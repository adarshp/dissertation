\chapter{Dark Matter}\label{ch:bsm}

\section{Calculation of the thermal relic density}

Assuming that the present-day abundance of WIMP dark matter is explained by a thermal history, the observed density of dark matter, known as the \emph{relic density} sets limits on particle dark matter candidates. We will briefly go over the standard calculation of the thermal relic density.

\newcommand{\fxv}{f(\mathbf{x},\mathbf{v})}

Consider the phase-space density of a single particle dark matter candidate in phase space, given by $\fxv$. That is, the probability of finding the particle with a position of $\mathbf{x}$ and velocity $\mathbf{v}$ is given by $\fxv d^3\mathbf{x}d^3\mathbf{v}$ \footnote{insert references}.
The time evolution of this phase-space density is given by Boltzmann's equation:
\begin{equation}
  \mathbf{L}f = \mathbf{C}f.
\end{equation}
Here, $\mathbf{L}$ is the Liouville operator, which in the non-relativistic limit reduces to 
\begin{equation}
  \mathbf{L}_\text{nr}f = \frac{\partial f}{\partial t} + \mathbf{\dot{x}}\frac{\partial f}{\partial\mathbf{x}} + \mathbf{\dot{v}}\frac{\partial f}{\partial \mathbf{v}} = \frac{d}{dt}
\end{equation}
The general-relativistic version of this operator is obtained by taking the total derivative along a world line parameterized by some affine parameter, say  $\lambda$.
\begin{equation}
  \mathbf{L} = \frac{d}{d\lambda} = \frac{dx^\mu}{d\lambda}\frac{\partial}{\partial x^\mu} + \frac{dp^\mu}{d\lambda}\frac{\partial}{\partial p^\mu}
\end{equation}
We can choose to normalize this parameter as follows:
\begin{equation}
  p^\mu = \frac{dx^\mu}{d\lambda} = m\frac{dx^\mu}{d\tau}
\end{equation}
where $\tau$ is the proper time. With this normalization, the geodesic equation becomes
\begin{equation}
  \frac{dp^\mu}{d\lambda}+\Gamma^{\mu}_{\alpha\beta}p^\alpha p^\beta = 0
\end{equation}
If we assume that there are no external non-gravitational forces, trajectory of the DM particle through phase space will be described by the geodesic. Thus, we can write the general-relativistic Liouville operator as:
\begin{equation}\label{eq:gr_liouville_operator}
  \mathbf{L} = p^\mu\frac{\partial}{\partial x^\mu} - \Gamma^\mu_{\alpha\beta}p^\alpha p^\beta\frac{\partial}{\partial p^\mu}
\end{equation}
The isotropy of the model implies that the phase-space density \emph{f} is devoid of directionality - it will depend only on the energy \emph{E} and time \emph{t}, and can be written as $f(E,t)$. Thus, all the partial derivatives in \eqref{eq:gr_liouville_operator} will vanish except for those corresponding to $\mu = 0$. If the connections $\Gamma^\mu_{\alpha\beta}$ are calculated for the FLRW metric (see \autoref{sec:sm_cosmology}), the Liouville operator simplifies to
\begin{equation}\label{eq:frw_liouville_operator}
  \mathbf{L}f = E\frac{\partial f}{\partial t} + H|\mathbf{p}|^2 \frac{\partial f}{\partial E}
\end{equation}
where $H$ is the Hubble parameter. The number density \emph{n} is given by the integral of $f(E,t)$ over all of phase-space, given by
\begin{align}
  n = 4\pi g\int dp p^2 f(E,t)
\end{align}
where $p = |\mathbf{p}|$, and $g$ is the number of spin degrees of freedom.
Dividing \eqref{eq:frw_liouville_operator} by E and integrating over the momentum coordinates gives us:
\begin{align}
  4\pi g\int dp p^2 \frac{\mathbf{L}f}{E} = 4\pi g\int dp p^2 \left(\frac{\partial f}{\partial t}- H\frac{p^2}{E}\frac{\partial f}{\partial E}\right)
\end{align}
Pulling the partial derivative outside the integral for the first term gives us
\begin{equation}
  \frac{dn}{dt} - 4\pi gH\int dp\frac{p^4}{E}\frac{\partial f}{\partial E}
\end{equation}
We know that $E^2 = p^2 - m^2$, thus $\frac{1}{E}\frac{\partial}{\partial E} = \frac{1}{p}\frac{\partial}{\partial p}$. Substituting this into the previous equation gives us:
\begin{equation}\label{eq:n_dot}
  \dot{n} - 4\pi gH\int dp p^3 \frac{\partial f}{\partial p}
\end{equation}
We can simplify the second term using integration by parts:
\begin{equation}
  \int dp p^3\frac{\partial f}{\partial p} = fp^3|_{0}^\infty - \int dp (3p^2 f)
\end{equation}
The first term on the right hand side of the above equation should evaluate to zero, since the probability of the particle having an infinite momentum must be zero. Substituting this expression into \eqref{eq:n_dot} and simplifying gives us
\begin{equation}\label{eq:simplified_n_density}
  \dot{n}+3Hn = 4\pi g\int \frac{dpp^2}{E}\mathbf{C}f
\end{equation}
The operator \textbf{C} is known as the \emph{collision} operator, and contains the information about how the dark matter particles interact with themselves and other particles. For a process of the form $1+2\rightarrow 3+4$, the collision term (on the right hand side of the above equation) for particle 1 takes the form  
\begin{equation}\label{eq:collision_term}
  \begin{split}
  &-\sum_\text{spins}\int [f_1 f_2(1\pm f_3)(1\pm f_4)|\mathcal{M}_{12\rightarrow 34}|^2 + f_3 f_4(1\pm f_1)(1\pm f_2)|\mathcal{M}_{34\rightarrow 12}|^2]\\
&\times(2\pi)^4\delta^4(p_1+p_2-p_3-p_4)d\Pi_1 d\Pi_2 d\Pi_3 d\Pi_4
\end{split}
\end{equation}
where
\[d\Pi_i = \frac{d^3 p_i}{(2\pi)^3 2E_i}\]
are phase-space integration factors. This can be simplified considerably by assuming the following:
\begin{itemize}
  \item The early universe is in thermal equilibrium, and thus the phase-space distribution $f$ takes either Fermi-Dirac or Bose-Einstein form.
  \item The temperature of each species, $T_i$ is low enough that the condition $T_i \ll E_i - \mu_i$, where $\mu_i$ is its chemical potential. This means that their phase-space distributions are approximately of the Maxwell-Boltzmann form, and we can neglect the statistical factors: $(1+f)\approx 1$.
\end{itemize}
The amplitude for the forward and backwards processes should be equal:
\[|\mathcal{M}_{12\rightarrow 34}|^2 = |\mathcal{M}_{34\rightarrow 12}|^2 = |\mathcal{M}|^2 \]
Then the expression in \eqref{eq:collision_term} simplifies to
\begin{equation}
-\sum_\text{spins}\int(f_1 f_2 - f_3 f_4)|\mathcal{M}|^2(2\pi)^4\delta(p_1 + p_2 - p_3 - p_4)d\Pi_1 d\Pi_2 d\Pi_3 d\Pi_4
\end{equation}
We can relate the matrix element to the cross section as follows:
\[\sum_\text{spins}|\mathcal{M}_{ij\rightarrow kl}|^2\times(2\pi)^4\delta^4(p_i + p_j - p_k -p_l)d\Pi_jd\Pi_k = 4g_ig_j\sigma_{ij}\sqrt{(p_i\cdot p_j)^2 - (m_im_j)^2}\]
where $\sigma_{ij}$ is the scattering cross section. Let us define the M\o ller velocity,
\[v_\text{M\o l} = \frac{\sqrt{(p_i\cdot p_j)^2 - (m_im_j)^2}}{E_iE_j}\]
Substituting this in the collision term and noting that $f_i(2E_i)d\Pi_i = dn_i$, we get the collision term as
\[g_1\int \mathbf{C}f_1\frac{d^3p_1}{(2\pi)^3} = -\int[(\sigma v_\text{M\o l})_{12}dn_1dn_2 - (\sigma v_\text{M\o l})_{34}dn_3dn_4]\]
The quantity $\sigma v_\text{M\o l}$ is relatively independent of the number densities $n_i$, and thus can be taken outside the integral, to give us
\[\dot{n}_1 + 3Hn_1 = -\langle\sigma v\rangle_{12}n_1n_2 + \langle\sigma v\rangle_{34}n_3n_4.\]
where $v = v_\text{M\o l}$. Let us now apply this result to the specific $2\rightarrow 2$ annihilation case shown in \autoref{fig:dm_annihilation}. The incoming DM particles are identical, with number density \emph{n}. When the DM particles are in equilibrium with the SM particles, a condition known as \emph{detailed balance} holds that relates the number densities and annihilation cross sections for the SM and DM particles:
\[\langle\sigma v\rangle_{12}n^2_\text{eq}=\langle\sigma v\rangle_{34}n_3^\text{eq}n_4^\text{eq}\]
Thus, the Boltzmann equation can be written as
\[\dot{n}+3Hn = \langle\sigma v\rangle (n_\text{eq}^2 - n^2)\]
where $\langle\sigma v\rangle = \langle\sigma v\rangle_{12}$. Intuitively, the number density $n$ will decrease naturally as the universe expands. If we want to focus on the effects of collision, we should define a \emph{comoving number density} $Y = n/s$, where $s$ is the total entropy density of the universe. If the dominant contribution to \emph{s} comes from thermal radiation, then $sa^3$ will be approximately constant. Rewriting the Boltzmann equation in terms of $Y$ gives us
\[\frac{dY}{dt} = \langle\sigma v\rangle s (Y_\text{eq}^2 - Y^2)\]
As the early radiation-dominated universe expanded, it also cooled down. It is expedient to write the above equation in terms of temperature, using the variable $x = m/T$, where \emph{m} is the mass of the DM particle. 
\[\frac{dY}{dx} = -\frac{\langle\sigma v\rangle s}{H(x)x}\]
The thermally averaged cross-section times velocity is given by
\begin{equation}\label{eq:thermal_sigma_v}
  \langle\sigma v\rangle = \frac{\int\sigma vdn_1^\text{eq}dn_2^\text{eq}}{\int dn_1^\text{eq}dn_2^\text{eq}}
  =\frac{\int \sigma v e^{-E_1/T}e^{-E_2/T}d^3p_1d^3p_2}{\int e^{-E_1/T}e^{-E_2/T}d^3p_1d^3p_2}
\end{equation}
The momentum-space volume element in spherical coordinates is
\[d^3p_1d^3p_2 = 4\pi |\mathbf{p}_1|dE_14\pi |\mathbf{p}_2|dE_2\frac{1}{2}d(\cos\theta)\]
where $\theta$ is the angle between the 3-momenta $\mathbf{p}_1$ and $\mathbf{p}_2$. We can then change the integration variables to $E_+, E_-, s$, given by
\begin{align}
  E_+ &= E_1 + E_2\\
  E_- &= E_1 - E_2,\\
  s &= 2m^2 + 2E_1E_2-2\mathbf{p}_1\mathbf{p}_2\cos\theta
\end{align}
Thus the volume element in these variables becomes
\[d^3p_1d^3p_2 = 2\pi^2E_1E_2dE_+dE_-ds\]
And the integration limits become
\begin{align}
  |E_1| &\leq \sqrt{1-\frac{4m^2}{s}}\sqrt{E_+^2 - s}\\
  E_+ &\geq \sqrt{s}\\
  s &\geq 4m^2
\end{align}
With these variables, the numerator in \eqref{eq:thermal_sigma_v} becomes
\begin{align*}
  &2\pi^2\int dE_+\int dE_-\int ds(\sigma v)E_1E_2e^{-E_+/T}\\
  =&4\pi^2\int ds\sigma \frac{1}{2}\sqrt{s(s-4m^2)}\sqrt{1-\frac{4m^2}{s}}\int dE_+e^{-E_+/T}\sqrt{E_+-s}\\
  =&2\pi^2T\int ds\sigma(s-4m^2)\sqrt{s}K_1(\sqrt{s}/T)
\end{align*}
Similarly, the denominator simplifies to $[4\pi m^2TK_2(m/T)]^2$. The functions $K_i$ are modified Bessel functions of order \emph{i}\footnote{The ordinary differential equation
  \begin{equation*}
    x^2\ddot{y} + x\dot{y} - (x^2+n^2)y = 0
  \end{equation*}
  admits solutions of the form $y = c_1 I_n(x) + c_2K_n(x)$. The functions $I_n$ and $K_n$ are known as modified Bessel functions of the first and second kinds, respectively.
}. Thus \eqref{eq:thermal_sigma_v} simplifies to
\[\langle\sigma v\rangle = \frac{1}{8m^4TK_2^2(m/T)}\int_{4m^2}^\infty\sigma(s-4m^2)\sqrt{s}K_1(\sqrt{s}/T)ds\]
In this non-relativistic limit, this simplifies to
\[\langle\sigma v\rangle \approx b_0 + \frac{3}{2}b_1x^{-1} + ...\]
The coefficients $b_0, b_1, ...$ correspond to \emph{s}-wave, \emph{p}-wave annihilation, and so on. An illustration of the thermal freeze-out mechanism is given in \autoref{fig:thermal_freeze_out} for \emph{s}-wave annihilation.

\strictpagecheck
\begin{figure}
  \begin{sidecaption}{Illustration of the freeze-out mechanism for \emph{s}-wave annihilation. The blue line represents the number density of dark matter, while the red dashed line represents its equilibrium value.\Adarsh{Label the curves, elaborate on this description.}}
  % \includegraphics[width=\textwidth]{images/relic_density.pdf}
    %% Creator: Matplotlib, PGF backend
%%
%% To include the figure in your LaTeX document, write
%%   \input{<filename>.pgf}
%%
%% Make sure the required packages are loaded in your preamble
%%   \usepackage{pgf}
%%
%% Figures using additional raster images can only be included by \input if
%% they are in the same directory as the main LaTeX file. For loading figures
%% from other directories you can use the `import` package
%%   \usepackage{import}
%% and then include the figures with
%%   \import{<path to file>}{<filename>.pgf}
%%
%% Matplotlib used the following preamble
%%   \usepackage{fontspec}
%%   \setmonofont{Andale Mono}
%%
\begingroup%
\makeatletter%
\begin{pgfpicture}%
\pgfpathrectangle{\pgfpointorigin}{\pgfqpoint{3.888197in}{2.403038in}}%
\pgfusepath{use as bounding box, clip}%
\begin{pgfscope}%
\pgfsetbuttcap%
\pgfsetmiterjoin%
\definecolor{currentfill}{rgb}{0.941176,0.941176,0.941176}%
\pgfsetfillcolor{currentfill}%
\pgfsetlinewidth{0.000000pt}%
\definecolor{currentstroke}{rgb}{0.941176,0.941176,0.941176}%
\pgfsetstrokecolor{currentstroke}%
\pgfsetdash{}{0pt}%
\pgfpathmoveto{\pgfqpoint{0.000000in}{0.000000in}}%
\pgfpathlineto{\pgfqpoint{3.888197in}{0.000000in}}%
\pgfpathlineto{\pgfqpoint{3.888197in}{2.403038in}}%
\pgfpathlineto{\pgfqpoint{0.000000in}{2.403038in}}%
\pgfpathclose%
\pgfusepath{fill}%
\end{pgfscope}%
\begin{pgfscope}%
\pgfsetbuttcap%
\pgfsetmiterjoin%
\definecolor{currentfill}{rgb}{0.941176,0.941176,0.941176}%
\pgfsetfillcolor{currentfill}%
\pgfsetlinewidth{0.000000pt}%
\definecolor{currentstroke}{rgb}{0.000000,0.000000,0.000000}%
\pgfsetstrokecolor{currentstroke}%
\pgfsetstrokeopacity{0.000000}%
\pgfsetdash{}{0pt}%
\pgfpathmoveto{\pgfqpoint{0.676752in}{0.518611in}}%
\pgfpathlineto{\pgfqpoint{3.738197in}{0.518611in}}%
\pgfpathlineto{\pgfqpoint{3.738197in}{2.253038in}}%
\pgfpathlineto{\pgfqpoint{0.676752in}{2.253038in}}%
\pgfpathclose%
\pgfusepath{fill}%
\end{pgfscope}%
\begin{pgfscope}%
\pgfpathrectangle{\pgfqpoint{0.676752in}{0.518611in}}{\pgfqpoint{3.061445in}{1.734427in}} %
\pgfusepath{clip}%
\pgfsetbuttcap%
\pgfsetroundjoin%
\pgfsetlinewidth{1.003750pt}%
\definecolor{currentstroke}{rgb}{0.796078,0.796078,0.796078}%
\pgfsetstrokecolor{currentstroke}%
\pgfsetdash{}{0pt}%
\pgfpathmoveto{\pgfqpoint{0.815909in}{0.518611in}}%
\pgfpathlineto{\pgfqpoint{0.815909in}{2.253038in}}%
\pgfusepath{stroke}%
\end{pgfscope}%
\begin{pgfscope}%
\pgftext[x=0.815909in,y=0.469999in,,top]{\setmainfont{Minion Pro}\rmfamily\fontsize{10.000000}{12.000000}\selectfont \(\displaystyle {10^{0}}\)}%
\end{pgfscope}%
\begin{pgfscope}%
\pgfpathrectangle{\pgfqpoint{0.676752in}{0.518611in}}{\pgfqpoint{3.061445in}{1.734427in}} %
\pgfusepath{clip}%
\pgfsetbuttcap%
\pgfsetroundjoin%
\pgfsetlinewidth{1.003750pt}%
\definecolor{currentstroke}{rgb}{0.796078,0.796078,0.796078}%
\pgfsetstrokecolor{currentstroke}%
\pgfsetdash{}{0pt}%
\pgfpathmoveto{\pgfqpoint{1.743619in}{0.518611in}}%
\pgfpathlineto{\pgfqpoint{1.743619in}{2.253038in}}%
\pgfusepath{stroke}%
\end{pgfscope}%
\begin{pgfscope}%
\pgftext[x=1.743619in,y=0.469999in,,top]{\setmainfont{Minion Pro}\rmfamily\fontsize{10.000000}{12.000000}\selectfont \(\displaystyle {10^{1}}\)}%
\end{pgfscope}%
\begin{pgfscope}%
\pgfpathrectangle{\pgfqpoint{0.676752in}{0.518611in}}{\pgfqpoint{3.061445in}{1.734427in}} %
\pgfusepath{clip}%
\pgfsetbuttcap%
\pgfsetroundjoin%
\pgfsetlinewidth{1.003750pt}%
\definecolor{currentstroke}{rgb}{0.796078,0.796078,0.796078}%
\pgfsetstrokecolor{currentstroke}%
\pgfsetdash{}{0pt}%
\pgfpathmoveto{\pgfqpoint{2.671330in}{0.518611in}}%
\pgfpathlineto{\pgfqpoint{2.671330in}{2.253038in}}%
\pgfusepath{stroke}%
\end{pgfscope}%
\begin{pgfscope}%
\pgftext[x=2.671330in,y=0.469999in,,top]{\setmainfont{Minion Pro}\rmfamily\fontsize{10.000000}{12.000000}\selectfont \(\displaystyle {10^{2}}\)}%
\end{pgfscope}%
\begin{pgfscope}%
\pgfpathrectangle{\pgfqpoint{0.676752in}{0.518611in}}{\pgfqpoint{3.061445in}{1.734427in}} %
\pgfusepath{clip}%
\pgfsetbuttcap%
\pgfsetroundjoin%
\pgfsetlinewidth{1.003750pt}%
\definecolor{currentstroke}{rgb}{0.796078,0.796078,0.796078}%
\pgfsetstrokecolor{currentstroke}%
\pgfsetdash{}{0pt}%
\pgfpathmoveto{\pgfqpoint{3.599040in}{0.518611in}}%
\pgfpathlineto{\pgfqpoint{3.599040in}{2.253038in}}%
\pgfusepath{stroke}%
\end{pgfscope}%
\begin{pgfscope}%
\pgftext[x=3.599040in,y=0.469999in,,top]{\setmainfont{Minion Pro}\rmfamily\fontsize{10.000000}{12.000000}\selectfont \(\displaystyle {10^{3}}\)}%
\end{pgfscope}%
\begin{pgfscope}%
\pgftext[x=2.207475in,y=0.282222in,,top]{\setmainfont{Minion Pro}\rmfamily\fontsize{10.000000}{12.000000}\selectfont x}%
\end{pgfscope}%
\begin{pgfscope}%
\pgfpathrectangle{\pgfqpoint{0.676752in}{0.518611in}}{\pgfqpoint{3.061445in}{1.734427in}} %
\pgfusepath{clip}%
\pgfsetbuttcap%
\pgfsetroundjoin%
\pgfsetlinewidth{1.003750pt}%
\definecolor{currentstroke}{rgb}{0.796078,0.796078,0.796078}%
\pgfsetstrokecolor{currentstroke}%
\pgfsetdash{}{0pt}%
\pgfpathmoveto{\pgfqpoint{0.676752in}{0.518611in}}%
\pgfpathlineto{\pgfqpoint{3.738197in}{0.518611in}}%
\pgfusepath{stroke}%
\end{pgfscope}%
\begin{pgfscope}%
\pgftext[x=0.340139in,y=0.469166in,left,base]{\setmainfont{Minion Pro}\rmfamily\fontsize{10.000000}{12.000000}\selectfont \(\displaystyle {10^{-5}}\)}%
\end{pgfscope}%
\begin{pgfscope}%
\pgfpathrectangle{\pgfqpoint{0.676752in}{0.518611in}}{\pgfqpoint{3.061445in}{1.734427in}} %
\pgfusepath{clip}%
\pgfsetbuttcap%
\pgfsetroundjoin%
\pgfsetlinewidth{1.003750pt}%
\definecolor{currentstroke}{rgb}{0.796078,0.796078,0.796078}%
\pgfsetstrokecolor{currentstroke}%
\pgfsetdash{}{0pt}%
\pgfpathmoveto{\pgfqpoint{0.676752in}{0.843816in}}%
\pgfpathlineto{\pgfqpoint{3.738197in}{0.843816in}}%
\pgfusepath{stroke}%
\end{pgfscope}%
\begin{pgfscope}%
\pgftext[x=0.340139in,y=0.794371in,left,base]{\setmainfont{Minion Pro}\rmfamily\fontsize{10.000000}{12.000000}\selectfont \(\displaystyle {10^{-2}}\)}%
\end{pgfscope}%
\begin{pgfscope}%
\pgfpathrectangle{\pgfqpoint{0.676752in}{0.518611in}}{\pgfqpoint{3.061445in}{1.734427in}} %
\pgfusepath{clip}%
\pgfsetbuttcap%
\pgfsetroundjoin%
\pgfsetlinewidth{1.003750pt}%
\definecolor{currentstroke}{rgb}{0.796078,0.796078,0.796078}%
\pgfsetstrokecolor{currentstroke}%
\pgfsetdash{}{0pt}%
\pgfpathmoveto{\pgfqpoint{0.676752in}{1.169021in}}%
\pgfpathlineto{\pgfqpoint{3.738197in}{1.169021in}}%
\pgfusepath{stroke}%
\end{pgfscope}%
\begin{pgfscope}%
\pgftext[x=0.426945in,y=1.119576in,left,base]{\setmainfont{Minion Pro}\rmfamily\fontsize{10.000000}{12.000000}\selectfont \(\displaystyle {10^{1}}\)}%
\end{pgfscope}%
\begin{pgfscope}%
\pgfpathrectangle{\pgfqpoint{0.676752in}{0.518611in}}{\pgfqpoint{3.061445in}{1.734427in}} %
\pgfusepath{clip}%
\pgfsetbuttcap%
\pgfsetroundjoin%
\pgfsetlinewidth{1.003750pt}%
\definecolor{currentstroke}{rgb}{0.796078,0.796078,0.796078}%
\pgfsetstrokecolor{currentstroke}%
\pgfsetdash{}{0pt}%
\pgfpathmoveto{\pgfqpoint{0.676752in}{1.494226in}}%
\pgfpathlineto{\pgfqpoint{3.738197in}{1.494226in}}%
\pgfusepath{stroke}%
\end{pgfscope}%
\begin{pgfscope}%
\pgftext[x=0.426945in,y=1.444782in,left,base]{\setmainfont{Minion Pro}\rmfamily\fontsize{10.000000}{12.000000}\selectfont \(\displaystyle {10^{4}}\)}%
\end{pgfscope}%
\begin{pgfscope}%
\pgfpathrectangle{\pgfqpoint{0.676752in}{0.518611in}}{\pgfqpoint{3.061445in}{1.734427in}} %
\pgfusepath{clip}%
\pgfsetbuttcap%
\pgfsetroundjoin%
\pgfsetlinewidth{1.003750pt}%
\definecolor{currentstroke}{rgb}{0.796078,0.796078,0.796078}%
\pgfsetstrokecolor{currentstroke}%
\pgfsetdash{}{0pt}%
\pgfpathmoveto{\pgfqpoint{0.676752in}{1.819431in}}%
\pgfpathlineto{\pgfqpoint{3.738197in}{1.819431in}}%
\pgfusepath{stroke}%
\end{pgfscope}%
\begin{pgfscope}%
\pgftext[x=0.426945in,y=1.769987in,left,base]{\setmainfont{Minion Pro}\rmfamily\fontsize{10.000000}{12.000000}\selectfont \(\displaystyle {10^{7}}\)}%
\end{pgfscope}%
\begin{pgfscope}%
\pgfpathrectangle{\pgfqpoint{0.676752in}{0.518611in}}{\pgfqpoint{3.061445in}{1.734427in}} %
\pgfusepath{clip}%
\pgfsetbuttcap%
\pgfsetroundjoin%
\pgfsetlinewidth{1.003750pt}%
\definecolor{currentstroke}{rgb}{0.796078,0.796078,0.796078}%
\pgfsetstrokecolor{currentstroke}%
\pgfsetdash{}{0pt}%
\pgfpathmoveto{\pgfqpoint{0.676752in}{2.144636in}}%
\pgfpathlineto{\pgfqpoint{3.738197in}{2.144636in}}%
\pgfusepath{stroke}%
\end{pgfscope}%
\begin{pgfscope}%
\pgftext[x=0.371582in,y=2.095192in,left,base]{\setmainfont{Minion Pro}\rmfamily\fontsize{10.000000}{12.000000}\selectfont \(\displaystyle {10^{10}}\)}%
\end{pgfscope}%
\begin{pgfscope}%
\pgftext[x=0.284583in,y=1.385824in,,bottom,rotate=90.000000]{\setmainfont{Minion Pro}\rmfamily\fontsize{10.000000}{12.000000}\selectfont Number density}%
\end{pgfscope}%
\begin{pgfscope}%
\pgfpathrectangle{\pgfqpoint{0.676752in}{0.518611in}}{\pgfqpoint{3.061445in}{1.734427in}} %
\pgfusepath{clip}%
\pgfsetbuttcap%
\pgfsetroundjoin%
\pgfsetlinewidth{0.501875pt}%
\definecolor{currentstroke}{rgb}{0.000000,0.560784,0.835294}%
\pgfsetstrokecolor{currentstroke}%
\pgfsetdash{}{0pt}%
\pgfpathmoveto{\pgfqpoint{0.815909in}{2.169617in}}%
\pgfpathlineto{\pgfqpoint{0.854276in}{2.172283in}}%
\pgfpathlineto{\pgfqpoint{0.889306in}{2.173719in}}%
\pgfpathlineto{\pgfqpoint{0.979150in}{2.175356in}}%
\pgfpathlineto{\pgfqpoint{1.052567in}{2.174113in}}%
\pgfpathlineto{\pgfqpoint{1.114645in}{2.170883in}}%
\pgfpathlineto{\pgfqpoint{1.168423in}{2.166198in}}%
\pgfpathlineto{\pgfqpoint{1.230509in}{2.158266in}}%
\pgfpathlineto{\pgfqpoint{1.284292in}{2.148878in}}%
\pgfpathlineto{\pgfqpoint{1.331735in}{2.138379in}}%
\pgfpathlineto{\pgfqpoint{1.374174in}{2.127003in}}%
\pgfpathlineto{\pgfqpoint{1.421619in}{2.111801in}}%
\pgfpathlineto{\pgfqpoint{1.464061in}{2.095722in}}%
\pgfpathlineto{\pgfqpoint{1.502454in}{2.078933in}}%
\pgfpathlineto{\pgfqpoint{1.544164in}{2.058022in}}%
\pgfpathlineto{\pgfqpoint{1.581959in}{2.036425in}}%
\pgfpathlineto{\pgfqpoint{1.616510in}{2.014260in}}%
\pgfpathlineto{\pgfqpoint{1.653398in}{1.987800in}}%
\pgfpathlineto{\pgfqpoint{1.687190in}{1.960798in}}%
\pgfpathlineto{\pgfqpoint{1.722629in}{1.929380in}}%
\pgfpathlineto{\pgfqpoint{1.755201in}{1.897461in}}%
\pgfpathlineto{\pgfqpoint{1.788949in}{1.861044in}}%
\pgfpathlineto{\pgfqpoint{1.820087in}{1.824169in}}%
\pgfpathlineto{\pgfqpoint{1.852078in}{1.782741in}}%
\pgfpathlineto{\pgfqpoint{1.884562in}{1.736697in}}%
\pgfpathlineto{\pgfqpoint{1.914621in}{1.690231in}}%
\pgfpathlineto{\pgfqpoint{1.945041in}{1.639139in}}%
\pgfpathlineto{\pgfqpoint{1.975594in}{1.583423in}}%
\pgfpathlineto{\pgfqpoint{2.006097in}{1.523247in}}%
\pgfpathlineto{\pgfqpoint{2.046019in}{1.438903in}}%
\pgfpathlineto{\pgfqpoint{2.075332in}{1.378767in}}%
\pgfpathlineto{\pgfqpoint{2.090930in}{1.351591in}}%
\pgfpathlineto{\pgfqpoint{2.104305in}{1.332491in}}%
\pgfpathlineto{\pgfqpoint{2.117250in}{1.317734in}}%
\pgfpathlineto{\pgfqpoint{2.131333in}{1.305156in}}%
\pgfpathlineto{\pgfqpoint{2.146424in}{1.294647in}}%
\pgfpathlineto{\pgfqpoint{2.165234in}{1.284466in}}%
\pgfpathlineto{\pgfqpoint{2.187240in}{1.275231in}}%
\pgfpathlineto{\pgfqpoint{2.214411in}{1.266374in}}%
\pgfpathlineto{\pgfqpoint{2.248005in}{1.257923in}}%
\pgfpathlineto{\pgfqpoint{2.288492in}{1.250103in}}%
\pgfpathlineto{\pgfqpoint{2.339286in}{1.242653in}}%
\pgfpathlineto{\pgfqpoint{2.402047in}{1.235782in}}%
\pgfpathlineto{\pgfqpoint{2.480964in}{1.229477in}}%
\pgfpathlineto{\pgfqpoint{2.580563in}{1.223853in}}%
\pgfpathlineto{\pgfqpoint{2.707903in}{1.218991in}}%
\pgfpathlineto{\pgfqpoint{2.874923in}{1.214952in}}%
\pgfpathlineto{\pgfqpoint{3.102571in}{1.211807in}}%
\pgfpathlineto{\pgfqpoint{3.435820in}{1.209606in}}%
\pgfpathlineto{\pgfqpoint{3.599040in}{1.209052in}}%
\pgfpathlineto{\pgfqpoint{3.599040in}{1.209052in}}%
\pgfusepath{stroke}%
\end{pgfscope}%
\begin{pgfscope}%
\pgfpathrectangle{\pgfqpoint{0.676752in}{0.518611in}}{\pgfqpoint{3.061445in}{1.734427in}} %
\pgfusepath{clip}%
\pgfsetbuttcap%
\pgfsetroundjoin%
\pgfsetlinewidth{0.501875pt}%
\definecolor{currentstroke}{rgb}{0.988235,0.309804,0.188235}%
\pgfsetstrokecolor{currentstroke}%
\pgfsetdash{{5.600000pt}{2.400000pt}}{0.000000pt}%
\pgfpathmoveto{\pgfqpoint{0.815909in}{2.170262in}}%
\pgfpathlineto{\pgfqpoint{0.854276in}{2.172283in}}%
\pgfpathlineto{\pgfqpoint{0.889306in}{2.173719in}}%
\pgfpathlineto{\pgfqpoint{0.979150in}{2.175356in}}%
\pgfpathlineto{\pgfqpoint{1.052567in}{2.174113in}}%
\pgfpathlineto{\pgfqpoint{1.114645in}{2.170883in}}%
\pgfpathlineto{\pgfqpoint{1.168423in}{2.166198in}}%
\pgfpathlineto{\pgfqpoint{1.230509in}{2.158266in}}%
\pgfpathlineto{\pgfqpoint{1.284292in}{2.148878in}}%
\pgfpathlineto{\pgfqpoint{1.331735in}{2.138379in}}%
\pgfpathlineto{\pgfqpoint{1.374174in}{2.127003in}}%
\pgfpathlineto{\pgfqpoint{1.421619in}{2.111801in}}%
\pgfpathlineto{\pgfqpoint{1.464061in}{2.095722in}}%
\pgfpathlineto{\pgfqpoint{1.502454in}{2.078933in}}%
\pgfpathlineto{\pgfqpoint{1.544164in}{2.058022in}}%
\pgfpathlineto{\pgfqpoint{1.581959in}{2.036425in}}%
\pgfpathlineto{\pgfqpoint{1.616510in}{2.014260in}}%
\pgfpathlineto{\pgfqpoint{1.653398in}{1.987800in}}%
\pgfpathlineto{\pgfqpoint{1.687190in}{1.960798in}}%
\pgfpathlineto{\pgfqpoint{1.722629in}{1.929380in}}%
\pgfpathlineto{\pgfqpoint{1.755201in}{1.897461in}}%
\pgfpathlineto{\pgfqpoint{1.788949in}{1.861044in}}%
\pgfpathlineto{\pgfqpoint{1.820087in}{1.824169in}}%
\pgfpathlineto{\pgfqpoint{1.852078in}{1.782740in}}%
\pgfpathlineto{\pgfqpoint{1.884562in}{1.736694in}}%
\pgfpathlineto{\pgfqpoint{1.914621in}{1.690223in}}%
\pgfpathlineto{\pgfqpoint{1.945041in}{1.639112in}}%
\pgfpathlineto{\pgfqpoint{1.975594in}{1.583321in}}%
\pgfpathlineto{\pgfqpoint{2.006097in}{1.522817in}}%
\pgfpathlineto{\pgfqpoint{2.036403in}{1.457575in}}%
\pgfpathlineto{\pgfqpoint{2.066399in}{1.387575in}}%
\pgfpathlineto{\pgfqpoint{2.095998in}{1.312801in}}%
\pgfpathlineto{\pgfqpoint{2.125135in}{1.233244in}}%
\pgfpathlineto{\pgfqpoint{2.155215in}{1.144444in}}%
\pgfpathlineto{\pgfqpoint{2.184554in}{1.050811in}}%
\pgfpathlineto{\pgfqpoint{2.214411in}{0.947862in}}%
\pgfpathlineto{\pgfqpoint{2.243374in}{0.840052in}}%
\pgfpathlineto{\pgfqpoint{2.272565in}{0.722875in}}%
\pgfpathlineto{\pgfqpoint{2.301803in}{0.596299in}}%
\pgfpathlineto{\pgfqpoint{2.320838in}{0.508611in}}%
\pgfpathlineto{\pgfqpoint{2.320838in}{0.508611in}}%
\pgfusepath{stroke}%
\end{pgfscope}%
\begin{pgfscope}%
\pgfsetrectcap%
\pgfsetmiterjoin%
\pgfsetlinewidth{3.011250pt}%
\definecolor{currentstroke}{rgb}{0.941176,0.941176,0.941176}%
\pgfsetstrokecolor{currentstroke}%
\pgfsetdash{}{0pt}%
\pgfpathmoveto{\pgfqpoint{0.676752in}{0.518611in}}%
\pgfpathlineto{\pgfqpoint{0.676752in}{2.253038in}}%
\pgfusepath{stroke}%
\end{pgfscope}%
\begin{pgfscope}%
\pgfsetrectcap%
\pgfsetmiterjoin%
\pgfsetlinewidth{3.011250pt}%
\definecolor{currentstroke}{rgb}{0.941176,0.941176,0.941176}%
\pgfsetstrokecolor{currentstroke}%
\pgfsetdash{}{0pt}%
\pgfpathmoveto{\pgfqpoint{3.738197in}{0.518611in}}%
\pgfpathlineto{\pgfqpoint{3.738197in}{2.253038in}}%
\pgfusepath{stroke}%
\end{pgfscope}%
\begin{pgfscope}%
\pgfsetrectcap%
\pgfsetmiterjoin%
\pgfsetlinewidth{3.011250pt}%
\definecolor{currentstroke}{rgb}{0.941176,0.941176,0.941176}%
\pgfsetstrokecolor{currentstroke}%
\pgfsetdash{}{0pt}%
\pgfpathmoveto{\pgfqpoint{0.676752in}{0.518611in}}%
\pgfpathlineto{\pgfqpoint{3.738197in}{0.518611in}}%
\pgfusepath{stroke}%
\end{pgfscope}%
\begin{pgfscope}%
\pgfsetrectcap%
\pgfsetmiterjoin%
\pgfsetlinewidth{3.011250pt}%
\definecolor{currentstroke}{rgb}{0.941176,0.941176,0.941176}%
\pgfsetstrokecolor{currentstroke}%
\pgfsetdash{}{0pt}%
\pgfpathmoveto{\pgfqpoint{0.676752in}{2.253038in}}%
\pgfpathlineto{\pgfqpoint{3.738197in}{2.253038in}}%
\pgfusepath{stroke}%
\end{pgfscope}%
\end{pgfpicture}%
\makeatother%
\endgroup%

  \end{sidecaption}
  \label{fig:thermal_freeze_out}
\end{figure}

\section{Bino Dark Matter}
Binos annihilate with each other through \emph{p}-wave processes\footnote{\Adarsh{Insert citation.}}, rather than \emph{s}-wave processes. Heavy binos do not annihilate efficiently enough to have a dark matter relic density low enough to be consistent with experimental observations.
