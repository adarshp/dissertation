\chapter{Two-Higgs Doublet Models}\label{ch:2HDMs}

Intuitively, it is not too hard to imagine, based on the complex structure of the fermion and gauge sectors, that the scalar sector of the Standard Model might well contain members other than a single $SU(2)$ doublet. The possible extensions to the scalar sector are constrained by the value of the parameter $\rho$. For a $SU(2)\times U(1)$ gauge theory, this takes the form (at tree level):

\begin{equation}
\rho = \frac{\sum_\text{i=1}^n\left[I_i(I_i+1)-\frac{1}{4}Y_i^2Y_i^2\right]v_i}{\sum_\text{i=1}^n\frac{1}{2}Y_i^2v_i}
\end{equation}
%
where $n$ is the number of scalar multiplets with isospin\footnote{Short for \emph{isotopic spin}} $I_i$, hypercharge $Y_i$, and vacuum expectation values of the neutral components $v_i$. This parameter has been experimentally measured to be almost exactly one. The simplest extensions to the scalar sector that satisfy this constraint involve adding additional scalar singlets and doublets. The ones that add a single additional $SU(2)$ scalar doublet field are collectively known as \emph{Two-Higgs Doublet Models}, or 2HDMs. For a detailed review of 2HDMs, see \citep{Branco2012}, whose treatment we follow closely here.

Adding an additional scalar doublet alleviates some of the problems with the Standard Model that we mentioned at the end of the previous chapter. 
% Strong CP problem
Lagrangians for Yang-Mills theories can have a renormalizable term that is gauge-invariant but but violates CP, of the form
\[\mathcal{L}_{\theta} = \theta\epsilon^{\mu\nu\alpha\beta}F_{\mu\nu}^aF_{\alpha\beta}^a\]
where $\theta$ is some angle, $F_{\mu\nu}$ is the field strength tensor, and $\epsilon^{\mu\nu\alpha\beta}$ is an antisymmetric tensor\footnote{Levi-Civita?}. This term is a total derivative, since it can be written as $2\theta\partial_\mu(\epsilon^{\mu\nu\alpha\beta}A_\nu^aF_{\alpha\beta}^a)$. Thus, it should not contribute to perturbative effects. However, this term can potentially contribute to non-perturbative effects. Additionally, this term can be modified by chiral rotations of the form $\psi\rightarrow \exp(i\gamma_5\theta_F)\psi$. Since physical observables should be independent of the choice of the basis, i.e. $\theta_F$, we should absorb it into $\theta$ by defining a basis-independent phase: $\bar{\theta} = \theta-\theta_F$. For $SU(2)$ and $U(1)$ gauge symmetries, this phase can be set to zero by performing appropriate chiral rotations of the fermion fields. However, no such choice exists for the term corresponding to the the $SU(3)$ group, and thus the CP-violating term in the QCD Lagrangian could in principle by non-zero. A non-zero value of $\bar{\theta}$ would be manifested as non-perturbative effects. For example, the neutron would then have a non-zero electric dipole moment. However, experiments have shown that $\bar{\theta}$ must be vanishingly small, with a stringent upper bound: $\bar{\theta}<10^{-10}$.

Adding an additional scalar doublet allows us to impose a global $U(1)$ symmetry on the Lagrangian, known as \emph{Peccei-Quinn} symmetry. If this symmetry is spontaneously broken, a Goldstone boson arises, which can then be chirally rotated such that $\bar{\theta}$ becomes effectively zero for the ground state.
% Baryon asymmetry
Another motivation for 2HDMs is their ability to explain the observed baryon asymmetry of the universe. That is, the amount of matter in the universe is much larger than the amount of antimatter in it. The CP violation in the weak sector the Standard Model cannot account for this imbalance, but 2HDMs, with their complex scalar sector and possible new sources of CP violation, can.
%
The strongest motivation for 2HDMs is, however, the potential to resolve the hierarchy problem. More specifically, supersymmetric theories provide a way to resolve the hierarchy problem, as discussed in \autoref{ch:supersymmetry}. These theories require an additional scalar doublet to give mass to both up- and down-type fermions, and for the cancellation of anomalies.

\section{The 2HDM Lagrangian}
The most general renormalizable scalar potential for two doublets $\Phi_1$ and $\Phi_2$ with hypercharge $+1$ is:

\begin{align*}
  V(\Phi_1,\Phi_2) &= m_{11}^2|\Phi_1|^2 + m_{22}^2|\Phi_2|^2 - m_{12}^2\left(\Phi_1^\dagger\Phi_2 + h.c.\right)\\
&+\frac{\lambda_1}{2}|\Phi_1|^4 + \frac{\lambda_2}{2}|\Phi_2|^4+\lambda_3|\Phi_1|^2|\Phi_2|^2 + \lambda_4|\Phi_1^\dagger\Phi_2|^2\\
&+\frac{\lambda_5}{2}\left[\left(\Phi_1^\dagger\Phi_2 \right)^2+\lambda_6|\Phi_1|^2(\Phi_1^\dagger\Phi_2)+\lambda_7|\Phi_2|^2(\Phi_1^\dagger\Phi_2) + h.c.\right]
\label{eq:2HDM_scalar_potential}
\end{align*}
where $h.c.$ stands for the hermitian conjugate. The parameters $m_{11}, m_{22}$, and $\lambda_{1,2,3,4}$ are real while $m_{12}$ and $\lambda_{5,6,7}$ can be complex. Naively, it would seem that this potential has 14 degrees of freedom - six from the real parameters, and eight from the complex parameters. However, it should be noted that we have the freedom to perform basis transformations, that is, we can write the potential in terms of new doublets $\Phi_a' = \sum_{b=1}^2U_{ab}\Phi_b$. where $U_{ab}$ is a $2\times 2$ unitary matrix. The condition of unitarity implies that $U$ has three degrees of freedom, which can absorb three out of the 14 degrees of freedom listed earlier. Thus, only 11 out of the original 14 degrees of freedom are physical.

In principle, we could proceed with these 11 parameters, however, there are a couple of reasons to attempt to reduce this number. The first is that a large number of free parameters makes a theory less falsifiable, thus reducing its predictive power. The second is that in order to distinguish between pseudoscalars and scalars, CP must be conserved in the Higgs sector. Finally, the potential in \eqref{eq:2HDM_scalar_potential} allows for tree-level flavor-changing neutral currents (FCNC), which are experimentally measured to be highly suppressed. These can be eliminated by introducing discrete or continuous symmetries. Imposing a discrete symmetry such as $\mathcal{Z}_2$, that is, the Lagrangian is invariant under the reflection of one of the doublets: $\Phi_i\rightarrow-\Phi_i$, removes the terms that are odd in $\Phi_i$. This effectively sets $\lambda_6=\lambda_7 = 0$. In principle, this should set $m_{12} = 0$ as well, but we retain this term since it breaks the $\mathcal{Z}_2$ symmetry softly, which relaxes the experimental bounds on the mass spectrum.
After imposing these constraints, all the remaining parameters $\lambda_{1,2,3,4,5}, m_{11,12,22}$ are real. From here on, we will only consider 2HDMs with these constraints. 

There are four such models, classified based on the coupling patterns of the fermions to the two Higgs doublets. In type-I 2HDMs, all the quarks couple to only one of the Higgs doublets (chosen by convention to be $\Phi_2$. In type-II 2HDMs, the up-type right-handed quarks (\emph{u,c,t}) couple to $\Phi_2$, and the down-type right-handed quarks (\emph{d,s,b}) couple to $\Phi_1$. In both of these models, the right-handed leptons couple to the same doublet as the down-type quarks. There are two other models that do not have tree-level FCNCs. The lepton-specific model is similar to the type-I model, except in this case, the right-handed leptons couple to $\Phi_1$. Similarly, the `flipped' model is similar to the type-II 2HDM, except that the leptons couple to $\Phi_2$. The coupling patterns for these models are collected in \autoref{tab:no_FCNC_2HDMs}.

\begin{margintable}
\small{
  \begin{tabular}{lccc}
	\toprule
    Model & $u_R^i$ & $d_R^i$  & $e_R^i$\\
    \midrule
    Type I          & $\Phi_2$ & $\Phi_2$ & $\Phi_2$\\
    Type II         & $\Phi_2$ & $\Phi_1$ & $\Phi_1$\\
    Lepton-specific & $\Phi_2$ & $\Phi_2$ & $\Phi_1$\\
    Flipped         & $\Phi_2$ & $\Phi_1$ & $\Phi_2$\\
    \bottomrule
  \end{tabular}}
  \caption{2HDMs with flavor conservation.}
  \label{tab:no_FCNC_2HDMs}
\end{margintable}

This potential is minimized for non-zero vacuum expectation values of $\Phi_i$:
\begin{align}
\langle\Phi_i\rangle_0=\vdoublet{0}{\frac{v_i}{\sqrt{2}}}
\end{align}
The complex scalar $SU(2)$ doublets $\Phi_i$ can be expressed in terms of eight real fields as follows:
\begin{equation}\label{eq:2HDM_doublet_components}
\Phi_i = \vdoublet{\phi_i^+}{\frac{1}{\sqrt{2}}(v_i+\rho_i+i\eta_i)}
\end{equation}
The process of electroweak symmetry breaking causes three of these fields to be `eaten' by the \emph{W} and \emph{Z} bosons, and the remaining five are manifested as physical scalar fields. These consist of a pair of CP-even neutral scalars \emph{h} and \emph{H}, a CP-odd pseudoscalar \emph{A}, and a charged scalar $H^\pm$.

\section{The 2HDM mass spectrum}

In this section, we will analyze the mass spectrum of flavor-conserving 2HDMs. To do so, we construct the mass matrices by taking derivatives of the scalar potential:
\[M_{ij} = \frac{\partial V(\Phi_1,\Phi_2)}{\partial\phi_i\partial\phi_j}\]
where $\phi_{i}$ can be any of the fields $\phi_i^+,\rho_i,\eta_i$ in \eqref{eq:2HDM_doublet_components}. We will also adopt the notation: 
Applying this procedure to the charged scalar components $\phi_i^\pm$, we obtain their mass matrix $M_{\phi^{\pm}}$:
\[M_{\phi^\pm} = \left[m_{12}^2-(\lambda_4+\lambda_5)v_1v_2\right]\fourmatrix{v_2/v_1}{-1}{-1}{v_1/v_2}\]
Diagonalizing this matrix gives us the mass of the charged Higgses:
\[m_{H^\pm}^2 = (v_1^2+v_2^2)[m_{12}^2/v_1v_2-(\lambda_4+\lambda_5)]\]
Similarly, the mass matrix for the pseudoscalars is given by
\[M_{\eta} = \frac{m_A^2}{v_1^2+v_2^2}\fourmatrix{v_2^2}{-v_1v_2}{-v_1v_2}{v_1^2}\]
Diagonalizing this matrix gives us a massless Goldstone boson $G^0$, corresponding to a zero eigenvalue, and a pseudoscalar Higgs boson $A$, with mass given by:
\[m_A^2 = (v_1^2+v_2^2)[m_{12}^2/v_1v_2-2\lambda_5]\]
The diagonalization process amounts to a rotation of the basis vectors by some angle.For the CP-odd and the charged scalars, this angle is the same, and is denoted as $\beta$. Let us now adopt the notation 
\[s_\theta,c_\theta,t_\theta = \sin\theta,\cos\theta,\tan\theta\]
for conciseness. In this notation, the mass eigenstates are given by
\begin{align}
\vdoublet{A}{G^0} = \fourmatrix{s_\beta}{-c_\beta}{c_\beta}{s_\beta}\vdoublet{\eta_1}{\eta_2}&&\text{and}&&
\vdoublet{H^\pm}{G^\pm} = \fourmatrix{s_\beta}{-c_\beta}{c_\beta}{s_\beta}\vdoublet{\phi_1^+}{\phi_2^+}.
\end{align}
The angle $\beta$ turns out to be a very important one for studying 2HDMs. It also represents the ratio of the vacuum expectation values of the neutral components of the two Higgs doublets: $t_\beta = v_2/v_1$.
Finally, the mass matrix for the CP-even scalars is given by:
\[M_{\rho} = -\fourmatrix{m_{12}^2\frac{v_2}{v_1}+\lambda_1v_1^2}{-m_{12}^2+\lambda_{345}v_1v_2}{-m_{12}^2+\lambda_{345}v_1v_2}{m_{12}^2\frac{v_2}{v_1}+\lambda_1v_2^2}\]
where $\lambda_{345} = \lambda_3+\lambda_4+\lambda_5$. This matrix is diagonalized by rotation of the basis vectors by the angle $\alpha$:
\[\vdoublet{h}{H} = \fourmatrix{s_\alpha}{-c_\alpha}{-c_\alpha}{-s_\alpha}\vdoublet{\rho_1}{\rho_2}\]
with the mass eigenstates denoted $h,H$. Traditionally, $h$ is taken to be the ligher of the two particles. 

%\begin{align}
%v &= v_1^2 + v_2^2
%\end{align}
Thus we see that the physical spectrum of 2HDMs contains five mass eigenstates: the CP-even higgses $h$ and $H$, the CP-odd pseudoscalar Higgs $A$, and a pair of charged Higgses $H^\pm$. Incidentally, the standard model Higgs is a combination of the CP-even scalars:
\begin{equation}
h_\text{SM} = \rho_1\cos\beta + \rho_2\sin\beta = h\sin(\alpha-\beta)-H\cos(\alpha-\beta)
\label{eq:h_SM}
\end{equation}


\section{Interactions}

\newcommand{\sbma}{s_{\beta-\alpha}}
\newcommand{\cbma}{c_{\beta-\alpha}}
\newcommand{\casb}{c_\alpha/s_\beta}
\newcommand{\sacb}{s_\alpha/c_\beta}
\newcommand{\sasb}{s_\alpha/s_\beta}
\begin{table}
  \[
    \begin{array}{lrrrr}
      \toprule
         & \text{Type I} & \text{Type II} & \text{Lepton-specific} & \text{Flipped}\\
         \midrule
      \xi_{hVV} & \sbma & \sbma & \sbma & \sbma \\
      \xi_{h}^u & \casb & \casb & \casb & \casb \\
      \xi_{h}^d & \casb & -\sacb & \casb & -\sasb \\
      \xi_{h}^l & \casb & -\sacb & -\sasb & -\casb \\
      \xi_{HVV} & \cbma & \cbma  & \cbma & \cbma \\
      \xi_{H}^u & \sasb & \sasb & \sasb & \sasb \\
      \xi_{H}^d & \sasb & \casb & \sasb & \casb \\
      \xi_{H}^l & \sasb & \casb & \casb & \sasb \\
      \xi_{AVV} & 0     & 0     & 0     & 0 \\
      \xi_{A}^u & 1/t_\beta & 1/t_\beta & 1/t_\beta & 1/t_\beta \\
      \xi_{A}^d & -1/t_\beta & t_\beta & -1/t_\beta & t_\beta \\
      \xi_{A}^l & -1/t_\beta & t_\beta & t_\beta & -1/t_\beta \\
      \bottomrule
\end{array}\]
\caption{List of the factors $\xi$ that determine the Yukawa couplings of the 2HDM Higgs bosons.}
\label{tab:xi_factors}
\end{table}

We will first consider the Yukawa interactions of the mass eigenstates with fermions, which are given by:
\begin{align*}
\mathcal{L}^{\mathrm{2HDM}}_{\text{Yukawa}} =& - \sum_{f = u, d, l} \frac{m_f}{v}
\left(\xi_h^f \overline{f}fh+\xi_H^f \overline{f}fH-i\xi_A^f \overline{f}\gamma_5fA \right)\\
&-\left\{\frac{\sqrt{2}V_{ud}}{v}\overline{u}\left(m_u\xi_A^uP_l+m_d\xi_A^dP_R\right)dH^+ + \frac{\sqrt{2}m_l\xi^l_A}{v}\overline{\nu}_Ll_RH^+ + h.c.\right\}
\end{align*}
where $f$ is a fermion with mass $m_f$, $u,d$ refer to up- and down-type quarks with masses $m_u,m_d$ and CKM mixing $V_{ud}$, $l$ is a lepton with mass $m_l$, and $\nu_L$ is a neutrino. 
As for the Yukawa couplings to the vector bosons, we can recover them by recalling that the standard model Higgs is given by
\begin{equation}
h_\text{SM} = hs_{\alpha-\beta}-Hc_{\alpha-\beta}.
\end{equation}
Thus the couplings of \emph{h} and \emph{H} become rescaled versions of standard model Higgs couplings to the vector bosons. If we represent these couplings by $g_{h_\text{SM}VV}$, where $VV$ can be $ZZ$ or $W^+W^-$, then the Yukawa couplings of the 2HDM Higgses are given by
\begin{align}
g_{\phi VV} = \xi_{\phi VV}g_{h_\text{SM}VV},
\end{align}
where $\phi = h/H/A$. The factors $\xi$ in the above expressions depend on the specific model being considered, and are listed in \autoref{tab:xi_factors}.

With a little bit of work (we omit the details for brevity), we can extract the following couplings (see \citep{Kling2016a} for details):
\begin{align*}
g_{\gamma H^+H^-} &= -ie(p_{H^+}-p_{H^-})^\mu\\
g_{ZH^+H^-} &= -i\frac{gc_{2\theta_w}}{2c_{\theta_w}}(p_{H^+}-p_{H^-})^\mu\\
g_{AH^\pm W^\mp} &= \frac{g}{2}(p_{H^+}-p_{A})^\mu\\
g_{hAZ} &= is_{\beta-\alpha}\frac{g}{2c_{\theta_w}}(p_A-p_h)^\mu\\
g_{hH^\pm W^\mp} &= -is_{\beta-\alpha}\frac{g}{2}(p_{H^\pm}-p_h)^\mu\\
g_{HAZ} &= ic_{\beta-\alpha}\frac{g}{2c_{\theta_w}}(p_A-p_H)^\mu\\
g_{HH^\pm W^\mp} &= -ic_{\beta-\alpha}\frac{g}{2}(p_{H^\pm}-p_H)^\mu\\
\end{align*}

The cubic and quartic Higgs couplings can also be worked out explicitly, but we will decline to do so and instead refer the reader to the general expressions in Appendix C of  \citep{Branco2012} and the specific expressions in section 2.3.3 of \citep{Kling2016a}. 

\section{The Type-II 2HDM}

As mentioned at the beginning of this chapter, The type-II 2HDM is of special interest since it has the same fermion-Higgs doublet coupling pattern as the MSSM. In fact, the MSSM can be viewed as a special case of a type-II 2HDM, one that incorporates supersymmetry. We will examine the MSSM in more detail in the next section, but will note that if we can recover the tree-level MSSM scalar potential from a type-II 2HDM by setting the parameters $\lambda_i$ to the following values:
\begin{align}
\lambda_{1,2} = \frac{g^2+g'^2}{2} &,& \lambda_3 = \frac{g^2-g'^2}{4} &,& \lambda_4 = -\frac{g^2}{2}&,&\lambda_{5,6,7} = 0.
\end{align}
It should be noted, however, that these relations do not hold beyond the tree-level for a generic non-supersymmetrized 2HDM.
