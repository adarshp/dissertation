\chapter{Light Charged Higgses at the LHC}\label{ch:LightChargedHiggs}

\def\h{h}
\def\H{H}
\def\A{A}
\newcommand{\sba}{\ensuremath{s_{\beta-\alpha}}}
\newcommand{\cba}{\ensuremath{\cos(\beta-\alpha)}}
\newcommand{\met}{\ensuremath{{\not\mathrel{E}}_T}}
\newcommand{\ifb}{\ensuremath{ \text{fb}^{-1} }}
\newcommand{\cmark}{\ding{51}}%
\newcommand{\xmark}{\ding{55}}%

In \autoref{sec:2HDMs}, we introduced a well-motivated class of models that serve as a framework for analyzing extended scalar sectors, known as Two-Higgs Doublet Models ($2$HDMs). In these models, the mass spectrum of the scalar sector consists of two CP-even Higgses \emph{h} and \emph{H}\footnote{Note that we use \emph{h} and \emph{H} to refer to the lighter or the heavier CP-even Higgs for models with two CP-even Higgs bosons. When there is no need to specify, we use \emph{H} to refer to the CP-even Higgses.}, a CP-odd Higgs \emph{A}, and a charged Higgs $H^\pm$. 

The discovery of one or more of these new particles would be a clear indication of an extended scalar sector, which can have important implications for electroweak symmetry breaking (EWSB). A number of experimental searches have been carried out to find these particles at the Large Electron-Positron Collider (LEP), the Tevatron experiment at Fermilab, and most recently, the LHC \cite{LEP_Higgs, Aad:2014vgg, Khachatryan:2014wca, TheATLAScollaboration:2013wia, CMS:2014cdp, Aad:2013hla, CMS:2014kga, Khachatryan:2015cwa}. These searches have been performed assuming that the new scalar states decay solely through conventional channels, that is, into SM particles. However, if they are kinematically allowed, `exotic' decay channels, in which a heavy Higgs boson decays into either a pair of lighter Higgses, or a Higgs and a gauge boson, can be the dominant decay channels in certain regions of parameter space, thus reducing the reach of the conventional search channels. Some of these channels have already been studied both theoretically \cite{Curtin:2013fra, Brownson:2013lka, Coleppa:2014hxa, Coleppa:2014cca,Li:2015lra,Dorsch:2014qja,Chen:2013emb,Chen:2014dma,Enberg:2014pua} and experimentally \cite{Aad:2015wra, CMS:2014yra,CMS:2013eua}. With the increase in center of mass collision energy of the LHC from 7 to 14 TeV, more of those exotic decay channels will become accessible at the LHC. It is therefore timely to study their reach more carefully.

In this study, we examine the prospects of detecting a light (that is, lighter than the top quark) charged Higgs boson that decays through the exotic decay channel
\[H^\pm\rightarrow (A/H)W^\pm\rightarrow (\tau\tau) W^\pm\]
and is produced via the decay of the top quark, $t\rightarrow H^+ b$.
The collider analysis in \cite{Coleppa:2014cca} studies the same exotic decay channel, but for a charged Higgs heavier than the top quark, produced in $H^\pm tb$ associated production.
The large rate of production of top quarks at the LHC should enable us to observe a sizable number of signal events. Current search strategies for light charged Higgses at the LHC assume that they decay either leptonically to the $\tau\nu$ final state, or hadronically to the $cs$ final state. The null search results obtained by both the ATLAS and CMS experiments exclude a light charged Higgs below a mass of about 160 GeV for most of the two-dimensional $m_{H^\pm}-t_{\beta}$ plane\sidefootnote{Here we use the abbreviated notations for trigonometric functions introduced in \autoref{sec:2HDMs}} in parameter space \cite{TheATLAScollaboration:2013wia,CMS:2014cdp}. However, if there exists a neutral Higgs (\emph{A/H}) light enough that the $H^{\pm} \rightarrow AW^{\pm}/HW^{\pm}$ channel is kinematically open, the branching ratios into the conventional final states $\tau\nu$ and $cs$ are suppressed and the exclusion bounds can be significantly weakened. Due to experimental challenges at low energies, such a light neutral Higgs has not been fully excluded yet. A relatively large region of $m_H^\pm > $ 150 GeV and $t_\beta \lesssim$ 20 is still allowed, while no limits exist for $m_H^\pm > $ 160 GeV.

The exotic decay channel of $H^\pm \to AW/HW$, on the other hand,  offers an additional opportunity for the detection of a light charged Higgs and closes the loophole of the current light charged Higgs searches.  While there are strong constraints on the mass of the light charged Higgs from flavor~\cite{Mahmoudi:2009zx,Coleppa:2013dya} and precision~\cite{Froggatt:1991qw,Froggatt:1992wt,Pomarol:1993mu,WahabElKaffas:2007xd,Haber:2010bw} observables, they are typically model dependent and could be relaxed when there are contributions from the other sectors of the model~\cite{Han:2013mga}. A direct search for a light charged Higgs, on the other hand, provides model-independent limits. It is thus worthwhile to fully explore the discovery and exclusion potential for the light charged Higgs at the LHC.

The rest of the chapter is structured as folllows. In \autoref{sec:motivation}, we briefly introduce the relevant couplings and branching ratios for our decay channel for the type II $2$HDM and highlight scenarios that permit a large branching ratio for the process $H^{\pm} \rightarrow AW/HW$. In \autoref{sec:limits}, we summarize the current experimental constraints on a light charged Higgs. In \autoref{sec:light_charged_analysis}, we present the details of our collider analysis, and in \autoref{subsec:ana_limits}, the model independent exclusion and discovery limits for the 14 TeV LHC with various luminosities.In \autoref{sec:implication}, we discuss the implications of our analysis for the Type II $2$HDM and translate our results into reaches in parameter space. We conclude in \autoref{sec:light_charged_conclusion}.
 
\section{Couplings and branching ratios} \label{sec:motivation}

\subsection{Decays of the top quark}
If the charged Higgs is light enough, the top quark decays primarily to the final states $W^\pm b$ or to $H^{\pm}b$. The first decay is controlled by the SM gauge coupling
\begin{equation}
g_{W^{\pm} tb} = \frac{g}{\sqrt{2}} \gamma^\mu \frac{1-\gamma_5}{2}, 
\end{equation}
where $g$ is the SM $SU(2)$ coupling. The second is a function of the parameter $t_{\beta}$ (see \autoref{sec:2HDMs} for details). More specifically, for a Type II $2$HDM, it takes the form (see \autoref{tab:xi_factors}) 
\begin{equation}
g_{H^{\pm} tb} = \frac{g}{2 \sqrt{2} m_W} \left[ (m_b t_{\beta} + m_t /t_\beta) \pm (m_b t_{\beta} - m_t /t_\beta ) \gamma_5 \right].
\end{equation}
\begin{marginfigure}[-1cm]
\centering
    \caption{Contours of the branching ratio of the top quark to the charged Higgs, in the $m_{H^{\pm}}-t_{\beta}$ plane.}
    \includegraphics[width=\textwidth]{./images/BR_tHpm.pdf}
    \legend{We see that the branching ratio reaches a minimum at $t_\beta = \sqrt{m_t/m_b} \approx $ 8.}
\label{fig:tBR}
\end{marginfigure}
\noindent This coupling is enhanced for both small and large values of $t_{\beta}$. In \autoref{fig:tBR}, we present contours of the branching ratio BR$(t \rightarrow H^{\pm} b)$ in the $m_{H^{\pm}}-t_{\beta}$ plane, calculated using the Two-Higgs Doublet Model Calculator ($2$HDMC) \cite{Eriksson:2009ws}. We see that it can reach values of 5\% and above for both large and small values of $t_{\beta}$, but reaches a minimum at $t_{\beta} = \sqrt{m_t/m_b}\sim$ 8. It also decreases rapidly as the charged Higgs mass approaches the mass of the top quark.

\subsection{Decays of the charged Higgs}
Conventionally, a light charged Higgs is assumed to either decay into the $\tau\nu$ or $cs$ final states, with the corresponding couplings given by
\begin{align*}
  g_{H^{\pm} \tau\nu} &= \frac{g}{2 \sqrt{2} m_W} m_\tau t_{\beta}(1\pm \gamma_5 ) \text{ and} \\
g_{H^{\pm} cs} &= \frac{g}{2 \sqrt{2} m_W} \left[ (m_s t_{\beta} + m_c /t_\beta ) \pm (m_s t_{\beta} - m_c /t_\beta ) \gamma_5 \right].
 \end{align*}
If there exists an additional neutral Higgs boson \emph{h} or $A$ that is lighter than the charged Higgs, then additional decay channels into $hW/AW$ open up. The relevant couplings are determined by the gauge structure, as well as the mixing angles $\alpha$ and $\beta$, taking the forms \cite{Gunion:1989we} 
\begin{align*}
  g_{H^{\pm} h W^{\mp}}= \frac{gc_{\beta-\alpha}}{2}(p_{\h}-p_{H^{\pm}})^\mu &&\text{~~ and }&&
 g_{H^{\pm} A W^{\mp}}= \frac{g}{2}(p_{A}-p_{H^{\pm}})^\mu,
 \end{align*}
 where $p_X^\mu$ is the incoming momentum for the corresponding particle \emph{X}. 

The $H^{\pm}\rightarrow hW$ channel for a light charged Higgs is open only if we demand the heavy CP-even neutral Higgs \emph{H} to be the SM-like Higgs. In this case $|c_{\beta-\alpha}| \sim$ 1 is preferred by experiments and the $H^{\pm}hW^{\pm}$ coupling is unsuppressed. The $H^{\pm} A W^{\pm}$ coupling is independent of $c_{\beta-\alpha}$ and always unsuppressed. In this scenario, the $H^{\pm} \rightarrow HW$ channel is kinematically forbidden.

In the generic $2$HDM, there are no mass relations between the charged scalars, the scalar and pseudoscalar states. Therefore, both the decays $H^{\pm} \rightarrow h W$ and $H^{\pm} \rightarrow A W$ can be accessible or even dominant in certain regions of the parameter space. It was shown in \cite{Coleppa:2013dya} that in the Type II $2$HDM with $\mathcal{Z}_2$ symmetry, imposing all experimental and theoretical constraints still leaves large regions in the parameter space that permit such exotic decays with unsuppressed decay branching ratios. 
\begin{marginfigure}[-12cm]
    \caption{Contours of the branching ratio for the decay $( H^{\pm} \rightarrow AW)$ in the Type II $2$HDM in $m_{H^\pm}-t_{\beta}$ plane.}
 	\includegraphics[width=\textwidth]{images/BR_HpmAW.pdf}
    \label{fig:br_hC_AW}
\legend{The branching ratio reaches large values for $t_\beta$ between 10 and 30, for $m_{H^\pm}$.}
\end{marginfigure}
\begin{marginfigure}[-1.5cm]
    \caption{The branching ratios of $H^{\pm} \rightarrow AW$ (red), $\tau\nu$ (green) and $cs$ (blue) as functions of $t_{\beta}$ for $m_{H^\pm} =$ 160 GeV.}
 	\includegraphics[width=\textwidth,trim=0 +0.05cm 0 0 ]{images/BR_Hpm_tanb_log.pdf}
\legend{The branching ratio BR$(H^\pm\rightarrow AW)$ dominates the branching ratios to $\tau\nu$ and $cs$ at low values of $t_\beta$.}
\label{fig:br_hC_AW_sm}
\end{marginfigure}

%Why mA = 70 GeV?
In \autoref{fig:br_hC_AW}, we show the contours of the branching ratio of the charged Higgs to the $AW$ final state in the $m_{H^{\pm}}-t_{\beta}$ plane, with $h$ being the SM-like Higgs and \emph{H} being heavy enough to be decoupled. This branching ratio reaches large values for $t_{\beta}$ between 10 and 30 and charged Higgs masses between 155 and 170 GeV. For small charged Higgs masses close to the $m_A + m_W$ threshold, the decay is kinematically suppressed. In \autoref{fig:br_hC_AW_sm}, we show the same branching ratio, but this time as a function of $t_\beta$, for a fixed charged Higgs mass of 160 GeV. For small values of $t_{\beta}$, the branching ratio of the $AW$ mode dominates that of the $\tau\nu$ mode. In both plots, we assume the existence of a pseudoscalar \emph{A} with a mass of 70 GeV. Similar results can be obtained for $H^{\pm} \rightarrow \h W$ with $m_{h} =$ 70 GeV,  $s_{\beta-\alpha}\sim$ 0 and \emph{A} decoupled. 

The MSSM Higgs mass spectrum is more restricted than that of a generic $2$HDM. At tree level, the mass matrix depends only on $m_A$ and $t_{\beta}$, and the mass of the charged Higgs mass is related to that of the pseudoscalar \emph{A} by the constraint $m_{H^{\pm}}^2 =m_A^2 + m_W^2$. Large loop corrections are therefore needed to increase the mass splitting enough to kinematically permit the decay $H^{\pm} \rightarrow AW$. In the non-decoupling region of the MSSM with \emph{H} being the SM-like Higgs, the CP-even Higgs \emph{h} can be light: $m_{h}<m_{H^{\pm}}-m_W$. The branching ratios can reach values up to 10\% \cite{Heinemeyer:2013tqa} in some regions of parameter space. In the NMSSM, where the Higgs sector is enlarged by an additional singlet, the decays $H^{\pm} \rightarrow A_iW/H_iW$ can be significant in certain regions of parameter space \cite{Christensen:2013dra,Drees:1999sb}.

\section{Literature review}\label{sec:limits}

Searches for a light charged Higgs boson have been performed by both ATLAS and CMS. The production mechanism considered is top pair production, with one top quark decaying to $bH^{\pm} $ while the other decays to $bW^\pm$. These studies focus on the $H^{\pm} \rightarrow \tau \nu$ decay channel, which is dominant in most regions of the parameter space in the absence of decays into lighter Higgses. Assuming that  BR$(H^{\pm} \rightarrow \tau \nu) =$ 100 \%, the null search results from CMS~\cite{CMS:2014cdp} imply upper bounds on the top quark branching ratio BR$(t \rightarrow H^{\pm} b)$ ranging from 1.2\% to 0.16\% for charged Higgs masses between 80 and 160 GeV. This result can be translated into bounds on the MSSM parameter space. The obtained exclusion limits for the MSSM $m_h^\text{max}$ scenario can be seen in \autoref{fig:lim_BR_Hpm_right} (region to the left of the red line). Only charged Higgs masses in the small region between 155 and 160 GeV around $t_{\beta} =$ 8 are still allowed. The results from ATLAS \cite{TheATLAScollaboration:2013wia} are similar.
A search with the $H^{\pm} \rightarrow cs$ decay channel has been performed by ATLAS \cite{Aad:2013hla} using 4.7 fb$^{-1}$ of integrated luminosity at 7 TeV and by CMS \cite{CMS:2014kga} using 19.7 fb$^{-1}$ of integrated luminosity at 8 TeV. Assuming BR$(H^{\pm} \rightarrow c s) =$ 100\%, the ATLAS results imply upper bounds on BR$(t \rightarrow b H^{\pm} )$ between 5\% to 1\% for charged Higgs masses between 90 and 150 GeV, while the CMS searches imply upper bounds between 2\% to 7\% for charged Higgs masses between 90 and 160 GeV.
\begin{marginfigure}[-15cm]
    \caption{Weakened CMS limits on the branching ratio of the top quark to the charged Higgs.}
 	\includegraphics[width=\textwidth,]{images/BR_tHpm_CMS.pdf}
    \legend{Here, we show the limit assuming a BR$(H^{\pm} \rightarrow \tau \nu) =$ 100\% (black line)~\cite{CMS:2014cdp}, as well as the weakened limits (green, blue, and red lines) in the Type II $2$HDM in the presence of a light neutral Higgs for various values of $t_{\beta}$.}
\label{fig:lim_BR_Hpm_left}
\end{marginfigure}
These limits get weaker once we consider realistic branching ratios smaller than 100\%. In \autoref{fig:lim_BR_Hpm_left}, we show how the CMS limits on the branching ratio BR$(t \rightarrow H^{\pm} b)$ can change significantly in the presence of an additional light neutral Higgs. The black curve shows the CMS limits presented in \cite{CMS:2014cdp} assuming a BR$(H^{\pm} \rightarrow \tau \nu) =$ 100\%. The modified limits assuming the existence of a 70 GeV CP-odd neutral Higgs are shown for $t_{\beta}=$ 1 (red), 7 (blue) and 50 (green). We can see that for large values of $t_{\beta}$, the limits stay almost unchanged, since $H^{\pm} \rightarrow \tau \nu$ is the dominant decay channel. However, for smaller values of $t_{\beta}$ these limits are weakened significantly.
\begin{marginfigure}[-5cm]
  \caption{Weakening of the CMS exclusion limits in the $m_{H^\pm}-t_\beta$.}
 	\includegraphics[width=\textwidth,trim=0 -0.1cm 0 0 ]{images/Limits_CMS.pdf}
    \legend{The combined yellow and cyan regions show the region excluded by CMS in the $m_{H^\pm}-t_{\beta}$ plane assuming a BR$(H^{\pm} \rightarrow \tau \nu) =$ 100\%. The weakened limits with a light neutral Higgs correspond to the cyan region.}
\label{fig:lim_BR_Hpm_right}
\end{marginfigure}
Similarly, \autoref{fig:lim_BR_Hpm_right} shows how the CMS limits in the $m_{H^{\pm}}-t_{\beta}$ plane weaken in the presence of an additional light Higgs. The union of the yellow and cyan regions is excluded by CMS assuming BR$(H^{\pm} \rightarrow \tau \nu) =$ 100\%, while only the cyan region is excluded assuming the branching ratios predicted by a Type II $2$HDM in the presence of a CP-odd neutral Higgs with a mass of 70 GeV. This weakening is most prominent in the low $t_\beta$ region - the lower limit on charged Higgs masses reduces to about 150 GeV for $t_{\beta}<$ 15. Therefore, the presence of exotic decay modes substantially weakens the current and projected future limits from searches based on the conventional $H^{\pm} \rightarrow \tau \nu, cs$ decay modes. 

A light charged Higgs could have a large impact on precision and flavor observables as well~\cite{Olive:2016xmw}. For example, the bounds on the rate of the decay $b\rightarrow s\gamma$ restrict the charged Higgs to be heavier than 300 GeV for a Type II $2$HDM. A detailed analysis of precision and flavor bounds on $2$HDMs can be found in \cite{Coleppa:2013dya,Mahmoudi:2009zx}. Flavor constraints on the Higgs sector are however typically model-dependent, and could be alleviated by contributions from other new particles in a particular model~\cite{Han:2013mga}. Since our focus in this work is on collider phenomenology, we consider light charged Higgses that satisfy the direct bounds set by collider searches for Higgs bosons.

Our study also assumes the existence of a light neutral Higgs $A/H$. This type of Higgs has been constrained by the $A/H \rightarrow \tau\tau$ searches at the LHC~\cite{Khachatryan:2014wca,Aad:2014vgg}, in particular, for $m_{A/H}>$ 90 GeV and relatively large values of $t_{\beta}$. No limit, however, exists for $m_{A/H}<$ 90 GeV due to the difficulties in the identification of the relatively soft taus and the overwhelming SM backgrounds for soft leptons and $\tau$-jets. Furthermore, LEP limits~\cite{LEP_Higgs} based on $VH$ associated production do not apply to the CP-odd $A$ or the non-SM like CP-even Higgs. The LEP limits based on \emph{AH} pair production also do not apply, as long as $m_A+m_H>$ 208 GeV. Therefore, in our analyses, we choose the daughter (neutral) Higgs mass to be 70 GeV\footnote{The mass of 70 GeV is also chosen to be less than half the mass of the SM Higgs (126 GeV). This prohibits the decay channel $h_\text{SM} \rightarrow AA$ and enforces consistency with the current measurements of branching ratios of the SM Higgs.}.

There have been other theoretical studies on light charged Higgses as well - the authors of \cite{Guedes:2012eu} and \cite{Hashemi:2013kga} analyze light charged Higgses produced via single top production and decaying to the $\tau\nu$ final state, while the authors of \cite{Hashemi:2011gy} and \cite{Das:2014fha} investigate light charged Higgses that are produced via the top pair production channel and decay to the $\mu\nu$ and $\gamma\gamma W$ final states.

Charged Higgses heavier than the top quark have been analyzed in detail in \cite{Coleppa:2014cca}, which considers the associated production channel $H^{\pm} tb$ and the exotic decay channel $H^{\pm}\rightarrow AW/HW^{\pm}$. Given that our study considers the same final state $bbWW(A/H)$ as the one in \cite{Coleppa:2014cca}, we can use a similar search strategy to find light charged Higgses that come from the decay of pair-produced top quarks.

\section{Collider analysis}\label{sec:light_charged_analysis}

In our analysis we study the exotic decay $H^{\pm} \rightarrow (A/H)W$ of light charged Higgs bosons $(m_{H^{\pm}}<m_t)$ produced via top decay. We consider two production mechanisms for the parent top quarks: $t$-channel single top production\footnote{We only consider the dominant $t$-channel single top mode since the $s$-channel mode suffers from a very small production rate and the $tW$ mode has a final state similar to that of the top pair production case.} ($tj$) and top pair production ($t\bar{t}$)~\cite{Kidonakis:2012db}.

The light neutral Higgs boson can either be the CP-even $H$ or the CP-odd $A$. In the analysis that follows, we use the decay $H^{\pm} \rightarrow A W^{\pm}$ as an illustration. Since we do not use angular correlations of the charged Higgs decay, the bounds obtained for the $H^{\pm} \rightarrow A W^{\pm}$ channel will apply to the $H^{\pm} \rightarrow H W^{\pm}$ channel as well.

The neutral Higgs boson $A$ itself will decay further. In this analysis we look at the decay $A \rightarrow\tau\tau$ for single top production and both the $\tau\tau$ and the $bb$ modes for top pair production. While the $bb$ mode has the advantage of a large branching ratio, the $\tau\tau$ case has smaller SM backgrounds and therefore leads to a cleaner signal. We study both leptonic ($\tau_\text{lep}$) and hadronic ($\tau_\text{had}$) decays of the tau lepton, and consider the following combinations of $\tau$ decays: $\tau_\text{had}\tau_\text{had}$, $\tau_\text{lep}\tau_\text{had}$ and $\tau_\text{lep}\tau_\text{lep}$. The $\tau_\text{lep}\tau_\text{had}$ case is particularly promising since we can utilize a same sign dilepton signal with the leptons from the decays of the $W$ and the $\tau$. 

We use Madgraph 5/MadEvent v1.5.11 \cite{Alwall:2014hca} to generate our signal and background events. These events are passed to Pythia v2.1.21 \cite{Sjostrand:2006za} to simulate initial and final state radiation, showering and hadronization. The events are further passed through Delphes 3.07 \cite{deFavereau:2013fsa} with the Snowmass combined LHC detector card \cite{Anderson:2013kxz} to simulate detector effects. The discovery reach and exclusion bounds have been determined using RooStats~\cite{Moneta:2010pm} and theta-auto \cite{theta_auto}. 

In this section, we will present model-\emph{independent} exclusion and discovery limits on $\sigma\times\text{BR}$ for both single top and top pair production with the final states $\tau\tau bW j$ and $\tau\tau bbWW$/$bbbbWW$. We consider parent charged Higgses with masses between 150 and 170 GeV and a daughter Higgs \emph{A} with a mass of 70 GeV. 

\subsection{Single Top Production}\label{subsec:light_charged_analysis_tj}
 
For single top production, we consider the channel
\[pp\rightarrow tj\rightarrow H^{\pm} bj\rightarrow AW^{\pm}bj \rightarrow \tau\tau W bj.\]
The dominant SM backgrounds are $W\tau\tau$ production, which we generate with up to two additional jets (including \emph{b} jets), and top pair production with both fully leptonic and semileptonic decay chains, which we generate with up to one additional jet. We also take into account the SM backgrounds $tj\tau\tau$ and $ttll$ with $l= (e, \mu, \tau)$. Our approach is to perform a traditional `cut-and-count' analysis, with the following selection cuts.

\begin{enumerate}

\item \emph{Identification cuts:} 

  \begin{itemize}
    \item \emph{Case A} ($\tau_\text{had}\tau_\text{had}$): We select events with exactly one lepton (\emph{e} or $\mu$), two $\tau$ tagged jets, fewer than two $b$ tagged jets, and at least one untagged jet. We also require the $\tau$-tagged jets to have opposite-sign charges.

\item \emph{Case B} ($\tau_\text{lep}\tau_\text{had}$): Same as case A, except that we require exactly two leptons instead of one. Additionally, we require that both of these leptons have charges of the same sign, which is opposite to the sign of the $\tau$ tagged jet.
\item \emph{Case C} ($\tau_\text{lep}\tau_\text{lep}$): Same as case A, except that we require exactly three leptons instead of one. No constraints are placed on the signs of the charges for this case.
\end{itemize}
The following criteria were also required for the identification of leptons, and jets: $|\eta_{l,b,\tau}| <$ 2.5, $|\eta_{j}| <$~5, $p_{T}(l_1, j, b) >$~20 GeV and $p_{T}(l_{2}) >$~10 GeV.

\item \emph{Neutrino reconstruction:} 
We reconstruct the momentum of the neutrino using the missing transverse momentum and the momentum of the hardest lepton as described in \cite{Aad:2012ux}, assuming that the missing energy is solely from the decay $W\rightarrow l \nu$.

\item \emph{Neutral Higgs candidate:} The $\tau$ jets (case A), the $\tau$ jet and the softer lepton (case B) or the two softer leptons (case C) are combined to form the neutral Higgs candidate. The reconstruction of the neutrino and the neutral Higgs candidate are relatively poor in cases B and C due to additional missing energy from the neutrino associated with the leptonic $\tau$ decay. 

\item \emph{Charged Higgs candidate:} The momenta of the neutral Higgs candidate, the reconstructed neutrino and the hardest lepton are combined to form the charged Higgs candidate. 

\item \emph{Mass cuts:} We place upper limits on the masses of the charged and neutral Higgs candidates, optimized for each mass combination. For $m_{H^{\pm}}=$160 GeV and $m_{A}=$ 70 GeV, we impose the cuts $m_{\tau\tau} <$ 48 GeV, and $m_{\tau\tau W} <$ 148 GeV.

\begin{marginfigure}
\centering
\caption{Normalized distributions of $\cos\theta^*$ (top) and the transverse momentum of the $tj$ system $p_{T}(tj)$ (bottom), for case A with $m_{H^{\pm}}=$ 160 GeV.}
\subbottom[$\cos\theta^*$]{\includegraphics[trim = {0 0 1.5cm 0}, clip, width = \textwidth]{images/angle_separated_rotated}\label{sf:angular_correlations}}
\subbottom[$p_T(tj)$]{\includegraphics[trim = {1.5cm 0 0 0}, clip, width = \textwidth]{images/pt_separated_rotated}\label{sf:tj}}
\legend{The imposed cuts are indicated by the vertical dashed lines.}
\end{marginfigure}

\item \emph{Angular correlation:} The angle $\theta^{*}$ between the momentum of the top quark in the \emph{tj} system's rest frame and the momentum of the \emph{tj} system in the lab frame acts as a unique kinematical signature of single top production \cite{Kling:2012up}. The differential distribution for $\cos\theta^{*}$ is shown in figure \autoref{sf:angular_correlations} for the signal (red), and the backgrounds $t\bar{t}$ (blue) and $W\tau\tau$ (green). The signal distribution tends to peak around $\cos\theta^* \approx$ -1 while the background distributions are flat. If the top quark could be reliably identified, the $\cos\theta^{*}$ distribution for the $t\bar{t}$ background would peak around $\cos\theta^{*}=$ 1, as shown in \cite{Kling:2012up}. However, in this study we approximate the top quark momentum by the momentum of the charged Higgs candidate, which results in a flat distribution of $\cos\theta^*$ for the $t\bar{t}$ system. In our analysis we require $\cos\theta^*<$ -0.8.
 
\item\emph{Top and recoil jet system momentum:} In single top production, we expect that the transverse momentum of the top quark and recoil jet should balance each other, as shown in figure \autoref{sf:tj} by the red curve. We require that the transverse momentum of the $tj$ system, $p_{T}(tj)$, must be less than 30 GeV. This further suppresses the top pair production background in the presence of additional jets coming from the second top quark.
\end{enumerate}
In \autoref{tab:tj}, we show a representative cut flow table for a signal benchmark point with $m_{H^{\pm}} =$ 160 GeV and $m_A =$ 70 GeV at the 14 TeV LHC. The first row shows the total cross section before cuts, calculated using MadGraph. The following rows show the cross sections for the signal and major backgrounds after applying the selection cuts discussed above. We have chosen a nominal value of 100 fb for $\sigma \times \text{BR}( p p \rightarrow H^{\pm} b j \rightarrow \tau \tau W bj)$ to illustrate the cut efficiencies\footnote{This is consistent with the branching ratio of the top quark into a charged Higgs with mass 160 GeV in a Type II $2$HDM, which is typically between 0.1\% and 1\% (see \autoref{fig:tBR}). Using the single top production cross section from \cite{Kidonakis:2012db}, $\sigma_{tj}=248$ pb, and assuming the branching ratios BR$(H^{\pm} \rightarrow AW^{\pm}) =$ 100\% and BR$(A \rightarrow \tau\tau)=$ 8.6 \% leads to the stated $\sigma\times$ BR of between 21 and 210 fb.}.

\begin{table}
\centering
\strictpagecheck
\begin{adjustwidth*}{0in}{-1.1in}
\caption{Representative cut flow table for the benchmark point with $m_{H^{\pm}}$ = 160 GeV and $m_A$ = 70 GeV at the 14 TeV LHC. (Single top production)}
  \begin{tabular}{clrrrrrr}
    \toprule
 Case   & Cut                           & $\sigma_\text{Signal}$ & $\sigma_{W(W)\tau\tau}$ & $\sigma_{t\bar{t}}$ & $\sigma_{tj\tau\tau/ttll}$ & $S/B$ & $S/\sqrt{B}$\\\midrule
        &  Original                             & 100                    & 2000                    & 6.3 $\times$ 10$^5$ & 257                        & -     & -	\\\midrule
    A   & Identification                & 0.29                   & 5.36                    & 130                 & 1.39                       & 0.002 & 0.43	\\
        & Mass cuts                     & 0.16                   & 0.34                    & 2.62                & 0.04                       & 0.05  & 1.55	\\
        & $\cos\theta^*$ and $p_{T,tj}$ & 0.07                   & 0.03                    & 0.07                & 0.001                      & 0.67  & 3.72	\\\midrule
    B   & Identification                & 0.25                   & 4.45                    & 2.46                & 1.33                       & 0.03  & 1.51	\\
        & Mass cuts                     & 0.11                   & 0.31                    & 0.20                & 0.05                       & 0.19  & 2.48	\\
        & $\cos\theta^*$ and $p_{T,tj}$ & 0.06                   & 0.04                    & 0.02                & 0.002                      & 0.91  & 3.99	\\\midrule
    C   & Identification                & 0.18                   & 3.07                    & 6.77                & 6.74                       & 0.01  & 0.78 \\
        & Mass cuts                     & 0.12                   & 0.55                    & 0.94                & 0.28                       & 0.07  & 1.63 \\
        & $\cos\theta^*$ and $p_{T,tj}$ & 0.07                   & 0.08                    & 0.10                & 0.01                       & 0.38  & 2.84 \\
\bottomrule
\end{tabular}
\legend{All cross sections are given in femtobarns. We have chosen a nominal value for $\sigma \times \text{BR}(pp \rightarrow tj \rightarrow H^{\pm} j b \rightarrow \tau\tau W b j )$ of 100 fb to illustrate the cut efficiencies for the signal process. The significance, $S/\sqrt{B}$ is calculated for an integrated luminosity of 300 fb$^{-1}$.}
\label{tab:tj}
\end{adjustwidth*}
\end{table}

We can see that the dominant backgrounds after particle identification are $t\overline{t}$ for cases A and C, and $W\tau\tau$ for case B. The reach is slightly better in case B since the same sign dilepton signature can efficiently reduce the $t\bar{t}$ background. Nevertheless, soft leptons from underlying events or $b$-decay can mimic the same sign dilepton signal. The obtained results are sensitive to the $\tau$ tagging efficiency as well as the misidentification rate. In our analyses, we have used a $\tau$ tagging efficiency of 60\% and a mistagging rate of 0.4\%, as suggested in \cite{Anderson:2013kxz}. A better rejection of non-$\tau$ initiated jets would increase the significance of this channel. 

%++++++++++++++++++++++++++++++++++++++++++++++++++++++++++++++++++++++++
\subsection{Top Pair Production}
 \label{sec:light_charged_analysis_tt}

We now turn to the top pair production channel, 
\begin{equation}
pp \rightarrow tt \rightarrow H^{\pm} tb \rightarrow AbbWW\rightarrow \tau\tau bbWW/bbbbWW.
\end{equation} 
As mentioned earlier, a detailed collider study with the same final states has been performed in \cite{Coleppa:2014cca} with a focus on high charged Higgs masses. We adopt the same strategy for the light charged Higgs case and refer to \cite{Coleppa:2014cca} for the details of the analysis. 

To analyze this channel, we consider the same decay modes for the neutral Higgs as we did for the single top production case: $\tau_\text{had}\tau_\text{had}$, $\tau_\text{had}\tau_\text{lep}$, $\tau_\text{lep}\tau_\text{lep}$, along with the additional mode $A\rightarrow bb$. We require that one of the two $W$ bosons from the decays of the charged Higgses decay leptonically, and the other hadronically, to strike a balance between reducing backgrounds and keeping a substantial production cross section.
 
The dominant SM background for the $\tau\tau$ channel is top pair production with  semi and fully leptonic decay chains. We also take into account $ttll$ production with $l = (e, \mu, \tau)$, as well as $W\tau\tau$ and $WW\tau\tau$. We ignored the subdominant backgrounds from single vector boson production, \emph{WW}, \emph{ZZ}, single top production, as well as multijet QCD background. Those backgrounds are either small or can be sufficiently suppressed by the cuts imposed. Similar backgrounds are considered for the \emph{bb} process. 

In \autoref{tab:tt}, we show a representative cut flow table for a signal benchmark point with $m_{H^{\pm}}=$ 160 GeV and $m_A=$ 70 GeV and the decay $A\rightarrow\tau\tau$ at the 14 TeV LHC, similar to \autoref{tab:tj}. We have chosen a nominal value for $\sigma \times \text{BR}(pp \rightarrow tt \rightarrow H^{\pm} tb \rightarrow \tau\tau bb WW)$ of 1000 fb to illustrate the cut efficiencies for the signal process. 

After the cuts, the dominant backgrounds are $t\bar{t}$ (for the $\tau_\text{had}\tau_\text{had}$, $\tau_\text{lep}\tau_\text{lep}$ cases) as well as $t\bar{t}ll$ (for the $\tau_\text{had}\tau_\text{lep}$ cases). Other backgrounds including vector bosons do not contribute much. Similar to the single top case, we find that the $\tau_\text{lep}\tau_\text{had}$ case gives the best reach.

\begin{table}
\centering
\begin{adjustwidth*}{0in}{-0.8in}
  \caption{Representative cut flow table for the same benchmark point as \autoref{tab:tj} and the decay mode $A\rightarrow\tau\tau$. (Top pair production)}
\begin{tabular}{clrrrrrr}
  \toprule
  Case    & Cut                                & $\sigma_\text{Signal}$ & $\sigma_{t\bar{t}}$ & $\sigma_{t\bar{t}ll}$ & $\sigma_{W(W)\tau\tau}$ & $S/B$ & $S/\sqrt{B}$	\\\midrule
          & Original                           & 1000                   & 6.3 $\times$ 10$^5$ & 247                   & 2000                    & -     & - \\\midrule
  A       & Identification                     & 4.1                    & 23.3                & 0.58                  & 0.078                   & 0.17  & 14.9		\\
          & $m_{\tau\tau}$ vs $m_{\tau\tau W}$ & 0.6                    & 0.31                & 0.021                 & 0.003                   & 1.9   & 18.8		\\\midrule
  B       & Identification                     & 3.3                    & 0.35                & 0.697                 & 0.072                   & 3.0   & 55.3		\\
          & $m_{\tau\tau}$ vs $m_{\tau\tau W}$ & 0.69                   & 0.035               & 0.042                 & 0.007                   & 8.1   & 41.1		\\\midrule
  C       & Identification                     & 3.1                    & 2.35                & 5.11                  & 0.058                   & 0.41  & 19.9		\\
          & $m_{\tau\tau}$ vs $m_{\tau\tau W}$ & 0.62                   & 0.25                & 0.16                  & 0.006                   & 1.4   & 16.5		\\\midrule
\end{tabular}
\legend{We have chosen a nominal value for $\sigma \times \text{BR}(pp \rightarrow tt \rightarrow H^{\pm} tb \rightarrow \tau\tau bbWW)$ of 1000 fb to illustrate the cut efficiencies for the signal process. The significance, $S/\sqrt{B}$ is calculated for an integrated luminosity of 300 fb$^{-1}$. See \cite{Coleppa:2014cca} for the details of the selection cuts used.}
\label{tab:tt}
\end{adjustwidth*}
\end{table}

\subsection{Model-independent limits}\label{subsec:ana_limits}

\begin{marginfigure}[-2.8in]
  \caption{Exclusion and discovery limits on $\sigma\times$BR$(t\rightarrow H^+b)$}
  \subbottom[Single top production]{\includegraphics[width=\textwidth]{images/lim_comb_tj.pdf}}
  \subbottom[Top pair production]{\includegraphics[width=\textwidth]{images/lim_comb_tt.pdf}}
\legend{The upper panel represents the single top channel and the lower panel represents the top pair production channel. The vertical axis on the left represents the limit on the quantity $\sigma\times$BR, while the vertical axis on the right represents the corresponding limits on the branching ratio of the top quark to the charged Higgs boson.}
  \label{fig:ana_limits}
\end{marginfigure}

Figure~\ref{fig:ana_limits} displays the exclusion (green) and discovery (red) limits at the 14 TeV LHC for the single top production and top pair channels, assuming BR$(H^\pm\rightarrow AW^\pm) =$ 100\% and BR$(A\rightarrow\tau\tau) =$ 8.6\%, and a systematic error of 10\%. The dot-dashed, solid and dashed lines show the results for integrated luminosities of 100, 300, and 1000 fb$^{-1}$ respectively. The left vertical axis shows limits on $\sigma\times$BR, and the right vertical axis shows the corresponding limits on BR$(t\rightarrow H^+ b)$. In these plots, we have combined all three cases of $\tau$ decays. While in the single top channel, cases A, B, and C contribute roughly the same to the overall significance, the sensitivity achieved in case B is substantially greater than in cases A and B in the top pair production channel. Due to the small number of events in both channels, the statistical error dominates over the systematic error in the background cross sections. Therefore, higher luminosities lead to better reaches. Assuming 300 fb$^{-1}$ of integrated luminosity, the exclusion limits on $\sigma\times\text{BR}$ are about 35 and 55 fb for the single top and top pair production processes respectively. The discovery reaches are about three times higher. 

Assuming BR$(H^{\pm} \rightarrow A W )=$ 100\% and BR$(A \rightarrow \tau\tau)=$ 8.6\%\footnote{Assuming $bb$ and $\tau\tau$ are the dominant decay modes of a light $A$, BR$(A \rightarrow \tau\tau)=8.6$\% in a Type II $2$HDM or the MSSM at medium to large $t_{\beta}$. This branching ratio decreases for small $t_{\beta}$ when the $cs$-channel is enhanced.}, we can reinterpret the limits on $\sigma\times\text{BR}$ as limits on the $\text{BR}(t \rightarrow H^{\pm} b)$, as indicated by the vertical axes on the right. While the limits on the cross section are better in the single top channel, the corresponding limits on $\text{BR}(t \rightarrow H^{\pm} b)$ are weaker due to the smaller cross section for single top production. The lowest achievable upper limits on BR$(t \rightarrow H^{\pm} b)$ are about 0.2\% for the single top process and 0.03\% for the top pair production process, respectively.

A study of the $A \rightarrow bb$ decay using the top pair production channel leads to worse results due to the significantly higher SM backgrounds. For the 14 TeV LHC with 300 fb$^{-1}$, the exclusion limit on $\sigma \times \text{BR}$ is about 7 pb for a charged Higgs with a mass of 160 GeV, assuming the existence of a light neutral Higgs with a mass of 70 GeV. Thus, given the typical ratio 
\[\frac{\text{BR}(A/H \rightarrow bb)}{\text{BR}(A/H\rightarrow \tau\tau)} \sim \frac{3m^2_b}{m^2_{\tau}},\]
we conclude that the reach in the $bb$ case is much worse than that of the $\tau\tau$ case.

We reiterate here that the limits on $\sigma \times \text{BR}$ are completely model independent. Whether or not discovery/exclusion is actually feasible in this channel should be answered within the context of a particular model, in which the theoretically predicted cross sections and branching ratios can be compared with the model-independent limits. We will do this in the following section for a Type II $2$HDM. 

\section{Implications for a Type II 2HDM}\label{sec:implication}

Two-Higgs Doublet Models allow us to interpret the observed Higgs signal either as the lighter CP-even Higgs ($h$-126) or the heavier CP-even Higgs ($H$-126). The authors of \cite{Coleppa:2013dya} have identified the regions of the parameter space of Type II $2$HDMs that survives all the existing experimental and theoretical constraints in both of these cases, assuming $m_{12}^2=$ 0. In the $h$-126 case, we are restricted to either a SM-like region at $s_{\beta-\alpha}=\pm$1 with $t_{\beta}<$ 4 or an extended region with 0.6 $<s_{\beta-\alpha}<$ 0.9 and 1.5 $<t_{\beta}<$ 4 with relatively unconstrained masses. In the $H$-126 case, a SM-like region around $s_{\beta-\alpha}=$ 0 and t$_{\beta}<$ 8, and an extended region with -0.8 $< s_{\beta-\alpha}<$ 0.05 and $t_{\beta}$ up to 30 or higher, survive all constraints. 
 
We can interpret the results of the previous section in two ways: the light neutral Higgs that the charged Higgs decays to could be either the CP-even Higgs $h$ or the CP-odd Higgs $A$. The decay mode $H^{\pm} \rightarrow HW$ is not possible given that $m_{H}\geq 126$ GeV. The decay $H^{\pm} \rightarrow AW$ is possible in both the $h$-126 and $H$-126 cases and the partial decay width is independent of $s_{\beta-\alpha}$. However, the branching ratio does depend on whether the channel $H^{\pm} \rightarrow \h W$ is open or not. For simplicity, we choose a representative benchmark point BP1, with $\left\{ {m_{H^{\pm}},m_{\A},m_{\h},m_{\H}}\right\}=$ \{160, 70, 126, 700\} such that only the decay mode $H^{\pm} \rightarrow AW$ is kinematically accessible. The decay width $H^{\pm} \rightarrow h W$ depends on $s_{\beta-\alpha}$ and is only sizable in the \emph{H}-126 case. We illustrate this case with a second benchmark point, BP2: $\left\{ {m_{H^{\pm}},m_{\A},m_{\h},m_{\H}}\right\}=$ \{160, 700, 70, 126\}, assuming that the CP-odd Higgs $A$ decouples. We list the benchmark points in \autoref{tab:classification}. 

\begin{table}
  \begin{sidecaption}{Benchmark points used for illustrating the discovery and exclusion limits in the context of the Type II $2$HDM. Also shown are the typical favored regions of $s_{\beta-\alpha}$ for each case (see \cite{Coleppa:2013dya}). } 
\begin{center}
 \begin{tabular}{ccccl}
 \toprule
 Masses (GeV) & \multicolumn{2}{c}{Kinematically allowed?} & Favored\\ \cmidrule{2-3}
 $\left\{ {m_{H^{\pm}},m_{\A},m_{\h},m_{\H}}\right\}$ & $H^{\pm}\rightarrow\A W$ & $H^{\pm}\rightarrow\h W$ & Region\\\midrule
  BP1: \{160, 70, 126, 700\} & Yes & No & $\sba\approx\pm$ 1 \\ \midrule
 BP2: \{160, 700, 70, 126\} & No & Yes & $\sba\approx$ 0 \\
 \bottomrule
 \end{tabular}
\end{center}
\end{sidecaption}
\label{tab:classification}
\end{table}


In \autoref{fig:imp_BP1_BR}, we show the branching faction BR$(H^{\pm} \rightarrow AW)$ for BP1, which is independent of $s_{\beta-\alpha}$ and decreases with increasing $t_{\beta}$ due to the enhancement of the $\tau\nu$ mode. The branching ratio can reach values of 90\% or larger for small $t_{\beta}<$ 4 and stays the dominant channel until $t_{\beta}=$ 12. 

\begin{marginfigure}[-3in]
 {\color{gray}\hrule}
 \vspace{\onelineskip}
 \caption{Contours of branching ratios for the benchmark point BP1.}
 \includegraphics[width=\textwidth]{images/BP1_BR.pdf}
 \legend{The branching ratio is independent of $s_{\beta-\alpha}$, and is dominated by the $\tau\nu$ mode at high values of $t_\beta$. For small $t_\beta$ less than 12, it is the dominant channel, reaching values of 90\% or larger for $t_\beta <$ 4.}
\label{fig:imp_BP1_BR}
\end{marginfigure}

\begin{marginfigure}[-0.5cm]
 \centering
 {\color{gray}\hrule}
 \vspace{\onelineskip}
\caption{Contours of branching ratios BR$(H^\pm\rightarrow hW^\pm)$ for the benchmark point BP2.}
\includegraphics[width=\textwidth]{images/BP2_BR.pdf}
\legend{This branching ratio reaches its maximum at $s_{\beta-\alpha = 0}$, since larger values of $|s_{\beta-\alpha}|$ lead to the suppression of the $H^\pm hW^\pm$ coupling.}
\vspace{\onelineskip}
{\color{gray}\hrule}
\label{fig:imp_BP2_BR}
\end{marginfigure}

Figure \ref{fig:imp_BP2_BR} shows the branching ratio, BR$(H^{\pm} \rightarrow hW)$, for BP2. It reaches maximal values around $s_{\beta-\alpha}=0$ and decreases for larger $|s_{\beta-\alpha}|$ compared to BP1 due to the suppressed $H^{\pm} hW$ coupling. 

\begin{figure}
 \centering
 {\color{gray}\hrule}
 \vspace{\onelineskip}
 \caption{The exclusion (yellow and cyan regions combined, bounded by dashed lines) and discovery (cyan region only bound by solid lines) reach achievable in the $t_{\beta}$ versus $\sba$ plane for the benchmark points 1 and 2, with an integrated luminosity of 300 fb$^{-1}$ at the 14 TeV LHC.}
 \subbottom[Benchmark point 1]{\includegraphics[width=0.49\textwidth]{images/BP1_Lim.pdf}}
 \subbottom[Benchmark point 2]{\includegraphics[width=0.49\textwidth]{images/BP2_Lim.pdf}}
 \legend{The red lines represent the reach achievable by the top pair production channel, while the blue lines represent the reach achivable by the single top channel.}
 \legend{Erratum: The plot on the right is labeled $H^\pm\rightarrow AW$, when it should actually be $H^\pm\rightarrow hW$.}
\label{fig:imp_BP_Lim}
 \vspace{\onelineskip}
{\color{gray}\hrule}
\end{figure}

In \autoref{fig:imp_BP_Lim}, we display the regions that can be excluded (yellow regions enclosed by the solid lines as well as the cyan regions) and discovered (cyan regions enclosed by the dashed lines) for BP1 (left panel) and BP2 (right panel) at the 14 TeV LHC with 300 fb$^{-1}$ integrated luminosity. The red lines represent the limits from the top pair production channel, and the blue lines represent the limits from the single top production channel.

For the benchmark point BP1 with $H^{\pm} \rightarrow AW^{\pm}$, the exclusion reach based on top pair production covers the entire parameter space, while discovery is possible for small $t_{\beta}<$ 6 and large $t_{\beta}>$ 18, independent of $\sba$. Intermediate values of $t_{\beta}$ have a reduced branching ratio BR$(t \rightarrow H^{\pm} b)$ (see \autoref{fig:tBR}) and therefore the total $\sigma\times$BR is suppressed. At high values of $t_{\beta}$, BR$(t \rightarrow H^{\pm}b)$ is enhanced enough to overcome the reduced branching ratio BR$(H^{\pm} \rightarrow AW)$. The search based on single top production is only effective in the small $t_{\beta}$ region, with an exclusion reach of $t_{\beta} <$ 4 and a discovery reach of $t_{\beta}<$ 2. 

The right panel of \autoref{fig:imp_BP_Lim} shows the reach for BP2. The exclusion region for top pair production covers the entire parameter space except for $|s_{\beta-\alpha}| >$ 0.85 and $t_{\beta}>$ 4. Discovery is possible for large $t_{\beta}>$ 18 with $|s_{\beta-\alpha}|<$ 0.5 and for small $t_{\beta}<$ 6. The reach from the single top production channel is limited to the small $t_{\beta}$ region.
\begin{figure}
  \begin{sidecaption}{Exclusion (yellow regions bounded by solid lines as well as the cyan regions) and discovery (cyan regions bounded by the dashed lines) imposed by the $tj$-channel (blue) and $tt$-channel (red) in the $m_{H^{\pm}}-t_{\beta}$ parameter space for 300 fb$^{-1}$ luminosity with $m_\A=$ 70 GeV. The same limits apply for $m_{h}=$ 70 GeV and $s_{\beta-\alpha}=$ 0 if \emph{A} is decoupled. The black hatched region indicates the region excluded by the CMS search based on $H^{\pm} \rightarrow \tau \nu$~\cite{CMS:2014cdp}.}[fig:imp_Limits]
 \centering
 	\includegraphics[width=0.7\textwidth]{images/LIMITS_AWCHANNEL.pdf}
  \end{sidecaption}
 \vspace{\onelineskip}
{\color{gray}\hrule}
\end{figure}

In \autoref{fig:imp_Limits}, we show the reach in the $m_{H^{\pm}}-t_{\beta}$ plane for $H^{\pm} \rightarrow AW$ with $m_A=$ 70 GeV with decays of $H^\pm$ to both $\h$ and $\H$ kinematically disallowed. These limits also apply for $H^{\pm} \rightarrow hW$ with $m_{h}=$ 70 GeV and $s_{\beta-\alpha}=$ 0 with a decoupled $A$. We display the exclusion (yellow regions enclosed by the solid lines as well as the cyan regions) and discovery limits (cyan regions enclosed by the dashed lines) for an integrated luminosity of 300 fb$^{-1}$ at the 14 TeV LHC. Superimposed are the current CMS limits (black hatched region)~\cite{CMS:2014cdp} which exclude the large $t_{\beta}$ region for $m_{H^\pm}<$ 160 GeV. 

The best reach is obtained by the top pair channel (region enclosed by the red lines). Charged Higgs masses can be excluded up to 167 GeV for all values of $t_{\beta}$, and up to 170 GeV for $t_{\beta}<$ 4 or $t_{\beta}>$ 29. Discovery is possible for both low $t_{\beta}<$ 6 with $m_{H^\pm}$ between 150 and 170 GeV and high $t_{\beta} >$ 17 with $m_{H^\pm}$ between 155 and 165 GeV. The reach is weakened for intermediate values of $t_{\beta}$ due to the reduced branching ratio BR$(t \rightarrow H^{\pm} b)$. The single top channel (region enclosed by the blue lines) only provides sensitivity in the low $t_{\beta}$ region, permitting exclusion for $t_{\beta} \lesssim$ 4 and discovery for $t_\beta < $ 3. 

We conclude this section with the following observations: 
\begin{itemize}
\item Both the $H^{\pm} \rightarrow A W$ channel (for the $\h$-126 case) and the $H^{\pm} \rightarrow \h W$ channel (for the $\H$-126 case) permit exclusion and discovery in large regions of the parameter space.

\item While the top pair production channel covers a large region of parameter space, the single top channel permits discovery/exclusion only in the low $t_{\beta}$ region.  
\end{itemize} 

%++++++++++++++++++++++++++++++++++++++++++++++++++++++++++++++++++++++++
\section{Conclusion}
\label{sec:light_charged_conclusion}
After the discovery of the first fundamental scalar by both the ATLAS and CMS collaboration, it is now time to carefully measure its properties to determine its nature. Current measurements still permit the possibility that the discovered signal is not the SM Higgs particle, but just one scalar particle contained in a larger Higgs sector, predicted by many extensions of the SM.

In this work we consider the prospects of discovering a charged Higgs lighter than the top quark, produced via top decay $(t \rightarrow H^{\pm} b)$. Due to the large single top and top pair production cross sections at the LHC, such a charged Higgs can be produced copiously. Assuming that the light charged Higgs predominantly decays into $\tau\nu$, both ATLAS and CMS exclude a light charged Higgs for most regions of the MSSM and Type II $2$HDM parameter spaces. However, the branching ratio BR$(H^{\pm} \rightarrow \tau\nu)$ can be significantly reduced once the exotic decay channel into a light Higgs $(H^{\pm} \rightarrow AW/HW)$, is open. This weakens the exclusion bounds from the $\tau\nu$ search, in particular for small and intermediate $t_{\beta}$, leaving the possibility of a light charged Higgs open. This loophole, however, can be closed when we consider the exotic charged Higgs decay channel $H^\pm \to AW/HW$.

Our analysis examines the decay mode $H^{\pm} \rightarrow AW/HW$ decay mode assuming that the lighter Higgs $A/H$ decays into either $\tau\tau$ or $bb$. While the top pair channel benefits from a large production cross section, the single top channel permits a cleaner signal due to its unique kinematic features. Assuming the existence of a light neutral Higgs with a mass of 70 GeV, the model independent exclusion limits on $\sigma\times$BR based on the $\tau\tau$ channel are approximately 35 fb for the single top channel and 55 fb for the top pair channel. The discovery reaches are about three times higher. Assuming $\text{BR}(H^{\pm} \rightarrow AW/HW)=$ 100\% and $\text{BR}(A/H \rightarrow \tau\tau)=$ 8.6\%, the exclusion limits on $\text{BR}(t \rightarrow H^+ b)$ are about 0.2\% and 0.03\% for single top and top pair production, respectively. A significantly worse reach is obtained in the $bb$ channel. 

We discuss the implications of the obtained exclusion and discovery bounds in the context of the Type II $2$HDM, focusing on two scenarios: the decay $H^{\pm} \rightarrow AW$ with a light $A$ in the $\h$-126 case and the decay $H^{\pm} \rightarrow \h W$ in the $\H$-126 case.  The top pair channel provides the best reach and permits discovery for both large $t_{\beta}>$ 17 around $m_{H^{\pm}}=$ 160 GeV and small $t_{\beta}<$ 6 over the entire mass range, while exclusion is possible in the entire $t_{\beta}$ versus $m_{H^{\pm}}$ plane except for charged Higgs masses close to the top quark mass threshold. The single top channel is sensitive in the low $t_{\beta}$ region and permits discovery for $t_{\beta}<$ 3. In particular, the low $t_{\beta}$ region is not constrained by searches in $\tau\nu$ channel, making $H^{\pm} \rightarrow AW/\h W$ a complementary channel for charged Higgs searches.

While most of the searches for additional Higgs bosons have focused on conventional decay channels, searches using exotic decay channels have started to garner additional interest~\cite{Curtin:2013fra, Brownson:2013lka, Coleppa:2014hxa, Coleppa:2014cca,Li:2015lra,Dorsch:2014qja,Chen:2013emb,Chen:2014dma,Enberg:2014pua,CMS:2014yra,Aad:2015wra,CMS:2013eua}. Studying all of the possibilities for the non-SM Higgs decays will allow us to exploit the full potential of the LHC and future colliders to improve our understanding of the nature of electroweak symmetry breaking.
\paragraph{Acknowledgments}
We would like to thank Baradhwaj Coleppa for his participation at the beginning of this project. We would also like to thank Peter Loch and Matthew Leone for helpful discussions. This work was supported in part by the Department of Energy under Grant~DE-FG02-13ER41976. 
