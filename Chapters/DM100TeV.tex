\chapter{A razor search for dark matter at 100 TeV}

\section{Background generation challenges}
One of the challenges we faced while performing this analysis was dealing with the sheer number of background events to generate and analyze. At a 100 TeV collider, the background cross-sections grow very large compared to their 14 TeV counterparts. For example, the cross section for the inclusive production of top quark pairs increases from $\sim975$ pb (N$^{3}$LO) to $\sim34,000$ pb (NLO) \citep{Mangano:2016jyj}, that is, increasing the collision energy by about seven times increases the number of events by a factor of more than 30! Multiplying this by 3 ab$^{-1}$ (the expected integrated luminosity from the first 10 years of running the FCC), gives us 96 billion top pair production events. Simulating this many events would require enormous amounts of computing time as well as huge amounts of storage. To alleviate the first problem, we generated events on the University of Arizona cluster, leveraging the power of many computing nodes to perform event generation simultaneously. But this alone would not be enough. 
\section{Goldstone equivalence theorem}

In high-energy environments, the branching ratios involving Higgsinos are simplified considerably using the \emph{Goldstone equivalence theorem}, defined below.

\begin{theorem}{Goldstone equivalence theorem}
At high energies, the amplitude for the emission or absorption of a longitudinally polarized gauge boson is equal to the the amplitude for the emission or absorption of the Goldstone boson that was 'eaten' by the gauge boson.
\end{theorem}
