\chapter{Exotic Higgs decays at 14 and 100 TeV}\label{ch:ExoticHiggs}

In the previous chapter, we showed how exotic decay modes can offer an complementary avenue for discovering charged Higgs bosons at the LHC. While that analysis was performed for a single BSM Higgs and a single exotic decay mode, the project that forms the basis of this chapter aims to do something much more ambitious, namely, to perform a comprehensive set of collider analyses for all the exotic decay modes for heavy Higgses in Type II $2$HDMs, both at the 14 TeV LHC, as well as a future 100 TeV hadron collider. The structure of this chapter is as follows. In \autoref{sec:setting_the_stage}, we present the general case for considering large mass splittings among heavy higgses, thus motivating the consideration of exotic Higgs decay channels. We also briefly discuss the physics potential of a future 100 TeV collider and provide a few references. In \autoref{sec:classifying_exotic_decays}, we summarize the most general decay modes for BSM Higgses as well as benchmark planes corresponding to a set of exotic decay modes that satisfy theoretical and experimental constraints. In \autoref{sec:literature_review}, we enumerate some of the existing studies in the literature on exotic decays. In \autoref{sec:analysis_details}, we describe the exotic decay modes we have studied so far, and the strategy we used to do so. In \autoref{sec:exo_higgs_results}, we present some of the preliminary results we have obtained. Finally, we conclude in \autoref{sec:exo_higgs_conclusion} with a summary and discussion of future work.

\section{Exotic Higgs decays: The Big Picture}\label{sec:setting_the_stage}
As pointed out in \autoref{ch:LightChargedHiggs}, most experimental analyses aimed at discovering BSM Higgses focus on conventional decay modes to Standard Model Particles. The reason for this is that most studies assume the scenario known as the \emph{decoupling limit}, where the states \emph{H,A,$H^\pm$} are much heavier than the state \emph{h}. Decoupling implies the phenomenon of \emph{alignment}, that is, \emph{h} is the SM Higgs boson, or equivalently, the angles $\beta$ and $\alpha$ satisfy the condition $\beta-\alpha = \pi/2$ \cite{Contino:2016spe}. The limit of decoupling with alignment is an attractive one, since it implies that the undiscovered Higgses are all much heavier than the SM Higgs, accounting for why they have not been found yet. However, in the decoupling limit, the splitting between the squared masses of heavy Higgses is constrained to be small, on the order of $\sim\mathcal{O}(v^2)$ \cite{Gunion:2002zf}.
This constraint on the size of the mass splittings implies that decays of heavy Higgses into final states that contain other heavy Higgses will be kinematically disfavored, and thus they will dominantly decay to final states with SM particles instead.

The relation between decoupling and alignment is not bidirectional, though - alignment does not necessarily imply decoupling. In the limit of alignment without decoupling, it is possible to have sizable mass splittings between the heavy Higgses. While this scenario is necessarily less generic than the decoupling limit, it is well worth considering, especially given that large mass splittings are consistent with a strongly first order electroweak phase transition \cite{Dorsch:2014qja}. This type of phase transition is necessary for the mechanism of electroweak baryogenesis, which can explain the observed level of baryon asymmetry in the universe. It is also possible to generate such mass splittings away from the alignment limit. While this scenario is experimentally disfavored \cite{Aad:2015pla}, it has not yet been excluded, and so we study it for completeness.

The presence of large mass splittings opens up the possibility that the heavy Higgses could decay into final states containing other heavy Higgses, that is, they could decay via exotic decay channels. As mentioned when we studied the exotic decay mode $H^\pm \rightarrow AW$ in \autoref{ch:LightChargedHiggs}, in certain regions of parameter space, once the exotic decay modes are allowed, they will dominate the conventional decay modes. This is illustrated in \autoref{fig:exotic_higgs_brs} for a few scenarios. Panels (a) and (b) show the branching ratios for the decays of \emph{A}, with $m_{12}^2 = m_H^2 s_\beta c_\beta$. Panels (c) and (d) show similar plots, this time for the branching ratios of $H^\pm$, with $m_{12}^2$ equal to $m_H^2 s_\beta c_\beta$ and $0$ respectively. The dashed lines indicate the branching ratios to SM particles in the absence of the exotic decay modes. These channels have begun to garner additional interest in recent years, with the publication of a number of original analyses on the topic \cite{Dorsch:2014qja,Dorsch:2016tab,Coleppa:2013xfa,Coleppa:2014hxa,Brownson:2013lka,Coleppa:2014cca,Kling:2015uba,Li:2015lra,Maitra:2014qea,Basso:2012st,Dermisek:2013cxa,Mohn:2005lda,Assamagan:2000ud}. 

\begin{marginfigure}[1cm]
  \centering
  \subbottom[]{\includegraphics[width=\textwidth]{images/BR-A-mH200-mA500-mC200.pdf}}
  \subbottom[]{\includegraphics[width=\textwidth]{images/BR-A-mH200-mA500-mC500.pdf}}
  \subbottom[]{\includegraphics[width=\textwidth]{images/BR-C-mH200-mA500-mC500.pdf}}
  \subbottom[]{\includegraphics[width=\textwidth]{images/BR-H-mH500-mA200-mC200.pdf}}
  \caption{branching ratios for exotic decays in Type II $2$HDMs, as a function of $t_\beta$, with $c_{\beta-\alpha} = 0$. Source: \cite{Kling:2016opi}.}
\label{fig:exotic_higgs_brs}
\end{marginfigure}

A 100 TeV collider would open up an immense number of physics opportunities not afforded to a 14 TeV collider \citep{Arkani-Hamed:2015vfh}. 
% Status of 100 TeV planning
Such a collider would be a natural next step after the LHC, and is being actively discussed in the particle physics community. The two major proposals being considered are the FCC-hh machine by CERN \citep{FCC-hh}, and the SppC by IHEP in China \citep{CEPC}. Executing a project of this size requires an immense amount of planning, over many years. Thus, now is an opportune time to begin to fully explore the physics potential of a 100 TeV collider. There have been a number of studies conducted already, and more are coming out in the literature on a regular basis. 
We will not attempt to enumerate them all, and instead point the interested reader to the references sections of the review papers \citep{Arkani-Hamed:2015vfh,Contino:2016spe,Golling:2016gvc,Mangano:2016jyj}.


\section{Classifying exotic decays}\label{sec:classifying_exotic_decays}

Taking all the possible combinations of parent and daughter Higgses yields the complete list of exotic decay modes in $2$HDMs, shown along with the dominant SM final states in \autoref{tab:exotic_decay_summary}. 
\begin{table}
\begin{adjustwidth*}{0in}{-2.1in}
\centering
 \caption{Summary of all the possible exotic decay modes and dominant final states for non-SM Higgs bosons}
  \begin{tabular}{llll}
  \toprule
 Parent Higgs & Decay type& Channels in $2$HDM & Main Final States  \\
 \midrule
               & \emph{HH} & $H\rightarrow AA, hh$                  & (\emph{bb/$\tau\tau$/WW/ZZ/$\gamma\gamma$})(\emph{bb/$\tau\tau$/WW/ZZ/$\gamma\gamma$}) \\
 Neutral Higgs & \emph{HZ} & $H\rightarrow AZ, A\rightarrow HZ, hZ$ & (\emph{ll/qq/$\nu\nu$})(\emph{bb/$\tau\tau$/WW/ZZ/$\gamma\gamma$}) \\
 \emph{H,A}    & $H^+H^-$  & $H\rightarrow H^+H^-$                  & (\emph{tb/$\tau\nu$/cs})(\emph{tb/$\tau\nu$/cs}) \\
               & $H^\pm W^\mp$  & $H/A\rightarrow H^\pm W^\mp$      & (\emph{l\nu/qq'})(\emph{tb/$\tau\nu$/cs}) \\
               \midrule
Charged Higgs  & $HW^\pm$  & $H^\pm\rightarrow hW^\pm/HW^\pm/AW^\pm$ & (\emph{l\nu/qq'})(\emph{bb/$\tau\tau$/WW/ZZ/$\gamma\gamma$}) \\
 \bottomrule
 \end{tabular}
   \legend{Here, we do not take experimental and theoretical constraints into account. In the second column, \emph{H} refers to any of the neutral Higgs bosons \emph{h,H}, and \emph{A}. Source: \cite{Contino:2016spe}.}
 \label{tab:exotic_decay_summary}
\end{adjustwidth*}
\end{table}

Given the large number of degrees of freedom in generic $2$HDMs, it is important for us to whittle down the number of possible scenarios to a more managable number by imposing some constraints. The authors of \cite{Kling:2016opi} have systematically categorized the scenarios (with differing mass orderings and splittings, values of the parameter $m_{12}$, and degrees of deviation from alignment) that are still viable after imposing all the relevant theoretical and experimental constraints. Based on this categorization, they have presented a set of benchmark planes to guide searches for new scalars at the LHC via exotic decay modes. These planes are summarized in \autoref{tab:benchmark_planes}. 

\begin{table}
  \centering
  \begin{sidecaption}{Summary of benchmark planes for exploring exotic Higgs decays in $2$HDMs that survive theoretical and experimental constraints. Source: \cite{Kling:2016opi}.}[tab:benchmark_planes]
  \begin{tabular}{lclr}
    \toprule
    Name    & Plane                   & Decays                                    & $m_{12}^2$\\\midrule
    IA      & $m_A > m_H = m_{H^\pm}$ & $A\rightarrow H^\pm W^\mp$                & $m_H^2s_\beta c_\beta$\\
            &                         & $A\rightarrow HZ$                         & 0 \\\midrule
    IB      & $m_A < m_H = m_{H^\pm}$ & $H\rightarrow AZ/AA$                      & 0\\
            &                         & $H^\pm\rightarrow AW^\pm$                 & \\\midrule
    IIA     & $m_H > m_A = m_{H^\pm}$ & $H\rightarrow AZ/AA$                      & 0\\
            &                         & $H\rightarrow H^+H^-/H^\pm W^\pm$         & \\\midrule
    IIB     & $m_H < m_A = m_{H^\pm}$ & $A\rightarrow HZ$                         & $m_H^2s_\beta c_\beta$\\
            &                         & $H^\pm\rightarrow HW^\pm$                 & 0 \\\midrule
    III     & $m_A = m_H = m_{H^\pm}$ & $A\rightarrow hZ,H^\pm\rightarrow hW^\pm$ & $m_H^2s_\beta c_\beta$\\
            & vs $c_{\beta-\alpha}$   & $H\rightarrow hh$                         & 0 \\
    \bottomrule
  \end{tabular}
\end{sidecaption}
\end{table}

Even after this constraining, there remain a plethora of possible final states, as evidenced by the last column of \autoref{tab:exotic_decay_summary}. In this work, we aim to answer the questions: which among these final states are the most promising to find new physics, and what are the best strategies to do so at a collider? In the next section, we summarize some of the existing studies on exotic Higgs decays.

% Methodology
\section{Literature review}\label{sec:literature_review}
There are a number of theoretical studies in the literature that examine exotic Higgs decays. We summarize some of the more recent ones. In \cite{Coleppa:2014hxa}, the exotic decay of a heavy neutral Higgs, $A/H\rightarrow HZ/AZ$ is studied. The same authors subsequently study the exotic decay of a heavy charged Higgs: $H^\pm\rightarrow (A/H)W^\pm$. The prospects of finding a charged Higgs boson lighter than the top quark are examined in \cite{Kling:2015uba} (which forms the basis of \autoref{ch:LightChargedHiggs}). The study in \cite{Li:2015lra} examines the exotic decay of a neutral heavy Higgs: $A/H\rightarrow W^\pm H^\mp$. Constraints are derived in the $m_{H^\pm}-t_\beta$ plane for the decay chain $H\rightarrow H^\pm W^\pm\rightarrow W^+W^- A$ in \cite{Dermisek:2013cxa}, while the authors of \cite{Basso:2012st} examine heavy charged Higgses produced in association with a \emph{W} boson and decaying into a lighter neutral Higgs. In {\cite{Maitra:2014qea} the signals for a fermiophobic charged Higgs arising in a $2$HDM with right handed neutrinos are examined. Finally, the whitepaper \cite{Brownson:2013lka} investigates the prospects of discovering a heavy neutral Higgs through the decay $A/H\rightarrow ZZ/Zh$, with the \emph{Z} decaying leptonically, and the \emph{h} decaying to either the \emph{bb} or $\tau\tau$ final states. All of these studies are performed for the 14 TeV LHC.

\section{Crafting features for our classifier}\label{sec:analysis_details}
Preliminary analyses have been carried out for two of the benchmark planes described in \autoref{tab:benchmark_planes}: benchmark plane IB, with $m_A < m_H = m_{H^\pm}$, and benchmark plane IIB, $m_H < m_A = m_{H^\pm}$, both assuming $m_{12}^2 = m_H^2s_\beta c_\beta$. For benchmark plane IB, we examine the production and decay chains for the neutral Higgs \emph{H}:
\begin{align*}
  bb/gg\rightarrow H\rightarrow AZ\rightarrow \tau\tau ll.
\end{align*}
For benchmark plane IIB, we consider the following production and decay chains for the neutral Higgs \emph{A}:
\begin{align*}
  bb/gg\rightarrow A\rightarrow HZ\rightarrow \tau\tau ll.
\end{align*}
In addition, we examine the following production and decay chain for a heavy charged Higgs $H^\pm$ in benchmark plane IIB:
\[pp\rightarrow tbH^\pm\rightarrow (Wb)b(HW)\rightarrow (Wb)b((\tau\tau)W),\]
where one of the \emph{W} bosons decays leptonically, and the other hadronically. This production and decay chain has been previously considered in \cite{Coleppa:2014hxa}. Events were generated using MadGraph 5 \cite{Alwall:2014hca}, interfaced with Pythia 6 \cite{Sjostrand:2006za} for showering and hadronization and Delphes \cite{deFavereau:2013fsa} for a fast, parameterized detector simulation.
Recognizing the power of statistical learning techniques to efficiently find non-linear decision boundaries in feature space, we use boosted decision tree classifiers to discriminate between signal and background events. For each channel, we provide the classifier with a mixture of low-level and high-level features. Low-level features include readily available information such as the momenta of the final state particles, whereas high-level features are functions of the low-level features designed to exploit kinematic relations unique to each decay topology. In practice, using high and low level features in conjunction with each other yields better results than using either of them alone.

\subsection{Exotic decays of neutral Higgses}\label{subsec:A_HZ_analysis}
For the exotic decay modes $(A/H)\rightarrow(H/A)Z$ with the final state $\tau\tau ll$, the following strategy was taken to prepare feature arrays for the boosted decision tree classifier. We consider two possibilities for the decay of the \emph{Z} boson: $Z\rightarrow bb$ (case A), and $Z\rightarrow l^+l^-$ (case B). The procedure is described below.
\begin{enumerate}
  \item \emph{Trigger.} We selected events that passed one of the following trigger criteria:
    \begin{enumerate}
      \item At least one lepton with $p_T$ > 30 GeV.
      \item Two leptons, with the hardest lepton having $p_T > $ 20 GeV and the second hardest lepton having $p_T > $ 10 GeV.
      \item A minimum missing transverse energy of 100 GeV.
    \end{enumerate}
  \item \emph{Identification}. We required events have at least two \emph{b}-tagged jets (case A), or at least one pair of oppositely charged $\tau$-tagged jets (case B). For both cases, we require exactly two leptons (electrons or muons) with the same flavor and opposite sign.
  \item\emph{Calculate and collect features}. For events that pass the trigger and identification criteria above, we construct the following set of features:
    \begin{enumerate}
      \item {\itshape \emph{Z} candidate}. We combine the momenta of the two same flavor, opposite sign leptons to form the \emph{Z} candidate, and calculate its invariant mass,  $m_{ll}$.
      \item {\itshape Lighter neutral Higgs candidate}. We combine the momenta of the two leading \emph{b}-tagged jets (case A) or the two $\tau$-tagged jets (case B) to form the lighter neutral Higgs candidate, and calculate its invariant mass, denoted either $m_{bb}$ or $m_{ll}$.
      \item {\itshape Heavier (parent) neutral Higgs candidate}. We combine the momenta of the \emph{Z} candidate and the \emph{A/H} candidate to form the charged Higgs candidate $H^\pm$, and calculate its invariant mass, denoted $m_{llbb}$ (case A) or $m_{ll\tau\tau}$ (case B).
      \item {\itshape Other features}: We also collect the following set of features: the transverse momenta of the leptons, and \emph{b}-tagged jets (case A), or the $\tau$-tagged jets (case B). Finally, we collect the missing transverse energy and hadronic transverse energy of the event.
    \end{enumerate}
\end{enumerate}
\subsection{Exotic decays of the charged Higgs}\label{subsec:Hpm_analysis}
The search channel involving the charged Higgs, has the production and decay chain: 
\[pp\rightarrow tbH^+\rightarrow (Wb)b(HW)\rightarrow l\nu bb\tau_h\tau_ljj.\]
Here, one of the \emph{W}'s and one of the $\tau$s decays hadronically, and the other decays leptonically.
We employed the following strategy to construct input features for our classifier. 
\begin{enumerate}
  \item \emph{Identification}. We required events have exactly one or two \emph{b}-tagged jets, at least two untagged jets, exactly one $\tau$-tagged jet, and exactly two leptons (electrons or muons) with the same charge which is opposite to that of the $\tau$-tagged jet.
  \item\emph{Calculate and collect features}. As a test of the software setup, we analysed this channel with only the following low-level features: the transverse momenta of the leptons, the leading \emph{b}-tagged jet, and the untagged jets, and the missing and hadronic transverse energies of the event. In the future, we aim to reconstruct the lighter Higgs \emph{H} and the charged Higgs $H^\pm$, building upon the strategy in \cite{Coleppa:2014hxa}, and using the calculated invariant masses as input features for the classifier.
\end{enumerate}

\section{Preliminary results}\label{sec:exo_higgs_results}
\begin{marginfigure}[1.1in]
\centering
{\color{gray}\hrule}\vspace{\onelineskip}
  \caption{Exclusion and discovery limits for the channel $A\rightarrow HZ\rightarrow \tau\tau ll$, for the 14 TeV LHC with 100 fb$^{-1}$ (solid red line) and 300 fb$^{-1}$ (dashed red line) and a 100 TeV collider with 1000 fb$^{-1}$ (solid blue line) and 3000 fb$^{-1}$ (dashed blue line).}
  \subbottom[$bb\rightarrow A$]{\includegraphics[width=\textwidth]{images/bb2A2tatall_dm200.pdf}}
  \subbottom[$gg\rightarrow A$]{\includegraphics[width=\textwidth]{images/gg2A2tatall_dm200.pdf}}
  \legend{Here we assume $m_{12}^2 = m_H^2s_\beta c_\beta$. The top panel represents the $bb\rightarrow A$ production channel, while the lower one represents the $gg\rightarrow A$ production channel.}
  \vspace{\onelineskip}{\color{gray}\hrule}
\label{fig:Honglei_A_tatall}
\end{marginfigure}
In this section, we will present some of the preliminary results of this project. In figure \ref{fig:Honglei_A_tatall}, we see the reach obtainable in the $A\rightarrow HZ\rightarrow \tau\tau ll$ channels in the $m_A = m_{H^\pm} - t_\beta$ plane, with $m_A - m_H = $ 200 GeV. The results are shown separately for the \emph{A} produced via $bb\rightarrow A$ (top panel), and $gg\rightarrow A$ (bottom panel). The $bb$ production channel has a significantly higher reach, being able to exclude points with $m_A = m_{H^\pm}$ up to 3.5 TeV for $t_\beta$ between 15 and 20, for a 100 TeV collider (combined green, brown, and cyan regions). The corresponding reach for the $gg$ production channel is only up to about 2 TeV. In figure \ref{fig:Honglei_H_tatall}, we see the reach obtainable in the $H\rightarrow AZ\rightarrow \tau\tau ll$ channels, with $m_H - m_A =$ 200 GeV. Similar to the results for $A\rightarrow HZ$, the results for the $bb$ production channel (left panel) are significantly better than those for the $gg$ production channel (right panel). The $bb$ production channel will be able to exclude points in the $m_H = m_{H^\pm} - t_\beta$ plane up to 3.5 TeV for a wide range of values o$t_\beta$ between 10 and 50. The reach for the $gg$ production channel is seen to be low for intermediate values of $t_\beta$ of about 20. This plane simply represents the region along the diagonal of benchmark plane IIB - this region was chosen since it survives all the theoretical and experimental constraints. Varying $\tan\beta$ does not vary the kinematics of the final state particles, just the total $\sigma\times$BR. 
\begin{figure}
{\color{gray}\hrule}\vspace{\onelineskip}
  \caption{Exclusion and discovery limits for the channel $H\rightarrow AZ\rightarrow \tau\tau ll$ for the 14 TeV LHC with 100 fb$^{-1}$ (solid red line) and 300 fb$^{-1}$ (dashed red line) and a 100 TeV collider with 1000 fb$^{-1}$ (solid blue line) and 3000 fb$^{-1}$ (dashed blue line). }
  \subbottom[$bb\rightarrow H$]{\includegraphics[width=0.49\textwidth]{images/bb2H2tatall_dm200.pdf}}
  \subbottom[$gg\rightarrow H$]{\includegraphics[width=0.49\textwidth]{images/gg2H2tatall_dm200.pdf}}
  \legend{Here we assume $m_{12}^2 = m_H^2s_\beta c_\beta$. The left panel represents the $bb\rightarrow H$ production channel, while the right represents the $gg\rightarrow H$ production channel.}
  \vspace{\onelineskip}{\color{gray}\hrule}
\label{fig:Honglei_H_tatall}
\end{figure}

\begin{figure}
{\color{gray}\hrule}\vspace{\onelineskip}
  \begin{sidecaption}{Exclusion and discovery reaches in the $m_{H^\pm}-m_{H}$ plane for the exotic decay $H^\pm\rightarrow HW^\pm$, at a 100 TeV collider with an integrated luminosity of 3 ab$^{-1}$. Here we assume $m_{12}^2 = m_H^2s_\beta c_\beta$.}[fig:charged_higgs_100_TeV]
%\includegraphics[width=\textwidth]{images/hC_HW_contours.pdf}
    %% Creator: Matplotlib, PGF backend
%%
%% To include the figure in your LaTeX document, write
%%   \input{<filename>.pgf}
%%
%% Make sure the required packages are loaded in your preamble
%%   \usepackage{pgf}
%%
%% Figures using additional raster images can only be included by \input if
%% they are in the same directory as the main LaTeX file. For loading figures
%% from other directories you can use the `import` package
%%   \usepackage{import}
%% and then include the figures with
%%   \import{<path to file>}{<filename>.pgf}
%%
%% Matplotlib used the following preamble
%%   \usepackage{fontspec}
%%   \setmainfont{Century Schoolbook L}
%%   \setmonofont{DejaVu Sans Mono}
%%
\begingroup%
\makeatletter%
\begin{pgfpicture}%
\pgfpathrectangle{\pgfpointorigin}{\pgfqpoint{3.888197in}{3.888197in}}%
\pgfusepath{use as bounding box, clip}%
\begin{pgfscope}%
\pgfsetbuttcap%
\pgfsetmiterjoin%
\definecolor{currentfill}{rgb}{1.000000,1.000000,1.000000}%
\pgfsetfillcolor{currentfill}%
\pgfsetlinewidth{0.000000pt}%
\definecolor{currentstroke}{rgb}{1.000000,1.000000,1.000000}%
\pgfsetstrokecolor{currentstroke}%
\pgfsetdash{}{0pt}%
\pgfpathmoveto{\pgfqpoint{0.000000in}{0.000000in}}%
\pgfpathlineto{\pgfqpoint{3.888197in}{0.000000in}}%
\pgfpathlineto{\pgfqpoint{3.888197in}{3.888197in}}%
\pgfpathlineto{\pgfqpoint{0.000000in}{3.888197in}}%
\pgfpathclose%
\pgfusepath{fill}%
\end{pgfscope}%
\begin{pgfscope}%
\pgfsetbuttcap%
\pgfsetmiterjoin%
\definecolor{currentfill}{rgb}{0.898039,0.898039,0.898039}%
\pgfsetfillcolor{currentfill}%
\pgfsetlinewidth{0.000000pt}%
\definecolor{currentstroke}{rgb}{0.000000,0.000000,0.000000}%
\pgfsetstrokecolor{currentstroke}%
\pgfsetstrokeopacity{0.000000}%
\pgfsetdash{}{0pt}%
\pgfpathmoveto{\pgfqpoint{0.769861in}{0.591389in}}%
\pgfpathlineto{\pgfqpoint{3.583602in}{0.591389in}}%
\pgfpathlineto{\pgfqpoint{3.583602in}{3.672833in}}%
\pgfpathlineto{\pgfqpoint{0.769861in}{3.672833in}}%
\pgfpathclose%
\pgfusepath{fill}%
\end{pgfscope}%
\begin{pgfscope}%
\pgfpathrectangle{\pgfqpoint{0.769861in}{0.591389in}}{\pgfqpoint{2.813741in}{3.081445in}} %
\pgfusepath{clip}%
\pgfsetrectcap%
\pgfsetroundjoin%
\pgfsetlinewidth{0.501875pt}%
\definecolor{currentstroke}{rgb}{1.000000,1.000000,1.000000}%
\pgfsetstrokecolor{currentstroke}%
\pgfsetdash{}{0pt}%
\pgfpathmoveto{\pgfqpoint{0.769861in}{0.591389in}}%
\pgfpathlineto{\pgfqpoint{0.769861in}{3.672833in}}%
\pgfusepath{stroke}%
\end{pgfscope}%
\begin{pgfscope}%
\pgfsetbuttcap%
\pgfsetroundjoin%
\definecolor{currentfill}{rgb}{0.333333,0.333333,0.333333}%
\pgfsetfillcolor{currentfill}%
\pgfsetlinewidth{0.501875pt}%
\definecolor{currentstroke}{rgb}{0.333333,0.333333,0.333333}%
\pgfsetstrokecolor{currentstroke}%
\pgfsetdash{}{0pt}%
\pgfsys@defobject{currentmarker}{\pgfqpoint{0.000000in}{-0.055556in}}{\pgfqpoint{0.000000in}{0.000000in}}{%
\pgfpathmoveto{\pgfqpoint{0.000000in}{0.000000in}}%
\pgfpathlineto{\pgfqpoint{0.000000in}{-0.055556in}}%
\pgfusepath{stroke,fill}%
}%
\begin{pgfscope}%
\pgfsys@transformshift{0.769861in}{0.591389in}%
\pgfsys@useobject{currentmarker}{}%
\end{pgfscope}%
\end{pgfscope}%
\begin{pgfscope}%
\pgfsetbuttcap%
\pgfsetroundjoin%
\definecolor{currentfill}{rgb}{0.333333,0.333333,0.333333}%
\pgfsetfillcolor{currentfill}%
\pgfsetlinewidth{0.501875pt}%
\definecolor{currentstroke}{rgb}{0.333333,0.333333,0.333333}%
\pgfsetstrokecolor{currentstroke}%
\pgfsetdash{}{0pt}%
\pgfsys@defobject{currentmarker}{\pgfqpoint{0.000000in}{0.000000in}}{\pgfqpoint{0.000000in}{0.055556in}}{%
\pgfpathmoveto{\pgfqpoint{0.000000in}{0.000000in}}%
\pgfpathlineto{\pgfqpoint{0.000000in}{0.055556in}}%
\pgfusepath{stroke,fill}%
}%
\begin{pgfscope}%
\pgfsys@transformshift{0.769861in}{3.672833in}%
\pgfsys@useobject{currentmarker}{}%
\end{pgfscope}%
\end{pgfscope}%
\begin{pgfscope}%
\definecolor{textcolor}{rgb}{0.333333,0.333333,0.333333}%
\pgfsetstrokecolor{textcolor}%
\pgfsetfillcolor{textcolor}%
\pgftext[x=0.769861in,y=0.480278in,,top]{\color{textcolor}\rmfamily\fontsize{10.000000}{12.000000}\selectfont 0}%
\end{pgfscope}%
\begin{pgfscope}%
\pgfpathrectangle{\pgfqpoint{0.769861in}{0.591389in}}{\pgfqpoint{2.813741in}{3.081445in}} %
\pgfusepath{clip}%
\pgfsetrectcap%
\pgfsetroundjoin%
\pgfsetlinewidth{0.501875pt}%
\definecolor{currentstroke}{rgb}{1.000000,1.000000,1.000000}%
\pgfsetstrokecolor{currentstroke}%
\pgfsetdash{}{0pt}%
\pgfpathmoveto{\pgfqpoint{1.473331in}{0.591389in}}%
\pgfpathlineto{\pgfqpoint{1.473331in}{3.672833in}}%
\pgfusepath{stroke}%
\end{pgfscope}%
\begin{pgfscope}%
\pgfsetbuttcap%
\pgfsetroundjoin%
\definecolor{currentfill}{rgb}{0.333333,0.333333,0.333333}%
\pgfsetfillcolor{currentfill}%
\pgfsetlinewidth{0.501875pt}%
\definecolor{currentstroke}{rgb}{0.333333,0.333333,0.333333}%
\pgfsetstrokecolor{currentstroke}%
\pgfsetdash{}{0pt}%
\pgfsys@defobject{currentmarker}{\pgfqpoint{0.000000in}{-0.055556in}}{\pgfqpoint{0.000000in}{0.000000in}}{%
\pgfpathmoveto{\pgfqpoint{0.000000in}{0.000000in}}%
\pgfpathlineto{\pgfqpoint{0.000000in}{-0.055556in}}%
\pgfusepath{stroke,fill}%
}%
\begin{pgfscope}%
\pgfsys@transformshift{1.473331in}{0.591389in}%
\pgfsys@useobject{currentmarker}{}%
\end{pgfscope}%
\end{pgfscope}%
\begin{pgfscope}%
\pgfsetbuttcap%
\pgfsetroundjoin%
\definecolor{currentfill}{rgb}{0.333333,0.333333,0.333333}%
\pgfsetfillcolor{currentfill}%
\pgfsetlinewidth{0.501875pt}%
\definecolor{currentstroke}{rgb}{0.333333,0.333333,0.333333}%
\pgfsetstrokecolor{currentstroke}%
\pgfsetdash{}{0pt}%
\pgfsys@defobject{currentmarker}{\pgfqpoint{0.000000in}{0.000000in}}{\pgfqpoint{0.000000in}{0.055556in}}{%
\pgfpathmoveto{\pgfqpoint{0.000000in}{0.000000in}}%
\pgfpathlineto{\pgfqpoint{0.000000in}{0.055556in}}%
\pgfusepath{stroke,fill}%
}%
\begin{pgfscope}%
\pgfsys@transformshift{1.473331in}{3.672833in}%
\pgfsys@useobject{currentmarker}{}%
\end{pgfscope}%
\end{pgfscope}%
\begin{pgfscope}%
\definecolor{textcolor}{rgb}{0.333333,0.333333,0.333333}%
\pgfsetstrokecolor{textcolor}%
\pgfsetfillcolor{textcolor}%
\pgftext[x=1.473331in,y=0.480278in,,top]{\color{textcolor}\rmfamily\fontsize{10.000000}{12.000000}\selectfont 500}%
\end{pgfscope}%
\begin{pgfscope}%
\pgfpathrectangle{\pgfqpoint{0.769861in}{0.591389in}}{\pgfqpoint{2.813741in}{3.081445in}} %
\pgfusepath{clip}%
\pgfsetrectcap%
\pgfsetroundjoin%
\pgfsetlinewidth{0.501875pt}%
\definecolor{currentstroke}{rgb}{1.000000,1.000000,1.000000}%
\pgfsetstrokecolor{currentstroke}%
\pgfsetdash{}{0pt}%
\pgfpathmoveto{\pgfqpoint{2.176802in}{0.591389in}}%
\pgfpathlineto{\pgfqpoint{2.176802in}{3.672833in}}%
\pgfusepath{stroke}%
\end{pgfscope}%
\begin{pgfscope}%
\pgfsetbuttcap%
\pgfsetroundjoin%
\definecolor{currentfill}{rgb}{0.333333,0.333333,0.333333}%
\pgfsetfillcolor{currentfill}%
\pgfsetlinewidth{0.501875pt}%
\definecolor{currentstroke}{rgb}{0.333333,0.333333,0.333333}%
\pgfsetstrokecolor{currentstroke}%
\pgfsetdash{}{0pt}%
\pgfsys@defobject{currentmarker}{\pgfqpoint{0.000000in}{-0.055556in}}{\pgfqpoint{0.000000in}{0.000000in}}{%
\pgfpathmoveto{\pgfqpoint{0.000000in}{0.000000in}}%
\pgfpathlineto{\pgfqpoint{0.000000in}{-0.055556in}}%
\pgfusepath{stroke,fill}%
}%
\begin{pgfscope}%
\pgfsys@transformshift{2.176802in}{0.591389in}%
\pgfsys@useobject{currentmarker}{}%
\end{pgfscope}%
\end{pgfscope}%
\begin{pgfscope}%
\pgfsetbuttcap%
\pgfsetroundjoin%
\definecolor{currentfill}{rgb}{0.333333,0.333333,0.333333}%
\pgfsetfillcolor{currentfill}%
\pgfsetlinewidth{0.501875pt}%
\definecolor{currentstroke}{rgb}{0.333333,0.333333,0.333333}%
\pgfsetstrokecolor{currentstroke}%
\pgfsetdash{}{0pt}%
\pgfsys@defobject{currentmarker}{\pgfqpoint{0.000000in}{0.000000in}}{\pgfqpoint{0.000000in}{0.055556in}}{%
\pgfpathmoveto{\pgfqpoint{0.000000in}{0.000000in}}%
\pgfpathlineto{\pgfqpoint{0.000000in}{0.055556in}}%
\pgfusepath{stroke,fill}%
}%
\begin{pgfscope}%
\pgfsys@transformshift{2.176802in}{3.672833in}%
\pgfsys@useobject{currentmarker}{}%
\end{pgfscope}%
\end{pgfscope}%
\begin{pgfscope}%
\definecolor{textcolor}{rgb}{0.333333,0.333333,0.333333}%
\pgfsetstrokecolor{textcolor}%
\pgfsetfillcolor{textcolor}%
\pgftext[x=2.176802in,y=0.480278in,,top]{\color{textcolor}\rmfamily\fontsize{10.000000}{12.000000}\selectfont 1000}%
\end{pgfscope}%
\begin{pgfscope}%
\pgfpathrectangle{\pgfqpoint{0.769861in}{0.591389in}}{\pgfqpoint{2.813741in}{3.081445in}} %
\pgfusepath{clip}%
\pgfsetrectcap%
\pgfsetroundjoin%
\pgfsetlinewidth{0.501875pt}%
\definecolor{currentstroke}{rgb}{1.000000,1.000000,1.000000}%
\pgfsetstrokecolor{currentstroke}%
\pgfsetdash{}{0pt}%
\pgfpathmoveto{\pgfqpoint{2.880272in}{0.591389in}}%
\pgfpathlineto{\pgfqpoint{2.880272in}{3.672833in}}%
\pgfusepath{stroke}%
\end{pgfscope}%
\begin{pgfscope}%
\pgfsetbuttcap%
\pgfsetroundjoin%
\definecolor{currentfill}{rgb}{0.333333,0.333333,0.333333}%
\pgfsetfillcolor{currentfill}%
\pgfsetlinewidth{0.501875pt}%
\definecolor{currentstroke}{rgb}{0.333333,0.333333,0.333333}%
\pgfsetstrokecolor{currentstroke}%
\pgfsetdash{}{0pt}%
\pgfsys@defobject{currentmarker}{\pgfqpoint{0.000000in}{-0.055556in}}{\pgfqpoint{0.000000in}{0.000000in}}{%
\pgfpathmoveto{\pgfqpoint{0.000000in}{0.000000in}}%
\pgfpathlineto{\pgfqpoint{0.000000in}{-0.055556in}}%
\pgfusepath{stroke,fill}%
}%
\begin{pgfscope}%
\pgfsys@transformshift{2.880272in}{0.591389in}%
\pgfsys@useobject{currentmarker}{}%
\end{pgfscope}%
\end{pgfscope}%
\begin{pgfscope}%
\pgfsetbuttcap%
\pgfsetroundjoin%
\definecolor{currentfill}{rgb}{0.333333,0.333333,0.333333}%
\pgfsetfillcolor{currentfill}%
\pgfsetlinewidth{0.501875pt}%
\definecolor{currentstroke}{rgb}{0.333333,0.333333,0.333333}%
\pgfsetstrokecolor{currentstroke}%
\pgfsetdash{}{0pt}%
\pgfsys@defobject{currentmarker}{\pgfqpoint{0.000000in}{0.000000in}}{\pgfqpoint{0.000000in}{0.055556in}}{%
\pgfpathmoveto{\pgfqpoint{0.000000in}{0.000000in}}%
\pgfpathlineto{\pgfqpoint{0.000000in}{0.055556in}}%
\pgfusepath{stroke,fill}%
}%
\begin{pgfscope}%
\pgfsys@transformshift{2.880272in}{3.672833in}%
\pgfsys@useobject{currentmarker}{}%
\end{pgfscope}%
\end{pgfscope}%
\begin{pgfscope}%
\definecolor{textcolor}{rgb}{0.333333,0.333333,0.333333}%
\pgfsetstrokecolor{textcolor}%
\pgfsetfillcolor{textcolor}%
\pgftext[x=2.880272in,y=0.480278in,,top]{\color{textcolor}\rmfamily\fontsize{10.000000}{12.000000}\selectfont 1500}%
\end{pgfscope}%
\begin{pgfscope}%
\definecolor{textcolor}{rgb}{0.333333,0.333333,0.333333}%
\pgfsetstrokecolor{textcolor}%
\pgfsetfillcolor{textcolor}%
\pgftext[x=2.176731in,y=0.280417in,,top]{\color{textcolor}\rmfamily\fontsize{10.000000}{12.000000}\selectfont \(\displaystyle m_{H^\pm} = m_A\) (GeV)}%
\end{pgfscope}%
\begin{pgfscope}%
\pgfpathrectangle{\pgfqpoint{0.769861in}{0.591389in}}{\pgfqpoint{2.813741in}{3.081445in}} %
\pgfusepath{clip}%
\pgfsetrectcap%
\pgfsetroundjoin%
\pgfsetlinewidth{0.501875pt}%
\definecolor{currentstroke}{rgb}{1.000000,1.000000,1.000000}%
\pgfsetstrokecolor{currentstroke}%
\pgfsetdash{}{0pt}%
\pgfpathmoveto{\pgfqpoint{0.769861in}{0.591389in}}%
\pgfpathlineto{\pgfqpoint{3.583602in}{0.591389in}}%
\pgfusepath{stroke}%
\end{pgfscope}%
\begin{pgfscope}%
\pgfsetbuttcap%
\pgfsetroundjoin%
\definecolor{currentfill}{rgb}{0.333333,0.333333,0.333333}%
\pgfsetfillcolor{currentfill}%
\pgfsetlinewidth{0.501875pt}%
\definecolor{currentstroke}{rgb}{0.333333,0.333333,0.333333}%
\pgfsetstrokecolor{currentstroke}%
\pgfsetdash{}{0pt}%
\pgfsys@defobject{currentmarker}{\pgfqpoint{-0.055556in}{0.000000in}}{\pgfqpoint{0.000000in}{0.000000in}}{%
\pgfpathmoveto{\pgfqpoint{0.000000in}{0.000000in}}%
\pgfpathlineto{\pgfqpoint{-0.055556in}{0.000000in}}%
\pgfusepath{stroke,fill}%
}%
\begin{pgfscope}%
\pgfsys@transformshift{0.769861in}{0.591389in}%
\pgfsys@useobject{currentmarker}{}%
\end{pgfscope}%
\end{pgfscope}%
\begin{pgfscope}%
\pgfsetbuttcap%
\pgfsetroundjoin%
\definecolor{currentfill}{rgb}{0.333333,0.333333,0.333333}%
\pgfsetfillcolor{currentfill}%
\pgfsetlinewidth{0.501875pt}%
\definecolor{currentstroke}{rgb}{0.333333,0.333333,0.333333}%
\pgfsetstrokecolor{currentstroke}%
\pgfsetdash{}{0pt}%
\pgfsys@defobject{currentmarker}{\pgfqpoint{0.000000in}{0.000000in}}{\pgfqpoint{0.055556in}{0.000000in}}{%
\pgfpathmoveto{\pgfqpoint{0.000000in}{0.000000in}}%
\pgfpathlineto{\pgfqpoint{0.055556in}{0.000000in}}%
\pgfusepath{stroke,fill}%
}%
\begin{pgfscope}%
\pgfsys@transformshift{3.583602in}{0.591389in}%
\pgfsys@useobject{currentmarker}{}%
\end{pgfscope}%
\end{pgfscope}%
\begin{pgfscope}%
\definecolor{textcolor}{rgb}{0.333333,0.333333,0.333333}%
\pgfsetstrokecolor{textcolor}%
\pgfsetfillcolor{textcolor}%
\pgftext[x=0.658750in,y=0.591389in,right,]{\color{textcolor}\rmfamily\fontsize{10.000000}{12.000000}\selectfont 0}%
\end{pgfscope}%
\begin{pgfscope}%
\pgfpathrectangle{\pgfqpoint{0.769861in}{0.591389in}}{\pgfqpoint{2.813741in}{3.081445in}} %
\pgfusepath{clip}%
\pgfsetrectcap%
\pgfsetroundjoin%
\pgfsetlinewidth{0.501875pt}%
\definecolor{currentstroke}{rgb}{1.000000,1.000000,1.000000}%
\pgfsetstrokecolor{currentstroke}%
\pgfsetdash{}{0pt}%
\pgfpathmoveto{\pgfqpoint{0.769861in}{1.361788in}}%
\pgfpathlineto{\pgfqpoint{3.583602in}{1.361788in}}%
\pgfusepath{stroke}%
\end{pgfscope}%
\begin{pgfscope}%
\pgfsetbuttcap%
\pgfsetroundjoin%
\definecolor{currentfill}{rgb}{0.333333,0.333333,0.333333}%
\pgfsetfillcolor{currentfill}%
\pgfsetlinewidth{0.501875pt}%
\definecolor{currentstroke}{rgb}{0.333333,0.333333,0.333333}%
\pgfsetstrokecolor{currentstroke}%
\pgfsetdash{}{0pt}%
\pgfsys@defobject{currentmarker}{\pgfqpoint{-0.055556in}{0.000000in}}{\pgfqpoint{0.000000in}{0.000000in}}{%
\pgfpathmoveto{\pgfqpoint{0.000000in}{0.000000in}}%
\pgfpathlineto{\pgfqpoint{-0.055556in}{0.000000in}}%
\pgfusepath{stroke,fill}%
}%
\begin{pgfscope}%
\pgfsys@transformshift{0.769861in}{1.361788in}%
\pgfsys@useobject{currentmarker}{}%
\end{pgfscope}%
\end{pgfscope}%
\begin{pgfscope}%
\pgfsetbuttcap%
\pgfsetroundjoin%
\definecolor{currentfill}{rgb}{0.333333,0.333333,0.333333}%
\pgfsetfillcolor{currentfill}%
\pgfsetlinewidth{0.501875pt}%
\definecolor{currentstroke}{rgb}{0.333333,0.333333,0.333333}%
\pgfsetstrokecolor{currentstroke}%
\pgfsetdash{}{0pt}%
\pgfsys@defobject{currentmarker}{\pgfqpoint{0.000000in}{0.000000in}}{\pgfqpoint{0.055556in}{0.000000in}}{%
\pgfpathmoveto{\pgfqpoint{0.000000in}{0.000000in}}%
\pgfpathlineto{\pgfqpoint{0.055556in}{0.000000in}}%
\pgfusepath{stroke,fill}%
}%
\begin{pgfscope}%
\pgfsys@transformshift{3.583602in}{1.361788in}%
\pgfsys@useobject{currentmarker}{}%
\end{pgfscope}%
\end{pgfscope}%
\begin{pgfscope}%
\definecolor{textcolor}{rgb}{0.333333,0.333333,0.333333}%
\pgfsetstrokecolor{textcolor}%
\pgfsetfillcolor{textcolor}%
\pgftext[x=0.658750in,y=1.361788in,right,]{\color{textcolor}\rmfamily\fontsize{10.000000}{12.000000}\selectfont 500}%
\end{pgfscope}%
\begin{pgfscope}%
\pgfpathrectangle{\pgfqpoint{0.769861in}{0.591389in}}{\pgfqpoint{2.813741in}{3.081445in}} %
\pgfusepath{clip}%
\pgfsetrectcap%
\pgfsetroundjoin%
\pgfsetlinewidth{0.501875pt}%
\definecolor{currentstroke}{rgb}{1.000000,1.000000,1.000000}%
\pgfsetstrokecolor{currentstroke}%
\pgfsetdash{}{0pt}%
\pgfpathmoveto{\pgfqpoint{0.769861in}{2.132188in}}%
\pgfpathlineto{\pgfqpoint{3.583602in}{2.132188in}}%
\pgfusepath{stroke}%
\end{pgfscope}%
\begin{pgfscope}%
\pgfsetbuttcap%
\pgfsetroundjoin%
\definecolor{currentfill}{rgb}{0.333333,0.333333,0.333333}%
\pgfsetfillcolor{currentfill}%
\pgfsetlinewidth{0.501875pt}%
\definecolor{currentstroke}{rgb}{0.333333,0.333333,0.333333}%
\pgfsetstrokecolor{currentstroke}%
\pgfsetdash{}{0pt}%
\pgfsys@defobject{currentmarker}{\pgfqpoint{-0.055556in}{0.000000in}}{\pgfqpoint{0.000000in}{0.000000in}}{%
\pgfpathmoveto{\pgfqpoint{0.000000in}{0.000000in}}%
\pgfpathlineto{\pgfqpoint{-0.055556in}{0.000000in}}%
\pgfusepath{stroke,fill}%
}%
\begin{pgfscope}%
\pgfsys@transformshift{0.769861in}{2.132188in}%
\pgfsys@useobject{currentmarker}{}%
\end{pgfscope}%
\end{pgfscope}%
\begin{pgfscope}%
\pgfsetbuttcap%
\pgfsetroundjoin%
\definecolor{currentfill}{rgb}{0.333333,0.333333,0.333333}%
\pgfsetfillcolor{currentfill}%
\pgfsetlinewidth{0.501875pt}%
\definecolor{currentstroke}{rgb}{0.333333,0.333333,0.333333}%
\pgfsetstrokecolor{currentstroke}%
\pgfsetdash{}{0pt}%
\pgfsys@defobject{currentmarker}{\pgfqpoint{0.000000in}{0.000000in}}{\pgfqpoint{0.055556in}{0.000000in}}{%
\pgfpathmoveto{\pgfqpoint{0.000000in}{0.000000in}}%
\pgfpathlineto{\pgfqpoint{0.055556in}{0.000000in}}%
\pgfusepath{stroke,fill}%
}%
\begin{pgfscope}%
\pgfsys@transformshift{3.583602in}{2.132188in}%
\pgfsys@useobject{currentmarker}{}%
\end{pgfscope}%
\end{pgfscope}%
\begin{pgfscope}%
\definecolor{textcolor}{rgb}{0.333333,0.333333,0.333333}%
\pgfsetstrokecolor{textcolor}%
\pgfsetfillcolor{textcolor}%
\pgftext[x=0.658750in,y=2.132188in,right,]{\color{textcolor}\rmfamily\fontsize{10.000000}{12.000000}\selectfont 1000}%
\end{pgfscope}%
\begin{pgfscope}%
\pgfpathrectangle{\pgfqpoint{0.769861in}{0.591389in}}{\pgfqpoint{2.813741in}{3.081445in}} %
\pgfusepath{clip}%
\pgfsetrectcap%
\pgfsetroundjoin%
\pgfsetlinewidth{0.501875pt}%
\definecolor{currentstroke}{rgb}{1.000000,1.000000,1.000000}%
\pgfsetstrokecolor{currentstroke}%
\pgfsetdash{}{0pt}%
\pgfpathmoveto{\pgfqpoint{0.769861in}{2.902588in}}%
\pgfpathlineto{\pgfqpoint{3.583602in}{2.902588in}}%
\pgfusepath{stroke}%
\end{pgfscope}%
\begin{pgfscope}%
\pgfsetbuttcap%
\pgfsetroundjoin%
\definecolor{currentfill}{rgb}{0.333333,0.333333,0.333333}%
\pgfsetfillcolor{currentfill}%
\pgfsetlinewidth{0.501875pt}%
\definecolor{currentstroke}{rgb}{0.333333,0.333333,0.333333}%
\pgfsetstrokecolor{currentstroke}%
\pgfsetdash{}{0pt}%
\pgfsys@defobject{currentmarker}{\pgfqpoint{-0.055556in}{0.000000in}}{\pgfqpoint{0.000000in}{0.000000in}}{%
\pgfpathmoveto{\pgfqpoint{0.000000in}{0.000000in}}%
\pgfpathlineto{\pgfqpoint{-0.055556in}{0.000000in}}%
\pgfusepath{stroke,fill}%
}%
\begin{pgfscope}%
\pgfsys@transformshift{0.769861in}{2.902588in}%
\pgfsys@useobject{currentmarker}{}%
\end{pgfscope}%
\end{pgfscope}%
\begin{pgfscope}%
\pgfsetbuttcap%
\pgfsetroundjoin%
\definecolor{currentfill}{rgb}{0.333333,0.333333,0.333333}%
\pgfsetfillcolor{currentfill}%
\pgfsetlinewidth{0.501875pt}%
\definecolor{currentstroke}{rgb}{0.333333,0.333333,0.333333}%
\pgfsetstrokecolor{currentstroke}%
\pgfsetdash{}{0pt}%
\pgfsys@defobject{currentmarker}{\pgfqpoint{0.000000in}{0.000000in}}{\pgfqpoint{0.055556in}{0.000000in}}{%
\pgfpathmoveto{\pgfqpoint{0.000000in}{0.000000in}}%
\pgfpathlineto{\pgfqpoint{0.055556in}{0.000000in}}%
\pgfusepath{stroke,fill}%
}%
\begin{pgfscope}%
\pgfsys@transformshift{3.583602in}{2.902588in}%
\pgfsys@useobject{currentmarker}{}%
\end{pgfscope}%
\end{pgfscope}%
\begin{pgfscope}%
\definecolor{textcolor}{rgb}{0.333333,0.333333,0.333333}%
\pgfsetstrokecolor{textcolor}%
\pgfsetfillcolor{textcolor}%
\pgftext[x=0.658750in,y=2.902588in,right,]{\color{textcolor}\rmfamily\fontsize{10.000000}{12.000000}\selectfont 1500}%
\end{pgfscope}%
\begin{pgfscope}%
\definecolor{textcolor}{rgb}{0.333333,0.333333,0.333333}%
\pgfsetstrokecolor{textcolor}%
\pgfsetfillcolor{textcolor}%
\pgftext[x=0.280417in,y=2.132111in,,bottom,rotate=90.000000]{\color{textcolor}\rmfamily\fontsize{10.000000}{12.000000}\selectfont \(\displaystyle m_{H}\) (GeV)}%
\end{pgfscope}%
\begin{pgfscope}%
\pgfpathrectangle{\pgfqpoint{0.769861in}{0.591389in}}{\pgfqpoint{2.813741in}{3.081445in}} %
\pgfusepath{clip}%
\pgfsetbuttcap%
\pgfsetroundjoin%
\pgfsetlinewidth{1.003750pt}%
\definecolor{currentstroke}{rgb}{0.000000,0.000000,0.500000}%
\pgfsetstrokecolor{currentstroke}%
\pgfsetdash{{6.000000pt}{6.000000pt}}{0.000000pt}%
\pgfpathmoveto{\pgfqpoint{3.337655in}{1.370089in}}%
\pgfpathlineto{\pgfqpoint{3.154241in}{1.371772in}}%
\pgfpathlineto{\pgfqpoint{2.970827in}{1.377774in}}%
\pgfpathlineto{\pgfqpoint{2.787413in}{1.365733in}}%
\pgfpathlineto{\pgfqpoint{2.603999in}{1.375257in}}%
\pgfpathlineto{\pgfqpoint{2.420585in}{1.370069in}}%
\pgfpathlineto{\pgfqpoint{2.293001in}{1.363773in}}%
\pgfusepath{stroke}%
\end{pgfscope}%
\begin{pgfscope}%
\pgfpathrectangle{\pgfqpoint{0.769861in}{0.591389in}}{\pgfqpoint{2.813741in}{3.081445in}} %
\pgfusepath{clip}%
\pgfsetbuttcap%
\pgfsetroundjoin%
\pgfsetlinewidth{1.003750pt}%
\definecolor{currentstroke}{rgb}{0.000000,0.000000,0.500000}%
\pgfsetstrokecolor{currentstroke}%
\pgfsetdash{{6.000000pt}{6.000000pt}}{0.000000pt}%
\pgfpathmoveto{\pgfqpoint{1.820726in}{1.288592in}}%
\pgfpathlineto{\pgfqpoint{1.686930in}{1.230836in}}%
\pgfpathlineto{\pgfqpoint{1.678644in}{1.224088in}}%
\pgfpathlineto{\pgfqpoint{1.503516in}{1.190518in}}%
\pgfpathlineto{\pgfqpoint{1.340890in}{1.013189in}}%
\pgfpathlineto{\pgfqpoint{1.320102in}{0.968553in}}%
\pgfpathlineto{\pgfqpoint{1.181843in}{0.802289in}}%
\pgfusepath{stroke}%
\end{pgfscope}%
\begin{pgfscope}%
\pgfpathrectangle{\pgfqpoint{0.769861in}{0.591389in}}{\pgfqpoint{2.813741in}{3.081445in}} %
\pgfusepath{clip}%
\pgfsetbuttcap%
\pgfsetroundjoin%
\pgfsetlinewidth{1.003750pt}%
\definecolor{currentstroke}{rgb}{0.500000,0.000000,0.000000}%
\pgfsetstrokecolor{currentstroke}%
\pgfsetdash{{6.000000pt}{6.000000pt}}{0.000000pt}%
\pgfpathmoveto{\pgfqpoint{3.337655in}{1.259325in}}%
\pgfpathlineto{\pgfqpoint{3.154241in}{1.264362in}}%
\pgfpathlineto{\pgfqpoint{2.970827in}{1.280504in}}%
\pgfpathlineto{\pgfqpoint{2.787413in}{1.248332in}}%
\pgfpathlineto{\pgfqpoint{2.603999in}{1.274942in}}%
\pgfpathlineto{\pgfqpoint{2.420585in}{1.262033in}}%
\pgfpathlineto{\pgfqpoint{2.378952in}{1.257112in}}%
\pgfusepath{stroke}%
\end{pgfscope}%
\begin{pgfscope}%
\pgfpathrectangle{\pgfqpoint{0.769861in}{0.591389in}}{\pgfqpoint{2.813741in}{3.081445in}} %
\pgfusepath{clip}%
\pgfsetbuttcap%
\pgfsetroundjoin%
\pgfsetlinewidth{1.003750pt}%
\definecolor{currentstroke}{rgb}{0.500000,0.000000,0.000000}%
\pgfsetstrokecolor{currentstroke}%
\pgfsetdash{{6.000000pt}{6.000000pt}}{0.000000pt}%
\pgfpathmoveto{\pgfqpoint{2.095928in}{1.222275in}}%
\pgfpathlineto{\pgfqpoint{2.053758in}{1.220941in}}%
\pgfpathlineto{\pgfqpoint{1.870344in}{1.209445in}}%
\pgfpathlineto{\pgfqpoint{1.686930in}{1.185830in}}%
\pgfpathlineto{\pgfqpoint{1.503516in}{1.115889in}}%
\pgfpathlineto{\pgfqpoint{1.409331in}{1.013189in}}%
\pgfpathlineto{\pgfqpoint{1.320102in}{0.821592in}}%
\pgfpathlineto{\pgfqpoint{1.304050in}{0.802289in}}%
\pgfusepath{stroke}%
\end{pgfscope}%
\begin{pgfscope}%
\pgfpathrectangle{\pgfqpoint{0.769861in}{0.591389in}}{\pgfqpoint{2.813741in}{3.081445in}} %
\pgfusepath{clip}%
\pgfsetbuttcap%
\pgfsetroundjoin%
\pgfsetlinewidth{1.003750pt}%
\definecolor{currentstroke}{rgb}{0.501961,0.501961,0.501961}%
\pgfsetstrokecolor{currentstroke}%
\pgfsetdash{{6.000000pt}{6.000000pt}}{0.000000pt}%
\pgfpathmoveto{\pgfqpoint{0.769861in}{0.591389in}}%
\pgfpathlineto{\pgfqpoint{3.583602in}{3.672833in}}%
\pgfpathlineto{\pgfqpoint{3.583602in}{3.672833in}}%
\pgfusepath{stroke}%
\end{pgfscope}%
\begin{pgfscope}%
\pgfsetrectcap%
\pgfsetmiterjoin%
\pgfsetlinewidth{1.003750pt}%
\definecolor{currentstroke}{rgb}{1.000000,1.000000,1.000000}%
\pgfsetstrokecolor{currentstroke}%
\pgfsetdash{}{0pt}%
\pgfpathmoveto{\pgfqpoint{0.769861in}{3.672833in}}%
\pgfpathlineto{\pgfqpoint{3.583602in}{3.672833in}}%
\pgfusepath{stroke}%
\end{pgfscope}%
\begin{pgfscope}%
\pgfsetrectcap%
\pgfsetmiterjoin%
\pgfsetlinewidth{1.003750pt}%
\definecolor{currentstroke}{rgb}{1.000000,1.000000,1.000000}%
\pgfsetstrokecolor{currentstroke}%
\pgfsetdash{}{0pt}%
\pgfpathmoveto{\pgfqpoint{3.583602in}{0.591389in}}%
\pgfpathlineto{\pgfqpoint{3.583602in}{3.672833in}}%
\pgfusepath{stroke}%
\end{pgfscope}%
\begin{pgfscope}%
\pgfsetrectcap%
\pgfsetmiterjoin%
\pgfsetlinewidth{1.003750pt}%
\definecolor{currentstroke}{rgb}{1.000000,1.000000,1.000000}%
\pgfsetstrokecolor{currentstroke}%
\pgfsetdash{}{0pt}%
\pgfpathmoveto{\pgfqpoint{0.769861in}{0.591389in}}%
\pgfpathlineto{\pgfqpoint{3.583602in}{0.591389in}}%
\pgfusepath{stroke}%
\end{pgfscope}%
\begin{pgfscope}%
\pgfsetrectcap%
\pgfsetmiterjoin%
\pgfsetlinewidth{1.003750pt}%
\definecolor{currentstroke}{rgb}{1.000000,1.000000,1.000000}%
\pgfsetstrokecolor{currentstroke}%
\pgfsetdash{}{0pt}%
\pgfpathmoveto{\pgfqpoint{0.769861in}{0.591389in}}%
\pgfpathlineto{\pgfqpoint{0.769861in}{3.672833in}}%
\pgfusepath{stroke}%
\end{pgfscope}%
\begin{pgfscope}%
\pgftext[x=1.828352in,y=1.897392in,left,base,rotate=46.000000]{\rmfamily\fontsize{10.000000}{12.000000}\selectfont \(\displaystyle m_{H^\pm} = m_A = m_H\)}%
\end{pgfscope}%
\begin{pgfscope}%
\definecolor{textcolor}{rgb}{0.000000,0.000000,0.500000}%
\pgfsetstrokecolor{textcolor}%
\pgfsetfillcolor{textcolor}%
\pgftext[x=1.894343in,y=1.286692in,left,base,rotate=7.449590]{\color{textcolor}\rmfamily\fontsize{10.000000}{12.000000}\selectfont \(\displaystyle 1.96\sigma\)}%
\end{pgfscope}%
\begin{pgfscope}%
\definecolor{textcolor}{rgb}{0.500000,0.000000,0.000000}%
\pgfsetstrokecolor{textcolor}%
\pgfsetfillcolor{textcolor}%
\pgftext[x=2.166490in,y=1.192423in,left,base,rotate=8.353546]{\color{textcolor}\rmfamily\fontsize{10.000000}{12.000000}\selectfont \(\displaystyle 5\sigma\)}%
\end{pgfscope}%
\end{pgfpicture}%
\makeatother%
\endgroup%

  \end{sidecaption}
\vspace{\onelineskip}{\color{gray}\hrule}
\end{figure}

For the charged Higgs channel $H^\pm\rightarrow HW$ described in the previous section, the reach for a 100 TeV collider is shown in \autoref{fig:charged_higgs_100_TeV}, in the $m_{H^\pm} = m_A$ vs $m_H$ plane. We see that with the simple set of input features described in \autoref{subsec:Hpm_analysis}, we are able to exclude charged Higgses with masses up to 2 TeV for neutral Higgses \emph{H} with masses up to 500 GeV. We expect that the addition of high-level features such as invariant masses will substantially improve the results. In all cases, the significance is calculated using the formula $S/\sqrt{B}$. For the 14 TeV LHC, we consider an integrated luminosity of 300 fb$^{-1}$, and for a 100 TeV collider, we consider an integrated luminosity of 3000 fb$^{-1}$.

\section{Conclusions and outlook}\label{sec:exo_higgs_conclusion}
In this chapter, we have provided some preliminary results from an ongoing project aimed at evaluating the prospects of discovering new heavy scalars from $2$HDMs at the 14 TeV LHC and a 100 TeV collider, in the physically well-motivated but underexamined scenario where they can have large mass splittings. The approach we take is to examine the resulting `exotic' decay modes of these heavy scalars, since they can dominate over the conventional decay modes in certain regions of parameter space once kinematically allowed. We began by examining two of the benchmark planes suggested in \cite{Kling:2016opi}. The first, benchmark plane IB, with $m_{A} < m_{H} = m_{H^\pm}$ was probed using the exotic decay $H\rightarrow AZ$, with the subsequent decays $A\rightarrow \tau\tau, Z\rightarrow ll$. The second plane, labeled IIB, with $m_H < m_A = m_{H^\pm}$, was investigated using the exotic decay modes $A\rightarrow HZ$ and $H^\pm \rightarrow HW^\pm$. For the first mode, we consider the final state $\tau\tau ll$, similar to the $H\rightarrow AZ$ case, while for the second, we consider the more complicated final state $bbWW\tau\tau$, where one $\tau$ and one \emph{W} decay hadronically, and the other $\tau$ and \emph{W} decay leptonically. We used boosted decision tree classifiers to separate signal and background events, using a mixture of low-level features and physically inspired high-level features that are unique to each of the channels. For the benchmark plane IB, we found that we are able to exclude points with $m_A = m_H^{\pm}$ up to 3.5 TeV at optimal points in the $m_A = m_{H^\pm}$ plane, with a fixed $m_A - m_H =$ 200 GeV, for a 100 TeV collider. For benchmark plane IIB, we found that the reach is higher - we can discover points with $m_H = m_{H^\pm}$ up to 3.5 TeV, again for the same mass difference between the parent and daughter Higgs. Preliminary results for the charged Higgs channel $H^\pm \rightarrow HW^\pm\rightarrow \tau\tau W^\pm$ show an exclusion reach in benchmark plane IIB up to charged Higgs masses of about 2 TeV, for $m_H$ up to 500 GeV. This region is complementary to the region explored by the channel $A\rightarrow HZ$ (\autoref{subsec:A_HZ_analysis}) in the benchmark plane IIB. These preliminary results look promising, and we aim to complete our investigation of all the benchmark planes in the near future. An extended scalar sector arises frequently in theories beyond the Standard Model, and would have important implications for the electroweak phase transition, electroweak baryogenesis, as well as the hierarchy problem. Thus, understanding its structure is vital, and exotic decays provide a complementary avenue to do so.
