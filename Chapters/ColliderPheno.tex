\chapter{Collider Phenomenology and Machine Learning}\label{ch:ColliderPheno}
In this chapter, we will discuss how to connect theory and experiment. We will begin by connecting scattering amplitudes and physically observed cross sections, after which we will briefly discuss detector components of the LHC, and then move on to the question of how we establish statistical significance given experimental data from detectors.

% Scattering amplitudes and cross sections
To observe physics at subatomic length scales, it is necessary to produce interactions on those scales. This requires colliding particles at speeds close to the speed of light. The energy of these collisions is so high that not only do the colliding particles scatter off of each other, but new particles can be created as well. 
The physical observable at particle colliders is a quantity known as the scattering cross section, denoted by $\sigma$. It is defined by the relation
\[R = \sigma\mathcal{L}\]
where \emph{R} is the rate of collisions per unit time, and $\mathcal{L}$ is the luminosity of the collider, that is, the number of particles passing through some cross-sectional area per unit time. The scattering cross-section $\sigma$ for a generic process with particles 1 and 2 colliding to produce a set of particles \emph{X} as the final state is given by the integral of the scattering amplitude for the process, $|\mathcal{M}_{12\rightarrow X}|^2$ over the phase space of \emph{X}\footnote{Here we omit the form factors that would come into play at a hadronic collider, without affecting our narrative.}. 
\begin{equation}
  \sigma_{12} = \int d\Pi_X |\mathcal{M}_{12\rightarrow X}|^2
\end{equation}
The dynamics of the collision event are contained in the matrix element $\mathcal{M}$, which will be influenced by the presence of BSM physics.
Thus, the typical quantities of interest are the \emph{differential} cross sections, $d\sigma/dx$, where \emph{x} represents some kinematic variable. 
Particle colliders are broadly categorized as either lepton or hadron colliders. Each one has its advantages and disadvantages. Lepton colliders have cleaner signals than hadron colliders due to the lack of large QCD backgrounds. On the other hand, hadron colliders are able to reach much larger center-of-mass energies, since leptons lose large amounts of energy via synchrotron radiation when forced along a circular path. In this work, we will focus on the phenomenology of hadron colliders, both present and future.
\section{Detector design for hadron colliders}
Particle detectors can be thought of as sophisticated videocameras with extremely high framerates. Conversely, the digital cameras that we use in our daily lives can be thought of as particle detectors, except designed for only one kind of particle, the photon. The response of a detector to an incident particle can take on a variety of forms. In a gaseous detector such as a  Geiger-Muller tube, an energetic incident particle will lead to the ionization of a large fraction of the gas molecules, followed by rapid recombination. This is manifested as an electrical pulse. The analogue of ionization for a solid-state detector (such as the ones used in regular digital cameras) is the creation of a large number of electron-hole pairs. The design of a particle detector will be based upon the particles it aims to detect as well as the desired precision and accuracy.
At the Large Hadron Collider at CERN, the two major detectors are ATLAS (A Toroidal LHC Apparatus) and CMS (Compact Muon Solenoid). They differ slightly in construction, but essentially probe the same physics. The components common to them are the following.
\paragraph{Tracking chamber}
The innermost part of the detector is the tracking chamber. It consists of layers of solid-state detectors that can accurately measure the paths of charged particles that are formed in the particle collision events. The presence of a strong magnetic field bends the paths of these particles, enabling us to learn about their charge and mass.
\paragraph{Calorimeter(s)}
If a particle is energetic enough to go beyond the tracking chamber, it enters the calorimeter region. A calorimeter consists of materials dense enough to completely absorb the energy of an incident particle and stop it in its tracks. At ATLAS and CMS, the calorimeter is actually a combination of two layers that are designed to stop different kinds of particles. The electromagnetic calorimeter is designed to measure the energy of electrons and photons, while the hadronic calorimeter is designed to stop (and you might have guessed this already) hadrons. 
\paragraph{Muon Detectors} The muon, being about 200 times heavier than the electron, experiences much less energy loss through bremsstrahlung, and is able to bypass both the tracking chamber and calorimeter layers. For this reason, muon detectors form the outermost layer of a particle detector.

A transverse slice of the CMS detector is shown in \autoref{fig:CMS_slice}, depicting the trajectories taken by different kinds of particles through the layers of the detector.
The specific design of a detector depends upon the vision of the collaboration running it. With a finite construction and operation budget, tradeoffs must inevitably be made. Although ATLAS and CMS are both state of the art multipurpose detectors, they have their strengths and weaknesses relative to each other based upon different sets of priorities. For this reason, we do not provide the specifics of their construction, and instead point the reader to \citep{Froidevaux2006} for a detailed review and comparison of the two.
For our purposes, the features of the collision events that we use to perform our analyses (particle momentum, missing transverse energy, etc.) can be measured with either of these detectors.
\begin{figure}[h]
  \begin{sidecaption}
    {Transverse slice of the CMS detector, showing the paths of various particles. Source: \citep{CMS_Slice}}
    \centering
    \includegraphics[trim={0 0 0 4cm},clip,width=\textwidth]{images/CMS_slice}
  \end{sidecaption}
  \label{fig:CMS_slice}
\end{figure}
\paragraph{Trigger} As mentioned in \autoref{ch:introduction}, there are hundreds of millions of collisions every second at the LHC. Recording and analyzing all of these events would be impractical, given that most of the collisions simply involve small deflections. What we are really interested in are the \emph{hard} collisions, with the final state particles having a high amount of momentum in the transverse plane. To filter out the uninteresting events, we employ a \emph{trigger} - that is, a condition that an event must satisfy to be stored for further analysis. For example, in \autoref{ch:DM_100_TeV}, we choose to trigger on a hard lepton, that is, one with a high transverse momentum.
\paragraph{Invisibles} An important point to note is that certain particles, such as neutrinos and dark matter candidates, escape the detector entirely, leaving no tracks or energy deposits. The existence of one of these `invisible' particles in a collision event can only be inferred from an observed imbalance of momenta of the final state particles in the transverse direction. Thus, for analyses involving neutrinos or dark matter, the kinematic quantity known as \emph{missing transverse energy}, denoted by $\slashed{E}_T$, is of utmost importance.

\section{Anatomy of a collider analysis}
At its heart, the goal of a collider analysis is to compare the predictions of the SM and BSM theories with actual experimental data, and estimate the level of compatibility between them. To do this well, we need precise theoretical predictions of the differential cross sections that we can expect to see at the collider. This is done with the help of Monte Carlo methods and detector simulations, as described below. 
\subsection{Parton-level event generation}
As mentioned earlier, the cross section for a scattering process is an integral of the scattering amplitude over the phase space of the final state particles. 
In general, these integrals do not have a closed form solution, and so we must resort to numerical integration. The simplest Monte Carlo integration method, the \emph{acceptance-rejection} method, involves randomly sampling points within the limits of integration with a uniform probability distribution, and testing whether they lie under the `curve' specified by the integrand. The fraction of points that pass this test, multiplied by the volume of this `bounding box' specified by the integration limits, gives the definite integral. This method is difficult to view in multiple dimensions, but not too hard to grasp in one or two dimensions - see \citep{Pyarelal2011} for a brief overview and implementation. However, drawing the random samples from a uniform probability distribution is not the most efficient method for performing Monte Carlo integrals - programs such MadGraph5 and MadEvent \citep{Alwall2014} do this in more sophisticated ways, determining the roughly most important regions of phase space and concentrating the sampling within those regions. These `points' are simply the particle collision events themselves, with coordinates given by the four-momenta of the final state particles. 
\subsection{Showering and hadronization}
At hadron colliders, dealing with the matrix element $\mathcal{M}$ alone will not suffice. We must also take into account nonperturbative QCD effects, such as the radiation of soft gluons, and the formation of complex hadronic final states. For example, even if an energetic quark is contained in an event generated by MadEvent, it will not be detected as an elementary particle. It will radiate gluons that themselves split to form new quarks, that subsequently form bound states. This collection of hadronic bound states is termed a jet, and has a momentum collinear with the momentum of the original quark. The identity of this quark can be determined (with a finite efficiency) from the properties of the jet. To handle these non-perturbative effects, we interface MadEvent with the program Pythia\citep{Sjostrand2006} which performs the steps of parton showering and hadronization.
\subsection{Detector simulation and reconstruction}
A full collider analysis carried out by experimentalists will involve detailed simulations of the detector response, using programs such as GEANT4 \citep{Agostinelli2003}. For our purposes, however, it is enough to parameterize the detector response at a higher level - for example, specifying a fixed probability for identifying a certain particle. This is done using the Delphes 3 framework \citep{DeFavereau2014a}, which provides a way to perform a fast, modular simulation of the detector response. 

\subsection{Hypothesis testing}
After signal and background samples have been generated using Monte Carlo methods and passed through a detector simulation, they are compared to actual experimental data. Doing this enables us to do one or more of the following:
\begin{itemize}
  \item Estimate some parameter, for example the mass of a particle, or a coupling strength.
  \item Set upper limits on the rate of occurrence of a process, and translating those limits into limits on the parameter space of BSM theories.
  \item Discover a new particle
  \item Compare theoretical and experimental differential cross sections using a goodness-of-fit test.
\end{itemize}
As it turns out, all of these can be subsumed into the larger framework of \emph{hypothesis testing} \citep{Heinrich}. 
%In our case, we most interested in discovering new particles, or failing that, excluding regions of parameter space for BSM theories.
In the following section, we will address how we answer the question ``Does this experimental data imply the existence of new physics (or conversely, rule it out) ?". This discussion is adapted from the one in \citep{Cranmer2015}. 

\section{Statistical significance in particle physics}
A hypothesis is a claim about what the experimental data will look like, given some model parameters. Let us consider hypothesis testing in the context of a simple counting experiment, in the frequentist interpretation. In this experiment, we examine the subset of a dataset $\mathcal{D}$ that resides in a region of the `data space' labeled the \emph{signal region} (SR), and contains $n_{SR}$ events. Our competing hypotheses are the `background-only' and `signal plus background' hypotheses, described in \autoref{tab:hypotheses}. 
\begin{table}
  \begin{tabular}{llll}
    \toprule
    Symbol & Statistical name & Physics name & Probability model\\
    \midrule
    $H_0$ & Null & Background-only & Pois($n_{SR}|\nu_B$)\\
    $H_1$ & Alternate & Signal+Background & Pois($n_{SR}|\nu_S+\nu_B$)\\
    \bottomrule
  \end{tabular}
  \caption{The two competing hypotheses for our simple number counting example. Source: \cite{Cranmer2015}.}
  \label{tab:hypotheses}
\end{table}
The probability models associated with these hypotheses are Poisson models\footnote{The exact probabiliity distribution for counting experiments is given by the binomial frequency function. In the limit of a large number of events, this approaches the Poisson frequency function, given by
  \[\text{Pois}(n|\nu) = \nu^n\frac{e^{-\nu}}{n!}\]
  which describes the probability of observing \emph{n} events given that the mean expected number of events is $\nu$.
}. The model for the 'background-only' hypothesis is Pois$(n_{SR}|\nu_B)$, that is, the probability of obtaining $n_{SR}$ events in the signal region when $\nu_B$ events are expected from the background process. The competing hypothesis, `signal plus background', on the other hand, predicts $\nu_S+\nu_B$ events in the signal region, with $\nu_S$ events from the signal process and $\nu_B$ events from the background process. The probability that the background-only hypothesis will produce at least $n_\text{SR}$ events is given by
\[p_\text{background only} = \sum_{n=n_\text{SR}}^\infty \text{Pois}(n|\nu_B)\]
The lower this probability is, the less likely it is that the background-only hypothesis can account for the observed number of events in the signal region. This quantity is also known colloquially as the \emph{p}-value - in this case, it is the \emph{p}-value for the background-only hypothesis. The corresponding \emph{p}-value for the signal + background hypothesis is given by:
\[p_\text{signal + background} = \sum_{n=n_\text{SR}}^\infty \text{Pois}(n|\nu_S + \nu_B)\]
For phenomenology, there are usually two \emph{p}-value thresholds of interest. The first is the exclusion threshold - if $p_\text{signal + background} < 0.05$, we consider the signal + background hypothesis `excluded'. The second is the discovery criterion, $p_\text{background only} < 2.87\times 10^{-7}$. When this condition is satisfied, we choose to reject the background-only hypothesis, and claim discovery of new physics. In practice, \emph{p}-values are reported in terms of equivalent \emph{Z}-values. The \emph{Z} value corresponding to a \emph{p}-value $p_0$ is given by
\[Z = \Phi^{-1}(1-p_0)\]
where $\Phi^{-1}$ is the inverse of the cumulative distribution function of the standard normal distribution (a Gaussian distribution with mean 0 and variance 1). The equivalent \emph{Z}-values for the exclusion and discovery \emph{p}-value thresholds are 1.96 and 5, respectively. In particle physics parlance, the equivalent \emph{Z} value is known as the \emph{significance}, and is reported as $Z\sigma$.
The significance is maximized by choosing signal regions with low values of $\nu_B$.
This can be seen from the expressions for $p_\text{background only}$. Lower values of $\nu_B$ lead to lower values of $p_\text{background only}$, minimizing the probability, denoted $\alpha$, that we will wrongly reject the background-only hypothesis when it is true. This is known as a Type-I error. However, in general, there can be multiple, or even an infinite number of such regions in data space. Moreover, this criterion does not let us say anything about the signal + background hypothesis. Thus, we also require that the choice of the signal region leads to a low probability (denoted $\beta$) of wrongly accepting the background-only hypothesis when the signal + background hypothesis is true (a Type-II error).

In the language of hypothesis testing, then, the job of a collider phenomenologist like myself is to find regions of data space that \emph{minimize} the probability of wrongly accepting the null hypothesis when the alternate hypothesis is true, given a \emph{fixed} probability of wrongly rejecting the null hypothesis when the alternate hypothesis is true.

Whew, that is a mouthful! In more concise terms, we would like to minimize $\beta$ for a given value of $\alpha$. The quantity $\alpha$ is known as the \emph{size} of the test, and the quantity $1-\beta$ is known as the \emph{power} of the test. And more importantly, in practical terms, this task boils down to finding regions of the data space that are densely populated by signal events (minimizing $\beta$) and sparsely populated by background events (minimizing $\alpha$). After isolating a promising signal region, we calculate the maximum achievable value of \emph{Z}. In general, the value of \emph{Z} will be calculated differently for claiming exclusion or discovery. However, we are not performing a full collider analysis, but rather a sort of `feasibility study', for which we simply use the asymptotic formula for \emph{Z} obtained by taking limit in which the probability distributions are Gaussian, and the signal rates are much smaller than the background rates.
\[Z \approx \frac{n_S}{\sqrt{n_B}}\]
where $n_S$ and $n_B$ are the number of signal and background events observed in the signal region respectively.

Traditionally, promising signal regions are found by formulating kinematic variables that can efficiently discriminate between signal and background events. These variables are designed based on our knowledge of the kinematics of the final state particles in the signal and background events. They include variables such as invariant mass, missing transverse energy, and transverse mass. While this approach has served us well so far, the models being examined are grown increasingly complex, with large parameter spaces, and with the increase in collision energy comes a much larger rate of background event production. The boundary of the optimal signal region in data space, can potentially be highly non-linear. The traditional `cut-and-count' strategy finds the signal region by applying a series of one or two dimensional selection cuts on the data. This approach can potentially miss higher-dimensional correlations in the data space. These correlations can be found efficiently through the use of \emph{machine learning} techniques, which we discuss in \autoref{sec:MachineLearning}.

\paragraph{Reach in parameter space} As mentioned earlier, new physics theories such as Two Higgs Doublet Models can have extremely large parameter spaces. Each point of a theory's parameter space can be considered as a new physics hypothesis. In our analyses, we aim to perform hypothesis tests for a large number of points in parameter space, and ascertain which of the points, or hypotheses, can be rejected, or alternatively, discovered. Doing this, we find contours of the the equivalent \emph{Z} values in the parameter space. Typically we show only the contours for $Z = 1.96$ and $Z=5$, that is, the discovery and exclusion contours. We term the area bounded by these contours the \emph{reach} of our analysis in parameter space. 

\section{Machine learning in particle physics}\label{sec:MachineLearning}
Separating signals and backgrounds in particle physics is, in the language of machine learning, a \emph{classification} problem. One can consider particle collision events as residing in a space, where the coordinates of each event are represented by various \emph{features} of the event. This is equivalent to the `data space' discussed in the previous section. These features could be high-level ones such as invariant masses, or low-level ones, such as the momentum components of individual final state particles. What we do when we perform a cut-and-count analysis is to try and isolate a region of this space that is rich in signal events. Typically, the most straightforward approach is to define a sort of `box' in feature space using rectangular, one-dimensional cuts. However, there is no guarantee that signal events are confined to such a box --- correlations between the features can distort the distribution of events in feature space. 
\strictpagecheck
\begin{figure}
  \begin{sidecaption}{Illustration of a non-linear decision boundary in two-dimensional feature space.}
  %% Creator: Matplotlib, PGF backend
%%
%% To include the figure in your LaTeX document, write
%%   \input{<filename>.pgf}
%%
%% Make sure the required packages are loaded in your preamble
%%   \usepackage{pgf}
%%
%% Figures using additional raster images can only be included by \input if
%% they are in the same directory as the main LaTeX file. For loading figures
%% from other directories you can use the `import` package
%%   \usepackage{import}
%% and then include the figures with
%%   \import{<path to file>}{<filename>.pgf}
%%
%% Matplotlib used the following preamble
%%   \usepackage{fontspec}
%%   \setmainfont{Minion Pro}
%%   \setsansfont{Lucida Grande}
%%   \setmonofont{Andale Mono}
%%
\begingroup%
\makeatletter%
\begin{pgfpicture}%
\pgfpathrectangle{\pgfpointorigin}{\pgfqpoint{3.880000in}{3.880000in}}%
\pgfusepath{use as bounding box, clip}%
\begin{pgfscope}%
\pgfsetbuttcap%
\pgfsetmiterjoin%
\definecolor{currentfill}{rgb}{0.941176,0.941176,0.941176}%
\pgfsetfillcolor{currentfill}%
\pgfsetlinewidth{0.000000pt}%
\definecolor{currentstroke}{rgb}{0.941176,0.941176,0.941176}%
\pgfsetstrokecolor{currentstroke}%
\pgfsetdash{}{0pt}%
\pgfpathmoveto{\pgfqpoint{0.000000in}{0.000000in}}%
\pgfpathlineto{\pgfqpoint{3.880000in}{0.000000in}}%
\pgfpathlineto{\pgfqpoint{3.880000in}{3.880000in}}%
\pgfpathlineto{\pgfqpoint{0.000000in}{3.880000in}}%
\pgfpathclose%
\pgfusepath{fill}%
\end{pgfscope}%
\begin{pgfscope}%
\pgfsetbuttcap%
\pgfsetmiterjoin%
\definecolor{currentfill}{rgb}{0.941176,0.941176,0.941176}%
\pgfsetfillcolor{currentfill}%
\pgfsetlinewidth{0.000000pt}%
\definecolor{currentstroke}{rgb}{0.000000,0.000000,0.000000}%
\pgfsetstrokecolor{currentstroke}%
\pgfsetstrokeopacity{0.000000}%
\pgfsetdash{}{0pt}%
\pgfpathmoveto{\pgfqpoint{0.511823in}{0.504323in}}%
\pgfpathlineto{\pgfqpoint{3.730000in}{0.504323in}}%
\pgfpathlineto{\pgfqpoint{3.730000in}{3.730000in}}%
\pgfpathlineto{\pgfqpoint{0.511823in}{3.730000in}}%
\pgfpathclose%
\pgfusepath{fill}%
\end{pgfscope}%
\begin{pgfscope}%
\pgfpathrectangle{\pgfqpoint{0.511823in}{0.504323in}}{\pgfqpoint{3.218177in}{3.225677in}} %
\pgfusepath{clip}%
\pgfsetbuttcap%
\pgfsetroundjoin%
\pgfsetlinewidth{1.003750pt}%
\definecolor{currentstroke}{rgb}{0.796078,0.796078,0.796078}%
\pgfsetstrokecolor{currentstroke}%
\pgfsetdash{}{0pt}%
\pgfpathmoveto{\pgfqpoint{0.686439in}{0.504323in}}%
\pgfpathlineto{\pgfqpoint{0.686439in}{3.730000in}}%
\pgfusepath{stroke}%
\end{pgfscope}%
\begin{pgfscope}%
\pgftext[x=0.686439in,y=0.455712in,,top]{\rmfamily\fontsize{10.000000}{12.000000}\selectfont 0}%
\end{pgfscope}%
\begin{pgfscope}%
\pgfpathrectangle{\pgfqpoint{0.511823in}{0.504323in}}{\pgfqpoint{3.218177in}{3.225677in}} %
\pgfusepath{clip}%
\pgfsetbuttcap%
\pgfsetroundjoin%
\pgfsetlinewidth{1.003750pt}%
\definecolor{currentstroke}{rgb}{0.796078,0.796078,0.796078}%
\pgfsetstrokecolor{currentstroke}%
\pgfsetdash{}{0pt}%
\pgfpathmoveto{\pgfqpoint{1.349754in}{0.504323in}}%
\pgfpathlineto{\pgfqpoint{1.349754in}{3.730000in}}%
\pgfusepath{stroke}%
\end{pgfscope}%
\begin{pgfscope}%
\pgftext[x=1.349754in,y=0.455712in,,top]{\rmfamily\fontsize{10.000000}{12.000000}\selectfont 5}%
\end{pgfscope}%
\begin{pgfscope}%
\pgfpathrectangle{\pgfqpoint{0.511823in}{0.504323in}}{\pgfqpoint{3.218177in}{3.225677in}} %
\pgfusepath{clip}%
\pgfsetbuttcap%
\pgfsetroundjoin%
\pgfsetlinewidth{1.003750pt}%
\definecolor{currentstroke}{rgb}{0.796078,0.796078,0.796078}%
\pgfsetstrokecolor{currentstroke}%
\pgfsetdash{}{0pt}%
\pgfpathmoveto{\pgfqpoint{2.013069in}{0.504323in}}%
\pgfpathlineto{\pgfqpoint{2.013069in}{3.730000in}}%
\pgfusepath{stroke}%
\end{pgfscope}%
\begin{pgfscope}%
\pgftext[x=2.013069in,y=0.455712in,,top]{\rmfamily\fontsize{10.000000}{12.000000}\selectfont 10}%
\end{pgfscope}%
\begin{pgfscope}%
\pgfpathrectangle{\pgfqpoint{0.511823in}{0.504323in}}{\pgfqpoint{3.218177in}{3.225677in}} %
\pgfusepath{clip}%
\pgfsetbuttcap%
\pgfsetroundjoin%
\pgfsetlinewidth{1.003750pt}%
\definecolor{currentstroke}{rgb}{0.796078,0.796078,0.796078}%
\pgfsetstrokecolor{currentstroke}%
\pgfsetdash{}{0pt}%
\pgfpathmoveto{\pgfqpoint{2.676384in}{0.504323in}}%
\pgfpathlineto{\pgfqpoint{2.676384in}{3.730000in}}%
\pgfusepath{stroke}%
\end{pgfscope}%
\begin{pgfscope}%
\pgftext[x=2.676384in,y=0.455712in,,top]{\rmfamily\fontsize{10.000000}{12.000000}\selectfont 15}%
\end{pgfscope}%
\begin{pgfscope}%
\pgfpathrectangle{\pgfqpoint{0.511823in}{0.504323in}}{\pgfqpoint{3.218177in}{3.225677in}} %
\pgfusepath{clip}%
\pgfsetbuttcap%
\pgfsetroundjoin%
\pgfsetlinewidth{1.003750pt}%
\definecolor{currentstroke}{rgb}{0.796078,0.796078,0.796078}%
\pgfsetstrokecolor{currentstroke}%
\pgfsetdash{}{0pt}%
\pgfpathmoveto{\pgfqpoint{3.339699in}{0.504323in}}%
\pgfpathlineto{\pgfqpoint{3.339699in}{3.730000in}}%
\pgfusepath{stroke}%
\end{pgfscope}%
\begin{pgfscope}%
\pgftext[x=3.339699in,y=0.455712in,,top]{\rmfamily\fontsize{10.000000}{12.000000}\selectfont 20}%
\end{pgfscope}%
\begin{pgfscope}%
\pgftext[x=2.120911in,y=0.267934in,,top]{\rmfamily\fontsize{10.000000}{12.000000}\selectfont \emph{x}}%
\end{pgfscope}%
\begin{pgfscope}%
\pgfpathrectangle{\pgfqpoint{0.511823in}{0.504323in}}{\pgfqpoint{3.218177in}{3.225677in}} %
\pgfusepath{clip}%
\pgfsetbuttcap%
\pgfsetroundjoin%
\pgfsetlinewidth{1.003750pt}%
\definecolor{currentstroke}{rgb}{0.796078,0.796078,0.796078}%
\pgfsetstrokecolor{currentstroke}%
\pgfsetdash{}{0pt}%
\pgfpathmoveto{\pgfqpoint{0.511823in}{0.681428in}}%
\pgfpathlineto{\pgfqpoint{3.730000in}{0.681428in}}%
\pgfusepath{stroke}%
\end{pgfscope}%
\begin{pgfscope}%
\pgftext[x=0.396545in,y=0.631984in,left,base]{\rmfamily\fontsize{10.000000}{12.000000}\selectfont 0}%
\end{pgfscope}%
\begin{pgfscope}%
\pgfpathrectangle{\pgfqpoint{0.511823in}{0.504323in}}{\pgfqpoint{3.218177in}{3.225677in}} %
\pgfusepath{clip}%
\pgfsetbuttcap%
\pgfsetroundjoin%
\pgfsetlinewidth{1.003750pt}%
\definecolor{currentstroke}{rgb}{0.796078,0.796078,0.796078}%
\pgfsetstrokecolor{currentstroke}%
\pgfsetdash{}{0pt}%
\pgfpathmoveto{\pgfqpoint{0.511823in}{1.298614in}}%
\pgfpathlineto{\pgfqpoint{3.730000in}{1.298614in}}%
\pgfusepath{stroke}%
\end{pgfscope}%
\begin{pgfscope}%
\pgftext[x=0.396545in,y=1.249170in,left,base]{\rmfamily\fontsize{10.000000}{12.000000}\selectfont 5}%
\end{pgfscope}%
\begin{pgfscope}%
\pgfpathrectangle{\pgfqpoint{0.511823in}{0.504323in}}{\pgfqpoint{3.218177in}{3.225677in}} %
\pgfusepath{clip}%
\pgfsetbuttcap%
\pgfsetroundjoin%
\pgfsetlinewidth{1.003750pt}%
\definecolor{currentstroke}{rgb}{0.796078,0.796078,0.796078}%
\pgfsetstrokecolor{currentstroke}%
\pgfsetdash{}{0pt}%
\pgfpathmoveto{\pgfqpoint{0.511823in}{1.915800in}}%
\pgfpathlineto{\pgfqpoint{3.730000in}{1.915800in}}%
\pgfusepath{stroke}%
\end{pgfscope}%
\begin{pgfscope}%
\pgftext[x=0.329878in,y=1.866356in,left,base]{\rmfamily\fontsize{10.000000}{12.000000}\selectfont 10}%
\end{pgfscope}%
\begin{pgfscope}%
\pgfpathrectangle{\pgfqpoint{0.511823in}{0.504323in}}{\pgfqpoint{3.218177in}{3.225677in}} %
\pgfusepath{clip}%
\pgfsetbuttcap%
\pgfsetroundjoin%
\pgfsetlinewidth{1.003750pt}%
\definecolor{currentstroke}{rgb}{0.796078,0.796078,0.796078}%
\pgfsetstrokecolor{currentstroke}%
\pgfsetdash{}{0pt}%
\pgfpathmoveto{\pgfqpoint{0.511823in}{2.532986in}}%
\pgfpathlineto{\pgfqpoint{3.730000in}{2.532986in}}%
\pgfusepath{stroke}%
\end{pgfscope}%
\begin{pgfscope}%
\pgftext[x=0.329878in,y=2.483542in,left,base]{\rmfamily\fontsize{10.000000}{12.000000}\selectfont 15}%
\end{pgfscope}%
\begin{pgfscope}%
\pgfpathrectangle{\pgfqpoint{0.511823in}{0.504323in}}{\pgfqpoint{3.218177in}{3.225677in}} %
\pgfusepath{clip}%
\pgfsetbuttcap%
\pgfsetroundjoin%
\pgfsetlinewidth{1.003750pt}%
\definecolor{currentstroke}{rgb}{0.796078,0.796078,0.796078}%
\pgfsetstrokecolor{currentstroke}%
\pgfsetdash{}{0pt}%
\pgfpathmoveto{\pgfqpoint{0.511823in}{3.150172in}}%
\pgfpathlineto{\pgfqpoint{3.730000in}{3.150172in}}%
\pgfusepath{stroke}%
\end{pgfscope}%
\begin{pgfscope}%
\pgftext[x=0.329878in,y=3.100728in,left,base]{\rmfamily\fontsize{10.000000}{12.000000}\selectfont 20}%
\end{pgfscope}%
\begin{pgfscope}%
\pgftext[x=0.274323in,y=2.117161in,,bottom,rotate=90.000000]{\rmfamily\fontsize{10.000000}{12.000000}\selectfont \emph{y}}%
\end{pgfscope}%
\begin{pgfscope}%
\pgfpathrectangle{\pgfqpoint{0.511823in}{0.504323in}}{\pgfqpoint{3.218177in}{3.225677in}} %
\pgfusepath{clip}%
\pgfsetbuttcap%
\pgfsetroundjoin%
\definecolor{currentfill}{rgb}{0.501961,0.000000,0.000000}%
\pgfsetfillcolor{currentfill}%
\pgfsetfillopacity{0.400000}%
\pgfsetlinewidth{0.501875pt}%
\definecolor{currentstroke}{rgb}{0.501961,0.000000,0.000000}%
\pgfsetstrokecolor{currentstroke}%
\pgfsetstrokeopacity{0.400000}%
\pgfsetdash{}{0pt}%
\pgfpathmoveto{\pgfqpoint{3.480634in}{0.639761in}}%
\pgfpathcurveto{\pgfqpoint{3.491685in}{0.639761in}}{\pgfqpoint{3.502284in}{0.644152in}}{\pgfqpoint{3.510097in}{0.651965in}}%
\pgfpathcurveto{\pgfqpoint{3.517911in}{0.659779in}}{\pgfqpoint{3.522301in}{0.670378in}}{\pgfqpoint{3.522301in}{0.681428in}}%
\pgfpathcurveto{\pgfqpoint{3.522301in}{0.692478in}}{\pgfqpoint{3.517911in}{0.703077in}}{\pgfqpoint{3.510097in}{0.710891in}}%
\pgfpathcurveto{\pgfqpoint{3.502284in}{0.718704in}}{\pgfqpoint{3.491685in}{0.723095in}}{\pgfqpoint{3.480634in}{0.723095in}}%
\pgfpathcurveto{\pgfqpoint{3.469584in}{0.723095in}}{\pgfqpoint{3.458985in}{0.718704in}}{\pgfqpoint{3.451172in}{0.710891in}}%
\pgfpathcurveto{\pgfqpoint{3.443358in}{0.703077in}}{\pgfqpoint{3.438968in}{0.692478in}}{\pgfqpoint{3.438968in}{0.681428in}}%
\pgfpathcurveto{\pgfqpoint{3.438968in}{0.670378in}}{\pgfqpoint{3.443358in}{0.659779in}}{\pgfqpoint{3.451172in}{0.651965in}}%
\pgfpathcurveto{\pgfqpoint{3.458985in}{0.644152in}}{\pgfqpoint{3.469584in}{0.639761in}}{\pgfqpoint{3.480634in}{0.639761in}}%
\pgfpathclose%
\pgfusepath{stroke,fill}%
\end{pgfscope}%
\begin{pgfscope}%
\pgfpathrectangle{\pgfqpoint{0.511823in}{0.504323in}}{\pgfqpoint{3.218177in}{3.225677in}} %
\pgfusepath{clip}%
\pgfsetbuttcap%
\pgfsetroundjoin%
\definecolor{currentfill}{rgb}{0.501961,0.000000,0.000000}%
\pgfsetfillcolor{currentfill}%
\pgfsetfillopacity{0.400000}%
\pgfsetlinewidth{0.501875pt}%
\definecolor{currentstroke}{rgb}{0.501961,0.000000,0.000000}%
\pgfsetstrokecolor{currentstroke}%
\pgfsetstrokeopacity{0.400000}%
\pgfsetdash{}{0pt}%
\pgfpathmoveto{\pgfqpoint{3.402503in}{0.647717in}}%
\pgfpathcurveto{\pgfqpoint{3.413553in}{0.647717in}}{\pgfqpoint{3.424152in}{0.652107in}}{\pgfqpoint{3.431965in}{0.659921in}}%
\pgfpathcurveto{\pgfqpoint{3.439779in}{0.667734in}}{\pgfqpoint{3.444169in}{0.678333in}}{\pgfqpoint{3.444169in}{0.689383in}}%
\pgfpathcurveto{\pgfqpoint{3.444169in}{0.700433in}}{\pgfqpoint{3.439779in}{0.711032in}}{\pgfqpoint{3.431965in}{0.718846in}}%
\pgfpathcurveto{\pgfqpoint{3.424152in}{0.726660in}}{\pgfqpoint{3.413553in}{0.731050in}}{\pgfqpoint{3.402503in}{0.731050in}}%
\pgfpathcurveto{\pgfqpoint{3.391452in}{0.731050in}}{\pgfqpoint{3.380853in}{0.726660in}}{\pgfqpoint{3.373040in}{0.718846in}}%
\pgfpathcurveto{\pgfqpoint{3.365226in}{0.711032in}}{\pgfqpoint{3.360836in}{0.700433in}}{\pgfqpoint{3.360836in}{0.689383in}}%
\pgfpathcurveto{\pgfqpoint{3.360836in}{0.678333in}}{\pgfqpoint{3.365226in}{0.667734in}}{\pgfqpoint{3.373040in}{0.659921in}}%
\pgfpathcurveto{\pgfqpoint{3.380853in}{0.652107in}}{\pgfqpoint{3.391452in}{0.647717in}}{\pgfqpoint{3.402503in}{0.647717in}}%
\pgfpathclose%
\pgfusepath{stroke,fill}%
\end{pgfscope}%
\begin{pgfscope}%
\pgfpathrectangle{\pgfqpoint{0.511823in}{0.504323in}}{\pgfqpoint{3.218177in}{3.225677in}} %
\pgfusepath{clip}%
\pgfsetbuttcap%
\pgfsetroundjoin%
\definecolor{currentfill}{rgb}{0.501961,0.000000,0.000000}%
\pgfsetfillcolor{currentfill}%
\pgfsetfillopacity{0.400000}%
\pgfsetlinewidth{0.501875pt}%
\definecolor{currentstroke}{rgb}{0.501961,0.000000,0.000000}%
\pgfsetstrokecolor{currentstroke}%
\pgfsetstrokeopacity{0.400000}%
\pgfsetdash{}{0pt}%
\pgfpathmoveto{\pgfqpoint{3.450365in}{0.655952in}}%
\pgfpathcurveto{\pgfqpoint{3.461415in}{0.655952in}}{\pgfqpoint{3.472014in}{0.660343in}}{\pgfqpoint{3.479828in}{0.668156in}}%
\pgfpathcurveto{\pgfqpoint{3.487641in}{0.675970in}}{\pgfqpoint{3.492032in}{0.686569in}}{\pgfqpoint{3.492032in}{0.697619in}}%
\pgfpathcurveto{\pgfqpoint{3.492032in}{0.708669in}}{\pgfqpoint{3.487641in}{0.719268in}}{\pgfqpoint{3.479828in}{0.727082in}}%
\pgfpathcurveto{\pgfqpoint{3.472014in}{0.734896in}}{\pgfqpoint{3.461415in}{0.739286in}}{\pgfqpoint{3.450365in}{0.739286in}}%
\pgfpathcurveto{\pgfqpoint{3.439315in}{0.739286in}}{\pgfqpoint{3.428716in}{0.734896in}}{\pgfqpoint{3.420902in}{0.727082in}}%
\pgfpathcurveto{\pgfqpoint{3.413088in}{0.719268in}}{\pgfqpoint{3.408698in}{0.708669in}}{\pgfqpoint{3.408698in}{0.697619in}}%
\pgfpathcurveto{\pgfqpoint{3.408698in}{0.686569in}}{\pgfqpoint{3.413088in}{0.675970in}}{\pgfqpoint{3.420902in}{0.668156in}}%
\pgfpathcurveto{\pgfqpoint{3.428716in}{0.660343in}}{\pgfqpoint{3.439315in}{0.655952in}}{\pgfqpoint{3.450365in}{0.655952in}}%
\pgfpathclose%
\pgfusepath{stroke,fill}%
\end{pgfscope}%
\begin{pgfscope}%
\pgfpathrectangle{\pgfqpoint{0.511823in}{0.504323in}}{\pgfqpoint{3.218177in}{3.225677in}} %
\pgfusepath{clip}%
\pgfsetbuttcap%
\pgfsetroundjoin%
\definecolor{currentfill}{rgb}{0.501961,0.000000,0.000000}%
\pgfsetfillcolor{currentfill}%
\pgfsetfillopacity{0.400000}%
\pgfsetlinewidth{0.501875pt}%
\definecolor{currentstroke}{rgb}{0.501961,0.000000,0.000000}%
\pgfsetstrokecolor{currentstroke}%
\pgfsetstrokeopacity{0.400000}%
\pgfsetdash{}{0pt}%
\pgfpathmoveto{\pgfqpoint{3.332399in}{0.663012in}}%
\pgfpathcurveto{\pgfqpoint{3.343449in}{0.663012in}}{\pgfqpoint{3.354048in}{0.667402in}}{\pgfqpoint{3.361862in}{0.675216in}}%
\pgfpathcurveto{\pgfqpoint{3.369675in}{0.683029in}}{\pgfqpoint{3.374066in}{0.693628in}}{\pgfqpoint{3.374066in}{0.704679in}}%
\pgfpathcurveto{\pgfqpoint{3.374066in}{0.715729in}}{\pgfqpoint{3.369675in}{0.726328in}}{\pgfqpoint{3.361862in}{0.734141in}}%
\pgfpathcurveto{\pgfqpoint{3.354048in}{0.741955in}}{\pgfqpoint{3.343449in}{0.746345in}}{\pgfqpoint{3.332399in}{0.746345in}}%
\pgfpathcurveto{\pgfqpoint{3.321349in}{0.746345in}}{\pgfqpoint{3.310750in}{0.741955in}}{\pgfqpoint{3.302936in}{0.734141in}}%
\pgfpathcurveto{\pgfqpoint{3.295122in}{0.726328in}}{\pgfqpoint{3.290732in}{0.715729in}}{\pgfqpoint{3.290732in}{0.704679in}}%
\pgfpathcurveto{\pgfqpoint{3.290732in}{0.693628in}}{\pgfqpoint{3.295122in}{0.683029in}}{\pgfqpoint{3.302936in}{0.675216in}}%
\pgfpathcurveto{\pgfqpoint{3.310750in}{0.667402in}}{\pgfqpoint{3.321349in}{0.663012in}}{\pgfqpoint{3.332399in}{0.663012in}}%
\pgfpathclose%
\pgfusepath{stroke,fill}%
\end{pgfscope}%
\begin{pgfscope}%
\pgfpathrectangle{\pgfqpoint{0.511823in}{0.504323in}}{\pgfqpoint{3.218177in}{3.225677in}} %
\pgfusepath{clip}%
\pgfsetbuttcap%
\pgfsetroundjoin%
\definecolor{currentfill}{rgb}{0.501961,0.000000,0.000000}%
\pgfsetfillcolor{currentfill}%
\pgfsetfillopacity{0.400000}%
\pgfsetlinewidth{0.501875pt}%
\definecolor{currentstroke}{rgb}{0.501961,0.000000,0.000000}%
\pgfsetstrokecolor{currentstroke}%
\pgfsetstrokeopacity{0.400000}%
\pgfsetdash{}{0pt}%
\pgfpathmoveto{\pgfqpoint{3.336088in}{0.670806in}}%
\pgfpathcurveto{\pgfqpoint{3.347138in}{0.670806in}}{\pgfqpoint{3.357737in}{0.675196in}}{\pgfqpoint{3.365551in}{0.683010in}}%
\pgfpathcurveto{\pgfqpoint{3.373364in}{0.690823in}}{\pgfqpoint{3.377755in}{0.701423in}}{\pgfqpoint{3.377755in}{0.712473in}}%
\pgfpathcurveto{\pgfqpoint{3.377755in}{0.723523in}}{\pgfqpoint{3.373364in}{0.734122in}}{\pgfqpoint{3.365551in}{0.741935in}}%
\pgfpathcurveto{\pgfqpoint{3.357737in}{0.749749in}}{\pgfqpoint{3.347138in}{0.754139in}}{\pgfqpoint{3.336088in}{0.754139in}}%
\pgfpathcurveto{\pgfqpoint{3.325038in}{0.754139in}}{\pgfqpoint{3.314439in}{0.749749in}}{\pgfqpoint{3.306625in}{0.741935in}}%
\pgfpathcurveto{\pgfqpoint{3.298811in}{0.734122in}}{\pgfqpoint{3.294421in}{0.723523in}}{\pgfqpoint{3.294421in}{0.712473in}}%
\pgfpathcurveto{\pgfqpoint{3.294421in}{0.701423in}}{\pgfqpoint{3.298811in}{0.690823in}}{\pgfqpoint{3.306625in}{0.683010in}}%
\pgfpathcurveto{\pgfqpoint{3.314439in}{0.675196in}}{\pgfqpoint{3.325038in}{0.670806in}}{\pgfqpoint{3.336088in}{0.670806in}}%
\pgfpathclose%
\pgfusepath{stroke,fill}%
\end{pgfscope}%
\begin{pgfscope}%
\pgfpathrectangle{\pgfqpoint{0.511823in}{0.504323in}}{\pgfqpoint{3.218177in}{3.225677in}} %
\pgfusepath{clip}%
\pgfsetbuttcap%
\pgfsetroundjoin%
\definecolor{currentfill}{rgb}{0.501961,0.000000,0.000000}%
\pgfsetfillcolor{currentfill}%
\pgfsetfillopacity{0.400000}%
\pgfsetlinewidth{0.501875pt}%
\definecolor{currentstroke}{rgb}{0.501961,0.000000,0.000000}%
\pgfsetstrokecolor{currentstroke}%
\pgfsetstrokeopacity{0.400000}%
\pgfsetdash{}{0pt}%
\pgfpathmoveto{\pgfqpoint{3.105860in}{0.675196in}}%
\pgfpathcurveto{\pgfqpoint{3.116910in}{0.675196in}}{\pgfqpoint{3.127509in}{0.679587in}}{\pgfqpoint{3.135322in}{0.687400in}}%
\pgfpathcurveto{\pgfqpoint{3.143136in}{0.695214in}}{\pgfqpoint{3.147526in}{0.705813in}}{\pgfqpoint{3.147526in}{0.716863in}}%
\pgfpathcurveto{\pgfqpoint{3.147526in}{0.727913in}}{\pgfqpoint{3.143136in}{0.738512in}}{\pgfqpoint{3.135322in}{0.746326in}}%
\pgfpathcurveto{\pgfqpoint{3.127509in}{0.754139in}}{\pgfqpoint{3.116910in}{0.758530in}}{\pgfqpoint{3.105860in}{0.758530in}}%
\pgfpathcurveto{\pgfqpoint{3.094810in}{0.758530in}}{\pgfqpoint{3.084210in}{0.754139in}}{\pgfqpoint{3.076397in}{0.746326in}}%
\pgfpathcurveto{\pgfqpoint{3.068583in}{0.738512in}}{\pgfqpoint{3.064193in}{0.727913in}}{\pgfqpoint{3.064193in}{0.716863in}}%
\pgfpathcurveto{\pgfqpoint{3.064193in}{0.705813in}}{\pgfqpoint{3.068583in}{0.695214in}}{\pgfqpoint{3.076397in}{0.687400in}}%
\pgfpathcurveto{\pgfqpoint{3.084210in}{0.679587in}}{\pgfqpoint{3.094810in}{0.675196in}}{\pgfqpoint{3.105860in}{0.675196in}}%
\pgfpathclose%
\pgfusepath{stroke,fill}%
\end{pgfscope}%
\begin{pgfscope}%
\pgfpathrectangle{\pgfqpoint{0.511823in}{0.504323in}}{\pgfqpoint{3.218177in}{3.225677in}} %
\pgfusepath{clip}%
\pgfsetbuttcap%
\pgfsetroundjoin%
\definecolor{currentfill}{rgb}{0.501961,0.000000,0.000000}%
\pgfsetfillcolor{currentfill}%
\pgfsetfillopacity{0.400000}%
\pgfsetlinewidth{0.501875pt}%
\definecolor{currentstroke}{rgb}{0.501961,0.000000,0.000000}%
\pgfsetstrokecolor{currentstroke}%
\pgfsetstrokeopacity{0.400000}%
\pgfsetdash{}{0pt}%
\pgfpathmoveto{\pgfqpoint{3.361603in}{0.686780in}}%
\pgfpathcurveto{\pgfqpoint{3.372653in}{0.686780in}}{\pgfqpoint{3.383252in}{0.691170in}}{\pgfqpoint{3.391066in}{0.698984in}}%
\pgfpathcurveto{\pgfqpoint{3.398880in}{0.706797in}}{\pgfqpoint{3.403270in}{0.717396in}}{\pgfqpoint{3.403270in}{0.728447in}}%
\pgfpathcurveto{\pgfqpoint{3.403270in}{0.739497in}}{\pgfqpoint{3.398880in}{0.750096in}}{\pgfqpoint{3.391066in}{0.757909in}}%
\pgfpathcurveto{\pgfqpoint{3.383252in}{0.765723in}}{\pgfqpoint{3.372653in}{0.770113in}}{\pgfqpoint{3.361603in}{0.770113in}}%
\pgfpathcurveto{\pgfqpoint{3.350553in}{0.770113in}}{\pgfqpoint{3.339954in}{0.765723in}}{\pgfqpoint{3.332140in}{0.757909in}}%
\pgfpathcurveto{\pgfqpoint{3.324327in}{0.750096in}}{\pgfqpoint{3.319936in}{0.739497in}}{\pgfqpoint{3.319936in}{0.728447in}}%
\pgfpathcurveto{\pgfqpoint{3.319936in}{0.717396in}}{\pgfqpoint{3.324327in}{0.706797in}}{\pgfqpoint{3.332140in}{0.698984in}}%
\pgfpathcurveto{\pgfqpoint{3.339954in}{0.691170in}}{\pgfqpoint{3.350553in}{0.686780in}}{\pgfqpoint{3.361603in}{0.686780in}}%
\pgfpathclose%
\pgfusepath{stroke,fill}%
\end{pgfscope}%
\begin{pgfscope}%
\pgfpathrectangle{\pgfqpoint{0.511823in}{0.504323in}}{\pgfqpoint{3.218177in}{3.225677in}} %
\pgfusepath{clip}%
\pgfsetbuttcap%
\pgfsetroundjoin%
\definecolor{currentfill}{rgb}{0.501961,0.000000,0.000000}%
\pgfsetfillcolor{currentfill}%
\pgfsetfillopacity{0.400000}%
\pgfsetlinewidth{0.501875pt}%
\definecolor{currentstroke}{rgb}{0.501961,0.000000,0.000000}%
\pgfsetstrokecolor{currentstroke}%
\pgfsetstrokeopacity{0.400000}%
\pgfsetdash{}{0pt}%
\pgfpathmoveto{\pgfqpoint{3.507135in}{0.697603in}}%
\pgfpathcurveto{\pgfqpoint{3.518185in}{0.697603in}}{\pgfqpoint{3.528784in}{0.701993in}}{\pgfqpoint{3.536598in}{0.709807in}}%
\pgfpathcurveto{\pgfqpoint{3.544412in}{0.717620in}}{\pgfqpoint{3.548802in}{0.728219in}}{\pgfqpoint{3.548802in}{0.739270in}}%
\pgfpathcurveto{\pgfqpoint{3.548802in}{0.750320in}}{\pgfqpoint{3.544412in}{0.760919in}}{\pgfqpoint{3.536598in}{0.768732in}}%
\pgfpathcurveto{\pgfqpoint{3.528784in}{0.776546in}}{\pgfqpoint{3.518185in}{0.780936in}}{\pgfqpoint{3.507135in}{0.780936in}}%
\pgfpathcurveto{\pgfqpoint{3.496085in}{0.780936in}}{\pgfqpoint{3.485486in}{0.776546in}}{\pgfqpoint{3.477672in}{0.768732in}}%
\pgfpathcurveto{\pgfqpoint{3.469859in}{0.760919in}}{\pgfqpoint{3.465469in}{0.750320in}}{\pgfqpoint{3.465469in}{0.739270in}}%
\pgfpathcurveto{\pgfqpoint{3.465469in}{0.728219in}}{\pgfqpoint{3.469859in}{0.717620in}}{\pgfqpoint{3.477672in}{0.709807in}}%
\pgfpathcurveto{\pgfqpoint{3.485486in}{0.701993in}}{\pgfqpoint{3.496085in}{0.697603in}}{\pgfqpoint{3.507135in}{0.697603in}}%
\pgfpathclose%
\pgfusepath{stroke,fill}%
\end{pgfscope}%
\begin{pgfscope}%
\pgfpathrectangle{\pgfqpoint{0.511823in}{0.504323in}}{\pgfqpoint{3.218177in}{3.225677in}} %
\pgfusepath{clip}%
\pgfsetbuttcap%
\pgfsetroundjoin%
\definecolor{currentfill}{rgb}{0.501961,0.000000,0.000000}%
\pgfsetfillcolor{currentfill}%
\pgfsetfillopacity{0.400000}%
\pgfsetlinewidth{0.501875pt}%
\definecolor{currentstroke}{rgb}{0.501961,0.000000,0.000000}%
\pgfsetstrokecolor{currentstroke}%
\pgfsetstrokeopacity{0.400000}%
\pgfsetdash{}{0pt}%
\pgfpathmoveto{\pgfqpoint{3.239664in}{0.699601in}}%
\pgfpathcurveto{\pgfqpoint{3.250715in}{0.699601in}}{\pgfqpoint{3.261314in}{0.703991in}}{\pgfqpoint{3.269127in}{0.711805in}}%
\pgfpathcurveto{\pgfqpoint{3.276941in}{0.719618in}}{\pgfqpoint{3.281331in}{0.730217in}}{\pgfqpoint{3.281331in}{0.741267in}}%
\pgfpathcurveto{\pgfqpoint{3.281331in}{0.752317in}}{\pgfqpoint{3.276941in}{0.762916in}}{\pgfqpoint{3.269127in}{0.770730in}}%
\pgfpathcurveto{\pgfqpoint{3.261314in}{0.778544in}}{\pgfqpoint{3.250715in}{0.782934in}}{\pgfqpoint{3.239664in}{0.782934in}}%
\pgfpathcurveto{\pgfqpoint{3.228614in}{0.782934in}}{\pgfqpoint{3.218015in}{0.778544in}}{\pgfqpoint{3.210202in}{0.770730in}}%
\pgfpathcurveto{\pgfqpoint{3.202388in}{0.762916in}}{\pgfqpoint{3.197998in}{0.752317in}}{\pgfqpoint{3.197998in}{0.741267in}}%
\pgfpathcurveto{\pgfqpoint{3.197998in}{0.730217in}}{\pgfqpoint{3.202388in}{0.719618in}}{\pgfqpoint{3.210202in}{0.711805in}}%
\pgfpathcurveto{\pgfqpoint{3.218015in}{0.703991in}}{\pgfqpoint{3.228614in}{0.699601in}}{\pgfqpoint{3.239664in}{0.699601in}}%
\pgfpathclose%
\pgfusepath{stroke,fill}%
\end{pgfscope}%
\begin{pgfscope}%
\pgfpathrectangle{\pgfqpoint{0.511823in}{0.504323in}}{\pgfqpoint{3.218177in}{3.225677in}} %
\pgfusepath{clip}%
\pgfsetbuttcap%
\pgfsetroundjoin%
\definecolor{currentfill}{rgb}{0.501961,0.000000,0.000000}%
\pgfsetfillcolor{currentfill}%
\pgfsetfillopacity{0.400000}%
\pgfsetlinewidth{0.501875pt}%
\definecolor{currentstroke}{rgb}{0.501961,0.000000,0.000000}%
\pgfsetstrokecolor{currentstroke}%
\pgfsetstrokeopacity{0.400000}%
\pgfsetdash{}{0pt}%
\pgfpathmoveto{\pgfqpoint{3.183515in}{0.705604in}}%
\pgfpathcurveto{\pgfqpoint{3.194565in}{0.705604in}}{\pgfqpoint{3.205164in}{0.709994in}}{\pgfqpoint{3.212978in}{0.717808in}}%
\pgfpathcurveto{\pgfqpoint{3.220791in}{0.725621in}}{\pgfqpoint{3.225181in}{0.736220in}}{\pgfqpoint{3.225181in}{0.747270in}}%
\pgfpathcurveto{\pgfqpoint{3.225181in}{0.758321in}}{\pgfqpoint{3.220791in}{0.768920in}}{\pgfqpoint{3.212978in}{0.776733in}}%
\pgfpathcurveto{\pgfqpoint{3.205164in}{0.784547in}}{\pgfqpoint{3.194565in}{0.788937in}}{\pgfqpoint{3.183515in}{0.788937in}}%
\pgfpathcurveto{\pgfqpoint{3.172465in}{0.788937in}}{\pgfqpoint{3.161866in}{0.784547in}}{\pgfqpoint{3.154052in}{0.776733in}}%
\pgfpathcurveto{\pgfqpoint{3.146238in}{0.768920in}}{\pgfqpoint{3.141848in}{0.758321in}}{\pgfqpoint{3.141848in}{0.747270in}}%
\pgfpathcurveto{\pgfqpoint{3.141848in}{0.736220in}}{\pgfqpoint{3.146238in}{0.725621in}}{\pgfqpoint{3.154052in}{0.717808in}}%
\pgfpathcurveto{\pgfqpoint{3.161866in}{0.709994in}}{\pgfqpoint{3.172465in}{0.705604in}}{\pgfqpoint{3.183515in}{0.705604in}}%
\pgfpathclose%
\pgfusepath{stroke,fill}%
\end{pgfscope}%
\begin{pgfscope}%
\pgfpathrectangle{\pgfqpoint{0.511823in}{0.504323in}}{\pgfqpoint{3.218177in}{3.225677in}} %
\pgfusepath{clip}%
\pgfsetbuttcap%
\pgfsetroundjoin%
\definecolor{currentfill}{rgb}{0.501961,0.000000,0.000000}%
\pgfsetfillcolor{currentfill}%
\pgfsetfillopacity{0.400000}%
\pgfsetlinewidth{0.501875pt}%
\definecolor{currentstroke}{rgb}{0.501961,0.000000,0.000000}%
\pgfsetstrokecolor{currentstroke}%
\pgfsetstrokeopacity{0.400000}%
\pgfsetdash{}{0pt}%
\pgfpathmoveto{\pgfqpoint{3.350482in}{0.717816in}}%
\pgfpathcurveto{\pgfqpoint{3.361532in}{0.717816in}}{\pgfqpoint{3.372132in}{0.722207in}}{\pgfqpoint{3.379945in}{0.730020in}}%
\pgfpathcurveto{\pgfqpoint{3.387759in}{0.737834in}}{\pgfqpoint{3.392149in}{0.748433in}}{\pgfqpoint{3.392149in}{0.759483in}}%
\pgfpathcurveto{\pgfqpoint{3.392149in}{0.770533in}}{\pgfqpoint{3.387759in}{0.781132in}}{\pgfqpoint{3.379945in}{0.788946in}}%
\pgfpathcurveto{\pgfqpoint{3.372132in}{0.796759in}}{\pgfqpoint{3.361532in}{0.801150in}}{\pgfqpoint{3.350482in}{0.801150in}}%
\pgfpathcurveto{\pgfqpoint{3.339432in}{0.801150in}}{\pgfqpoint{3.328833in}{0.796759in}}{\pgfqpoint{3.321020in}{0.788946in}}%
\pgfpathcurveto{\pgfqpoint{3.313206in}{0.781132in}}{\pgfqpoint{3.308816in}{0.770533in}}{\pgfqpoint{3.308816in}{0.759483in}}%
\pgfpathcurveto{\pgfqpoint{3.308816in}{0.748433in}}{\pgfqpoint{3.313206in}{0.737834in}}{\pgfqpoint{3.321020in}{0.730020in}}%
\pgfpathcurveto{\pgfqpoint{3.328833in}{0.722207in}}{\pgfqpoint{3.339432in}{0.717816in}}{\pgfqpoint{3.350482in}{0.717816in}}%
\pgfpathclose%
\pgfusepath{stroke,fill}%
\end{pgfscope}%
\begin{pgfscope}%
\pgfpathrectangle{\pgfqpoint{0.511823in}{0.504323in}}{\pgfqpoint{3.218177in}{3.225677in}} %
\pgfusepath{clip}%
\pgfsetbuttcap%
\pgfsetroundjoin%
\definecolor{currentfill}{rgb}{0.501961,0.000000,0.000000}%
\pgfsetfillcolor{currentfill}%
\pgfsetfillopacity{0.400000}%
\pgfsetlinewidth{0.501875pt}%
\definecolor{currentstroke}{rgb}{0.501961,0.000000,0.000000}%
\pgfsetstrokecolor{currentstroke}%
\pgfsetstrokeopacity{0.400000}%
\pgfsetdash{}{0pt}%
\pgfpathmoveto{\pgfqpoint{3.488514in}{0.730077in}}%
\pgfpathcurveto{\pgfqpoint{3.499564in}{0.730077in}}{\pgfqpoint{3.510163in}{0.734467in}}{\pgfqpoint{3.517977in}{0.742281in}}%
\pgfpathcurveto{\pgfqpoint{3.525791in}{0.750094in}}{\pgfqpoint{3.530181in}{0.760693in}}{\pgfqpoint{3.530181in}{0.771743in}}%
\pgfpathcurveto{\pgfqpoint{3.530181in}{0.782793in}}{\pgfqpoint{3.525791in}{0.793393in}}{\pgfqpoint{3.517977in}{0.801206in}}%
\pgfpathcurveto{\pgfqpoint{3.510163in}{0.809020in}}{\pgfqpoint{3.499564in}{0.813410in}}{\pgfqpoint{3.488514in}{0.813410in}}%
\pgfpathcurveto{\pgfqpoint{3.477464in}{0.813410in}}{\pgfqpoint{3.466865in}{0.809020in}}{\pgfqpoint{3.459051in}{0.801206in}}%
\pgfpathcurveto{\pgfqpoint{3.451238in}{0.793393in}}{\pgfqpoint{3.446847in}{0.782793in}}{\pgfqpoint{3.446847in}{0.771743in}}%
\pgfpathcurveto{\pgfqpoint{3.446847in}{0.760693in}}{\pgfqpoint{3.451238in}{0.750094in}}{\pgfqpoint{3.459051in}{0.742281in}}%
\pgfpathcurveto{\pgfqpoint{3.466865in}{0.734467in}}{\pgfqpoint{3.477464in}{0.730077in}}{\pgfqpoint{3.488514in}{0.730077in}}%
\pgfpathclose%
\pgfusepath{stroke,fill}%
\end{pgfscope}%
\begin{pgfscope}%
\pgfpathrectangle{\pgfqpoint{0.511823in}{0.504323in}}{\pgfqpoint{3.218177in}{3.225677in}} %
\pgfusepath{clip}%
\pgfsetbuttcap%
\pgfsetroundjoin%
\definecolor{currentfill}{rgb}{0.501961,0.000000,0.000000}%
\pgfsetfillcolor{currentfill}%
\pgfsetfillopacity{0.400000}%
\pgfsetlinewidth{0.501875pt}%
\definecolor{currentstroke}{rgb}{0.501961,0.000000,0.000000}%
\pgfsetstrokecolor{currentstroke}%
\pgfsetstrokeopacity{0.400000}%
\pgfsetdash{}{0pt}%
\pgfpathmoveto{\pgfqpoint{3.497760in}{0.738620in}}%
\pgfpathcurveto{\pgfqpoint{3.508810in}{0.738620in}}{\pgfqpoint{3.519409in}{0.743010in}}{\pgfqpoint{3.527223in}{0.750824in}}%
\pgfpathcurveto{\pgfqpoint{3.535037in}{0.758637in}}{\pgfqpoint{3.539427in}{0.769236in}}{\pgfqpoint{3.539427in}{0.780286in}}%
\pgfpathcurveto{\pgfqpoint{3.539427in}{0.791337in}}{\pgfqpoint{3.535037in}{0.801936in}}{\pgfqpoint{3.527223in}{0.809749in}}%
\pgfpathcurveto{\pgfqpoint{3.519409in}{0.817563in}}{\pgfqpoint{3.508810in}{0.821953in}}{\pgfqpoint{3.497760in}{0.821953in}}%
\pgfpathcurveto{\pgfqpoint{3.486710in}{0.821953in}}{\pgfqpoint{3.476111in}{0.817563in}}{\pgfqpoint{3.468297in}{0.809749in}}%
\pgfpathcurveto{\pgfqpoint{3.460484in}{0.801936in}}{\pgfqpoint{3.456094in}{0.791337in}}{\pgfqpoint{3.456094in}{0.780286in}}%
\pgfpathcurveto{\pgfqpoint{3.456094in}{0.769236in}}{\pgfqpoint{3.460484in}{0.758637in}}{\pgfqpoint{3.468297in}{0.750824in}}%
\pgfpathcurveto{\pgfqpoint{3.476111in}{0.743010in}}{\pgfqpoint{3.486710in}{0.738620in}}{\pgfqpoint{3.497760in}{0.738620in}}%
\pgfpathclose%
\pgfusepath{stroke,fill}%
\end{pgfscope}%
\begin{pgfscope}%
\pgfpathrectangle{\pgfqpoint{0.511823in}{0.504323in}}{\pgfqpoint{3.218177in}{3.225677in}} %
\pgfusepath{clip}%
\pgfsetbuttcap%
\pgfsetroundjoin%
\definecolor{currentfill}{rgb}{0.501961,0.000000,0.000000}%
\pgfsetfillcolor{currentfill}%
\pgfsetfillopacity{0.400000}%
\pgfsetlinewidth{0.501875pt}%
\definecolor{currentstroke}{rgb}{0.501961,0.000000,0.000000}%
\pgfsetstrokecolor{currentstroke}%
\pgfsetstrokeopacity{0.400000}%
\pgfsetdash{}{0pt}%
\pgfpathmoveto{\pgfqpoint{3.138588in}{0.733183in}}%
\pgfpathcurveto{\pgfqpoint{3.149639in}{0.733183in}}{\pgfqpoint{3.160238in}{0.737573in}}{\pgfqpoint{3.168051in}{0.745387in}}%
\pgfpathcurveto{\pgfqpoint{3.175865in}{0.753201in}}{\pgfqpoint{3.180255in}{0.763800in}}{\pgfqpoint{3.180255in}{0.774850in}}%
\pgfpathcurveto{\pgfqpoint{3.180255in}{0.785900in}}{\pgfqpoint{3.175865in}{0.796499in}}{\pgfqpoint{3.168051in}{0.804313in}}%
\pgfpathcurveto{\pgfqpoint{3.160238in}{0.812126in}}{\pgfqpoint{3.149639in}{0.816517in}}{\pgfqpoint{3.138588in}{0.816517in}}%
\pgfpathcurveto{\pgfqpoint{3.127538in}{0.816517in}}{\pgfqpoint{3.116939in}{0.812126in}}{\pgfqpoint{3.109126in}{0.804313in}}%
\pgfpathcurveto{\pgfqpoint{3.101312in}{0.796499in}}{\pgfqpoint{3.096922in}{0.785900in}}{\pgfqpoint{3.096922in}{0.774850in}}%
\pgfpathcurveto{\pgfqpoint{3.096922in}{0.763800in}}{\pgfqpoint{3.101312in}{0.753201in}}{\pgfqpoint{3.109126in}{0.745387in}}%
\pgfpathcurveto{\pgfqpoint{3.116939in}{0.737573in}}{\pgfqpoint{3.127538in}{0.733183in}}{\pgfqpoint{3.138588in}{0.733183in}}%
\pgfpathclose%
\pgfusepath{stroke,fill}%
\end{pgfscope}%
\begin{pgfscope}%
\pgfpathrectangle{\pgfqpoint{0.511823in}{0.504323in}}{\pgfqpoint{3.218177in}{3.225677in}} %
\pgfusepath{clip}%
\pgfsetbuttcap%
\pgfsetroundjoin%
\definecolor{currentfill}{rgb}{0.501961,0.000000,0.000000}%
\pgfsetfillcolor{currentfill}%
\pgfsetfillopacity{0.400000}%
\pgfsetlinewidth{0.501875pt}%
\definecolor{currentstroke}{rgb}{0.501961,0.000000,0.000000}%
\pgfsetstrokecolor{currentstroke}%
\pgfsetstrokeopacity{0.400000}%
\pgfsetdash{}{0pt}%
\pgfpathmoveto{\pgfqpoint{3.396766in}{0.750972in}}%
\pgfpathcurveto{\pgfqpoint{3.407816in}{0.750972in}}{\pgfqpoint{3.418416in}{0.755362in}}{\pgfqpoint{3.426229in}{0.763176in}}%
\pgfpathcurveto{\pgfqpoint{3.434043in}{0.770990in}}{\pgfqpoint{3.438433in}{0.781589in}}{\pgfqpoint{3.438433in}{0.792639in}}%
\pgfpathcurveto{\pgfqpoint{3.438433in}{0.803689in}}{\pgfqpoint{3.434043in}{0.814288in}}{\pgfqpoint{3.426229in}{0.822102in}}%
\pgfpathcurveto{\pgfqpoint{3.418416in}{0.829915in}}{\pgfqpoint{3.407816in}{0.834305in}}{\pgfqpoint{3.396766in}{0.834305in}}%
\pgfpathcurveto{\pgfqpoint{3.385716in}{0.834305in}}{\pgfqpoint{3.375117in}{0.829915in}}{\pgfqpoint{3.367304in}{0.822102in}}%
\pgfpathcurveto{\pgfqpoint{3.359490in}{0.814288in}}{\pgfqpoint{3.355100in}{0.803689in}}{\pgfqpoint{3.355100in}{0.792639in}}%
\pgfpathcurveto{\pgfqpoint{3.355100in}{0.781589in}}{\pgfqpoint{3.359490in}{0.770990in}}{\pgfqpoint{3.367304in}{0.763176in}}%
\pgfpathcurveto{\pgfqpoint{3.375117in}{0.755362in}}{\pgfqpoint{3.385716in}{0.750972in}}{\pgfqpoint{3.396766in}{0.750972in}}%
\pgfpathclose%
\pgfusepath{stroke,fill}%
\end{pgfscope}%
\begin{pgfscope}%
\pgfpathrectangle{\pgfqpoint{0.511823in}{0.504323in}}{\pgfqpoint{3.218177in}{3.225677in}} %
\pgfusepath{clip}%
\pgfsetbuttcap%
\pgfsetroundjoin%
\definecolor{currentfill}{rgb}{0.501961,0.000000,0.000000}%
\pgfsetfillcolor{currentfill}%
\pgfsetfillopacity{0.400000}%
\pgfsetlinewidth{0.501875pt}%
\definecolor{currentstroke}{rgb}{0.501961,0.000000,0.000000}%
\pgfsetstrokecolor{currentstroke}%
\pgfsetstrokeopacity{0.400000}%
\pgfsetdash{}{0pt}%
\pgfpathmoveto{\pgfqpoint{3.249779in}{0.752464in}}%
\pgfpathcurveto{\pgfqpoint{3.260829in}{0.752464in}}{\pgfqpoint{3.271428in}{0.756855in}}{\pgfqpoint{3.279242in}{0.764668in}}%
\pgfpathcurveto{\pgfqpoint{3.287056in}{0.772482in}}{\pgfqpoint{3.291446in}{0.783081in}}{\pgfqpoint{3.291446in}{0.794131in}}%
\pgfpathcurveto{\pgfqpoint{3.291446in}{0.805181in}}{\pgfqpoint{3.287056in}{0.815780in}}{\pgfqpoint{3.279242in}{0.823594in}}%
\pgfpathcurveto{\pgfqpoint{3.271428in}{0.831408in}}{\pgfqpoint{3.260829in}{0.835798in}}{\pgfqpoint{3.249779in}{0.835798in}}%
\pgfpathcurveto{\pgfqpoint{3.238729in}{0.835798in}}{\pgfqpoint{3.228130in}{0.831408in}}{\pgfqpoint{3.220316in}{0.823594in}}%
\pgfpathcurveto{\pgfqpoint{3.212503in}{0.815780in}}{\pgfqpoint{3.208112in}{0.805181in}}{\pgfqpoint{3.208112in}{0.794131in}}%
\pgfpathcurveto{\pgfqpoint{3.208112in}{0.783081in}}{\pgfqpoint{3.212503in}{0.772482in}}{\pgfqpoint{3.220316in}{0.764668in}}%
\pgfpathcurveto{\pgfqpoint{3.228130in}{0.756855in}}{\pgfqpoint{3.238729in}{0.752464in}}{\pgfqpoint{3.249779in}{0.752464in}}%
\pgfpathclose%
\pgfusepath{stroke,fill}%
\end{pgfscope}%
\begin{pgfscope}%
\pgfpathrectangle{\pgfqpoint{0.511823in}{0.504323in}}{\pgfqpoint{3.218177in}{3.225677in}} %
\pgfusepath{clip}%
\pgfsetbuttcap%
\pgfsetroundjoin%
\definecolor{currentfill}{rgb}{0.501961,0.000000,0.000000}%
\pgfsetfillcolor{currentfill}%
\pgfsetfillopacity{0.400000}%
\pgfsetlinewidth{0.501875pt}%
\definecolor{currentstroke}{rgb}{0.501961,0.000000,0.000000}%
\pgfsetstrokecolor{currentstroke}%
\pgfsetstrokeopacity{0.400000}%
\pgfsetdash{}{0pt}%
\pgfpathmoveto{\pgfqpoint{3.373098in}{0.765774in}}%
\pgfpathcurveto{\pgfqpoint{3.384148in}{0.765774in}}{\pgfqpoint{3.394747in}{0.770165in}}{\pgfqpoint{3.402561in}{0.777978in}}%
\pgfpathcurveto{\pgfqpoint{3.410375in}{0.785792in}}{\pgfqpoint{3.414765in}{0.796391in}}{\pgfqpoint{3.414765in}{0.807441in}}%
\pgfpathcurveto{\pgfqpoint{3.414765in}{0.818491in}}{\pgfqpoint{3.410375in}{0.829090in}}{\pgfqpoint{3.402561in}{0.836904in}}%
\pgfpathcurveto{\pgfqpoint{3.394747in}{0.844717in}}{\pgfqpoint{3.384148in}{0.849108in}}{\pgfqpoint{3.373098in}{0.849108in}}%
\pgfpathcurveto{\pgfqpoint{3.362048in}{0.849108in}}{\pgfqpoint{3.351449in}{0.844717in}}{\pgfqpoint{3.343635in}{0.836904in}}%
\pgfpathcurveto{\pgfqpoint{3.335822in}{0.829090in}}{\pgfqpoint{3.331432in}{0.818491in}}{\pgfqpoint{3.331432in}{0.807441in}}%
\pgfpathcurveto{\pgfqpoint{3.331432in}{0.796391in}}{\pgfqpoint{3.335822in}{0.785792in}}{\pgfqpoint{3.343635in}{0.777978in}}%
\pgfpathcurveto{\pgfqpoint{3.351449in}{0.770165in}}{\pgfqpoint{3.362048in}{0.765774in}}{\pgfqpoint{3.373098in}{0.765774in}}%
\pgfpathclose%
\pgfusepath{stroke,fill}%
\end{pgfscope}%
\begin{pgfscope}%
\pgfpathrectangle{\pgfqpoint{0.511823in}{0.504323in}}{\pgfqpoint{3.218177in}{3.225677in}} %
\pgfusepath{clip}%
\pgfsetbuttcap%
\pgfsetroundjoin%
\definecolor{currentfill}{rgb}{0.501961,0.000000,0.000000}%
\pgfsetfillcolor{currentfill}%
\pgfsetfillopacity{0.400000}%
\pgfsetlinewidth{0.501875pt}%
\definecolor{currentstroke}{rgb}{0.501961,0.000000,0.000000}%
\pgfsetstrokecolor{currentstroke}%
\pgfsetstrokeopacity{0.400000}%
\pgfsetdash{}{0pt}%
\pgfpathmoveto{\pgfqpoint{3.036816in}{0.756904in}}%
\pgfpathcurveto{\pgfqpoint{3.047866in}{0.756904in}}{\pgfqpoint{3.058465in}{0.761295in}}{\pgfqpoint{3.066279in}{0.769108in}}%
\pgfpathcurveto{\pgfqpoint{3.074093in}{0.776922in}}{\pgfqpoint{3.078483in}{0.787521in}}{\pgfqpoint{3.078483in}{0.798571in}}%
\pgfpathcurveto{\pgfqpoint{3.078483in}{0.809621in}}{\pgfqpoint{3.074093in}{0.820220in}}{\pgfqpoint{3.066279in}{0.828034in}}%
\pgfpathcurveto{\pgfqpoint{3.058465in}{0.835848in}}{\pgfqpoint{3.047866in}{0.840238in}}{\pgfqpoint{3.036816in}{0.840238in}}%
\pgfpathcurveto{\pgfqpoint{3.025766in}{0.840238in}}{\pgfqpoint{3.015167in}{0.835848in}}{\pgfqpoint{3.007353in}{0.828034in}}%
\pgfpathcurveto{\pgfqpoint{2.999540in}{0.820220in}}{\pgfqpoint{2.995150in}{0.809621in}}{\pgfqpoint{2.995150in}{0.798571in}}%
\pgfpathcurveto{\pgfqpoint{2.995150in}{0.787521in}}{\pgfqpoint{2.999540in}{0.776922in}}{\pgfqpoint{3.007353in}{0.769108in}}%
\pgfpathcurveto{\pgfqpoint{3.015167in}{0.761295in}}{\pgfqpoint{3.025766in}{0.756904in}}{\pgfqpoint{3.036816in}{0.756904in}}%
\pgfpathclose%
\pgfusepath{stroke,fill}%
\end{pgfscope}%
\begin{pgfscope}%
\pgfpathrectangle{\pgfqpoint{0.511823in}{0.504323in}}{\pgfqpoint{3.218177in}{3.225677in}} %
\pgfusepath{clip}%
\pgfsetbuttcap%
\pgfsetroundjoin%
\definecolor{currentfill}{rgb}{0.501961,0.000000,0.000000}%
\pgfsetfillcolor{currentfill}%
\pgfsetfillopacity{0.400000}%
\pgfsetlinewidth{0.501875pt}%
\definecolor{currentstroke}{rgb}{0.501961,0.000000,0.000000}%
\pgfsetstrokecolor{currentstroke}%
\pgfsetstrokeopacity{0.400000}%
\pgfsetdash{}{0pt}%
\pgfpathmoveto{\pgfqpoint{3.457532in}{0.786014in}}%
\pgfpathcurveto{\pgfqpoint{3.468582in}{0.786014in}}{\pgfqpoint{3.479181in}{0.790404in}}{\pgfqpoint{3.486995in}{0.798218in}}%
\pgfpathcurveto{\pgfqpoint{3.494809in}{0.806032in}}{\pgfqpoint{3.499199in}{0.816631in}}{\pgfqpoint{3.499199in}{0.827681in}}%
\pgfpathcurveto{\pgfqpoint{3.499199in}{0.838731in}}{\pgfqpoint{3.494809in}{0.849330in}}{\pgfqpoint{3.486995in}{0.857144in}}%
\pgfpathcurveto{\pgfqpoint{3.479181in}{0.864957in}}{\pgfqpoint{3.468582in}{0.869348in}}{\pgfqpoint{3.457532in}{0.869348in}}%
\pgfpathcurveto{\pgfqpoint{3.446482in}{0.869348in}}{\pgfqpoint{3.435883in}{0.864957in}}{\pgfqpoint{3.428069in}{0.857144in}}%
\pgfpathcurveto{\pgfqpoint{3.420256in}{0.849330in}}{\pgfqpoint{3.415866in}{0.838731in}}{\pgfqpoint{3.415866in}{0.827681in}}%
\pgfpathcurveto{\pgfqpoint{3.415866in}{0.816631in}}{\pgfqpoint{3.420256in}{0.806032in}}{\pgfqpoint{3.428069in}{0.798218in}}%
\pgfpathcurveto{\pgfqpoint{3.435883in}{0.790404in}}{\pgfqpoint{3.446482in}{0.786014in}}{\pgfqpoint{3.457532in}{0.786014in}}%
\pgfpathclose%
\pgfusepath{stroke,fill}%
\end{pgfscope}%
\begin{pgfscope}%
\pgfpathrectangle{\pgfqpoint{0.511823in}{0.504323in}}{\pgfqpoint{3.218177in}{3.225677in}} %
\pgfusepath{clip}%
\pgfsetbuttcap%
\pgfsetroundjoin%
\definecolor{currentfill}{rgb}{0.501961,0.000000,0.000000}%
\pgfsetfillcolor{currentfill}%
\pgfsetfillopacity{0.400000}%
\pgfsetlinewidth{0.501875pt}%
\definecolor{currentstroke}{rgb}{0.501961,0.000000,0.000000}%
\pgfsetstrokecolor{currentstroke}%
\pgfsetstrokeopacity{0.400000}%
\pgfsetdash{}{0pt}%
\pgfpathmoveto{\pgfqpoint{3.282569in}{0.784410in}}%
\pgfpathcurveto{\pgfqpoint{3.293619in}{0.784410in}}{\pgfqpoint{3.304218in}{0.788800in}}{\pgfqpoint{3.312031in}{0.796614in}}%
\pgfpathcurveto{\pgfqpoint{3.319845in}{0.804427in}}{\pgfqpoint{3.324235in}{0.815026in}}{\pgfqpoint{3.324235in}{0.826076in}}%
\pgfpathcurveto{\pgfqpoint{3.324235in}{0.837127in}}{\pgfqpoint{3.319845in}{0.847726in}}{\pgfqpoint{3.312031in}{0.855539in}}%
\pgfpathcurveto{\pgfqpoint{3.304218in}{0.863353in}}{\pgfqpoint{3.293619in}{0.867743in}}{\pgfqpoint{3.282569in}{0.867743in}}%
\pgfpathcurveto{\pgfqpoint{3.271519in}{0.867743in}}{\pgfqpoint{3.260919in}{0.863353in}}{\pgfqpoint{3.253106in}{0.855539in}}%
\pgfpathcurveto{\pgfqpoint{3.245292in}{0.847726in}}{\pgfqpoint{3.240902in}{0.837127in}}{\pgfqpoint{3.240902in}{0.826076in}}%
\pgfpathcurveto{\pgfqpoint{3.240902in}{0.815026in}}{\pgfqpoint{3.245292in}{0.804427in}}{\pgfqpoint{3.253106in}{0.796614in}}%
\pgfpathcurveto{\pgfqpoint{3.260919in}{0.788800in}}{\pgfqpoint{3.271519in}{0.784410in}}{\pgfqpoint{3.282569in}{0.784410in}}%
\pgfpathclose%
\pgfusepath{stroke,fill}%
\end{pgfscope}%
\begin{pgfscope}%
\pgfpathrectangle{\pgfqpoint{0.511823in}{0.504323in}}{\pgfqpoint{3.218177in}{3.225677in}} %
\pgfusepath{clip}%
\pgfsetbuttcap%
\pgfsetroundjoin%
\definecolor{currentfill}{rgb}{0.501961,0.000000,0.000000}%
\pgfsetfillcolor{currentfill}%
\pgfsetfillopacity{0.400000}%
\pgfsetlinewidth{0.501875pt}%
\definecolor{currentstroke}{rgb}{0.501961,0.000000,0.000000}%
\pgfsetstrokecolor{currentstroke}%
\pgfsetstrokeopacity{0.400000}%
\pgfsetdash{}{0pt}%
\pgfpathmoveto{\pgfqpoint{3.361586in}{0.796677in}}%
\pgfpathcurveto{\pgfqpoint{3.372636in}{0.796677in}}{\pgfqpoint{3.383235in}{0.801068in}}{\pgfqpoint{3.391049in}{0.808881in}}%
\pgfpathcurveto{\pgfqpoint{3.398863in}{0.816695in}}{\pgfqpoint{3.403253in}{0.827294in}}{\pgfqpoint{3.403253in}{0.838344in}}%
\pgfpathcurveto{\pgfqpoint{3.403253in}{0.849394in}}{\pgfqpoint{3.398863in}{0.859993in}}{\pgfqpoint{3.391049in}{0.867807in}}%
\pgfpathcurveto{\pgfqpoint{3.383235in}{0.875621in}}{\pgfqpoint{3.372636in}{0.880011in}}{\pgfqpoint{3.361586in}{0.880011in}}%
\pgfpathcurveto{\pgfqpoint{3.350536in}{0.880011in}}{\pgfqpoint{3.339937in}{0.875621in}}{\pgfqpoint{3.332123in}{0.867807in}}%
\pgfpathcurveto{\pgfqpoint{3.324310in}{0.859993in}}{\pgfqpoint{3.319919in}{0.849394in}}{\pgfqpoint{3.319919in}{0.838344in}}%
\pgfpathcurveto{\pgfqpoint{3.319919in}{0.827294in}}{\pgfqpoint{3.324310in}{0.816695in}}{\pgfqpoint{3.332123in}{0.808881in}}%
\pgfpathcurveto{\pgfqpoint{3.339937in}{0.801068in}}{\pgfqpoint{3.350536in}{0.796677in}}{\pgfqpoint{3.361586in}{0.796677in}}%
\pgfpathclose%
\pgfusepath{stroke,fill}%
\end{pgfscope}%
\begin{pgfscope}%
\pgfpathrectangle{\pgfqpoint{0.511823in}{0.504323in}}{\pgfqpoint{3.218177in}{3.225677in}} %
\pgfusepath{clip}%
\pgfsetbuttcap%
\pgfsetroundjoin%
\definecolor{currentfill}{rgb}{0.501961,0.000000,0.000000}%
\pgfsetfillcolor{currentfill}%
\pgfsetfillopacity{0.400000}%
\pgfsetlinewidth{0.501875pt}%
\definecolor{currentstroke}{rgb}{0.501961,0.000000,0.000000}%
\pgfsetstrokecolor{currentstroke}%
\pgfsetstrokeopacity{0.400000}%
\pgfsetdash{}{0pt}%
\pgfpathmoveto{\pgfqpoint{3.310123in}{0.801376in}}%
\pgfpathcurveto{\pgfqpoint{3.321173in}{0.801376in}}{\pgfqpoint{3.331772in}{0.805766in}}{\pgfqpoint{3.339586in}{0.813579in}}%
\pgfpathcurveto{\pgfqpoint{3.347400in}{0.821393in}}{\pgfqpoint{3.351790in}{0.831992in}}{\pgfqpoint{3.351790in}{0.843042in}}%
\pgfpathcurveto{\pgfqpoint{3.351790in}{0.854092in}}{\pgfqpoint{3.347400in}{0.864691in}}{\pgfqpoint{3.339586in}{0.872505in}}%
\pgfpathcurveto{\pgfqpoint{3.331772in}{0.880319in}}{\pgfqpoint{3.321173in}{0.884709in}}{\pgfqpoint{3.310123in}{0.884709in}}%
\pgfpathcurveto{\pgfqpoint{3.299073in}{0.884709in}}{\pgfqpoint{3.288474in}{0.880319in}}{\pgfqpoint{3.280660in}{0.872505in}}%
\pgfpathcurveto{\pgfqpoint{3.272847in}{0.864691in}}{\pgfqpoint{3.268457in}{0.854092in}}{\pgfqpoint{3.268457in}{0.843042in}}%
\pgfpathcurveto{\pgfqpoint{3.268457in}{0.831992in}}{\pgfqpoint{3.272847in}{0.821393in}}{\pgfqpoint{3.280660in}{0.813579in}}%
\pgfpathcurveto{\pgfqpoint{3.288474in}{0.805766in}}{\pgfqpoint{3.299073in}{0.801376in}}{\pgfqpoint{3.310123in}{0.801376in}}%
\pgfpathclose%
\pgfusepath{stroke,fill}%
\end{pgfscope}%
\begin{pgfscope}%
\pgfpathrectangle{\pgfqpoint{0.511823in}{0.504323in}}{\pgfqpoint{3.218177in}{3.225677in}} %
\pgfusepath{clip}%
\pgfsetbuttcap%
\pgfsetroundjoin%
\definecolor{currentfill}{rgb}{0.501961,0.000000,0.000000}%
\pgfsetfillcolor{currentfill}%
\pgfsetfillopacity{0.400000}%
\pgfsetlinewidth{0.501875pt}%
\definecolor{currentstroke}{rgb}{0.501961,0.000000,0.000000}%
\pgfsetstrokecolor{currentstroke}%
\pgfsetstrokeopacity{0.400000}%
\pgfsetdash{}{0pt}%
\pgfpathmoveto{\pgfqpoint{3.196499in}{0.801762in}}%
\pgfpathcurveto{\pgfqpoint{3.207549in}{0.801762in}}{\pgfqpoint{3.218148in}{0.806153in}}{\pgfqpoint{3.225962in}{0.813966in}}%
\pgfpathcurveto{\pgfqpoint{3.233776in}{0.821780in}}{\pgfqpoint{3.238166in}{0.832379in}}{\pgfqpoint{3.238166in}{0.843429in}}%
\pgfpathcurveto{\pgfqpoint{3.238166in}{0.854479in}}{\pgfqpoint{3.233776in}{0.865078in}}{\pgfqpoint{3.225962in}{0.872892in}}%
\pgfpathcurveto{\pgfqpoint{3.218148in}{0.880705in}}{\pgfqpoint{3.207549in}{0.885096in}}{\pgfqpoint{3.196499in}{0.885096in}}%
\pgfpathcurveto{\pgfqpoint{3.185449in}{0.885096in}}{\pgfqpoint{3.174850in}{0.880705in}}{\pgfqpoint{3.167036in}{0.872892in}}%
\pgfpathcurveto{\pgfqpoint{3.159223in}{0.865078in}}{\pgfqpoint{3.154832in}{0.854479in}}{\pgfqpoint{3.154832in}{0.843429in}}%
\pgfpathcurveto{\pgfqpoint{3.154832in}{0.832379in}}{\pgfqpoint{3.159223in}{0.821780in}}{\pgfqpoint{3.167036in}{0.813966in}}%
\pgfpathcurveto{\pgfqpoint{3.174850in}{0.806153in}}{\pgfqpoint{3.185449in}{0.801762in}}{\pgfqpoint{3.196499in}{0.801762in}}%
\pgfpathclose%
\pgfusepath{stroke,fill}%
\end{pgfscope}%
\begin{pgfscope}%
\pgfpathrectangle{\pgfqpoint{0.511823in}{0.504323in}}{\pgfqpoint{3.218177in}{3.225677in}} %
\pgfusepath{clip}%
\pgfsetbuttcap%
\pgfsetroundjoin%
\definecolor{currentfill}{rgb}{0.501961,0.000000,0.000000}%
\pgfsetfillcolor{currentfill}%
\pgfsetfillopacity{0.400000}%
\pgfsetlinewidth{0.501875pt}%
\definecolor{currentstroke}{rgb}{0.501961,0.000000,0.000000}%
\pgfsetstrokecolor{currentstroke}%
\pgfsetstrokeopacity{0.400000}%
\pgfsetdash{}{0pt}%
\pgfpathmoveto{\pgfqpoint{3.402285in}{0.823038in}}%
\pgfpathcurveto{\pgfqpoint{3.413335in}{0.823038in}}{\pgfqpoint{3.423934in}{0.827429in}}{\pgfqpoint{3.431748in}{0.835242in}}%
\pgfpathcurveto{\pgfqpoint{3.439561in}{0.843056in}}{\pgfqpoint{3.443952in}{0.853655in}}{\pgfqpoint{3.443952in}{0.864705in}}%
\pgfpathcurveto{\pgfqpoint{3.443952in}{0.875755in}}{\pgfqpoint{3.439561in}{0.886354in}}{\pgfqpoint{3.431748in}{0.894168in}}%
\pgfpathcurveto{\pgfqpoint{3.423934in}{0.901982in}}{\pgfqpoint{3.413335in}{0.906372in}}{\pgfqpoint{3.402285in}{0.906372in}}%
\pgfpathcurveto{\pgfqpoint{3.391235in}{0.906372in}}{\pgfqpoint{3.380636in}{0.901982in}}{\pgfqpoint{3.372822in}{0.894168in}}%
\pgfpathcurveto{\pgfqpoint{3.365009in}{0.886354in}}{\pgfqpoint{3.360618in}{0.875755in}}{\pgfqpoint{3.360618in}{0.864705in}}%
\pgfpathcurveto{\pgfqpoint{3.360618in}{0.853655in}}{\pgfqpoint{3.365009in}{0.843056in}}{\pgfqpoint{3.372822in}{0.835242in}}%
\pgfpathcurveto{\pgfqpoint{3.380636in}{0.827429in}}{\pgfqpoint{3.391235in}{0.823038in}}{\pgfqpoint{3.402285in}{0.823038in}}%
\pgfpathclose%
\pgfusepath{stroke,fill}%
\end{pgfscope}%
\begin{pgfscope}%
\pgfpathrectangle{\pgfqpoint{0.511823in}{0.504323in}}{\pgfqpoint{3.218177in}{3.225677in}} %
\pgfusepath{clip}%
\pgfsetbuttcap%
\pgfsetroundjoin%
\definecolor{currentfill}{rgb}{0.501961,0.000000,0.000000}%
\pgfsetfillcolor{currentfill}%
\pgfsetfillopacity{0.400000}%
\pgfsetlinewidth{0.501875pt}%
\definecolor{currentstroke}{rgb}{0.501961,0.000000,0.000000}%
\pgfsetstrokecolor{currentstroke}%
\pgfsetstrokeopacity{0.400000}%
\pgfsetdash{}{0pt}%
\pgfpathmoveto{\pgfqpoint{3.268413in}{0.821608in}}%
\pgfpathcurveto{\pgfqpoint{3.279464in}{0.821608in}}{\pgfqpoint{3.290063in}{0.825999in}}{\pgfqpoint{3.297876in}{0.833812in}}%
\pgfpathcurveto{\pgfqpoint{3.305690in}{0.841626in}}{\pgfqpoint{3.310080in}{0.852225in}}{\pgfqpoint{3.310080in}{0.863275in}}%
\pgfpathcurveto{\pgfqpoint{3.310080in}{0.874325in}}{\pgfqpoint{3.305690in}{0.884924in}}{\pgfqpoint{3.297876in}{0.892738in}}%
\pgfpathcurveto{\pgfqpoint{3.290063in}{0.900551in}}{\pgfqpoint{3.279464in}{0.904942in}}{\pgfqpoint{3.268413in}{0.904942in}}%
\pgfpathcurveto{\pgfqpoint{3.257363in}{0.904942in}}{\pgfqpoint{3.246764in}{0.900551in}}{\pgfqpoint{3.238951in}{0.892738in}}%
\pgfpathcurveto{\pgfqpoint{3.231137in}{0.884924in}}{\pgfqpoint{3.226747in}{0.874325in}}{\pgfqpoint{3.226747in}{0.863275in}}%
\pgfpathcurveto{\pgfqpoint{3.226747in}{0.852225in}}{\pgfqpoint{3.231137in}{0.841626in}}{\pgfqpoint{3.238951in}{0.833812in}}%
\pgfpathcurveto{\pgfqpoint{3.246764in}{0.825999in}}{\pgfqpoint{3.257363in}{0.821608in}}{\pgfqpoint{3.268413in}{0.821608in}}%
\pgfpathclose%
\pgfusepath{stroke,fill}%
\end{pgfscope}%
\begin{pgfscope}%
\pgfpathrectangle{\pgfqpoint{0.511823in}{0.504323in}}{\pgfqpoint{3.218177in}{3.225677in}} %
\pgfusepath{clip}%
\pgfsetbuttcap%
\pgfsetroundjoin%
\definecolor{currentfill}{rgb}{0.501961,0.000000,0.000000}%
\pgfsetfillcolor{currentfill}%
\pgfsetfillopacity{0.400000}%
\pgfsetlinewidth{0.501875pt}%
\definecolor{currentstroke}{rgb}{0.501961,0.000000,0.000000}%
\pgfsetstrokecolor{currentstroke}%
\pgfsetstrokeopacity{0.400000}%
\pgfsetdash{}{0pt}%
\pgfpathmoveto{\pgfqpoint{3.240443in}{0.827164in}}%
\pgfpathcurveto{\pgfqpoint{3.251493in}{0.827164in}}{\pgfqpoint{3.262092in}{0.831554in}}{\pgfqpoint{3.269906in}{0.839368in}}%
\pgfpathcurveto{\pgfqpoint{3.277719in}{0.847181in}}{\pgfqpoint{3.282109in}{0.857780in}}{\pgfqpoint{3.282109in}{0.868830in}}%
\pgfpathcurveto{\pgfqpoint{3.282109in}{0.879880in}}{\pgfqpoint{3.277719in}{0.890479in}}{\pgfqpoint{3.269906in}{0.898293in}}%
\pgfpathcurveto{\pgfqpoint{3.262092in}{0.906107in}}{\pgfqpoint{3.251493in}{0.910497in}}{\pgfqpoint{3.240443in}{0.910497in}}%
\pgfpathcurveto{\pgfqpoint{3.229393in}{0.910497in}}{\pgfqpoint{3.218794in}{0.906107in}}{\pgfqpoint{3.210980in}{0.898293in}}%
\pgfpathcurveto{\pgfqpoint{3.203166in}{0.890479in}}{\pgfqpoint{3.198776in}{0.879880in}}{\pgfqpoint{3.198776in}{0.868830in}}%
\pgfpathcurveto{\pgfqpoint{3.198776in}{0.857780in}}{\pgfqpoint{3.203166in}{0.847181in}}{\pgfqpoint{3.210980in}{0.839368in}}%
\pgfpathcurveto{\pgfqpoint{3.218794in}{0.831554in}}{\pgfqpoint{3.229393in}{0.827164in}}{\pgfqpoint{3.240443in}{0.827164in}}%
\pgfpathclose%
\pgfusepath{stroke,fill}%
\end{pgfscope}%
\begin{pgfscope}%
\pgfpathrectangle{\pgfqpoint{0.511823in}{0.504323in}}{\pgfqpoint{3.218177in}{3.225677in}} %
\pgfusepath{clip}%
\pgfsetbuttcap%
\pgfsetroundjoin%
\definecolor{currentfill}{rgb}{0.501961,0.000000,0.000000}%
\pgfsetfillcolor{currentfill}%
\pgfsetfillopacity{0.400000}%
\pgfsetlinewidth{0.501875pt}%
\definecolor{currentstroke}{rgb}{0.501961,0.000000,0.000000}%
\pgfsetstrokecolor{currentstroke}%
\pgfsetstrokeopacity{0.400000}%
\pgfsetdash{}{0pt}%
\pgfpathmoveto{\pgfqpoint{3.417976in}{0.848243in}}%
\pgfpathcurveto{\pgfqpoint{3.429026in}{0.848243in}}{\pgfqpoint{3.439625in}{0.852633in}}{\pgfqpoint{3.447438in}{0.860447in}}%
\pgfpathcurveto{\pgfqpoint{3.455252in}{0.868260in}}{\pgfqpoint{3.459642in}{0.878859in}}{\pgfqpoint{3.459642in}{0.889909in}}%
\pgfpathcurveto{\pgfqpoint{3.459642in}{0.900959in}}{\pgfqpoint{3.455252in}{0.911558in}}{\pgfqpoint{3.447438in}{0.919372in}}%
\pgfpathcurveto{\pgfqpoint{3.439625in}{0.927186in}}{\pgfqpoint{3.429026in}{0.931576in}}{\pgfqpoint{3.417976in}{0.931576in}}%
\pgfpathcurveto{\pgfqpoint{3.406925in}{0.931576in}}{\pgfqpoint{3.396326in}{0.927186in}}{\pgfqpoint{3.388513in}{0.919372in}}%
\pgfpathcurveto{\pgfqpoint{3.380699in}{0.911558in}}{\pgfqpoint{3.376309in}{0.900959in}}{\pgfqpoint{3.376309in}{0.889909in}}%
\pgfpathcurveto{\pgfqpoint{3.376309in}{0.878859in}}{\pgfqpoint{3.380699in}{0.868260in}}{\pgfqpoint{3.388513in}{0.860447in}}%
\pgfpathcurveto{\pgfqpoint{3.396326in}{0.852633in}}{\pgfqpoint{3.406925in}{0.848243in}}{\pgfqpoint{3.417976in}{0.848243in}}%
\pgfpathclose%
\pgfusepath{stroke,fill}%
\end{pgfscope}%
\begin{pgfscope}%
\pgfpathrectangle{\pgfqpoint{0.511823in}{0.504323in}}{\pgfqpoint{3.218177in}{3.225677in}} %
\pgfusepath{clip}%
\pgfsetbuttcap%
\pgfsetroundjoin%
\definecolor{currentfill}{rgb}{0.501961,0.000000,0.000000}%
\pgfsetfillcolor{currentfill}%
\pgfsetfillopacity{0.400000}%
\pgfsetlinewidth{0.501875pt}%
\definecolor{currentstroke}{rgb}{0.501961,0.000000,0.000000}%
\pgfsetstrokecolor{currentstroke}%
\pgfsetstrokeopacity{0.400000}%
\pgfsetdash{}{0pt}%
\pgfpathmoveto{\pgfqpoint{3.319071in}{0.848459in}}%
\pgfpathcurveto{\pgfqpoint{3.330121in}{0.848459in}}{\pgfqpoint{3.340720in}{0.852849in}}{\pgfqpoint{3.348534in}{0.860663in}}%
\pgfpathcurveto{\pgfqpoint{3.356347in}{0.868476in}}{\pgfqpoint{3.360738in}{0.879075in}}{\pgfqpoint{3.360738in}{0.890125in}}%
\pgfpathcurveto{\pgfqpoint{3.360738in}{0.901176in}}{\pgfqpoint{3.356347in}{0.911775in}}{\pgfqpoint{3.348534in}{0.919588in}}%
\pgfpathcurveto{\pgfqpoint{3.340720in}{0.927402in}}{\pgfqpoint{3.330121in}{0.931792in}}{\pgfqpoint{3.319071in}{0.931792in}}%
\pgfpathcurveto{\pgfqpoint{3.308021in}{0.931792in}}{\pgfqpoint{3.297422in}{0.927402in}}{\pgfqpoint{3.289608in}{0.919588in}}%
\pgfpathcurveto{\pgfqpoint{3.281795in}{0.911775in}}{\pgfqpoint{3.277404in}{0.901176in}}{\pgfqpoint{3.277404in}{0.890125in}}%
\pgfpathcurveto{\pgfqpoint{3.277404in}{0.879075in}}{\pgfqpoint{3.281795in}{0.868476in}}{\pgfqpoint{3.289608in}{0.860663in}}%
\pgfpathcurveto{\pgfqpoint{3.297422in}{0.852849in}}{\pgfqpoint{3.308021in}{0.848459in}}{\pgfqpoint{3.319071in}{0.848459in}}%
\pgfpathclose%
\pgfusepath{stroke,fill}%
\end{pgfscope}%
\begin{pgfscope}%
\pgfpathrectangle{\pgfqpoint{0.511823in}{0.504323in}}{\pgfqpoint{3.218177in}{3.225677in}} %
\pgfusepath{clip}%
\pgfsetbuttcap%
\pgfsetroundjoin%
\definecolor{currentfill}{rgb}{0.501961,0.000000,0.000000}%
\pgfsetfillcolor{currentfill}%
\pgfsetfillopacity{0.400000}%
\pgfsetlinewidth{0.501875pt}%
\definecolor{currentstroke}{rgb}{0.501961,0.000000,0.000000}%
\pgfsetstrokecolor{currentstroke}%
\pgfsetstrokeopacity{0.400000}%
\pgfsetdash{}{0pt}%
\pgfpathmoveto{\pgfqpoint{3.236960in}{0.849476in}}%
\pgfpathcurveto{\pgfqpoint{3.248011in}{0.849476in}}{\pgfqpoint{3.258610in}{0.853866in}}{\pgfqpoint{3.266423in}{0.861680in}}%
\pgfpathcurveto{\pgfqpoint{3.274237in}{0.869494in}}{\pgfqpoint{3.278627in}{0.880093in}}{\pgfqpoint{3.278627in}{0.891143in}}%
\pgfpathcurveto{\pgfqpoint{3.278627in}{0.902193in}}{\pgfqpoint{3.274237in}{0.912792in}}{\pgfqpoint{3.266423in}{0.920606in}}%
\pgfpathcurveto{\pgfqpoint{3.258610in}{0.928419in}}{\pgfqpoint{3.248011in}{0.932810in}}{\pgfqpoint{3.236960in}{0.932810in}}%
\pgfpathcurveto{\pgfqpoint{3.225910in}{0.932810in}}{\pgfqpoint{3.215311in}{0.928419in}}{\pgfqpoint{3.207498in}{0.920606in}}%
\pgfpathcurveto{\pgfqpoint{3.199684in}{0.912792in}}{\pgfqpoint{3.195294in}{0.902193in}}{\pgfqpoint{3.195294in}{0.891143in}}%
\pgfpathcurveto{\pgfqpoint{3.195294in}{0.880093in}}{\pgfqpoint{3.199684in}{0.869494in}}{\pgfqpoint{3.207498in}{0.861680in}}%
\pgfpathcurveto{\pgfqpoint{3.215311in}{0.853866in}}{\pgfqpoint{3.225910in}{0.849476in}}{\pgfqpoint{3.236960in}{0.849476in}}%
\pgfpathclose%
\pgfusepath{stroke,fill}%
\end{pgfscope}%
\begin{pgfscope}%
\pgfpathrectangle{\pgfqpoint{0.511823in}{0.504323in}}{\pgfqpoint{3.218177in}{3.225677in}} %
\pgfusepath{clip}%
\pgfsetbuttcap%
\pgfsetroundjoin%
\definecolor{currentfill}{rgb}{0.501961,0.000000,0.000000}%
\pgfsetfillcolor{currentfill}%
\pgfsetfillopacity{0.400000}%
\pgfsetlinewidth{0.501875pt}%
\definecolor{currentstroke}{rgb}{0.501961,0.000000,0.000000}%
\pgfsetstrokecolor{currentstroke}%
\pgfsetstrokeopacity{0.400000}%
\pgfsetdash{}{0pt}%
\pgfpathmoveto{\pgfqpoint{3.452741in}{0.875387in}}%
\pgfpathcurveto{\pgfqpoint{3.463791in}{0.875387in}}{\pgfqpoint{3.474391in}{0.879777in}}{\pgfqpoint{3.482204in}{0.887591in}}%
\pgfpathcurveto{\pgfqpoint{3.490018in}{0.895404in}}{\pgfqpoint{3.494408in}{0.906003in}}{\pgfqpoint{3.494408in}{0.917053in}}%
\pgfpathcurveto{\pgfqpoint{3.494408in}{0.928103in}}{\pgfqpoint{3.490018in}{0.938702in}}{\pgfqpoint{3.482204in}{0.946516in}}%
\pgfpathcurveto{\pgfqpoint{3.474391in}{0.954330in}}{\pgfqpoint{3.463791in}{0.958720in}}{\pgfqpoint{3.452741in}{0.958720in}}%
\pgfpathcurveto{\pgfqpoint{3.441691in}{0.958720in}}{\pgfqpoint{3.431092in}{0.954330in}}{\pgfqpoint{3.423279in}{0.946516in}}%
\pgfpathcurveto{\pgfqpoint{3.415465in}{0.938702in}}{\pgfqpoint{3.411075in}{0.928103in}}{\pgfqpoint{3.411075in}{0.917053in}}%
\pgfpathcurveto{\pgfqpoint{3.411075in}{0.906003in}}{\pgfqpoint{3.415465in}{0.895404in}}{\pgfqpoint{3.423279in}{0.887591in}}%
\pgfpathcurveto{\pgfqpoint{3.431092in}{0.879777in}}{\pgfqpoint{3.441691in}{0.875387in}}{\pgfqpoint{3.452741in}{0.875387in}}%
\pgfpathclose%
\pgfusepath{stroke,fill}%
\end{pgfscope}%
\begin{pgfscope}%
\pgfpathrectangle{\pgfqpoint{0.511823in}{0.504323in}}{\pgfqpoint{3.218177in}{3.225677in}} %
\pgfusepath{clip}%
\pgfsetbuttcap%
\pgfsetroundjoin%
\definecolor{currentfill}{rgb}{0.501961,0.000000,0.000000}%
\pgfsetfillcolor{currentfill}%
\pgfsetfillopacity{0.400000}%
\pgfsetlinewidth{0.501875pt}%
\definecolor{currentstroke}{rgb}{0.501961,0.000000,0.000000}%
\pgfsetstrokecolor{currentstroke}%
\pgfsetstrokeopacity{0.400000}%
\pgfsetdash{}{0pt}%
\pgfpathmoveto{\pgfqpoint{3.249899in}{0.865683in}}%
\pgfpathcurveto{\pgfqpoint{3.260950in}{0.865683in}}{\pgfqpoint{3.271549in}{0.870073in}}{\pgfqpoint{3.279362in}{0.877887in}}%
\pgfpathcurveto{\pgfqpoint{3.287176in}{0.885700in}}{\pgfqpoint{3.291566in}{0.896299in}}{\pgfqpoint{3.291566in}{0.907349in}}%
\pgfpathcurveto{\pgfqpoint{3.291566in}{0.918399in}}{\pgfqpoint{3.287176in}{0.928998in}}{\pgfqpoint{3.279362in}{0.936812in}}%
\pgfpathcurveto{\pgfqpoint{3.271549in}{0.944626in}}{\pgfqpoint{3.260950in}{0.949016in}}{\pgfqpoint{3.249899in}{0.949016in}}%
\pgfpathcurveto{\pgfqpoint{3.238849in}{0.949016in}}{\pgfqpoint{3.228250in}{0.944626in}}{\pgfqpoint{3.220437in}{0.936812in}}%
\pgfpathcurveto{\pgfqpoint{3.212623in}{0.928998in}}{\pgfqpoint{3.208233in}{0.918399in}}{\pgfqpoint{3.208233in}{0.907349in}}%
\pgfpathcurveto{\pgfqpoint{3.208233in}{0.896299in}}{\pgfqpoint{3.212623in}{0.885700in}}{\pgfqpoint{3.220437in}{0.877887in}}%
\pgfpathcurveto{\pgfqpoint{3.228250in}{0.870073in}}{\pgfqpoint{3.238849in}{0.865683in}}{\pgfqpoint{3.249899in}{0.865683in}}%
\pgfpathclose%
\pgfusepath{stroke,fill}%
\end{pgfscope}%
\begin{pgfscope}%
\pgfpathrectangle{\pgfqpoint{0.511823in}{0.504323in}}{\pgfqpoint{3.218177in}{3.225677in}} %
\pgfusepath{clip}%
\pgfsetbuttcap%
\pgfsetroundjoin%
\definecolor{currentfill}{rgb}{0.501961,0.000000,0.000000}%
\pgfsetfillcolor{currentfill}%
\pgfsetfillopacity{0.400000}%
\pgfsetlinewidth{0.501875pt}%
\definecolor{currentstroke}{rgb}{0.501961,0.000000,0.000000}%
\pgfsetstrokecolor{currentstroke}%
\pgfsetstrokeopacity{0.400000}%
\pgfsetdash{}{0pt}%
\pgfpathmoveto{\pgfqpoint{3.322937in}{0.879913in}}%
\pgfpathcurveto{\pgfqpoint{3.333987in}{0.879913in}}{\pgfqpoint{3.344586in}{0.884304in}}{\pgfqpoint{3.352400in}{0.892117in}}%
\pgfpathcurveto{\pgfqpoint{3.360213in}{0.899931in}}{\pgfqpoint{3.364604in}{0.910530in}}{\pgfqpoint{3.364604in}{0.921580in}}%
\pgfpathcurveto{\pgfqpoint{3.364604in}{0.932630in}}{\pgfqpoint{3.360213in}{0.943229in}}{\pgfqpoint{3.352400in}{0.951043in}}%
\pgfpathcurveto{\pgfqpoint{3.344586in}{0.958856in}}{\pgfqpoint{3.333987in}{0.963247in}}{\pgfqpoint{3.322937in}{0.963247in}}%
\pgfpathcurveto{\pgfqpoint{3.311887in}{0.963247in}}{\pgfqpoint{3.301288in}{0.958856in}}{\pgfqpoint{3.293474in}{0.951043in}}%
\pgfpathcurveto{\pgfqpoint{3.285661in}{0.943229in}}{\pgfqpoint{3.281270in}{0.932630in}}{\pgfqpoint{3.281270in}{0.921580in}}%
\pgfpathcurveto{\pgfqpoint{3.281270in}{0.910530in}}{\pgfqpoint{3.285661in}{0.899931in}}{\pgfqpoint{3.293474in}{0.892117in}}%
\pgfpathcurveto{\pgfqpoint{3.301288in}{0.884304in}}{\pgfqpoint{3.311887in}{0.879913in}}{\pgfqpoint{3.322937in}{0.879913in}}%
\pgfpathclose%
\pgfusepath{stroke,fill}%
\end{pgfscope}%
\begin{pgfscope}%
\pgfpathrectangle{\pgfqpoint{0.511823in}{0.504323in}}{\pgfqpoint{3.218177in}{3.225677in}} %
\pgfusepath{clip}%
\pgfsetbuttcap%
\pgfsetroundjoin%
\definecolor{currentfill}{rgb}{0.501961,0.000000,0.000000}%
\pgfsetfillcolor{currentfill}%
\pgfsetfillopacity{0.400000}%
\pgfsetlinewidth{0.501875pt}%
\definecolor{currentstroke}{rgb}{0.501961,0.000000,0.000000}%
\pgfsetstrokecolor{currentstroke}%
\pgfsetstrokeopacity{0.400000}%
\pgfsetdash{}{0pt}%
\pgfpathmoveto{\pgfqpoint{3.423160in}{0.897138in}}%
\pgfpathcurveto{\pgfqpoint{3.434210in}{0.897138in}}{\pgfqpoint{3.444809in}{0.901528in}}{\pgfqpoint{3.452623in}{0.909341in}}%
\pgfpathcurveto{\pgfqpoint{3.460436in}{0.917155in}}{\pgfqpoint{3.464827in}{0.927754in}}{\pgfqpoint{3.464827in}{0.938804in}}%
\pgfpathcurveto{\pgfqpoint{3.464827in}{0.949854in}}{\pgfqpoint{3.460436in}{0.960453in}}{\pgfqpoint{3.452623in}{0.968267in}}%
\pgfpathcurveto{\pgfqpoint{3.444809in}{0.976081in}}{\pgfqpoint{3.434210in}{0.980471in}}{\pgfqpoint{3.423160in}{0.980471in}}%
\pgfpathcurveto{\pgfqpoint{3.412110in}{0.980471in}}{\pgfqpoint{3.401511in}{0.976081in}}{\pgfqpoint{3.393697in}{0.968267in}}%
\pgfpathcurveto{\pgfqpoint{3.385883in}{0.960453in}}{\pgfqpoint{3.381493in}{0.949854in}}{\pgfqpoint{3.381493in}{0.938804in}}%
\pgfpathcurveto{\pgfqpoint{3.381493in}{0.927754in}}{\pgfqpoint{3.385883in}{0.917155in}}{\pgfqpoint{3.393697in}{0.909341in}}%
\pgfpathcurveto{\pgfqpoint{3.401511in}{0.901528in}}{\pgfqpoint{3.412110in}{0.897138in}}{\pgfqpoint{3.423160in}{0.897138in}}%
\pgfpathclose%
\pgfusepath{stroke,fill}%
\end{pgfscope}%
\begin{pgfscope}%
\pgfpathrectangle{\pgfqpoint{0.511823in}{0.504323in}}{\pgfqpoint{3.218177in}{3.225677in}} %
\pgfusepath{clip}%
\pgfsetbuttcap%
\pgfsetroundjoin%
\definecolor{currentfill}{rgb}{0.501961,0.000000,0.000000}%
\pgfsetfillcolor{currentfill}%
\pgfsetfillopacity{0.400000}%
\pgfsetlinewidth{0.501875pt}%
\definecolor{currentstroke}{rgb}{0.501961,0.000000,0.000000}%
\pgfsetstrokecolor{currentstroke}%
\pgfsetstrokeopacity{0.400000}%
\pgfsetdash{}{0pt}%
\pgfpathmoveto{\pgfqpoint{3.449917in}{0.907833in}}%
\pgfpathcurveto{\pgfqpoint{3.460967in}{0.907833in}}{\pgfqpoint{3.471566in}{0.912224in}}{\pgfqpoint{3.479380in}{0.920037in}}%
\pgfpathcurveto{\pgfqpoint{3.487193in}{0.927851in}}{\pgfqpoint{3.491584in}{0.938450in}}{\pgfqpoint{3.491584in}{0.949500in}}%
\pgfpathcurveto{\pgfqpoint{3.491584in}{0.960550in}}{\pgfqpoint{3.487193in}{0.971149in}}{\pgfqpoint{3.479380in}{0.978963in}}%
\pgfpathcurveto{\pgfqpoint{3.471566in}{0.986777in}}{\pgfqpoint{3.460967in}{0.991167in}}{\pgfqpoint{3.449917in}{0.991167in}}%
\pgfpathcurveto{\pgfqpoint{3.438867in}{0.991167in}}{\pgfqpoint{3.428268in}{0.986777in}}{\pgfqpoint{3.420454in}{0.978963in}}%
\pgfpathcurveto{\pgfqpoint{3.412641in}{0.971149in}}{\pgfqpoint{3.408250in}{0.960550in}}{\pgfqpoint{3.408250in}{0.949500in}}%
\pgfpathcurveto{\pgfqpoint{3.408250in}{0.938450in}}{\pgfqpoint{3.412641in}{0.927851in}}{\pgfqpoint{3.420454in}{0.920037in}}%
\pgfpathcurveto{\pgfqpoint{3.428268in}{0.912224in}}{\pgfqpoint{3.438867in}{0.907833in}}{\pgfqpoint{3.449917in}{0.907833in}}%
\pgfpathclose%
\pgfusepath{stroke,fill}%
\end{pgfscope}%
\begin{pgfscope}%
\pgfpathrectangle{\pgfqpoint{0.511823in}{0.504323in}}{\pgfqpoint{3.218177in}{3.225677in}} %
\pgfusepath{clip}%
\pgfsetbuttcap%
\pgfsetroundjoin%
\definecolor{currentfill}{rgb}{0.501961,0.000000,0.000000}%
\pgfsetfillcolor{currentfill}%
\pgfsetfillopacity{0.400000}%
\pgfsetlinewidth{0.501875pt}%
\definecolor{currentstroke}{rgb}{0.501961,0.000000,0.000000}%
\pgfsetstrokecolor{currentstroke}%
\pgfsetstrokeopacity{0.400000}%
\pgfsetdash{}{0pt}%
\pgfpathmoveto{\pgfqpoint{3.482823in}{0.919308in}}%
\pgfpathcurveto{\pgfqpoint{3.493873in}{0.919308in}}{\pgfqpoint{3.504472in}{0.923698in}}{\pgfqpoint{3.512286in}{0.931512in}}%
\pgfpathcurveto{\pgfqpoint{3.520100in}{0.939325in}}{\pgfqpoint{3.524490in}{0.949924in}}{\pgfqpoint{3.524490in}{0.960974in}}%
\pgfpathcurveto{\pgfqpoint{3.524490in}{0.972025in}}{\pgfqpoint{3.520100in}{0.982624in}}{\pgfqpoint{3.512286in}{0.990437in}}%
\pgfpathcurveto{\pgfqpoint{3.504472in}{0.998251in}}{\pgfqpoint{3.493873in}{1.002641in}}{\pgfqpoint{3.482823in}{1.002641in}}%
\pgfpathcurveto{\pgfqpoint{3.471773in}{1.002641in}}{\pgfqpoint{3.461174in}{0.998251in}}{\pgfqpoint{3.453360in}{0.990437in}}%
\pgfpathcurveto{\pgfqpoint{3.445547in}{0.982624in}}{\pgfqpoint{3.441156in}{0.972025in}}{\pgfqpoint{3.441156in}{0.960974in}}%
\pgfpathcurveto{\pgfqpoint{3.441156in}{0.949924in}}{\pgfqpoint{3.445547in}{0.939325in}}{\pgfqpoint{3.453360in}{0.931512in}}%
\pgfpathcurveto{\pgfqpoint{3.461174in}{0.923698in}}{\pgfqpoint{3.471773in}{0.919308in}}{\pgfqpoint{3.482823in}{0.919308in}}%
\pgfpathclose%
\pgfusepath{stroke,fill}%
\end{pgfscope}%
\begin{pgfscope}%
\pgfpathrectangle{\pgfqpoint{0.511823in}{0.504323in}}{\pgfqpoint{3.218177in}{3.225677in}} %
\pgfusepath{clip}%
\pgfsetbuttcap%
\pgfsetroundjoin%
\definecolor{currentfill}{rgb}{0.501961,0.000000,0.000000}%
\pgfsetfillcolor{currentfill}%
\pgfsetfillopacity{0.400000}%
\pgfsetlinewidth{0.501875pt}%
\definecolor{currentstroke}{rgb}{0.501961,0.000000,0.000000}%
\pgfsetstrokecolor{currentstroke}%
\pgfsetstrokeopacity{0.400000}%
\pgfsetdash{}{0pt}%
\pgfpathmoveto{\pgfqpoint{3.542068in}{0.933694in}}%
\pgfpathcurveto{\pgfqpoint{3.553118in}{0.933694in}}{\pgfqpoint{3.563717in}{0.938084in}}{\pgfqpoint{3.571531in}{0.945898in}}%
\pgfpathcurveto{\pgfqpoint{3.579344in}{0.953711in}}{\pgfqpoint{3.583735in}{0.964310in}}{\pgfqpoint{3.583735in}{0.975361in}}%
\pgfpathcurveto{\pgfqpoint{3.583735in}{0.986411in}}{\pgfqpoint{3.579344in}{0.997010in}}{\pgfqpoint{3.571531in}{1.004823in}}%
\pgfpathcurveto{\pgfqpoint{3.563717in}{1.012637in}}{\pgfqpoint{3.553118in}{1.017027in}}{\pgfqpoint{3.542068in}{1.017027in}}%
\pgfpathcurveto{\pgfqpoint{3.531018in}{1.017027in}}{\pgfqpoint{3.520419in}{1.012637in}}{\pgfqpoint{3.512605in}{1.004823in}}%
\pgfpathcurveto{\pgfqpoint{3.504792in}{0.997010in}}{\pgfqpoint{3.500401in}{0.986411in}}{\pgfqpoint{3.500401in}{0.975361in}}%
\pgfpathcurveto{\pgfqpoint{3.500401in}{0.964310in}}{\pgfqpoint{3.504792in}{0.953711in}}{\pgfqpoint{3.512605in}{0.945898in}}%
\pgfpathcurveto{\pgfqpoint{3.520419in}{0.938084in}}{\pgfqpoint{3.531018in}{0.933694in}}{\pgfqpoint{3.542068in}{0.933694in}}%
\pgfpathclose%
\pgfusepath{stroke,fill}%
\end{pgfscope}%
\begin{pgfscope}%
\pgfpathrectangle{\pgfqpoint{0.511823in}{0.504323in}}{\pgfqpoint{3.218177in}{3.225677in}} %
\pgfusepath{clip}%
\pgfsetbuttcap%
\pgfsetroundjoin%
\definecolor{currentfill}{rgb}{0.501961,0.000000,0.000000}%
\pgfsetfillcolor{currentfill}%
\pgfsetfillopacity{0.400000}%
\pgfsetlinewidth{0.501875pt}%
\definecolor{currentstroke}{rgb}{0.501961,0.000000,0.000000}%
\pgfsetstrokecolor{currentstroke}%
\pgfsetstrokeopacity{0.400000}%
\pgfsetdash{}{0pt}%
\pgfpathmoveto{\pgfqpoint{3.328610in}{0.919559in}}%
\pgfpathcurveto{\pgfqpoint{3.339660in}{0.919559in}}{\pgfqpoint{3.350259in}{0.923949in}}{\pgfqpoint{3.358073in}{0.931763in}}%
\pgfpathcurveto{\pgfqpoint{3.365886in}{0.939576in}}{\pgfqpoint{3.370277in}{0.950175in}}{\pgfqpoint{3.370277in}{0.961225in}}%
\pgfpathcurveto{\pgfqpoint{3.370277in}{0.972276in}}{\pgfqpoint{3.365886in}{0.982875in}}{\pgfqpoint{3.358073in}{0.990688in}}%
\pgfpathcurveto{\pgfqpoint{3.350259in}{0.998502in}}{\pgfqpoint{3.339660in}{1.002892in}}{\pgfqpoint{3.328610in}{1.002892in}}%
\pgfpathcurveto{\pgfqpoint{3.317560in}{1.002892in}}{\pgfqpoint{3.306961in}{0.998502in}}{\pgfqpoint{3.299147in}{0.990688in}}%
\pgfpathcurveto{\pgfqpoint{3.291334in}{0.982875in}}{\pgfqpoint{3.286943in}{0.972276in}}{\pgfqpoint{3.286943in}{0.961225in}}%
\pgfpathcurveto{\pgfqpoint{3.286943in}{0.950175in}}{\pgfqpoint{3.291334in}{0.939576in}}{\pgfqpoint{3.299147in}{0.931763in}}%
\pgfpathcurveto{\pgfqpoint{3.306961in}{0.923949in}}{\pgfqpoint{3.317560in}{0.919559in}}{\pgfqpoint{3.328610in}{0.919559in}}%
\pgfpathclose%
\pgfusepath{stroke,fill}%
\end{pgfscope}%
\begin{pgfscope}%
\pgfpathrectangle{\pgfqpoint{0.511823in}{0.504323in}}{\pgfqpoint{3.218177in}{3.225677in}} %
\pgfusepath{clip}%
\pgfsetbuttcap%
\pgfsetroundjoin%
\definecolor{currentfill}{rgb}{0.501961,0.000000,0.000000}%
\pgfsetfillcolor{currentfill}%
\pgfsetfillopacity{0.400000}%
\pgfsetlinewidth{0.501875pt}%
\definecolor{currentstroke}{rgb}{0.501961,0.000000,0.000000}%
\pgfsetstrokecolor{currentstroke}%
\pgfsetstrokeopacity{0.400000}%
\pgfsetdash{}{0pt}%
\pgfpathmoveto{\pgfqpoint{3.216552in}{0.915202in}}%
\pgfpathcurveto{\pgfqpoint{3.227602in}{0.915202in}}{\pgfqpoint{3.238201in}{0.919592in}}{\pgfqpoint{3.246015in}{0.927405in}}%
\pgfpathcurveto{\pgfqpoint{3.253828in}{0.935219in}}{\pgfqpoint{3.258219in}{0.945818in}}{\pgfqpoint{3.258219in}{0.956868in}}%
\pgfpathcurveto{\pgfqpoint{3.258219in}{0.967918in}}{\pgfqpoint{3.253828in}{0.978517in}}{\pgfqpoint{3.246015in}{0.986331in}}%
\pgfpathcurveto{\pgfqpoint{3.238201in}{0.994145in}}{\pgfqpoint{3.227602in}{0.998535in}}{\pgfqpoint{3.216552in}{0.998535in}}%
\pgfpathcurveto{\pgfqpoint{3.205502in}{0.998535in}}{\pgfqpoint{3.194903in}{0.994145in}}{\pgfqpoint{3.187089in}{0.986331in}}%
\pgfpathcurveto{\pgfqpoint{3.179276in}{0.978517in}}{\pgfqpoint{3.174885in}{0.967918in}}{\pgfqpoint{3.174885in}{0.956868in}}%
\pgfpathcurveto{\pgfqpoint{3.174885in}{0.945818in}}{\pgfqpoint{3.179276in}{0.935219in}}{\pgfqpoint{3.187089in}{0.927405in}}%
\pgfpathcurveto{\pgfqpoint{3.194903in}{0.919592in}}{\pgfqpoint{3.205502in}{0.915202in}}{\pgfqpoint{3.216552in}{0.915202in}}%
\pgfpathclose%
\pgfusepath{stroke,fill}%
\end{pgfscope}%
\begin{pgfscope}%
\pgfpathrectangle{\pgfqpoint{0.511823in}{0.504323in}}{\pgfqpoint{3.218177in}{3.225677in}} %
\pgfusepath{clip}%
\pgfsetbuttcap%
\pgfsetroundjoin%
\definecolor{currentfill}{rgb}{0.501961,0.000000,0.000000}%
\pgfsetfillcolor{currentfill}%
\pgfsetfillopacity{0.400000}%
\pgfsetlinewidth{0.501875pt}%
\definecolor{currentstroke}{rgb}{0.501961,0.000000,0.000000}%
\pgfsetstrokecolor{currentstroke}%
\pgfsetstrokeopacity{0.400000}%
\pgfsetdash{}{0pt}%
\pgfpathmoveto{\pgfqpoint{3.537810in}{0.958644in}}%
\pgfpathcurveto{\pgfqpoint{3.548860in}{0.958644in}}{\pgfqpoint{3.559459in}{0.963035in}}{\pgfqpoint{3.567273in}{0.970848in}}%
\pgfpathcurveto{\pgfqpoint{3.575087in}{0.978662in}}{\pgfqpoint{3.579477in}{0.989261in}}{\pgfqpoint{3.579477in}{1.000311in}}%
\pgfpathcurveto{\pgfqpoint{3.579477in}{1.011361in}}{\pgfqpoint{3.575087in}{1.021960in}}{\pgfqpoint{3.567273in}{1.029774in}}%
\pgfpathcurveto{\pgfqpoint{3.559459in}{1.037587in}}{\pgfqpoint{3.548860in}{1.041978in}}{\pgfqpoint{3.537810in}{1.041978in}}%
\pgfpathcurveto{\pgfqpoint{3.526760in}{1.041978in}}{\pgfqpoint{3.516161in}{1.037587in}}{\pgfqpoint{3.508347in}{1.029774in}}%
\pgfpathcurveto{\pgfqpoint{3.500534in}{1.021960in}}{\pgfqpoint{3.496144in}{1.011361in}}{\pgfqpoint{3.496144in}{1.000311in}}%
\pgfpathcurveto{\pgfqpoint{3.496144in}{0.989261in}}{\pgfqpoint{3.500534in}{0.978662in}}{\pgfqpoint{3.508347in}{0.970848in}}%
\pgfpathcurveto{\pgfqpoint{3.516161in}{0.963035in}}{\pgfqpoint{3.526760in}{0.958644in}}{\pgfqpoint{3.537810in}{0.958644in}}%
\pgfpathclose%
\pgfusepath{stroke,fill}%
\end{pgfscope}%
\begin{pgfscope}%
\pgfpathrectangle{\pgfqpoint{0.511823in}{0.504323in}}{\pgfqpoint{3.218177in}{3.225677in}} %
\pgfusepath{clip}%
\pgfsetbuttcap%
\pgfsetroundjoin%
\definecolor{currentfill}{rgb}{0.501961,0.000000,0.000000}%
\pgfsetfillcolor{currentfill}%
\pgfsetfillopacity{0.400000}%
\pgfsetlinewidth{0.501875pt}%
\definecolor{currentstroke}{rgb}{0.501961,0.000000,0.000000}%
\pgfsetstrokecolor{currentstroke}%
\pgfsetstrokeopacity{0.400000}%
\pgfsetdash{}{0pt}%
\pgfpathmoveto{\pgfqpoint{3.502955in}{0.963118in}}%
\pgfpathcurveto{\pgfqpoint{3.514005in}{0.963118in}}{\pgfqpoint{3.524604in}{0.967508in}}{\pgfqpoint{3.532418in}{0.975322in}}%
\pgfpathcurveto{\pgfqpoint{3.540231in}{0.983136in}}{\pgfqpoint{3.544622in}{0.993735in}}{\pgfqpoint{3.544622in}{1.004785in}}%
\pgfpathcurveto{\pgfqpoint{3.544622in}{1.015835in}}{\pgfqpoint{3.540231in}{1.026434in}}{\pgfqpoint{3.532418in}{1.034248in}}%
\pgfpathcurveto{\pgfqpoint{3.524604in}{1.042061in}}{\pgfqpoint{3.514005in}{1.046451in}}{\pgfqpoint{3.502955in}{1.046451in}}%
\pgfpathcurveto{\pgfqpoint{3.491905in}{1.046451in}}{\pgfqpoint{3.481306in}{1.042061in}}{\pgfqpoint{3.473492in}{1.034248in}}%
\pgfpathcurveto{\pgfqpoint{3.465679in}{1.026434in}}{\pgfqpoint{3.461288in}{1.015835in}}{\pgfqpoint{3.461288in}{1.004785in}}%
\pgfpathcurveto{\pgfqpoint{3.461288in}{0.993735in}}{\pgfqpoint{3.465679in}{0.983136in}}{\pgfqpoint{3.473492in}{0.975322in}}%
\pgfpathcurveto{\pgfqpoint{3.481306in}{0.967508in}}{\pgfqpoint{3.491905in}{0.963118in}}{\pgfqpoint{3.502955in}{0.963118in}}%
\pgfpathclose%
\pgfusepath{stroke,fill}%
\end{pgfscope}%
\begin{pgfscope}%
\pgfpathrectangle{\pgfqpoint{0.511823in}{0.504323in}}{\pgfqpoint{3.218177in}{3.225677in}} %
\pgfusepath{clip}%
\pgfsetbuttcap%
\pgfsetroundjoin%
\definecolor{currentfill}{rgb}{0.501961,0.000000,0.000000}%
\pgfsetfillcolor{currentfill}%
\pgfsetfillopacity{0.400000}%
\pgfsetlinewidth{0.501875pt}%
\definecolor{currentstroke}{rgb}{0.501961,0.000000,0.000000}%
\pgfsetstrokecolor{currentstroke}%
\pgfsetstrokeopacity{0.400000}%
\pgfsetdash{}{0pt}%
\pgfpathmoveto{\pgfqpoint{3.279294in}{0.945153in}}%
\pgfpathcurveto{\pgfqpoint{3.290344in}{0.945153in}}{\pgfqpoint{3.300943in}{0.949544in}}{\pgfqpoint{3.308757in}{0.957357in}}%
\pgfpathcurveto{\pgfqpoint{3.316570in}{0.965171in}}{\pgfqpoint{3.320961in}{0.975770in}}{\pgfqpoint{3.320961in}{0.986820in}}%
\pgfpathcurveto{\pgfqpoint{3.320961in}{0.997870in}}{\pgfqpoint{3.316570in}{1.008469in}}{\pgfqpoint{3.308757in}{1.016283in}}%
\pgfpathcurveto{\pgfqpoint{3.300943in}{1.024096in}}{\pgfqpoint{3.290344in}{1.028487in}}{\pgfqpoint{3.279294in}{1.028487in}}%
\pgfpathcurveto{\pgfqpoint{3.268244in}{1.028487in}}{\pgfqpoint{3.257645in}{1.024096in}}{\pgfqpoint{3.249831in}{1.016283in}}%
\pgfpathcurveto{\pgfqpoint{3.242018in}{1.008469in}}{\pgfqpoint{3.237627in}{0.997870in}}{\pgfqpoint{3.237627in}{0.986820in}}%
\pgfpathcurveto{\pgfqpoint{3.237627in}{0.975770in}}{\pgfqpoint{3.242018in}{0.965171in}}{\pgfqpoint{3.249831in}{0.957357in}}%
\pgfpathcurveto{\pgfqpoint{3.257645in}{0.949544in}}{\pgfqpoint{3.268244in}{0.945153in}}{\pgfqpoint{3.279294in}{0.945153in}}%
\pgfpathclose%
\pgfusepath{stroke,fill}%
\end{pgfscope}%
\begin{pgfscope}%
\pgfpathrectangle{\pgfqpoint{0.511823in}{0.504323in}}{\pgfqpoint{3.218177in}{3.225677in}} %
\pgfusepath{clip}%
\pgfsetbuttcap%
\pgfsetroundjoin%
\definecolor{currentfill}{rgb}{0.501961,0.000000,0.000000}%
\pgfsetfillcolor{currentfill}%
\pgfsetfillopacity{0.400000}%
\pgfsetlinewidth{0.501875pt}%
\definecolor{currentstroke}{rgb}{0.501961,0.000000,0.000000}%
\pgfsetstrokecolor{currentstroke}%
\pgfsetstrokeopacity{0.400000}%
\pgfsetdash{}{0pt}%
\pgfpathmoveto{\pgfqpoint{3.409882in}{0.968642in}}%
\pgfpathcurveto{\pgfqpoint{3.420932in}{0.968642in}}{\pgfqpoint{3.431531in}{0.973032in}}{\pgfqpoint{3.439345in}{0.980846in}}%
\pgfpathcurveto{\pgfqpoint{3.447159in}{0.988660in}}{\pgfqpoint{3.451549in}{0.999259in}}{\pgfqpoint{3.451549in}{1.010309in}}%
\pgfpathcurveto{\pgfqpoint{3.451549in}{1.021359in}}{\pgfqpoint{3.447159in}{1.031958in}}{\pgfqpoint{3.439345in}{1.039772in}}%
\pgfpathcurveto{\pgfqpoint{3.431531in}{1.047585in}}{\pgfqpoint{3.420932in}{1.051975in}}{\pgfqpoint{3.409882in}{1.051975in}}%
\pgfpathcurveto{\pgfqpoint{3.398832in}{1.051975in}}{\pgfqpoint{3.388233in}{1.047585in}}{\pgfqpoint{3.380419in}{1.039772in}}%
\pgfpathcurveto{\pgfqpoint{3.372606in}{1.031958in}}{\pgfqpoint{3.368215in}{1.021359in}}{\pgfqpoint{3.368215in}{1.010309in}}%
\pgfpathcurveto{\pgfqpoint{3.368215in}{0.999259in}}{\pgfqpoint{3.372606in}{0.988660in}}{\pgfqpoint{3.380419in}{0.980846in}}%
\pgfpathcurveto{\pgfqpoint{3.388233in}{0.973032in}}{\pgfqpoint{3.398832in}{0.968642in}}{\pgfqpoint{3.409882in}{0.968642in}}%
\pgfpathclose%
\pgfusepath{stroke,fill}%
\end{pgfscope}%
\begin{pgfscope}%
\pgfpathrectangle{\pgfqpoint{0.511823in}{0.504323in}}{\pgfqpoint{3.218177in}{3.225677in}} %
\pgfusepath{clip}%
\pgfsetbuttcap%
\pgfsetroundjoin%
\definecolor{currentfill}{rgb}{0.501961,0.000000,0.000000}%
\pgfsetfillcolor{currentfill}%
\pgfsetfillopacity{0.400000}%
\pgfsetlinewidth{0.501875pt}%
\definecolor{currentstroke}{rgb}{0.501961,0.000000,0.000000}%
\pgfsetstrokecolor{currentstroke}%
\pgfsetstrokeopacity{0.400000}%
\pgfsetdash{}{0pt}%
\pgfpathmoveto{\pgfqpoint{3.266887in}{0.959063in}}%
\pgfpathcurveto{\pgfqpoint{3.277937in}{0.959063in}}{\pgfqpoint{3.288536in}{0.963453in}}{\pgfqpoint{3.296350in}{0.971267in}}%
\pgfpathcurveto{\pgfqpoint{3.304164in}{0.979080in}}{\pgfqpoint{3.308554in}{0.989679in}}{\pgfqpoint{3.308554in}{1.000729in}}%
\pgfpathcurveto{\pgfqpoint{3.308554in}{1.011780in}}{\pgfqpoint{3.304164in}{1.022379in}}{\pgfqpoint{3.296350in}{1.030192in}}%
\pgfpathcurveto{\pgfqpoint{3.288536in}{1.038006in}}{\pgfqpoint{3.277937in}{1.042396in}}{\pgfqpoint{3.266887in}{1.042396in}}%
\pgfpathcurveto{\pgfqpoint{3.255837in}{1.042396in}}{\pgfqpoint{3.245238in}{1.038006in}}{\pgfqpoint{3.237425in}{1.030192in}}%
\pgfpathcurveto{\pgfqpoint{3.229611in}{1.022379in}}{\pgfqpoint{3.225221in}{1.011780in}}{\pgfqpoint{3.225221in}{1.000729in}}%
\pgfpathcurveto{\pgfqpoint{3.225221in}{0.989679in}}{\pgfqpoint{3.229611in}{0.979080in}}{\pgfqpoint{3.237425in}{0.971267in}}%
\pgfpathcurveto{\pgfqpoint{3.245238in}{0.963453in}}{\pgfqpoint{3.255837in}{0.959063in}}{\pgfqpoint{3.266887in}{0.959063in}}%
\pgfpathclose%
\pgfusepath{stroke,fill}%
\end{pgfscope}%
\begin{pgfscope}%
\pgfpathrectangle{\pgfqpoint{0.511823in}{0.504323in}}{\pgfqpoint{3.218177in}{3.225677in}} %
\pgfusepath{clip}%
\pgfsetbuttcap%
\pgfsetroundjoin%
\definecolor{currentfill}{rgb}{0.501961,0.000000,0.000000}%
\pgfsetfillcolor{currentfill}%
\pgfsetfillopacity{0.400000}%
\pgfsetlinewidth{0.501875pt}%
\definecolor{currentstroke}{rgb}{0.501961,0.000000,0.000000}%
\pgfsetstrokecolor{currentstroke}%
\pgfsetstrokeopacity{0.400000}%
\pgfsetdash{}{0pt}%
\pgfpathmoveto{\pgfqpoint{3.110723in}{0.946969in}}%
\pgfpathcurveto{\pgfqpoint{3.121773in}{0.946969in}}{\pgfqpoint{3.132372in}{0.951359in}}{\pgfqpoint{3.140186in}{0.959172in}}%
\pgfpathcurveto{\pgfqpoint{3.148000in}{0.966986in}}{\pgfqpoint{3.152390in}{0.977585in}}{\pgfqpoint{3.152390in}{0.988635in}}%
\pgfpathcurveto{\pgfqpoint{3.152390in}{0.999685in}}{\pgfqpoint{3.148000in}{1.010284in}}{\pgfqpoint{3.140186in}{1.018098in}}%
\pgfpathcurveto{\pgfqpoint{3.132372in}{1.025912in}}{\pgfqpoint{3.121773in}{1.030302in}}{\pgfqpoint{3.110723in}{1.030302in}}%
\pgfpathcurveto{\pgfqpoint{3.099673in}{1.030302in}}{\pgfqpoint{3.089074in}{1.025912in}}{\pgfqpoint{3.081261in}{1.018098in}}%
\pgfpathcurveto{\pgfqpoint{3.073447in}{1.010284in}}{\pgfqpoint{3.069057in}{0.999685in}}{\pgfqpoint{3.069057in}{0.988635in}}%
\pgfpathcurveto{\pgfqpoint{3.069057in}{0.977585in}}{\pgfqpoint{3.073447in}{0.966986in}}{\pgfqpoint{3.081261in}{0.959172in}}%
\pgfpathcurveto{\pgfqpoint{3.089074in}{0.951359in}}{\pgfqpoint{3.099673in}{0.946969in}}{\pgfqpoint{3.110723in}{0.946969in}}%
\pgfpathclose%
\pgfusepath{stroke,fill}%
\end{pgfscope}%
\begin{pgfscope}%
\pgfpathrectangle{\pgfqpoint{0.511823in}{0.504323in}}{\pgfqpoint{3.218177in}{3.225677in}} %
\pgfusepath{clip}%
\pgfsetbuttcap%
\pgfsetroundjoin%
\definecolor{currentfill}{rgb}{0.501961,0.000000,0.000000}%
\pgfsetfillcolor{currentfill}%
\pgfsetfillopacity{0.400000}%
\pgfsetlinewidth{0.501875pt}%
\definecolor{currentstroke}{rgb}{0.501961,0.000000,0.000000}%
\pgfsetstrokecolor{currentstroke}%
\pgfsetstrokeopacity{0.400000}%
\pgfsetdash{}{0pt}%
\pgfpathmoveto{\pgfqpoint{3.450698in}{0.998301in}}%
\pgfpathcurveto{\pgfqpoint{3.461748in}{0.998301in}}{\pgfqpoint{3.472347in}{1.002691in}}{\pgfqpoint{3.480161in}{1.010504in}}%
\pgfpathcurveto{\pgfqpoint{3.487975in}{1.018318in}}{\pgfqpoint{3.492365in}{1.028917in}}{\pgfqpoint{3.492365in}{1.039967in}}%
\pgfpathcurveto{\pgfqpoint{3.492365in}{1.051017in}}{\pgfqpoint{3.487975in}{1.061616in}}{\pgfqpoint{3.480161in}{1.069430in}}%
\pgfpathcurveto{\pgfqpoint{3.472347in}{1.077244in}}{\pgfqpoint{3.461748in}{1.081634in}}{\pgfqpoint{3.450698in}{1.081634in}}%
\pgfpathcurveto{\pgfqpoint{3.439648in}{1.081634in}}{\pgfqpoint{3.429049in}{1.077244in}}{\pgfqpoint{3.421235in}{1.069430in}}%
\pgfpathcurveto{\pgfqpoint{3.413422in}{1.061616in}}{\pgfqpoint{3.409032in}{1.051017in}}{\pgfqpoint{3.409032in}{1.039967in}}%
\pgfpathcurveto{\pgfqpoint{3.409032in}{1.028917in}}{\pgfqpoint{3.413422in}{1.018318in}}{\pgfqpoint{3.421235in}{1.010504in}}%
\pgfpathcurveto{\pgfqpoint{3.429049in}{1.002691in}}{\pgfqpoint{3.439648in}{0.998301in}}{\pgfqpoint{3.450698in}{0.998301in}}%
\pgfpathclose%
\pgfusepath{stroke,fill}%
\end{pgfscope}%
\begin{pgfscope}%
\pgfpathrectangle{\pgfqpoint{0.511823in}{0.504323in}}{\pgfqpoint{3.218177in}{3.225677in}} %
\pgfusepath{clip}%
\pgfsetbuttcap%
\pgfsetroundjoin%
\definecolor{currentfill}{rgb}{0.501961,0.000000,0.000000}%
\pgfsetfillcolor{currentfill}%
\pgfsetfillopacity{0.400000}%
\pgfsetlinewidth{0.501875pt}%
\definecolor{currentstroke}{rgb}{0.501961,0.000000,0.000000}%
\pgfsetstrokecolor{currentstroke}%
\pgfsetstrokeopacity{0.400000}%
\pgfsetdash{}{0pt}%
\pgfpathmoveto{\pgfqpoint{3.358572in}{0.994334in}}%
\pgfpathcurveto{\pgfqpoint{3.369622in}{0.994334in}}{\pgfqpoint{3.380221in}{0.998724in}}{\pgfqpoint{3.388035in}{1.006537in}}%
\pgfpathcurveto{\pgfqpoint{3.395848in}{1.014351in}}{\pgfqpoint{3.400239in}{1.024950in}}{\pgfqpoint{3.400239in}{1.036000in}}%
\pgfpathcurveto{\pgfqpoint{3.400239in}{1.047050in}}{\pgfqpoint{3.395848in}{1.057649in}}{\pgfqpoint{3.388035in}{1.065463in}}%
\pgfpathcurveto{\pgfqpoint{3.380221in}{1.073277in}}{\pgfqpoint{3.369622in}{1.077667in}}{\pgfqpoint{3.358572in}{1.077667in}}%
\pgfpathcurveto{\pgfqpoint{3.347522in}{1.077667in}}{\pgfqpoint{3.336923in}{1.073277in}}{\pgfqpoint{3.329109in}{1.065463in}}%
\pgfpathcurveto{\pgfqpoint{3.321296in}{1.057649in}}{\pgfqpoint{3.316905in}{1.047050in}}{\pgfqpoint{3.316905in}{1.036000in}}%
\pgfpathcurveto{\pgfqpoint{3.316905in}{1.024950in}}{\pgfqpoint{3.321296in}{1.014351in}}{\pgfqpoint{3.329109in}{1.006537in}}%
\pgfpathcurveto{\pgfqpoint{3.336923in}{0.998724in}}{\pgfqpoint{3.347522in}{0.994334in}}{\pgfqpoint{3.358572in}{0.994334in}}%
\pgfpathclose%
\pgfusepath{stroke,fill}%
\end{pgfscope}%
\begin{pgfscope}%
\pgfpathrectangle{\pgfqpoint{0.511823in}{0.504323in}}{\pgfqpoint{3.218177in}{3.225677in}} %
\pgfusepath{clip}%
\pgfsetbuttcap%
\pgfsetroundjoin%
\definecolor{currentfill}{rgb}{0.501961,0.000000,0.000000}%
\pgfsetfillcolor{currentfill}%
\pgfsetfillopacity{0.400000}%
\pgfsetlinewidth{0.501875pt}%
\definecolor{currentstroke}{rgb}{0.501961,0.000000,0.000000}%
\pgfsetstrokecolor{currentstroke}%
\pgfsetstrokeopacity{0.400000}%
\pgfsetdash{}{0pt}%
\pgfpathmoveto{\pgfqpoint{3.327997in}{0.998174in}}%
\pgfpathcurveto{\pgfqpoint{3.339047in}{0.998174in}}{\pgfqpoint{3.349646in}{1.002565in}}{\pgfqpoint{3.357460in}{1.010378in}}%
\pgfpathcurveto{\pgfqpoint{3.365273in}{1.018192in}}{\pgfqpoint{3.369664in}{1.028791in}}{\pgfqpoint{3.369664in}{1.039841in}}%
\pgfpathcurveto{\pgfqpoint{3.369664in}{1.050891in}}{\pgfqpoint{3.365273in}{1.061490in}}{\pgfqpoint{3.357460in}{1.069304in}}%
\pgfpathcurveto{\pgfqpoint{3.349646in}{1.077118in}}{\pgfqpoint{3.339047in}{1.081508in}}{\pgfqpoint{3.327997in}{1.081508in}}%
\pgfpathcurveto{\pgfqpoint{3.316947in}{1.081508in}}{\pgfqpoint{3.306348in}{1.077118in}}{\pgfqpoint{3.298534in}{1.069304in}}%
\pgfpathcurveto{\pgfqpoint{3.290721in}{1.061490in}}{\pgfqpoint{3.286330in}{1.050891in}}{\pgfqpoint{3.286330in}{1.039841in}}%
\pgfpathcurveto{\pgfqpoint{3.286330in}{1.028791in}}{\pgfqpoint{3.290721in}{1.018192in}}{\pgfqpoint{3.298534in}{1.010378in}}%
\pgfpathcurveto{\pgfqpoint{3.306348in}{1.002565in}}{\pgfqpoint{3.316947in}{0.998174in}}{\pgfqpoint{3.327997in}{0.998174in}}%
\pgfpathclose%
\pgfusepath{stroke,fill}%
\end{pgfscope}%
\begin{pgfscope}%
\pgfpathrectangle{\pgfqpoint{0.511823in}{0.504323in}}{\pgfqpoint{3.218177in}{3.225677in}} %
\pgfusepath{clip}%
\pgfsetbuttcap%
\pgfsetroundjoin%
\definecolor{currentfill}{rgb}{0.501961,0.000000,0.000000}%
\pgfsetfillcolor{currentfill}%
\pgfsetfillopacity{0.400000}%
\pgfsetlinewidth{0.501875pt}%
\definecolor{currentstroke}{rgb}{0.501961,0.000000,0.000000}%
\pgfsetstrokecolor{currentstroke}%
\pgfsetstrokeopacity{0.400000}%
\pgfsetdash{}{0pt}%
\pgfpathmoveto{\pgfqpoint{3.532465in}{1.034434in}}%
\pgfpathcurveto{\pgfqpoint{3.543515in}{1.034434in}}{\pgfqpoint{3.554114in}{1.038825in}}{\pgfqpoint{3.561928in}{1.046638in}}%
\pgfpathcurveto{\pgfqpoint{3.569742in}{1.054452in}}{\pgfqpoint{3.574132in}{1.065051in}}{\pgfqpoint{3.574132in}{1.076101in}}%
\pgfpathcurveto{\pgfqpoint{3.574132in}{1.087151in}}{\pgfqpoint{3.569742in}{1.097750in}}{\pgfqpoint{3.561928in}{1.105564in}}%
\pgfpathcurveto{\pgfqpoint{3.554114in}{1.113377in}}{\pgfqpoint{3.543515in}{1.117768in}}{\pgfqpoint{3.532465in}{1.117768in}}%
\pgfpathcurveto{\pgfqpoint{3.521415in}{1.117768in}}{\pgfqpoint{3.510816in}{1.113377in}}{\pgfqpoint{3.503002in}{1.105564in}}%
\pgfpathcurveto{\pgfqpoint{3.495189in}{1.097750in}}{\pgfqpoint{3.490798in}{1.087151in}}{\pgfqpoint{3.490798in}{1.076101in}}%
\pgfpathcurveto{\pgfqpoint{3.490798in}{1.065051in}}{\pgfqpoint{3.495189in}{1.054452in}}{\pgfqpoint{3.503002in}{1.046638in}}%
\pgfpathcurveto{\pgfqpoint{3.510816in}{1.038825in}}{\pgfqpoint{3.521415in}{1.034434in}}{\pgfqpoint{3.532465in}{1.034434in}}%
\pgfpathclose%
\pgfusepath{stroke,fill}%
\end{pgfscope}%
\begin{pgfscope}%
\pgfpathrectangle{\pgfqpoint{0.511823in}{0.504323in}}{\pgfqpoint{3.218177in}{3.225677in}} %
\pgfusepath{clip}%
\pgfsetbuttcap%
\pgfsetroundjoin%
\definecolor{currentfill}{rgb}{0.501961,0.000000,0.000000}%
\pgfsetfillcolor{currentfill}%
\pgfsetfillopacity{0.400000}%
\pgfsetlinewidth{0.501875pt}%
\definecolor{currentstroke}{rgb}{0.501961,0.000000,0.000000}%
\pgfsetstrokecolor{currentstroke}%
\pgfsetstrokeopacity{0.400000}%
\pgfsetdash{}{0pt}%
\pgfpathmoveto{\pgfqpoint{3.497371in}{1.037988in}}%
\pgfpathcurveto{\pgfqpoint{3.508421in}{1.037988in}}{\pgfqpoint{3.519020in}{1.042378in}}{\pgfqpoint{3.526834in}{1.050192in}}%
\pgfpathcurveto{\pgfqpoint{3.534648in}{1.058005in}}{\pgfqpoint{3.539038in}{1.068604in}}{\pgfqpoint{3.539038in}{1.079654in}}%
\pgfpathcurveto{\pgfqpoint{3.539038in}{1.090704in}}{\pgfqpoint{3.534648in}{1.101304in}}{\pgfqpoint{3.526834in}{1.109117in}}%
\pgfpathcurveto{\pgfqpoint{3.519020in}{1.116931in}}{\pgfqpoint{3.508421in}{1.121321in}}{\pgfqpoint{3.497371in}{1.121321in}}%
\pgfpathcurveto{\pgfqpoint{3.486321in}{1.121321in}}{\pgfqpoint{3.475722in}{1.116931in}}{\pgfqpoint{3.467909in}{1.109117in}}%
\pgfpathcurveto{\pgfqpoint{3.460095in}{1.101304in}}{\pgfqpoint{3.455705in}{1.090704in}}{\pgfqpoint{3.455705in}{1.079654in}}%
\pgfpathcurveto{\pgfqpoint{3.455705in}{1.068604in}}{\pgfqpoint{3.460095in}{1.058005in}}{\pgfqpoint{3.467909in}{1.050192in}}%
\pgfpathcurveto{\pgfqpoint{3.475722in}{1.042378in}}{\pgfqpoint{3.486321in}{1.037988in}}{\pgfqpoint{3.497371in}{1.037988in}}%
\pgfpathclose%
\pgfusepath{stroke,fill}%
\end{pgfscope}%
\begin{pgfscope}%
\pgfpathrectangle{\pgfqpoint{0.511823in}{0.504323in}}{\pgfqpoint{3.218177in}{3.225677in}} %
\pgfusepath{clip}%
\pgfsetbuttcap%
\pgfsetroundjoin%
\definecolor{currentfill}{rgb}{0.501961,0.000000,0.000000}%
\pgfsetfillcolor{currentfill}%
\pgfsetfillopacity{0.400000}%
\pgfsetlinewidth{0.501875pt}%
\definecolor{currentstroke}{rgb}{0.501961,0.000000,0.000000}%
\pgfsetstrokecolor{currentstroke}%
\pgfsetstrokeopacity{0.400000}%
\pgfsetdash{}{0pt}%
\pgfpathmoveto{\pgfqpoint{3.224090in}{1.006880in}}%
\pgfpathcurveto{\pgfqpoint{3.235140in}{1.006880in}}{\pgfqpoint{3.245739in}{1.011271in}}{\pgfqpoint{3.253552in}{1.019084in}}%
\pgfpathcurveto{\pgfqpoint{3.261366in}{1.026898in}}{\pgfqpoint{3.265756in}{1.037497in}}{\pgfqpoint{3.265756in}{1.048547in}}%
\pgfpathcurveto{\pgfqpoint{3.265756in}{1.059597in}}{\pgfqpoint{3.261366in}{1.070196in}}{\pgfqpoint{3.253552in}{1.078010in}}%
\pgfpathcurveto{\pgfqpoint{3.245739in}{1.085823in}}{\pgfqpoint{3.235140in}{1.090214in}}{\pgfqpoint{3.224090in}{1.090214in}}%
\pgfpathcurveto{\pgfqpoint{3.213039in}{1.090214in}}{\pgfqpoint{3.202440in}{1.085823in}}{\pgfqpoint{3.194627in}{1.078010in}}%
\pgfpathcurveto{\pgfqpoint{3.186813in}{1.070196in}}{\pgfqpoint{3.182423in}{1.059597in}}{\pgfqpoint{3.182423in}{1.048547in}}%
\pgfpathcurveto{\pgfqpoint{3.182423in}{1.037497in}}{\pgfqpoint{3.186813in}{1.026898in}}{\pgfqpoint{3.194627in}{1.019084in}}%
\pgfpathcurveto{\pgfqpoint{3.202440in}{1.011271in}}{\pgfqpoint{3.213039in}{1.006880in}}{\pgfqpoint{3.224090in}{1.006880in}}%
\pgfpathclose%
\pgfusepath{stroke,fill}%
\end{pgfscope}%
\begin{pgfscope}%
\pgfpathrectangle{\pgfqpoint{0.511823in}{0.504323in}}{\pgfqpoint{3.218177in}{3.225677in}} %
\pgfusepath{clip}%
\pgfsetbuttcap%
\pgfsetroundjoin%
\definecolor{currentfill}{rgb}{0.501961,0.000000,0.000000}%
\pgfsetfillcolor{currentfill}%
\pgfsetfillopacity{0.400000}%
\pgfsetlinewidth{0.501875pt}%
\definecolor{currentstroke}{rgb}{0.501961,0.000000,0.000000}%
\pgfsetstrokecolor{currentstroke}%
\pgfsetstrokeopacity{0.400000}%
\pgfsetdash{}{0pt}%
\pgfpathmoveto{\pgfqpoint{3.161660in}{1.005278in}}%
\pgfpathcurveto{\pgfqpoint{3.172710in}{1.005278in}}{\pgfqpoint{3.183309in}{1.009668in}}{\pgfqpoint{3.191123in}{1.017481in}}%
\pgfpathcurveto{\pgfqpoint{3.198936in}{1.025295in}}{\pgfqpoint{3.203327in}{1.035894in}}{\pgfqpoint{3.203327in}{1.046944in}}%
\pgfpathcurveto{\pgfqpoint{3.203327in}{1.057994in}}{\pgfqpoint{3.198936in}{1.068593in}}{\pgfqpoint{3.191123in}{1.076407in}}%
\pgfpathcurveto{\pgfqpoint{3.183309in}{1.084221in}}{\pgfqpoint{3.172710in}{1.088611in}}{\pgfqpoint{3.161660in}{1.088611in}}%
\pgfpathcurveto{\pgfqpoint{3.150610in}{1.088611in}}{\pgfqpoint{3.140011in}{1.084221in}}{\pgfqpoint{3.132197in}{1.076407in}}%
\pgfpathcurveto{\pgfqpoint{3.124384in}{1.068593in}}{\pgfqpoint{3.119993in}{1.057994in}}{\pgfqpoint{3.119993in}{1.046944in}}%
\pgfpathcurveto{\pgfqpoint{3.119993in}{1.035894in}}{\pgfqpoint{3.124384in}{1.025295in}}{\pgfqpoint{3.132197in}{1.017481in}}%
\pgfpathcurveto{\pgfqpoint{3.140011in}{1.009668in}}{\pgfqpoint{3.150610in}{1.005278in}}{\pgfqpoint{3.161660in}{1.005278in}}%
\pgfpathclose%
\pgfusepath{stroke,fill}%
\end{pgfscope}%
\begin{pgfscope}%
\pgfpathrectangle{\pgfqpoint{0.511823in}{0.504323in}}{\pgfqpoint{3.218177in}{3.225677in}} %
\pgfusepath{clip}%
\pgfsetbuttcap%
\pgfsetroundjoin%
\definecolor{currentfill}{rgb}{0.501961,0.000000,0.000000}%
\pgfsetfillcolor{currentfill}%
\pgfsetfillopacity{0.400000}%
\pgfsetlinewidth{0.501875pt}%
\definecolor{currentstroke}{rgb}{0.501961,0.000000,0.000000}%
\pgfsetstrokecolor{currentstroke}%
\pgfsetstrokeopacity{0.400000}%
\pgfsetdash{}{0pt}%
\pgfpathmoveto{\pgfqpoint{3.057456in}{0.997013in}}%
\pgfpathcurveto{\pgfqpoint{3.068506in}{0.997013in}}{\pgfqpoint{3.079105in}{1.001403in}}{\pgfqpoint{3.086918in}{1.009217in}}%
\pgfpathcurveto{\pgfqpoint{3.094732in}{1.017030in}}{\pgfqpoint{3.099122in}{1.027629in}}{\pgfqpoint{3.099122in}{1.038680in}}%
\pgfpathcurveto{\pgfqpoint{3.099122in}{1.049730in}}{\pgfqpoint{3.094732in}{1.060329in}}{\pgfqpoint{3.086918in}{1.068142in}}%
\pgfpathcurveto{\pgfqpoint{3.079105in}{1.075956in}}{\pgfqpoint{3.068506in}{1.080346in}}{\pgfqpoint{3.057456in}{1.080346in}}%
\pgfpathcurveto{\pgfqpoint{3.046406in}{1.080346in}}{\pgfqpoint{3.035806in}{1.075956in}}{\pgfqpoint{3.027993in}{1.068142in}}%
\pgfpathcurveto{\pgfqpoint{3.020179in}{1.060329in}}{\pgfqpoint{3.015789in}{1.049730in}}{\pgfqpoint{3.015789in}{1.038680in}}%
\pgfpathcurveto{\pgfqpoint{3.015789in}{1.027629in}}{\pgfqpoint{3.020179in}{1.017030in}}{\pgfqpoint{3.027993in}{1.009217in}}%
\pgfpathcurveto{\pgfqpoint{3.035806in}{1.001403in}}{\pgfqpoint{3.046406in}{0.997013in}}{\pgfqpoint{3.057456in}{0.997013in}}%
\pgfpathclose%
\pgfusepath{stroke,fill}%
\end{pgfscope}%
\begin{pgfscope}%
\pgfpathrectangle{\pgfqpoint{0.511823in}{0.504323in}}{\pgfqpoint{3.218177in}{3.225677in}} %
\pgfusepath{clip}%
\pgfsetbuttcap%
\pgfsetroundjoin%
\definecolor{currentfill}{rgb}{0.501961,0.000000,0.000000}%
\pgfsetfillcolor{currentfill}%
\pgfsetfillopacity{0.400000}%
\pgfsetlinewidth{0.501875pt}%
\definecolor{currentstroke}{rgb}{0.501961,0.000000,0.000000}%
\pgfsetstrokecolor{currentstroke}%
\pgfsetstrokeopacity{0.400000}%
\pgfsetdash{}{0pt}%
\pgfpathmoveto{\pgfqpoint{3.305047in}{1.042193in}}%
\pgfpathcurveto{\pgfqpoint{3.316097in}{1.042193in}}{\pgfqpoint{3.326696in}{1.046584in}}{\pgfqpoint{3.334510in}{1.054397in}}%
\pgfpathcurveto{\pgfqpoint{3.342323in}{1.062211in}}{\pgfqpoint{3.346713in}{1.072810in}}{\pgfqpoint{3.346713in}{1.083860in}}%
\pgfpathcurveto{\pgfqpoint{3.346713in}{1.094910in}}{\pgfqpoint{3.342323in}{1.105509in}}{\pgfqpoint{3.334510in}{1.113323in}}%
\pgfpathcurveto{\pgfqpoint{3.326696in}{1.121137in}}{\pgfqpoint{3.316097in}{1.125527in}}{\pgfqpoint{3.305047in}{1.125527in}}%
\pgfpathcurveto{\pgfqpoint{3.293997in}{1.125527in}}{\pgfqpoint{3.283398in}{1.121137in}}{\pgfqpoint{3.275584in}{1.113323in}}%
\pgfpathcurveto{\pgfqpoint{3.267770in}{1.105509in}}{\pgfqpoint{3.263380in}{1.094910in}}{\pgfqpoint{3.263380in}{1.083860in}}%
\pgfpathcurveto{\pgfqpoint{3.263380in}{1.072810in}}{\pgfqpoint{3.267770in}{1.062211in}}{\pgfqpoint{3.275584in}{1.054397in}}%
\pgfpathcurveto{\pgfqpoint{3.283398in}{1.046584in}}{\pgfqpoint{3.293997in}{1.042193in}}{\pgfqpoint{3.305047in}{1.042193in}}%
\pgfpathclose%
\pgfusepath{stroke,fill}%
\end{pgfscope}%
\begin{pgfscope}%
\pgfpathrectangle{\pgfqpoint{0.511823in}{0.504323in}}{\pgfqpoint{3.218177in}{3.225677in}} %
\pgfusepath{clip}%
\pgfsetbuttcap%
\pgfsetroundjoin%
\definecolor{currentfill}{rgb}{0.501961,0.000000,0.000000}%
\pgfsetfillcolor{currentfill}%
\pgfsetfillopacity{0.400000}%
\pgfsetlinewidth{0.501875pt}%
\definecolor{currentstroke}{rgb}{0.501961,0.000000,0.000000}%
\pgfsetstrokecolor{currentstroke}%
\pgfsetstrokeopacity{0.400000}%
\pgfsetdash{}{0pt}%
\pgfpathmoveto{\pgfqpoint{3.252345in}{1.041819in}}%
\pgfpathcurveto{\pgfqpoint{3.263395in}{1.041819in}}{\pgfqpoint{3.273994in}{1.046209in}}{\pgfqpoint{3.281808in}{1.054023in}}%
\pgfpathcurveto{\pgfqpoint{3.289621in}{1.061836in}}{\pgfqpoint{3.294012in}{1.072435in}}{\pgfqpoint{3.294012in}{1.083485in}}%
\pgfpathcurveto{\pgfqpoint{3.294012in}{1.094536in}}{\pgfqpoint{3.289621in}{1.105135in}}{\pgfqpoint{3.281808in}{1.112948in}}%
\pgfpathcurveto{\pgfqpoint{3.273994in}{1.120762in}}{\pgfqpoint{3.263395in}{1.125152in}}{\pgfqpoint{3.252345in}{1.125152in}}%
\pgfpathcurveto{\pgfqpoint{3.241295in}{1.125152in}}{\pgfqpoint{3.230696in}{1.120762in}}{\pgfqpoint{3.222882in}{1.112948in}}%
\pgfpathcurveto{\pgfqpoint{3.215069in}{1.105135in}}{\pgfqpoint{3.210678in}{1.094536in}}{\pgfqpoint{3.210678in}{1.083485in}}%
\pgfpathcurveto{\pgfqpoint{3.210678in}{1.072435in}}{\pgfqpoint{3.215069in}{1.061836in}}{\pgfqpoint{3.222882in}{1.054023in}}%
\pgfpathcurveto{\pgfqpoint{3.230696in}{1.046209in}}{\pgfqpoint{3.241295in}{1.041819in}}{\pgfqpoint{3.252345in}{1.041819in}}%
\pgfpathclose%
\pgfusepath{stroke,fill}%
\end{pgfscope}%
\begin{pgfscope}%
\pgfpathrectangle{\pgfqpoint{0.511823in}{0.504323in}}{\pgfqpoint{3.218177in}{3.225677in}} %
\pgfusepath{clip}%
\pgfsetbuttcap%
\pgfsetroundjoin%
\definecolor{currentfill}{rgb}{0.501961,0.000000,0.000000}%
\pgfsetfillcolor{currentfill}%
\pgfsetfillopacity{0.400000}%
\pgfsetlinewidth{0.501875pt}%
\definecolor{currentstroke}{rgb}{0.501961,0.000000,0.000000}%
\pgfsetstrokecolor{currentstroke}%
\pgfsetstrokeopacity{0.400000}%
\pgfsetdash{}{0pt}%
\pgfpathmoveto{\pgfqpoint{3.348947in}{1.064979in}}%
\pgfpathcurveto{\pgfqpoint{3.359997in}{1.064979in}}{\pgfqpoint{3.370596in}{1.069370in}}{\pgfqpoint{3.378409in}{1.077183in}}%
\pgfpathcurveto{\pgfqpoint{3.386223in}{1.084997in}}{\pgfqpoint{3.390613in}{1.095596in}}{\pgfqpoint{3.390613in}{1.106646in}}%
\pgfpathcurveto{\pgfqpoint{3.390613in}{1.117696in}}{\pgfqpoint{3.386223in}{1.128295in}}{\pgfqpoint{3.378409in}{1.136109in}}%
\pgfpathcurveto{\pgfqpoint{3.370596in}{1.143922in}}{\pgfqpoint{3.359997in}{1.148313in}}{\pgfqpoint{3.348947in}{1.148313in}}%
\pgfpathcurveto{\pgfqpoint{3.337897in}{1.148313in}}{\pgfqpoint{3.327298in}{1.143922in}}{\pgfqpoint{3.319484in}{1.136109in}}%
\pgfpathcurveto{\pgfqpoint{3.311670in}{1.128295in}}{\pgfqpoint{3.307280in}{1.117696in}}{\pgfqpoint{3.307280in}{1.106646in}}%
\pgfpathcurveto{\pgfqpoint{3.307280in}{1.095596in}}{\pgfqpoint{3.311670in}{1.084997in}}{\pgfqpoint{3.319484in}{1.077183in}}%
\pgfpathcurveto{\pgfqpoint{3.327298in}{1.069370in}}{\pgfqpoint{3.337897in}{1.064979in}}{\pgfqpoint{3.348947in}{1.064979in}}%
\pgfpathclose%
\pgfusepath{stroke,fill}%
\end{pgfscope}%
\begin{pgfscope}%
\pgfpathrectangle{\pgfqpoint{0.511823in}{0.504323in}}{\pgfqpoint{3.218177in}{3.225677in}} %
\pgfusepath{clip}%
\pgfsetbuttcap%
\pgfsetroundjoin%
\definecolor{currentfill}{rgb}{0.501961,0.000000,0.000000}%
\pgfsetfillcolor{currentfill}%
\pgfsetfillopacity{0.400000}%
\pgfsetlinewidth{0.501875pt}%
\definecolor{currentstroke}{rgb}{0.501961,0.000000,0.000000}%
\pgfsetstrokecolor{currentstroke}%
\pgfsetstrokeopacity{0.400000}%
\pgfsetdash{}{0pt}%
\pgfpathmoveto{\pgfqpoint{3.301712in}{1.065326in}}%
\pgfpathcurveto{\pgfqpoint{3.312762in}{1.065326in}}{\pgfqpoint{3.323361in}{1.069716in}}{\pgfqpoint{3.331174in}{1.077530in}}%
\pgfpathcurveto{\pgfqpoint{3.338988in}{1.085343in}}{\pgfqpoint{3.343378in}{1.095942in}}{\pgfqpoint{3.343378in}{1.106992in}}%
\pgfpathcurveto{\pgfqpoint{3.343378in}{1.118042in}}{\pgfqpoint{3.338988in}{1.128641in}}{\pgfqpoint{3.331174in}{1.136455in}}%
\pgfpathcurveto{\pgfqpoint{3.323361in}{1.144269in}}{\pgfqpoint{3.312762in}{1.148659in}}{\pgfqpoint{3.301712in}{1.148659in}}%
\pgfpathcurveto{\pgfqpoint{3.290661in}{1.148659in}}{\pgfqpoint{3.280062in}{1.144269in}}{\pgfqpoint{3.272249in}{1.136455in}}%
\pgfpathcurveto{\pgfqpoint{3.264435in}{1.128641in}}{\pgfqpoint{3.260045in}{1.118042in}}{\pgfqpoint{3.260045in}{1.106992in}}%
\pgfpathcurveto{\pgfqpoint{3.260045in}{1.095942in}}{\pgfqpoint{3.264435in}{1.085343in}}{\pgfqpoint{3.272249in}{1.077530in}}%
\pgfpathcurveto{\pgfqpoint{3.280062in}{1.069716in}}{\pgfqpoint{3.290661in}{1.065326in}}{\pgfqpoint{3.301712in}{1.065326in}}%
\pgfpathclose%
\pgfusepath{stroke,fill}%
\end{pgfscope}%
\begin{pgfscope}%
\pgfpathrectangle{\pgfqpoint{0.511823in}{0.504323in}}{\pgfqpoint{3.218177in}{3.225677in}} %
\pgfusepath{clip}%
\pgfsetbuttcap%
\pgfsetroundjoin%
\definecolor{currentfill}{rgb}{0.501961,0.000000,0.000000}%
\pgfsetfillcolor{currentfill}%
\pgfsetfillopacity{0.400000}%
\pgfsetlinewidth{0.501875pt}%
\definecolor{currentstroke}{rgb}{0.501961,0.000000,0.000000}%
\pgfsetstrokecolor{currentstroke}%
\pgfsetstrokeopacity{0.400000}%
\pgfsetdash{}{0pt}%
\pgfpathmoveto{\pgfqpoint{3.212859in}{1.058498in}}%
\pgfpathcurveto{\pgfqpoint{3.223909in}{1.058498in}}{\pgfqpoint{3.234508in}{1.062888in}}{\pgfqpoint{3.242322in}{1.070702in}}%
\pgfpathcurveto{\pgfqpoint{3.250135in}{1.078515in}}{\pgfqpoint{3.254526in}{1.089114in}}{\pgfqpoint{3.254526in}{1.100164in}}%
\pgfpathcurveto{\pgfqpoint{3.254526in}{1.111214in}}{\pgfqpoint{3.250135in}{1.121813in}}{\pgfqpoint{3.242322in}{1.129627in}}%
\pgfpathcurveto{\pgfqpoint{3.234508in}{1.137441in}}{\pgfqpoint{3.223909in}{1.141831in}}{\pgfqpoint{3.212859in}{1.141831in}}%
\pgfpathcurveto{\pgfqpoint{3.201809in}{1.141831in}}{\pgfqpoint{3.191210in}{1.137441in}}{\pgfqpoint{3.183396in}{1.129627in}}%
\pgfpathcurveto{\pgfqpoint{3.175583in}{1.121813in}}{\pgfqpoint{3.171192in}{1.111214in}}{\pgfqpoint{3.171192in}{1.100164in}}%
\pgfpathcurveto{\pgfqpoint{3.171192in}{1.089114in}}{\pgfqpoint{3.175583in}{1.078515in}}{\pgfqpoint{3.183396in}{1.070702in}}%
\pgfpathcurveto{\pgfqpoint{3.191210in}{1.062888in}}{\pgfqpoint{3.201809in}{1.058498in}}{\pgfqpoint{3.212859in}{1.058498in}}%
\pgfpathclose%
\pgfusepath{stroke,fill}%
\end{pgfscope}%
\begin{pgfscope}%
\pgfpathrectangle{\pgfqpoint{0.511823in}{0.504323in}}{\pgfqpoint{3.218177in}{3.225677in}} %
\pgfusepath{clip}%
\pgfsetbuttcap%
\pgfsetroundjoin%
\definecolor{currentfill}{rgb}{0.501961,0.000000,0.000000}%
\pgfsetfillcolor{currentfill}%
\pgfsetfillopacity{0.400000}%
\pgfsetlinewidth{0.501875pt}%
\definecolor{currentstroke}{rgb}{0.501961,0.000000,0.000000}%
\pgfsetstrokecolor{currentstroke}%
\pgfsetstrokeopacity{0.400000}%
\pgfsetdash{}{0pt}%
\pgfpathmoveto{\pgfqpoint{3.244322in}{1.071446in}}%
\pgfpathcurveto{\pgfqpoint{3.255372in}{1.071446in}}{\pgfqpoint{3.265971in}{1.075837in}}{\pgfqpoint{3.273785in}{1.083650in}}%
\pgfpathcurveto{\pgfqpoint{3.281598in}{1.091464in}}{\pgfqpoint{3.285989in}{1.102063in}}{\pgfqpoint{3.285989in}{1.113113in}}%
\pgfpathcurveto{\pgfqpoint{3.285989in}{1.124163in}}{\pgfqpoint{3.281598in}{1.134762in}}{\pgfqpoint{3.273785in}{1.142576in}}%
\pgfpathcurveto{\pgfqpoint{3.265971in}{1.150390in}}{\pgfqpoint{3.255372in}{1.154780in}}{\pgfqpoint{3.244322in}{1.154780in}}%
\pgfpathcurveto{\pgfqpoint{3.233272in}{1.154780in}}{\pgfqpoint{3.222673in}{1.150390in}}{\pgfqpoint{3.214859in}{1.142576in}}%
\pgfpathcurveto{\pgfqpoint{3.207046in}{1.134762in}}{\pgfqpoint{3.202655in}{1.124163in}}{\pgfqpoint{3.202655in}{1.113113in}}%
\pgfpathcurveto{\pgfqpoint{3.202655in}{1.102063in}}{\pgfqpoint{3.207046in}{1.091464in}}{\pgfqpoint{3.214859in}{1.083650in}}%
\pgfpathcurveto{\pgfqpoint{3.222673in}{1.075837in}}{\pgfqpoint{3.233272in}{1.071446in}}{\pgfqpoint{3.244322in}{1.071446in}}%
\pgfpathclose%
\pgfusepath{stroke,fill}%
\end{pgfscope}%
\begin{pgfscope}%
\pgfpathrectangle{\pgfqpoint{0.511823in}{0.504323in}}{\pgfqpoint{3.218177in}{3.225677in}} %
\pgfusepath{clip}%
\pgfsetbuttcap%
\pgfsetroundjoin%
\definecolor{currentfill}{rgb}{0.501961,0.000000,0.000000}%
\pgfsetfillcolor{currentfill}%
\pgfsetfillopacity{0.400000}%
\pgfsetlinewidth{0.501875pt}%
\definecolor{currentstroke}{rgb}{0.501961,0.000000,0.000000}%
\pgfsetstrokecolor{currentstroke}%
\pgfsetstrokeopacity{0.400000}%
\pgfsetdash{}{0pt}%
\pgfpathmoveto{\pgfqpoint{3.092345in}{1.053081in}}%
\pgfpathcurveto{\pgfqpoint{3.103395in}{1.053081in}}{\pgfqpoint{3.113994in}{1.057471in}}{\pgfqpoint{3.121808in}{1.065285in}}%
\pgfpathcurveto{\pgfqpoint{3.129621in}{1.073098in}}{\pgfqpoint{3.134012in}{1.083697in}}{\pgfqpoint{3.134012in}{1.094747in}}%
\pgfpathcurveto{\pgfqpoint{3.134012in}{1.105798in}}{\pgfqpoint{3.129621in}{1.116397in}}{\pgfqpoint{3.121808in}{1.124210in}}%
\pgfpathcurveto{\pgfqpoint{3.113994in}{1.132024in}}{\pgfqpoint{3.103395in}{1.136414in}}{\pgfqpoint{3.092345in}{1.136414in}}%
\pgfpathcurveto{\pgfqpoint{3.081295in}{1.136414in}}{\pgfqpoint{3.070696in}{1.132024in}}{\pgfqpoint{3.062882in}{1.124210in}}%
\pgfpathcurveto{\pgfqpoint{3.055069in}{1.116397in}}{\pgfqpoint{3.050678in}{1.105798in}}{\pgfqpoint{3.050678in}{1.094747in}}%
\pgfpathcurveto{\pgfqpoint{3.050678in}{1.083697in}}{\pgfqpoint{3.055069in}{1.073098in}}{\pgfqpoint{3.062882in}{1.065285in}}%
\pgfpathcurveto{\pgfqpoint{3.070696in}{1.057471in}}{\pgfqpoint{3.081295in}{1.053081in}}{\pgfqpoint{3.092345in}{1.053081in}}%
\pgfpathclose%
\pgfusepath{stroke,fill}%
\end{pgfscope}%
\begin{pgfscope}%
\pgfpathrectangle{\pgfqpoint{0.511823in}{0.504323in}}{\pgfqpoint{3.218177in}{3.225677in}} %
\pgfusepath{clip}%
\pgfsetbuttcap%
\pgfsetroundjoin%
\definecolor{currentfill}{rgb}{0.501961,0.000000,0.000000}%
\pgfsetfillcolor{currentfill}%
\pgfsetfillopacity{0.400000}%
\pgfsetlinewidth{0.501875pt}%
\definecolor{currentstroke}{rgb}{0.501961,0.000000,0.000000}%
\pgfsetstrokecolor{currentstroke}%
\pgfsetstrokeopacity{0.400000}%
\pgfsetdash{}{0pt}%
\pgfpathmoveto{\pgfqpoint{3.454228in}{1.123638in}}%
\pgfpathcurveto{\pgfqpoint{3.465278in}{1.123638in}}{\pgfqpoint{3.475877in}{1.128028in}}{\pgfqpoint{3.483691in}{1.135842in}}%
\pgfpathcurveto{\pgfqpoint{3.491505in}{1.143655in}}{\pgfqpoint{3.495895in}{1.154255in}}{\pgfqpoint{3.495895in}{1.165305in}}%
\pgfpathcurveto{\pgfqpoint{3.495895in}{1.176355in}}{\pgfqpoint{3.491505in}{1.186954in}}{\pgfqpoint{3.483691in}{1.194767in}}%
\pgfpathcurveto{\pgfqpoint{3.475877in}{1.202581in}}{\pgfqpoint{3.465278in}{1.206971in}}{\pgfqpoint{3.454228in}{1.206971in}}%
\pgfpathcurveto{\pgfqpoint{3.443178in}{1.206971in}}{\pgfqpoint{3.432579in}{1.202581in}}{\pgfqpoint{3.424765in}{1.194767in}}%
\pgfpathcurveto{\pgfqpoint{3.416952in}{1.186954in}}{\pgfqpoint{3.412562in}{1.176355in}}{\pgfqpoint{3.412562in}{1.165305in}}%
\pgfpathcurveto{\pgfqpoint{3.412562in}{1.154255in}}{\pgfqpoint{3.416952in}{1.143655in}}{\pgfqpoint{3.424765in}{1.135842in}}%
\pgfpathcurveto{\pgfqpoint{3.432579in}{1.128028in}}{\pgfqpoint{3.443178in}{1.123638in}}{\pgfqpoint{3.454228in}{1.123638in}}%
\pgfpathclose%
\pgfusepath{stroke,fill}%
\end{pgfscope}%
\begin{pgfscope}%
\pgfpathrectangle{\pgfqpoint{0.511823in}{0.504323in}}{\pgfqpoint{3.218177in}{3.225677in}} %
\pgfusepath{clip}%
\pgfsetbuttcap%
\pgfsetroundjoin%
\definecolor{currentfill}{rgb}{0.501961,0.000000,0.000000}%
\pgfsetfillcolor{currentfill}%
\pgfsetfillopacity{0.400000}%
\pgfsetlinewidth{0.501875pt}%
\definecolor{currentstroke}{rgb}{0.501961,0.000000,0.000000}%
\pgfsetstrokecolor{currentstroke}%
\pgfsetstrokeopacity{0.400000}%
\pgfsetdash{}{0pt}%
\pgfpathmoveto{\pgfqpoint{3.223121in}{1.090932in}}%
\pgfpathcurveto{\pgfqpoint{3.234171in}{1.090932in}}{\pgfqpoint{3.244770in}{1.095322in}}{\pgfqpoint{3.252584in}{1.103136in}}%
\pgfpathcurveto{\pgfqpoint{3.260398in}{1.110949in}}{\pgfqpoint{3.264788in}{1.121548in}}{\pgfqpoint{3.264788in}{1.132598in}}%
\pgfpathcurveto{\pgfqpoint{3.264788in}{1.143648in}}{\pgfqpoint{3.260398in}{1.154247in}}{\pgfqpoint{3.252584in}{1.162061in}}%
\pgfpathcurveto{\pgfqpoint{3.244770in}{1.169875in}}{\pgfqpoint{3.234171in}{1.174265in}}{\pgfqpoint{3.223121in}{1.174265in}}%
\pgfpathcurveto{\pgfqpoint{3.212071in}{1.174265in}}{\pgfqpoint{3.201472in}{1.169875in}}{\pgfqpoint{3.193659in}{1.162061in}}%
\pgfpathcurveto{\pgfqpoint{3.185845in}{1.154247in}}{\pgfqpoint{3.181455in}{1.143648in}}{\pgfqpoint{3.181455in}{1.132598in}}%
\pgfpathcurveto{\pgfqpoint{3.181455in}{1.121548in}}{\pgfqpoint{3.185845in}{1.110949in}}{\pgfqpoint{3.193659in}{1.103136in}}%
\pgfpathcurveto{\pgfqpoint{3.201472in}{1.095322in}}{\pgfqpoint{3.212071in}{1.090932in}}{\pgfqpoint{3.223121in}{1.090932in}}%
\pgfpathclose%
\pgfusepath{stroke,fill}%
\end{pgfscope}%
\begin{pgfscope}%
\pgfpathrectangle{\pgfqpoint{0.511823in}{0.504323in}}{\pgfqpoint{3.218177in}{3.225677in}} %
\pgfusepath{clip}%
\pgfsetbuttcap%
\pgfsetroundjoin%
\definecolor{currentfill}{rgb}{0.501961,0.000000,0.000000}%
\pgfsetfillcolor{currentfill}%
\pgfsetfillopacity{0.400000}%
\pgfsetlinewidth{0.501875pt}%
\definecolor{currentstroke}{rgb}{0.501961,0.000000,0.000000}%
\pgfsetstrokecolor{currentstroke}%
\pgfsetstrokeopacity{0.400000}%
\pgfsetdash{}{0pt}%
\pgfpathmoveto{\pgfqpoint{3.370678in}{1.125330in}}%
\pgfpathcurveto{\pgfqpoint{3.381728in}{1.125330in}}{\pgfqpoint{3.392327in}{1.129720in}}{\pgfqpoint{3.400141in}{1.137534in}}%
\pgfpathcurveto{\pgfqpoint{3.407954in}{1.145348in}}{\pgfqpoint{3.412344in}{1.155947in}}{\pgfqpoint{3.412344in}{1.166997in}}%
\pgfpathcurveto{\pgfqpoint{3.412344in}{1.178047in}}{\pgfqpoint{3.407954in}{1.188646in}}{\pgfqpoint{3.400141in}{1.196460in}}%
\pgfpathcurveto{\pgfqpoint{3.392327in}{1.204273in}}{\pgfqpoint{3.381728in}{1.208663in}}{\pgfqpoint{3.370678in}{1.208663in}}%
\pgfpathcurveto{\pgfqpoint{3.359628in}{1.208663in}}{\pgfqpoint{3.349029in}{1.204273in}}{\pgfqpoint{3.341215in}{1.196460in}}%
\pgfpathcurveto{\pgfqpoint{3.333401in}{1.188646in}}{\pgfqpoint{3.329011in}{1.178047in}}{\pgfqpoint{3.329011in}{1.166997in}}%
\pgfpathcurveto{\pgfqpoint{3.329011in}{1.155947in}}{\pgfqpoint{3.333401in}{1.145348in}}{\pgfqpoint{3.341215in}{1.137534in}}%
\pgfpathcurveto{\pgfqpoint{3.349029in}{1.129720in}}{\pgfqpoint{3.359628in}{1.125330in}}{\pgfqpoint{3.370678in}{1.125330in}}%
\pgfpathclose%
\pgfusepath{stroke,fill}%
\end{pgfscope}%
\begin{pgfscope}%
\pgfpathrectangle{\pgfqpoint{0.511823in}{0.504323in}}{\pgfqpoint{3.218177in}{3.225677in}} %
\pgfusepath{clip}%
\pgfsetbuttcap%
\pgfsetroundjoin%
\definecolor{currentfill}{rgb}{0.501961,0.000000,0.000000}%
\pgfsetfillcolor{currentfill}%
\pgfsetfillopacity{0.400000}%
\pgfsetlinewidth{0.501875pt}%
\definecolor{currentstroke}{rgb}{0.501961,0.000000,0.000000}%
\pgfsetstrokecolor{currentstroke}%
\pgfsetstrokeopacity{0.400000}%
\pgfsetdash{}{0pt}%
\pgfpathmoveto{\pgfqpoint{3.203725in}{1.102785in}}%
\pgfpathcurveto{\pgfqpoint{3.214775in}{1.102785in}}{\pgfqpoint{3.225374in}{1.107176in}}{\pgfqpoint{3.233188in}{1.114989in}}%
\pgfpathcurveto{\pgfqpoint{3.241001in}{1.122803in}}{\pgfqpoint{3.245391in}{1.133402in}}{\pgfqpoint{3.245391in}{1.144452in}}%
\pgfpathcurveto{\pgfqpoint{3.245391in}{1.155502in}}{\pgfqpoint{3.241001in}{1.166101in}}{\pgfqpoint{3.233188in}{1.173915in}}%
\pgfpathcurveto{\pgfqpoint{3.225374in}{1.181728in}}{\pgfqpoint{3.214775in}{1.186119in}}{\pgfqpoint{3.203725in}{1.186119in}}%
\pgfpathcurveto{\pgfqpoint{3.192675in}{1.186119in}}{\pgfqpoint{3.182076in}{1.181728in}}{\pgfqpoint{3.174262in}{1.173915in}}%
\pgfpathcurveto{\pgfqpoint{3.166448in}{1.166101in}}{\pgfqpoint{3.162058in}{1.155502in}}{\pgfqpoint{3.162058in}{1.144452in}}%
\pgfpathcurveto{\pgfqpoint{3.162058in}{1.133402in}}{\pgfqpoint{3.166448in}{1.122803in}}{\pgfqpoint{3.174262in}{1.114989in}}%
\pgfpathcurveto{\pgfqpoint{3.182076in}{1.107176in}}{\pgfqpoint{3.192675in}{1.102785in}}{\pgfqpoint{3.203725in}{1.102785in}}%
\pgfpathclose%
\pgfusepath{stroke,fill}%
\end{pgfscope}%
\begin{pgfscope}%
\pgfpathrectangle{\pgfqpoint{0.511823in}{0.504323in}}{\pgfqpoint{3.218177in}{3.225677in}} %
\pgfusepath{clip}%
\pgfsetbuttcap%
\pgfsetroundjoin%
\definecolor{currentfill}{rgb}{0.501961,0.000000,0.000000}%
\pgfsetfillcolor{currentfill}%
\pgfsetfillopacity{0.400000}%
\pgfsetlinewidth{0.501875pt}%
\definecolor{currentstroke}{rgb}{0.501961,0.000000,0.000000}%
\pgfsetstrokecolor{currentstroke}%
\pgfsetstrokeopacity{0.400000}%
\pgfsetdash{}{0pt}%
\pgfpathmoveto{\pgfqpoint{3.296910in}{1.127875in}}%
\pgfpathcurveto{\pgfqpoint{3.307960in}{1.127875in}}{\pgfqpoint{3.318559in}{1.132266in}}{\pgfqpoint{3.326372in}{1.140079in}}%
\pgfpathcurveto{\pgfqpoint{3.334186in}{1.147893in}}{\pgfqpoint{3.338576in}{1.158492in}}{\pgfqpoint{3.338576in}{1.169542in}}%
\pgfpathcurveto{\pgfqpoint{3.338576in}{1.180592in}}{\pgfqpoint{3.334186in}{1.191191in}}{\pgfqpoint{3.326372in}{1.199005in}}%
\pgfpathcurveto{\pgfqpoint{3.318559in}{1.206818in}}{\pgfqpoint{3.307960in}{1.211209in}}{\pgfqpoint{3.296910in}{1.211209in}}%
\pgfpathcurveto{\pgfqpoint{3.285859in}{1.211209in}}{\pgfqpoint{3.275260in}{1.206818in}}{\pgfqpoint{3.267447in}{1.199005in}}%
\pgfpathcurveto{\pgfqpoint{3.259633in}{1.191191in}}{\pgfqpoint{3.255243in}{1.180592in}}{\pgfqpoint{3.255243in}{1.169542in}}%
\pgfpathcurveto{\pgfqpoint{3.255243in}{1.158492in}}{\pgfqpoint{3.259633in}{1.147893in}}{\pgfqpoint{3.267447in}{1.140079in}}%
\pgfpathcurveto{\pgfqpoint{3.275260in}{1.132266in}}{\pgfqpoint{3.285859in}{1.127875in}}{\pgfqpoint{3.296910in}{1.127875in}}%
\pgfpathclose%
\pgfusepath{stroke,fill}%
\end{pgfscope}%
\begin{pgfscope}%
\pgfpathrectangle{\pgfqpoint{0.511823in}{0.504323in}}{\pgfqpoint{3.218177in}{3.225677in}} %
\pgfusepath{clip}%
\pgfsetbuttcap%
\pgfsetroundjoin%
\definecolor{currentfill}{rgb}{0.501961,0.000000,0.000000}%
\pgfsetfillcolor{currentfill}%
\pgfsetfillopacity{0.400000}%
\pgfsetlinewidth{0.501875pt}%
\definecolor{currentstroke}{rgb}{0.501961,0.000000,0.000000}%
\pgfsetstrokecolor{currentstroke}%
\pgfsetstrokeopacity{0.400000}%
\pgfsetdash{}{0pt}%
\pgfpathmoveto{\pgfqpoint{3.367932in}{1.149332in}}%
\pgfpathcurveto{\pgfqpoint{3.378982in}{1.149332in}}{\pgfqpoint{3.389581in}{1.153722in}}{\pgfqpoint{3.397395in}{1.161536in}}%
\pgfpathcurveto{\pgfqpoint{3.405208in}{1.169349in}}{\pgfqpoint{3.409599in}{1.179948in}}{\pgfqpoint{3.409599in}{1.190998in}}%
\pgfpathcurveto{\pgfqpoint{3.409599in}{1.202049in}}{\pgfqpoint{3.405208in}{1.212648in}}{\pgfqpoint{3.397395in}{1.220461in}}%
\pgfpathcurveto{\pgfqpoint{3.389581in}{1.228275in}}{\pgfqpoint{3.378982in}{1.232665in}}{\pgfqpoint{3.367932in}{1.232665in}}%
\pgfpathcurveto{\pgfqpoint{3.356882in}{1.232665in}}{\pgfqpoint{3.346283in}{1.228275in}}{\pgfqpoint{3.338469in}{1.220461in}}%
\pgfpathcurveto{\pgfqpoint{3.330655in}{1.212648in}}{\pgfqpoint{3.326265in}{1.202049in}}{\pgfqpoint{3.326265in}{1.190998in}}%
\pgfpathcurveto{\pgfqpoint{3.326265in}{1.179948in}}{\pgfqpoint{3.330655in}{1.169349in}}{\pgfqpoint{3.338469in}{1.161536in}}%
\pgfpathcurveto{\pgfqpoint{3.346283in}{1.153722in}}{\pgfqpoint{3.356882in}{1.149332in}}{\pgfqpoint{3.367932in}{1.149332in}}%
\pgfpathclose%
\pgfusepath{stroke,fill}%
\end{pgfscope}%
\begin{pgfscope}%
\pgfpathrectangle{\pgfqpoint{0.511823in}{0.504323in}}{\pgfqpoint{3.218177in}{3.225677in}} %
\pgfusepath{clip}%
\pgfsetbuttcap%
\pgfsetroundjoin%
\definecolor{currentfill}{rgb}{0.501961,0.000000,0.000000}%
\pgfsetfillcolor{currentfill}%
\pgfsetfillopacity{0.400000}%
\pgfsetlinewidth{0.501875pt}%
\definecolor{currentstroke}{rgb}{0.501961,0.000000,0.000000}%
\pgfsetstrokecolor{currentstroke}%
\pgfsetstrokeopacity{0.400000}%
\pgfsetdash{}{0pt}%
\pgfpathmoveto{\pgfqpoint{3.248786in}{1.134513in}}%
\pgfpathcurveto{\pgfqpoint{3.259836in}{1.134513in}}{\pgfqpoint{3.270435in}{1.138904in}}{\pgfqpoint{3.278249in}{1.146717in}}%
\pgfpathcurveto{\pgfqpoint{3.286062in}{1.154531in}}{\pgfqpoint{3.290453in}{1.165130in}}{\pgfqpoint{3.290453in}{1.176180in}}%
\pgfpathcurveto{\pgfqpoint{3.290453in}{1.187230in}}{\pgfqpoint{3.286062in}{1.197829in}}{\pgfqpoint{3.278249in}{1.205643in}}%
\pgfpathcurveto{\pgfqpoint{3.270435in}{1.213456in}}{\pgfqpoint{3.259836in}{1.217847in}}{\pgfqpoint{3.248786in}{1.217847in}}%
\pgfpathcurveto{\pgfqpoint{3.237736in}{1.217847in}}{\pgfqpoint{3.227137in}{1.213456in}}{\pgfqpoint{3.219323in}{1.205643in}}%
\pgfpathcurveto{\pgfqpoint{3.211510in}{1.197829in}}{\pgfqpoint{3.207119in}{1.187230in}}{\pgfqpoint{3.207119in}{1.176180in}}%
\pgfpathcurveto{\pgfqpoint{3.207119in}{1.165130in}}{\pgfqpoint{3.211510in}{1.154531in}}{\pgfqpoint{3.219323in}{1.146717in}}%
\pgfpathcurveto{\pgfqpoint{3.227137in}{1.138904in}}{\pgfqpoint{3.237736in}{1.134513in}}{\pgfqpoint{3.248786in}{1.134513in}}%
\pgfpathclose%
\pgfusepath{stroke,fill}%
\end{pgfscope}%
\begin{pgfscope}%
\pgfpathrectangle{\pgfqpoint{0.511823in}{0.504323in}}{\pgfqpoint{3.218177in}{3.225677in}} %
\pgfusepath{clip}%
\pgfsetbuttcap%
\pgfsetroundjoin%
\definecolor{currentfill}{rgb}{0.501961,0.000000,0.000000}%
\pgfsetfillcolor{currentfill}%
\pgfsetfillopacity{0.400000}%
\pgfsetlinewidth{0.501875pt}%
\definecolor{currentstroke}{rgb}{0.501961,0.000000,0.000000}%
\pgfsetstrokecolor{currentstroke}%
\pgfsetstrokeopacity{0.400000}%
\pgfsetdash{}{0pt}%
\pgfpathmoveto{\pgfqpoint{3.176818in}{1.128231in}}%
\pgfpathcurveto{\pgfqpoint{3.187868in}{1.128231in}}{\pgfqpoint{3.198467in}{1.132621in}}{\pgfqpoint{3.206281in}{1.140435in}}%
\pgfpathcurveto{\pgfqpoint{3.214094in}{1.148248in}}{\pgfqpoint{3.218485in}{1.158847in}}{\pgfqpoint{3.218485in}{1.169897in}}%
\pgfpathcurveto{\pgfqpoint{3.218485in}{1.180947in}}{\pgfqpoint{3.214094in}{1.191547in}}{\pgfqpoint{3.206281in}{1.199360in}}%
\pgfpathcurveto{\pgfqpoint{3.198467in}{1.207174in}}{\pgfqpoint{3.187868in}{1.211564in}}{\pgfqpoint{3.176818in}{1.211564in}}%
\pgfpathcurveto{\pgfqpoint{3.165768in}{1.211564in}}{\pgfqpoint{3.155169in}{1.207174in}}{\pgfqpoint{3.147355in}{1.199360in}}%
\pgfpathcurveto{\pgfqpoint{3.139542in}{1.191547in}}{\pgfqpoint{3.135151in}{1.180947in}}{\pgfqpoint{3.135151in}{1.169897in}}%
\pgfpathcurveto{\pgfqpoint{3.135151in}{1.158847in}}{\pgfqpoint{3.139542in}{1.148248in}}{\pgfqpoint{3.147355in}{1.140435in}}%
\pgfpathcurveto{\pgfqpoint{3.155169in}{1.132621in}}{\pgfqpoint{3.165768in}{1.128231in}}{\pgfqpoint{3.176818in}{1.128231in}}%
\pgfpathclose%
\pgfusepath{stroke,fill}%
\end{pgfscope}%
\begin{pgfscope}%
\pgfpathrectangle{\pgfqpoint{0.511823in}{0.504323in}}{\pgfqpoint{3.218177in}{3.225677in}} %
\pgfusepath{clip}%
\pgfsetbuttcap%
\pgfsetroundjoin%
\definecolor{currentfill}{rgb}{0.501961,0.000000,0.000000}%
\pgfsetfillcolor{currentfill}%
\pgfsetfillopacity{0.400000}%
\pgfsetlinewidth{0.501875pt}%
\definecolor{currentstroke}{rgb}{0.501961,0.000000,0.000000}%
\pgfsetstrokecolor{currentstroke}%
\pgfsetstrokeopacity{0.400000}%
\pgfsetdash{}{0pt}%
\pgfpathmoveto{\pgfqpoint{3.582540in}{1.216676in}}%
\pgfpathcurveto{\pgfqpoint{3.593590in}{1.216676in}}{\pgfqpoint{3.604189in}{1.221066in}}{\pgfqpoint{3.612003in}{1.228879in}}%
\pgfpathcurveto{\pgfqpoint{3.619817in}{1.236693in}}{\pgfqpoint{3.624207in}{1.247292in}}{\pgfqpoint{3.624207in}{1.258342in}}%
\pgfpathcurveto{\pgfqpoint{3.624207in}{1.269392in}}{\pgfqpoint{3.619817in}{1.279991in}}{\pgfqpoint{3.612003in}{1.287805in}}%
\pgfpathcurveto{\pgfqpoint{3.604189in}{1.295619in}}{\pgfqpoint{3.593590in}{1.300009in}}{\pgfqpoint{3.582540in}{1.300009in}}%
\pgfpathcurveto{\pgfqpoint{3.571490in}{1.300009in}}{\pgfqpoint{3.560891in}{1.295619in}}{\pgfqpoint{3.553077in}{1.287805in}}%
\pgfpathcurveto{\pgfqpoint{3.545264in}{1.279991in}}{\pgfqpoint{3.540874in}{1.269392in}}{\pgfqpoint{3.540874in}{1.258342in}}%
\pgfpathcurveto{\pgfqpoint{3.540874in}{1.247292in}}{\pgfqpoint{3.545264in}{1.236693in}}{\pgfqpoint{3.553077in}{1.228879in}}%
\pgfpathcurveto{\pgfqpoint{3.560891in}{1.221066in}}{\pgfqpoint{3.571490in}{1.216676in}}{\pgfqpoint{3.582540in}{1.216676in}}%
\pgfpathclose%
\pgfusepath{stroke,fill}%
\end{pgfscope}%
\begin{pgfscope}%
\pgfpathrectangle{\pgfqpoint{0.511823in}{0.504323in}}{\pgfqpoint{3.218177in}{3.225677in}} %
\pgfusepath{clip}%
\pgfsetbuttcap%
\pgfsetroundjoin%
\definecolor{currentfill}{rgb}{0.501961,0.000000,0.000000}%
\pgfsetfillcolor{currentfill}%
\pgfsetfillopacity{0.400000}%
\pgfsetlinewidth{0.501875pt}%
\definecolor{currentstroke}{rgb}{0.501961,0.000000,0.000000}%
\pgfsetstrokecolor{currentstroke}%
\pgfsetstrokeopacity{0.400000}%
\pgfsetdash{}{0pt}%
\pgfpathmoveto{\pgfqpoint{3.253414in}{1.158981in}}%
\pgfpathcurveto{\pgfqpoint{3.264464in}{1.158981in}}{\pgfqpoint{3.275063in}{1.163371in}}{\pgfqpoint{3.282877in}{1.171185in}}%
\pgfpathcurveto{\pgfqpoint{3.290691in}{1.178998in}}{\pgfqpoint{3.295081in}{1.189598in}}{\pgfqpoint{3.295081in}{1.200648in}}%
\pgfpathcurveto{\pgfqpoint{3.295081in}{1.211698in}}{\pgfqpoint{3.290691in}{1.222297in}}{\pgfqpoint{3.282877in}{1.230110in}}%
\pgfpathcurveto{\pgfqpoint{3.275063in}{1.237924in}}{\pgfqpoint{3.264464in}{1.242314in}}{\pgfqpoint{3.253414in}{1.242314in}}%
\pgfpathcurveto{\pgfqpoint{3.242364in}{1.242314in}}{\pgfqpoint{3.231765in}{1.237924in}}{\pgfqpoint{3.223951in}{1.230110in}}%
\pgfpathcurveto{\pgfqpoint{3.216138in}{1.222297in}}{\pgfqpoint{3.211747in}{1.211698in}}{\pgfqpoint{3.211747in}{1.200648in}}%
\pgfpathcurveto{\pgfqpoint{3.211747in}{1.189598in}}{\pgfqpoint{3.216138in}{1.178998in}}{\pgfqpoint{3.223951in}{1.171185in}}%
\pgfpathcurveto{\pgfqpoint{3.231765in}{1.163371in}}{\pgfqpoint{3.242364in}{1.158981in}}{\pgfqpoint{3.253414in}{1.158981in}}%
\pgfpathclose%
\pgfusepath{stroke,fill}%
\end{pgfscope}%
\begin{pgfscope}%
\pgfpathrectangle{\pgfqpoint{0.511823in}{0.504323in}}{\pgfqpoint{3.218177in}{3.225677in}} %
\pgfusepath{clip}%
\pgfsetbuttcap%
\pgfsetroundjoin%
\definecolor{currentfill}{rgb}{0.501961,0.000000,0.000000}%
\pgfsetfillcolor{currentfill}%
\pgfsetfillopacity{0.400000}%
\pgfsetlinewidth{0.501875pt}%
\definecolor{currentstroke}{rgb}{0.501961,0.000000,0.000000}%
\pgfsetstrokecolor{currentstroke}%
\pgfsetstrokeopacity{0.400000}%
\pgfsetdash{}{0pt}%
\pgfpathmoveto{\pgfqpoint{3.350975in}{1.186893in}}%
\pgfpathcurveto{\pgfqpoint{3.362025in}{1.186893in}}{\pgfqpoint{3.372624in}{1.191284in}}{\pgfqpoint{3.380438in}{1.199097in}}%
\pgfpathcurveto{\pgfqpoint{3.388252in}{1.206911in}}{\pgfqpoint{3.392642in}{1.217510in}}{\pgfqpoint{3.392642in}{1.228560in}}%
\pgfpathcurveto{\pgfqpoint{3.392642in}{1.239610in}}{\pgfqpoint{3.388252in}{1.250209in}}{\pgfqpoint{3.380438in}{1.258023in}}%
\pgfpathcurveto{\pgfqpoint{3.372624in}{1.265836in}}{\pgfqpoint{3.362025in}{1.270227in}}{\pgfqpoint{3.350975in}{1.270227in}}%
\pgfpathcurveto{\pgfqpoint{3.339925in}{1.270227in}}{\pgfqpoint{3.329326in}{1.265836in}}{\pgfqpoint{3.321512in}{1.258023in}}%
\pgfpathcurveto{\pgfqpoint{3.313699in}{1.250209in}}{\pgfqpoint{3.309308in}{1.239610in}}{\pgfqpoint{3.309308in}{1.228560in}}%
\pgfpathcurveto{\pgfqpoint{3.309308in}{1.217510in}}{\pgfqpoint{3.313699in}{1.206911in}}{\pgfqpoint{3.321512in}{1.199097in}}%
\pgfpathcurveto{\pgfqpoint{3.329326in}{1.191284in}}{\pgfqpoint{3.339925in}{1.186893in}}{\pgfqpoint{3.350975in}{1.186893in}}%
\pgfpathclose%
\pgfusepath{stroke,fill}%
\end{pgfscope}%
\begin{pgfscope}%
\pgfpathrectangle{\pgfqpoint{0.511823in}{0.504323in}}{\pgfqpoint{3.218177in}{3.225677in}} %
\pgfusepath{clip}%
\pgfsetbuttcap%
\pgfsetroundjoin%
\definecolor{currentfill}{rgb}{0.501961,0.000000,0.000000}%
\pgfsetfillcolor{currentfill}%
\pgfsetfillopacity{0.400000}%
\pgfsetlinewidth{0.501875pt}%
\definecolor{currentstroke}{rgb}{0.501961,0.000000,0.000000}%
\pgfsetstrokecolor{currentstroke}%
\pgfsetstrokeopacity{0.400000}%
\pgfsetdash{}{0pt}%
\pgfpathmoveto{\pgfqpoint{3.265067in}{1.177179in}}%
\pgfpathcurveto{\pgfqpoint{3.276117in}{1.177179in}}{\pgfqpoint{3.286716in}{1.181569in}}{\pgfqpoint{3.294530in}{1.189383in}}%
\pgfpathcurveto{\pgfqpoint{3.302343in}{1.197197in}}{\pgfqpoint{3.306733in}{1.207796in}}{\pgfqpoint{3.306733in}{1.218846in}}%
\pgfpathcurveto{\pgfqpoint{3.306733in}{1.229896in}}{\pgfqpoint{3.302343in}{1.240495in}}{\pgfqpoint{3.294530in}{1.248309in}}%
\pgfpathcurveto{\pgfqpoint{3.286716in}{1.256122in}}{\pgfqpoint{3.276117in}{1.260512in}}{\pgfqpoint{3.265067in}{1.260512in}}%
\pgfpathcurveto{\pgfqpoint{3.254017in}{1.260512in}}{\pgfqpoint{3.243418in}{1.256122in}}{\pgfqpoint{3.235604in}{1.248309in}}%
\pgfpathcurveto{\pgfqpoint{3.227790in}{1.240495in}}{\pgfqpoint{3.223400in}{1.229896in}}{\pgfqpoint{3.223400in}{1.218846in}}%
\pgfpathcurveto{\pgfqpoint{3.223400in}{1.207796in}}{\pgfqpoint{3.227790in}{1.197197in}}{\pgfqpoint{3.235604in}{1.189383in}}%
\pgfpathcurveto{\pgfqpoint{3.243418in}{1.181569in}}{\pgfqpoint{3.254017in}{1.177179in}}{\pgfqpoint{3.265067in}{1.177179in}}%
\pgfpathclose%
\pgfusepath{stroke,fill}%
\end{pgfscope}%
\begin{pgfscope}%
\pgfpathrectangle{\pgfqpoint{0.511823in}{0.504323in}}{\pgfqpoint{3.218177in}{3.225677in}} %
\pgfusepath{clip}%
\pgfsetbuttcap%
\pgfsetroundjoin%
\definecolor{currentfill}{rgb}{0.501961,0.000000,0.000000}%
\pgfsetfillcolor{currentfill}%
\pgfsetfillopacity{0.400000}%
\pgfsetlinewidth{0.501875pt}%
\definecolor{currentstroke}{rgb}{0.501961,0.000000,0.000000}%
\pgfsetstrokecolor{currentstroke}%
\pgfsetstrokeopacity{0.400000}%
\pgfsetdash{}{0pt}%
\pgfpathmoveto{\pgfqpoint{3.173679in}{1.165789in}}%
\pgfpathcurveto{\pgfqpoint{3.184729in}{1.165789in}}{\pgfqpoint{3.195328in}{1.170179in}}{\pgfqpoint{3.203142in}{1.177993in}}%
\pgfpathcurveto{\pgfqpoint{3.210956in}{1.185806in}}{\pgfqpoint{3.215346in}{1.196405in}}{\pgfqpoint{3.215346in}{1.207455in}}%
\pgfpathcurveto{\pgfqpoint{3.215346in}{1.218506in}}{\pgfqpoint{3.210956in}{1.229105in}}{\pgfqpoint{3.203142in}{1.236918in}}%
\pgfpathcurveto{\pgfqpoint{3.195328in}{1.244732in}}{\pgfqpoint{3.184729in}{1.249122in}}{\pgfqpoint{3.173679in}{1.249122in}}%
\pgfpathcurveto{\pgfqpoint{3.162629in}{1.249122in}}{\pgfqpoint{3.152030in}{1.244732in}}{\pgfqpoint{3.144216in}{1.236918in}}%
\pgfpathcurveto{\pgfqpoint{3.136403in}{1.229105in}}{\pgfqpoint{3.132013in}{1.218506in}}{\pgfqpoint{3.132013in}{1.207455in}}%
\pgfpathcurveto{\pgfqpoint{3.132013in}{1.196405in}}{\pgfqpoint{3.136403in}{1.185806in}}{\pgfqpoint{3.144216in}{1.177993in}}%
\pgfpathcurveto{\pgfqpoint{3.152030in}{1.170179in}}{\pgfqpoint{3.162629in}{1.165789in}}{\pgfqpoint{3.173679in}{1.165789in}}%
\pgfpathclose%
\pgfusepath{stroke,fill}%
\end{pgfscope}%
\begin{pgfscope}%
\pgfpathrectangle{\pgfqpoint{0.511823in}{0.504323in}}{\pgfqpoint{3.218177in}{3.225677in}} %
\pgfusepath{clip}%
\pgfsetbuttcap%
\pgfsetroundjoin%
\definecolor{currentfill}{rgb}{0.501961,0.000000,0.000000}%
\pgfsetfillcolor{currentfill}%
\pgfsetfillopacity{0.400000}%
\pgfsetlinewidth{0.501875pt}%
\definecolor{currentstroke}{rgb}{0.501961,0.000000,0.000000}%
\pgfsetstrokecolor{currentstroke}%
\pgfsetstrokeopacity{0.400000}%
\pgfsetdash{}{0pt}%
\pgfpathmoveto{\pgfqpoint{3.441737in}{1.230974in}}%
\pgfpathcurveto{\pgfqpoint{3.452787in}{1.230974in}}{\pgfqpoint{3.463386in}{1.235364in}}{\pgfqpoint{3.471200in}{1.243178in}}%
\pgfpathcurveto{\pgfqpoint{3.479013in}{1.250991in}}{\pgfqpoint{3.483404in}{1.261590in}}{\pgfqpoint{3.483404in}{1.272640in}}%
\pgfpathcurveto{\pgfqpoint{3.483404in}{1.283691in}}{\pgfqpoint{3.479013in}{1.294290in}}{\pgfqpoint{3.471200in}{1.302103in}}%
\pgfpathcurveto{\pgfqpoint{3.463386in}{1.309917in}}{\pgfqpoint{3.452787in}{1.314307in}}{\pgfqpoint{3.441737in}{1.314307in}}%
\pgfpathcurveto{\pgfqpoint{3.430687in}{1.314307in}}{\pgfqpoint{3.420088in}{1.309917in}}{\pgfqpoint{3.412274in}{1.302103in}}%
\pgfpathcurveto{\pgfqpoint{3.404461in}{1.294290in}}{\pgfqpoint{3.400070in}{1.283691in}}{\pgfqpoint{3.400070in}{1.272640in}}%
\pgfpathcurveto{\pgfqpoint{3.400070in}{1.261590in}}{\pgfqpoint{3.404461in}{1.250991in}}{\pgfqpoint{3.412274in}{1.243178in}}%
\pgfpathcurveto{\pgfqpoint{3.420088in}{1.235364in}}{\pgfqpoint{3.430687in}{1.230974in}}{\pgfqpoint{3.441737in}{1.230974in}}%
\pgfpathclose%
\pgfusepath{stroke,fill}%
\end{pgfscope}%
\begin{pgfscope}%
\pgfpathrectangle{\pgfqpoint{0.511823in}{0.504323in}}{\pgfqpoint{3.218177in}{3.225677in}} %
\pgfusepath{clip}%
\pgfsetbuttcap%
\pgfsetroundjoin%
\definecolor{currentfill}{rgb}{0.501961,0.000000,0.000000}%
\pgfsetfillcolor{currentfill}%
\pgfsetfillopacity{0.400000}%
\pgfsetlinewidth{0.501875pt}%
\definecolor{currentstroke}{rgb}{0.501961,0.000000,0.000000}%
\pgfsetstrokecolor{currentstroke}%
\pgfsetstrokeopacity{0.400000}%
\pgfsetdash{}{0pt}%
\pgfpathmoveto{\pgfqpoint{3.320482in}{1.213087in}}%
\pgfpathcurveto{\pgfqpoint{3.331532in}{1.213087in}}{\pgfqpoint{3.342131in}{1.217477in}}{\pgfqpoint{3.349944in}{1.225291in}}%
\pgfpathcurveto{\pgfqpoint{3.357758in}{1.233104in}}{\pgfqpoint{3.362148in}{1.243703in}}{\pgfqpoint{3.362148in}{1.254754in}}%
\pgfpathcurveto{\pgfqpoint{3.362148in}{1.265804in}}{\pgfqpoint{3.357758in}{1.276403in}}{\pgfqpoint{3.349944in}{1.284216in}}%
\pgfpathcurveto{\pgfqpoint{3.342131in}{1.292030in}}{\pgfqpoint{3.331532in}{1.296420in}}{\pgfqpoint{3.320482in}{1.296420in}}%
\pgfpathcurveto{\pgfqpoint{3.309431in}{1.296420in}}{\pgfqpoint{3.298832in}{1.292030in}}{\pgfqpoint{3.291019in}{1.284216in}}%
\pgfpathcurveto{\pgfqpoint{3.283205in}{1.276403in}}{\pgfqpoint{3.278815in}{1.265804in}}{\pgfqpoint{3.278815in}{1.254754in}}%
\pgfpathcurveto{\pgfqpoint{3.278815in}{1.243703in}}{\pgfqpoint{3.283205in}{1.233104in}}{\pgfqpoint{3.291019in}{1.225291in}}%
\pgfpathcurveto{\pgfqpoint{3.298832in}{1.217477in}}{\pgfqpoint{3.309431in}{1.213087in}}{\pgfqpoint{3.320482in}{1.213087in}}%
\pgfpathclose%
\pgfusepath{stroke,fill}%
\end{pgfscope}%
\begin{pgfscope}%
\pgfpathrectangle{\pgfqpoint{0.511823in}{0.504323in}}{\pgfqpoint{3.218177in}{3.225677in}} %
\pgfusepath{clip}%
\pgfsetbuttcap%
\pgfsetroundjoin%
\definecolor{currentfill}{rgb}{0.501961,0.000000,0.000000}%
\pgfsetfillcolor{currentfill}%
\pgfsetfillopacity{0.400000}%
\pgfsetlinewidth{0.501875pt}%
\definecolor{currentstroke}{rgb}{0.501961,0.000000,0.000000}%
\pgfsetstrokecolor{currentstroke}%
\pgfsetstrokeopacity{0.400000}%
\pgfsetdash{}{0pt}%
\pgfpathmoveto{\pgfqpoint{3.312043in}{1.219367in}}%
\pgfpathcurveto{\pgfqpoint{3.323093in}{1.219367in}}{\pgfqpoint{3.333692in}{1.223757in}}{\pgfqpoint{3.341505in}{1.231571in}}%
\pgfpathcurveto{\pgfqpoint{3.349319in}{1.239385in}}{\pgfqpoint{3.353709in}{1.249984in}}{\pgfqpoint{3.353709in}{1.261034in}}%
\pgfpathcurveto{\pgfqpoint{3.353709in}{1.272084in}}{\pgfqpoint{3.349319in}{1.282683in}}{\pgfqpoint{3.341505in}{1.290497in}}%
\pgfpathcurveto{\pgfqpoint{3.333692in}{1.298310in}}{\pgfqpoint{3.323093in}{1.302701in}}{\pgfqpoint{3.312043in}{1.302701in}}%
\pgfpathcurveto{\pgfqpoint{3.300992in}{1.302701in}}{\pgfqpoint{3.290393in}{1.298310in}}{\pgfqpoint{3.282580in}{1.290497in}}%
\pgfpathcurveto{\pgfqpoint{3.274766in}{1.282683in}}{\pgfqpoint{3.270376in}{1.272084in}}{\pgfqpoint{3.270376in}{1.261034in}}%
\pgfpathcurveto{\pgfqpoint{3.270376in}{1.249984in}}{\pgfqpoint{3.274766in}{1.239385in}}{\pgfqpoint{3.282580in}{1.231571in}}%
\pgfpathcurveto{\pgfqpoint{3.290393in}{1.223757in}}{\pgfqpoint{3.300992in}{1.219367in}}{\pgfqpoint{3.312043in}{1.219367in}}%
\pgfpathclose%
\pgfusepath{stroke,fill}%
\end{pgfscope}%
\begin{pgfscope}%
\pgfpathrectangle{\pgfqpoint{0.511823in}{0.504323in}}{\pgfqpoint{3.218177in}{3.225677in}} %
\pgfusepath{clip}%
\pgfsetbuttcap%
\pgfsetroundjoin%
\definecolor{currentfill}{rgb}{0.501961,0.000000,0.000000}%
\pgfsetfillcolor{currentfill}%
\pgfsetfillopacity{0.400000}%
\pgfsetlinewidth{0.501875pt}%
\definecolor{currentstroke}{rgb}{0.501961,0.000000,0.000000}%
\pgfsetstrokecolor{currentstroke}%
\pgfsetstrokeopacity{0.400000}%
\pgfsetdash{}{0pt}%
\pgfpathmoveto{\pgfqpoint{2.972576in}{1.151508in}}%
\pgfpathcurveto{\pgfqpoint{2.983627in}{1.151508in}}{\pgfqpoint{2.994226in}{1.155898in}}{\pgfqpoint{3.002039in}{1.163712in}}%
\pgfpathcurveto{\pgfqpoint{3.009853in}{1.171525in}}{\pgfqpoint{3.014243in}{1.182124in}}{\pgfqpoint{3.014243in}{1.193174in}}%
\pgfpathcurveto{\pgfqpoint{3.014243in}{1.204225in}}{\pgfqpoint{3.009853in}{1.214824in}}{\pgfqpoint{3.002039in}{1.222637in}}%
\pgfpathcurveto{\pgfqpoint{2.994226in}{1.230451in}}{\pgfqpoint{2.983627in}{1.234841in}}{\pgfqpoint{2.972576in}{1.234841in}}%
\pgfpathcurveto{\pgfqpoint{2.961526in}{1.234841in}}{\pgfqpoint{2.950927in}{1.230451in}}{\pgfqpoint{2.943114in}{1.222637in}}%
\pgfpathcurveto{\pgfqpoint{2.935300in}{1.214824in}}{\pgfqpoint{2.930910in}{1.204225in}}{\pgfqpoint{2.930910in}{1.193174in}}%
\pgfpathcurveto{\pgfqpoint{2.930910in}{1.182124in}}{\pgfqpoint{2.935300in}{1.171525in}}{\pgfqpoint{2.943114in}{1.163712in}}%
\pgfpathcurveto{\pgfqpoint{2.950927in}{1.155898in}}{\pgfqpoint{2.961526in}{1.151508in}}{\pgfqpoint{2.972576in}{1.151508in}}%
\pgfpathclose%
\pgfusepath{stroke,fill}%
\end{pgfscope}%
\begin{pgfscope}%
\pgfpathrectangle{\pgfqpoint{0.511823in}{0.504323in}}{\pgfqpoint{3.218177in}{3.225677in}} %
\pgfusepath{clip}%
\pgfsetbuttcap%
\pgfsetroundjoin%
\definecolor{currentfill}{rgb}{0.501961,0.000000,0.000000}%
\pgfsetfillcolor{currentfill}%
\pgfsetfillopacity{0.400000}%
\pgfsetlinewidth{0.501875pt}%
\definecolor{currentstroke}{rgb}{0.501961,0.000000,0.000000}%
\pgfsetstrokecolor{currentstroke}%
\pgfsetstrokeopacity{0.400000}%
\pgfsetdash{}{0pt}%
\pgfpathmoveto{\pgfqpoint{3.301140in}{1.233164in}}%
\pgfpathcurveto{\pgfqpoint{3.312190in}{1.233164in}}{\pgfqpoint{3.322789in}{1.237554in}}{\pgfqpoint{3.330603in}{1.245368in}}%
\pgfpathcurveto{\pgfqpoint{3.338416in}{1.253181in}}{\pgfqpoint{3.342807in}{1.263780in}}{\pgfqpoint{3.342807in}{1.274830in}}%
\pgfpathcurveto{\pgfqpoint{3.342807in}{1.285881in}}{\pgfqpoint{3.338416in}{1.296480in}}{\pgfqpoint{3.330603in}{1.304293in}}%
\pgfpathcurveto{\pgfqpoint{3.322789in}{1.312107in}}{\pgfqpoint{3.312190in}{1.316497in}}{\pgfqpoint{3.301140in}{1.316497in}}%
\pgfpathcurveto{\pgfqpoint{3.290090in}{1.316497in}}{\pgfqpoint{3.279491in}{1.312107in}}{\pgfqpoint{3.271677in}{1.304293in}}%
\pgfpathcurveto{\pgfqpoint{3.263864in}{1.296480in}}{\pgfqpoint{3.259473in}{1.285881in}}{\pgfqpoint{3.259473in}{1.274830in}}%
\pgfpathcurveto{\pgfqpoint{3.259473in}{1.263780in}}{\pgfqpoint{3.263864in}{1.253181in}}{\pgfqpoint{3.271677in}{1.245368in}}%
\pgfpathcurveto{\pgfqpoint{3.279491in}{1.237554in}}{\pgfqpoint{3.290090in}{1.233164in}}{\pgfqpoint{3.301140in}{1.233164in}}%
\pgfpathclose%
\pgfusepath{stroke,fill}%
\end{pgfscope}%
\begin{pgfscope}%
\pgfpathrectangle{\pgfqpoint{0.511823in}{0.504323in}}{\pgfqpoint{3.218177in}{3.225677in}} %
\pgfusepath{clip}%
\pgfsetbuttcap%
\pgfsetroundjoin%
\definecolor{currentfill}{rgb}{0.501961,0.000000,0.000000}%
\pgfsetfillcolor{currentfill}%
\pgfsetfillopacity{0.400000}%
\pgfsetlinewidth{0.501875pt}%
\definecolor{currentstroke}{rgb}{0.501961,0.000000,0.000000}%
\pgfsetstrokecolor{currentstroke}%
\pgfsetstrokeopacity{0.400000}%
\pgfsetdash{}{0pt}%
\pgfpathmoveto{\pgfqpoint{3.370261in}{1.257186in}}%
\pgfpathcurveto{\pgfqpoint{3.381312in}{1.257186in}}{\pgfqpoint{3.391911in}{1.261576in}}{\pgfqpoint{3.399724in}{1.269390in}}%
\pgfpathcurveto{\pgfqpoint{3.407538in}{1.277203in}}{\pgfqpoint{3.411928in}{1.287802in}}{\pgfqpoint{3.411928in}{1.298852in}}%
\pgfpathcurveto{\pgfqpoint{3.411928in}{1.309903in}}{\pgfqpoint{3.407538in}{1.320502in}}{\pgfqpoint{3.399724in}{1.328315in}}%
\pgfpathcurveto{\pgfqpoint{3.391911in}{1.336129in}}{\pgfqpoint{3.381312in}{1.340519in}}{\pgfqpoint{3.370261in}{1.340519in}}%
\pgfpathcurveto{\pgfqpoint{3.359211in}{1.340519in}}{\pgfqpoint{3.348612in}{1.336129in}}{\pgfqpoint{3.340799in}{1.328315in}}%
\pgfpathcurveto{\pgfqpoint{3.332985in}{1.320502in}}{\pgfqpoint{3.328595in}{1.309903in}}{\pgfqpoint{3.328595in}{1.298852in}}%
\pgfpathcurveto{\pgfqpoint{3.328595in}{1.287802in}}{\pgfqpoint{3.332985in}{1.277203in}}{\pgfqpoint{3.340799in}{1.269390in}}%
\pgfpathcurveto{\pgfqpoint{3.348612in}{1.261576in}}{\pgfqpoint{3.359211in}{1.257186in}}{\pgfqpoint{3.370261in}{1.257186in}}%
\pgfpathclose%
\pgfusepath{stroke,fill}%
\end{pgfscope}%
\begin{pgfscope}%
\pgfpathrectangle{\pgfqpoint{0.511823in}{0.504323in}}{\pgfqpoint{3.218177in}{3.225677in}} %
\pgfusepath{clip}%
\pgfsetbuttcap%
\pgfsetroundjoin%
\definecolor{currentfill}{rgb}{0.501961,0.000000,0.000000}%
\pgfsetfillcolor{currentfill}%
\pgfsetfillopacity{0.400000}%
\pgfsetlinewidth{0.501875pt}%
\definecolor{currentstroke}{rgb}{0.501961,0.000000,0.000000}%
\pgfsetstrokecolor{currentstroke}%
\pgfsetstrokeopacity{0.400000}%
\pgfsetdash{}{0pt}%
\pgfpathmoveto{\pgfqpoint{3.296538in}{1.248344in}}%
\pgfpathcurveto{\pgfqpoint{3.307588in}{1.248344in}}{\pgfqpoint{3.318187in}{1.252734in}}{\pgfqpoint{3.326001in}{1.260548in}}%
\pgfpathcurveto{\pgfqpoint{3.333814in}{1.268361in}}{\pgfqpoint{3.338205in}{1.278960in}}{\pgfqpoint{3.338205in}{1.290011in}}%
\pgfpathcurveto{\pgfqpoint{3.338205in}{1.301061in}}{\pgfqpoint{3.333814in}{1.311660in}}{\pgfqpoint{3.326001in}{1.319473in}}%
\pgfpathcurveto{\pgfqpoint{3.318187in}{1.327287in}}{\pgfqpoint{3.307588in}{1.331677in}}{\pgfqpoint{3.296538in}{1.331677in}}%
\pgfpathcurveto{\pgfqpoint{3.285488in}{1.331677in}}{\pgfqpoint{3.274889in}{1.327287in}}{\pgfqpoint{3.267075in}{1.319473in}}%
\pgfpathcurveto{\pgfqpoint{3.259261in}{1.311660in}}{\pgfqpoint{3.254871in}{1.301061in}}{\pgfqpoint{3.254871in}{1.290011in}}%
\pgfpathcurveto{\pgfqpoint{3.254871in}{1.278960in}}{\pgfqpoint{3.259261in}{1.268361in}}{\pgfqpoint{3.267075in}{1.260548in}}%
\pgfpathcurveto{\pgfqpoint{3.274889in}{1.252734in}}{\pgfqpoint{3.285488in}{1.248344in}}{\pgfqpoint{3.296538in}{1.248344in}}%
\pgfpathclose%
\pgfusepath{stroke,fill}%
\end{pgfscope}%
\begin{pgfscope}%
\pgfpathrectangle{\pgfqpoint{0.511823in}{0.504323in}}{\pgfqpoint{3.218177in}{3.225677in}} %
\pgfusepath{clip}%
\pgfsetbuttcap%
\pgfsetroundjoin%
\definecolor{currentfill}{rgb}{0.501961,0.000000,0.000000}%
\pgfsetfillcolor{currentfill}%
\pgfsetfillopacity{0.400000}%
\pgfsetlinewidth{0.501875pt}%
\definecolor{currentstroke}{rgb}{0.501961,0.000000,0.000000}%
\pgfsetstrokecolor{currentstroke}%
\pgfsetstrokeopacity{0.400000}%
\pgfsetdash{}{0pt}%
\pgfpathmoveto{\pgfqpoint{3.238992in}{1.242878in}}%
\pgfpathcurveto{\pgfqpoint{3.250042in}{1.242878in}}{\pgfqpoint{3.260641in}{1.247269in}}{\pgfqpoint{3.268455in}{1.255082in}}%
\pgfpathcurveto{\pgfqpoint{3.276269in}{1.262896in}}{\pgfqpoint{3.280659in}{1.273495in}}{\pgfqpoint{3.280659in}{1.284545in}}%
\pgfpathcurveto{\pgfqpoint{3.280659in}{1.295595in}}{\pgfqpoint{3.276269in}{1.306194in}}{\pgfqpoint{3.268455in}{1.314008in}}%
\pgfpathcurveto{\pgfqpoint{3.260641in}{1.321822in}}{\pgfqpoint{3.250042in}{1.326212in}}{\pgfqpoint{3.238992in}{1.326212in}}%
\pgfpathcurveto{\pgfqpoint{3.227942in}{1.326212in}}{\pgfqpoint{3.217343in}{1.321822in}}{\pgfqpoint{3.209529in}{1.314008in}}%
\pgfpathcurveto{\pgfqpoint{3.201716in}{1.306194in}}{\pgfqpoint{3.197326in}{1.295595in}}{\pgfqpoint{3.197326in}{1.284545in}}%
\pgfpathcurveto{\pgfqpoint{3.197326in}{1.273495in}}{\pgfqpoint{3.201716in}{1.262896in}}{\pgfqpoint{3.209529in}{1.255082in}}%
\pgfpathcurveto{\pgfqpoint{3.217343in}{1.247269in}}{\pgfqpoint{3.227942in}{1.242878in}}{\pgfqpoint{3.238992in}{1.242878in}}%
\pgfpathclose%
\pgfusepath{stroke,fill}%
\end{pgfscope}%
\begin{pgfscope}%
\pgfpathrectangle{\pgfqpoint{0.511823in}{0.504323in}}{\pgfqpoint{3.218177in}{3.225677in}} %
\pgfusepath{clip}%
\pgfsetbuttcap%
\pgfsetroundjoin%
\definecolor{currentfill}{rgb}{0.501961,0.000000,0.000000}%
\pgfsetfillcolor{currentfill}%
\pgfsetfillopacity{0.400000}%
\pgfsetlinewidth{0.501875pt}%
\definecolor{currentstroke}{rgb}{0.501961,0.000000,0.000000}%
\pgfsetstrokecolor{currentstroke}%
\pgfsetstrokeopacity{0.400000}%
\pgfsetdash{}{0pt}%
\pgfpathmoveto{\pgfqpoint{3.409265in}{1.291607in}}%
\pgfpathcurveto{\pgfqpoint{3.420315in}{1.291607in}}{\pgfqpoint{3.430914in}{1.295997in}}{\pgfqpoint{3.438728in}{1.303811in}}%
\pgfpathcurveto{\pgfqpoint{3.446542in}{1.311624in}}{\pgfqpoint{3.450932in}{1.322223in}}{\pgfqpoint{3.450932in}{1.333273in}}%
\pgfpathcurveto{\pgfqpoint{3.450932in}{1.344324in}}{\pgfqpoint{3.446542in}{1.354923in}}{\pgfqpoint{3.438728in}{1.362736in}}%
\pgfpathcurveto{\pgfqpoint{3.430914in}{1.370550in}}{\pgfqpoint{3.420315in}{1.374940in}}{\pgfqpoint{3.409265in}{1.374940in}}%
\pgfpathcurveto{\pgfqpoint{3.398215in}{1.374940in}}{\pgfqpoint{3.387616in}{1.370550in}}{\pgfqpoint{3.379802in}{1.362736in}}%
\pgfpathcurveto{\pgfqpoint{3.371989in}{1.354923in}}{\pgfqpoint{3.367598in}{1.344324in}}{\pgfqpoint{3.367598in}{1.333273in}}%
\pgfpathcurveto{\pgfqpoint{3.367598in}{1.322223in}}{\pgfqpoint{3.371989in}{1.311624in}}{\pgfqpoint{3.379802in}{1.303811in}}%
\pgfpathcurveto{\pgfqpoint{3.387616in}{1.295997in}}{\pgfqpoint{3.398215in}{1.291607in}}{\pgfqpoint{3.409265in}{1.291607in}}%
\pgfpathclose%
\pgfusepath{stroke,fill}%
\end{pgfscope}%
\begin{pgfscope}%
\pgfpathrectangle{\pgfqpoint{0.511823in}{0.504323in}}{\pgfqpoint{3.218177in}{3.225677in}} %
\pgfusepath{clip}%
\pgfsetbuttcap%
\pgfsetroundjoin%
\definecolor{currentfill}{rgb}{0.501961,0.000000,0.000000}%
\pgfsetfillcolor{currentfill}%
\pgfsetfillopacity{0.400000}%
\pgfsetlinewidth{0.501875pt}%
\definecolor{currentstroke}{rgb}{0.501961,0.000000,0.000000}%
\pgfsetstrokecolor{currentstroke}%
\pgfsetstrokeopacity{0.400000}%
\pgfsetdash{}{0pt}%
\pgfpathmoveto{\pgfqpoint{3.285510in}{1.270103in}}%
\pgfpathcurveto{\pgfqpoint{3.296560in}{1.270103in}}{\pgfqpoint{3.307159in}{1.274493in}}{\pgfqpoint{3.314972in}{1.282307in}}%
\pgfpathcurveto{\pgfqpoint{3.322786in}{1.290120in}}{\pgfqpoint{3.327176in}{1.300719in}}{\pgfqpoint{3.327176in}{1.311769in}}%
\pgfpathcurveto{\pgfqpoint{3.327176in}{1.322820in}}{\pgfqpoint{3.322786in}{1.333419in}}{\pgfqpoint{3.314972in}{1.341232in}}%
\pgfpathcurveto{\pgfqpoint{3.307159in}{1.349046in}}{\pgfqpoint{3.296560in}{1.353436in}}{\pgfqpoint{3.285510in}{1.353436in}}%
\pgfpathcurveto{\pgfqpoint{3.274459in}{1.353436in}}{\pgfqpoint{3.263860in}{1.349046in}}{\pgfqpoint{3.256047in}{1.341232in}}%
\pgfpathcurveto{\pgfqpoint{3.248233in}{1.333419in}}{\pgfqpoint{3.243843in}{1.322820in}}{\pgfqpoint{3.243843in}{1.311769in}}%
\pgfpathcurveto{\pgfqpoint{3.243843in}{1.300719in}}{\pgfqpoint{3.248233in}{1.290120in}}{\pgfqpoint{3.256047in}{1.282307in}}%
\pgfpathcurveto{\pgfqpoint{3.263860in}{1.274493in}}{\pgfqpoint{3.274459in}{1.270103in}}{\pgfqpoint{3.285510in}{1.270103in}}%
\pgfpathclose%
\pgfusepath{stroke,fill}%
\end{pgfscope}%
\begin{pgfscope}%
\pgfpathrectangle{\pgfqpoint{0.511823in}{0.504323in}}{\pgfqpoint{3.218177in}{3.225677in}} %
\pgfusepath{clip}%
\pgfsetbuttcap%
\pgfsetroundjoin%
\definecolor{currentfill}{rgb}{0.501961,0.000000,0.000000}%
\pgfsetfillcolor{currentfill}%
\pgfsetfillopacity{0.400000}%
\pgfsetlinewidth{0.501875pt}%
\definecolor{currentstroke}{rgb}{0.501961,0.000000,0.000000}%
\pgfsetstrokecolor{currentstroke}%
\pgfsetstrokeopacity{0.400000}%
\pgfsetdash{}{0pt}%
\pgfpathmoveto{\pgfqpoint{3.326269in}{1.288252in}}%
\pgfpathcurveto{\pgfqpoint{3.337319in}{1.288252in}}{\pgfqpoint{3.347918in}{1.292642in}}{\pgfqpoint{3.355732in}{1.300456in}}%
\pgfpathcurveto{\pgfqpoint{3.363545in}{1.308270in}}{\pgfqpoint{3.367936in}{1.318869in}}{\pgfqpoint{3.367936in}{1.329919in}}%
\pgfpathcurveto{\pgfqpoint{3.367936in}{1.340969in}}{\pgfqpoint{3.363545in}{1.351568in}}{\pgfqpoint{3.355732in}{1.359382in}}%
\pgfpathcurveto{\pgfqpoint{3.347918in}{1.367195in}}{\pgfqpoint{3.337319in}{1.371585in}}{\pgfqpoint{3.326269in}{1.371585in}}%
\pgfpathcurveto{\pgfqpoint{3.315219in}{1.371585in}}{\pgfqpoint{3.304620in}{1.367195in}}{\pgfqpoint{3.296806in}{1.359382in}}%
\pgfpathcurveto{\pgfqpoint{3.288993in}{1.351568in}}{\pgfqpoint{3.284602in}{1.340969in}}{\pgfqpoint{3.284602in}{1.329919in}}%
\pgfpathcurveto{\pgfqpoint{3.284602in}{1.318869in}}{\pgfqpoint{3.288993in}{1.308270in}}{\pgfqpoint{3.296806in}{1.300456in}}%
\pgfpathcurveto{\pgfqpoint{3.304620in}{1.292642in}}{\pgfqpoint{3.315219in}{1.288252in}}{\pgfqpoint{3.326269in}{1.288252in}}%
\pgfpathclose%
\pgfusepath{stroke,fill}%
\end{pgfscope}%
\begin{pgfscope}%
\pgfpathrectangle{\pgfqpoint{0.511823in}{0.504323in}}{\pgfqpoint{3.218177in}{3.225677in}} %
\pgfusepath{clip}%
\pgfsetbuttcap%
\pgfsetroundjoin%
\definecolor{currentfill}{rgb}{0.501961,0.000000,0.000000}%
\pgfsetfillcolor{currentfill}%
\pgfsetfillopacity{0.400000}%
\pgfsetlinewidth{0.501875pt}%
\definecolor{currentstroke}{rgb}{0.501961,0.000000,0.000000}%
\pgfsetstrokecolor{currentstroke}%
\pgfsetstrokeopacity{0.400000}%
\pgfsetdash{}{0pt}%
\pgfpathmoveto{\pgfqpoint{3.382512in}{1.310523in}}%
\pgfpathcurveto{\pgfqpoint{3.393562in}{1.310523in}}{\pgfqpoint{3.404161in}{1.314913in}}{\pgfqpoint{3.411975in}{1.322727in}}%
\pgfpathcurveto{\pgfqpoint{3.419788in}{1.330540in}}{\pgfqpoint{3.424178in}{1.341139in}}{\pgfqpoint{3.424178in}{1.352189in}}%
\pgfpathcurveto{\pgfqpoint{3.424178in}{1.363239in}}{\pgfqpoint{3.419788in}{1.373839in}}{\pgfqpoint{3.411975in}{1.381652in}}%
\pgfpathcurveto{\pgfqpoint{3.404161in}{1.389466in}}{\pgfqpoint{3.393562in}{1.393856in}}{\pgfqpoint{3.382512in}{1.393856in}}%
\pgfpathcurveto{\pgfqpoint{3.371462in}{1.393856in}}{\pgfqpoint{3.360863in}{1.389466in}}{\pgfqpoint{3.353049in}{1.381652in}}%
\pgfpathcurveto{\pgfqpoint{3.345235in}{1.373839in}}{\pgfqpoint{3.340845in}{1.363239in}}{\pgfqpoint{3.340845in}{1.352189in}}%
\pgfpathcurveto{\pgfqpoint{3.340845in}{1.341139in}}{\pgfqpoint{3.345235in}{1.330540in}}{\pgfqpoint{3.353049in}{1.322727in}}%
\pgfpathcurveto{\pgfqpoint{3.360863in}{1.314913in}}{\pgfqpoint{3.371462in}{1.310523in}}{\pgfqpoint{3.382512in}{1.310523in}}%
\pgfpathclose%
\pgfusepath{stroke,fill}%
\end{pgfscope}%
\begin{pgfscope}%
\pgfpathrectangle{\pgfqpoint{0.511823in}{0.504323in}}{\pgfqpoint{3.218177in}{3.225677in}} %
\pgfusepath{clip}%
\pgfsetbuttcap%
\pgfsetroundjoin%
\definecolor{currentfill}{rgb}{0.501961,0.000000,0.000000}%
\pgfsetfillcolor{currentfill}%
\pgfsetfillopacity{0.400000}%
\pgfsetlinewidth{0.501875pt}%
\definecolor{currentstroke}{rgb}{0.501961,0.000000,0.000000}%
\pgfsetstrokecolor{currentstroke}%
\pgfsetstrokeopacity{0.400000}%
\pgfsetdash{}{0pt}%
\pgfpathmoveto{\pgfqpoint{3.375648in}{1.317262in}}%
\pgfpathcurveto{\pgfqpoint{3.386698in}{1.317262in}}{\pgfqpoint{3.397297in}{1.321652in}}{\pgfqpoint{3.405110in}{1.329466in}}%
\pgfpathcurveto{\pgfqpoint{3.412924in}{1.337279in}}{\pgfqpoint{3.417314in}{1.347878in}}{\pgfqpoint{3.417314in}{1.358929in}}%
\pgfpathcurveto{\pgfqpoint{3.417314in}{1.369979in}}{\pgfqpoint{3.412924in}{1.380578in}}{\pgfqpoint{3.405110in}{1.388391in}}%
\pgfpathcurveto{\pgfqpoint{3.397297in}{1.396205in}}{\pgfqpoint{3.386698in}{1.400595in}}{\pgfqpoint{3.375648in}{1.400595in}}%
\pgfpathcurveto{\pgfqpoint{3.364597in}{1.400595in}}{\pgfqpoint{3.353998in}{1.396205in}}{\pgfqpoint{3.346185in}{1.388391in}}%
\pgfpathcurveto{\pgfqpoint{3.338371in}{1.380578in}}{\pgfqpoint{3.333981in}{1.369979in}}{\pgfqpoint{3.333981in}{1.358929in}}%
\pgfpathcurveto{\pgfqpoint{3.333981in}{1.347878in}}{\pgfqpoint{3.338371in}{1.337279in}}{\pgfqpoint{3.346185in}{1.329466in}}%
\pgfpathcurveto{\pgfqpoint{3.353998in}{1.321652in}}{\pgfqpoint{3.364597in}{1.317262in}}{\pgfqpoint{3.375648in}{1.317262in}}%
\pgfpathclose%
\pgfusepath{stroke,fill}%
\end{pgfscope}%
\begin{pgfscope}%
\pgfpathrectangle{\pgfqpoint{0.511823in}{0.504323in}}{\pgfqpoint{3.218177in}{3.225677in}} %
\pgfusepath{clip}%
\pgfsetbuttcap%
\pgfsetroundjoin%
\definecolor{currentfill}{rgb}{0.501961,0.000000,0.000000}%
\pgfsetfillcolor{currentfill}%
\pgfsetfillopacity{0.400000}%
\pgfsetlinewidth{0.501875pt}%
\definecolor{currentstroke}{rgb}{0.501961,0.000000,0.000000}%
\pgfsetstrokecolor{currentstroke}%
\pgfsetstrokeopacity{0.400000}%
\pgfsetdash{}{0pt}%
\pgfpathmoveto{\pgfqpoint{3.077139in}{1.249580in}}%
\pgfpathcurveto{\pgfqpoint{3.088189in}{1.249580in}}{\pgfqpoint{3.098788in}{1.253970in}}{\pgfqpoint{3.106602in}{1.261784in}}%
\pgfpathcurveto{\pgfqpoint{3.114416in}{1.269597in}}{\pgfqpoint{3.118806in}{1.280196in}}{\pgfqpoint{3.118806in}{1.291246in}}%
\pgfpathcurveto{\pgfqpoint{3.118806in}{1.302297in}}{\pgfqpoint{3.114416in}{1.312896in}}{\pgfqpoint{3.106602in}{1.320709in}}%
\pgfpathcurveto{\pgfqpoint{3.098788in}{1.328523in}}{\pgfqpoint{3.088189in}{1.332913in}}{\pgfqpoint{3.077139in}{1.332913in}}%
\pgfpathcurveto{\pgfqpoint{3.066089in}{1.332913in}}{\pgfqpoint{3.055490in}{1.328523in}}{\pgfqpoint{3.047676in}{1.320709in}}%
\pgfpathcurveto{\pgfqpoint{3.039863in}{1.312896in}}{\pgfqpoint{3.035472in}{1.302297in}}{\pgfqpoint{3.035472in}{1.291246in}}%
\pgfpathcurveto{\pgfqpoint{3.035472in}{1.280196in}}{\pgfqpoint{3.039863in}{1.269597in}}{\pgfqpoint{3.047676in}{1.261784in}}%
\pgfpathcurveto{\pgfqpoint{3.055490in}{1.253970in}}{\pgfqpoint{3.066089in}{1.249580in}}{\pgfqpoint{3.077139in}{1.249580in}}%
\pgfpathclose%
\pgfusepath{stroke,fill}%
\end{pgfscope}%
\begin{pgfscope}%
\pgfpathrectangle{\pgfqpoint{0.511823in}{0.504323in}}{\pgfqpoint{3.218177in}{3.225677in}} %
\pgfusepath{clip}%
\pgfsetbuttcap%
\pgfsetroundjoin%
\definecolor{currentfill}{rgb}{0.501961,0.000000,0.000000}%
\pgfsetfillcolor{currentfill}%
\pgfsetfillopacity{0.400000}%
\pgfsetlinewidth{0.501875pt}%
\definecolor{currentstroke}{rgb}{0.501961,0.000000,0.000000}%
\pgfsetstrokecolor{currentstroke}%
\pgfsetstrokeopacity{0.400000}%
\pgfsetdash{}{0pt}%
\pgfpathmoveto{\pgfqpoint{3.286054in}{1.311063in}}%
\pgfpathcurveto{\pgfqpoint{3.297105in}{1.311063in}}{\pgfqpoint{3.307704in}{1.315454in}}{\pgfqpoint{3.315517in}{1.323267in}}%
\pgfpathcurveto{\pgfqpoint{3.323331in}{1.331081in}}{\pgfqpoint{3.327721in}{1.341680in}}{\pgfqpoint{3.327721in}{1.352730in}}%
\pgfpathcurveto{\pgfqpoint{3.327721in}{1.363780in}}{\pgfqpoint{3.323331in}{1.374379in}}{\pgfqpoint{3.315517in}{1.382193in}}%
\pgfpathcurveto{\pgfqpoint{3.307704in}{1.390006in}}{\pgfqpoint{3.297105in}{1.394397in}}{\pgfqpoint{3.286054in}{1.394397in}}%
\pgfpathcurveto{\pgfqpoint{3.275004in}{1.394397in}}{\pgfqpoint{3.264405in}{1.390006in}}{\pgfqpoint{3.256592in}{1.382193in}}%
\pgfpathcurveto{\pgfqpoint{3.248778in}{1.374379in}}{\pgfqpoint{3.244388in}{1.363780in}}{\pgfqpoint{3.244388in}{1.352730in}}%
\pgfpathcurveto{\pgfqpoint{3.244388in}{1.341680in}}{\pgfqpoint{3.248778in}{1.331081in}}{\pgfqpoint{3.256592in}{1.323267in}}%
\pgfpathcurveto{\pgfqpoint{3.264405in}{1.315454in}}{\pgfqpoint{3.275004in}{1.311063in}}{\pgfqpoint{3.286054in}{1.311063in}}%
\pgfpathclose%
\pgfusepath{stroke,fill}%
\end{pgfscope}%
\begin{pgfscope}%
\pgfpathrectangle{\pgfqpoint{0.511823in}{0.504323in}}{\pgfqpoint{3.218177in}{3.225677in}} %
\pgfusepath{clip}%
\pgfsetbuttcap%
\pgfsetroundjoin%
\definecolor{currentfill}{rgb}{0.501961,0.000000,0.000000}%
\pgfsetfillcolor{currentfill}%
\pgfsetfillopacity{0.400000}%
\pgfsetlinewidth{0.501875pt}%
\definecolor{currentstroke}{rgb}{0.501961,0.000000,0.000000}%
\pgfsetstrokecolor{currentstroke}%
\pgfsetstrokeopacity{0.400000}%
\pgfsetdash{}{0pt}%
\pgfpathmoveto{\pgfqpoint{3.143545in}{1.282021in}}%
\pgfpathcurveto{\pgfqpoint{3.154595in}{1.282021in}}{\pgfqpoint{3.165194in}{1.286411in}}{\pgfqpoint{3.173008in}{1.294225in}}%
\pgfpathcurveto{\pgfqpoint{3.180822in}{1.302038in}}{\pgfqpoint{3.185212in}{1.312637in}}{\pgfqpoint{3.185212in}{1.323688in}}%
\pgfpathcurveto{\pgfqpoint{3.185212in}{1.334738in}}{\pgfqpoint{3.180822in}{1.345337in}}{\pgfqpoint{3.173008in}{1.353150in}}%
\pgfpathcurveto{\pgfqpoint{3.165194in}{1.360964in}}{\pgfqpoint{3.154595in}{1.365354in}}{\pgfqpoint{3.143545in}{1.365354in}}%
\pgfpathcurveto{\pgfqpoint{3.132495in}{1.365354in}}{\pgfqpoint{3.121896in}{1.360964in}}{\pgfqpoint{3.114082in}{1.353150in}}%
\pgfpathcurveto{\pgfqpoint{3.106269in}{1.345337in}}{\pgfqpoint{3.101879in}{1.334738in}}{\pgfqpoint{3.101879in}{1.323688in}}%
\pgfpathcurveto{\pgfqpoint{3.101879in}{1.312637in}}{\pgfqpoint{3.106269in}{1.302038in}}{\pgfqpoint{3.114082in}{1.294225in}}%
\pgfpathcurveto{\pgfqpoint{3.121896in}{1.286411in}}{\pgfqpoint{3.132495in}{1.282021in}}{\pgfqpoint{3.143545in}{1.282021in}}%
\pgfpathclose%
\pgfusepath{stroke,fill}%
\end{pgfscope}%
\begin{pgfscope}%
\pgfpathrectangle{\pgfqpoint{0.511823in}{0.504323in}}{\pgfqpoint{3.218177in}{3.225677in}} %
\pgfusepath{clip}%
\pgfsetbuttcap%
\pgfsetroundjoin%
\definecolor{currentfill}{rgb}{0.501961,0.000000,0.000000}%
\pgfsetfillcolor{currentfill}%
\pgfsetfillopacity{0.400000}%
\pgfsetlinewidth{0.501875pt}%
\definecolor{currentstroke}{rgb}{0.501961,0.000000,0.000000}%
\pgfsetstrokecolor{currentstroke}%
\pgfsetstrokeopacity{0.400000}%
\pgfsetdash{}{0pt}%
\pgfpathmoveto{\pgfqpoint{2.982343in}{1.247146in}}%
\pgfpathcurveto{\pgfqpoint{2.993393in}{1.247146in}}{\pgfqpoint{3.003992in}{1.251537in}}{\pgfqpoint{3.011806in}{1.259350in}}%
\pgfpathcurveto{\pgfqpoint{3.019620in}{1.267164in}}{\pgfqpoint{3.024010in}{1.277763in}}{\pgfqpoint{3.024010in}{1.288813in}}%
\pgfpathcurveto{\pgfqpoint{3.024010in}{1.299863in}}{\pgfqpoint{3.019620in}{1.310462in}}{\pgfqpoint{3.011806in}{1.318276in}}%
\pgfpathcurveto{\pgfqpoint{3.003992in}{1.326089in}}{\pgfqpoint{2.993393in}{1.330480in}}{\pgfqpoint{2.982343in}{1.330480in}}%
\pgfpathcurveto{\pgfqpoint{2.971293in}{1.330480in}}{\pgfqpoint{2.960694in}{1.326089in}}{\pgfqpoint{2.952880in}{1.318276in}}%
\pgfpathcurveto{\pgfqpoint{2.945067in}{1.310462in}}{\pgfqpoint{2.940677in}{1.299863in}}{\pgfqpoint{2.940677in}{1.288813in}}%
\pgfpathcurveto{\pgfqpoint{2.940677in}{1.277763in}}{\pgfqpoint{2.945067in}{1.267164in}}{\pgfqpoint{2.952880in}{1.259350in}}%
\pgfpathcurveto{\pgfqpoint{2.960694in}{1.251537in}}{\pgfqpoint{2.971293in}{1.247146in}}{\pgfqpoint{2.982343in}{1.247146in}}%
\pgfpathclose%
\pgfusepath{stroke,fill}%
\end{pgfscope}%
\begin{pgfscope}%
\pgfpathrectangle{\pgfqpoint{0.511823in}{0.504323in}}{\pgfqpoint{3.218177in}{3.225677in}} %
\pgfusepath{clip}%
\pgfsetbuttcap%
\pgfsetroundjoin%
\definecolor{currentfill}{rgb}{0.501961,0.000000,0.000000}%
\pgfsetfillcolor{currentfill}%
\pgfsetfillopacity{0.400000}%
\pgfsetlinewidth{0.501875pt}%
\definecolor{currentstroke}{rgb}{0.501961,0.000000,0.000000}%
\pgfsetstrokecolor{currentstroke}%
\pgfsetstrokeopacity{0.400000}%
\pgfsetdash{}{0pt}%
\pgfpathmoveto{\pgfqpoint{3.272639in}{1.332139in}}%
\pgfpathcurveto{\pgfqpoint{3.283690in}{1.332139in}}{\pgfqpoint{3.294289in}{1.336530in}}{\pgfqpoint{3.302102in}{1.344343in}}%
\pgfpathcurveto{\pgfqpoint{3.309916in}{1.352157in}}{\pgfqpoint{3.314306in}{1.362756in}}{\pgfqpoint{3.314306in}{1.373806in}}%
\pgfpathcurveto{\pgfqpoint{3.314306in}{1.384856in}}{\pgfqpoint{3.309916in}{1.395455in}}{\pgfqpoint{3.302102in}{1.403269in}}%
\pgfpathcurveto{\pgfqpoint{3.294289in}{1.411082in}}{\pgfqpoint{3.283690in}{1.415473in}}{\pgfqpoint{3.272639in}{1.415473in}}%
\pgfpathcurveto{\pgfqpoint{3.261589in}{1.415473in}}{\pgfqpoint{3.250990in}{1.411082in}}{\pgfqpoint{3.243177in}{1.403269in}}%
\pgfpathcurveto{\pgfqpoint{3.235363in}{1.395455in}}{\pgfqpoint{3.230973in}{1.384856in}}{\pgfqpoint{3.230973in}{1.373806in}}%
\pgfpathcurveto{\pgfqpoint{3.230973in}{1.362756in}}{\pgfqpoint{3.235363in}{1.352157in}}{\pgfqpoint{3.243177in}{1.344343in}}%
\pgfpathcurveto{\pgfqpoint{3.250990in}{1.336530in}}{\pgfqpoint{3.261589in}{1.332139in}}{\pgfqpoint{3.272639in}{1.332139in}}%
\pgfpathclose%
\pgfusepath{stroke,fill}%
\end{pgfscope}%
\begin{pgfscope}%
\pgfpathrectangle{\pgfqpoint{0.511823in}{0.504323in}}{\pgfqpoint{3.218177in}{3.225677in}} %
\pgfusepath{clip}%
\pgfsetbuttcap%
\pgfsetroundjoin%
\definecolor{currentfill}{rgb}{0.501961,0.000000,0.000000}%
\pgfsetfillcolor{currentfill}%
\pgfsetfillopacity{0.400000}%
\pgfsetlinewidth{0.501875pt}%
\definecolor{currentstroke}{rgb}{0.501961,0.000000,0.000000}%
\pgfsetstrokecolor{currentstroke}%
\pgfsetstrokeopacity{0.400000}%
\pgfsetdash{}{0pt}%
\pgfpathmoveto{\pgfqpoint{3.012562in}{1.269895in}}%
\pgfpathcurveto{\pgfqpoint{3.023612in}{1.269895in}}{\pgfqpoint{3.034211in}{1.274285in}}{\pgfqpoint{3.042024in}{1.282099in}}%
\pgfpathcurveto{\pgfqpoint{3.049838in}{1.289913in}}{\pgfqpoint{3.054228in}{1.300512in}}{\pgfqpoint{3.054228in}{1.311562in}}%
\pgfpathcurveto{\pgfqpoint{3.054228in}{1.322612in}}{\pgfqpoint{3.049838in}{1.333211in}}{\pgfqpoint{3.042024in}{1.341025in}}%
\pgfpathcurveto{\pgfqpoint{3.034211in}{1.348838in}}{\pgfqpoint{3.023612in}{1.353228in}}{\pgfqpoint{3.012562in}{1.353228in}}%
\pgfpathcurveto{\pgfqpoint{3.001511in}{1.353228in}}{\pgfqpoint{2.990912in}{1.348838in}}{\pgfqpoint{2.983099in}{1.341025in}}%
\pgfpathcurveto{\pgfqpoint{2.975285in}{1.333211in}}{\pgfqpoint{2.970895in}{1.322612in}}{\pgfqpoint{2.970895in}{1.311562in}}%
\pgfpathcurveto{\pgfqpoint{2.970895in}{1.300512in}}{\pgfqpoint{2.975285in}{1.289913in}}{\pgfqpoint{2.983099in}{1.282099in}}%
\pgfpathcurveto{\pgfqpoint{2.990912in}{1.274285in}}{\pgfqpoint{3.001511in}{1.269895in}}{\pgfqpoint{3.012562in}{1.269895in}}%
\pgfpathclose%
\pgfusepath{stroke,fill}%
\end{pgfscope}%
\begin{pgfscope}%
\pgfpathrectangle{\pgfqpoint{0.511823in}{0.504323in}}{\pgfqpoint{3.218177in}{3.225677in}} %
\pgfusepath{clip}%
\pgfsetbuttcap%
\pgfsetroundjoin%
\definecolor{currentfill}{rgb}{0.501961,0.000000,0.000000}%
\pgfsetfillcolor{currentfill}%
\pgfsetfillopacity{0.400000}%
\pgfsetlinewidth{0.501875pt}%
\definecolor{currentstroke}{rgb}{0.501961,0.000000,0.000000}%
\pgfsetstrokecolor{currentstroke}%
\pgfsetstrokeopacity{0.400000}%
\pgfsetdash{}{0pt}%
\pgfpathmoveto{\pgfqpoint{3.172218in}{1.321050in}}%
\pgfpathcurveto{\pgfqpoint{3.183268in}{1.321050in}}{\pgfqpoint{3.193867in}{1.325441in}}{\pgfqpoint{3.201681in}{1.333254in}}%
\pgfpathcurveto{\pgfqpoint{3.209494in}{1.341068in}}{\pgfqpoint{3.213885in}{1.351667in}}{\pgfqpoint{3.213885in}{1.362717in}}%
\pgfpathcurveto{\pgfqpoint{3.213885in}{1.373767in}}{\pgfqpoint{3.209494in}{1.384366in}}{\pgfqpoint{3.201681in}{1.392180in}}%
\pgfpathcurveto{\pgfqpoint{3.193867in}{1.399993in}}{\pgfqpoint{3.183268in}{1.404384in}}{\pgfqpoint{3.172218in}{1.404384in}}%
\pgfpathcurveto{\pgfqpoint{3.161168in}{1.404384in}}{\pgfqpoint{3.150569in}{1.399993in}}{\pgfqpoint{3.142755in}{1.392180in}}%
\pgfpathcurveto{\pgfqpoint{3.134942in}{1.384366in}}{\pgfqpoint{3.130551in}{1.373767in}}{\pgfqpoint{3.130551in}{1.362717in}}%
\pgfpathcurveto{\pgfqpoint{3.130551in}{1.351667in}}{\pgfqpoint{3.134942in}{1.341068in}}{\pgfqpoint{3.142755in}{1.333254in}}%
\pgfpathcurveto{\pgfqpoint{3.150569in}{1.325441in}}{\pgfqpoint{3.161168in}{1.321050in}}{\pgfqpoint{3.172218in}{1.321050in}}%
\pgfpathclose%
\pgfusepath{stroke,fill}%
\end{pgfscope}%
\begin{pgfscope}%
\pgfpathrectangle{\pgfqpoint{0.511823in}{0.504323in}}{\pgfqpoint{3.218177in}{3.225677in}} %
\pgfusepath{clip}%
\pgfsetbuttcap%
\pgfsetroundjoin%
\definecolor{currentfill}{rgb}{0.501961,0.000000,0.000000}%
\pgfsetfillcolor{currentfill}%
\pgfsetfillopacity{0.400000}%
\pgfsetlinewidth{0.501875pt}%
\definecolor{currentstroke}{rgb}{0.501961,0.000000,0.000000}%
\pgfsetstrokecolor{currentstroke}%
\pgfsetstrokeopacity{0.400000}%
\pgfsetdash{}{0pt}%
\pgfpathmoveto{\pgfqpoint{2.910188in}{1.256320in}}%
\pgfpathcurveto{\pgfqpoint{2.921238in}{1.256320in}}{\pgfqpoint{2.931837in}{1.260710in}}{\pgfqpoint{2.939651in}{1.268524in}}%
\pgfpathcurveto{\pgfqpoint{2.947465in}{1.276337in}}{\pgfqpoint{2.951855in}{1.286936in}}{\pgfqpoint{2.951855in}{1.297986in}}%
\pgfpathcurveto{\pgfqpoint{2.951855in}{1.309036in}}{\pgfqpoint{2.947465in}{1.319636in}}{\pgfqpoint{2.939651in}{1.327449in}}%
\pgfpathcurveto{\pgfqpoint{2.931837in}{1.335263in}}{\pgfqpoint{2.921238in}{1.339653in}}{\pgfqpoint{2.910188in}{1.339653in}}%
\pgfpathcurveto{\pgfqpoint{2.899138in}{1.339653in}}{\pgfqpoint{2.888539in}{1.335263in}}{\pgfqpoint{2.880725in}{1.327449in}}%
\pgfpathcurveto{\pgfqpoint{2.872912in}{1.319636in}}{\pgfqpoint{2.868522in}{1.309036in}}{\pgfqpoint{2.868522in}{1.297986in}}%
\pgfpathcurveto{\pgfqpoint{2.868522in}{1.286936in}}{\pgfqpoint{2.872912in}{1.276337in}}{\pgfqpoint{2.880725in}{1.268524in}}%
\pgfpathcurveto{\pgfqpoint{2.888539in}{1.260710in}}{\pgfqpoint{2.899138in}{1.256320in}}{\pgfqpoint{2.910188in}{1.256320in}}%
\pgfpathclose%
\pgfusepath{stroke,fill}%
\end{pgfscope}%
\begin{pgfscope}%
\pgfpathrectangle{\pgfqpoint{0.511823in}{0.504323in}}{\pgfqpoint{3.218177in}{3.225677in}} %
\pgfusepath{clip}%
\pgfsetbuttcap%
\pgfsetroundjoin%
\definecolor{currentfill}{rgb}{0.501961,0.000000,0.000000}%
\pgfsetfillcolor{currentfill}%
\pgfsetfillopacity{0.400000}%
\pgfsetlinewidth{0.501875pt}%
\definecolor{currentstroke}{rgb}{0.501961,0.000000,0.000000}%
\pgfsetstrokecolor{currentstroke}%
\pgfsetstrokeopacity{0.400000}%
\pgfsetdash{}{0pt}%
\pgfpathmoveto{\pgfqpoint{3.097122in}{1.315844in}}%
\pgfpathcurveto{\pgfqpoint{3.108172in}{1.315844in}}{\pgfqpoint{3.118771in}{1.320234in}}{\pgfqpoint{3.126585in}{1.328048in}}%
\pgfpathcurveto{\pgfqpoint{3.134398in}{1.335862in}}{\pgfqpoint{3.138789in}{1.346461in}}{\pgfqpoint{3.138789in}{1.357511in}}%
\pgfpathcurveto{\pgfqpoint{3.138789in}{1.368561in}}{\pgfqpoint{3.134398in}{1.379160in}}{\pgfqpoint{3.126585in}{1.386974in}}%
\pgfpathcurveto{\pgfqpoint{3.118771in}{1.394787in}}{\pgfqpoint{3.108172in}{1.399177in}}{\pgfqpoint{3.097122in}{1.399177in}}%
\pgfpathcurveto{\pgfqpoint{3.086072in}{1.399177in}}{\pgfqpoint{3.075473in}{1.394787in}}{\pgfqpoint{3.067659in}{1.386974in}}%
\pgfpathcurveto{\pgfqpoint{3.059845in}{1.379160in}}{\pgfqpoint{3.055455in}{1.368561in}}{\pgfqpoint{3.055455in}{1.357511in}}%
\pgfpathcurveto{\pgfqpoint{3.055455in}{1.346461in}}{\pgfqpoint{3.059845in}{1.335862in}}{\pgfqpoint{3.067659in}{1.328048in}}%
\pgfpathcurveto{\pgfqpoint{3.075473in}{1.320234in}}{\pgfqpoint{3.086072in}{1.315844in}}{\pgfqpoint{3.097122in}{1.315844in}}%
\pgfpathclose%
\pgfusepath{stroke,fill}%
\end{pgfscope}%
\begin{pgfscope}%
\pgfpathrectangle{\pgfqpoint{0.511823in}{0.504323in}}{\pgfqpoint{3.218177in}{3.225677in}} %
\pgfusepath{clip}%
\pgfsetbuttcap%
\pgfsetroundjoin%
\definecolor{currentfill}{rgb}{0.501961,0.000000,0.000000}%
\pgfsetfillcolor{currentfill}%
\pgfsetfillopacity{0.400000}%
\pgfsetlinewidth{0.501875pt}%
\definecolor{currentstroke}{rgb}{0.501961,0.000000,0.000000}%
\pgfsetstrokecolor{currentstroke}%
\pgfsetstrokeopacity{0.400000}%
\pgfsetdash{}{0pt}%
\pgfpathmoveto{\pgfqpoint{3.164770in}{1.342742in}}%
\pgfpathcurveto{\pgfqpoint{3.175820in}{1.342742in}}{\pgfqpoint{3.186419in}{1.347133in}}{\pgfqpoint{3.194233in}{1.354946in}}%
\pgfpathcurveto{\pgfqpoint{3.202047in}{1.362760in}}{\pgfqpoint{3.206437in}{1.373359in}}{\pgfqpoint{3.206437in}{1.384409in}}%
\pgfpathcurveto{\pgfqpoint{3.206437in}{1.395459in}}{\pgfqpoint{3.202047in}{1.406058in}}{\pgfqpoint{3.194233in}{1.413872in}}%
\pgfpathcurveto{\pgfqpoint{3.186419in}{1.421685in}}{\pgfqpoint{3.175820in}{1.426076in}}{\pgfqpoint{3.164770in}{1.426076in}}%
\pgfpathcurveto{\pgfqpoint{3.153720in}{1.426076in}}{\pgfqpoint{3.143121in}{1.421685in}}{\pgfqpoint{3.135307in}{1.413872in}}%
\pgfpathcurveto{\pgfqpoint{3.127494in}{1.406058in}}{\pgfqpoint{3.123104in}{1.395459in}}{\pgfqpoint{3.123104in}{1.384409in}}%
\pgfpathcurveto{\pgfqpoint{3.123104in}{1.373359in}}{\pgfqpoint{3.127494in}{1.362760in}}{\pgfqpoint{3.135307in}{1.354946in}}%
\pgfpathcurveto{\pgfqpoint{3.143121in}{1.347133in}}{\pgfqpoint{3.153720in}{1.342742in}}{\pgfqpoint{3.164770in}{1.342742in}}%
\pgfpathclose%
\pgfusepath{stroke,fill}%
\end{pgfscope}%
\begin{pgfscope}%
\pgfpathrectangle{\pgfqpoint{0.511823in}{0.504323in}}{\pgfqpoint{3.218177in}{3.225677in}} %
\pgfusepath{clip}%
\pgfsetbuttcap%
\pgfsetroundjoin%
\definecolor{currentfill}{rgb}{0.501961,0.000000,0.000000}%
\pgfsetfillcolor{currentfill}%
\pgfsetfillopacity{0.400000}%
\pgfsetlinewidth{0.501875pt}%
\definecolor{currentstroke}{rgb}{0.501961,0.000000,0.000000}%
\pgfsetstrokecolor{currentstroke}%
\pgfsetstrokeopacity{0.400000}%
\pgfsetdash{}{0pt}%
\pgfpathmoveto{\pgfqpoint{3.220956in}{1.366801in}}%
\pgfpathcurveto{\pgfqpoint{3.232006in}{1.366801in}}{\pgfqpoint{3.242605in}{1.371191in}}{\pgfqpoint{3.250419in}{1.379005in}}%
\pgfpathcurveto{\pgfqpoint{3.258233in}{1.386818in}}{\pgfqpoint{3.262623in}{1.397417in}}{\pgfqpoint{3.262623in}{1.408467in}}%
\pgfpathcurveto{\pgfqpoint{3.262623in}{1.419517in}}{\pgfqpoint{3.258233in}{1.430117in}}{\pgfqpoint{3.250419in}{1.437930in}}%
\pgfpathcurveto{\pgfqpoint{3.242605in}{1.445744in}}{\pgfqpoint{3.232006in}{1.450134in}}{\pgfqpoint{3.220956in}{1.450134in}}%
\pgfpathcurveto{\pgfqpoint{3.209906in}{1.450134in}}{\pgfqpoint{3.199307in}{1.445744in}}{\pgfqpoint{3.191493in}{1.437930in}}%
\pgfpathcurveto{\pgfqpoint{3.183680in}{1.430117in}}{\pgfqpoint{3.179289in}{1.419517in}}{\pgfqpoint{3.179289in}{1.408467in}}%
\pgfpathcurveto{\pgfqpoint{3.179289in}{1.397417in}}{\pgfqpoint{3.183680in}{1.386818in}}{\pgfqpoint{3.191493in}{1.379005in}}%
\pgfpathcurveto{\pgfqpoint{3.199307in}{1.371191in}}{\pgfqpoint{3.209906in}{1.366801in}}{\pgfqpoint{3.220956in}{1.366801in}}%
\pgfpathclose%
\pgfusepath{stroke,fill}%
\end{pgfscope}%
\begin{pgfscope}%
\pgfpathrectangle{\pgfqpoint{0.511823in}{0.504323in}}{\pgfqpoint{3.218177in}{3.225677in}} %
\pgfusepath{clip}%
\pgfsetbuttcap%
\pgfsetroundjoin%
\definecolor{currentfill}{rgb}{0.501961,0.000000,0.000000}%
\pgfsetfillcolor{currentfill}%
\pgfsetfillopacity{0.400000}%
\pgfsetlinewidth{0.501875pt}%
\definecolor{currentstroke}{rgb}{0.501961,0.000000,0.000000}%
\pgfsetstrokecolor{currentstroke}%
\pgfsetstrokeopacity{0.400000}%
\pgfsetdash{}{0pt}%
\pgfpathmoveto{\pgfqpoint{3.151092in}{1.354673in}}%
\pgfpathcurveto{\pgfqpoint{3.162143in}{1.354673in}}{\pgfqpoint{3.172742in}{1.359063in}}{\pgfqpoint{3.180555in}{1.366877in}}%
\pgfpathcurveto{\pgfqpoint{3.188369in}{1.374690in}}{\pgfqpoint{3.192759in}{1.385289in}}{\pgfqpoint{3.192759in}{1.396339in}}%
\pgfpathcurveto{\pgfqpoint{3.192759in}{1.407389in}}{\pgfqpoint{3.188369in}{1.417988in}}{\pgfqpoint{3.180555in}{1.425802in}}%
\pgfpathcurveto{\pgfqpoint{3.172742in}{1.433616in}}{\pgfqpoint{3.162143in}{1.438006in}}{\pgfqpoint{3.151092in}{1.438006in}}%
\pgfpathcurveto{\pgfqpoint{3.140042in}{1.438006in}}{\pgfqpoint{3.129443in}{1.433616in}}{\pgfqpoint{3.121630in}{1.425802in}}%
\pgfpathcurveto{\pgfqpoint{3.113816in}{1.417988in}}{\pgfqpoint{3.109426in}{1.407389in}}{\pgfqpoint{3.109426in}{1.396339in}}%
\pgfpathcurveto{\pgfqpoint{3.109426in}{1.385289in}}{\pgfqpoint{3.113816in}{1.374690in}}{\pgfqpoint{3.121630in}{1.366877in}}%
\pgfpathcurveto{\pgfqpoint{3.129443in}{1.359063in}}{\pgfqpoint{3.140042in}{1.354673in}}{\pgfqpoint{3.151092in}{1.354673in}}%
\pgfpathclose%
\pgfusepath{stroke,fill}%
\end{pgfscope}%
\begin{pgfscope}%
\pgfpathrectangle{\pgfqpoint{0.511823in}{0.504323in}}{\pgfqpoint{3.218177in}{3.225677in}} %
\pgfusepath{clip}%
\pgfsetbuttcap%
\pgfsetroundjoin%
\definecolor{currentfill}{rgb}{0.501961,0.000000,0.000000}%
\pgfsetfillcolor{currentfill}%
\pgfsetfillopacity{0.400000}%
\pgfsetlinewidth{0.501875pt}%
\definecolor{currentstroke}{rgb}{0.501961,0.000000,0.000000}%
\pgfsetstrokecolor{currentstroke}%
\pgfsetstrokeopacity{0.400000}%
\pgfsetdash{}{0pt}%
\pgfpathmoveto{\pgfqpoint{3.100002in}{1.347617in}}%
\pgfpathcurveto{\pgfqpoint{3.111053in}{1.347617in}}{\pgfqpoint{3.121652in}{1.352007in}}{\pgfqpoint{3.129465in}{1.359821in}}%
\pgfpathcurveto{\pgfqpoint{3.137279in}{1.367635in}}{\pgfqpoint{3.141669in}{1.378234in}}{\pgfqpoint{3.141669in}{1.389284in}}%
\pgfpathcurveto{\pgfqpoint{3.141669in}{1.400334in}}{\pgfqpoint{3.137279in}{1.410933in}}{\pgfqpoint{3.129465in}{1.418747in}}%
\pgfpathcurveto{\pgfqpoint{3.121652in}{1.426560in}}{\pgfqpoint{3.111053in}{1.430950in}}{\pgfqpoint{3.100002in}{1.430950in}}%
\pgfpathcurveto{\pgfqpoint{3.088952in}{1.430950in}}{\pgfqpoint{3.078353in}{1.426560in}}{\pgfqpoint{3.070540in}{1.418747in}}%
\pgfpathcurveto{\pgfqpoint{3.062726in}{1.410933in}}{\pgfqpoint{3.058336in}{1.400334in}}{\pgfqpoint{3.058336in}{1.389284in}}%
\pgfpathcurveto{\pgfqpoint{3.058336in}{1.378234in}}{\pgfqpoint{3.062726in}{1.367635in}}{\pgfqpoint{3.070540in}{1.359821in}}%
\pgfpathcurveto{\pgfqpoint{3.078353in}{1.352007in}}{\pgfqpoint{3.088952in}{1.347617in}}{\pgfqpoint{3.100002in}{1.347617in}}%
\pgfpathclose%
\pgfusepath{stroke,fill}%
\end{pgfscope}%
\begin{pgfscope}%
\pgfpathrectangle{\pgfqpoint{0.511823in}{0.504323in}}{\pgfqpoint{3.218177in}{3.225677in}} %
\pgfusepath{clip}%
\pgfsetbuttcap%
\pgfsetroundjoin%
\definecolor{currentfill}{rgb}{0.501961,0.000000,0.000000}%
\pgfsetfillcolor{currentfill}%
\pgfsetfillopacity{0.400000}%
\pgfsetlinewidth{0.501875pt}%
\definecolor{currentstroke}{rgb}{0.501961,0.000000,0.000000}%
\pgfsetstrokecolor{currentstroke}%
\pgfsetstrokeopacity{0.400000}%
\pgfsetdash{}{0pt}%
\pgfpathmoveto{\pgfqpoint{3.100555in}{1.355560in}}%
\pgfpathcurveto{\pgfqpoint{3.111605in}{1.355560in}}{\pgfqpoint{3.122204in}{1.359951in}}{\pgfqpoint{3.130017in}{1.367764in}}%
\pgfpathcurveto{\pgfqpoint{3.137831in}{1.375578in}}{\pgfqpoint{3.142221in}{1.386177in}}{\pgfqpoint{3.142221in}{1.397227in}}%
\pgfpathcurveto{\pgfqpoint{3.142221in}{1.408277in}}{\pgfqpoint{3.137831in}{1.418876in}}{\pgfqpoint{3.130017in}{1.426690in}}%
\pgfpathcurveto{\pgfqpoint{3.122204in}{1.434503in}}{\pgfqpoint{3.111605in}{1.438894in}}{\pgfqpoint{3.100555in}{1.438894in}}%
\pgfpathcurveto{\pgfqpoint{3.089505in}{1.438894in}}{\pgfqpoint{3.078905in}{1.434503in}}{\pgfqpoint{3.071092in}{1.426690in}}%
\pgfpathcurveto{\pgfqpoint{3.063278in}{1.418876in}}{\pgfqpoint{3.058888in}{1.408277in}}{\pgfqpoint{3.058888in}{1.397227in}}%
\pgfpathcurveto{\pgfqpoint{3.058888in}{1.386177in}}{\pgfqpoint{3.063278in}{1.375578in}}{\pgfqpoint{3.071092in}{1.367764in}}%
\pgfpathcurveto{\pgfqpoint{3.078905in}{1.359951in}}{\pgfqpoint{3.089505in}{1.355560in}}{\pgfqpoint{3.100555in}{1.355560in}}%
\pgfpathclose%
\pgfusepath{stroke,fill}%
\end{pgfscope}%
\begin{pgfscope}%
\pgfpathrectangle{\pgfqpoint{0.511823in}{0.504323in}}{\pgfqpoint{3.218177in}{3.225677in}} %
\pgfusepath{clip}%
\pgfsetbuttcap%
\pgfsetroundjoin%
\definecolor{currentfill}{rgb}{0.501961,0.000000,0.000000}%
\pgfsetfillcolor{currentfill}%
\pgfsetfillopacity{0.400000}%
\pgfsetlinewidth{0.501875pt}%
\definecolor{currentstroke}{rgb}{0.501961,0.000000,0.000000}%
\pgfsetstrokecolor{currentstroke}%
\pgfsetstrokeopacity{0.400000}%
\pgfsetdash{}{0pt}%
\pgfpathmoveto{\pgfqpoint{3.405373in}{1.454722in}}%
\pgfpathcurveto{\pgfqpoint{3.416423in}{1.454722in}}{\pgfqpoint{3.427022in}{1.459112in}}{\pgfqpoint{3.434836in}{1.466926in}}%
\pgfpathcurveto{\pgfqpoint{3.442650in}{1.474739in}}{\pgfqpoint{3.447040in}{1.485338in}}{\pgfqpoint{3.447040in}{1.496389in}}%
\pgfpathcurveto{\pgfqpoint{3.447040in}{1.507439in}}{\pgfqpoint{3.442650in}{1.518038in}}{\pgfqpoint{3.434836in}{1.525851in}}%
\pgfpathcurveto{\pgfqpoint{3.427022in}{1.533665in}}{\pgfqpoint{3.416423in}{1.538055in}}{\pgfqpoint{3.405373in}{1.538055in}}%
\pgfpathcurveto{\pgfqpoint{3.394323in}{1.538055in}}{\pgfqpoint{3.383724in}{1.533665in}}{\pgfqpoint{3.375911in}{1.525851in}}%
\pgfpathcurveto{\pgfqpoint{3.368097in}{1.518038in}}{\pgfqpoint{3.363707in}{1.507439in}}{\pgfqpoint{3.363707in}{1.496389in}}%
\pgfpathcurveto{\pgfqpoint{3.363707in}{1.485338in}}{\pgfqpoint{3.368097in}{1.474739in}}{\pgfqpoint{3.375911in}{1.466926in}}%
\pgfpathcurveto{\pgfqpoint{3.383724in}{1.459112in}}{\pgfqpoint{3.394323in}{1.454722in}}{\pgfqpoint{3.405373in}{1.454722in}}%
\pgfpathclose%
\pgfusepath{stroke,fill}%
\end{pgfscope}%
\begin{pgfscope}%
\pgfpathrectangle{\pgfqpoint{0.511823in}{0.504323in}}{\pgfqpoint{3.218177in}{3.225677in}} %
\pgfusepath{clip}%
\pgfsetbuttcap%
\pgfsetroundjoin%
\definecolor{currentfill}{rgb}{0.501961,0.000000,0.000000}%
\pgfsetfillcolor{currentfill}%
\pgfsetfillopacity{0.400000}%
\pgfsetlinewidth{0.501875pt}%
\definecolor{currentstroke}{rgb}{0.501961,0.000000,0.000000}%
\pgfsetstrokecolor{currentstroke}%
\pgfsetstrokeopacity{0.400000}%
\pgfsetdash{}{0pt}%
\pgfpathmoveto{\pgfqpoint{3.149953in}{1.386136in}}%
\pgfpathcurveto{\pgfqpoint{3.161004in}{1.386136in}}{\pgfqpoint{3.171603in}{1.390526in}}{\pgfqpoint{3.179416in}{1.398340in}}%
\pgfpathcurveto{\pgfqpoint{3.187230in}{1.406154in}}{\pgfqpoint{3.191620in}{1.416753in}}{\pgfqpoint{3.191620in}{1.427803in}}%
\pgfpathcurveto{\pgfqpoint{3.191620in}{1.438853in}}{\pgfqpoint{3.187230in}{1.449452in}}{\pgfqpoint{3.179416in}{1.457265in}}%
\pgfpathcurveto{\pgfqpoint{3.171603in}{1.465079in}}{\pgfqpoint{3.161004in}{1.469469in}}{\pgfqpoint{3.149953in}{1.469469in}}%
\pgfpathcurveto{\pgfqpoint{3.138903in}{1.469469in}}{\pgfqpoint{3.128304in}{1.465079in}}{\pgfqpoint{3.120491in}{1.457265in}}%
\pgfpathcurveto{\pgfqpoint{3.112677in}{1.449452in}}{\pgfqpoint{3.108287in}{1.438853in}}{\pgfqpoint{3.108287in}{1.427803in}}%
\pgfpathcurveto{\pgfqpoint{3.108287in}{1.416753in}}{\pgfqpoint{3.112677in}{1.406154in}}{\pgfqpoint{3.120491in}{1.398340in}}%
\pgfpathcurveto{\pgfqpoint{3.128304in}{1.390526in}}{\pgfqpoint{3.138903in}{1.386136in}}{\pgfqpoint{3.149953in}{1.386136in}}%
\pgfpathclose%
\pgfusepath{stroke,fill}%
\end{pgfscope}%
\begin{pgfscope}%
\pgfpathrectangle{\pgfqpoint{0.511823in}{0.504323in}}{\pgfqpoint{3.218177in}{3.225677in}} %
\pgfusepath{clip}%
\pgfsetbuttcap%
\pgfsetroundjoin%
\definecolor{currentfill}{rgb}{0.501961,0.000000,0.000000}%
\pgfsetfillcolor{currentfill}%
\pgfsetfillopacity{0.400000}%
\pgfsetlinewidth{0.501875pt}%
\definecolor{currentstroke}{rgb}{0.501961,0.000000,0.000000}%
\pgfsetstrokecolor{currentstroke}%
\pgfsetstrokeopacity{0.400000}%
\pgfsetdash{}{0pt}%
\pgfpathmoveto{\pgfqpoint{3.194388in}{1.407731in}}%
\pgfpathcurveto{\pgfqpoint{3.205438in}{1.407731in}}{\pgfqpoint{3.216037in}{1.412121in}}{\pgfqpoint{3.223850in}{1.419935in}}%
\pgfpathcurveto{\pgfqpoint{3.231664in}{1.427749in}}{\pgfqpoint{3.236054in}{1.438348in}}{\pgfqpoint{3.236054in}{1.449398in}}%
\pgfpathcurveto{\pgfqpoint{3.236054in}{1.460448in}}{\pgfqpoint{3.231664in}{1.471047in}}{\pgfqpoint{3.223850in}{1.478861in}}%
\pgfpathcurveto{\pgfqpoint{3.216037in}{1.486674in}}{\pgfqpoint{3.205438in}{1.491065in}}{\pgfqpoint{3.194388in}{1.491065in}}%
\pgfpathcurveto{\pgfqpoint{3.183337in}{1.491065in}}{\pgfqpoint{3.172738in}{1.486674in}}{\pgfqpoint{3.164925in}{1.478861in}}%
\pgfpathcurveto{\pgfqpoint{3.157111in}{1.471047in}}{\pgfqpoint{3.152721in}{1.460448in}}{\pgfqpoint{3.152721in}{1.449398in}}%
\pgfpathcurveto{\pgfqpoint{3.152721in}{1.438348in}}{\pgfqpoint{3.157111in}{1.427749in}}{\pgfqpoint{3.164925in}{1.419935in}}%
\pgfpathcurveto{\pgfqpoint{3.172738in}{1.412121in}}{\pgfqpoint{3.183337in}{1.407731in}}{\pgfqpoint{3.194388in}{1.407731in}}%
\pgfpathclose%
\pgfusepath{stroke,fill}%
\end{pgfscope}%
\begin{pgfscope}%
\pgfpathrectangle{\pgfqpoint{0.511823in}{0.504323in}}{\pgfqpoint{3.218177in}{3.225677in}} %
\pgfusepath{clip}%
\pgfsetbuttcap%
\pgfsetroundjoin%
\definecolor{currentfill}{rgb}{0.501961,0.000000,0.000000}%
\pgfsetfillcolor{currentfill}%
\pgfsetfillopacity{0.400000}%
\pgfsetlinewidth{0.501875pt}%
\definecolor{currentstroke}{rgb}{0.501961,0.000000,0.000000}%
\pgfsetstrokecolor{currentstroke}%
\pgfsetstrokeopacity{0.400000}%
\pgfsetdash{}{0pt}%
\pgfpathmoveto{\pgfqpoint{3.441286in}{1.492287in}}%
\pgfpathcurveto{\pgfqpoint{3.452336in}{1.492287in}}{\pgfqpoint{3.462935in}{1.496677in}}{\pgfqpoint{3.470749in}{1.504491in}}%
\pgfpathcurveto{\pgfqpoint{3.478562in}{1.512305in}}{\pgfqpoint{3.482953in}{1.522904in}}{\pgfqpoint{3.482953in}{1.533954in}}%
\pgfpathcurveto{\pgfqpoint{3.482953in}{1.545004in}}{\pgfqpoint{3.478562in}{1.555603in}}{\pgfqpoint{3.470749in}{1.563417in}}%
\pgfpathcurveto{\pgfqpoint{3.462935in}{1.571230in}}{\pgfqpoint{3.452336in}{1.575621in}}{\pgfqpoint{3.441286in}{1.575621in}}%
\pgfpathcurveto{\pgfqpoint{3.430236in}{1.575621in}}{\pgfqpoint{3.419637in}{1.571230in}}{\pgfqpoint{3.411823in}{1.563417in}}%
\pgfpathcurveto{\pgfqpoint{3.404010in}{1.555603in}}{\pgfqpoint{3.399619in}{1.545004in}}{\pgfqpoint{3.399619in}{1.533954in}}%
\pgfpathcurveto{\pgfqpoint{3.399619in}{1.522904in}}{\pgfqpoint{3.404010in}{1.512305in}}{\pgfqpoint{3.411823in}{1.504491in}}%
\pgfpathcurveto{\pgfqpoint{3.419637in}{1.496677in}}{\pgfqpoint{3.430236in}{1.492287in}}{\pgfqpoint{3.441286in}{1.492287in}}%
\pgfpathclose%
\pgfusepath{stroke,fill}%
\end{pgfscope}%
\begin{pgfscope}%
\pgfpathrectangle{\pgfqpoint{0.511823in}{0.504323in}}{\pgfqpoint{3.218177in}{3.225677in}} %
\pgfusepath{clip}%
\pgfsetbuttcap%
\pgfsetroundjoin%
\definecolor{currentfill}{rgb}{0.501961,0.000000,0.000000}%
\pgfsetfillcolor{currentfill}%
\pgfsetfillopacity{0.400000}%
\pgfsetlinewidth{0.501875pt}%
\definecolor{currentstroke}{rgb}{0.501961,0.000000,0.000000}%
\pgfsetstrokecolor{currentstroke}%
\pgfsetstrokeopacity{0.400000}%
\pgfsetdash{}{0pt}%
\pgfpathmoveto{\pgfqpoint{2.982315in}{1.357729in}}%
\pgfpathcurveto{\pgfqpoint{2.993366in}{1.357729in}}{\pgfqpoint{3.003965in}{1.362119in}}{\pgfqpoint{3.011778in}{1.369932in}}%
\pgfpathcurveto{\pgfqpoint{3.019592in}{1.377746in}}{\pgfqpoint{3.023982in}{1.388345in}}{\pgfqpoint{3.023982in}{1.399395in}}%
\pgfpathcurveto{\pgfqpoint{3.023982in}{1.410445in}}{\pgfqpoint{3.019592in}{1.421044in}}{\pgfqpoint{3.011778in}{1.428858in}}%
\pgfpathcurveto{\pgfqpoint{3.003965in}{1.436672in}}{\pgfqpoint{2.993366in}{1.441062in}}{\pgfqpoint{2.982315in}{1.441062in}}%
\pgfpathcurveto{\pgfqpoint{2.971265in}{1.441062in}}{\pgfqpoint{2.960666in}{1.436672in}}{\pgfqpoint{2.952853in}{1.428858in}}%
\pgfpathcurveto{\pgfqpoint{2.945039in}{1.421044in}}{\pgfqpoint{2.940649in}{1.410445in}}{\pgfqpoint{2.940649in}{1.399395in}}%
\pgfpathcurveto{\pgfqpoint{2.940649in}{1.388345in}}{\pgfqpoint{2.945039in}{1.377746in}}{\pgfqpoint{2.952853in}{1.369932in}}%
\pgfpathcurveto{\pgfqpoint{2.960666in}{1.362119in}}{\pgfqpoint{2.971265in}{1.357729in}}{\pgfqpoint{2.982315in}{1.357729in}}%
\pgfpathclose%
\pgfusepath{stroke,fill}%
\end{pgfscope}%
\begin{pgfscope}%
\pgfpathrectangle{\pgfqpoint{0.511823in}{0.504323in}}{\pgfqpoint{3.218177in}{3.225677in}} %
\pgfusepath{clip}%
\pgfsetbuttcap%
\pgfsetroundjoin%
\definecolor{currentfill}{rgb}{0.501961,0.000000,0.000000}%
\pgfsetfillcolor{currentfill}%
\pgfsetfillopacity{0.400000}%
\pgfsetlinewidth{0.501875pt}%
\definecolor{currentstroke}{rgb}{0.501961,0.000000,0.000000}%
\pgfsetstrokecolor{currentstroke}%
\pgfsetstrokeopacity{0.400000}%
\pgfsetdash{}{0pt}%
\pgfpathmoveto{\pgfqpoint{2.938891in}{1.351499in}}%
\pgfpathcurveto{\pgfqpoint{2.949941in}{1.351499in}}{\pgfqpoint{2.960540in}{1.355890in}}{\pgfqpoint{2.968354in}{1.363703in}}%
\pgfpathcurveto{\pgfqpoint{2.976167in}{1.371517in}}{\pgfqpoint{2.980558in}{1.382116in}}{\pgfqpoint{2.980558in}{1.393166in}}%
\pgfpathcurveto{\pgfqpoint{2.980558in}{1.404216in}}{\pgfqpoint{2.976167in}{1.414815in}}{\pgfqpoint{2.968354in}{1.422629in}}%
\pgfpathcurveto{\pgfqpoint{2.960540in}{1.430442in}}{\pgfqpoint{2.949941in}{1.434833in}}{\pgfqpoint{2.938891in}{1.434833in}}%
\pgfpathcurveto{\pgfqpoint{2.927841in}{1.434833in}}{\pgfqpoint{2.917242in}{1.430442in}}{\pgfqpoint{2.909428in}{1.422629in}}%
\pgfpathcurveto{\pgfqpoint{2.901615in}{1.414815in}}{\pgfqpoint{2.897224in}{1.404216in}}{\pgfqpoint{2.897224in}{1.393166in}}%
\pgfpathcurveto{\pgfqpoint{2.897224in}{1.382116in}}{\pgfqpoint{2.901615in}{1.371517in}}{\pgfqpoint{2.909428in}{1.363703in}}%
\pgfpathcurveto{\pgfqpoint{2.917242in}{1.355890in}}{\pgfqpoint{2.927841in}{1.351499in}}{\pgfqpoint{2.938891in}{1.351499in}}%
\pgfpathclose%
\pgfusepath{stroke,fill}%
\end{pgfscope}%
\begin{pgfscope}%
\pgfpathrectangle{\pgfqpoint{0.511823in}{0.504323in}}{\pgfqpoint{3.218177in}{3.225677in}} %
\pgfusepath{clip}%
\pgfsetbuttcap%
\pgfsetroundjoin%
\definecolor{currentfill}{rgb}{0.501961,0.000000,0.000000}%
\pgfsetfillcolor{currentfill}%
\pgfsetfillopacity{0.400000}%
\pgfsetlinewidth{0.501875pt}%
\definecolor{currentstroke}{rgb}{0.501961,0.000000,0.000000}%
\pgfsetstrokecolor{currentstroke}%
\pgfsetstrokeopacity{0.400000}%
\pgfsetdash{}{0pt}%
\pgfpathmoveto{\pgfqpoint{3.067813in}{1.400024in}}%
\pgfpathcurveto{\pgfqpoint{3.078863in}{1.400024in}}{\pgfqpoint{3.089462in}{1.404415in}}{\pgfqpoint{3.097276in}{1.412228in}}%
\pgfpathcurveto{\pgfqpoint{3.105089in}{1.420042in}}{\pgfqpoint{3.109480in}{1.430641in}}{\pgfqpoint{3.109480in}{1.441691in}}%
\pgfpathcurveto{\pgfqpoint{3.109480in}{1.452741in}}{\pgfqpoint{3.105089in}{1.463340in}}{\pgfqpoint{3.097276in}{1.471154in}}%
\pgfpathcurveto{\pgfqpoint{3.089462in}{1.478967in}}{\pgfqpoint{3.078863in}{1.483358in}}{\pgfqpoint{3.067813in}{1.483358in}}%
\pgfpathcurveto{\pgfqpoint{3.056763in}{1.483358in}}{\pgfqpoint{3.046164in}{1.478967in}}{\pgfqpoint{3.038350in}{1.471154in}}%
\pgfpathcurveto{\pgfqpoint{3.030537in}{1.463340in}}{\pgfqpoint{3.026146in}{1.452741in}}{\pgfqpoint{3.026146in}{1.441691in}}%
\pgfpathcurveto{\pgfqpoint{3.026146in}{1.430641in}}{\pgfqpoint{3.030537in}{1.420042in}}{\pgfqpoint{3.038350in}{1.412228in}}%
\pgfpathcurveto{\pgfqpoint{3.046164in}{1.404415in}}{\pgfqpoint{3.056763in}{1.400024in}}{\pgfqpoint{3.067813in}{1.400024in}}%
\pgfpathclose%
\pgfusepath{stroke,fill}%
\end{pgfscope}%
\begin{pgfscope}%
\pgfpathrectangle{\pgfqpoint{0.511823in}{0.504323in}}{\pgfqpoint{3.218177in}{3.225677in}} %
\pgfusepath{clip}%
\pgfsetbuttcap%
\pgfsetroundjoin%
\definecolor{currentfill}{rgb}{0.501961,0.000000,0.000000}%
\pgfsetfillcolor{currentfill}%
\pgfsetfillopacity{0.400000}%
\pgfsetlinewidth{0.501875pt}%
\definecolor{currentstroke}{rgb}{0.501961,0.000000,0.000000}%
\pgfsetstrokecolor{currentstroke}%
\pgfsetstrokeopacity{0.400000}%
\pgfsetdash{}{0pt}%
\pgfpathmoveto{\pgfqpoint{3.143755in}{1.432323in}}%
\pgfpathcurveto{\pgfqpoint{3.154806in}{1.432323in}}{\pgfqpoint{3.165405in}{1.436713in}}{\pgfqpoint{3.173218in}{1.444527in}}%
\pgfpathcurveto{\pgfqpoint{3.181032in}{1.452340in}}{\pgfqpoint{3.185422in}{1.462939in}}{\pgfqpoint{3.185422in}{1.473989in}}%
\pgfpathcurveto{\pgfqpoint{3.185422in}{1.485039in}}{\pgfqpoint{3.181032in}{1.495639in}}{\pgfqpoint{3.173218in}{1.503452in}}%
\pgfpathcurveto{\pgfqpoint{3.165405in}{1.511266in}}{\pgfqpoint{3.154806in}{1.515656in}}{\pgfqpoint{3.143755in}{1.515656in}}%
\pgfpathcurveto{\pgfqpoint{3.132705in}{1.515656in}}{\pgfqpoint{3.122106in}{1.511266in}}{\pgfqpoint{3.114293in}{1.503452in}}%
\pgfpathcurveto{\pgfqpoint{3.106479in}{1.495639in}}{\pgfqpoint{3.102089in}{1.485039in}}{\pgfqpoint{3.102089in}{1.473989in}}%
\pgfpathcurveto{\pgfqpoint{3.102089in}{1.462939in}}{\pgfqpoint{3.106479in}{1.452340in}}{\pgfqpoint{3.114293in}{1.444527in}}%
\pgfpathcurveto{\pgfqpoint{3.122106in}{1.436713in}}{\pgfqpoint{3.132705in}{1.432323in}}{\pgfqpoint{3.143755in}{1.432323in}}%
\pgfpathclose%
\pgfusepath{stroke,fill}%
\end{pgfscope}%
\begin{pgfscope}%
\pgfpathrectangle{\pgfqpoint{0.511823in}{0.504323in}}{\pgfqpoint{3.218177in}{3.225677in}} %
\pgfusepath{clip}%
\pgfsetbuttcap%
\pgfsetroundjoin%
\definecolor{currentfill}{rgb}{0.501961,0.000000,0.000000}%
\pgfsetfillcolor{currentfill}%
\pgfsetfillopacity{0.400000}%
\pgfsetlinewidth{0.501875pt}%
\definecolor{currentstroke}{rgb}{0.501961,0.000000,0.000000}%
\pgfsetstrokecolor{currentstroke}%
\pgfsetstrokeopacity{0.400000}%
\pgfsetdash{}{0pt}%
\pgfpathmoveto{\pgfqpoint{3.046934in}{1.408848in}}%
\pgfpathcurveto{\pgfqpoint{3.057984in}{1.408848in}}{\pgfqpoint{3.068583in}{1.413238in}}{\pgfqpoint{3.076397in}{1.421052in}}%
\pgfpathcurveto{\pgfqpoint{3.084210in}{1.428865in}}{\pgfqpoint{3.088601in}{1.439464in}}{\pgfqpoint{3.088601in}{1.450514in}}%
\pgfpathcurveto{\pgfqpoint{3.088601in}{1.461565in}}{\pgfqpoint{3.084210in}{1.472164in}}{\pgfqpoint{3.076397in}{1.479977in}}%
\pgfpathcurveto{\pgfqpoint{3.068583in}{1.487791in}}{\pgfqpoint{3.057984in}{1.492181in}}{\pgfqpoint{3.046934in}{1.492181in}}%
\pgfpathcurveto{\pgfqpoint{3.035884in}{1.492181in}}{\pgfqpoint{3.025285in}{1.487791in}}{\pgfqpoint{3.017471in}{1.479977in}}%
\pgfpathcurveto{\pgfqpoint{3.009658in}{1.472164in}}{\pgfqpoint{3.005267in}{1.461565in}}{\pgfqpoint{3.005267in}{1.450514in}}%
\pgfpathcurveto{\pgfqpoint{3.005267in}{1.439464in}}{\pgfqpoint{3.009658in}{1.428865in}}{\pgfqpoint{3.017471in}{1.421052in}}%
\pgfpathcurveto{\pgfqpoint{3.025285in}{1.413238in}}{\pgfqpoint{3.035884in}{1.408848in}}{\pgfqpoint{3.046934in}{1.408848in}}%
\pgfpathclose%
\pgfusepath{stroke,fill}%
\end{pgfscope}%
\begin{pgfscope}%
\pgfpathrectangle{\pgfqpoint{0.511823in}{0.504323in}}{\pgfqpoint{3.218177in}{3.225677in}} %
\pgfusepath{clip}%
\pgfsetbuttcap%
\pgfsetroundjoin%
\definecolor{currentfill}{rgb}{0.501961,0.000000,0.000000}%
\pgfsetfillcolor{currentfill}%
\pgfsetfillopacity{0.400000}%
\pgfsetlinewidth{0.501875pt}%
\definecolor{currentstroke}{rgb}{0.501961,0.000000,0.000000}%
\pgfsetstrokecolor{currentstroke}%
\pgfsetstrokeopacity{0.400000}%
\pgfsetdash{}{0pt}%
\pgfpathmoveto{\pgfqpoint{3.182446in}{1.461216in}}%
\pgfpathcurveto{\pgfqpoint{3.193496in}{1.461216in}}{\pgfqpoint{3.204095in}{1.465606in}}{\pgfqpoint{3.211909in}{1.473420in}}%
\pgfpathcurveto{\pgfqpoint{3.219723in}{1.481234in}}{\pgfqpoint{3.224113in}{1.491833in}}{\pgfqpoint{3.224113in}{1.502883in}}%
\pgfpathcurveto{\pgfqpoint{3.224113in}{1.513933in}}{\pgfqpoint{3.219723in}{1.524532in}}{\pgfqpoint{3.211909in}{1.532346in}}%
\pgfpathcurveto{\pgfqpoint{3.204095in}{1.540159in}}{\pgfqpoint{3.193496in}{1.544549in}}{\pgfqpoint{3.182446in}{1.544549in}}%
\pgfpathcurveto{\pgfqpoint{3.171396in}{1.544549in}}{\pgfqpoint{3.160797in}{1.540159in}}{\pgfqpoint{3.152984in}{1.532346in}}%
\pgfpathcurveto{\pgfqpoint{3.145170in}{1.524532in}}{\pgfqpoint{3.140780in}{1.513933in}}{\pgfqpoint{3.140780in}{1.502883in}}%
\pgfpathcurveto{\pgfqpoint{3.140780in}{1.491833in}}{\pgfqpoint{3.145170in}{1.481234in}}{\pgfqpoint{3.152984in}{1.473420in}}%
\pgfpathcurveto{\pgfqpoint{3.160797in}{1.465606in}}{\pgfqpoint{3.171396in}{1.461216in}}{\pgfqpoint{3.182446in}{1.461216in}}%
\pgfpathclose%
\pgfusepath{stroke,fill}%
\end{pgfscope}%
\begin{pgfscope}%
\pgfpathrectangle{\pgfqpoint{0.511823in}{0.504323in}}{\pgfqpoint{3.218177in}{3.225677in}} %
\pgfusepath{clip}%
\pgfsetbuttcap%
\pgfsetroundjoin%
\definecolor{currentfill}{rgb}{0.501961,0.000000,0.000000}%
\pgfsetfillcolor{currentfill}%
\pgfsetfillopacity{0.400000}%
\pgfsetlinewidth{0.501875pt}%
\definecolor{currentstroke}{rgb}{0.501961,0.000000,0.000000}%
\pgfsetstrokecolor{currentstroke}%
\pgfsetstrokeopacity{0.400000}%
\pgfsetdash{}{0pt}%
\pgfpathmoveto{\pgfqpoint{3.173884in}{1.466604in}}%
\pgfpathcurveto{\pgfqpoint{3.184934in}{1.466604in}}{\pgfqpoint{3.195533in}{1.470995in}}{\pgfqpoint{3.203347in}{1.478808in}}%
\pgfpathcurveto{\pgfqpoint{3.211160in}{1.486622in}}{\pgfqpoint{3.215551in}{1.497221in}}{\pgfqpoint{3.215551in}{1.508271in}}%
\pgfpathcurveto{\pgfqpoint{3.215551in}{1.519321in}}{\pgfqpoint{3.211160in}{1.529920in}}{\pgfqpoint{3.203347in}{1.537734in}}%
\pgfpathcurveto{\pgfqpoint{3.195533in}{1.545548in}}{\pgfqpoint{3.184934in}{1.549938in}}{\pgfqpoint{3.173884in}{1.549938in}}%
\pgfpathcurveto{\pgfqpoint{3.162834in}{1.549938in}}{\pgfqpoint{3.152235in}{1.545548in}}{\pgfqpoint{3.144421in}{1.537734in}}%
\pgfpathcurveto{\pgfqpoint{3.136608in}{1.529920in}}{\pgfqpoint{3.132217in}{1.519321in}}{\pgfqpoint{3.132217in}{1.508271in}}%
\pgfpathcurveto{\pgfqpoint{3.132217in}{1.497221in}}{\pgfqpoint{3.136608in}{1.486622in}}{\pgfqpoint{3.144421in}{1.478808in}}%
\pgfpathcurveto{\pgfqpoint{3.152235in}{1.470995in}}{\pgfqpoint{3.162834in}{1.466604in}}{\pgfqpoint{3.173884in}{1.466604in}}%
\pgfpathclose%
\pgfusepath{stroke,fill}%
\end{pgfscope}%
\begin{pgfscope}%
\pgfpathrectangle{\pgfqpoint{0.511823in}{0.504323in}}{\pgfqpoint{3.218177in}{3.225677in}} %
\pgfusepath{clip}%
\pgfsetbuttcap%
\pgfsetroundjoin%
\definecolor{currentfill}{rgb}{0.501961,0.000000,0.000000}%
\pgfsetfillcolor{currentfill}%
\pgfsetfillopacity{0.400000}%
\pgfsetlinewidth{0.501875pt}%
\definecolor{currentstroke}{rgb}{0.501961,0.000000,0.000000}%
\pgfsetstrokecolor{currentstroke}%
\pgfsetstrokeopacity{0.400000}%
\pgfsetdash{}{0pt}%
\pgfpathmoveto{\pgfqpoint{2.988618in}{1.412633in}}%
\pgfpathcurveto{\pgfqpoint{2.999668in}{1.412633in}}{\pgfqpoint{3.010267in}{1.417023in}}{\pgfqpoint{3.018080in}{1.424837in}}%
\pgfpathcurveto{\pgfqpoint{3.025894in}{1.432650in}}{\pgfqpoint{3.030284in}{1.443249in}}{\pgfqpoint{3.030284in}{1.454300in}}%
\pgfpathcurveto{\pgfqpoint{3.030284in}{1.465350in}}{\pgfqpoint{3.025894in}{1.475949in}}{\pgfqpoint{3.018080in}{1.483762in}}%
\pgfpathcurveto{\pgfqpoint{3.010267in}{1.491576in}}{\pgfqpoint{2.999668in}{1.495966in}}{\pgfqpoint{2.988618in}{1.495966in}}%
\pgfpathcurveto{\pgfqpoint{2.977567in}{1.495966in}}{\pgfqpoint{2.966968in}{1.491576in}}{\pgfqpoint{2.959155in}{1.483762in}}%
\pgfpathcurveto{\pgfqpoint{2.951341in}{1.475949in}}{\pgfqpoint{2.946951in}{1.465350in}}{\pgfqpoint{2.946951in}{1.454300in}}%
\pgfpathcurveto{\pgfqpoint{2.946951in}{1.443249in}}{\pgfqpoint{2.951341in}{1.432650in}}{\pgfqpoint{2.959155in}{1.424837in}}%
\pgfpathcurveto{\pgfqpoint{2.966968in}{1.417023in}}{\pgfqpoint{2.977567in}{1.412633in}}{\pgfqpoint{2.988618in}{1.412633in}}%
\pgfpathclose%
\pgfusepath{stroke,fill}%
\end{pgfscope}%
\begin{pgfscope}%
\pgfpathrectangle{\pgfqpoint{0.511823in}{0.504323in}}{\pgfqpoint{3.218177in}{3.225677in}} %
\pgfusepath{clip}%
\pgfsetbuttcap%
\pgfsetroundjoin%
\definecolor{currentfill}{rgb}{0.501961,0.000000,0.000000}%
\pgfsetfillcolor{currentfill}%
\pgfsetfillopacity{0.400000}%
\pgfsetlinewidth{0.501875pt}%
\definecolor{currentstroke}{rgb}{0.501961,0.000000,0.000000}%
\pgfsetstrokecolor{currentstroke}%
\pgfsetstrokeopacity{0.400000}%
\pgfsetdash{}{0pt}%
\pgfpathmoveto{\pgfqpoint{3.199410in}{1.491727in}}%
\pgfpathcurveto{\pgfqpoint{3.210460in}{1.491727in}}{\pgfqpoint{3.221060in}{1.496117in}}{\pgfqpoint{3.228873in}{1.503931in}}%
\pgfpathcurveto{\pgfqpoint{3.236687in}{1.511744in}}{\pgfqpoint{3.241077in}{1.522343in}}{\pgfqpoint{3.241077in}{1.533394in}}%
\pgfpathcurveto{\pgfqpoint{3.241077in}{1.544444in}}{\pgfqpoint{3.236687in}{1.555043in}}{\pgfqpoint{3.228873in}{1.562856in}}%
\pgfpathcurveto{\pgfqpoint{3.221060in}{1.570670in}}{\pgfqpoint{3.210460in}{1.575060in}}{\pgfqpoint{3.199410in}{1.575060in}}%
\pgfpathcurveto{\pgfqpoint{3.188360in}{1.575060in}}{\pgfqpoint{3.177761in}{1.570670in}}{\pgfqpoint{3.169948in}{1.562856in}}%
\pgfpathcurveto{\pgfqpoint{3.162134in}{1.555043in}}{\pgfqpoint{3.157744in}{1.544444in}}{\pgfqpoint{3.157744in}{1.533394in}}%
\pgfpathcurveto{\pgfqpoint{3.157744in}{1.522343in}}{\pgfqpoint{3.162134in}{1.511744in}}{\pgfqpoint{3.169948in}{1.503931in}}%
\pgfpathcurveto{\pgfqpoint{3.177761in}{1.496117in}}{\pgfqpoint{3.188360in}{1.491727in}}{\pgfqpoint{3.199410in}{1.491727in}}%
\pgfpathclose%
\pgfusepath{stroke,fill}%
\end{pgfscope}%
\begin{pgfscope}%
\pgfpathrectangle{\pgfqpoint{0.511823in}{0.504323in}}{\pgfqpoint{3.218177in}{3.225677in}} %
\pgfusepath{clip}%
\pgfsetbuttcap%
\pgfsetroundjoin%
\definecolor{currentfill}{rgb}{0.501961,0.000000,0.000000}%
\pgfsetfillcolor{currentfill}%
\pgfsetfillopacity{0.400000}%
\pgfsetlinewidth{0.501875pt}%
\definecolor{currentstroke}{rgb}{0.501961,0.000000,0.000000}%
\pgfsetstrokecolor{currentstroke}%
\pgfsetstrokeopacity{0.400000}%
\pgfsetdash{}{0pt}%
\pgfpathmoveto{\pgfqpoint{2.997601in}{1.430985in}}%
\pgfpathcurveto{\pgfqpoint{3.008651in}{1.430985in}}{\pgfqpoint{3.019250in}{1.435375in}}{\pgfqpoint{3.027063in}{1.443189in}}%
\pgfpathcurveto{\pgfqpoint{3.034877in}{1.451002in}}{\pgfqpoint{3.039267in}{1.461601in}}{\pgfqpoint{3.039267in}{1.472651in}}%
\pgfpathcurveto{\pgfqpoint{3.039267in}{1.483702in}}{\pgfqpoint{3.034877in}{1.494301in}}{\pgfqpoint{3.027063in}{1.502114in}}%
\pgfpathcurveto{\pgfqpoint{3.019250in}{1.509928in}}{\pgfqpoint{3.008651in}{1.514318in}}{\pgfqpoint{2.997601in}{1.514318in}}%
\pgfpathcurveto{\pgfqpoint{2.986551in}{1.514318in}}{\pgfqpoint{2.975951in}{1.509928in}}{\pgfqpoint{2.968138in}{1.502114in}}%
\pgfpathcurveto{\pgfqpoint{2.960324in}{1.494301in}}{\pgfqpoint{2.955934in}{1.483702in}}{\pgfqpoint{2.955934in}{1.472651in}}%
\pgfpathcurveto{\pgfqpoint{2.955934in}{1.461601in}}{\pgfqpoint{2.960324in}{1.451002in}}{\pgfqpoint{2.968138in}{1.443189in}}%
\pgfpathcurveto{\pgfqpoint{2.975951in}{1.435375in}}{\pgfqpoint{2.986551in}{1.430985in}}{\pgfqpoint{2.997601in}{1.430985in}}%
\pgfpathclose%
\pgfusepath{stroke,fill}%
\end{pgfscope}%
\begin{pgfscope}%
\pgfpathrectangle{\pgfqpoint{0.511823in}{0.504323in}}{\pgfqpoint{3.218177in}{3.225677in}} %
\pgfusepath{clip}%
\pgfsetbuttcap%
\pgfsetroundjoin%
\definecolor{currentfill}{rgb}{0.501961,0.000000,0.000000}%
\pgfsetfillcolor{currentfill}%
\pgfsetfillopacity{0.400000}%
\pgfsetlinewidth{0.501875pt}%
\definecolor{currentstroke}{rgb}{0.501961,0.000000,0.000000}%
\pgfsetstrokecolor{currentstroke}%
\pgfsetstrokeopacity{0.400000}%
\pgfsetdash{}{0pt}%
\pgfpathmoveto{\pgfqpoint{3.260412in}{1.529528in}}%
\pgfpathcurveto{\pgfqpoint{3.271462in}{1.529528in}}{\pgfqpoint{3.282061in}{1.533918in}}{\pgfqpoint{3.289875in}{1.541731in}}%
\pgfpathcurveto{\pgfqpoint{3.297688in}{1.549545in}}{\pgfqpoint{3.302078in}{1.560144in}}{\pgfqpoint{3.302078in}{1.571194in}}%
\pgfpathcurveto{\pgfqpoint{3.302078in}{1.582244in}}{\pgfqpoint{3.297688in}{1.592843in}}{\pgfqpoint{3.289875in}{1.600657in}}%
\pgfpathcurveto{\pgfqpoint{3.282061in}{1.608471in}}{\pgfqpoint{3.271462in}{1.612861in}}{\pgfqpoint{3.260412in}{1.612861in}}%
\pgfpathcurveto{\pgfqpoint{3.249362in}{1.612861in}}{\pgfqpoint{3.238763in}{1.608471in}}{\pgfqpoint{3.230949in}{1.600657in}}%
\pgfpathcurveto{\pgfqpoint{3.223135in}{1.592843in}}{\pgfqpoint{3.218745in}{1.582244in}}{\pgfqpoint{3.218745in}{1.571194in}}%
\pgfpathcurveto{\pgfqpoint{3.218745in}{1.560144in}}{\pgfqpoint{3.223135in}{1.549545in}}{\pgfqpoint{3.230949in}{1.541731in}}%
\pgfpathcurveto{\pgfqpoint{3.238763in}{1.533918in}}{\pgfqpoint{3.249362in}{1.529528in}}{\pgfqpoint{3.260412in}{1.529528in}}%
\pgfpathclose%
\pgfusepath{stroke,fill}%
\end{pgfscope}%
\begin{pgfscope}%
\pgfpathrectangle{\pgfqpoint{0.511823in}{0.504323in}}{\pgfqpoint{3.218177in}{3.225677in}} %
\pgfusepath{clip}%
\pgfsetbuttcap%
\pgfsetroundjoin%
\definecolor{currentfill}{rgb}{0.501961,0.000000,0.000000}%
\pgfsetfillcolor{currentfill}%
\pgfsetfillopacity{0.400000}%
\pgfsetlinewidth{0.501875pt}%
\definecolor{currentstroke}{rgb}{0.501961,0.000000,0.000000}%
\pgfsetstrokecolor{currentstroke}%
\pgfsetstrokeopacity{0.400000}%
\pgfsetdash{}{0pt}%
\pgfpathmoveto{\pgfqpoint{3.197920in}{1.516307in}}%
\pgfpathcurveto{\pgfqpoint{3.208970in}{1.516307in}}{\pgfqpoint{3.219569in}{1.520697in}}{\pgfqpoint{3.227383in}{1.528511in}}%
\pgfpathcurveto{\pgfqpoint{3.235197in}{1.536324in}}{\pgfqpoint{3.239587in}{1.546923in}}{\pgfqpoint{3.239587in}{1.557974in}}%
\pgfpathcurveto{\pgfqpoint{3.239587in}{1.569024in}}{\pgfqpoint{3.235197in}{1.579623in}}{\pgfqpoint{3.227383in}{1.587436in}}%
\pgfpathcurveto{\pgfqpoint{3.219569in}{1.595250in}}{\pgfqpoint{3.208970in}{1.599640in}}{\pgfqpoint{3.197920in}{1.599640in}}%
\pgfpathcurveto{\pgfqpoint{3.186870in}{1.599640in}}{\pgfqpoint{3.176271in}{1.595250in}}{\pgfqpoint{3.168458in}{1.587436in}}%
\pgfpathcurveto{\pgfqpoint{3.160644in}{1.579623in}}{\pgfqpoint{3.156254in}{1.569024in}}{\pgfqpoint{3.156254in}{1.557974in}}%
\pgfpathcurveto{\pgfqpoint{3.156254in}{1.546923in}}{\pgfqpoint{3.160644in}{1.536324in}}{\pgfqpoint{3.168458in}{1.528511in}}%
\pgfpathcurveto{\pgfqpoint{3.176271in}{1.520697in}}{\pgfqpoint{3.186870in}{1.516307in}}{\pgfqpoint{3.197920in}{1.516307in}}%
\pgfpathclose%
\pgfusepath{stroke,fill}%
\end{pgfscope}%
\begin{pgfscope}%
\pgfpathrectangle{\pgfqpoint{0.511823in}{0.504323in}}{\pgfqpoint{3.218177in}{3.225677in}} %
\pgfusepath{clip}%
\pgfsetbuttcap%
\pgfsetroundjoin%
\definecolor{currentfill}{rgb}{0.501961,0.000000,0.000000}%
\pgfsetfillcolor{currentfill}%
\pgfsetfillopacity{0.400000}%
\pgfsetlinewidth{0.501875pt}%
\definecolor{currentstroke}{rgb}{0.501961,0.000000,0.000000}%
\pgfsetstrokecolor{currentstroke}%
\pgfsetstrokeopacity{0.400000}%
\pgfsetdash{}{0pt}%
\pgfpathmoveto{\pgfqpoint{3.439036in}{1.609667in}}%
\pgfpathcurveto{\pgfqpoint{3.450086in}{1.609667in}}{\pgfqpoint{3.460685in}{1.614058in}}{\pgfqpoint{3.468499in}{1.621871in}}%
\pgfpathcurveto{\pgfqpoint{3.476312in}{1.629685in}}{\pgfqpoint{3.480702in}{1.640284in}}{\pgfqpoint{3.480702in}{1.651334in}}%
\pgfpathcurveto{\pgfqpoint{3.480702in}{1.662384in}}{\pgfqpoint{3.476312in}{1.672983in}}{\pgfqpoint{3.468499in}{1.680797in}}%
\pgfpathcurveto{\pgfqpoint{3.460685in}{1.688610in}}{\pgfqpoint{3.450086in}{1.693001in}}{\pgfqpoint{3.439036in}{1.693001in}}%
\pgfpathcurveto{\pgfqpoint{3.427986in}{1.693001in}}{\pgfqpoint{3.417387in}{1.688610in}}{\pgfqpoint{3.409573in}{1.680797in}}%
\pgfpathcurveto{\pgfqpoint{3.401759in}{1.672983in}}{\pgfqpoint{3.397369in}{1.662384in}}{\pgfqpoint{3.397369in}{1.651334in}}%
\pgfpathcurveto{\pgfqpoint{3.397369in}{1.640284in}}{\pgfqpoint{3.401759in}{1.629685in}}{\pgfqpoint{3.409573in}{1.621871in}}%
\pgfpathcurveto{\pgfqpoint{3.417387in}{1.614058in}}{\pgfqpoint{3.427986in}{1.609667in}}{\pgfqpoint{3.439036in}{1.609667in}}%
\pgfpathclose%
\pgfusepath{stroke,fill}%
\end{pgfscope}%
\begin{pgfscope}%
\pgfpathrectangle{\pgfqpoint{0.511823in}{0.504323in}}{\pgfqpoint{3.218177in}{3.225677in}} %
\pgfusepath{clip}%
\pgfsetbuttcap%
\pgfsetroundjoin%
\definecolor{currentfill}{rgb}{0.501961,0.000000,0.000000}%
\pgfsetfillcolor{currentfill}%
\pgfsetfillopacity{0.400000}%
\pgfsetlinewidth{0.501875pt}%
\definecolor{currentstroke}{rgb}{0.501961,0.000000,0.000000}%
\pgfsetstrokecolor{currentstroke}%
\pgfsetstrokeopacity{0.400000}%
\pgfsetdash{}{0pt}%
\pgfpathmoveto{\pgfqpoint{3.266494in}{1.557521in}}%
\pgfpathcurveto{\pgfqpoint{3.277544in}{1.557521in}}{\pgfqpoint{3.288143in}{1.561912in}}{\pgfqpoint{3.295957in}{1.569725in}}%
\pgfpathcurveto{\pgfqpoint{3.303770in}{1.577539in}}{\pgfqpoint{3.308161in}{1.588138in}}{\pgfqpoint{3.308161in}{1.599188in}}%
\pgfpathcurveto{\pgfqpoint{3.308161in}{1.610238in}}{\pgfqpoint{3.303770in}{1.620837in}}{\pgfqpoint{3.295957in}{1.628651in}}%
\pgfpathcurveto{\pgfqpoint{3.288143in}{1.636465in}}{\pgfqpoint{3.277544in}{1.640855in}}{\pgfqpoint{3.266494in}{1.640855in}}%
\pgfpathcurveto{\pgfqpoint{3.255444in}{1.640855in}}{\pgfqpoint{3.244845in}{1.636465in}}{\pgfqpoint{3.237031in}{1.628651in}}%
\pgfpathcurveto{\pgfqpoint{3.229218in}{1.620837in}}{\pgfqpoint{3.224827in}{1.610238in}}{\pgfqpoint{3.224827in}{1.599188in}}%
\pgfpathcurveto{\pgfqpoint{3.224827in}{1.588138in}}{\pgfqpoint{3.229218in}{1.577539in}}{\pgfqpoint{3.237031in}{1.569725in}}%
\pgfpathcurveto{\pgfqpoint{3.244845in}{1.561912in}}{\pgfqpoint{3.255444in}{1.557521in}}{\pgfqpoint{3.266494in}{1.557521in}}%
\pgfpathclose%
\pgfusepath{stroke,fill}%
\end{pgfscope}%
\begin{pgfscope}%
\pgfpathrectangle{\pgfqpoint{0.511823in}{0.504323in}}{\pgfqpoint{3.218177in}{3.225677in}} %
\pgfusepath{clip}%
\pgfsetbuttcap%
\pgfsetroundjoin%
\definecolor{currentfill}{rgb}{0.501961,0.000000,0.000000}%
\pgfsetfillcolor{currentfill}%
\pgfsetfillopacity{0.400000}%
\pgfsetlinewidth{0.501875pt}%
\definecolor{currentstroke}{rgb}{0.501961,0.000000,0.000000}%
\pgfsetstrokecolor{currentstroke}%
\pgfsetstrokeopacity{0.400000}%
\pgfsetdash{}{0pt}%
\pgfpathmoveto{\pgfqpoint{3.135684in}{1.519223in}}%
\pgfpathcurveto{\pgfqpoint{3.146734in}{1.519223in}}{\pgfqpoint{3.157333in}{1.523613in}}{\pgfqpoint{3.165146in}{1.531427in}}%
\pgfpathcurveto{\pgfqpoint{3.172960in}{1.539240in}}{\pgfqpoint{3.177350in}{1.549839in}}{\pgfqpoint{3.177350in}{1.560889in}}%
\pgfpathcurveto{\pgfqpoint{3.177350in}{1.571939in}}{\pgfqpoint{3.172960in}{1.582538in}}{\pgfqpoint{3.165146in}{1.590352in}}%
\pgfpathcurveto{\pgfqpoint{3.157333in}{1.598166in}}{\pgfqpoint{3.146734in}{1.602556in}}{\pgfqpoint{3.135684in}{1.602556in}}%
\pgfpathcurveto{\pgfqpoint{3.124633in}{1.602556in}}{\pgfqpoint{3.114034in}{1.598166in}}{\pgfqpoint{3.106221in}{1.590352in}}%
\pgfpathcurveto{\pgfqpoint{3.098407in}{1.582538in}}{\pgfqpoint{3.094017in}{1.571939in}}{\pgfqpoint{3.094017in}{1.560889in}}%
\pgfpathcurveto{\pgfqpoint{3.094017in}{1.549839in}}{\pgfqpoint{3.098407in}{1.539240in}}{\pgfqpoint{3.106221in}{1.531427in}}%
\pgfpathcurveto{\pgfqpoint{3.114034in}{1.523613in}}{\pgfqpoint{3.124633in}{1.519223in}}{\pgfqpoint{3.135684in}{1.519223in}}%
\pgfpathclose%
\pgfusepath{stroke,fill}%
\end{pgfscope}%
\begin{pgfscope}%
\pgfpathrectangle{\pgfqpoint{0.511823in}{0.504323in}}{\pgfqpoint{3.218177in}{3.225677in}} %
\pgfusepath{clip}%
\pgfsetbuttcap%
\pgfsetroundjoin%
\definecolor{currentfill}{rgb}{0.501961,0.000000,0.000000}%
\pgfsetfillcolor{currentfill}%
\pgfsetfillopacity{0.400000}%
\pgfsetlinewidth{0.501875pt}%
\definecolor{currentstroke}{rgb}{0.501961,0.000000,0.000000}%
\pgfsetstrokecolor{currentstroke}%
\pgfsetstrokeopacity{0.400000}%
\pgfsetdash{}{0pt}%
\pgfpathmoveto{\pgfqpoint{3.200465in}{1.550954in}}%
\pgfpathcurveto{\pgfqpoint{3.211515in}{1.550954in}}{\pgfqpoint{3.222114in}{1.555345in}}{\pgfqpoint{3.229928in}{1.563158in}}%
\pgfpathcurveto{\pgfqpoint{3.237741in}{1.570972in}}{\pgfqpoint{3.242132in}{1.581571in}}{\pgfqpoint{3.242132in}{1.592621in}}%
\pgfpathcurveto{\pgfqpoint{3.242132in}{1.603671in}}{\pgfqpoint{3.237741in}{1.614270in}}{\pgfqpoint{3.229928in}{1.622084in}}%
\pgfpathcurveto{\pgfqpoint{3.222114in}{1.629898in}}{\pgfqpoint{3.211515in}{1.634288in}}{\pgfqpoint{3.200465in}{1.634288in}}%
\pgfpathcurveto{\pgfqpoint{3.189415in}{1.634288in}}{\pgfqpoint{3.178816in}{1.629898in}}{\pgfqpoint{3.171002in}{1.622084in}}%
\pgfpathcurveto{\pgfqpoint{3.163188in}{1.614270in}}{\pgfqpoint{3.158798in}{1.603671in}}{\pgfqpoint{3.158798in}{1.592621in}}%
\pgfpathcurveto{\pgfqpoint{3.158798in}{1.581571in}}{\pgfqpoint{3.163188in}{1.570972in}}{\pgfqpoint{3.171002in}{1.563158in}}%
\pgfpathcurveto{\pgfqpoint{3.178816in}{1.555345in}}{\pgfqpoint{3.189415in}{1.550954in}}{\pgfqpoint{3.200465in}{1.550954in}}%
\pgfpathclose%
\pgfusepath{stroke,fill}%
\end{pgfscope}%
\begin{pgfscope}%
\pgfpathrectangle{\pgfqpoint{0.511823in}{0.504323in}}{\pgfqpoint{3.218177in}{3.225677in}} %
\pgfusepath{clip}%
\pgfsetbuttcap%
\pgfsetroundjoin%
\definecolor{currentfill}{rgb}{0.501961,0.000000,0.000000}%
\pgfsetfillcolor{currentfill}%
\pgfsetfillopacity{0.400000}%
\pgfsetlinewidth{0.501875pt}%
\definecolor{currentstroke}{rgb}{0.501961,0.000000,0.000000}%
\pgfsetstrokecolor{currentstroke}%
\pgfsetstrokeopacity{0.400000}%
\pgfsetdash{}{0pt}%
\pgfpathmoveto{\pgfqpoint{3.172962in}{1.549385in}}%
\pgfpathcurveto{\pgfqpoint{3.184012in}{1.549385in}}{\pgfqpoint{3.194611in}{1.553775in}}{\pgfqpoint{3.202425in}{1.561589in}}%
\pgfpathcurveto{\pgfqpoint{3.210239in}{1.569402in}}{\pgfqpoint{3.214629in}{1.580001in}}{\pgfqpoint{3.214629in}{1.591051in}}%
\pgfpathcurveto{\pgfqpoint{3.214629in}{1.602101in}}{\pgfqpoint{3.210239in}{1.612700in}}{\pgfqpoint{3.202425in}{1.620514in}}%
\pgfpathcurveto{\pgfqpoint{3.194611in}{1.628328in}}{\pgfqpoint{3.184012in}{1.632718in}}{\pgfqpoint{3.172962in}{1.632718in}}%
\pgfpathcurveto{\pgfqpoint{3.161912in}{1.632718in}}{\pgfqpoint{3.151313in}{1.628328in}}{\pgfqpoint{3.143499in}{1.620514in}}%
\pgfpathcurveto{\pgfqpoint{3.135686in}{1.612700in}}{\pgfqpoint{3.131295in}{1.602101in}}{\pgfqpoint{3.131295in}{1.591051in}}%
\pgfpathcurveto{\pgfqpoint{3.131295in}{1.580001in}}{\pgfqpoint{3.135686in}{1.569402in}}{\pgfqpoint{3.143499in}{1.561589in}}%
\pgfpathcurveto{\pgfqpoint{3.151313in}{1.553775in}}{\pgfqpoint{3.161912in}{1.549385in}}{\pgfqpoint{3.172962in}{1.549385in}}%
\pgfpathclose%
\pgfusepath{stroke,fill}%
\end{pgfscope}%
\begin{pgfscope}%
\pgfpathrectangle{\pgfqpoint{0.511823in}{0.504323in}}{\pgfqpoint{3.218177in}{3.225677in}} %
\pgfusepath{clip}%
\pgfsetbuttcap%
\pgfsetroundjoin%
\definecolor{currentfill}{rgb}{0.501961,0.000000,0.000000}%
\pgfsetfillcolor{currentfill}%
\pgfsetfillopacity{0.400000}%
\pgfsetlinewidth{0.501875pt}%
\definecolor{currentstroke}{rgb}{0.501961,0.000000,0.000000}%
\pgfsetstrokecolor{currentstroke}%
\pgfsetstrokeopacity{0.400000}%
\pgfsetdash{}{0pt}%
\pgfpathmoveto{\pgfqpoint{3.060495in}{1.516280in}}%
\pgfpathcurveto{\pgfqpoint{3.071545in}{1.516280in}}{\pgfqpoint{3.082144in}{1.520670in}}{\pgfqpoint{3.089957in}{1.528484in}}%
\pgfpathcurveto{\pgfqpoint{3.097771in}{1.536298in}}{\pgfqpoint{3.102161in}{1.546897in}}{\pgfqpoint{3.102161in}{1.557947in}}%
\pgfpathcurveto{\pgfqpoint{3.102161in}{1.568997in}}{\pgfqpoint{3.097771in}{1.579596in}}{\pgfqpoint{3.089957in}{1.587409in}}%
\pgfpathcurveto{\pgfqpoint{3.082144in}{1.595223in}}{\pgfqpoint{3.071545in}{1.599613in}}{\pgfqpoint{3.060495in}{1.599613in}}%
\pgfpathcurveto{\pgfqpoint{3.049445in}{1.599613in}}{\pgfqpoint{3.038845in}{1.595223in}}{\pgfqpoint{3.031032in}{1.587409in}}%
\pgfpathcurveto{\pgfqpoint{3.023218in}{1.579596in}}{\pgfqpoint{3.018828in}{1.568997in}}{\pgfqpoint{3.018828in}{1.557947in}}%
\pgfpathcurveto{\pgfqpoint{3.018828in}{1.546897in}}{\pgfqpoint{3.023218in}{1.536298in}}{\pgfqpoint{3.031032in}{1.528484in}}%
\pgfpathcurveto{\pgfqpoint{3.038845in}{1.520670in}}{\pgfqpoint{3.049445in}{1.516280in}}{\pgfqpoint{3.060495in}{1.516280in}}%
\pgfpathclose%
\pgfusepath{stroke,fill}%
\end{pgfscope}%
\begin{pgfscope}%
\pgfpathrectangle{\pgfqpoint{0.511823in}{0.504323in}}{\pgfqpoint{3.218177in}{3.225677in}} %
\pgfusepath{clip}%
\pgfsetbuttcap%
\pgfsetroundjoin%
\definecolor{currentfill}{rgb}{0.501961,0.000000,0.000000}%
\pgfsetfillcolor{currentfill}%
\pgfsetfillopacity{0.400000}%
\pgfsetlinewidth{0.501875pt}%
\definecolor{currentstroke}{rgb}{0.501961,0.000000,0.000000}%
\pgfsetstrokecolor{currentstroke}%
\pgfsetstrokeopacity{0.400000}%
\pgfsetdash{}{0pt}%
\pgfpathmoveto{\pgfqpoint{2.969668in}{1.490496in}}%
\pgfpathcurveto{\pgfqpoint{2.980718in}{1.490496in}}{\pgfqpoint{2.991317in}{1.494886in}}{\pgfqpoint{2.999130in}{1.502700in}}%
\pgfpathcurveto{\pgfqpoint{3.006944in}{1.510514in}}{\pgfqpoint{3.011334in}{1.521113in}}{\pgfqpoint{3.011334in}{1.532163in}}%
\pgfpathcurveto{\pgfqpoint{3.011334in}{1.543213in}}{\pgfqpoint{3.006944in}{1.553812in}}{\pgfqpoint{2.999130in}{1.561626in}}%
\pgfpathcurveto{\pgfqpoint{2.991317in}{1.569439in}}{\pgfqpoint{2.980718in}{1.573830in}}{\pgfqpoint{2.969668in}{1.573830in}}%
\pgfpathcurveto{\pgfqpoint{2.958617in}{1.573830in}}{\pgfqpoint{2.948018in}{1.569439in}}{\pgfqpoint{2.940205in}{1.561626in}}%
\pgfpathcurveto{\pgfqpoint{2.932391in}{1.553812in}}{\pgfqpoint{2.928001in}{1.543213in}}{\pgfqpoint{2.928001in}{1.532163in}}%
\pgfpathcurveto{\pgfqpoint{2.928001in}{1.521113in}}{\pgfqpoint{2.932391in}{1.510514in}}{\pgfqpoint{2.940205in}{1.502700in}}%
\pgfpathcurveto{\pgfqpoint{2.948018in}{1.494886in}}{\pgfqpoint{2.958617in}{1.490496in}}{\pgfqpoint{2.969668in}{1.490496in}}%
\pgfpathclose%
\pgfusepath{stroke,fill}%
\end{pgfscope}%
\begin{pgfscope}%
\pgfpathrectangle{\pgfqpoint{0.511823in}{0.504323in}}{\pgfqpoint{3.218177in}{3.225677in}} %
\pgfusepath{clip}%
\pgfsetbuttcap%
\pgfsetroundjoin%
\definecolor{currentfill}{rgb}{0.501961,0.000000,0.000000}%
\pgfsetfillcolor{currentfill}%
\pgfsetfillopacity{0.400000}%
\pgfsetlinewidth{0.501875pt}%
\definecolor{currentstroke}{rgb}{0.501961,0.000000,0.000000}%
\pgfsetstrokecolor{currentstroke}%
\pgfsetstrokeopacity{0.400000}%
\pgfsetdash{}{0pt}%
\pgfpathmoveto{\pgfqpoint{3.280702in}{1.615216in}}%
\pgfpathcurveto{\pgfqpoint{3.291752in}{1.615216in}}{\pgfqpoint{3.302351in}{1.619607in}}{\pgfqpoint{3.310165in}{1.627420in}}%
\pgfpathcurveto{\pgfqpoint{3.317979in}{1.635234in}}{\pgfqpoint{3.322369in}{1.645833in}}{\pgfqpoint{3.322369in}{1.656883in}}%
\pgfpathcurveto{\pgfqpoint{3.322369in}{1.667933in}}{\pgfqpoint{3.317979in}{1.678532in}}{\pgfqpoint{3.310165in}{1.686346in}}%
\pgfpathcurveto{\pgfqpoint{3.302351in}{1.694160in}}{\pgfqpoint{3.291752in}{1.698550in}}{\pgfqpoint{3.280702in}{1.698550in}}%
\pgfpathcurveto{\pgfqpoint{3.269652in}{1.698550in}}{\pgfqpoint{3.259053in}{1.694160in}}{\pgfqpoint{3.251240in}{1.686346in}}%
\pgfpathcurveto{\pgfqpoint{3.243426in}{1.678532in}}{\pgfqpoint{3.239036in}{1.667933in}}{\pgfqpoint{3.239036in}{1.656883in}}%
\pgfpathcurveto{\pgfqpoint{3.239036in}{1.645833in}}{\pgfqpoint{3.243426in}{1.635234in}}{\pgfqpoint{3.251240in}{1.627420in}}%
\pgfpathcurveto{\pgfqpoint{3.259053in}{1.619607in}}{\pgfqpoint{3.269652in}{1.615216in}}{\pgfqpoint{3.280702in}{1.615216in}}%
\pgfpathclose%
\pgfusepath{stroke,fill}%
\end{pgfscope}%
\begin{pgfscope}%
\pgfpathrectangle{\pgfqpoint{0.511823in}{0.504323in}}{\pgfqpoint{3.218177in}{3.225677in}} %
\pgfusepath{clip}%
\pgfsetbuttcap%
\pgfsetroundjoin%
\definecolor{currentfill}{rgb}{0.501961,0.000000,0.000000}%
\pgfsetfillcolor{currentfill}%
\pgfsetfillopacity{0.400000}%
\pgfsetlinewidth{0.501875pt}%
\definecolor{currentstroke}{rgb}{0.501961,0.000000,0.000000}%
\pgfsetstrokecolor{currentstroke}%
\pgfsetstrokeopacity{0.400000}%
\pgfsetdash{}{0pt}%
\pgfpathmoveto{\pgfqpoint{3.329010in}{1.642396in}}%
\pgfpathcurveto{\pgfqpoint{3.340060in}{1.642396in}}{\pgfqpoint{3.350659in}{1.646786in}}{\pgfqpoint{3.358473in}{1.654600in}}%
\pgfpathcurveto{\pgfqpoint{3.366286in}{1.662413in}}{\pgfqpoint{3.370677in}{1.673012in}}{\pgfqpoint{3.370677in}{1.684063in}}%
\pgfpathcurveto{\pgfqpoint{3.370677in}{1.695113in}}{\pgfqpoint{3.366286in}{1.705712in}}{\pgfqpoint{3.358473in}{1.713525in}}%
\pgfpathcurveto{\pgfqpoint{3.350659in}{1.721339in}}{\pgfqpoint{3.340060in}{1.725729in}}{\pgfqpoint{3.329010in}{1.725729in}}%
\pgfpathcurveto{\pgfqpoint{3.317960in}{1.725729in}}{\pgfqpoint{3.307361in}{1.721339in}}{\pgfqpoint{3.299547in}{1.713525in}}%
\pgfpathcurveto{\pgfqpoint{3.291734in}{1.705712in}}{\pgfqpoint{3.287343in}{1.695113in}}{\pgfqpoint{3.287343in}{1.684063in}}%
\pgfpathcurveto{\pgfqpoint{3.287343in}{1.673012in}}{\pgfqpoint{3.291734in}{1.662413in}}{\pgfqpoint{3.299547in}{1.654600in}}%
\pgfpathcurveto{\pgfqpoint{3.307361in}{1.646786in}}{\pgfqpoint{3.317960in}{1.642396in}}{\pgfqpoint{3.329010in}{1.642396in}}%
\pgfpathclose%
\pgfusepath{stroke,fill}%
\end{pgfscope}%
\begin{pgfscope}%
\pgfpathrectangle{\pgfqpoint{0.511823in}{0.504323in}}{\pgfqpoint{3.218177in}{3.225677in}} %
\pgfusepath{clip}%
\pgfsetbuttcap%
\pgfsetroundjoin%
\definecolor{currentfill}{rgb}{0.501961,0.000000,0.000000}%
\pgfsetfillcolor{currentfill}%
\pgfsetfillopacity{0.400000}%
\pgfsetlinewidth{0.501875pt}%
\definecolor{currentstroke}{rgb}{0.501961,0.000000,0.000000}%
\pgfsetstrokecolor{currentstroke}%
\pgfsetstrokeopacity{0.400000}%
\pgfsetdash{}{0pt}%
\pgfpathmoveto{\pgfqpoint{3.082364in}{1.557010in}}%
\pgfpathcurveto{\pgfqpoint{3.093415in}{1.557010in}}{\pgfqpoint{3.104014in}{1.561400in}}{\pgfqpoint{3.111827in}{1.569213in}}%
\pgfpathcurveto{\pgfqpoint{3.119641in}{1.577027in}}{\pgfqpoint{3.124031in}{1.587626in}}{\pgfqpoint{3.124031in}{1.598676in}}%
\pgfpathcurveto{\pgfqpoint{3.124031in}{1.609726in}}{\pgfqpoint{3.119641in}{1.620325in}}{\pgfqpoint{3.111827in}{1.628139in}}%
\pgfpathcurveto{\pgfqpoint{3.104014in}{1.635953in}}{\pgfqpoint{3.093415in}{1.640343in}}{\pgfqpoint{3.082364in}{1.640343in}}%
\pgfpathcurveto{\pgfqpoint{3.071314in}{1.640343in}}{\pgfqpoint{3.060715in}{1.635953in}}{\pgfqpoint{3.052902in}{1.628139in}}%
\pgfpathcurveto{\pgfqpoint{3.045088in}{1.620325in}}{\pgfqpoint{3.040698in}{1.609726in}}{\pgfqpoint{3.040698in}{1.598676in}}%
\pgfpathcurveto{\pgfqpoint{3.040698in}{1.587626in}}{\pgfqpoint{3.045088in}{1.577027in}}{\pgfqpoint{3.052902in}{1.569213in}}%
\pgfpathcurveto{\pgfqpoint{3.060715in}{1.561400in}}{\pgfqpoint{3.071314in}{1.557010in}}{\pgfqpoint{3.082364in}{1.557010in}}%
\pgfpathclose%
\pgfusepath{stroke,fill}%
\end{pgfscope}%
\begin{pgfscope}%
\pgfpathrectangle{\pgfqpoint{0.511823in}{0.504323in}}{\pgfqpoint{3.218177in}{3.225677in}} %
\pgfusepath{clip}%
\pgfsetbuttcap%
\pgfsetroundjoin%
\definecolor{currentfill}{rgb}{0.501961,0.000000,0.000000}%
\pgfsetfillcolor{currentfill}%
\pgfsetfillopacity{0.400000}%
\pgfsetlinewidth{0.501875pt}%
\definecolor{currentstroke}{rgb}{0.501961,0.000000,0.000000}%
\pgfsetstrokecolor{currentstroke}%
\pgfsetstrokeopacity{0.400000}%
\pgfsetdash{}{0pt}%
\pgfpathmoveto{\pgfqpoint{3.104428in}{1.573748in}}%
\pgfpathcurveto{\pgfqpoint{3.115478in}{1.573748in}}{\pgfqpoint{3.126077in}{1.578139in}}{\pgfqpoint{3.133891in}{1.585952in}}%
\pgfpathcurveto{\pgfqpoint{3.141704in}{1.593766in}}{\pgfqpoint{3.146095in}{1.604365in}}{\pgfqpoint{3.146095in}{1.615415in}}%
\pgfpathcurveto{\pgfqpoint{3.146095in}{1.626465in}}{\pgfqpoint{3.141704in}{1.637064in}}{\pgfqpoint{3.133891in}{1.644878in}}%
\pgfpathcurveto{\pgfqpoint{3.126077in}{1.652691in}}{\pgfqpoint{3.115478in}{1.657082in}}{\pgfqpoint{3.104428in}{1.657082in}}%
\pgfpathcurveto{\pgfqpoint{3.093378in}{1.657082in}}{\pgfqpoint{3.082779in}{1.652691in}}{\pgfqpoint{3.074965in}{1.644878in}}%
\pgfpathcurveto{\pgfqpoint{3.067152in}{1.637064in}}{\pgfqpoint{3.062761in}{1.626465in}}{\pgfqpoint{3.062761in}{1.615415in}}%
\pgfpathcurveto{\pgfqpoint{3.062761in}{1.604365in}}{\pgfqpoint{3.067152in}{1.593766in}}{\pgfqpoint{3.074965in}{1.585952in}}%
\pgfpathcurveto{\pgfqpoint{3.082779in}{1.578139in}}{\pgfqpoint{3.093378in}{1.573748in}}{\pgfqpoint{3.104428in}{1.573748in}}%
\pgfpathclose%
\pgfusepath{stroke,fill}%
\end{pgfscope}%
\begin{pgfscope}%
\pgfpathrectangle{\pgfqpoint{0.511823in}{0.504323in}}{\pgfqpoint{3.218177in}{3.225677in}} %
\pgfusepath{clip}%
\pgfsetbuttcap%
\pgfsetroundjoin%
\definecolor{currentfill}{rgb}{0.501961,0.000000,0.000000}%
\pgfsetfillcolor{currentfill}%
\pgfsetfillopacity{0.400000}%
\pgfsetlinewidth{0.501875pt}%
\definecolor{currentstroke}{rgb}{0.501961,0.000000,0.000000}%
\pgfsetstrokecolor{currentstroke}%
\pgfsetstrokeopacity{0.400000}%
\pgfsetdash{}{0pt}%
\pgfpathmoveto{\pgfqpoint{3.073553in}{1.570030in}}%
\pgfpathcurveto{\pgfqpoint{3.084604in}{1.570030in}}{\pgfqpoint{3.095203in}{1.574420in}}{\pgfqpoint{3.103016in}{1.582234in}}%
\pgfpathcurveto{\pgfqpoint{3.110830in}{1.590047in}}{\pgfqpoint{3.115220in}{1.600646in}}{\pgfqpoint{3.115220in}{1.611697in}}%
\pgfpathcurveto{\pgfqpoint{3.115220in}{1.622747in}}{\pgfqpoint{3.110830in}{1.633346in}}{\pgfqpoint{3.103016in}{1.641159in}}%
\pgfpathcurveto{\pgfqpoint{3.095203in}{1.648973in}}{\pgfqpoint{3.084604in}{1.653363in}}{\pgfqpoint{3.073553in}{1.653363in}}%
\pgfpathcurveto{\pgfqpoint{3.062503in}{1.653363in}}{\pgfqpoint{3.051904in}{1.648973in}}{\pgfqpoint{3.044091in}{1.641159in}}%
\pgfpathcurveto{\pgfqpoint{3.036277in}{1.633346in}}{\pgfqpoint{3.031887in}{1.622747in}}{\pgfqpoint{3.031887in}{1.611697in}}%
\pgfpathcurveto{\pgfqpoint{3.031887in}{1.600646in}}{\pgfqpoint{3.036277in}{1.590047in}}{\pgfqpoint{3.044091in}{1.582234in}}%
\pgfpathcurveto{\pgfqpoint{3.051904in}{1.574420in}}{\pgfqpoint{3.062503in}{1.570030in}}{\pgfqpoint{3.073553in}{1.570030in}}%
\pgfpathclose%
\pgfusepath{stroke,fill}%
\end{pgfscope}%
\begin{pgfscope}%
\pgfpathrectangle{\pgfqpoint{0.511823in}{0.504323in}}{\pgfqpoint{3.218177in}{3.225677in}} %
\pgfusepath{clip}%
\pgfsetbuttcap%
\pgfsetroundjoin%
\definecolor{currentfill}{rgb}{0.501961,0.000000,0.000000}%
\pgfsetfillcolor{currentfill}%
\pgfsetfillopacity{0.400000}%
\pgfsetlinewidth{0.501875pt}%
\definecolor{currentstroke}{rgb}{0.501961,0.000000,0.000000}%
\pgfsetstrokecolor{currentstroke}%
\pgfsetstrokeopacity{0.400000}%
\pgfsetdash{}{0pt}%
\pgfpathmoveto{\pgfqpoint{3.253174in}{1.648877in}}%
\pgfpathcurveto{\pgfqpoint{3.264224in}{1.648877in}}{\pgfqpoint{3.274823in}{1.653268in}}{\pgfqpoint{3.282637in}{1.661081in}}%
\pgfpathcurveto{\pgfqpoint{3.290451in}{1.668895in}}{\pgfqpoint{3.294841in}{1.679494in}}{\pgfqpoint{3.294841in}{1.690544in}}%
\pgfpathcurveto{\pgfqpoint{3.294841in}{1.701594in}}{\pgfqpoint{3.290451in}{1.712193in}}{\pgfqpoint{3.282637in}{1.720007in}}%
\pgfpathcurveto{\pgfqpoint{3.274823in}{1.727820in}}{\pgfqpoint{3.264224in}{1.732211in}}{\pgfqpoint{3.253174in}{1.732211in}}%
\pgfpathcurveto{\pgfqpoint{3.242124in}{1.732211in}}{\pgfqpoint{3.231525in}{1.727820in}}{\pgfqpoint{3.223711in}{1.720007in}}%
\pgfpathcurveto{\pgfqpoint{3.215898in}{1.712193in}}{\pgfqpoint{3.211508in}{1.701594in}}{\pgfqpoint{3.211508in}{1.690544in}}%
\pgfpathcurveto{\pgfqpoint{3.211508in}{1.679494in}}{\pgfqpoint{3.215898in}{1.668895in}}{\pgfqpoint{3.223711in}{1.661081in}}%
\pgfpathcurveto{\pgfqpoint{3.231525in}{1.653268in}}{\pgfqpoint{3.242124in}{1.648877in}}{\pgfqpoint{3.253174in}{1.648877in}}%
\pgfpathclose%
\pgfusepath{stroke,fill}%
\end{pgfscope}%
\begin{pgfscope}%
\pgfpathrectangle{\pgfqpoint{0.511823in}{0.504323in}}{\pgfqpoint{3.218177in}{3.225677in}} %
\pgfusepath{clip}%
\pgfsetbuttcap%
\pgfsetroundjoin%
\definecolor{currentfill}{rgb}{0.501961,0.000000,0.000000}%
\pgfsetfillcolor{currentfill}%
\pgfsetfillopacity{0.400000}%
\pgfsetlinewidth{0.501875pt}%
\definecolor{currentstroke}{rgb}{0.501961,0.000000,0.000000}%
\pgfsetstrokecolor{currentstroke}%
\pgfsetstrokeopacity{0.400000}%
\pgfsetdash{}{0pt}%
\pgfpathmoveto{\pgfqpoint{3.274118in}{1.666056in}}%
\pgfpathcurveto{\pgfqpoint{3.285168in}{1.666056in}}{\pgfqpoint{3.295767in}{1.670446in}}{\pgfqpoint{3.303581in}{1.678260in}}%
\pgfpathcurveto{\pgfqpoint{3.311395in}{1.686073in}}{\pgfqpoint{3.315785in}{1.696672in}}{\pgfqpoint{3.315785in}{1.707722in}}%
\pgfpathcurveto{\pgfqpoint{3.315785in}{1.718773in}}{\pgfqpoint{3.311395in}{1.729372in}}{\pgfqpoint{3.303581in}{1.737185in}}%
\pgfpathcurveto{\pgfqpoint{3.295767in}{1.744999in}}{\pgfqpoint{3.285168in}{1.749389in}}{\pgfqpoint{3.274118in}{1.749389in}}%
\pgfpathcurveto{\pgfqpoint{3.263068in}{1.749389in}}{\pgfqpoint{3.252469in}{1.744999in}}{\pgfqpoint{3.244655in}{1.737185in}}%
\pgfpathcurveto{\pgfqpoint{3.236842in}{1.729372in}}{\pgfqpoint{3.232451in}{1.718773in}}{\pgfqpoint{3.232451in}{1.707722in}}%
\pgfpathcurveto{\pgfqpoint{3.232451in}{1.696672in}}{\pgfqpoint{3.236842in}{1.686073in}}{\pgfqpoint{3.244655in}{1.678260in}}%
\pgfpathcurveto{\pgfqpoint{3.252469in}{1.670446in}}{\pgfqpoint{3.263068in}{1.666056in}}{\pgfqpoint{3.274118in}{1.666056in}}%
\pgfpathclose%
\pgfusepath{stroke,fill}%
\end{pgfscope}%
\begin{pgfscope}%
\pgfpathrectangle{\pgfqpoint{0.511823in}{0.504323in}}{\pgfqpoint{3.218177in}{3.225677in}} %
\pgfusepath{clip}%
\pgfsetbuttcap%
\pgfsetroundjoin%
\definecolor{currentfill}{rgb}{0.501961,0.000000,0.000000}%
\pgfsetfillcolor{currentfill}%
\pgfsetfillopacity{0.400000}%
\pgfsetlinewidth{0.501875pt}%
\definecolor{currentstroke}{rgb}{0.501961,0.000000,0.000000}%
\pgfsetstrokecolor{currentstroke}%
\pgfsetstrokeopacity{0.400000}%
\pgfsetdash{}{0pt}%
\pgfpathmoveto{\pgfqpoint{3.016876in}{1.572108in}}%
\pgfpathcurveto{\pgfqpoint{3.027926in}{1.572108in}}{\pgfqpoint{3.038525in}{1.576498in}}{\pgfqpoint{3.046338in}{1.584312in}}%
\pgfpathcurveto{\pgfqpoint{3.054152in}{1.592126in}}{\pgfqpoint{3.058542in}{1.602725in}}{\pgfqpoint{3.058542in}{1.613775in}}%
\pgfpathcurveto{\pgfqpoint{3.058542in}{1.624825in}}{\pgfqpoint{3.054152in}{1.635424in}}{\pgfqpoint{3.046338in}{1.643238in}}%
\pgfpathcurveto{\pgfqpoint{3.038525in}{1.651051in}}{\pgfqpoint{3.027926in}{1.655441in}}{\pgfqpoint{3.016876in}{1.655441in}}%
\pgfpathcurveto{\pgfqpoint{3.005825in}{1.655441in}}{\pgfqpoint{2.995226in}{1.651051in}}{\pgfqpoint{2.987413in}{1.643238in}}%
\pgfpathcurveto{\pgfqpoint{2.979599in}{1.635424in}}{\pgfqpoint{2.975209in}{1.624825in}}{\pgfqpoint{2.975209in}{1.613775in}}%
\pgfpathcurveto{\pgfqpoint{2.975209in}{1.602725in}}{\pgfqpoint{2.979599in}{1.592126in}}{\pgfqpoint{2.987413in}{1.584312in}}%
\pgfpathcurveto{\pgfqpoint{2.995226in}{1.576498in}}{\pgfqpoint{3.005825in}{1.572108in}}{\pgfqpoint{3.016876in}{1.572108in}}%
\pgfpathclose%
\pgfusepath{stroke,fill}%
\end{pgfscope}%
\begin{pgfscope}%
\pgfpathrectangle{\pgfqpoint{0.511823in}{0.504323in}}{\pgfqpoint{3.218177in}{3.225677in}} %
\pgfusepath{clip}%
\pgfsetbuttcap%
\pgfsetroundjoin%
\definecolor{currentfill}{rgb}{0.501961,0.000000,0.000000}%
\pgfsetfillcolor{currentfill}%
\pgfsetfillopacity{0.400000}%
\pgfsetlinewidth{0.501875pt}%
\definecolor{currentstroke}{rgb}{0.501961,0.000000,0.000000}%
\pgfsetstrokecolor{currentstroke}%
\pgfsetstrokeopacity{0.400000}%
\pgfsetdash{}{0pt}%
\pgfpathmoveto{\pgfqpoint{3.078436in}{1.605049in}}%
\pgfpathcurveto{\pgfqpoint{3.089486in}{1.605049in}}{\pgfqpoint{3.100085in}{1.609440in}}{\pgfqpoint{3.107898in}{1.617253in}}%
\pgfpathcurveto{\pgfqpoint{3.115712in}{1.625067in}}{\pgfqpoint{3.120102in}{1.635666in}}{\pgfqpoint{3.120102in}{1.646716in}}%
\pgfpathcurveto{\pgfqpoint{3.120102in}{1.657766in}}{\pgfqpoint{3.115712in}{1.668365in}}{\pgfqpoint{3.107898in}{1.676179in}}%
\pgfpathcurveto{\pgfqpoint{3.100085in}{1.683992in}}{\pgfqpoint{3.089486in}{1.688383in}}{\pgfqpoint{3.078436in}{1.688383in}}%
\pgfpathcurveto{\pgfqpoint{3.067385in}{1.688383in}}{\pgfqpoint{3.056786in}{1.683992in}}{\pgfqpoint{3.048973in}{1.676179in}}%
\pgfpathcurveto{\pgfqpoint{3.041159in}{1.668365in}}{\pgfqpoint{3.036769in}{1.657766in}}{\pgfqpoint{3.036769in}{1.646716in}}%
\pgfpathcurveto{\pgfqpoint{3.036769in}{1.635666in}}{\pgfqpoint{3.041159in}{1.625067in}}{\pgfqpoint{3.048973in}{1.617253in}}%
\pgfpathcurveto{\pgfqpoint{3.056786in}{1.609440in}}{\pgfqpoint{3.067385in}{1.605049in}}{\pgfqpoint{3.078436in}{1.605049in}}%
\pgfpathclose%
\pgfusepath{stroke,fill}%
\end{pgfscope}%
\begin{pgfscope}%
\pgfpathrectangle{\pgfqpoint{0.511823in}{0.504323in}}{\pgfqpoint{3.218177in}{3.225677in}} %
\pgfusepath{clip}%
\pgfsetbuttcap%
\pgfsetroundjoin%
\definecolor{currentfill}{rgb}{0.501961,0.000000,0.000000}%
\pgfsetfillcolor{currentfill}%
\pgfsetfillopacity{0.400000}%
\pgfsetlinewidth{0.501875pt}%
\definecolor{currentstroke}{rgb}{0.501961,0.000000,0.000000}%
\pgfsetstrokecolor{currentstroke}%
\pgfsetstrokeopacity{0.400000}%
\pgfsetdash{}{0pt}%
\pgfpathmoveto{\pgfqpoint{3.051360in}{1.602364in}}%
\pgfpathcurveto{\pgfqpoint{3.062410in}{1.602364in}}{\pgfqpoint{3.073009in}{1.606754in}}{\pgfqpoint{3.080822in}{1.614568in}}%
\pgfpathcurveto{\pgfqpoint{3.088636in}{1.622381in}}{\pgfqpoint{3.093026in}{1.632980in}}{\pgfqpoint{3.093026in}{1.644031in}}%
\pgfpathcurveto{\pgfqpoint{3.093026in}{1.655081in}}{\pgfqpoint{3.088636in}{1.665680in}}{\pgfqpoint{3.080822in}{1.673493in}}%
\pgfpathcurveto{\pgfqpoint{3.073009in}{1.681307in}}{\pgfqpoint{3.062410in}{1.685697in}}{\pgfqpoint{3.051360in}{1.685697in}}%
\pgfpathcurveto{\pgfqpoint{3.040309in}{1.685697in}}{\pgfqpoint{3.029710in}{1.681307in}}{\pgfqpoint{3.021897in}{1.673493in}}%
\pgfpathcurveto{\pgfqpoint{3.014083in}{1.665680in}}{\pgfqpoint{3.009693in}{1.655081in}}{\pgfqpoint{3.009693in}{1.644031in}}%
\pgfpathcurveto{\pgfqpoint{3.009693in}{1.632980in}}{\pgfqpoint{3.014083in}{1.622381in}}{\pgfqpoint{3.021897in}{1.614568in}}%
\pgfpathcurveto{\pgfqpoint{3.029710in}{1.606754in}}{\pgfqpoint{3.040309in}{1.602364in}}{\pgfqpoint{3.051360in}{1.602364in}}%
\pgfpathclose%
\pgfusepath{stroke,fill}%
\end{pgfscope}%
\begin{pgfscope}%
\pgfpathrectangle{\pgfqpoint{0.511823in}{0.504323in}}{\pgfqpoint{3.218177in}{3.225677in}} %
\pgfusepath{clip}%
\pgfsetbuttcap%
\pgfsetroundjoin%
\definecolor{currentfill}{rgb}{0.501961,0.000000,0.000000}%
\pgfsetfillcolor{currentfill}%
\pgfsetfillopacity{0.400000}%
\pgfsetlinewidth{0.501875pt}%
\definecolor{currentstroke}{rgb}{0.501961,0.000000,0.000000}%
\pgfsetstrokecolor{currentstroke}%
\pgfsetstrokeopacity{0.400000}%
\pgfsetdash{}{0pt}%
\pgfpathmoveto{\pgfqpoint{3.139169in}{1.646676in}}%
\pgfpathcurveto{\pgfqpoint{3.150219in}{1.646676in}}{\pgfqpoint{3.160818in}{1.651066in}}{\pgfqpoint{3.168631in}{1.658880in}}%
\pgfpathcurveto{\pgfqpoint{3.176445in}{1.666693in}}{\pgfqpoint{3.180835in}{1.677292in}}{\pgfqpoint{3.180835in}{1.688342in}}%
\pgfpathcurveto{\pgfqpoint{3.180835in}{1.699392in}}{\pgfqpoint{3.176445in}{1.709992in}}{\pgfqpoint{3.168631in}{1.717805in}}%
\pgfpathcurveto{\pgfqpoint{3.160818in}{1.725619in}}{\pgfqpoint{3.150219in}{1.730009in}}{\pgfqpoint{3.139169in}{1.730009in}}%
\pgfpathcurveto{\pgfqpoint{3.128118in}{1.730009in}}{\pgfqpoint{3.117519in}{1.725619in}}{\pgfqpoint{3.109706in}{1.717805in}}%
\pgfpathcurveto{\pgfqpoint{3.101892in}{1.709992in}}{\pgfqpoint{3.097502in}{1.699392in}}{\pgfqpoint{3.097502in}{1.688342in}}%
\pgfpathcurveto{\pgfqpoint{3.097502in}{1.677292in}}{\pgfqpoint{3.101892in}{1.666693in}}{\pgfqpoint{3.109706in}{1.658880in}}%
\pgfpathcurveto{\pgfqpoint{3.117519in}{1.651066in}}{\pgfqpoint{3.128118in}{1.646676in}}{\pgfqpoint{3.139169in}{1.646676in}}%
\pgfpathclose%
\pgfusepath{stroke,fill}%
\end{pgfscope}%
\begin{pgfscope}%
\pgfpathrectangle{\pgfqpoint{0.511823in}{0.504323in}}{\pgfqpoint{3.218177in}{3.225677in}} %
\pgfusepath{clip}%
\pgfsetbuttcap%
\pgfsetroundjoin%
\definecolor{currentfill}{rgb}{0.501961,0.000000,0.000000}%
\pgfsetfillcolor{currentfill}%
\pgfsetfillopacity{0.400000}%
\pgfsetlinewidth{0.501875pt}%
\definecolor{currentstroke}{rgb}{0.501961,0.000000,0.000000}%
\pgfsetstrokecolor{currentstroke}%
\pgfsetstrokeopacity{0.400000}%
\pgfsetdash{}{0pt}%
\pgfpathmoveto{\pgfqpoint{3.202962in}{1.681683in}}%
\pgfpathcurveto{\pgfqpoint{3.214013in}{1.681683in}}{\pgfqpoint{3.224612in}{1.686073in}}{\pgfqpoint{3.232425in}{1.693887in}}%
\pgfpathcurveto{\pgfqpoint{3.240239in}{1.701700in}}{\pgfqpoint{3.244629in}{1.712299in}}{\pgfqpoint{3.244629in}{1.723350in}}%
\pgfpathcurveto{\pgfqpoint{3.244629in}{1.734400in}}{\pgfqpoint{3.240239in}{1.744999in}}{\pgfqpoint{3.232425in}{1.752812in}}%
\pgfpathcurveto{\pgfqpoint{3.224612in}{1.760626in}}{\pgfqpoint{3.214013in}{1.765016in}}{\pgfqpoint{3.202962in}{1.765016in}}%
\pgfpathcurveto{\pgfqpoint{3.191912in}{1.765016in}}{\pgfqpoint{3.181313in}{1.760626in}}{\pgfqpoint{3.173500in}{1.752812in}}%
\pgfpathcurveto{\pgfqpoint{3.165686in}{1.744999in}}{\pgfqpoint{3.161296in}{1.734400in}}{\pgfqpoint{3.161296in}{1.723350in}}%
\pgfpathcurveto{\pgfqpoint{3.161296in}{1.712299in}}{\pgfqpoint{3.165686in}{1.701700in}}{\pgfqpoint{3.173500in}{1.693887in}}%
\pgfpathcurveto{\pgfqpoint{3.181313in}{1.686073in}}{\pgfqpoint{3.191912in}{1.681683in}}{\pgfqpoint{3.202962in}{1.681683in}}%
\pgfpathclose%
\pgfusepath{stroke,fill}%
\end{pgfscope}%
\begin{pgfscope}%
\pgfpathrectangle{\pgfqpoint{0.511823in}{0.504323in}}{\pgfqpoint{3.218177in}{3.225677in}} %
\pgfusepath{clip}%
\pgfsetbuttcap%
\pgfsetroundjoin%
\definecolor{currentfill}{rgb}{0.501961,0.000000,0.000000}%
\pgfsetfillcolor{currentfill}%
\pgfsetfillopacity{0.400000}%
\pgfsetlinewidth{0.501875pt}%
\definecolor{currentstroke}{rgb}{0.501961,0.000000,0.000000}%
\pgfsetstrokecolor{currentstroke}%
\pgfsetstrokeopacity{0.400000}%
\pgfsetdash{}{0pt}%
\pgfpathmoveto{\pgfqpoint{3.235960in}{1.704303in}}%
\pgfpathcurveto{\pgfqpoint{3.247010in}{1.704303in}}{\pgfqpoint{3.257609in}{1.708694in}}{\pgfqpoint{3.265422in}{1.716507in}}%
\pgfpathcurveto{\pgfqpoint{3.273236in}{1.724321in}}{\pgfqpoint{3.277626in}{1.734920in}}{\pgfqpoint{3.277626in}{1.745970in}}%
\pgfpathcurveto{\pgfqpoint{3.277626in}{1.757020in}}{\pgfqpoint{3.273236in}{1.767619in}}{\pgfqpoint{3.265422in}{1.775433in}}%
\pgfpathcurveto{\pgfqpoint{3.257609in}{1.783246in}}{\pgfqpoint{3.247010in}{1.787637in}}{\pgfqpoint{3.235960in}{1.787637in}}%
\pgfpathcurveto{\pgfqpoint{3.224909in}{1.787637in}}{\pgfqpoint{3.214310in}{1.783246in}}{\pgfqpoint{3.206497in}{1.775433in}}%
\pgfpathcurveto{\pgfqpoint{3.198683in}{1.767619in}}{\pgfqpoint{3.194293in}{1.757020in}}{\pgfqpoint{3.194293in}{1.745970in}}%
\pgfpathcurveto{\pgfqpoint{3.194293in}{1.734920in}}{\pgfqpoint{3.198683in}{1.724321in}}{\pgfqpoint{3.206497in}{1.716507in}}%
\pgfpathcurveto{\pgfqpoint{3.214310in}{1.708694in}}{\pgfqpoint{3.224909in}{1.704303in}}{\pgfqpoint{3.235960in}{1.704303in}}%
\pgfpathclose%
\pgfusepath{stroke,fill}%
\end{pgfscope}%
\begin{pgfscope}%
\pgfpathrectangle{\pgfqpoint{0.511823in}{0.504323in}}{\pgfqpoint{3.218177in}{3.225677in}} %
\pgfusepath{clip}%
\pgfsetbuttcap%
\pgfsetroundjoin%
\definecolor{currentfill}{rgb}{0.501961,0.000000,0.000000}%
\pgfsetfillcolor{currentfill}%
\pgfsetfillopacity{0.400000}%
\pgfsetlinewidth{0.501875pt}%
\definecolor{currentstroke}{rgb}{0.501961,0.000000,0.000000}%
\pgfsetstrokecolor{currentstroke}%
\pgfsetstrokeopacity{0.400000}%
\pgfsetdash{}{0pt}%
\pgfpathmoveto{\pgfqpoint{3.069634in}{1.643253in}}%
\pgfpathcurveto{\pgfqpoint{3.080684in}{1.643253in}}{\pgfqpoint{3.091283in}{1.647643in}}{\pgfqpoint{3.099096in}{1.655456in}}%
\pgfpathcurveto{\pgfqpoint{3.106910in}{1.663270in}}{\pgfqpoint{3.111300in}{1.673869in}}{\pgfqpoint{3.111300in}{1.684919in}}%
\pgfpathcurveto{\pgfqpoint{3.111300in}{1.695969in}}{\pgfqpoint{3.106910in}{1.706568in}}{\pgfqpoint{3.099096in}{1.714382in}}%
\pgfpathcurveto{\pgfqpoint{3.091283in}{1.722196in}}{\pgfqpoint{3.080684in}{1.726586in}}{\pgfqpoint{3.069634in}{1.726586in}}%
\pgfpathcurveto{\pgfqpoint{3.058584in}{1.726586in}}{\pgfqpoint{3.047985in}{1.722196in}}{\pgfqpoint{3.040171in}{1.714382in}}%
\pgfpathcurveto{\pgfqpoint{3.032357in}{1.706568in}}{\pgfqpoint{3.027967in}{1.695969in}}{\pgfqpoint{3.027967in}{1.684919in}}%
\pgfpathcurveto{\pgfqpoint{3.027967in}{1.673869in}}{\pgfqpoint{3.032357in}{1.663270in}}{\pgfqpoint{3.040171in}{1.655456in}}%
\pgfpathcurveto{\pgfqpoint{3.047985in}{1.647643in}}{\pgfqpoint{3.058584in}{1.643253in}}{\pgfqpoint{3.069634in}{1.643253in}}%
\pgfpathclose%
\pgfusepath{stroke,fill}%
\end{pgfscope}%
\begin{pgfscope}%
\pgfpathrectangle{\pgfqpoint{0.511823in}{0.504323in}}{\pgfqpoint{3.218177in}{3.225677in}} %
\pgfusepath{clip}%
\pgfsetbuttcap%
\pgfsetroundjoin%
\definecolor{currentfill}{rgb}{0.501961,0.000000,0.000000}%
\pgfsetfillcolor{currentfill}%
\pgfsetfillopacity{0.400000}%
\pgfsetlinewidth{0.501875pt}%
\definecolor{currentstroke}{rgb}{0.501961,0.000000,0.000000}%
\pgfsetstrokecolor{currentstroke}%
\pgfsetstrokeopacity{0.400000}%
\pgfsetdash{}{0pt}%
\pgfpathmoveto{\pgfqpoint{2.980782in}{1.613947in}}%
\pgfpathcurveto{\pgfqpoint{2.991832in}{1.613947in}}{\pgfqpoint{3.002431in}{1.618338in}}{\pgfqpoint{3.010244in}{1.626151in}}%
\pgfpathcurveto{\pgfqpoint{3.018058in}{1.633965in}}{\pgfqpoint{3.022448in}{1.644564in}}{\pgfqpoint{3.022448in}{1.655614in}}%
\pgfpathcurveto{\pgfqpoint{3.022448in}{1.666664in}}{\pgfqpoint{3.018058in}{1.677263in}}{\pgfqpoint{3.010244in}{1.685077in}}%
\pgfpathcurveto{\pgfqpoint{3.002431in}{1.692891in}}{\pgfqpoint{2.991832in}{1.697281in}}{\pgfqpoint{2.980782in}{1.697281in}}%
\pgfpathcurveto{\pgfqpoint{2.969731in}{1.697281in}}{\pgfqpoint{2.959132in}{1.692891in}}{\pgfqpoint{2.951319in}{1.685077in}}%
\pgfpathcurveto{\pgfqpoint{2.943505in}{1.677263in}}{\pgfqpoint{2.939115in}{1.666664in}}{\pgfqpoint{2.939115in}{1.655614in}}%
\pgfpathcurveto{\pgfqpoint{2.939115in}{1.644564in}}{\pgfqpoint{2.943505in}{1.633965in}}{\pgfqpoint{2.951319in}{1.626151in}}%
\pgfpathcurveto{\pgfqpoint{2.959132in}{1.618338in}}{\pgfqpoint{2.969731in}{1.613947in}}{\pgfqpoint{2.980782in}{1.613947in}}%
\pgfpathclose%
\pgfusepath{stroke,fill}%
\end{pgfscope}%
\begin{pgfscope}%
\pgfpathrectangle{\pgfqpoint{0.511823in}{0.504323in}}{\pgfqpoint{3.218177in}{3.225677in}} %
\pgfusepath{clip}%
\pgfsetbuttcap%
\pgfsetroundjoin%
\definecolor{currentfill}{rgb}{0.501961,0.000000,0.000000}%
\pgfsetfillcolor{currentfill}%
\pgfsetfillopacity{0.400000}%
\pgfsetlinewidth{0.501875pt}%
\definecolor{currentstroke}{rgb}{0.501961,0.000000,0.000000}%
\pgfsetstrokecolor{currentstroke}%
\pgfsetstrokeopacity{0.400000}%
\pgfsetdash{}{0pt}%
\pgfpathmoveto{\pgfqpoint{3.161110in}{1.699286in}}%
\pgfpathcurveto{\pgfqpoint{3.172160in}{1.699286in}}{\pgfqpoint{3.182759in}{1.703676in}}{\pgfqpoint{3.190572in}{1.711490in}}%
\pgfpathcurveto{\pgfqpoint{3.198386in}{1.719303in}}{\pgfqpoint{3.202776in}{1.729902in}}{\pgfqpoint{3.202776in}{1.740952in}}%
\pgfpathcurveto{\pgfqpoint{3.202776in}{1.752003in}}{\pgfqpoint{3.198386in}{1.762602in}}{\pgfqpoint{3.190572in}{1.770415in}}%
\pgfpathcurveto{\pgfqpoint{3.182759in}{1.778229in}}{\pgfqpoint{3.172160in}{1.782619in}}{\pgfqpoint{3.161110in}{1.782619in}}%
\pgfpathcurveto{\pgfqpoint{3.150060in}{1.782619in}}{\pgfqpoint{3.139460in}{1.778229in}}{\pgfqpoint{3.131647in}{1.770415in}}%
\pgfpathcurveto{\pgfqpoint{3.123833in}{1.762602in}}{\pgfqpoint{3.119443in}{1.752003in}}{\pgfqpoint{3.119443in}{1.740952in}}%
\pgfpathcurveto{\pgfqpoint{3.119443in}{1.729902in}}{\pgfqpoint{3.123833in}{1.719303in}}{\pgfqpoint{3.131647in}{1.711490in}}%
\pgfpathcurveto{\pgfqpoint{3.139460in}{1.703676in}}{\pgfqpoint{3.150060in}{1.699286in}}{\pgfqpoint{3.161110in}{1.699286in}}%
\pgfpathclose%
\pgfusepath{stroke,fill}%
\end{pgfscope}%
\begin{pgfscope}%
\pgfpathrectangle{\pgfqpoint{0.511823in}{0.504323in}}{\pgfqpoint{3.218177in}{3.225677in}} %
\pgfusepath{clip}%
\pgfsetbuttcap%
\pgfsetroundjoin%
\definecolor{currentfill}{rgb}{0.501961,0.000000,0.000000}%
\pgfsetfillcolor{currentfill}%
\pgfsetfillopacity{0.400000}%
\pgfsetlinewidth{0.501875pt}%
\definecolor{currentstroke}{rgb}{0.501961,0.000000,0.000000}%
\pgfsetstrokecolor{currentstroke}%
\pgfsetstrokeopacity{0.400000}%
\pgfsetdash{}{0pt}%
\pgfpathmoveto{\pgfqpoint{3.164186in}{1.709410in}}%
\pgfpathcurveto{\pgfqpoint{3.175236in}{1.709410in}}{\pgfqpoint{3.185835in}{1.713800in}}{\pgfqpoint{3.193649in}{1.721614in}}%
\pgfpathcurveto{\pgfqpoint{3.201463in}{1.729427in}}{\pgfqpoint{3.205853in}{1.740026in}}{\pgfqpoint{3.205853in}{1.751076in}}%
\pgfpathcurveto{\pgfqpoint{3.205853in}{1.762126in}}{\pgfqpoint{3.201463in}{1.772726in}}{\pgfqpoint{3.193649in}{1.780539in}}%
\pgfpathcurveto{\pgfqpoint{3.185835in}{1.788353in}}{\pgfqpoint{3.175236in}{1.792743in}}{\pgfqpoint{3.164186in}{1.792743in}}%
\pgfpathcurveto{\pgfqpoint{3.153136in}{1.792743in}}{\pgfqpoint{3.142537in}{1.788353in}}{\pgfqpoint{3.134724in}{1.780539in}}%
\pgfpathcurveto{\pgfqpoint{3.126910in}{1.772726in}}{\pgfqpoint{3.122520in}{1.762126in}}{\pgfqpoint{3.122520in}{1.751076in}}%
\pgfpathcurveto{\pgfqpoint{3.122520in}{1.740026in}}{\pgfqpoint{3.126910in}{1.729427in}}{\pgfqpoint{3.134724in}{1.721614in}}%
\pgfpathcurveto{\pgfqpoint{3.142537in}{1.713800in}}{\pgfqpoint{3.153136in}{1.709410in}}{\pgfqpoint{3.164186in}{1.709410in}}%
\pgfpathclose%
\pgfusepath{stroke,fill}%
\end{pgfscope}%
\begin{pgfscope}%
\pgfpathrectangle{\pgfqpoint{0.511823in}{0.504323in}}{\pgfqpoint{3.218177in}{3.225677in}} %
\pgfusepath{clip}%
\pgfsetbuttcap%
\pgfsetroundjoin%
\definecolor{currentfill}{rgb}{0.501961,0.000000,0.000000}%
\pgfsetfillcolor{currentfill}%
\pgfsetfillopacity{0.400000}%
\pgfsetlinewidth{0.501875pt}%
\definecolor{currentstroke}{rgb}{0.501961,0.000000,0.000000}%
\pgfsetstrokecolor{currentstroke}%
\pgfsetstrokeopacity{0.400000}%
\pgfsetdash{}{0pt}%
\pgfpathmoveto{\pgfqpoint{3.047086in}{1.667272in}}%
\pgfpathcurveto{\pgfqpoint{3.058136in}{1.667272in}}{\pgfqpoint{3.068735in}{1.671662in}}{\pgfqpoint{3.076548in}{1.679476in}}%
\pgfpathcurveto{\pgfqpoint{3.084362in}{1.687290in}}{\pgfqpoint{3.088752in}{1.697889in}}{\pgfqpoint{3.088752in}{1.708939in}}%
\pgfpathcurveto{\pgfqpoint{3.088752in}{1.719989in}}{\pgfqpoint{3.084362in}{1.730588in}}{\pgfqpoint{3.076548in}{1.738402in}}%
\pgfpathcurveto{\pgfqpoint{3.068735in}{1.746215in}}{\pgfqpoint{3.058136in}{1.750605in}}{\pgfqpoint{3.047086in}{1.750605in}}%
\pgfpathcurveto{\pgfqpoint{3.036036in}{1.750605in}}{\pgfqpoint{3.025437in}{1.746215in}}{\pgfqpoint{3.017623in}{1.738402in}}%
\pgfpathcurveto{\pgfqpoint{3.009809in}{1.730588in}}{\pgfqpoint{3.005419in}{1.719989in}}{\pgfqpoint{3.005419in}{1.708939in}}%
\pgfpathcurveto{\pgfqpoint{3.005419in}{1.697889in}}{\pgfqpoint{3.009809in}{1.687290in}}{\pgfqpoint{3.017623in}{1.679476in}}%
\pgfpathcurveto{\pgfqpoint{3.025437in}{1.671662in}}{\pgfqpoint{3.036036in}{1.667272in}}{\pgfqpoint{3.047086in}{1.667272in}}%
\pgfpathclose%
\pgfusepath{stroke,fill}%
\end{pgfscope}%
\begin{pgfscope}%
\pgfpathrectangle{\pgfqpoint{0.511823in}{0.504323in}}{\pgfqpoint{3.218177in}{3.225677in}} %
\pgfusepath{clip}%
\pgfsetbuttcap%
\pgfsetroundjoin%
\definecolor{currentfill}{rgb}{0.501961,0.000000,0.000000}%
\pgfsetfillcolor{currentfill}%
\pgfsetfillopacity{0.400000}%
\pgfsetlinewidth{0.501875pt}%
\definecolor{currentstroke}{rgb}{0.501961,0.000000,0.000000}%
\pgfsetstrokecolor{currentstroke}%
\pgfsetstrokeopacity{0.400000}%
\pgfsetdash{}{0pt}%
\pgfpathmoveto{\pgfqpoint{3.105904in}{1.701524in}}%
\pgfpathcurveto{\pgfqpoint{3.116954in}{1.701524in}}{\pgfqpoint{3.127553in}{1.705914in}}{\pgfqpoint{3.135367in}{1.713728in}}%
\pgfpathcurveto{\pgfqpoint{3.143180in}{1.721541in}}{\pgfqpoint{3.147571in}{1.732140in}}{\pgfqpoint{3.147571in}{1.743190in}}%
\pgfpathcurveto{\pgfqpoint{3.147571in}{1.754241in}}{\pgfqpoint{3.143180in}{1.764840in}}{\pgfqpoint{3.135367in}{1.772653in}}%
\pgfpathcurveto{\pgfqpoint{3.127553in}{1.780467in}}{\pgfqpoint{3.116954in}{1.784857in}}{\pgfqpoint{3.105904in}{1.784857in}}%
\pgfpathcurveto{\pgfqpoint{3.094854in}{1.784857in}}{\pgfqpoint{3.084255in}{1.780467in}}{\pgfqpoint{3.076441in}{1.772653in}}%
\pgfpathcurveto{\pgfqpoint{3.068627in}{1.764840in}}{\pgfqpoint{3.064237in}{1.754241in}}{\pgfqpoint{3.064237in}{1.743190in}}%
\pgfpathcurveto{\pgfqpoint{3.064237in}{1.732140in}}{\pgfqpoint{3.068627in}{1.721541in}}{\pgfqpoint{3.076441in}{1.713728in}}%
\pgfpathcurveto{\pgfqpoint{3.084255in}{1.705914in}}{\pgfqpoint{3.094854in}{1.701524in}}{\pgfqpoint{3.105904in}{1.701524in}}%
\pgfpathclose%
\pgfusepath{stroke,fill}%
\end{pgfscope}%
\begin{pgfscope}%
\pgfpathrectangle{\pgfqpoint{0.511823in}{0.504323in}}{\pgfqpoint{3.218177in}{3.225677in}} %
\pgfusepath{clip}%
\pgfsetbuttcap%
\pgfsetroundjoin%
\definecolor{currentfill}{rgb}{0.501961,0.000000,0.000000}%
\pgfsetfillcolor{currentfill}%
\pgfsetfillopacity{0.400000}%
\pgfsetlinewidth{0.501875pt}%
\definecolor{currentstroke}{rgb}{0.501961,0.000000,0.000000}%
\pgfsetstrokecolor{currentstroke}%
\pgfsetstrokeopacity{0.400000}%
\pgfsetdash{}{0pt}%
\pgfpathmoveto{\pgfqpoint{3.125107in}{1.718695in}}%
\pgfpathcurveto{\pgfqpoint{3.136157in}{1.718695in}}{\pgfqpoint{3.146756in}{1.723086in}}{\pgfqpoint{3.154569in}{1.730899in}}%
\pgfpathcurveto{\pgfqpoint{3.162383in}{1.738713in}}{\pgfqpoint{3.166773in}{1.749312in}}{\pgfqpoint{3.166773in}{1.760362in}}%
\pgfpathcurveto{\pgfqpoint{3.166773in}{1.771412in}}{\pgfqpoint{3.162383in}{1.782011in}}{\pgfqpoint{3.154569in}{1.789825in}}%
\pgfpathcurveto{\pgfqpoint{3.146756in}{1.797638in}}{\pgfqpoint{3.136157in}{1.802029in}}{\pgfqpoint{3.125107in}{1.802029in}}%
\pgfpathcurveto{\pgfqpoint{3.114056in}{1.802029in}}{\pgfqpoint{3.103457in}{1.797638in}}{\pgfqpoint{3.095644in}{1.789825in}}%
\pgfpathcurveto{\pgfqpoint{3.087830in}{1.782011in}}{\pgfqpoint{3.083440in}{1.771412in}}{\pgfqpoint{3.083440in}{1.760362in}}%
\pgfpathcurveto{\pgfqpoint{3.083440in}{1.749312in}}{\pgfqpoint{3.087830in}{1.738713in}}{\pgfqpoint{3.095644in}{1.730899in}}%
\pgfpathcurveto{\pgfqpoint{3.103457in}{1.723086in}}{\pgfqpoint{3.114056in}{1.718695in}}{\pgfqpoint{3.125107in}{1.718695in}}%
\pgfpathclose%
\pgfusepath{stroke,fill}%
\end{pgfscope}%
\begin{pgfscope}%
\pgfpathrectangle{\pgfqpoint{0.511823in}{0.504323in}}{\pgfqpoint{3.218177in}{3.225677in}} %
\pgfusepath{clip}%
\pgfsetbuttcap%
\pgfsetroundjoin%
\definecolor{currentfill}{rgb}{0.501961,0.000000,0.000000}%
\pgfsetfillcolor{currentfill}%
\pgfsetfillopacity{0.400000}%
\pgfsetlinewidth{0.501875pt}%
\definecolor{currentstroke}{rgb}{0.501961,0.000000,0.000000}%
\pgfsetstrokecolor{currentstroke}%
\pgfsetstrokeopacity{0.400000}%
\pgfsetdash{}{0pt}%
\pgfpathmoveto{\pgfqpoint{3.183966in}{1.753719in}}%
\pgfpathcurveto{\pgfqpoint{3.195016in}{1.753719in}}{\pgfqpoint{3.205615in}{1.758109in}}{\pgfqpoint{3.213429in}{1.765923in}}%
\pgfpathcurveto{\pgfqpoint{3.221243in}{1.773737in}}{\pgfqpoint{3.225633in}{1.784336in}}{\pgfqpoint{3.225633in}{1.795386in}}%
\pgfpathcurveto{\pgfqpoint{3.225633in}{1.806436in}}{\pgfqpoint{3.221243in}{1.817035in}}{\pgfqpoint{3.213429in}{1.824849in}}%
\pgfpathcurveto{\pgfqpoint{3.205615in}{1.832662in}}{\pgfqpoint{3.195016in}{1.837053in}}{\pgfqpoint{3.183966in}{1.837053in}}%
\pgfpathcurveto{\pgfqpoint{3.172916in}{1.837053in}}{\pgfqpoint{3.162317in}{1.832662in}}{\pgfqpoint{3.154503in}{1.824849in}}%
\pgfpathcurveto{\pgfqpoint{3.146690in}{1.817035in}}{\pgfqpoint{3.142300in}{1.806436in}}{\pgfqpoint{3.142300in}{1.795386in}}%
\pgfpathcurveto{\pgfqpoint{3.142300in}{1.784336in}}{\pgfqpoint{3.146690in}{1.773737in}}{\pgfqpoint{3.154503in}{1.765923in}}%
\pgfpathcurveto{\pgfqpoint{3.162317in}{1.758109in}}{\pgfqpoint{3.172916in}{1.753719in}}{\pgfqpoint{3.183966in}{1.753719in}}%
\pgfpathclose%
\pgfusepath{stroke,fill}%
\end{pgfscope}%
\begin{pgfscope}%
\pgfpathrectangle{\pgfqpoint{0.511823in}{0.504323in}}{\pgfqpoint{3.218177in}{3.225677in}} %
\pgfusepath{clip}%
\pgfsetbuttcap%
\pgfsetroundjoin%
\definecolor{currentfill}{rgb}{0.501961,0.000000,0.000000}%
\pgfsetfillcolor{currentfill}%
\pgfsetfillopacity{0.400000}%
\pgfsetlinewidth{0.501875pt}%
\definecolor{currentstroke}{rgb}{0.501961,0.000000,0.000000}%
\pgfsetstrokecolor{currentstroke}%
\pgfsetstrokeopacity{0.400000}%
\pgfsetdash{}{0pt}%
\pgfpathmoveto{\pgfqpoint{3.083571in}{1.717588in}}%
\pgfpathcurveto{\pgfqpoint{3.094621in}{1.717588in}}{\pgfqpoint{3.105220in}{1.721978in}}{\pgfqpoint{3.113034in}{1.729792in}}%
\pgfpathcurveto{\pgfqpoint{3.120847in}{1.737606in}}{\pgfqpoint{3.125238in}{1.748205in}}{\pgfqpoint{3.125238in}{1.759255in}}%
\pgfpathcurveto{\pgfqpoint{3.125238in}{1.770305in}}{\pgfqpoint{3.120847in}{1.780904in}}{\pgfqpoint{3.113034in}{1.788718in}}%
\pgfpathcurveto{\pgfqpoint{3.105220in}{1.796531in}}{\pgfqpoint{3.094621in}{1.800921in}}{\pgfqpoint{3.083571in}{1.800921in}}%
\pgfpathcurveto{\pgfqpoint{3.072521in}{1.800921in}}{\pgfqpoint{3.061922in}{1.796531in}}{\pgfqpoint{3.054108in}{1.788718in}}%
\pgfpathcurveto{\pgfqpoint{3.046295in}{1.780904in}}{\pgfqpoint{3.041904in}{1.770305in}}{\pgfqpoint{3.041904in}{1.759255in}}%
\pgfpathcurveto{\pgfqpoint{3.041904in}{1.748205in}}{\pgfqpoint{3.046295in}{1.737606in}}{\pgfqpoint{3.054108in}{1.729792in}}%
\pgfpathcurveto{\pgfqpoint{3.061922in}{1.721978in}}{\pgfqpoint{3.072521in}{1.717588in}}{\pgfqpoint{3.083571in}{1.717588in}}%
\pgfpathclose%
\pgfusepath{stroke,fill}%
\end{pgfscope}%
\begin{pgfscope}%
\pgfpathrectangle{\pgfqpoint{0.511823in}{0.504323in}}{\pgfqpoint{3.218177in}{3.225677in}} %
\pgfusepath{clip}%
\pgfsetbuttcap%
\pgfsetroundjoin%
\definecolor{currentfill}{rgb}{0.501961,0.000000,0.000000}%
\pgfsetfillcolor{currentfill}%
\pgfsetfillopacity{0.400000}%
\pgfsetlinewidth{0.501875pt}%
\definecolor{currentstroke}{rgb}{0.501961,0.000000,0.000000}%
\pgfsetstrokecolor{currentstroke}%
\pgfsetstrokeopacity{0.400000}%
\pgfsetdash{}{0pt}%
\pgfpathmoveto{\pgfqpoint{3.129508in}{1.747083in}}%
\pgfpathcurveto{\pgfqpoint{3.140558in}{1.747083in}}{\pgfqpoint{3.151157in}{1.751473in}}{\pgfqpoint{3.158971in}{1.759287in}}%
\pgfpathcurveto{\pgfqpoint{3.166784in}{1.767100in}}{\pgfqpoint{3.171174in}{1.777699in}}{\pgfqpoint{3.171174in}{1.788750in}}%
\pgfpathcurveto{\pgfqpoint{3.171174in}{1.799800in}}{\pgfqpoint{3.166784in}{1.810399in}}{\pgfqpoint{3.158971in}{1.818212in}}%
\pgfpathcurveto{\pgfqpoint{3.151157in}{1.826026in}}{\pgfqpoint{3.140558in}{1.830416in}}{\pgfqpoint{3.129508in}{1.830416in}}%
\pgfpathcurveto{\pgfqpoint{3.118458in}{1.830416in}}{\pgfqpoint{3.107859in}{1.826026in}}{\pgfqpoint{3.100045in}{1.818212in}}%
\pgfpathcurveto{\pgfqpoint{3.092231in}{1.810399in}}{\pgfqpoint{3.087841in}{1.799800in}}{\pgfqpoint{3.087841in}{1.788750in}}%
\pgfpathcurveto{\pgfqpoint{3.087841in}{1.777699in}}{\pgfqpoint{3.092231in}{1.767100in}}{\pgfqpoint{3.100045in}{1.759287in}}%
\pgfpathcurveto{\pgfqpoint{3.107859in}{1.751473in}}{\pgfqpoint{3.118458in}{1.747083in}}{\pgfqpoint{3.129508in}{1.747083in}}%
\pgfpathclose%
\pgfusepath{stroke,fill}%
\end{pgfscope}%
\begin{pgfscope}%
\pgfpathrectangle{\pgfqpoint{0.511823in}{0.504323in}}{\pgfqpoint{3.218177in}{3.225677in}} %
\pgfusepath{clip}%
\pgfsetbuttcap%
\pgfsetroundjoin%
\definecolor{currentfill}{rgb}{0.501961,0.000000,0.000000}%
\pgfsetfillcolor{currentfill}%
\pgfsetfillopacity{0.400000}%
\pgfsetlinewidth{0.501875pt}%
\definecolor{currentstroke}{rgb}{0.501961,0.000000,0.000000}%
\pgfsetstrokecolor{currentstroke}%
\pgfsetstrokeopacity{0.400000}%
\pgfsetdash{}{0pt}%
\pgfpathmoveto{\pgfqpoint{3.089115in}{1.737495in}}%
\pgfpathcurveto{\pgfqpoint{3.100165in}{1.737495in}}{\pgfqpoint{3.110764in}{1.741886in}}{\pgfqpoint{3.118577in}{1.749699in}}%
\pgfpathcurveto{\pgfqpoint{3.126391in}{1.757513in}}{\pgfqpoint{3.130781in}{1.768112in}}{\pgfqpoint{3.130781in}{1.779162in}}%
\pgfpathcurveto{\pgfqpoint{3.130781in}{1.790212in}}{\pgfqpoint{3.126391in}{1.800811in}}{\pgfqpoint{3.118577in}{1.808625in}}%
\pgfpathcurveto{\pgfqpoint{3.110764in}{1.816438in}}{\pgfqpoint{3.100165in}{1.820829in}}{\pgfqpoint{3.089115in}{1.820829in}}%
\pgfpathcurveto{\pgfqpoint{3.078064in}{1.820829in}}{\pgfqpoint{3.067465in}{1.816438in}}{\pgfqpoint{3.059652in}{1.808625in}}%
\pgfpathcurveto{\pgfqpoint{3.051838in}{1.800811in}}{\pgfqpoint{3.047448in}{1.790212in}}{\pgfqpoint{3.047448in}{1.779162in}}%
\pgfpathcurveto{\pgfqpoint{3.047448in}{1.768112in}}{\pgfqpoint{3.051838in}{1.757513in}}{\pgfqpoint{3.059652in}{1.749699in}}%
\pgfpathcurveto{\pgfqpoint{3.067465in}{1.741886in}}{\pgfqpoint{3.078064in}{1.737495in}}{\pgfqpoint{3.089115in}{1.737495in}}%
\pgfpathclose%
\pgfusepath{stroke,fill}%
\end{pgfscope}%
\begin{pgfscope}%
\pgfpathrectangle{\pgfqpoint{0.511823in}{0.504323in}}{\pgfqpoint{3.218177in}{3.225677in}} %
\pgfusepath{clip}%
\pgfsetbuttcap%
\pgfsetroundjoin%
\definecolor{currentfill}{rgb}{0.501961,0.000000,0.000000}%
\pgfsetfillcolor{currentfill}%
\pgfsetfillopacity{0.400000}%
\pgfsetlinewidth{0.501875pt}%
\definecolor{currentstroke}{rgb}{0.501961,0.000000,0.000000}%
\pgfsetstrokecolor{currentstroke}%
\pgfsetstrokeopacity{0.400000}%
\pgfsetdash{}{0pt}%
\pgfpathmoveto{\pgfqpoint{2.949255in}{1.681835in}}%
\pgfpathcurveto{\pgfqpoint{2.960305in}{1.681835in}}{\pgfqpoint{2.970904in}{1.686225in}}{\pgfqpoint{2.978717in}{1.694038in}}%
\pgfpathcurveto{\pgfqpoint{2.986531in}{1.701852in}}{\pgfqpoint{2.990921in}{1.712451in}}{\pgfqpoint{2.990921in}{1.723501in}}%
\pgfpathcurveto{\pgfqpoint{2.990921in}{1.734551in}}{\pgfqpoint{2.986531in}{1.745150in}}{\pgfqpoint{2.978717in}{1.752964in}}%
\pgfpathcurveto{\pgfqpoint{2.970904in}{1.760778in}}{\pgfqpoint{2.960305in}{1.765168in}}{\pgfqpoint{2.949255in}{1.765168in}}%
\pgfpathcurveto{\pgfqpoint{2.938205in}{1.765168in}}{\pgfqpoint{2.927605in}{1.760778in}}{\pgfqpoint{2.919792in}{1.752964in}}%
\pgfpathcurveto{\pgfqpoint{2.911978in}{1.745150in}}{\pgfqpoint{2.907588in}{1.734551in}}{\pgfqpoint{2.907588in}{1.723501in}}%
\pgfpathcurveto{\pgfqpoint{2.907588in}{1.712451in}}{\pgfqpoint{2.911978in}{1.701852in}}{\pgfqpoint{2.919792in}{1.694038in}}%
\pgfpathcurveto{\pgfqpoint{2.927605in}{1.686225in}}{\pgfqpoint{2.938205in}{1.681835in}}{\pgfqpoint{2.949255in}{1.681835in}}%
\pgfpathclose%
\pgfusepath{stroke,fill}%
\end{pgfscope}%
\begin{pgfscope}%
\pgfpathrectangle{\pgfqpoint{0.511823in}{0.504323in}}{\pgfqpoint{3.218177in}{3.225677in}} %
\pgfusepath{clip}%
\pgfsetbuttcap%
\pgfsetroundjoin%
\definecolor{currentfill}{rgb}{0.501961,0.000000,0.000000}%
\pgfsetfillcolor{currentfill}%
\pgfsetfillopacity{0.400000}%
\pgfsetlinewidth{0.501875pt}%
\definecolor{currentstroke}{rgb}{0.501961,0.000000,0.000000}%
\pgfsetstrokecolor{currentstroke}%
\pgfsetstrokeopacity{0.400000}%
\pgfsetdash{}{0pt}%
\pgfpathmoveto{\pgfqpoint{3.144044in}{1.780515in}}%
\pgfpathcurveto{\pgfqpoint{3.155094in}{1.780515in}}{\pgfqpoint{3.165693in}{1.784905in}}{\pgfqpoint{3.173506in}{1.792718in}}%
\pgfpathcurveto{\pgfqpoint{3.181320in}{1.800532in}}{\pgfqpoint{3.185710in}{1.811131in}}{\pgfqpoint{3.185710in}{1.822181in}}%
\pgfpathcurveto{\pgfqpoint{3.185710in}{1.833231in}}{\pgfqpoint{3.181320in}{1.843830in}}{\pgfqpoint{3.173506in}{1.851644in}}%
\pgfpathcurveto{\pgfqpoint{3.165693in}{1.859458in}}{\pgfqpoint{3.155094in}{1.863848in}}{\pgfqpoint{3.144044in}{1.863848in}}%
\pgfpathcurveto{\pgfqpoint{3.132993in}{1.863848in}}{\pgfqpoint{3.122394in}{1.859458in}}{\pgfqpoint{3.114581in}{1.851644in}}%
\pgfpathcurveto{\pgfqpoint{3.106767in}{1.843830in}}{\pgfqpoint{3.102377in}{1.833231in}}{\pgfqpoint{3.102377in}{1.822181in}}%
\pgfpathcurveto{\pgfqpoint{3.102377in}{1.811131in}}{\pgfqpoint{3.106767in}{1.800532in}}{\pgfqpoint{3.114581in}{1.792718in}}%
\pgfpathcurveto{\pgfqpoint{3.122394in}{1.784905in}}{\pgfqpoint{3.132993in}{1.780515in}}{\pgfqpoint{3.144044in}{1.780515in}}%
\pgfpathclose%
\pgfusepath{stroke,fill}%
\end{pgfscope}%
\begin{pgfscope}%
\pgfpathrectangle{\pgfqpoint{0.511823in}{0.504323in}}{\pgfqpoint{3.218177in}{3.225677in}} %
\pgfusepath{clip}%
\pgfsetbuttcap%
\pgfsetroundjoin%
\definecolor{currentfill}{rgb}{0.501961,0.000000,0.000000}%
\pgfsetfillcolor{currentfill}%
\pgfsetfillopacity{0.400000}%
\pgfsetlinewidth{0.501875pt}%
\definecolor{currentstroke}{rgb}{0.501961,0.000000,0.000000}%
\pgfsetstrokecolor{currentstroke}%
\pgfsetstrokeopacity{0.400000}%
\pgfsetdash{}{0pt}%
\pgfpathmoveto{\pgfqpoint{3.068562in}{1.754205in}}%
\pgfpathcurveto{\pgfqpoint{3.079612in}{1.754205in}}{\pgfqpoint{3.090211in}{1.758596in}}{\pgfqpoint{3.098025in}{1.766409in}}%
\pgfpathcurveto{\pgfqpoint{3.105838in}{1.774223in}}{\pgfqpoint{3.110229in}{1.784822in}}{\pgfqpoint{3.110229in}{1.795872in}}%
\pgfpathcurveto{\pgfqpoint{3.110229in}{1.806922in}}{\pgfqpoint{3.105838in}{1.817521in}}{\pgfqpoint{3.098025in}{1.825335in}}%
\pgfpathcurveto{\pgfqpoint{3.090211in}{1.833148in}}{\pgfqpoint{3.079612in}{1.837539in}}{\pgfqpoint{3.068562in}{1.837539in}}%
\pgfpathcurveto{\pgfqpoint{3.057512in}{1.837539in}}{\pgfqpoint{3.046913in}{1.833148in}}{\pgfqpoint{3.039099in}{1.825335in}}%
\pgfpathcurveto{\pgfqpoint{3.031286in}{1.817521in}}{\pgfqpoint{3.026895in}{1.806922in}}{\pgfqpoint{3.026895in}{1.795872in}}%
\pgfpathcurveto{\pgfqpoint{3.026895in}{1.784822in}}{\pgfqpoint{3.031286in}{1.774223in}}{\pgfqpoint{3.039099in}{1.766409in}}%
\pgfpathcurveto{\pgfqpoint{3.046913in}{1.758596in}}{\pgfqpoint{3.057512in}{1.754205in}}{\pgfqpoint{3.068562in}{1.754205in}}%
\pgfpathclose%
\pgfusepath{stroke,fill}%
\end{pgfscope}%
\begin{pgfscope}%
\pgfpathrectangle{\pgfqpoint{0.511823in}{0.504323in}}{\pgfqpoint{3.218177in}{3.225677in}} %
\pgfusepath{clip}%
\pgfsetbuttcap%
\pgfsetroundjoin%
\definecolor{currentfill}{rgb}{0.501961,0.000000,0.000000}%
\pgfsetfillcolor{currentfill}%
\pgfsetfillopacity{0.400000}%
\pgfsetlinewidth{0.501875pt}%
\definecolor{currentstroke}{rgb}{0.501961,0.000000,0.000000}%
\pgfsetstrokecolor{currentstroke}%
\pgfsetstrokeopacity{0.400000}%
\pgfsetdash{}{0pt}%
\pgfpathmoveto{\pgfqpoint{3.177783in}{1.814459in}}%
\pgfpathcurveto{\pgfqpoint{3.188833in}{1.814459in}}{\pgfqpoint{3.199432in}{1.818849in}}{\pgfqpoint{3.207245in}{1.826663in}}%
\pgfpathcurveto{\pgfqpoint{3.215059in}{1.834477in}}{\pgfqpoint{3.219449in}{1.845076in}}{\pgfqpoint{3.219449in}{1.856126in}}%
\pgfpathcurveto{\pgfqpoint{3.219449in}{1.867176in}}{\pgfqpoint{3.215059in}{1.877775in}}{\pgfqpoint{3.207245in}{1.885589in}}%
\pgfpathcurveto{\pgfqpoint{3.199432in}{1.893402in}}{\pgfqpoint{3.188833in}{1.897792in}}{\pgfqpoint{3.177783in}{1.897792in}}%
\pgfpathcurveto{\pgfqpoint{3.166732in}{1.897792in}}{\pgfqpoint{3.156133in}{1.893402in}}{\pgfqpoint{3.148320in}{1.885589in}}%
\pgfpathcurveto{\pgfqpoint{3.140506in}{1.877775in}}{\pgfqpoint{3.136116in}{1.867176in}}{\pgfqpoint{3.136116in}{1.856126in}}%
\pgfpathcurveto{\pgfqpoint{3.136116in}{1.845076in}}{\pgfqpoint{3.140506in}{1.834477in}}{\pgfqpoint{3.148320in}{1.826663in}}%
\pgfpathcurveto{\pgfqpoint{3.156133in}{1.818849in}}{\pgfqpoint{3.166732in}{1.814459in}}{\pgfqpoint{3.177783in}{1.814459in}}%
\pgfpathclose%
\pgfusepath{stroke,fill}%
\end{pgfscope}%
\begin{pgfscope}%
\pgfpathrectangle{\pgfqpoint{0.511823in}{0.504323in}}{\pgfqpoint{3.218177in}{3.225677in}} %
\pgfusepath{clip}%
\pgfsetbuttcap%
\pgfsetroundjoin%
\definecolor{currentfill}{rgb}{0.501961,0.000000,0.000000}%
\pgfsetfillcolor{currentfill}%
\pgfsetfillopacity{0.400000}%
\pgfsetlinewidth{0.501875pt}%
\definecolor{currentstroke}{rgb}{0.501961,0.000000,0.000000}%
\pgfsetstrokecolor{currentstroke}%
\pgfsetstrokeopacity{0.400000}%
\pgfsetdash{}{0pt}%
\pgfpathmoveto{\pgfqpoint{2.909105in}{1.695969in}}%
\pgfpathcurveto{\pgfqpoint{2.920155in}{1.695969in}}{\pgfqpoint{2.930754in}{1.700360in}}{\pgfqpoint{2.938568in}{1.708173in}}%
\pgfpathcurveto{\pgfqpoint{2.946381in}{1.715987in}}{\pgfqpoint{2.950771in}{1.726586in}}{\pgfqpoint{2.950771in}{1.737636in}}%
\pgfpathcurveto{\pgfqpoint{2.950771in}{1.748686in}}{\pgfqpoint{2.946381in}{1.759285in}}{\pgfqpoint{2.938568in}{1.767099in}}%
\pgfpathcurveto{\pgfqpoint{2.930754in}{1.774912in}}{\pgfqpoint{2.920155in}{1.779303in}}{\pgfqpoint{2.909105in}{1.779303in}}%
\pgfpathcurveto{\pgfqpoint{2.898055in}{1.779303in}}{\pgfqpoint{2.887456in}{1.774912in}}{\pgfqpoint{2.879642in}{1.767099in}}%
\pgfpathcurveto{\pgfqpoint{2.871828in}{1.759285in}}{\pgfqpoint{2.867438in}{1.748686in}}{\pgfqpoint{2.867438in}{1.737636in}}%
\pgfpathcurveto{\pgfqpoint{2.867438in}{1.726586in}}{\pgfqpoint{2.871828in}{1.715987in}}{\pgfqpoint{2.879642in}{1.708173in}}%
\pgfpathcurveto{\pgfqpoint{2.887456in}{1.700360in}}{\pgfqpoint{2.898055in}{1.695969in}}{\pgfqpoint{2.909105in}{1.695969in}}%
\pgfpathclose%
\pgfusepath{stroke,fill}%
\end{pgfscope}%
\begin{pgfscope}%
\pgfpathrectangle{\pgfqpoint{0.511823in}{0.504323in}}{\pgfqpoint{3.218177in}{3.225677in}} %
\pgfusepath{clip}%
\pgfsetbuttcap%
\pgfsetroundjoin%
\definecolor{currentfill}{rgb}{0.501961,0.000000,0.000000}%
\pgfsetfillcolor{currentfill}%
\pgfsetfillopacity{0.400000}%
\pgfsetlinewidth{0.501875pt}%
\definecolor{currentstroke}{rgb}{0.501961,0.000000,0.000000}%
\pgfsetstrokecolor{currentstroke}%
\pgfsetstrokeopacity{0.400000}%
\pgfsetdash{}{0pt}%
\pgfpathmoveto{\pgfqpoint{2.962653in}{1.729835in}}%
\pgfpathcurveto{\pgfqpoint{2.973703in}{1.729835in}}{\pgfqpoint{2.984302in}{1.734225in}}{\pgfqpoint{2.992116in}{1.742039in}}%
\pgfpathcurveto{\pgfqpoint{2.999930in}{1.749853in}}{\pgfqpoint{3.004320in}{1.760452in}}{\pgfqpoint{3.004320in}{1.771502in}}%
\pgfpathcurveto{\pgfqpoint{3.004320in}{1.782552in}}{\pgfqpoint{2.999930in}{1.793151in}}{\pgfqpoint{2.992116in}{1.800965in}}%
\pgfpathcurveto{\pgfqpoint{2.984302in}{1.808778in}}{\pgfqpoint{2.973703in}{1.813168in}}{\pgfqpoint{2.962653in}{1.813168in}}%
\pgfpathcurveto{\pgfqpoint{2.951603in}{1.813168in}}{\pgfqpoint{2.941004in}{1.808778in}}{\pgfqpoint{2.933191in}{1.800965in}}%
\pgfpathcurveto{\pgfqpoint{2.925377in}{1.793151in}}{\pgfqpoint{2.920987in}{1.782552in}}{\pgfqpoint{2.920987in}{1.771502in}}%
\pgfpathcurveto{\pgfqpoint{2.920987in}{1.760452in}}{\pgfqpoint{2.925377in}{1.749853in}}{\pgfqpoint{2.933191in}{1.742039in}}%
\pgfpathcurveto{\pgfqpoint{2.941004in}{1.734225in}}{\pgfqpoint{2.951603in}{1.729835in}}{\pgfqpoint{2.962653in}{1.729835in}}%
\pgfpathclose%
\pgfusepath{stroke,fill}%
\end{pgfscope}%
\begin{pgfscope}%
\pgfpathrectangle{\pgfqpoint{0.511823in}{0.504323in}}{\pgfqpoint{3.218177in}{3.225677in}} %
\pgfusepath{clip}%
\pgfsetbuttcap%
\pgfsetroundjoin%
\definecolor{currentfill}{rgb}{0.501961,0.000000,0.000000}%
\pgfsetfillcolor{currentfill}%
\pgfsetfillopacity{0.400000}%
\pgfsetlinewidth{0.501875pt}%
\definecolor{currentstroke}{rgb}{0.501961,0.000000,0.000000}%
\pgfsetstrokecolor{currentstroke}%
\pgfsetstrokeopacity{0.400000}%
\pgfsetdash{}{0pt}%
\pgfpathmoveto{\pgfqpoint{3.080089in}{1.794957in}}%
\pgfpathcurveto{\pgfqpoint{3.091139in}{1.794957in}}{\pgfqpoint{3.101738in}{1.799348in}}{\pgfqpoint{3.109552in}{1.807161in}}%
\pgfpathcurveto{\pgfqpoint{3.117366in}{1.814975in}}{\pgfqpoint{3.121756in}{1.825574in}}{\pgfqpoint{3.121756in}{1.836624in}}%
\pgfpathcurveto{\pgfqpoint{3.121756in}{1.847674in}}{\pgfqpoint{3.117366in}{1.858273in}}{\pgfqpoint{3.109552in}{1.866087in}}%
\pgfpathcurveto{\pgfqpoint{3.101738in}{1.873901in}}{\pgfqpoint{3.091139in}{1.878291in}}{\pgfqpoint{3.080089in}{1.878291in}}%
\pgfpathcurveto{\pgfqpoint{3.069039in}{1.878291in}}{\pgfqpoint{3.058440in}{1.873901in}}{\pgfqpoint{3.050627in}{1.866087in}}%
\pgfpathcurveto{\pgfqpoint{3.042813in}{1.858273in}}{\pgfqpoint{3.038423in}{1.847674in}}{\pgfqpoint{3.038423in}{1.836624in}}%
\pgfpathcurveto{\pgfqpoint{3.038423in}{1.825574in}}{\pgfqpoint{3.042813in}{1.814975in}}{\pgfqpoint{3.050627in}{1.807161in}}%
\pgfpathcurveto{\pgfqpoint{3.058440in}{1.799348in}}{\pgfqpoint{3.069039in}{1.794957in}}{\pgfqpoint{3.080089in}{1.794957in}}%
\pgfpathclose%
\pgfusepath{stroke,fill}%
\end{pgfscope}%
\begin{pgfscope}%
\pgfpathrectangle{\pgfqpoint{0.511823in}{0.504323in}}{\pgfqpoint{3.218177in}{3.225677in}} %
\pgfusepath{clip}%
\pgfsetbuttcap%
\pgfsetroundjoin%
\definecolor{currentfill}{rgb}{0.501961,0.000000,0.000000}%
\pgfsetfillcolor{currentfill}%
\pgfsetfillopacity{0.400000}%
\pgfsetlinewidth{0.501875pt}%
\definecolor{currentstroke}{rgb}{0.501961,0.000000,0.000000}%
\pgfsetstrokecolor{currentstroke}%
\pgfsetstrokeopacity{0.400000}%
\pgfsetdash{}{0pt}%
\pgfpathmoveto{\pgfqpoint{2.836048in}{1.685184in}}%
\pgfpathcurveto{\pgfqpoint{2.847098in}{1.685184in}}{\pgfqpoint{2.857697in}{1.689574in}}{\pgfqpoint{2.865510in}{1.697388in}}%
\pgfpathcurveto{\pgfqpoint{2.873324in}{1.705201in}}{\pgfqpoint{2.877714in}{1.715801in}}{\pgfqpoint{2.877714in}{1.726851in}}%
\pgfpathcurveto{\pgfqpoint{2.877714in}{1.737901in}}{\pgfqpoint{2.873324in}{1.748500in}}{\pgfqpoint{2.865510in}{1.756313in}}%
\pgfpathcurveto{\pgfqpoint{2.857697in}{1.764127in}}{\pgfqpoint{2.847098in}{1.768517in}}{\pgfqpoint{2.836048in}{1.768517in}}%
\pgfpathcurveto{\pgfqpoint{2.824998in}{1.768517in}}{\pgfqpoint{2.814399in}{1.764127in}}{\pgfqpoint{2.806585in}{1.756313in}}%
\pgfpathcurveto{\pgfqpoint{2.798771in}{1.748500in}}{\pgfqpoint{2.794381in}{1.737901in}}{\pgfqpoint{2.794381in}{1.726851in}}%
\pgfpathcurveto{\pgfqpoint{2.794381in}{1.715801in}}{\pgfqpoint{2.798771in}{1.705201in}}{\pgfqpoint{2.806585in}{1.697388in}}%
\pgfpathcurveto{\pgfqpoint{2.814399in}{1.689574in}}{\pgfqpoint{2.824998in}{1.685184in}}{\pgfqpoint{2.836048in}{1.685184in}}%
\pgfpathclose%
\pgfusepath{stroke,fill}%
\end{pgfscope}%
\begin{pgfscope}%
\pgfpathrectangle{\pgfqpoint{0.511823in}{0.504323in}}{\pgfqpoint{3.218177in}{3.225677in}} %
\pgfusepath{clip}%
\pgfsetbuttcap%
\pgfsetroundjoin%
\definecolor{currentfill}{rgb}{0.501961,0.000000,0.000000}%
\pgfsetfillcolor{currentfill}%
\pgfsetfillopacity{0.400000}%
\pgfsetlinewidth{0.501875pt}%
\definecolor{currentstroke}{rgb}{0.501961,0.000000,0.000000}%
\pgfsetstrokecolor{currentstroke}%
\pgfsetstrokeopacity{0.400000}%
\pgfsetdash{}{0pt}%
\pgfpathmoveto{\pgfqpoint{3.018267in}{1.782513in}}%
\pgfpathcurveto{\pgfqpoint{3.029317in}{1.782513in}}{\pgfqpoint{3.039916in}{1.786903in}}{\pgfqpoint{3.047730in}{1.794717in}}%
\pgfpathcurveto{\pgfqpoint{3.055544in}{1.802531in}}{\pgfqpoint{3.059934in}{1.813130in}}{\pgfqpoint{3.059934in}{1.824180in}}%
\pgfpathcurveto{\pgfqpoint{3.059934in}{1.835230in}}{\pgfqpoint{3.055544in}{1.845829in}}{\pgfqpoint{3.047730in}{1.853643in}}%
\pgfpathcurveto{\pgfqpoint{3.039916in}{1.861456in}}{\pgfqpoint{3.029317in}{1.865846in}}{\pgfqpoint{3.018267in}{1.865846in}}%
\pgfpathcurveto{\pgfqpoint{3.007217in}{1.865846in}}{\pgfqpoint{2.996618in}{1.861456in}}{\pgfqpoint{2.988804in}{1.853643in}}%
\pgfpathcurveto{\pgfqpoint{2.980991in}{1.845829in}}{\pgfqpoint{2.976600in}{1.835230in}}{\pgfqpoint{2.976600in}{1.824180in}}%
\pgfpathcurveto{\pgfqpoint{2.976600in}{1.813130in}}{\pgfqpoint{2.980991in}{1.802531in}}{\pgfqpoint{2.988804in}{1.794717in}}%
\pgfpathcurveto{\pgfqpoint{2.996618in}{1.786903in}}{\pgfqpoint{3.007217in}{1.782513in}}{\pgfqpoint{3.018267in}{1.782513in}}%
\pgfpathclose%
\pgfusepath{stroke,fill}%
\end{pgfscope}%
\begin{pgfscope}%
\pgfpathrectangle{\pgfqpoint{0.511823in}{0.504323in}}{\pgfqpoint{3.218177in}{3.225677in}} %
\pgfusepath{clip}%
\pgfsetbuttcap%
\pgfsetroundjoin%
\definecolor{currentfill}{rgb}{0.501961,0.000000,0.000000}%
\pgfsetfillcolor{currentfill}%
\pgfsetfillopacity{0.400000}%
\pgfsetlinewidth{0.501875pt}%
\definecolor{currentstroke}{rgb}{0.501961,0.000000,0.000000}%
\pgfsetstrokecolor{currentstroke}%
\pgfsetstrokeopacity{0.400000}%
\pgfsetdash{}{0pt}%
\pgfpathmoveto{\pgfqpoint{2.977572in}{1.771156in}}%
\pgfpathcurveto{\pgfqpoint{2.988622in}{1.771156in}}{\pgfqpoint{2.999221in}{1.775547in}}{\pgfqpoint{3.007035in}{1.783360in}}%
\pgfpathcurveto{\pgfqpoint{3.014849in}{1.791174in}}{\pgfqpoint{3.019239in}{1.801773in}}{\pgfqpoint{3.019239in}{1.812823in}}%
\pgfpathcurveto{\pgfqpoint{3.019239in}{1.823873in}}{\pgfqpoint{3.014849in}{1.834472in}}{\pgfqpoint{3.007035in}{1.842286in}}%
\pgfpathcurveto{\pgfqpoint{2.999221in}{1.850100in}}{\pgfqpoint{2.988622in}{1.854490in}}{\pgfqpoint{2.977572in}{1.854490in}}%
\pgfpathcurveto{\pgfqpoint{2.966522in}{1.854490in}}{\pgfqpoint{2.955923in}{1.850100in}}{\pgfqpoint{2.948109in}{1.842286in}}%
\pgfpathcurveto{\pgfqpoint{2.940296in}{1.834472in}}{\pgfqpoint{2.935906in}{1.823873in}}{\pgfqpoint{2.935906in}{1.812823in}}%
\pgfpathcurveto{\pgfqpoint{2.935906in}{1.801773in}}{\pgfqpoint{2.940296in}{1.791174in}}{\pgfqpoint{2.948109in}{1.783360in}}%
\pgfpathcurveto{\pgfqpoint{2.955923in}{1.775547in}}{\pgfqpoint{2.966522in}{1.771156in}}{\pgfqpoint{2.977572in}{1.771156in}}%
\pgfpathclose%
\pgfusepath{stroke,fill}%
\end{pgfscope}%
\begin{pgfscope}%
\pgfpathrectangle{\pgfqpoint{0.511823in}{0.504323in}}{\pgfqpoint{3.218177in}{3.225677in}} %
\pgfusepath{clip}%
\pgfsetbuttcap%
\pgfsetroundjoin%
\definecolor{currentfill}{rgb}{0.501961,0.000000,0.000000}%
\pgfsetfillcolor{currentfill}%
\pgfsetfillopacity{0.400000}%
\pgfsetlinewidth{0.501875pt}%
\definecolor{currentstroke}{rgb}{0.501961,0.000000,0.000000}%
\pgfsetstrokecolor{currentstroke}%
\pgfsetstrokeopacity{0.400000}%
\pgfsetdash{}{0pt}%
\pgfpathmoveto{\pgfqpoint{2.977581in}{1.779776in}}%
\pgfpathcurveto{\pgfqpoint{2.988631in}{1.779776in}}{\pgfqpoint{2.999230in}{1.784166in}}{\pgfqpoint{3.007044in}{1.791980in}}%
\pgfpathcurveto{\pgfqpoint{3.014857in}{1.799794in}}{\pgfqpoint{3.019248in}{1.810393in}}{\pgfqpoint{3.019248in}{1.821443in}}%
\pgfpathcurveto{\pgfqpoint{3.019248in}{1.832493in}}{\pgfqpoint{3.014857in}{1.843092in}}{\pgfqpoint{3.007044in}{1.850906in}}%
\pgfpathcurveto{\pgfqpoint{2.999230in}{1.858719in}}{\pgfqpoint{2.988631in}{1.863109in}}{\pgfqpoint{2.977581in}{1.863109in}}%
\pgfpathcurveto{\pgfqpoint{2.966531in}{1.863109in}}{\pgfqpoint{2.955932in}{1.858719in}}{\pgfqpoint{2.948118in}{1.850906in}}%
\pgfpathcurveto{\pgfqpoint{2.940305in}{1.843092in}}{\pgfqpoint{2.935914in}{1.832493in}}{\pgfqpoint{2.935914in}{1.821443in}}%
\pgfpathcurveto{\pgfqpoint{2.935914in}{1.810393in}}{\pgfqpoint{2.940305in}{1.799794in}}{\pgfqpoint{2.948118in}{1.791980in}}%
\pgfpathcurveto{\pgfqpoint{2.955932in}{1.784166in}}{\pgfqpoint{2.966531in}{1.779776in}}{\pgfqpoint{2.977581in}{1.779776in}}%
\pgfpathclose%
\pgfusepath{stroke,fill}%
\end{pgfscope}%
\begin{pgfscope}%
\pgfpathrectangle{\pgfqpoint{0.511823in}{0.504323in}}{\pgfqpoint{3.218177in}{3.225677in}} %
\pgfusepath{clip}%
\pgfsetbuttcap%
\pgfsetroundjoin%
\definecolor{currentfill}{rgb}{0.501961,0.000000,0.000000}%
\pgfsetfillcolor{currentfill}%
\pgfsetfillopacity{0.400000}%
\pgfsetlinewidth{0.501875pt}%
\definecolor{currentstroke}{rgb}{0.501961,0.000000,0.000000}%
\pgfsetstrokecolor{currentstroke}%
\pgfsetstrokeopacity{0.400000}%
\pgfsetdash{}{0pt}%
\pgfpathmoveto{\pgfqpoint{2.992731in}{1.796016in}}%
\pgfpathcurveto{\pgfqpoint{3.003781in}{1.796016in}}{\pgfqpoint{3.014380in}{1.800406in}}{\pgfqpoint{3.022194in}{1.808220in}}%
\pgfpathcurveto{\pgfqpoint{3.030008in}{1.816033in}}{\pgfqpoint{3.034398in}{1.826633in}}{\pgfqpoint{3.034398in}{1.837683in}}%
\pgfpathcurveto{\pgfqpoint{3.034398in}{1.848733in}}{\pgfqpoint{3.030008in}{1.859332in}}{\pgfqpoint{3.022194in}{1.867145in}}%
\pgfpathcurveto{\pgfqpoint{3.014380in}{1.874959in}}{\pgfqpoint{3.003781in}{1.879349in}}{\pgfqpoint{2.992731in}{1.879349in}}%
\pgfpathcurveto{\pgfqpoint{2.981681in}{1.879349in}}{\pgfqpoint{2.971082in}{1.874959in}}{\pgfqpoint{2.963268in}{1.867145in}}%
\pgfpathcurveto{\pgfqpoint{2.955455in}{1.859332in}}{\pgfqpoint{2.951065in}{1.848733in}}{\pgfqpoint{2.951065in}{1.837683in}}%
\pgfpathcurveto{\pgfqpoint{2.951065in}{1.826633in}}{\pgfqpoint{2.955455in}{1.816033in}}{\pgfqpoint{2.963268in}{1.808220in}}%
\pgfpathcurveto{\pgfqpoint{2.971082in}{1.800406in}}{\pgfqpoint{2.981681in}{1.796016in}}{\pgfqpoint{2.992731in}{1.796016in}}%
\pgfpathclose%
\pgfusepath{stroke,fill}%
\end{pgfscope}%
\begin{pgfscope}%
\pgfpathrectangle{\pgfqpoint{0.511823in}{0.504323in}}{\pgfqpoint{3.218177in}{3.225677in}} %
\pgfusepath{clip}%
\pgfsetbuttcap%
\pgfsetroundjoin%
\definecolor{currentfill}{rgb}{0.501961,0.000000,0.000000}%
\pgfsetfillcolor{currentfill}%
\pgfsetfillopacity{0.400000}%
\pgfsetlinewidth{0.501875pt}%
\definecolor{currentstroke}{rgb}{0.501961,0.000000,0.000000}%
\pgfsetstrokecolor{currentstroke}%
\pgfsetstrokeopacity{0.400000}%
\pgfsetdash{}{0pt}%
\pgfpathmoveto{\pgfqpoint{2.758304in}{1.686330in}}%
\pgfpathcurveto{\pgfqpoint{2.769354in}{1.686330in}}{\pgfqpoint{2.779953in}{1.690720in}}{\pgfqpoint{2.787766in}{1.698534in}}%
\pgfpathcurveto{\pgfqpoint{2.795580in}{1.706347in}}{\pgfqpoint{2.799970in}{1.716946in}}{\pgfqpoint{2.799970in}{1.727996in}}%
\pgfpathcurveto{\pgfqpoint{2.799970in}{1.739047in}}{\pgfqpoint{2.795580in}{1.749646in}}{\pgfqpoint{2.787766in}{1.757459in}}%
\pgfpathcurveto{\pgfqpoint{2.779953in}{1.765273in}}{\pgfqpoint{2.769354in}{1.769663in}}{\pgfqpoint{2.758304in}{1.769663in}}%
\pgfpathcurveto{\pgfqpoint{2.747253in}{1.769663in}}{\pgfqpoint{2.736654in}{1.765273in}}{\pgfqpoint{2.728841in}{1.757459in}}%
\pgfpathcurveto{\pgfqpoint{2.721027in}{1.749646in}}{\pgfqpoint{2.716637in}{1.739047in}}{\pgfqpoint{2.716637in}{1.727996in}}%
\pgfpathcurveto{\pgfqpoint{2.716637in}{1.716946in}}{\pgfqpoint{2.721027in}{1.706347in}}{\pgfqpoint{2.728841in}{1.698534in}}%
\pgfpathcurveto{\pgfqpoint{2.736654in}{1.690720in}}{\pgfqpoint{2.747253in}{1.686330in}}{\pgfqpoint{2.758304in}{1.686330in}}%
\pgfpathclose%
\pgfusepath{stroke,fill}%
\end{pgfscope}%
\begin{pgfscope}%
\pgfpathrectangle{\pgfqpoint{0.511823in}{0.504323in}}{\pgfqpoint{3.218177in}{3.225677in}} %
\pgfusepath{clip}%
\pgfsetbuttcap%
\pgfsetroundjoin%
\definecolor{currentfill}{rgb}{0.501961,0.000000,0.000000}%
\pgfsetfillcolor{currentfill}%
\pgfsetfillopacity{0.400000}%
\pgfsetlinewidth{0.501875pt}%
\definecolor{currentstroke}{rgb}{0.501961,0.000000,0.000000}%
\pgfsetstrokecolor{currentstroke}%
\pgfsetstrokeopacity{0.400000}%
\pgfsetdash{}{0pt}%
\pgfpathmoveto{\pgfqpoint{3.019073in}{1.826914in}}%
\pgfpathcurveto{\pgfqpoint{3.030123in}{1.826914in}}{\pgfqpoint{3.040722in}{1.831304in}}{\pgfqpoint{3.048535in}{1.839118in}}%
\pgfpathcurveto{\pgfqpoint{3.056349in}{1.846931in}}{\pgfqpoint{3.060739in}{1.857531in}}{\pgfqpoint{3.060739in}{1.868581in}}%
\pgfpathcurveto{\pgfqpoint{3.060739in}{1.879631in}}{\pgfqpoint{3.056349in}{1.890230in}}{\pgfqpoint{3.048535in}{1.898043in}}%
\pgfpathcurveto{\pgfqpoint{3.040722in}{1.905857in}}{\pgfqpoint{3.030123in}{1.910247in}}{\pgfqpoint{3.019073in}{1.910247in}}%
\pgfpathcurveto{\pgfqpoint{3.008022in}{1.910247in}}{\pgfqpoint{2.997423in}{1.905857in}}{\pgfqpoint{2.989610in}{1.898043in}}%
\pgfpathcurveto{\pgfqpoint{2.981796in}{1.890230in}}{\pgfqpoint{2.977406in}{1.879631in}}{\pgfqpoint{2.977406in}{1.868581in}}%
\pgfpathcurveto{\pgfqpoint{2.977406in}{1.857531in}}{\pgfqpoint{2.981796in}{1.846931in}}{\pgfqpoint{2.989610in}{1.839118in}}%
\pgfpathcurveto{\pgfqpoint{2.997423in}{1.831304in}}{\pgfqpoint{3.008022in}{1.826914in}}{\pgfqpoint{3.019073in}{1.826914in}}%
\pgfpathclose%
\pgfusepath{stroke,fill}%
\end{pgfscope}%
\begin{pgfscope}%
\pgfpathrectangle{\pgfqpoint{0.511823in}{0.504323in}}{\pgfqpoint{3.218177in}{3.225677in}} %
\pgfusepath{clip}%
\pgfsetbuttcap%
\pgfsetroundjoin%
\definecolor{currentfill}{rgb}{0.501961,0.000000,0.000000}%
\pgfsetfillcolor{currentfill}%
\pgfsetfillopacity{0.400000}%
\pgfsetlinewidth{0.501875pt}%
\definecolor{currentstroke}{rgb}{0.501961,0.000000,0.000000}%
\pgfsetstrokecolor{currentstroke}%
\pgfsetstrokeopacity{0.400000}%
\pgfsetdash{}{0pt}%
\pgfpathmoveto{\pgfqpoint{3.018835in}{1.835684in}}%
\pgfpathcurveto{\pgfqpoint{3.029885in}{1.835684in}}{\pgfqpoint{3.040484in}{1.840074in}}{\pgfqpoint{3.048298in}{1.847888in}}%
\pgfpathcurveto{\pgfqpoint{3.056111in}{1.855701in}}{\pgfqpoint{3.060501in}{1.866300in}}{\pgfqpoint{3.060501in}{1.877350in}}%
\pgfpathcurveto{\pgfqpoint{3.060501in}{1.888400in}}{\pgfqpoint{3.056111in}{1.898999in}}{\pgfqpoint{3.048298in}{1.906813in}}%
\pgfpathcurveto{\pgfqpoint{3.040484in}{1.914627in}}{\pgfqpoint{3.029885in}{1.919017in}}{\pgfqpoint{3.018835in}{1.919017in}}%
\pgfpathcurveto{\pgfqpoint{3.007785in}{1.919017in}}{\pgfqpoint{2.997186in}{1.914627in}}{\pgfqpoint{2.989372in}{1.906813in}}%
\pgfpathcurveto{\pgfqpoint{2.981558in}{1.898999in}}{\pgfqpoint{2.977168in}{1.888400in}}{\pgfqpoint{2.977168in}{1.877350in}}%
\pgfpathcurveto{\pgfqpoint{2.977168in}{1.866300in}}{\pgfqpoint{2.981558in}{1.855701in}}{\pgfqpoint{2.989372in}{1.847888in}}%
\pgfpathcurveto{\pgfqpoint{2.997186in}{1.840074in}}{\pgfqpoint{3.007785in}{1.835684in}}{\pgfqpoint{3.018835in}{1.835684in}}%
\pgfpathclose%
\pgfusepath{stroke,fill}%
\end{pgfscope}%
\begin{pgfscope}%
\pgfpathrectangle{\pgfqpoint{0.511823in}{0.504323in}}{\pgfqpoint{3.218177in}{3.225677in}} %
\pgfusepath{clip}%
\pgfsetbuttcap%
\pgfsetroundjoin%
\definecolor{currentfill}{rgb}{0.501961,0.000000,0.000000}%
\pgfsetfillcolor{currentfill}%
\pgfsetfillopacity{0.400000}%
\pgfsetlinewidth{0.501875pt}%
\definecolor{currentstroke}{rgb}{0.501961,0.000000,0.000000}%
\pgfsetstrokecolor{currentstroke}%
\pgfsetstrokeopacity{0.400000}%
\pgfsetdash{}{0pt}%
\pgfpathmoveto{\pgfqpoint{3.127907in}{1.900949in}}%
\pgfpathcurveto{\pgfqpoint{3.138958in}{1.900949in}}{\pgfqpoint{3.149557in}{1.905339in}}{\pgfqpoint{3.157370in}{1.913153in}}%
\pgfpathcurveto{\pgfqpoint{3.165184in}{1.920966in}}{\pgfqpoint{3.169574in}{1.931565in}}{\pgfqpoint{3.169574in}{1.942616in}}%
\pgfpathcurveto{\pgfqpoint{3.169574in}{1.953666in}}{\pgfqpoint{3.165184in}{1.964265in}}{\pgfqpoint{3.157370in}{1.972078in}}%
\pgfpathcurveto{\pgfqpoint{3.149557in}{1.979892in}}{\pgfqpoint{3.138958in}{1.984282in}}{\pgfqpoint{3.127907in}{1.984282in}}%
\pgfpathcurveto{\pgfqpoint{3.116857in}{1.984282in}}{\pgfqpoint{3.106258in}{1.979892in}}{\pgfqpoint{3.098445in}{1.972078in}}%
\pgfpathcurveto{\pgfqpoint{3.090631in}{1.964265in}}{\pgfqpoint{3.086241in}{1.953666in}}{\pgfqpoint{3.086241in}{1.942616in}}%
\pgfpathcurveto{\pgfqpoint{3.086241in}{1.931565in}}{\pgfqpoint{3.090631in}{1.920966in}}{\pgfqpoint{3.098445in}{1.913153in}}%
\pgfpathcurveto{\pgfqpoint{3.106258in}{1.905339in}}{\pgfqpoint{3.116857in}{1.900949in}}{\pgfqpoint{3.127907in}{1.900949in}}%
\pgfpathclose%
\pgfusepath{stroke,fill}%
\end{pgfscope}%
\begin{pgfscope}%
\pgfpathrectangle{\pgfqpoint{0.511823in}{0.504323in}}{\pgfqpoint{3.218177in}{3.225677in}} %
\pgfusepath{clip}%
\pgfsetbuttcap%
\pgfsetroundjoin%
\definecolor{currentfill}{rgb}{0.501961,0.000000,0.000000}%
\pgfsetfillcolor{currentfill}%
\pgfsetfillopacity{0.400000}%
\pgfsetlinewidth{0.501875pt}%
\definecolor{currentstroke}{rgb}{0.501961,0.000000,0.000000}%
\pgfsetstrokecolor{currentstroke}%
\pgfsetstrokeopacity{0.400000}%
\pgfsetdash{}{0pt}%
\pgfpathmoveto{\pgfqpoint{3.169283in}{1.931853in}}%
\pgfpathcurveto{\pgfqpoint{3.180333in}{1.931853in}}{\pgfqpoint{3.190933in}{1.936243in}}{\pgfqpoint{3.198746in}{1.944057in}}%
\pgfpathcurveto{\pgfqpoint{3.206560in}{1.951870in}}{\pgfqpoint{3.210950in}{1.962469in}}{\pgfqpoint{3.210950in}{1.973519in}}%
\pgfpathcurveto{\pgfqpoint{3.210950in}{1.984570in}}{\pgfqpoint{3.206560in}{1.995169in}}{\pgfqpoint{3.198746in}{2.002982in}}%
\pgfpathcurveto{\pgfqpoint{3.190933in}{2.010796in}}{\pgfqpoint{3.180333in}{2.015186in}}{\pgfqpoint{3.169283in}{2.015186in}}%
\pgfpathcurveto{\pgfqpoint{3.158233in}{2.015186in}}{\pgfqpoint{3.147634in}{2.010796in}}{\pgfqpoint{3.139821in}{2.002982in}}%
\pgfpathcurveto{\pgfqpoint{3.132007in}{1.995169in}}{\pgfqpoint{3.127617in}{1.984570in}}{\pgfqpoint{3.127617in}{1.973519in}}%
\pgfpathcurveto{\pgfqpoint{3.127617in}{1.962469in}}{\pgfqpoint{3.132007in}{1.951870in}}{\pgfqpoint{3.139821in}{1.944057in}}%
\pgfpathcurveto{\pgfqpoint{3.147634in}{1.936243in}}{\pgfqpoint{3.158233in}{1.931853in}}{\pgfqpoint{3.169283in}{1.931853in}}%
\pgfpathclose%
\pgfusepath{stroke,fill}%
\end{pgfscope}%
\begin{pgfscope}%
\pgfpathrectangle{\pgfqpoint{0.511823in}{0.504323in}}{\pgfqpoint{3.218177in}{3.225677in}} %
\pgfusepath{clip}%
\pgfsetbuttcap%
\pgfsetroundjoin%
\definecolor{currentfill}{rgb}{0.501961,0.000000,0.000000}%
\pgfsetfillcolor{currentfill}%
\pgfsetfillopacity{0.400000}%
\pgfsetlinewidth{0.501875pt}%
\definecolor{currentstroke}{rgb}{0.501961,0.000000,0.000000}%
\pgfsetstrokecolor{currentstroke}%
\pgfsetstrokeopacity{0.400000}%
\pgfsetdash{}{0pt}%
\pgfpathmoveto{\pgfqpoint{3.039020in}{1.873125in}}%
\pgfpathcurveto{\pgfqpoint{3.050070in}{1.873125in}}{\pgfqpoint{3.060669in}{1.877515in}}{\pgfqpoint{3.068483in}{1.885329in}}%
\pgfpathcurveto{\pgfqpoint{3.076296in}{1.893142in}}{\pgfqpoint{3.080687in}{1.903741in}}{\pgfqpoint{3.080687in}{1.914792in}}%
\pgfpathcurveto{\pgfqpoint{3.080687in}{1.925842in}}{\pgfqpoint{3.076296in}{1.936441in}}{\pgfqpoint{3.068483in}{1.944254in}}%
\pgfpathcurveto{\pgfqpoint{3.060669in}{1.952068in}}{\pgfqpoint{3.050070in}{1.956458in}}{\pgfqpoint{3.039020in}{1.956458in}}%
\pgfpathcurveto{\pgfqpoint{3.027970in}{1.956458in}}{\pgfqpoint{3.017371in}{1.952068in}}{\pgfqpoint{3.009557in}{1.944254in}}%
\pgfpathcurveto{\pgfqpoint{3.001744in}{1.936441in}}{\pgfqpoint{2.997353in}{1.925842in}}{\pgfqpoint{2.997353in}{1.914792in}}%
\pgfpathcurveto{\pgfqpoint{2.997353in}{1.903741in}}{\pgfqpoint{3.001744in}{1.893142in}}{\pgfqpoint{3.009557in}{1.885329in}}%
\pgfpathcurveto{\pgfqpoint{3.017371in}{1.877515in}}{\pgfqpoint{3.027970in}{1.873125in}}{\pgfqpoint{3.039020in}{1.873125in}}%
\pgfpathclose%
\pgfusepath{stroke,fill}%
\end{pgfscope}%
\begin{pgfscope}%
\pgfpathrectangle{\pgfqpoint{0.511823in}{0.504323in}}{\pgfqpoint{3.218177in}{3.225677in}} %
\pgfusepath{clip}%
\pgfsetbuttcap%
\pgfsetroundjoin%
\definecolor{currentfill}{rgb}{0.501961,0.000000,0.000000}%
\pgfsetfillcolor{currentfill}%
\pgfsetfillopacity{0.400000}%
\pgfsetlinewidth{0.501875pt}%
\definecolor{currentstroke}{rgb}{0.501961,0.000000,0.000000}%
\pgfsetstrokecolor{currentstroke}%
\pgfsetstrokeopacity{0.400000}%
\pgfsetdash{}{0pt}%
\pgfpathmoveto{\pgfqpoint{3.100674in}{1.914780in}}%
\pgfpathcurveto{\pgfqpoint{3.111724in}{1.914780in}}{\pgfqpoint{3.122323in}{1.919171in}}{\pgfqpoint{3.130137in}{1.926984in}}%
\pgfpathcurveto{\pgfqpoint{3.137951in}{1.934798in}}{\pgfqpoint{3.142341in}{1.945397in}}{\pgfqpoint{3.142341in}{1.956447in}}%
\pgfpathcurveto{\pgfqpoint{3.142341in}{1.967497in}}{\pgfqpoint{3.137951in}{1.978096in}}{\pgfqpoint{3.130137in}{1.985910in}}%
\pgfpathcurveto{\pgfqpoint{3.122323in}{1.993723in}}{\pgfqpoint{3.111724in}{1.998114in}}{\pgfqpoint{3.100674in}{1.998114in}}%
\pgfpathcurveto{\pgfqpoint{3.089624in}{1.998114in}}{\pgfqpoint{3.079025in}{1.993723in}}{\pgfqpoint{3.071211in}{1.985910in}}%
\pgfpathcurveto{\pgfqpoint{3.063398in}{1.978096in}}{\pgfqpoint{3.059008in}{1.967497in}}{\pgfqpoint{3.059008in}{1.956447in}}%
\pgfpathcurveto{\pgfqpoint{3.059008in}{1.945397in}}{\pgfqpoint{3.063398in}{1.934798in}}{\pgfqpoint{3.071211in}{1.926984in}}%
\pgfpathcurveto{\pgfqpoint{3.079025in}{1.919171in}}{\pgfqpoint{3.089624in}{1.914780in}}{\pgfqpoint{3.100674in}{1.914780in}}%
\pgfpathclose%
\pgfusepath{stroke,fill}%
\end{pgfscope}%
\begin{pgfscope}%
\pgfpathrectangle{\pgfqpoint{0.511823in}{0.504323in}}{\pgfqpoint{3.218177in}{3.225677in}} %
\pgfusepath{clip}%
\pgfsetbuttcap%
\pgfsetroundjoin%
\definecolor{currentfill}{rgb}{0.501961,0.000000,0.000000}%
\pgfsetfillcolor{currentfill}%
\pgfsetfillopacity{0.400000}%
\pgfsetlinewidth{0.501875pt}%
\definecolor{currentstroke}{rgb}{0.501961,0.000000,0.000000}%
\pgfsetstrokecolor{currentstroke}%
\pgfsetstrokeopacity{0.400000}%
\pgfsetdash{}{0pt}%
\pgfpathmoveto{\pgfqpoint{2.714082in}{1.718478in}}%
\pgfpathcurveto{\pgfqpoint{2.725132in}{1.718478in}}{\pgfqpoint{2.735731in}{1.722868in}}{\pgfqpoint{2.743545in}{1.730681in}}%
\pgfpathcurveto{\pgfqpoint{2.751358in}{1.738495in}}{\pgfqpoint{2.755749in}{1.749094in}}{\pgfqpoint{2.755749in}{1.760144in}}%
\pgfpathcurveto{\pgfqpoint{2.755749in}{1.771194in}}{\pgfqpoint{2.751358in}{1.781793in}}{\pgfqpoint{2.743545in}{1.789607in}}%
\pgfpathcurveto{\pgfqpoint{2.735731in}{1.797421in}}{\pgfqpoint{2.725132in}{1.801811in}}{\pgfqpoint{2.714082in}{1.801811in}}%
\pgfpathcurveto{\pgfqpoint{2.703032in}{1.801811in}}{\pgfqpoint{2.692433in}{1.797421in}}{\pgfqpoint{2.684619in}{1.789607in}}%
\pgfpathcurveto{\pgfqpoint{2.676806in}{1.781793in}}{\pgfqpoint{2.672415in}{1.771194in}}{\pgfqpoint{2.672415in}{1.760144in}}%
\pgfpathcurveto{\pgfqpoint{2.672415in}{1.749094in}}{\pgfqpoint{2.676806in}{1.738495in}}{\pgfqpoint{2.684619in}{1.730681in}}%
\pgfpathcurveto{\pgfqpoint{2.692433in}{1.722868in}}{\pgfqpoint{2.703032in}{1.718478in}}{\pgfqpoint{2.714082in}{1.718478in}}%
\pgfpathclose%
\pgfusepath{stroke,fill}%
\end{pgfscope}%
\begin{pgfscope}%
\pgfpathrectangle{\pgfqpoint{0.511823in}{0.504323in}}{\pgfqpoint{3.218177in}{3.225677in}} %
\pgfusepath{clip}%
\pgfsetbuttcap%
\pgfsetroundjoin%
\definecolor{currentfill}{rgb}{0.501961,0.000000,0.000000}%
\pgfsetfillcolor{currentfill}%
\pgfsetfillopacity{0.400000}%
\pgfsetlinewidth{0.501875pt}%
\definecolor{currentstroke}{rgb}{0.501961,0.000000,0.000000}%
\pgfsetstrokecolor{currentstroke}%
\pgfsetstrokeopacity{0.400000}%
\pgfsetdash{}{0pt}%
\pgfpathmoveto{\pgfqpoint{3.048515in}{1.905594in}}%
\pgfpathcurveto{\pgfqpoint{3.059565in}{1.905594in}}{\pgfqpoint{3.070164in}{1.909985in}}{\pgfqpoint{3.077978in}{1.917798in}}%
\pgfpathcurveto{\pgfqpoint{3.085792in}{1.925612in}}{\pgfqpoint{3.090182in}{1.936211in}}{\pgfqpoint{3.090182in}{1.947261in}}%
\pgfpathcurveto{\pgfqpoint{3.090182in}{1.958311in}}{\pgfqpoint{3.085792in}{1.968910in}}{\pgfqpoint{3.077978in}{1.976724in}}%
\pgfpathcurveto{\pgfqpoint{3.070164in}{1.984538in}}{\pgfqpoint{3.059565in}{1.988928in}}{\pgfqpoint{3.048515in}{1.988928in}}%
\pgfpathcurveto{\pgfqpoint{3.037465in}{1.988928in}}{\pgfqpoint{3.026866in}{1.984538in}}{\pgfqpoint{3.019052in}{1.976724in}}%
\pgfpathcurveto{\pgfqpoint{3.011239in}{1.968910in}}{\pgfqpoint{3.006849in}{1.958311in}}{\pgfqpoint{3.006849in}{1.947261in}}%
\pgfpathcurveto{\pgfqpoint{3.006849in}{1.936211in}}{\pgfqpoint{3.011239in}{1.925612in}}{\pgfqpoint{3.019052in}{1.917798in}}%
\pgfpathcurveto{\pgfqpoint{3.026866in}{1.909985in}}{\pgfqpoint{3.037465in}{1.905594in}}{\pgfqpoint{3.048515in}{1.905594in}}%
\pgfpathclose%
\pgfusepath{stroke,fill}%
\end{pgfscope}%
\begin{pgfscope}%
\pgfpathrectangle{\pgfqpoint{0.511823in}{0.504323in}}{\pgfqpoint{3.218177in}{3.225677in}} %
\pgfusepath{clip}%
\pgfsetbuttcap%
\pgfsetroundjoin%
\definecolor{currentfill}{rgb}{0.501961,0.000000,0.000000}%
\pgfsetfillcolor{currentfill}%
\pgfsetfillopacity{0.400000}%
\pgfsetlinewidth{0.501875pt}%
\definecolor{currentstroke}{rgb}{0.501961,0.000000,0.000000}%
\pgfsetstrokecolor{currentstroke}%
\pgfsetstrokeopacity{0.400000}%
\pgfsetdash{}{0pt}%
\pgfpathmoveto{\pgfqpoint{3.108261in}{1.947076in}}%
\pgfpathcurveto{\pgfqpoint{3.119311in}{1.947076in}}{\pgfqpoint{3.129910in}{1.951466in}}{\pgfqpoint{3.137724in}{1.959280in}}%
\pgfpathcurveto{\pgfqpoint{3.145538in}{1.967093in}}{\pgfqpoint{3.149928in}{1.977693in}}{\pgfqpoint{3.149928in}{1.988743in}}%
\pgfpathcurveto{\pgfqpoint{3.149928in}{1.999793in}}{\pgfqpoint{3.145538in}{2.010392in}}{\pgfqpoint{3.137724in}{2.018205in}}%
\pgfpathcurveto{\pgfqpoint{3.129910in}{2.026019in}}{\pgfqpoint{3.119311in}{2.030409in}}{\pgfqpoint{3.108261in}{2.030409in}}%
\pgfpathcurveto{\pgfqpoint{3.097211in}{2.030409in}}{\pgfqpoint{3.086612in}{2.026019in}}{\pgfqpoint{3.078799in}{2.018205in}}%
\pgfpathcurveto{\pgfqpoint{3.070985in}{2.010392in}}{\pgfqpoint{3.066595in}{1.999793in}}{\pgfqpoint{3.066595in}{1.988743in}}%
\pgfpathcurveto{\pgfqpoint{3.066595in}{1.977693in}}{\pgfqpoint{3.070985in}{1.967093in}}{\pgfqpoint{3.078799in}{1.959280in}}%
\pgfpathcurveto{\pgfqpoint{3.086612in}{1.951466in}}{\pgfqpoint{3.097211in}{1.947076in}}{\pgfqpoint{3.108261in}{1.947076in}}%
\pgfpathclose%
\pgfusepath{stroke,fill}%
\end{pgfscope}%
\begin{pgfscope}%
\pgfpathrectangle{\pgfqpoint{0.511823in}{0.504323in}}{\pgfqpoint{3.218177in}{3.225677in}} %
\pgfusepath{clip}%
\pgfsetbuttcap%
\pgfsetroundjoin%
\definecolor{currentfill}{rgb}{0.501961,0.000000,0.000000}%
\pgfsetfillcolor{currentfill}%
\pgfsetfillopacity{0.400000}%
\pgfsetlinewidth{0.501875pt}%
\definecolor{currentstroke}{rgb}{0.501961,0.000000,0.000000}%
\pgfsetstrokecolor{currentstroke}%
\pgfsetstrokeopacity{0.400000}%
\pgfsetdash{}{0pt}%
\pgfpathmoveto{\pgfqpoint{3.076906in}{1.939525in}}%
\pgfpathcurveto{\pgfqpoint{3.087956in}{1.939525in}}{\pgfqpoint{3.098555in}{1.943916in}}{\pgfqpoint{3.106369in}{1.951729in}}%
\pgfpathcurveto{\pgfqpoint{3.114182in}{1.959543in}}{\pgfqpoint{3.118573in}{1.970142in}}{\pgfqpoint{3.118573in}{1.981192in}}%
\pgfpathcurveto{\pgfqpoint{3.118573in}{1.992242in}}{\pgfqpoint{3.114182in}{2.002841in}}{\pgfqpoint{3.106369in}{2.010655in}}%
\pgfpathcurveto{\pgfqpoint{3.098555in}{2.018469in}}{\pgfqpoint{3.087956in}{2.022859in}}{\pgfqpoint{3.076906in}{2.022859in}}%
\pgfpathcurveto{\pgfqpoint{3.065856in}{2.022859in}}{\pgfqpoint{3.055257in}{2.018469in}}{\pgfqpoint{3.047443in}{2.010655in}}%
\pgfpathcurveto{\pgfqpoint{3.039629in}{2.002841in}}{\pgfqpoint{3.035239in}{1.992242in}}{\pgfqpoint{3.035239in}{1.981192in}}%
\pgfpathcurveto{\pgfqpoint{3.035239in}{1.970142in}}{\pgfqpoint{3.039629in}{1.959543in}}{\pgfqpoint{3.047443in}{1.951729in}}%
\pgfpathcurveto{\pgfqpoint{3.055257in}{1.943916in}}{\pgfqpoint{3.065856in}{1.939525in}}{\pgfqpoint{3.076906in}{1.939525in}}%
\pgfpathclose%
\pgfusepath{stroke,fill}%
\end{pgfscope}%
\begin{pgfscope}%
\pgfpathrectangle{\pgfqpoint{0.511823in}{0.504323in}}{\pgfqpoint{3.218177in}{3.225677in}} %
\pgfusepath{clip}%
\pgfsetbuttcap%
\pgfsetroundjoin%
\definecolor{currentfill}{rgb}{0.501961,0.000000,0.000000}%
\pgfsetfillcolor{currentfill}%
\pgfsetfillopacity{0.400000}%
\pgfsetlinewidth{0.501875pt}%
\definecolor{currentstroke}{rgb}{0.501961,0.000000,0.000000}%
\pgfsetstrokecolor{currentstroke}%
\pgfsetstrokeopacity{0.400000}%
\pgfsetdash{}{0pt}%
\pgfpathmoveto{\pgfqpoint{3.000135in}{1.906891in}}%
\pgfpathcurveto{\pgfqpoint{3.011186in}{1.906891in}}{\pgfqpoint{3.021785in}{1.911281in}}{\pgfqpoint{3.029598in}{1.919095in}}%
\pgfpathcurveto{\pgfqpoint{3.037412in}{1.926908in}}{\pgfqpoint{3.041802in}{1.937507in}}{\pgfqpoint{3.041802in}{1.948558in}}%
\pgfpathcurveto{\pgfqpoint{3.041802in}{1.959608in}}{\pgfqpoint{3.037412in}{1.970207in}}{\pgfqpoint{3.029598in}{1.978020in}}%
\pgfpathcurveto{\pgfqpoint{3.021785in}{1.985834in}}{\pgfqpoint{3.011186in}{1.990224in}}{\pgfqpoint{3.000135in}{1.990224in}}%
\pgfpathcurveto{\pgfqpoint{2.989085in}{1.990224in}}{\pgfqpoint{2.978486in}{1.985834in}}{\pgfqpoint{2.970673in}{1.978020in}}%
\pgfpathcurveto{\pgfqpoint{2.962859in}{1.970207in}}{\pgfqpoint{2.958469in}{1.959608in}}{\pgfqpoint{2.958469in}{1.948558in}}%
\pgfpathcurveto{\pgfqpoint{2.958469in}{1.937507in}}{\pgfqpoint{2.962859in}{1.926908in}}{\pgfqpoint{2.970673in}{1.919095in}}%
\pgfpathcurveto{\pgfqpoint{2.978486in}{1.911281in}}{\pgfqpoint{2.989085in}{1.906891in}}{\pgfqpoint{3.000135in}{1.906891in}}%
\pgfpathclose%
\pgfusepath{stroke,fill}%
\end{pgfscope}%
\begin{pgfscope}%
\pgfpathrectangle{\pgfqpoint{0.511823in}{0.504323in}}{\pgfqpoint{3.218177in}{3.225677in}} %
\pgfusepath{clip}%
\pgfsetbuttcap%
\pgfsetroundjoin%
\definecolor{currentfill}{rgb}{0.501961,0.000000,0.000000}%
\pgfsetfillcolor{currentfill}%
\pgfsetfillopacity{0.400000}%
\pgfsetlinewidth{0.501875pt}%
\definecolor{currentstroke}{rgb}{0.501961,0.000000,0.000000}%
\pgfsetstrokecolor{currentstroke}%
\pgfsetstrokeopacity{0.400000}%
\pgfsetdash{}{0pt}%
\pgfpathmoveto{\pgfqpoint{2.928160in}{1.876330in}}%
\pgfpathcurveto{\pgfqpoint{2.939211in}{1.876330in}}{\pgfqpoint{2.949810in}{1.880720in}}{\pgfqpoint{2.957623in}{1.888534in}}%
\pgfpathcurveto{\pgfqpoint{2.965437in}{1.896347in}}{\pgfqpoint{2.969827in}{1.906946in}}{\pgfqpoint{2.969827in}{1.917997in}}%
\pgfpathcurveto{\pgfqpoint{2.969827in}{1.929047in}}{\pgfqpoint{2.965437in}{1.939646in}}{\pgfqpoint{2.957623in}{1.947459in}}%
\pgfpathcurveto{\pgfqpoint{2.949810in}{1.955273in}}{\pgfqpoint{2.939211in}{1.959663in}}{\pgfqpoint{2.928160in}{1.959663in}}%
\pgfpathcurveto{\pgfqpoint{2.917110in}{1.959663in}}{\pgfqpoint{2.906511in}{1.955273in}}{\pgfqpoint{2.898698in}{1.947459in}}%
\pgfpathcurveto{\pgfqpoint{2.890884in}{1.939646in}}{\pgfqpoint{2.886494in}{1.929047in}}{\pgfqpoint{2.886494in}{1.917997in}}%
\pgfpathcurveto{\pgfqpoint{2.886494in}{1.906946in}}{\pgfqpoint{2.890884in}{1.896347in}}{\pgfqpoint{2.898698in}{1.888534in}}%
\pgfpathcurveto{\pgfqpoint{2.906511in}{1.880720in}}{\pgfqpoint{2.917110in}{1.876330in}}{\pgfqpoint{2.928160in}{1.876330in}}%
\pgfpathclose%
\pgfusepath{stroke,fill}%
\end{pgfscope}%
\begin{pgfscope}%
\pgfpathrectangle{\pgfqpoint{0.511823in}{0.504323in}}{\pgfqpoint{3.218177in}{3.225677in}} %
\pgfusepath{clip}%
\pgfsetbuttcap%
\pgfsetroundjoin%
\definecolor{currentfill}{rgb}{0.501961,0.000000,0.000000}%
\pgfsetfillcolor{currentfill}%
\pgfsetfillopacity{0.400000}%
\pgfsetlinewidth{0.501875pt}%
\definecolor{currentstroke}{rgb}{0.501961,0.000000,0.000000}%
\pgfsetstrokecolor{currentstroke}%
\pgfsetstrokeopacity{0.400000}%
\pgfsetdash{}{0pt}%
\pgfpathmoveto{\pgfqpoint{2.922957in}{1.882329in}}%
\pgfpathcurveto{\pgfqpoint{2.934007in}{1.882329in}}{\pgfqpoint{2.944606in}{1.886719in}}{\pgfqpoint{2.952420in}{1.894533in}}%
\pgfpathcurveto{\pgfqpoint{2.960233in}{1.902347in}}{\pgfqpoint{2.964624in}{1.912946in}}{\pgfqpoint{2.964624in}{1.923996in}}%
\pgfpathcurveto{\pgfqpoint{2.964624in}{1.935046in}}{\pgfqpoint{2.960233in}{1.945645in}}{\pgfqpoint{2.952420in}{1.953459in}}%
\pgfpathcurveto{\pgfqpoint{2.944606in}{1.961272in}}{\pgfqpoint{2.934007in}{1.965662in}}{\pgfqpoint{2.922957in}{1.965662in}}%
\pgfpathcurveto{\pgfqpoint{2.911907in}{1.965662in}}{\pgfqpoint{2.901308in}{1.961272in}}{\pgfqpoint{2.893494in}{1.953459in}}%
\pgfpathcurveto{\pgfqpoint{2.885680in}{1.945645in}}{\pgfqpoint{2.881290in}{1.935046in}}{\pgfqpoint{2.881290in}{1.923996in}}%
\pgfpathcurveto{\pgfqpoint{2.881290in}{1.912946in}}{\pgfqpoint{2.885680in}{1.902347in}}{\pgfqpoint{2.893494in}{1.894533in}}%
\pgfpathcurveto{\pgfqpoint{2.901308in}{1.886719in}}{\pgfqpoint{2.911907in}{1.882329in}}{\pgfqpoint{2.922957in}{1.882329in}}%
\pgfpathclose%
\pgfusepath{stroke,fill}%
\end{pgfscope}%
\begin{pgfscope}%
\pgfpathrectangle{\pgfqpoint{0.511823in}{0.504323in}}{\pgfqpoint{3.218177in}{3.225677in}} %
\pgfusepath{clip}%
\pgfsetbuttcap%
\pgfsetroundjoin%
\definecolor{currentfill}{rgb}{0.501961,0.000000,0.000000}%
\pgfsetfillcolor{currentfill}%
\pgfsetfillopacity{0.400000}%
\pgfsetlinewidth{0.501875pt}%
\definecolor{currentstroke}{rgb}{0.501961,0.000000,0.000000}%
\pgfsetstrokecolor{currentstroke}%
\pgfsetstrokeopacity{0.400000}%
\pgfsetdash{}{0pt}%
\pgfpathmoveto{\pgfqpoint{2.962843in}{1.913551in}}%
\pgfpathcurveto{\pgfqpoint{2.973893in}{1.913551in}}{\pgfqpoint{2.984492in}{1.917941in}}{\pgfqpoint{2.992306in}{1.925755in}}%
\pgfpathcurveto{\pgfqpoint{3.000120in}{1.933569in}}{\pgfqpoint{3.004510in}{1.944168in}}{\pgfqpoint{3.004510in}{1.955218in}}%
\pgfpathcurveto{\pgfqpoint{3.004510in}{1.966268in}}{\pgfqpoint{3.000120in}{1.976867in}}{\pgfqpoint{2.992306in}{1.984681in}}%
\pgfpathcurveto{\pgfqpoint{2.984492in}{1.992494in}}{\pgfqpoint{2.973893in}{1.996884in}}{\pgfqpoint{2.962843in}{1.996884in}}%
\pgfpathcurveto{\pgfqpoint{2.951793in}{1.996884in}}{\pgfqpoint{2.941194in}{1.992494in}}{\pgfqpoint{2.933381in}{1.984681in}}%
\pgfpathcurveto{\pgfqpoint{2.925567in}{1.976867in}}{\pgfqpoint{2.921177in}{1.966268in}}{\pgfqpoint{2.921177in}{1.955218in}}%
\pgfpathcurveto{\pgfqpoint{2.921177in}{1.944168in}}{\pgfqpoint{2.925567in}{1.933569in}}{\pgfqpoint{2.933381in}{1.925755in}}%
\pgfpathcurveto{\pgfqpoint{2.941194in}{1.917941in}}{\pgfqpoint{2.951793in}{1.913551in}}{\pgfqpoint{2.962843in}{1.913551in}}%
\pgfpathclose%
\pgfusepath{stroke,fill}%
\end{pgfscope}%
\begin{pgfscope}%
\pgfpathrectangle{\pgfqpoint{0.511823in}{0.504323in}}{\pgfqpoint{3.218177in}{3.225677in}} %
\pgfusepath{clip}%
\pgfsetbuttcap%
\pgfsetroundjoin%
\definecolor{currentfill}{rgb}{0.501961,0.000000,0.000000}%
\pgfsetfillcolor{currentfill}%
\pgfsetfillopacity{0.400000}%
\pgfsetlinewidth{0.501875pt}%
\definecolor{currentstroke}{rgb}{0.501961,0.000000,0.000000}%
\pgfsetstrokecolor{currentstroke}%
\pgfsetstrokeopacity{0.400000}%
\pgfsetdash{}{0pt}%
\pgfpathmoveto{\pgfqpoint{3.004556in}{1.946155in}}%
\pgfpathcurveto{\pgfqpoint{3.015606in}{1.946155in}}{\pgfqpoint{3.026205in}{1.950545in}}{\pgfqpoint{3.034018in}{1.958358in}}%
\pgfpathcurveto{\pgfqpoint{3.041832in}{1.966172in}}{\pgfqpoint{3.046222in}{1.976771in}}{\pgfqpoint{3.046222in}{1.987821in}}%
\pgfpathcurveto{\pgfqpoint{3.046222in}{1.998871in}}{\pgfqpoint{3.041832in}{2.009470in}}{\pgfqpoint{3.034018in}{2.017284in}}%
\pgfpathcurveto{\pgfqpoint{3.026205in}{2.025098in}}{\pgfqpoint{3.015606in}{2.029488in}}{\pgfqpoint{3.004556in}{2.029488in}}%
\pgfpathcurveto{\pgfqpoint{2.993505in}{2.029488in}}{\pgfqpoint{2.982906in}{2.025098in}}{\pgfqpoint{2.975093in}{2.017284in}}%
\pgfpathcurveto{\pgfqpoint{2.967279in}{2.009470in}}{\pgfqpoint{2.962889in}{1.998871in}}{\pgfqpoint{2.962889in}{1.987821in}}%
\pgfpathcurveto{\pgfqpoint{2.962889in}{1.976771in}}{\pgfqpoint{2.967279in}{1.966172in}}{\pgfqpoint{2.975093in}{1.958358in}}%
\pgfpathcurveto{\pgfqpoint{2.982906in}{1.950545in}}{\pgfqpoint{2.993505in}{1.946155in}}{\pgfqpoint{3.004556in}{1.946155in}}%
\pgfpathclose%
\pgfusepath{stroke,fill}%
\end{pgfscope}%
\begin{pgfscope}%
\pgfpathrectangle{\pgfqpoint{0.511823in}{0.504323in}}{\pgfqpoint{3.218177in}{3.225677in}} %
\pgfusepath{clip}%
\pgfsetbuttcap%
\pgfsetroundjoin%
\definecolor{currentfill}{rgb}{0.501961,0.000000,0.000000}%
\pgfsetfillcolor{currentfill}%
\pgfsetfillopacity{0.400000}%
\pgfsetlinewidth{0.501875pt}%
\definecolor{currentstroke}{rgb}{0.501961,0.000000,0.000000}%
\pgfsetstrokecolor{currentstroke}%
\pgfsetstrokeopacity{0.400000}%
\pgfsetdash{}{0pt}%
\pgfpathmoveto{\pgfqpoint{3.016289in}{1.962112in}}%
\pgfpathcurveto{\pgfqpoint{3.027339in}{1.962112in}}{\pgfqpoint{3.037938in}{1.966502in}}{\pgfqpoint{3.045751in}{1.974316in}}%
\pgfpathcurveto{\pgfqpoint{3.053565in}{1.982130in}}{\pgfqpoint{3.057955in}{1.992729in}}{\pgfqpoint{3.057955in}{2.003779in}}%
\pgfpathcurveto{\pgfqpoint{3.057955in}{2.014829in}}{\pgfqpoint{3.053565in}{2.025428in}}{\pgfqpoint{3.045751in}{2.033242in}}%
\pgfpathcurveto{\pgfqpoint{3.037938in}{2.041055in}}{\pgfqpoint{3.027339in}{2.045445in}}{\pgfqpoint{3.016289in}{2.045445in}}%
\pgfpathcurveto{\pgfqpoint{3.005238in}{2.045445in}}{\pgfqpoint{2.994639in}{2.041055in}}{\pgfqpoint{2.986826in}{2.033242in}}%
\pgfpathcurveto{\pgfqpoint{2.979012in}{2.025428in}}{\pgfqpoint{2.974622in}{2.014829in}}{\pgfqpoint{2.974622in}{2.003779in}}%
\pgfpathcurveto{\pgfqpoint{2.974622in}{1.992729in}}{\pgfqpoint{2.979012in}{1.982130in}}{\pgfqpoint{2.986826in}{1.974316in}}%
\pgfpathcurveto{\pgfqpoint{2.994639in}{1.966502in}}{\pgfqpoint{3.005238in}{1.962112in}}{\pgfqpoint{3.016289in}{1.962112in}}%
\pgfpathclose%
\pgfusepath{stroke,fill}%
\end{pgfscope}%
\begin{pgfscope}%
\pgfpathrectangle{\pgfqpoint{0.511823in}{0.504323in}}{\pgfqpoint{3.218177in}{3.225677in}} %
\pgfusepath{clip}%
\pgfsetbuttcap%
\pgfsetroundjoin%
\definecolor{currentfill}{rgb}{0.501961,0.000000,0.000000}%
\pgfsetfillcolor{currentfill}%
\pgfsetfillopacity{0.400000}%
\pgfsetlinewidth{0.501875pt}%
\definecolor{currentstroke}{rgb}{0.501961,0.000000,0.000000}%
\pgfsetstrokecolor{currentstroke}%
\pgfsetstrokeopacity{0.400000}%
\pgfsetdash{}{0pt}%
\pgfpathmoveto{\pgfqpoint{2.936499in}{1.925886in}}%
\pgfpathcurveto{\pgfqpoint{2.947550in}{1.925886in}}{\pgfqpoint{2.958149in}{1.930277in}}{\pgfqpoint{2.965962in}{1.938090in}}%
\pgfpathcurveto{\pgfqpoint{2.973776in}{1.945904in}}{\pgfqpoint{2.978166in}{1.956503in}}{\pgfqpoint{2.978166in}{1.967553in}}%
\pgfpathcurveto{\pgfqpoint{2.978166in}{1.978603in}}{\pgfqpoint{2.973776in}{1.989202in}}{\pgfqpoint{2.965962in}{1.997016in}}%
\pgfpathcurveto{\pgfqpoint{2.958149in}{2.004829in}}{\pgfqpoint{2.947550in}{2.009220in}}{\pgfqpoint{2.936499in}{2.009220in}}%
\pgfpathcurveto{\pgfqpoint{2.925449in}{2.009220in}}{\pgfqpoint{2.914850in}{2.004829in}}{\pgfqpoint{2.907037in}{1.997016in}}%
\pgfpathcurveto{\pgfqpoint{2.899223in}{1.989202in}}{\pgfqpoint{2.894833in}{1.978603in}}{\pgfqpoint{2.894833in}{1.967553in}}%
\pgfpathcurveto{\pgfqpoint{2.894833in}{1.956503in}}{\pgfqpoint{2.899223in}{1.945904in}}{\pgfqpoint{2.907037in}{1.938090in}}%
\pgfpathcurveto{\pgfqpoint{2.914850in}{1.930277in}}{\pgfqpoint{2.925449in}{1.925886in}}{\pgfqpoint{2.936499in}{1.925886in}}%
\pgfpathclose%
\pgfusepath{stroke,fill}%
\end{pgfscope}%
\begin{pgfscope}%
\pgfpathrectangle{\pgfqpoint{0.511823in}{0.504323in}}{\pgfqpoint{3.218177in}{3.225677in}} %
\pgfusepath{clip}%
\pgfsetbuttcap%
\pgfsetroundjoin%
\definecolor{currentfill}{rgb}{0.501961,0.000000,0.000000}%
\pgfsetfillcolor{currentfill}%
\pgfsetfillopacity{0.400000}%
\pgfsetlinewidth{0.501875pt}%
\definecolor{currentstroke}{rgb}{0.501961,0.000000,0.000000}%
\pgfsetstrokecolor{currentstroke}%
\pgfsetstrokeopacity{0.400000}%
\pgfsetdash{}{0pt}%
\pgfpathmoveto{\pgfqpoint{2.801051in}{1.857012in}}%
\pgfpathcurveto{\pgfqpoint{2.812101in}{1.857012in}}{\pgfqpoint{2.822700in}{1.861402in}}{\pgfqpoint{2.830514in}{1.869216in}}%
\pgfpathcurveto{\pgfqpoint{2.838328in}{1.877030in}}{\pgfqpoint{2.842718in}{1.887629in}}{\pgfqpoint{2.842718in}{1.898679in}}%
\pgfpathcurveto{\pgfqpoint{2.842718in}{1.909729in}}{\pgfqpoint{2.838328in}{1.920328in}}{\pgfqpoint{2.830514in}{1.928142in}}%
\pgfpathcurveto{\pgfqpoint{2.822700in}{1.935955in}}{\pgfqpoint{2.812101in}{1.940346in}}{\pgfqpoint{2.801051in}{1.940346in}}%
\pgfpathcurveto{\pgfqpoint{2.790001in}{1.940346in}}{\pgfqpoint{2.779402in}{1.935955in}}{\pgfqpoint{2.771588in}{1.928142in}}%
\pgfpathcurveto{\pgfqpoint{2.763775in}{1.920328in}}{\pgfqpoint{2.759385in}{1.909729in}}{\pgfqpoint{2.759385in}{1.898679in}}%
\pgfpathcurveto{\pgfqpoint{2.759385in}{1.887629in}}{\pgfqpoint{2.763775in}{1.877030in}}{\pgfqpoint{2.771588in}{1.869216in}}%
\pgfpathcurveto{\pgfqpoint{2.779402in}{1.861402in}}{\pgfqpoint{2.790001in}{1.857012in}}{\pgfqpoint{2.801051in}{1.857012in}}%
\pgfpathclose%
\pgfusepath{stroke,fill}%
\end{pgfscope}%
\begin{pgfscope}%
\pgfpathrectangle{\pgfqpoint{0.511823in}{0.504323in}}{\pgfqpoint{3.218177in}{3.225677in}} %
\pgfusepath{clip}%
\pgfsetbuttcap%
\pgfsetroundjoin%
\definecolor{currentfill}{rgb}{0.501961,0.000000,0.000000}%
\pgfsetfillcolor{currentfill}%
\pgfsetfillopacity{0.400000}%
\pgfsetlinewidth{0.501875pt}%
\definecolor{currentstroke}{rgb}{0.501961,0.000000,0.000000}%
\pgfsetstrokecolor{currentstroke}%
\pgfsetstrokeopacity{0.400000}%
\pgfsetdash{}{0pt}%
\pgfpathmoveto{\pgfqpoint{3.143683in}{2.064215in}}%
\pgfpathcurveto{\pgfqpoint{3.154733in}{2.064215in}}{\pgfqpoint{3.165332in}{2.068605in}}{\pgfqpoint{3.173145in}{2.076419in}}%
\pgfpathcurveto{\pgfqpoint{3.180959in}{2.084233in}}{\pgfqpoint{3.185349in}{2.094832in}}{\pgfqpoint{3.185349in}{2.105882in}}%
\pgfpathcurveto{\pgfqpoint{3.185349in}{2.116932in}}{\pgfqpoint{3.180959in}{2.127531in}}{\pgfqpoint{3.173145in}{2.135345in}}%
\pgfpathcurveto{\pgfqpoint{3.165332in}{2.143158in}}{\pgfqpoint{3.154733in}{2.147549in}}{\pgfqpoint{3.143683in}{2.147549in}}%
\pgfpathcurveto{\pgfqpoint{3.132633in}{2.147549in}}{\pgfqpoint{3.122034in}{2.143158in}}{\pgfqpoint{3.114220in}{2.135345in}}%
\pgfpathcurveto{\pgfqpoint{3.106406in}{2.127531in}}{\pgfqpoint{3.102016in}{2.116932in}}{\pgfqpoint{3.102016in}{2.105882in}}%
\pgfpathcurveto{\pgfqpoint{3.102016in}{2.094832in}}{\pgfqpoint{3.106406in}{2.084233in}}{\pgfqpoint{3.114220in}{2.076419in}}%
\pgfpathcurveto{\pgfqpoint{3.122034in}{2.068605in}}{\pgfqpoint{3.132633in}{2.064215in}}{\pgfqpoint{3.143683in}{2.064215in}}%
\pgfpathclose%
\pgfusepath{stroke,fill}%
\end{pgfscope}%
\begin{pgfscope}%
\pgfpathrectangle{\pgfqpoint{0.511823in}{0.504323in}}{\pgfqpoint{3.218177in}{3.225677in}} %
\pgfusepath{clip}%
\pgfsetbuttcap%
\pgfsetroundjoin%
\definecolor{currentfill}{rgb}{0.501961,0.000000,0.000000}%
\pgfsetfillcolor{currentfill}%
\pgfsetfillopacity{0.400000}%
\pgfsetlinewidth{0.501875pt}%
\definecolor{currentstroke}{rgb}{0.501961,0.000000,0.000000}%
\pgfsetstrokecolor{currentstroke}%
\pgfsetstrokeopacity{0.400000}%
\pgfsetdash{}{0pt}%
\pgfpathmoveto{\pgfqpoint{2.836722in}{1.895031in}}%
\pgfpathcurveto{\pgfqpoint{2.847772in}{1.895031in}}{\pgfqpoint{2.858371in}{1.899422in}}{\pgfqpoint{2.866185in}{1.907235in}}%
\pgfpathcurveto{\pgfqpoint{2.873998in}{1.915049in}}{\pgfqpoint{2.878388in}{1.925648in}}{\pgfqpoint{2.878388in}{1.936698in}}%
\pgfpathcurveto{\pgfqpoint{2.878388in}{1.947748in}}{\pgfqpoint{2.873998in}{1.958347in}}{\pgfqpoint{2.866185in}{1.966161in}}%
\pgfpathcurveto{\pgfqpoint{2.858371in}{1.973974in}}{\pgfqpoint{2.847772in}{1.978365in}}{\pgfqpoint{2.836722in}{1.978365in}}%
\pgfpathcurveto{\pgfqpoint{2.825672in}{1.978365in}}{\pgfqpoint{2.815073in}{1.973974in}}{\pgfqpoint{2.807259in}{1.966161in}}%
\pgfpathcurveto{\pgfqpoint{2.799445in}{1.958347in}}{\pgfqpoint{2.795055in}{1.947748in}}{\pgfqpoint{2.795055in}{1.936698in}}%
\pgfpathcurveto{\pgfqpoint{2.795055in}{1.925648in}}{\pgfqpoint{2.799445in}{1.915049in}}{\pgfqpoint{2.807259in}{1.907235in}}%
\pgfpathcurveto{\pgfqpoint{2.815073in}{1.899422in}}{\pgfqpoint{2.825672in}{1.895031in}}{\pgfqpoint{2.836722in}{1.895031in}}%
\pgfpathclose%
\pgfusepath{stroke,fill}%
\end{pgfscope}%
\begin{pgfscope}%
\pgfpathrectangle{\pgfqpoint{0.511823in}{0.504323in}}{\pgfqpoint{3.218177in}{3.225677in}} %
\pgfusepath{clip}%
\pgfsetbuttcap%
\pgfsetroundjoin%
\definecolor{currentfill}{rgb}{0.501961,0.000000,0.000000}%
\pgfsetfillcolor{currentfill}%
\pgfsetfillopacity{0.400000}%
\pgfsetlinewidth{0.501875pt}%
\definecolor{currentstroke}{rgb}{0.501961,0.000000,0.000000}%
\pgfsetstrokecolor{currentstroke}%
\pgfsetstrokeopacity{0.400000}%
\pgfsetdash{}{0pt}%
\pgfpathmoveto{\pgfqpoint{2.629121in}{1.781786in}}%
\pgfpathcurveto{\pgfqpoint{2.640171in}{1.781786in}}{\pgfqpoint{2.650770in}{1.786176in}}{\pgfqpoint{2.658584in}{1.793990in}}%
\pgfpathcurveto{\pgfqpoint{2.666397in}{1.801803in}}{\pgfqpoint{2.670787in}{1.812402in}}{\pgfqpoint{2.670787in}{1.823452in}}%
\pgfpathcurveto{\pgfqpoint{2.670787in}{1.834502in}}{\pgfqpoint{2.666397in}{1.845101in}}{\pgfqpoint{2.658584in}{1.852915in}}%
\pgfpathcurveto{\pgfqpoint{2.650770in}{1.860729in}}{\pgfqpoint{2.640171in}{1.865119in}}{\pgfqpoint{2.629121in}{1.865119in}}%
\pgfpathcurveto{\pgfqpoint{2.618071in}{1.865119in}}{\pgfqpoint{2.607472in}{1.860729in}}{\pgfqpoint{2.599658in}{1.852915in}}%
\pgfpathcurveto{\pgfqpoint{2.591844in}{1.845101in}}{\pgfqpoint{2.587454in}{1.834502in}}{\pgfqpoint{2.587454in}{1.823452in}}%
\pgfpathcurveto{\pgfqpoint{2.587454in}{1.812402in}}{\pgfqpoint{2.591844in}{1.801803in}}{\pgfqpoint{2.599658in}{1.793990in}}%
\pgfpathcurveto{\pgfqpoint{2.607472in}{1.786176in}}{\pgfqpoint{2.618071in}{1.781786in}}{\pgfqpoint{2.629121in}{1.781786in}}%
\pgfpathclose%
\pgfusepath{stroke,fill}%
\end{pgfscope}%
\begin{pgfscope}%
\pgfpathrectangle{\pgfqpoint{0.511823in}{0.504323in}}{\pgfqpoint{3.218177in}{3.225677in}} %
\pgfusepath{clip}%
\pgfsetbuttcap%
\pgfsetroundjoin%
\definecolor{currentfill}{rgb}{0.501961,0.000000,0.000000}%
\pgfsetfillcolor{currentfill}%
\pgfsetfillopacity{0.400000}%
\pgfsetlinewidth{0.501875pt}%
\definecolor{currentstroke}{rgb}{0.501961,0.000000,0.000000}%
\pgfsetstrokecolor{currentstroke}%
\pgfsetstrokeopacity{0.400000}%
\pgfsetdash{}{0pt}%
\pgfpathmoveto{\pgfqpoint{3.084664in}{2.059429in}}%
\pgfpathcurveto{\pgfqpoint{3.095714in}{2.059429in}}{\pgfqpoint{3.106313in}{2.063819in}}{\pgfqpoint{3.114127in}{2.071633in}}%
\pgfpathcurveto{\pgfqpoint{3.121940in}{2.079446in}}{\pgfqpoint{3.126330in}{2.090045in}}{\pgfqpoint{3.126330in}{2.101096in}}%
\pgfpathcurveto{\pgfqpoint{3.126330in}{2.112146in}}{\pgfqpoint{3.121940in}{2.122745in}}{\pgfqpoint{3.114127in}{2.130558in}}%
\pgfpathcurveto{\pgfqpoint{3.106313in}{2.138372in}}{\pgfqpoint{3.095714in}{2.142762in}}{\pgfqpoint{3.084664in}{2.142762in}}%
\pgfpathcurveto{\pgfqpoint{3.073614in}{2.142762in}}{\pgfqpoint{3.063015in}{2.138372in}}{\pgfqpoint{3.055201in}{2.130558in}}%
\pgfpathcurveto{\pgfqpoint{3.047387in}{2.122745in}}{\pgfqpoint{3.042997in}{2.112146in}}{\pgfqpoint{3.042997in}{2.101096in}}%
\pgfpathcurveto{\pgfqpoint{3.042997in}{2.090045in}}{\pgfqpoint{3.047387in}{2.079446in}}{\pgfqpoint{3.055201in}{2.071633in}}%
\pgfpathcurveto{\pgfqpoint{3.063015in}{2.063819in}}{\pgfqpoint{3.073614in}{2.059429in}}{\pgfqpoint{3.084664in}{2.059429in}}%
\pgfpathclose%
\pgfusepath{stroke,fill}%
\end{pgfscope}%
\begin{pgfscope}%
\pgfpathrectangle{\pgfqpoint{0.511823in}{0.504323in}}{\pgfqpoint{3.218177in}{3.225677in}} %
\pgfusepath{clip}%
\pgfsetbuttcap%
\pgfsetroundjoin%
\definecolor{currentfill}{rgb}{0.501961,0.000000,0.000000}%
\pgfsetfillcolor{currentfill}%
\pgfsetfillopacity{0.400000}%
\pgfsetlinewidth{0.501875pt}%
\definecolor{currentstroke}{rgb}{0.501961,0.000000,0.000000}%
\pgfsetstrokecolor{currentstroke}%
\pgfsetstrokeopacity{0.400000}%
\pgfsetdash{}{0pt}%
\pgfpathmoveto{\pgfqpoint{3.033724in}{2.038951in}}%
\pgfpathcurveto{\pgfqpoint{3.044774in}{2.038951in}}{\pgfqpoint{3.055373in}{2.043342in}}{\pgfqpoint{3.063187in}{2.051155in}}%
\pgfpathcurveto{\pgfqpoint{3.071000in}{2.058969in}}{\pgfqpoint{3.075390in}{2.069568in}}{\pgfqpoint{3.075390in}{2.080618in}}%
\pgfpathcurveto{\pgfqpoint{3.075390in}{2.091668in}}{\pgfqpoint{3.071000in}{2.102267in}}{\pgfqpoint{3.063187in}{2.110081in}}%
\pgfpathcurveto{\pgfqpoint{3.055373in}{2.117895in}}{\pgfqpoint{3.044774in}{2.122285in}}{\pgfqpoint{3.033724in}{2.122285in}}%
\pgfpathcurveto{\pgfqpoint{3.022674in}{2.122285in}}{\pgfqpoint{3.012075in}{2.117895in}}{\pgfqpoint{3.004261in}{2.110081in}}%
\pgfpathcurveto{\pgfqpoint{2.996447in}{2.102267in}}{\pgfqpoint{2.992057in}{2.091668in}}{\pgfqpoint{2.992057in}{2.080618in}}%
\pgfpathcurveto{\pgfqpoint{2.992057in}{2.069568in}}{\pgfqpoint{2.996447in}{2.058969in}}{\pgfqpoint{3.004261in}{2.051155in}}%
\pgfpathcurveto{\pgfqpoint{3.012075in}{2.043342in}}{\pgfqpoint{3.022674in}{2.038951in}}{\pgfqpoint{3.033724in}{2.038951in}}%
\pgfpathclose%
\pgfusepath{stroke,fill}%
\end{pgfscope}%
\begin{pgfscope}%
\pgfpathrectangle{\pgfqpoint{0.511823in}{0.504323in}}{\pgfqpoint{3.218177in}{3.225677in}} %
\pgfusepath{clip}%
\pgfsetbuttcap%
\pgfsetroundjoin%
\definecolor{currentfill}{rgb}{0.501961,0.000000,0.000000}%
\pgfsetfillcolor{currentfill}%
\pgfsetfillopacity{0.400000}%
\pgfsetlinewidth{0.501875pt}%
\definecolor{currentstroke}{rgb}{0.501961,0.000000,0.000000}%
\pgfsetstrokecolor{currentstroke}%
\pgfsetstrokeopacity{0.400000}%
\pgfsetdash{}{0pt}%
\pgfpathmoveto{\pgfqpoint{2.937062in}{1.990649in}}%
\pgfpathcurveto{\pgfqpoint{2.948112in}{1.990649in}}{\pgfqpoint{2.958711in}{1.995039in}}{\pgfqpoint{2.966525in}{2.002853in}}%
\pgfpathcurveto{\pgfqpoint{2.974338in}{2.010666in}}{\pgfqpoint{2.978729in}{2.021265in}}{\pgfqpoint{2.978729in}{2.032315in}}%
\pgfpathcurveto{\pgfqpoint{2.978729in}{2.043366in}}{\pgfqpoint{2.974338in}{2.053965in}}{\pgfqpoint{2.966525in}{2.061778in}}%
\pgfpathcurveto{\pgfqpoint{2.958711in}{2.069592in}}{\pgfqpoint{2.948112in}{2.073982in}}{\pgfqpoint{2.937062in}{2.073982in}}%
\pgfpathcurveto{\pgfqpoint{2.926012in}{2.073982in}}{\pgfqpoint{2.915413in}{2.069592in}}{\pgfqpoint{2.907599in}{2.061778in}}%
\pgfpathcurveto{\pgfqpoint{2.899786in}{2.053965in}}{\pgfqpoint{2.895395in}{2.043366in}}{\pgfqpoint{2.895395in}{2.032315in}}%
\pgfpathcurveto{\pgfqpoint{2.895395in}{2.021265in}}{\pgfqpoint{2.899786in}{2.010666in}}{\pgfqpoint{2.907599in}{2.002853in}}%
\pgfpathcurveto{\pgfqpoint{2.915413in}{1.995039in}}{\pgfqpoint{2.926012in}{1.990649in}}{\pgfqpoint{2.937062in}{1.990649in}}%
\pgfpathclose%
\pgfusepath{stroke,fill}%
\end{pgfscope}%
\begin{pgfscope}%
\pgfpathrectangle{\pgfqpoint{0.511823in}{0.504323in}}{\pgfqpoint{3.218177in}{3.225677in}} %
\pgfusepath{clip}%
\pgfsetbuttcap%
\pgfsetroundjoin%
\definecolor{currentfill}{rgb}{0.501961,0.000000,0.000000}%
\pgfsetfillcolor{currentfill}%
\pgfsetfillopacity{0.400000}%
\pgfsetlinewidth{0.501875pt}%
\definecolor{currentstroke}{rgb}{0.501961,0.000000,0.000000}%
\pgfsetstrokecolor{currentstroke}%
\pgfsetstrokeopacity{0.400000}%
\pgfsetdash{}{0pt}%
\pgfpathmoveto{\pgfqpoint{3.019740in}{2.049972in}}%
\pgfpathcurveto{\pgfqpoint{3.030790in}{2.049972in}}{\pgfqpoint{3.041389in}{2.054362in}}{\pgfqpoint{3.049202in}{2.062176in}}%
\pgfpathcurveto{\pgfqpoint{3.057016in}{2.069990in}}{\pgfqpoint{3.061406in}{2.080589in}}{\pgfqpoint{3.061406in}{2.091639in}}%
\pgfpathcurveto{\pgfqpoint{3.061406in}{2.102689in}}{\pgfqpoint{3.057016in}{2.113288in}}{\pgfqpoint{3.049202in}{2.121102in}}%
\pgfpathcurveto{\pgfqpoint{3.041389in}{2.128915in}}{\pgfqpoint{3.030790in}{2.133305in}}{\pgfqpoint{3.019740in}{2.133305in}}%
\pgfpathcurveto{\pgfqpoint{3.008690in}{2.133305in}}{\pgfqpoint{2.998090in}{2.128915in}}{\pgfqpoint{2.990277in}{2.121102in}}%
\pgfpathcurveto{\pgfqpoint{2.982463in}{2.113288in}}{\pgfqpoint{2.978073in}{2.102689in}}{\pgfqpoint{2.978073in}{2.091639in}}%
\pgfpathcurveto{\pgfqpoint{2.978073in}{2.080589in}}{\pgfqpoint{2.982463in}{2.069990in}}{\pgfqpoint{2.990277in}{2.062176in}}%
\pgfpathcurveto{\pgfqpoint{2.998090in}{2.054362in}}{\pgfqpoint{3.008690in}{2.049972in}}{\pgfqpoint{3.019740in}{2.049972in}}%
\pgfpathclose%
\pgfusepath{stroke,fill}%
\end{pgfscope}%
\begin{pgfscope}%
\pgfpathrectangle{\pgfqpoint{0.511823in}{0.504323in}}{\pgfqpoint{3.218177in}{3.225677in}} %
\pgfusepath{clip}%
\pgfsetbuttcap%
\pgfsetroundjoin%
\definecolor{currentfill}{rgb}{0.501961,0.000000,0.000000}%
\pgfsetfillcolor{currentfill}%
\pgfsetfillopacity{0.400000}%
\pgfsetlinewidth{0.501875pt}%
\definecolor{currentstroke}{rgb}{0.501961,0.000000,0.000000}%
\pgfsetstrokecolor{currentstroke}%
\pgfsetstrokeopacity{0.400000}%
\pgfsetdash{}{0pt}%
\pgfpathmoveto{\pgfqpoint{2.956326in}{2.021119in}}%
\pgfpathcurveto{\pgfqpoint{2.967376in}{2.021119in}}{\pgfqpoint{2.977975in}{2.025509in}}{\pgfqpoint{2.985789in}{2.033323in}}%
\pgfpathcurveto{\pgfqpoint{2.993602in}{2.041136in}}{\pgfqpoint{2.997992in}{2.051735in}}{\pgfqpoint{2.997992in}{2.062785in}}%
\pgfpathcurveto{\pgfqpoint{2.997992in}{2.073836in}}{\pgfqpoint{2.993602in}{2.084435in}}{\pgfqpoint{2.985789in}{2.092248in}}%
\pgfpathcurveto{\pgfqpoint{2.977975in}{2.100062in}}{\pgfqpoint{2.967376in}{2.104452in}}{\pgfqpoint{2.956326in}{2.104452in}}%
\pgfpathcurveto{\pgfqpoint{2.945276in}{2.104452in}}{\pgfqpoint{2.934677in}{2.100062in}}{\pgfqpoint{2.926863in}{2.092248in}}%
\pgfpathcurveto{\pgfqpoint{2.919049in}{2.084435in}}{\pgfqpoint{2.914659in}{2.073836in}}{\pgfqpoint{2.914659in}{2.062785in}}%
\pgfpathcurveto{\pgfqpoint{2.914659in}{2.051735in}}{\pgfqpoint{2.919049in}{2.041136in}}{\pgfqpoint{2.926863in}{2.033323in}}%
\pgfpathcurveto{\pgfqpoint{2.934677in}{2.025509in}}{\pgfqpoint{2.945276in}{2.021119in}}{\pgfqpoint{2.956326in}{2.021119in}}%
\pgfpathclose%
\pgfusepath{stroke,fill}%
\end{pgfscope}%
\begin{pgfscope}%
\pgfpathrectangle{\pgfqpoint{0.511823in}{0.504323in}}{\pgfqpoint{3.218177in}{3.225677in}} %
\pgfusepath{clip}%
\pgfsetbuttcap%
\pgfsetroundjoin%
\definecolor{currentfill}{rgb}{0.501961,0.000000,0.000000}%
\pgfsetfillcolor{currentfill}%
\pgfsetfillopacity{0.400000}%
\pgfsetlinewidth{0.501875pt}%
\definecolor{currentstroke}{rgb}{0.501961,0.000000,0.000000}%
\pgfsetstrokecolor{currentstroke}%
\pgfsetstrokeopacity{0.400000}%
\pgfsetdash{}{0pt}%
\pgfpathmoveto{\pgfqpoint{2.810097in}{1.941030in}}%
\pgfpathcurveto{\pgfqpoint{2.821147in}{1.941030in}}{\pgfqpoint{2.831746in}{1.945420in}}{\pgfqpoint{2.839560in}{1.953233in}}%
\pgfpathcurveto{\pgfqpoint{2.847374in}{1.961047in}}{\pgfqpoint{2.851764in}{1.971646in}}{\pgfqpoint{2.851764in}{1.982696in}}%
\pgfpathcurveto{\pgfqpoint{2.851764in}{1.993746in}}{\pgfqpoint{2.847374in}{2.004345in}}{\pgfqpoint{2.839560in}{2.012159in}}%
\pgfpathcurveto{\pgfqpoint{2.831746in}{2.019973in}}{\pgfqpoint{2.821147in}{2.024363in}}{\pgfqpoint{2.810097in}{2.024363in}}%
\pgfpathcurveto{\pgfqpoint{2.799047in}{2.024363in}}{\pgfqpoint{2.788448in}{2.019973in}}{\pgfqpoint{2.780634in}{2.012159in}}%
\pgfpathcurveto{\pgfqpoint{2.772821in}{2.004345in}}{\pgfqpoint{2.768431in}{1.993746in}}{\pgfqpoint{2.768431in}{1.982696in}}%
\pgfpathcurveto{\pgfqpoint{2.768431in}{1.971646in}}{\pgfqpoint{2.772821in}{1.961047in}}{\pgfqpoint{2.780634in}{1.953233in}}%
\pgfpathcurveto{\pgfqpoint{2.788448in}{1.945420in}}{\pgfqpoint{2.799047in}{1.941030in}}{\pgfqpoint{2.810097in}{1.941030in}}%
\pgfpathclose%
\pgfusepath{stroke,fill}%
\end{pgfscope}%
\begin{pgfscope}%
\pgfpathrectangle{\pgfqpoint{0.511823in}{0.504323in}}{\pgfqpoint{3.218177in}{3.225677in}} %
\pgfusepath{clip}%
\pgfsetbuttcap%
\pgfsetroundjoin%
\definecolor{currentfill}{rgb}{0.501961,0.000000,0.000000}%
\pgfsetfillcolor{currentfill}%
\pgfsetfillopacity{0.400000}%
\pgfsetlinewidth{0.501875pt}%
\definecolor{currentstroke}{rgb}{0.501961,0.000000,0.000000}%
\pgfsetstrokecolor{currentstroke}%
\pgfsetstrokeopacity{0.400000}%
\pgfsetdash{}{0pt}%
\pgfpathmoveto{\pgfqpoint{2.910607in}{2.011976in}}%
\pgfpathcurveto{\pgfqpoint{2.921657in}{2.011976in}}{\pgfqpoint{2.932257in}{2.016366in}}{\pgfqpoint{2.940070in}{2.024180in}}%
\pgfpathcurveto{\pgfqpoint{2.947884in}{2.031994in}}{\pgfqpoint{2.952274in}{2.042593in}}{\pgfqpoint{2.952274in}{2.053643in}}%
\pgfpathcurveto{\pgfqpoint{2.952274in}{2.064693in}}{\pgfqpoint{2.947884in}{2.075292in}}{\pgfqpoint{2.940070in}{2.083106in}}%
\pgfpathcurveto{\pgfqpoint{2.932257in}{2.090919in}}{\pgfqpoint{2.921657in}{2.095309in}}{\pgfqpoint{2.910607in}{2.095309in}}%
\pgfpathcurveto{\pgfqpoint{2.899557in}{2.095309in}}{\pgfqpoint{2.888958in}{2.090919in}}{\pgfqpoint{2.881145in}{2.083106in}}%
\pgfpathcurveto{\pgfqpoint{2.873331in}{2.075292in}}{\pgfqpoint{2.868941in}{2.064693in}}{\pgfqpoint{2.868941in}{2.053643in}}%
\pgfpathcurveto{\pgfqpoint{2.868941in}{2.042593in}}{\pgfqpoint{2.873331in}{2.031994in}}{\pgfqpoint{2.881145in}{2.024180in}}%
\pgfpathcurveto{\pgfqpoint{2.888958in}{2.016366in}}{\pgfqpoint{2.899557in}{2.011976in}}{\pgfqpoint{2.910607in}{2.011976in}}%
\pgfpathclose%
\pgfusepath{stroke,fill}%
\end{pgfscope}%
\begin{pgfscope}%
\pgfpathrectangle{\pgfqpoint{0.511823in}{0.504323in}}{\pgfqpoint{3.218177in}{3.225677in}} %
\pgfusepath{clip}%
\pgfsetbuttcap%
\pgfsetroundjoin%
\definecolor{currentfill}{rgb}{0.501961,0.000000,0.000000}%
\pgfsetfillcolor{currentfill}%
\pgfsetfillopacity{0.400000}%
\pgfsetlinewidth{0.501875pt}%
\definecolor{currentstroke}{rgb}{0.501961,0.000000,0.000000}%
\pgfsetstrokecolor{currentstroke}%
\pgfsetstrokeopacity{0.400000}%
\pgfsetdash{}{0pt}%
\pgfpathmoveto{\pgfqpoint{2.756770in}{1.925813in}}%
\pgfpathcurveto{\pgfqpoint{2.767820in}{1.925813in}}{\pgfqpoint{2.778419in}{1.930203in}}{\pgfqpoint{2.786232in}{1.938017in}}%
\pgfpathcurveto{\pgfqpoint{2.794046in}{1.945831in}}{\pgfqpoint{2.798436in}{1.956430in}}{\pgfqpoint{2.798436in}{1.967480in}}%
\pgfpathcurveto{\pgfqpoint{2.798436in}{1.978530in}}{\pgfqpoint{2.794046in}{1.989129in}}{\pgfqpoint{2.786232in}{1.996943in}}%
\pgfpathcurveto{\pgfqpoint{2.778419in}{2.004756in}}{\pgfqpoint{2.767820in}{2.009147in}}{\pgfqpoint{2.756770in}{2.009147in}}%
\pgfpathcurveto{\pgfqpoint{2.745719in}{2.009147in}}{\pgfqpoint{2.735120in}{2.004756in}}{\pgfqpoint{2.727307in}{1.996943in}}%
\pgfpathcurveto{\pgfqpoint{2.719493in}{1.989129in}}{\pgfqpoint{2.715103in}{1.978530in}}{\pgfqpoint{2.715103in}{1.967480in}}%
\pgfpathcurveto{\pgfqpoint{2.715103in}{1.956430in}}{\pgfqpoint{2.719493in}{1.945831in}}{\pgfqpoint{2.727307in}{1.938017in}}%
\pgfpathcurveto{\pgfqpoint{2.735120in}{1.930203in}}{\pgfqpoint{2.745719in}{1.925813in}}{\pgfqpoint{2.756770in}{1.925813in}}%
\pgfpathclose%
\pgfusepath{stroke,fill}%
\end{pgfscope}%
\begin{pgfscope}%
\pgfpathrectangle{\pgfqpoint{0.511823in}{0.504323in}}{\pgfqpoint{3.218177in}{3.225677in}} %
\pgfusepath{clip}%
\pgfsetbuttcap%
\pgfsetroundjoin%
\definecolor{currentfill}{rgb}{0.501961,0.000000,0.000000}%
\pgfsetfillcolor{currentfill}%
\pgfsetfillopacity{0.400000}%
\pgfsetlinewidth{0.501875pt}%
\definecolor{currentstroke}{rgb}{0.501961,0.000000,0.000000}%
\pgfsetstrokecolor{currentstroke}%
\pgfsetstrokeopacity{0.400000}%
\pgfsetdash{}{0pt}%
\pgfpathmoveto{\pgfqpoint{2.816927in}{1.972223in}}%
\pgfpathcurveto{\pgfqpoint{2.827977in}{1.972223in}}{\pgfqpoint{2.838577in}{1.976613in}}{\pgfqpoint{2.846390in}{1.984426in}}%
\pgfpathcurveto{\pgfqpoint{2.854204in}{1.992240in}}{\pgfqpoint{2.858594in}{2.002839in}}{\pgfqpoint{2.858594in}{2.013889in}}%
\pgfpathcurveto{\pgfqpoint{2.858594in}{2.024939in}}{\pgfqpoint{2.854204in}{2.035538in}}{\pgfqpoint{2.846390in}{2.043352in}}%
\pgfpathcurveto{\pgfqpoint{2.838577in}{2.051166in}}{\pgfqpoint{2.827977in}{2.055556in}}{\pgfqpoint{2.816927in}{2.055556in}}%
\pgfpathcurveto{\pgfqpoint{2.805877in}{2.055556in}}{\pgfqpoint{2.795278in}{2.051166in}}{\pgfqpoint{2.787465in}{2.043352in}}%
\pgfpathcurveto{\pgfqpoint{2.779651in}{2.035538in}}{\pgfqpoint{2.775261in}{2.024939in}}{\pgfqpoint{2.775261in}{2.013889in}}%
\pgfpathcurveto{\pgfqpoint{2.775261in}{2.002839in}}{\pgfqpoint{2.779651in}{1.992240in}}{\pgfqpoint{2.787465in}{1.984426in}}%
\pgfpathcurveto{\pgfqpoint{2.795278in}{1.976613in}}{\pgfqpoint{2.805877in}{1.972223in}}{\pgfqpoint{2.816927in}{1.972223in}}%
\pgfpathclose%
\pgfusepath{stroke,fill}%
\end{pgfscope}%
\begin{pgfscope}%
\pgfpathrectangle{\pgfqpoint{0.511823in}{0.504323in}}{\pgfqpoint{3.218177in}{3.225677in}} %
\pgfusepath{clip}%
\pgfsetbuttcap%
\pgfsetroundjoin%
\definecolor{currentfill}{rgb}{0.501961,0.000000,0.000000}%
\pgfsetfillcolor{currentfill}%
\pgfsetfillopacity{0.400000}%
\pgfsetlinewidth{0.501875pt}%
\definecolor{currentstroke}{rgb}{0.501961,0.000000,0.000000}%
\pgfsetstrokecolor{currentstroke}%
\pgfsetstrokeopacity{0.400000}%
\pgfsetdash{}{0pt}%
\pgfpathmoveto{\pgfqpoint{2.788579in}{1.963451in}}%
\pgfpathcurveto{\pgfqpoint{2.799629in}{1.963451in}}{\pgfqpoint{2.810228in}{1.967841in}}{\pgfqpoint{2.818041in}{1.975654in}}%
\pgfpathcurveto{\pgfqpoint{2.825855in}{1.983468in}}{\pgfqpoint{2.830245in}{1.994067in}}{\pgfqpoint{2.830245in}{2.005117in}}%
\pgfpathcurveto{\pgfqpoint{2.830245in}{2.016167in}}{\pgfqpoint{2.825855in}{2.026766in}}{\pgfqpoint{2.818041in}{2.034580in}}%
\pgfpathcurveto{\pgfqpoint{2.810228in}{2.042394in}}{\pgfqpoint{2.799629in}{2.046784in}}{\pgfqpoint{2.788579in}{2.046784in}}%
\pgfpathcurveto{\pgfqpoint{2.777528in}{2.046784in}}{\pgfqpoint{2.766929in}{2.042394in}}{\pgfqpoint{2.759116in}{2.034580in}}%
\pgfpathcurveto{\pgfqpoint{2.751302in}{2.026766in}}{\pgfqpoint{2.746912in}{2.016167in}}{\pgfqpoint{2.746912in}{2.005117in}}%
\pgfpathcurveto{\pgfqpoint{2.746912in}{1.994067in}}{\pgfqpoint{2.751302in}{1.983468in}}{\pgfqpoint{2.759116in}{1.975654in}}%
\pgfpathcurveto{\pgfqpoint{2.766929in}{1.967841in}}{\pgfqpoint{2.777528in}{1.963451in}}{\pgfqpoint{2.788579in}{1.963451in}}%
\pgfpathclose%
\pgfusepath{stroke,fill}%
\end{pgfscope}%
\begin{pgfscope}%
\pgfpathrectangle{\pgfqpoint{0.511823in}{0.504323in}}{\pgfqpoint{3.218177in}{3.225677in}} %
\pgfusepath{clip}%
\pgfsetbuttcap%
\pgfsetroundjoin%
\definecolor{currentfill}{rgb}{0.501961,0.000000,0.000000}%
\pgfsetfillcolor{currentfill}%
\pgfsetfillopacity{0.400000}%
\pgfsetlinewidth{0.501875pt}%
\definecolor{currentstroke}{rgb}{0.501961,0.000000,0.000000}%
\pgfsetstrokecolor{currentstroke}%
\pgfsetstrokeopacity{0.400000}%
\pgfsetdash{}{0pt}%
\pgfpathmoveto{\pgfqpoint{2.896239in}{2.040700in}}%
\pgfpathcurveto{\pgfqpoint{2.907290in}{2.040700in}}{\pgfqpoint{2.917889in}{2.045090in}}{\pgfqpoint{2.925702in}{2.052904in}}%
\pgfpathcurveto{\pgfqpoint{2.933516in}{2.060718in}}{\pgfqpoint{2.937906in}{2.071317in}}{\pgfqpoint{2.937906in}{2.082367in}}%
\pgfpathcurveto{\pgfqpoint{2.937906in}{2.093417in}}{\pgfqpoint{2.933516in}{2.104016in}}{\pgfqpoint{2.925702in}{2.111830in}}%
\pgfpathcurveto{\pgfqpoint{2.917889in}{2.119643in}}{\pgfqpoint{2.907290in}{2.124033in}}{\pgfqpoint{2.896239in}{2.124033in}}%
\pgfpathcurveto{\pgfqpoint{2.885189in}{2.124033in}}{\pgfqpoint{2.874590in}{2.119643in}}{\pgfqpoint{2.866777in}{2.111830in}}%
\pgfpathcurveto{\pgfqpoint{2.858963in}{2.104016in}}{\pgfqpoint{2.854573in}{2.093417in}}{\pgfqpoint{2.854573in}{2.082367in}}%
\pgfpathcurveto{\pgfqpoint{2.854573in}{2.071317in}}{\pgfqpoint{2.858963in}{2.060718in}}{\pgfqpoint{2.866777in}{2.052904in}}%
\pgfpathcurveto{\pgfqpoint{2.874590in}{2.045090in}}{\pgfqpoint{2.885189in}{2.040700in}}{\pgfqpoint{2.896239in}{2.040700in}}%
\pgfpathclose%
\pgfusepath{stroke,fill}%
\end{pgfscope}%
\begin{pgfscope}%
\pgfpathrectangle{\pgfqpoint{0.511823in}{0.504323in}}{\pgfqpoint{3.218177in}{3.225677in}} %
\pgfusepath{clip}%
\pgfsetbuttcap%
\pgfsetroundjoin%
\definecolor{currentfill}{rgb}{0.501961,0.000000,0.000000}%
\pgfsetfillcolor{currentfill}%
\pgfsetfillopacity{0.400000}%
\pgfsetlinewidth{0.501875pt}%
\definecolor{currentstroke}{rgb}{0.501961,0.000000,0.000000}%
\pgfsetstrokecolor{currentstroke}%
\pgfsetstrokeopacity{0.400000}%
\pgfsetdash{}{0pt}%
\pgfpathmoveto{\pgfqpoint{2.990726in}{2.110505in}}%
\pgfpathcurveto{\pgfqpoint{3.001776in}{2.110505in}}{\pgfqpoint{3.012375in}{2.114895in}}{\pgfqpoint{3.020189in}{2.122709in}}%
\pgfpathcurveto{\pgfqpoint{3.028003in}{2.130523in}}{\pgfqpoint{3.032393in}{2.141122in}}{\pgfqpoint{3.032393in}{2.152172in}}%
\pgfpathcurveto{\pgfqpoint{3.032393in}{2.163222in}}{\pgfqpoint{3.028003in}{2.173821in}}{\pgfqpoint{3.020189in}{2.181635in}}%
\pgfpathcurveto{\pgfqpoint{3.012375in}{2.189448in}}{\pgfqpoint{3.001776in}{2.193839in}}{\pgfqpoint{2.990726in}{2.193839in}}%
\pgfpathcurveto{\pgfqpoint{2.979676in}{2.193839in}}{\pgfqpoint{2.969077in}{2.189448in}}{\pgfqpoint{2.961263in}{2.181635in}}%
\pgfpathcurveto{\pgfqpoint{2.953450in}{2.173821in}}{\pgfqpoint{2.949060in}{2.163222in}}{\pgfqpoint{2.949060in}{2.152172in}}%
\pgfpathcurveto{\pgfqpoint{2.949060in}{2.141122in}}{\pgfqpoint{2.953450in}{2.130523in}}{\pgfqpoint{2.961263in}{2.122709in}}%
\pgfpathcurveto{\pgfqpoint{2.969077in}{2.114895in}}{\pgfqpoint{2.979676in}{2.110505in}}{\pgfqpoint{2.990726in}{2.110505in}}%
\pgfpathclose%
\pgfusepath{stroke,fill}%
\end{pgfscope}%
\begin{pgfscope}%
\pgfpathrectangle{\pgfqpoint{0.511823in}{0.504323in}}{\pgfqpoint{3.218177in}{3.225677in}} %
\pgfusepath{clip}%
\pgfsetbuttcap%
\pgfsetroundjoin%
\definecolor{currentfill}{rgb}{0.501961,0.000000,0.000000}%
\pgfsetfillcolor{currentfill}%
\pgfsetfillopacity{0.400000}%
\pgfsetlinewidth{0.501875pt}%
\definecolor{currentstroke}{rgb}{0.501961,0.000000,0.000000}%
\pgfsetstrokecolor{currentstroke}%
\pgfsetstrokeopacity{0.400000}%
\pgfsetdash{}{0pt}%
\pgfpathmoveto{\pgfqpoint{2.799205in}{1.997384in}}%
\pgfpathcurveto{\pgfqpoint{2.810256in}{1.997384in}}{\pgfqpoint{2.820855in}{2.001774in}}{\pgfqpoint{2.828668in}{2.009588in}}%
\pgfpathcurveto{\pgfqpoint{2.836482in}{2.017402in}}{\pgfqpoint{2.840872in}{2.028001in}}{\pgfqpoint{2.840872in}{2.039051in}}%
\pgfpathcurveto{\pgfqpoint{2.840872in}{2.050101in}}{\pgfqpoint{2.836482in}{2.060700in}}{\pgfqpoint{2.828668in}{2.068514in}}%
\pgfpathcurveto{\pgfqpoint{2.820855in}{2.076327in}}{\pgfqpoint{2.810256in}{2.080718in}}{\pgfqpoint{2.799205in}{2.080718in}}%
\pgfpathcurveto{\pgfqpoint{2.788155in}{2.080718in}}{\pgfqpoint{2.777556in}{2.076327in}}{\pgfqpoint{2.769743in}{2.068514in}}%
\pgfpathcurveto{\pgfqpoint{2.761929in}{2.060700in}}{\pgfqpoint{2.757539in}{2.050101in}}{\pgfqpoint{2.757539in}{2.039051in}}%
\pgfpathcurveto{\pgfqpoint{2.757539in}{2.028001in}}{\pgfqpoint{2.761929in}{2.017402in}}{\pgfqpoint{2.769743in}{2.009588in}}%
\pgfpathcurveto{\pgfqpoint{2.777556in}{2.001774in}}{\pgfqpoint{2.788155in}{1.997384in}}{\pgfqpoint{2.799205in}{1.997384in}}%
\pgfpathclose%
\pgfusepath{stroke,fill}%
\end{pgfscope}%
\begin{pgfscope}%
\pgfpathrectangle{\pgfqpoint{0.511823in}{0.504323in}}{\pgfqpoint{3.218177in}{3.225677in}} %
\pgfusepath{clip}%
\pgfsetbuttcap%
\pgfsetroundjoin%
\definecolor{currentfill}{rgb}{0.501961,0.000000,0.000000}%
\pgfsetfillcolor{currentfill}%
\pgfsetfillopacity{0.400000}%
\pgfsetlinewidth{0.501875pt}%
\definecolor{currentstroke}{rgb}{0.501961,0.000000,0.000000}%
\pgfsetstrokecolor{currentstroke}%
\pgfsetstrokeopacity{0.400000}%
\pgfsetdash{}{0pt}%
\pgfpathmoveto{\pgfqpoint{2.915044in}{2.081482in}}%
\pgfpathcurveto{\pgfqpoint{2.926094in}{2.081482in}}{\pgfqpoint{2.936693in}{2.085872in}}{\pgfqpoint{2.944507in}{2.093686in}}%
\pgfpathcurveto{\pgfqpoint{2.952321in}{2.101499in}}{\pgfqpoint{2.956711in}{2.112098in}}{\pgfqpoint{2.956711in}{2.123148in}}%
\pgfpathcurveto{\pgfqpoint{2.956711in}{2.134198in}}{\pgfqpoint{2.952321in}{2.144798in}}{\pgfqpoint{2.944507in}{2.152611in}}%
\pgfpathcurveto{\pgfqpoint{2.936693in}{2.160425in}}{\pgfqpoint{2.926094in}{2.164815in}}{\pgfqpoint{2.915044in}{2.164815in}}%
\pgfpathcurveto{\pgfqpoint{2.903994in}{2.164815in}}{\pgfqpoint{2.893395in}{2.160425in}}{\pgfqpoint{2.885581in}{2.152611in}}%
\pgfpathcurveto{\pgfqpoint{2.877768in}{2.144798in}}{\pgfqpoint{2.873377in}{2.134198in}}{\pgfqpoint{2.873377in}{2.123148in}}%
\pgfpathcurveto{\pgfqpoint{2.873377in}{2.112098in}}{\pgfqpoint{2.877768in}{2.101499in}}{\pgfqpoint{2.885581in}{2.093686in}}%
\pgfpathcurveto{\pgfqpoint{2.893395in}{2.085872in}}{\pgfqpoint{2.903994in}{2.081482in}}{\pgfqpoint{2.915044in}{2.081482in}}%
\pgfpathclose%
\pgfusepath{stroke,fill}%
\end{pgfscope}%
\begin{pgfscope}%
\pgfpathrectangle{\pgfqpoint{0.511823in}{0.504323in}}{\pgfqpoint{3.218177in}{3.225677in}} %
\pgfusepath{clip}%
\pgfsetbuttcap%
\pgfsetroundjoin%
\definecolor{currentfill}{rgb}{0.501961,0.000000,0.000000}%
\pgfsetfillcolor{currentfill}%
\pgfsetfillopacity{0.400000}%
\pgfsetlinewidth{0.501875pt}%
\definecolor{currentstroke}{rgb}{0.501961,0.000000,0.000000}%
\pgfsetstrokecolor{currentstroke}%
\pgfsetstrokeopacity{0.400000}%
\pgfsetdash{}{0pt}%
\pgfpathmoveto{\pgfqpoint{2.628404in}{1.904506in}}%
\pgfpathcurveto{\pgfqpoint{2.639454in}{1.904506in}}{\pgfqpoint{2.650053in}{1.908896in}}{\pgfqpoint{2.657867in}{1.916710in}}%
\pgfpathcurveto{\pgfqpoint{2.665681in}{1.924523in}}{\pgfqpoint{2.670071in}{1.935122in}}{\pgfqpoint{2.670071in}{1.946172in}}%
\pgfpathcurveto{\pgfqpoint{2.670071in}{1.957223in}}{\pgfqpoint{2.665681in}{1.967822in}}{\pgfqpoint{2.657867in}{1.975635in}}%
\pgfpathcurveto{\pgfqpoint{2.650053in}{1.983449in}}{\pgfqpoint{2.639454in}{1.987839in}}{\pgfqpoint{2.628404in}{1.987839in}}%
\pgfpathcurveto{\pgfqpoint{2.617354in}{1.987839in}}{\pgfqpoint{2.606755in}{1.983449in}}{\pgfqpoint{2.598941in}{1.975635in}}%
\pgfpathcurveto{\pgfqpoint{2.591128in}{1.967822in}}{\pgfqpoint{2.586738in}{1.957223in}}{\pgfqpoint{2.586738in}{1.946172in}}%
\pgfpathcurveto{\pgfqpoint{2.586738in}{1.935122in}}{\pgfqpoint{2.591128in}{1.924523in}}{\pgfqpoint{2.598941in}{1.916710in}}%
\pgfpathcurveto{\pgfqpoint{2.606755in}{1.908896in}}{\pgfqpoint{2.617354in}{1.904506in}}{\pgfqpoint{2.628404in}{1.904506in}}%
\pgfpathclose%
\pgfusepath{stroke,fill}%
\end{pgfscope}%
\begin{pgfscope}%
\pgfpathrectangle{\pgfqpoint{0.511823in}{0.504323in}}{\pgfqpoint{3.218177in}{3.225677in}} %
\pgfusepath{clip}%
\pgfsetbuttcap%
\pgfsetroundjoin%
\definecolor{currentfill}{rgb}{0.501961,0.000000,0.000000}%
\pgfsetfillcolor{currentfill}%
\pgfsetfillopacity{0.400000}%
\pgfsetlinewidth{0.501875pt}%
\definecolor{currentstroke}{rgb}{0.501961,0.000000,0.000000}%
\pgfsetstrokecolor{currentstroke}%
\pgfsetstrokeopacity{0.400000}%
\pgfsetdash{}{0pt}%
\pgfpathmoveto{\pgfqpoint{2.739068in}{1.985555in}}%
\pgfpathcurveto{\pgfqpoint{2.750118in}{1.985555in}}{\pgfqpoint{2.760717in}{1.989945in}}{\pgfqpoint{2.768531in}{1.997759in}}%
\pgfpathcurveto{\pgfqpoint{2.776344in}{2.005573in}}{\pgfqpoint{2.780735in}{2.016172in}}{\pgfqpoint{2.780735in}{2.027222in}}%
\pgfpathcurveto{\pgfqpoint{2.780735in}{2.038272in}}{\pgfqpoint{2.776344in}{2.048871in}}{\pgfqpoint{2.768531in}{2.056685in}}%
\pgfpathcurveto{\pgfqpoint{2.760717in}{2.064498in}}{\pgfqpoint{2.750118in}{2.068888in}}{\pgfqpoint{2.739068in}{2.068888in}}%
\pgfpathcurveto{\pgfqpoint{2.728018in}{2.068888in}}{\pgfqpoint{2.717419in}{2.064498in}}{\pgfqpoint{2.709605in}{2.056685in}}%
\pgfpathcurveto{\pgfqpoint{2.701791in}{2.048871in}}{\pgfqpoint{2.697401in}{2.038272in}}{\pgfqpoint{2.697401in}{2.027222in}}%
\pgfpathcurveto{\pgfqpoint{2.697401in}{2.016172in}}{\pgfqpoint{2.701791in}{2.005573in}}{\pgfqpoint{2.709605in}{1.997759in}}%
\pgfpathcurveto{\pgfqpoint{2.717419in}{1.989945in}}{\pgfqpoint{2.728018in}{1.985555in}}{\pgfqpoint{2.739068in}{1.985555in}}%
\pgfpathclose%
\pgfusepath{stroke,fill}%
\end{pgfscope}%
\begin{pgfscope}%
\pgfpathrectangle{\pgfqpoint{0.511823in}{0.504323in}}{\pgfqpoint{3.218177in}{3.225677in}} %
\pgfusepath{clip}%
\pgfsetbuttcap%
\pgfsetroundjoin%
\definecolor{currentfill}{rgb}{0.501961,0.000000,0.000000}%
\pgfsetfillcolor{currentfill}%
\pgfsetfillopacity{0.400000}%
\pgfsetlinewidth{0.501875pt}%
\definecolor{currentstroke}{rgb}{0.501961,0.000000,0.000000}%
\pgfsetstrokecolor{currentstroke}%
\pgfsetstrokeopacity{0.400000}%
\pgfsetdash{}{0pt}%
\pgfpathmoveto{\pgfqpoint{2.908433in}{2.106359in}}%
\pgfpathcurveto{\pgfqpoint{2.919483in}{2.106359in}}{\pgfqpoint{2.930082in}{2.110750in}}{\pgfqpoint{2.937895in}{2.118563in}}%
\pgfpathcurveto{\pgfqpoint{2.945709in}{2.126377in}}{\pgfqpoint{2.950099in}{2.136976in}}{\pgfqpoint{2.950099in}{2.148026in}}%
\pgfpathcurveto{\pgfqpoint{2.950099in}{2.159076in}}{\pgfqpoint{2.945709in}{2.169675in}}{\pgfqpoint{2.937895in}{2.177489in}}%
\pgfpathcurveto{\pgfqpoint{2.930082in}{2.185302in}}{\pgfqpoint{2.919483in}{2.189693in}}{\pgfqpoint{2.908433in}{2.189693in}}%
\pgfpathcurveto{\pgfqpoint{2.897382in}{2.189693in}}{\pgfqpoint{2.886783in}{2.185302in}}{\pgfqpoint{2.878970in}{2.177489in}}%
\pgfpathcurveto{\pgfqpoint{2.871156in}{2.169675in}}{\pgfqpoint{2.866766in}{2.159076in}}{\pgfqpoint{2.866766in}{2.148026in}}%
\pgfpathcurveto{\pgfqpoint{2.866766in}{2.136976in}}{\pgfqpoint{2.871156in}{2.126377in}}{\pgfqpoint{2.878970in}{2.118563in}}%
\pgfpathcurveto{\pgfqpoint{2.886783in}{2.110750in}}{\pgfqpoint{2.897382in}{2.106359in}}{\pgfqpoint{2.908433in}{2.106359in}}%
\pgfpathclose%
\pgfusepath{stroke,fill}%
\end{pgfscope}%
\begin{pgfscope}%
\pgfpathrectangle{\pgfqpoint{0.511823in}{0.504323in}}{\pgfqpoint{3.218177in}{3.225677in}} %
\pgfusepath{clip}%
\pgfsetbuttcap%
\pgfsetroundjoin%
\definecolor{currentfill}{rgb}{0.501961,0.000000,0.000000}%
\pgfsetfillcolor{currentfill}%
\pgfsetfillopacity{0.400000}%
\pgfsetlinewidth{0.501875pt}%
\definecolor{currentstroke}{rgb}{0.501961,0.000000,0.000000}%
\pgfsetstrokecolor{currentstroke}%
\pgfsetstrokeopacity{0.400000}%
\pgfsetdash{}{0pt}%
\pgfpathmoveto{\pgfqpoint{2.949071in}{2.143167in}}%
\pgfpathcurveto{\pgfqpoint{2.960121in}{2.143167in}}{\pgfqpoint{2.970720in}{2.147557in}}{\pgfqpoint{2.978534in}{2.155371in}}%
\pgfpathcurveto{\pgfqpoint{2.986348in}{2.163184in}}{\pgfqpoint{2.990738in}{2.173783in}}{\pgfqpoint{2.990738in}{2.184833in}}%
\pgfpathcurveto{\pgfqpoint{2.990738in}{2.195884in}}{\pgfqpoint{2.986348in}{2.206483in}}{\pgfqpoint{2.978534in}{2.214296in}}%
\pgfpathcurveto{\pgfqpoint{2.970720in}{2.222110in}}{\pgfqpoint{2.960121in}{2.226500in}}{\pgfqpoint{2.949071in}{2.226500in}}%
\pgfpathcurveto{\pgfqpoint{2.938021in}{2.226500in}}{\pgfqpoint{2.927422in}{2.222110in}}{\pgfqpoint{2.919608in}{2.214296in}}%
\pgfpathcurveto{\pgfqpoint{2.911795in}{2.206483in}}{\pgfqpoint{2.907404in}{2.195884in}}{\pgfqpoint{2.907404in}{2.184833in}}%
\pgfpathcurveto{\pgfqpoint{2.907404in}{2.173783in}}{\pgfqpoint{2.911795in}{2.163184in}}{\pgfqpoint{2.919608in}{2.155371in}}%
\pgfpathcurveto{\pgfqpoint{2.927422in}{2.147557in}}{\pgfqpoint{2.938021in}{2.143167in}}{\pgfqpoint{2.949071in}{2.143167in}}%
\pgfpathclose%
\pgfusepath{stroke,fill}%
\end{pgfscope}%
\begin{pgfscope}%
\pgfpathrectangle{\pgfqpoint{0.511823in}{0.504323in}}{\pgfqpoint{3.218177in}{3.225677in}} %
\pgfusepath{clip}%
\pgfsetbuttcap%
\pgfsetroundjoin%
\definecolor{currentfill}{rgb}{0.501961,0.000000,0.000000}%
\pgfsetfillcolor{currentfill}%
\pgfsetfillopacity{0.400000}%
\pgfsetlinewidth{0.501875pt}%
\definecolor{currentstroke}{rgb}{0.501961,0.000000,0.000000}%
\pgfsetstrokecolor{currentstroke}%
\pgfsetstrokeopacity{0.400000}%
\pgfsetdash{}{0pt}%
\pgfpathmoveto{\pgfqpoint{2.942406in}{2.148738in}}%
\pgfpathcurveto{\pgfqpoint{2.953457in}{2.148738in}}{\pgfqpoint{2.964056in}{2.153128in}}{\pgfqpoint{2.971869in}{2.160942in}}%
\pgfpathcurveto{\pgfqpoint{2.979683in}{2.168756in}}{\pgfqpoint{2.984073in}{2.179355in}}{\pgfqpoint{2.984073in}{2.190405in}}%
\pgfpathcurveto{\pgfqpoint{2.984073in}{2.201455in}}{\pgfqpoint{2.979683in}{2.212054in}}{\pgfqpoint{2.971869in}{2.219868in}}%
\pgfpathcurveto{\pgfqpoint{2.964056in}{2.227681in}}{\pgfqpoint{2.953457in}{2.232072in}}{\pgfqpoint{2.942406in}{2.232072in}}%
\pgfpathcurveto{\pgfqpoint{2.931356in}{2.232072in}}{\pgfqpoint{2.920757in}{2.227681in}}{\pgfqpoint{2.912944in}{2.219868in}}%
\pgfpathcurveto{\pgfqpoint{2.905130in}{2.212054in}}{\pgfqpoint{2.900740in}{2.201455in}}{\pgfqpoint{2.900740in}{2.190405in}}%
\pgfpathcurveto{\pgfqpoint{2.900740in}{2.179355in}}{\pgfqpoint{2.905130in}{2.168756in}}{\pgfqpoint{2.912944in}{2.160942in}}%
\pgfpathcurveto{\pgfqpoint{2.920757in}{2.153128in}}{\pgfqpoint{2.931356in}{2.148738in}}{\pgfqpoint{2.942406in}{2.148738in}}%
\pgfpathclose%
\pgfusepath{stroke,fill}%
\end{pgfscope}%
\begin{pgfscope}%
\pgfpathrectangle{\pgfqpoint{0.511823in}{0.504323in}}{\pgfqpoint{3.218177in}{3.225677in}} %
\pgfusepath{clip}%
\pgfsetbuttcap%
\pgfsetroundjoin%
\definecolor{currentfill}{rgb}{0.501961,0.000000,0.000000}%
\pgfsetfillcolor{currentfill}%
\pgfsetfillopacity{0.400000}%
\pgfsetlinewidth{0.501875pt}%
\definecolor{currentstroke}{rgb}{0.501961,0.000000,0.000000}%
\pgfsetstrokecolor{currentstroke}%
\pgfsetstrokeopacity{0.400000}%
\pgfsetdash{}{0pt}%
\pgfpathmoveto{\pgfqpoint{2.950848in}{2.164467in}}%
\pgfpathcurveto{\pgfqpoint{2.961898in}{2.164467in}}{\pgfqpoint{2.972497in}{2.168857in}}{\pgfqpoint{2.980310in}{2.176671in}}%
\pgfpathcurveto{\pgfqpoint{2.988124in}{2.184485in}}{\pgfqpoint{2.992514in}{2.195084in}}{\pgfqpoint{2.992514in}{2.206134in}}%
\pgfpathcurveto{\pgfqpoint{2.992514in}{2.217184in}}{\pgfqpoint{2.988124in}{2.227783in}}{\pgfqpoint{2.980310in}{2.235597in}}%
\pgfpathcurveto{\pgfqpoint{2.972497in}{2.243410in}}{\pgfqpoint{2.961898in}{2.247800in}}{\pgfqpoint{2.950848in}{2.247800in}}%
\pgfpathcurveto{\pgfqpoint{2.939798in}{2.247800in}}{\pgfqpoint{2.929199in}{2.243410in}}{\pgfqpoint{2.921385in}{2.235597in}}%
\pgfpathcurveto{\pgfqpoint{2.913571in}{2.227783in}}{\pgfqpoint{2.909181in}{2.217184in}}{\pgfqpoint{2.909181in}{2.206134in}}%
\pgfpathcurveto{\pgfqpoint{2.909181in}{2.195084in}}{\pgfqpoint{2.913571in}{2.184485in}}{\pgfqpoint{2.921385in}{2.176671in}}%
\pgfpathcurveto{\pgfqpoint{2.929199in}{2.168857in}}{\pgfqpoint{2.939798in}{2.164467in}}{\pgfqpoint{2.950848in}{2.164467in}}%
\pgfpathclose%
\pgfusepath{stroke,fill}%
\end{pgfscope}%
\begin{pgfscope}%
\pgfpathrectangle{\pgfqpoint{0.511823in}{0.504323in}}{\pgfqpoint{3.218177in}{3.225677in}} %
\pgfusepath{clip}%
\pgfsetbuttcap%
\pgfsetroundjoin%
\definecolor{currentfill}{rgb}{0.501961,0.000000,0.000000}%
\pgfsetfillcolor{currentfill}%
\pgfsetfillopacity{0.400000}%
\pgfsetlinewidth{0.501875pt}%
\definecolor{currentstroke}{rgb}{0.501961,0.000000,0.000000}%
\pgfsetstrokecolor{currentstroke}%
\pgfsetstrokeopacity{0.400000}%
\pgfsetdash{}{0pt}%
\pgfpathmoveto{\pgfqpoint{2.655286in}{1.974262in}}%
\pgfpathcurveto{\pgfqpoint{2.666336in}{1.974262in}}{\pgfqpoint{2.676935in}{1.978652in}}{\pgfqpoint{2.684749in}{1.986466in}}%
\pgfpathcurveto{\pgfqpoint{2.692562in}{1.994279in}}{\pgfqpoint{2.696952in}{2.004878in}}{\pgfqpoint{2.696952in}{2.015928in}}%
\pgfpathcurveto{\pgfqpoint{2.696952in}{2.026978in}}{\pgfqpoint{2.692562in}{2.037577in}}{\pgfqpoint{2.684749in}{2.045391in}}%
\pgfpathcurveto{\pgfqpoint{2.676935in}{2.053205in}}{\pgfqpoint{2.666336in}{2.057595in}}{\pgfqpoint{2.655286in}{2.057595in}}%
\pgfpathcurveto{\pgfqpoint{2.644236in}{2.057595in}}{\pgfqpoint{2.633637in}{2.053205in}}{\pgfqpoint{2.625823in}{2.045391in}}%
\pgfpathcurveto{\pgfqpoint{2.618009in}{2.037577in}}{\pgfqpoint{2.613619in}{2.026978in}}{\pgfqpoint{2.613619in}{2.015928in}}%
\pgfpathcurveto{\pgfqpoint{2.613619in}{2.004878in}}{\pgfqpoint{2.618009in}{1.994279in}}{\pgfqpoint{2.625823in}{1.986466in}}%
\pgfpathcurveto{\pgfqpoint{2.633637in}{1.978652in}}{\pgfqpoint{2.644236in}{1.974262in}}{\pgfqpoint{2.655286in}{1.974262in}}%
\pgfpathclose%
\pgfusepath{stroke,fill}%
\end{pgfscope}%
\begin{pgfscope}%
\pgfpathrectangle{\pgfqpoint{0.511823in}{0.504323in}}{\pgfqpoint{3.218177in}{3.225677in}} %
\pgfusepath{clip}%
\pgfsetbuttcap%
\pgfsetroundjoin%
\definecolor{currentfill}{rgb}{0.501961,0.000000,0.000000}%
\pgfsetfillcolor{currentfill}%
\pgfsetfillopacity{0.400000}%
\pgfsetlinewidth{0.501875pt}%
\definecolor{currentstroke}{rgb}{0.501961,0.000000,0.000000}%
\pgfsetstrokecolor{currentstroke}%
\pgfsetstrokeopacity{0.400000}%
\pgfsetdash{}{0pt}%
\pgfpathmoveto{\pgfqpoint{2.770379in}{2.061637in}}%
\pgfpathcurveto{\pgfqpoint{2.781429in}{2.061637in}}{\pgfqpoint{2.792029in}{2.066028in}}{\pgfqpoint{2.799842in}{2.073841in}}%
\pgfpathcurveto{\pgfqpoint{2.807656in}{2.081655in}}{\pgfqpoint{2.812046in}{2.092254in}}{\pgfqpoint{2.812046in}{2.103304in}}%
\pgfpathcurveto{\pgfqpoint{2.812046in}{2.114354in}}{\pgfqpoint{2.807656in}{2.124953in}}{\pgfqpoint{2.799842in}{2.132767in}}%
\pgfpathcurveto{\pgfqpoint{2.792029in}{2.140580in}}{\pgfqpoint{2.781429in}{2.144971in}}{\pgfqpoint{2.770379in}{2.144971in}}%
\pgfpathcurveto{\pgfqpoint{2.759329in}{2.144971in}}{\pgfqpoint{2.748730in}{2.140580in}}{\pgfqpoint{2.740917in}{2.132767in}}%
\pgfpathcurveto{\pgfqpoint{2.733103in}{2.124953in}}{\pgfqpoint{2.728713in}{2.114354in}}{\pgfqpoint{2.728713in}{2.103304in}}%
\pgfpathcurveto{\pgfqpoint{2.728713in}{2.092254in}}{\pgfqpoint{2.733103in}{2.081655in}}{\pgfqpoint{2.740917in}{2.073841in}}%
\pgfpathcurveto{\pgfqpoint{2.748730in}{2.066028in}}{\pgfqpoint{2.759329in}{2.061637in}}{\pgfqpoint{2.770379in}{2.061637in}}%
\pgfpathclose%
\pgfusepath{stroke,fill}%
\end{pgfscope}%
\begin{pgfscope}%
\pgfpathrectangle{\pgfqpoint{0.511823in}{0.504323in}}{\pgfqpoint{3.218177in}{3.225677in}} %
\pgfusepath{clip}%
\pgfsetbuttcap%
\pgfsetroundjoin%
\definecolor{currentfill}{rgb}{0.501961,0.000000,0.000000}%
\pgfsetfillcolor{currentfill}%
\pgfsetfillopacity{0.400000}%
\pgfsetlinewidth{0.501875pt}%
\definecolor{currentstroke}{rgb}{0.501961,0.000000,0.000000}%
\pgfsetstrokecolor{currentstroke}%
\pgfsetstrokeopacity{0.400000}%
\pgfsetdash{}{0pt}%
\pgfpathmoveto{\pgfqpoint{2.545767in}{1.916778in}}%
\pgfpathcurveto{\pgfqpoint{2.556818in}{1.916778in}}{\pgfqpoint{2.567417in}{1.921168in}}{\pgfqpoint{2.575230in}{1.928982in}}%
\pgfpathcurveto{\pgfqpoint{2.583044in}{1.936796in}}{\pgfqpoint{2.587434in}{1.947395in}}{\pgfqpoint{2.587434in}{1.958445in}}%
\pgfpathcurveto{\pgfqpoint{2.587434in}{1.969495in}}{\pgfqpoint{2.583044in}{1.980094in}}{\pgfqpoint{2.575230in}{1.987907in}}%
\pgfpathcurveto{\pgfqpoint{2.567417in}{1.995721in}}{\pgfqpoint{2.556818in}{2.000111in}}{\pgfqpoint{2.545767in}{2.000111in}}%
\pgfpathcurveto{\pgfqpoint{2.534717in}{2.000111in}}{\pgfqpoint{2.524118in}{1.995721in}}{\pgfqpoint{2.516305in}{1.987907in}}%
\pgfpathcurveto{\pgfqpoint{2.508491in}{1.980094in}}{\pgfqpoint{2.504101in}{1.969495in}}{\pgfqpoint{2.504101in}{1.958445in}}%
\pgfpathcurveto{\pgfqpoint{2.504101in}{1.947395in}}{\pgfqpoint{2.508491in}{1.936796in}}{\pgfqpoint{2.516305in}{1.928982in}}%
\pgfpathcurveto{\pgfqpoint{2.524118in}{1.921168in}}{\pgfqpoint{2.534717in}{1.916778in}}{\pgfqpoint{2.545767in}{1.916778in}}%
\pgfpathclose%
\pgfusepath{stroke,fill}%
\end{pgfscope}%
\begin{pgfscope}%
\pgfpathrectangle{\pgfqpoint{0.511823in}{0.504323in}}{\pgfqpoint{3.218177in}{3.225677in}} %
\pgfusepath{clip}%
\pgfsetbuttcap%
\pgfsetroundjoin%
\definecolor{currentfill}{rgb}{0.501961,0.000000,0.000000}%
\pgfsetfillcolor{currentfill}%
\pgfsetfillopacity{0.400000}%
\pgfsetlinewidth{0.501875pt}%
\definecolor{currentstroke}{rgb}{0.501961,0.000000,0.000000}%
\pgfsetstrokecolor{currentstroke}%
\pgfsetstrokeopacity{0.400000}%
\pgfsetdash{}{0pt}%
\pgfpathmoveto{\pgfqpoint{2.799558in}{2.100670in}}%
\pgfpathcurveto{\pgfqpoint{2.810609in}{2.100670in}}{\pgfqpoint{2.821208in}{2.105060in}}{\pgfqpoint{2.829021in}{2.112874in}}%
\pgfpathcurveto{\pgfqpoint{2.836835in}{2.120687in}}{\pgfqpoint{2.841225in}{2.131286in}}{\pgfqpoint{2.841225in}{2.142336in}}%
\pgfpathcurveto{\pgfqpoint{2.841225in}{2.153386in}}{\pgfqpoint{2.836835in}{2.163985in}}{\pgfqpoint{2.829021in}{2.171799in}}%
\pgfpathcurveto{\pgfqpoint{2.821208in}{2.179613in}}{\pgfqpoint{2.810609in}{2.184003in}}{\pgfqpoint{2.799558in}{2.184003in}}%
\pgfpathcurveto{\pgfqpoint{2.788508in}{2.184003in}}{\pgfqpoint{2.777909in}{2.179613in}}{\pgfqpoint{2.770096in}{2.171799in}}%
\pgfpathcurveto{\pgfqpoint{2.762282in}{2.163985in}}{\pgfqpoint{2.757892in}{2.153386in}}{\pgfqpoint{2.757892in}{2.142336in}}%
\pgfpathcurveto{\pgfqpoint{2.757892in}{2.131286in}}{\pgfqpoint{2.762282in}{2.120687in}}{\pgfqpoint{2.770096in}{2.112874in}}%
\pgfpathcurveto{\pgfqpoint{2.777909in}{2.105060in}}{\pgfqpoint{2.788508in}{2.100670in}}{\pgfqpoint{2.799558in}{2.100670in}}%
\pgfpathclose%
\pgfusepath{stroke,fill}%
\end{pgfscope}%
\begin{pgfscope}%
\pgfpathrectangle{\pgfqpoint{0.511823in}{0.504323in}}{\pgfqpoint{3.218177in}{3.225677in}} %
\pgfusepath{clip}%
\pgfsetbuttcap%
\pgfsetroundjoin%
\definecolor{currentfill}{rgb}{0.501961,0.000000,0.000000}%
\pgfsetfillcolor{currentfill}%
\pgfsetfillopacity{0.400000}%
\pgfsetlinewidth{0.501875pt}%
\definecolor{currentstroke}{rgb}{0.501961,0.000000,0.000000}%
\pgfsetstrokecolor{currentstroke}%
\pgfsetstrokeopacity{0.400000}%
\pgfsetdash{}{0pt}%
\pgfpathmoveto{\pgfqpoint{2.836119in}{2.135741in}}%
\pgfpathcurveto{\pgfqpoint{2.847169in}{2.135741in}}{\pgfqpoint{2.857768in}{2.140132in}}{\pgfqpoint{2.865582in}{2.147945in}}%
\pgfpathcurveto{\pgfqpoint{2.873395in}{2.155759in}}{\pgfqpoint{2.877786in}{2.166358in}}{\pgfqpoint{2.877786in}{2.177408in}}%
\pgfpathcurveto{\pgfqpoint{2.877786in}{2.188458in}}{\pgfqpoint{2.873395in}{2.199057in}}{\pgfqpoint{2.865582in}{2.206871in}}%
\pgfpathcurveto{\pgfqpoint{2.857768in}{2.214684in}}{\pgfqpoint{2.847169in}{2.219075in}}{\pgfqpoint{2.836119in}{2.219075in}}%
\pgfpathcurveto{\pgfqpoint{2.825069in}{2.219075in}}{\pgfqpoint{2.814470in}{2.214684in}}{\pgfqpoint{2.806656in}{2.206871in}}%
\pgfpathcurveto{\pgfqpoint{2.798843in}{2.199057in}}{\pgfqpoint{2.794452in}{2.188458in}}{\pgfqpoint{2.794452in}{2.177408in}}%
\pgfpathcurveto{\pgfqpoint{2.794452in}{2.166358in}}{\pgfqpoint{2.798843in}{2.155759in}}{\pgfqpoint{2.806656in}{2.147945in}}%
\pgfpathcurveto{\pgfqpoint{2.814470in}{2.140132in}}{\pgfqpoint{2.825069in}{2.135741in}}{\pgfqpoint{2.836119in}{2.135741in}}%
\pgfpathclose%
\pgfusepath{stroke,fill}%
\end{pgfscope}%
\begin{pgfscope}%
\pgfpathrectangle{\pgfqpoint{0.511823in}{0.504323in}}{\pgfqpoint{3.218177in}{3.225677in}} %
\pgfusepath{clip}%
\pgfsetbuttcap%
\pgfsetroundjoin%
\definecolor{currentfill}{rgb}{0.501961,0.000000,0.000000}%
\pgfsetfillcolor{currentfill}%
\pgfsetfillopacity{0.400000}%
\pgfsetlinewidth{0.501875pt}%
\definecolor{currentstroke}{rgb}{0.501961,0.000000,0.000000}%
\pgfsetstrokecolor{currentstroke}%
\pgfsetstrokeopacity{0.400000}%
\pgfsetdash{}{0pt}%
\pgfpathmoveto{\pgfqpoint{2.770165in}{2.099383in}}%
\pgfpathcurveto{\pgfqpoint{2.781215in}{2.099383in}}{\pgfqpoint{2.791814in}{2.103773in}}{\pgfqpoint{2.799628in}{2.111587in}}%
\pgfpathcurveto{\pgfqpoint{2.807442in}{2.119401in}}{\pgfqpoint{2.811832in}{2.130000in}}{\pgfqpoint{2.811832in}{2.141050in}}%
\pgfpathcurveto{\pgfqpoint{2.811832in}{2.152100in}}{\pgfqpoint{2.807442in}{2.162699in}}{\pgfqpoint{2.799628in}{2.170513in}}%
\pgfpathcurveto{\pgfqpoint{2.791814in}{2.178326in}}{\pgfqpoint{2.781215in}{2.182716in}}{\pgfqpoint{2.770165in}{2.182716in}}%
\pgfpathcurveto{\pgfqpoint{2.759115in}{2.182716in}}{\pgfqpoint{2.748516in}{2.178326in}}{\pgfqpoint{2.740702in}{2.170513in}}%
\pgfpathcurveto{\pgfqpoint{2.732889in}{2.162699in}}{\pgfqpoint{2.728498in}{2.152100in}}{\pgfqpoint{2.728498in}{2.141050in}}%
\pgfpathcurveto{\pgfqpoint{2.728498in}{2.130000in}}{\pgfqpoint{2.732889in}{2.119401in}}{\pgfqpoint{2.740702in}{2.111587in}}%
\pgfpathcurveto{\pgfqpoint{2.748516in}{2.103773in}}{\pgfqpoint{2.759115in}{2.099383in}}{\pgfqpoint{2.770165in}{2.099383in}}%
\pgfpathclose%
\pgfusepath{stroke,fill}%
\end{pgfscope}%
\begin{pgfscope}%
\pgfpathrectangle{\pgfqpoint{0.511823in}{0.504323in}}{\pgfqpoint{3.218177in}{3.225677in}} %
\pgfusepath{clip}%
\pgfsetbuttcap%
\pgfsetroundjoin%
\definecolor{currentfill}{rgb}{0.501961,0.000000,0.000000}%
\pgfsetfillcolor{currentfill}%
\pgfsetfillopacity{0.400000}%
\pgfsetlinewidth{0.501875pt}%
\definecolor{currentstroke}{rgb}{0.501961,0.000000,0.000000}%
\pgfsetstrokecolor{currentstroke}%
\pgfsetstrokeopacity{0.400000}%
\pgfsetdash{}{0pt}%
\pgfpathmoveto{\pgfqpoint{2.703111in}{2.061689in}}%
\pgfpathcurveto{\pgfqpoint{2.714161in}{2.061689in}}{\pgfqpoint{2.724760in}{2.066080in}}{\pgfqpoint{2.732574in}{2.073893in}}%
\pgfpathcurveto{\pgfqpoint{2.740388in}{2.081707in}}{\pgfqpoint{2.744778in}{2.092306in}}{\pgfqpoint{2.744778in}{2.103356in}}%
\pgfpathcurveto{\pgfqpoint{2.744778in}{2.114406in}}{\pgfqpoint{2.740388in}{2.125005in}}{\pgfqpoint{2.732574in}{2.132819in}}%
\pgfpathcurveto{\pgfqpoint{2.724760in}{2.140632in}}{\pgfqpoint{2.714161in}{2.145023in}}{\pgfqpoint{2.703111in}{2.145023in}}%
\pgfpathcurveto{\pgfqpoint{2.692061in}{2.145023in}}{\pgfqpoint{2.681462in}{2.140632in}}{\pgfqpoint{2.673648in}{2.132819in}}%
\pgfpathcurveto{\pgfqpoint{2.665835in}{2.125005in}}{\pgfqpoint{2.661445in}{2.114406in}}{\pgfqpoint{2.661445in}{2.103356in}}%
\pgfpathcurveto{\pgfqpoint{2.661445in}{2.092306in}}{\pgfqpoint{2.665835in}{2.081707in}}{\pgfqpoint{2.673648in}{2.073893in}}%
\pgfpathcurveto{\pgfqpoint{2.681462in}{2.066080in}}{\pgfqpoint{2.692061in}{2.061689in}}{\pgfqpoint{2.703111in}{2.061689in}}%
\pgfpathclose%
\pgfusepath{stroke,fill}%
\end{pgfscope}%
\begin{pgfscope}%
\pgfpathrectangle{\pgfqpoint{0.511823in}{0.504323in}}{\pgfqpoint{3.218177in}{3.225677in}} %
\pgfusepath{clip}%
\pgfsetbuttcap%
\pgfsetroundjoin%
\definecolor{currentfill}{rgb}{0.501961,0.000000,0.000000}%
\pgfsetfillcolor{currentfill}%
\pgfsetfillopacity{0.400000}%
\pgfsetlinewidth{0.501875pt}%
\definecolor{currentstroke}{rgb}{0.501961,0.000000,0.000000}%
\pgfsetstrokecolor{currentstroke}%
\pgfsetstrokeopacity{0.400000}%
\pgfsetdash{}{0pt}%
\pgfpathmoveto{\pgfqpoint{2.841367in}{2.169131in}}%
\pgfpathcurveto{\pgfqpoint{2.852417in}{2.169131in}}{\pgfqpoint{2.863016in}{2.173522in}}{\pgfqpoint{2.870829in}{2.181335in}}%
\pgfpathcurveto{\pgfqpoint{2.878643in}{2.189149in}}{\pgfqpoint{2.883033in}{2.199748in}}{\pgfqpoint{2.883033in}{2.210798in}}%
\pgfpathcurveto{\pgfqpoint{2.883033in}{2.221848in}}{\pgfqpoint{2.878643in}{2.232447in}}{\pgfqpoint{2.870829in}{2.240261in}}%
\pgfpathcurveto{\pgfqpoint{2.863016in}{2.248074in}}{\pgfqpoint{2.852417in}{2.252465in}}{\pgfqpoint{2.841367in}{2.252465in}}%
\pgfpathcurveto{\pgfqpoint{2.830316in}{2.252465in}}{\pgfqpoint{2.819717in}{2.248074in}}{\pgfqpoint{2.811904in}{2.240261in}}%
\pgfpathcurveto{\pgfqpoint{2.804090in}{2.232447in}}{\pgfqpoint{2.799700in}{2.221848in}}{\pgfqpoint{2.799700in}{2.210798in}}%
\pgfpathcurveto{\pgfqpoint{2.799700in}{2.199748in}}{\pgfqpoint{2.804090in}{2.189149in}}{\pgfqpoint{2.811904in}{2.181335in}}%
\pgfpathcurveto{\pgfqpoint{2.819717in}{2.173522in}}{\pgfqpoint{2.830316in}{2.169131in}}{\pgfqpoint{2.841367in}{2.169131in}}%
\pgfpathclose%
\pgfusepath{stroke,fill}%
\end{pgfscope}%
\begin{pgfscope}%
\pgfpathrectangle{\pgfqpoint{0.511823in}{0.504323in}}{\pgfqpoint{3.218177in}{3.225677in}} %
\pgfusepath{clip}%
\pgfsetbuttcap%
\pgfsetroundjoin%
\definecolor{currentfill}{rgb}{0.501961,0.000000,0.000000}%
\pgfsetfillcolor{currentfill}%
\pgfsetfillopacity{0.400000}%
\pgfsetlinewidth{0.501875pt}%
\definecolor{currentstroke}{rgb}{0.501961,0.000000,0.000000}%
\pgfsetstrokecolor{currentstroke}%
\pgfsetstrokeopacity{0.400000}%
\pgfsetdash{}{0pt}%
\pgfpathmoveto{\pgfqpoint{2.846933in}{2.183116in}}%
\pgfpathcurveto{\pgfqpoint{2.857984in}{2.183116in}}{\pgfqpoint{2.868583in}{2.187506in}}{\pgfqpoint{2.876396in}{2.195320in}}%
\pgfpathcurveto{\pgfqpoint{2.884210in}{2.203133in}}{\pgfqpoint{2.888600in}{2.213732in}}{\pgfqpoint{2.888600in}{2.224783in}}%
\pgfpathcurveto{\pgfqpoint{2.888600in}{2.235833in}}{\pgfqpoint{2.884210in}{2.246432in}}{\pgfqpoint{2.876396in}{2.254245in}}%
\pgfpathcurveto{\pgfqpoint{2.868583in}{2.262059in}}{\pgfqpoint{2.857984in}{2.266449in}}{\pgfqpoint{2.846933in}{2.266449in}}%
\pgfpathcurveto{\pgfqpoint{2.835883in}{2.266449in}}{\pgfqpoint{2.825284in}{2.262059in}}{\pgfqpoint{2.817471in}{2.254245in}}%
\pgfpathcurveto{\pgfqpoint{2.809657in}{2.246432in}}{\pgfqpoint{2.805267in}{2.235833in}}{\pgfqpoint{2.805267in}{2.224783in}}%
\pgfpathcurveto{\pgfqpoint{2.805267in}{2.213732in}}{\pgfqpoint{2.809657in}{2.203133in}}{\pgfqpoint{2.817471in}{2.195320in}}%
\pgfpathcurveto{\pgfqpoint{2.825284in}{2.187506in}}{\pgfqpoint{2.835883in}{2.183116in}}{\pgfqpoint{2.846933in}{2.183116in}}%
\pgfpathclose%
\pgfusepath{stroke,fill}%
\end{pgfscope}%
\begin{pgfscope}%
\pgfpathrectangle{\pgfqpoint{0.511823in}{0.504323in}}{\pgfqpoint{3.218177in}{3.225677in}} %
\pgfusepath{clip}%
\pgfsetbuttcap%
\pgfsetroundjoin%
\definecolor{currentfill}{rgb}{0.501961,0.000000,0.000000}%
\pgfsetfillcolor{currentfill}%
\pgfsetfillopacity{0.400000}%
\pgfsetlinewidth{0.501875pt}%
\definecolor{currentstroke}{rgb}{0.501961,0.000000,0.000000}%
\pgfsetstrokecolor{currentstroke}%
\pgfsetstrokeopacity{0.400000}%
\pgfsetdash{}{0pt}%
\pgfpathmoveto{\pgfqpoint{2.799915in}{2.159391in}}%
\pgfpathcurveto{\pgfqpoint{2.810966in}{2.159391in}}{\pgfqpoint{2.821565in}{2.163782in}}{\pgfqpoint{2.829378in}{2.171595in}}%
\pgfpathcurveto{\pgfqpoint{2.837192in}{2.179409in}}{\pgfqpoint{2.841582in}{2.190008in}}{\pgfqpoint{2.841582in}{2.201058in}}%
\pgfpathcurveto{\pgfqpoint{2.841582in}{2.212108in}}{\pgfqpoint{2.837192in}{2.222707in}}{\pgfqpoint{2.829378in}{2.230521in}}%
\pgfpathcurveto{\pgfqpoint{2.821565in}{2.238334in}}{\pgfqpoint{2.810966in}{2.242725in}}{\pgfqpoint{2.799915in}{2.242725in}}%
\pgfpathcurveto{\pgfqpoint{2.788865in}{2.242725in}}{\pgfqpoint{2.778266in}{2.238334in}}{\pgfqpoint{2.770453in}{2.230521in}}%
\pgfpathcurveto{\pgfqpoint{2.762639in}{2.222707in}}{\pgfqpoint{2.758249in}{2.212108in}}{\pgfqpoint{2.758249in}{2.201058in}}%
\pgfpathcurveto{\pgfqpoint{2.758249in}{2.190008in}}{\pgfqpoint{2.762639in}{2.179409in}}{\pgfqpoint{2.770453in}{2.171595in}}%
\pgfpathcurveto{\pgfqpoint{2.778266in}{2.163782in}}{\pgfqpoint{2.788865in}{2.159391in}}{\pgfqpoint{2.799915in}{2.159391in}}%
\pgfpathclose%
\pgfusepath{stroke,fill}%
\end{pgfscope}%
\begin{pgfscope}%
\pgfpathrectangle{\pgfqpoint{0.511823in}{0.504323in}}{\pgfqpoint{3.218177in}{3.225677in}} %
\pgfusepath{clip}%
\pgfsetbuttcap%
\pgfsetroundjoin%
\definecolor{currentfill}{rgb}{0.501961,0.000000,0.000000}%
\pgfsetfillcolor{currentfill}%
\pgfsetfillopacity{0.400000}%
\pgfsetlinewidth{0.501875pt}%
\definecolor{currentstroke}{rgb}{0.501961,0.000000,0.000000}%
\pgfsetstrokecolor{currentstroke}%
\pgfsetstrokeopacity{0.400000}%
\pgfsetdash{}{0pt}%
\pgfpathmoveto{\pgfqpoint{2.717595in}{2.109727in}}%
\pgfpathcurveto{\pgfqpoint{2.728645in}{2.109727in}}{\pgfqpoint{2.739244in}{2.114117in}}{\pgfqpoint{2.747058in}{2.121931in}}%
\pgfpathcurveto{\pgfqpoint{2.754872in}{2.129744in}}{\pgfqpoint{2.759262in}{2.140343in}}{\pgfqpoint{2.759262in}{2.151393in}}%
\pgfpathcurveto{\pgfqpoint{2.759262in}{2.162443in}}{\pgfqpoint{2.754872in}{2.173043in}}{\pgfqpoint{2.747058in}{2.180856in}}%
\pgfpathcurveto{\pgfqpoint{2.739244in}{2.188670in}}{\pgfqpoint{2.728645in}{2.193060in}}{\pgfqpoint{2.717595in}{2.193060in}}%
\pgfpathcurveto{\pgfqpoint{2.706545in}{2.193060in}}{\pgfqpoint{2.695946in}{2.188670in}}{\pgfqpoint{2.688133in}{2.180856in}}%
\pgfpathcurveto{\pgfqpoint{2.680319in}{2.173043in}}{\pgfqpoint{2.675929in}{2.162443in}}{\pgfqpoint{2.675929in}{2.151393in}}%
\pgfpathcurveto{\pgfqpoint{2.675929in}{2.140343in}}{\pgfqpoint{2.680319in}{2.129744in}}{\pgfqpoint{2.688133in}{2.121931in}}%
\pgfpathcurveto{\pgfqpoint{2.695946in}{2.114117in}}{\pgfqpoint{2.706545in}{2.109727in}}{\pgfqpoint{2.717595in}{2.109727in}}%
\pgfpathclose%
\pgfusepath{stroke,fill}%
\end{pgfscope}%
\begin{pgfscope}%
\pgfpathrectangle{\pgfqpoint{0.511823in}{0.504323in}}{\pgfqpoint{3.218177in}{3.225677in}} %
\pgfusepath{clip}%
\pgfsetbuttcap%
\pgfsetroundjoin%
\definecolor{currentfill}{rgb}{0.501961,0.000000,0.000000}%
\pgfsetfillcolor{currentfill}%
\pgfsetfillopacity{0.400000}%
\pgfsetlinewidth{0.501875pt}%
\definecolor{currentstroke}{rgb}{0.501961,0.000000,0.000000}%
\pgfsetstrokecolor{currentstroke}%
\pgfsetstrokeopacity{0.400000}%
\pgfsetdash{}{0pt}%
\pgfpathmoveto{\pgfqpoint{2.780133in}{2.164852in}}%
\pgfpathcurveto{\pgfqpoint{2.791183in}{2.164852in}}{\pgfqpoint{2.801782in}{2.169243in}}{\pgfqpoint{2.809596in}{2.177056in}}%
\pgfpathcurveto{\pgfqpoint{2.817410in}{2.184870in}}{\pgfqpoint{2.821800in}{2.195469in}}{\pgfqpoint{2.821800in}{2.206519in}}%
\pgfpathcurveto{\pgfqpoint{2.821800in}{2.217569in}}{\pgfqpoint{2.817410in}{2.228168in}}{\pgfqpoint{2.809596in}{2.235982in}}%
\pgfpathcurveto{\pgfqpoint{2.801782in}{2.243795in}}{\pgfqpoint{2.791183in}{2.248186in}}{\pgfqpoint{2.780133in}{2.248186in}}%
\pgfpathcurveto{\pgfqpoint{2.769083in}{2.248186in}}{\pgfqpoint{2.758484in}{2.243795in}}{\pgfqpoint{2.750670in}{2.235982in}}%
\pgfpathcurveto{\pgfqpoint{2.742857in}{2.228168in}}{\pgfqpoint{2.738466in}{2.217569in}}{\pgfqpoint{2.738466in}{2.206519in}}%
\pgfpathcurveto{\pgfqpoint{2.738466in}{2.195469in}}{\pgfqpoint{2.742857in}{2.184870in}}{\pgfqpoint{2.750670in}{2.177056in}}%
\pgfpathcurveto{\pgfqpoint{2.758484in}{2.169243in}}{\pgfqpoint{2.769083in}{2.164852in}}{\pgfqpoint{2.780133in}{2.164852in}}%
\pgfpathclose%
\pgfusepath{stroke,fill}%
\end{pgfscope}%
\begin{pgfscope}%
\pgfpathrectangle{\pgfqpoint{0.511823in}{0.504323in}}{\pgfqpoint{3.218177in}{3.225677in}} %
\pgfusepath{clip}%
\pgfsetbuttcap%
\pgfsetroundjoin%
\definecolor{currentfill}{rgb}{0.501961,0.000000,0.000000}%
\pgfsetfillcolor{currentfill}%
\pgfsetfillopacity{0.400000}%
\pgfsetlinewidth{0.501875pt}%
\definecolor{currentstroke}{rgb}{0.501961,0.000000,0.000000}%
\pgfsetstrokecolor{currentstroke}%
\pgfsetstrokeopacity{0.400000}%
\pgfsetdash{}{0pt}%
\pgfpathmoveto{\pgfqpoint{2.779365in}{2.174205in}}%
\pgfpathcurveto{\pgfqpoint{2.790415in}{2.174205in}}{\pgfqpoint{2.801014in}{2.178595in}}{\pgfqpoint{2.808828in}{2.186408in}}%
\pgfpathcurveto{\pgfqpoint{2.816642in}{2.194222in}}{\pgfqpoint{2.821032in}{2.204821in}}{\pgfqpoint{2.821032in}{2.215871in}}%
\pgfpathcurveto{\pgfqpoint{2.821032in}{2.226921in}}{\pgfqpoint{2.816642in}{2.237520in}}{\pgfqpoint{2.808828in}{2.245334in}}%
\pgfpathcurveto{\pgfqpoint{2.801014in}{2.253148in}}{\pgfqpoint{2.790415in}{2.257538in}}{\pgfqpoint{2.779365in}{2.257538in}}%
\pgfpathcurveto{\pgfqpoint{2.768315in}{2.257538in}}{\pgfqpoint{2.757716in}{2.253148in}}{\pgfqpoint{2.749903in}{2.245334in}}%
\pgfpathcurveto{\pgfqpoint{2.742089in}{2.237520in}}{\pgfqpoint{2.737699in}{2.226921in}}{\pgfqpoint{2.737699in}{2.215871in}}%
\pgfpathcurveto{\pgfqpoint{2.737699in}{2.204821in}}{\pgfqpoint{2.742089in}{2.194222in}}{\pgfqpoint{2.749903in}{2.186408in}}%
\pgfpathcurveto{\pgfqpoint{2.757716in}{2.178595in}}{\pgfqpoint{2.768315in}{2.174205in}}{\pgfqpoint{2.779365in}{2.174205in}}%
\pgfpathclose%
\pgfusepath{stroke,fill}%
\end{pgfscope}%
\begin{pgfscope}%
\pgfpathrectangle{\pgfqpoint{0.511823in}{0.504323in}}{\pgfqpoint{3.218177in}{3.225677in}} %
\pgfusepath{clip}%
\pgfsetbuttcap%
\pgfsetroundjoin%
\definecolor{currentfill}{rgb}{0.501961,0.000000,0.000000}%
\pgfsetfillcolor{currentfill}%
\pgfsetfillopacity{0.400000}%
\pgfsetlinewidth{0.501875pt}%
\definecolor{currentstroke}{rgb}{0.501961,0.000000,0.000000}%
\pgfsetstrokecolor{currentstroke}%
\pgfsetstrokeopacity{0.400000}%
\pgfsetdash{}{0pt}%
\pgfpathmoveto{\pgfqpoint{2.793081in}{2.194287in}}%
\pgfpathcurveto{\pgfqpoint{2.804131in}{2.194287in}}{\pgfqpoint{2.814730in}{2.198677in}}{\pgfqpoint{2.822544in}{2.206491in}}%
\pgfpathcurveto{\pgfqpoint{2.830358in}{2.214304in}}{\pgfqpoint{2.834748in}{2.224903in}}{\pgfqpoint{2.834748in}{2.235953in}}%
\pgfpathcurveto{\pgfqpoint{2.834748in}{2.247003in}}{\pgfqpoint{2.830358in}{2.257602in}}{\pgfqpoint{2.822544in}{2.265416in}}%
\pgfpathcurveto{\pgfqpoint{2.814730in}{2.273230in}}{\pgfqpoint{2.804131in}{2.277620in}}{\pgfqpoint{2.793081in}{2.277620in}}%
\pgfpathcurveto{\pgfqpoint{2.782031in}{2.277620in}}{\pgfqpoint{2.771432in}{2.273230in}}{\pgfqpoint{2.763618in}{2.265416in}}%
\pgfpathcurveto{\pgfqpoint{2.755805in}{2.257602in}}{\pgfqpoint{2.751415in}{2.247003in}}{\pgfqpoint{2.751415in}{2.235953in}}%
\pgfpathcurveto{\pgfqpoint{2.751415in}{2.224903in}}{\pgfqpoint{2.755805in}{2.214304in}}{\pgfqpoint{2.763618in}{2.206491in}}%
\pgfpathcurveto{\pgfqpoint{2.771432in}{2.198677in}}{\pgfqpoint{2.782031in}{2.194287in}}{\pgfqpoint{2.793081in}{2.194287in}}%
\pgfpathclose%
\pgfusepath{stroke,fill}%
\end{pgfscope}%
\begin{pgfscope}%
\pgfpathrectangle{\pgfqpoint{0.511823in}{0.504323in}}{\pgfqpoint{3.218177in}{3.225677in}} %
\pgfusepath{clip}%
\pgfsetbuttcap%
\pgfsetroundjoin%
\definecolor{currentfill}{rgb}{0.501961,0.000000,0.000000}%
\pgfsetfillcolor{currentfill}%
\pgfsetfillopacity{0.400000}%
\pgfsetlinewidth{0.501875pt}%
\definecolor{currentstroke}{rgb}{0.501961,0.000000,0.000000}%
\pgfsetstrokecolor{currentstroke}%
\pgfsetstrokeopacity{0.400000}%
\pgfsetdash{}{0pt}%
\pgfpathmoveto{\pgfqpoint{2.824607in}{2.227777in}}%
\pgfpathcurveto{\pgfqpoint{2.835657in}{2.227777in}}{\pgfqpoint{2.846256in}{2.232168in}}{\pgfqpoint{2.854070in}{2.239981in}}%
\pgfpathcurveto{\pgfqpoint{2.861884in}{2.247795in}}{\pgfqpoint{2.866274in}{2.258394in}}{\pgfqpoint{2.866274in}{2.269444in}}%
\pgfpathcurveto{\pgfqpoint{2.866274in}{2.280494in}}{\pgfqpoint{2.861884in}{2.291093in}}{\pgfqpoint{2.854070in}{2.298907in}}%
\pgfpathcurveto{\pgfqpoint{2.846256in}{2.306720in}}{\pgfqpoint{2.835657in}{2.311111in}}{\pgfqpoint{2.824607in}{2.311111in}}%
\pgfpathcurveto{\pgfqpoint{2.813557in}{2.311111in}}{\pgfqpoint{2.802958in}{2.306720in}}{\pgfqpoint{2.795144in}{2.298907in}}%
\pgfpathcurveto{\pgfqpoint{2.787331in}{2.291093in}}{\pgfqpoint{2.782941in}{2.280494in}}{\pgfqpoint{2.782941in}{2.269444in}}%
\pgfpathcurveto{\pgfqpoint{2.782941in}{2.258394in}}{\pgfqpoint{2.787331in}{2.247795in}}{\pgfqpoint{2.795144in}{2.239981in}}%
\pgfpathcurveto{\pgfqpoint{2.802958in}{2.232168in}}{\pgfqpoint{2.813557in}{2.227777in}}{\pgfqpoint{2.824607in}{2.227777in}}%
\pgfpathclose%
\pgfusepath{stroke,fill}%
\end{pgfscope}%
\begin{pgfscope}%
\pgfpathrectangle{\pgfqpoint{0.511823in}{0.504323in}}{\pgfqpoint{3.218177in}{3.225677in}} %
\pgfusepath{clip}%
\pgfsetbuttcap%
\pgfsetroundjoin%
\definecolor{currentfill}{rgb}{0.501961,0.000000,0.000000}%
\pgfsetfillcolor{currentfill}%
\pgfsetfillopacity{0.400000}%
\pgfsetlinewidth{0.501875pt}%
\definecolor{currentstroke}{rgb}{0.501961,0.000000,0.000000}%
\pgfsetstrokecolor{currentstroke}%
\pgfsetstrokeopacity{0.400000}%
\pgfsetdash{}{0pt}%
\pgfpathmoveto{\pgfqpoint{2.752094in}{2.183852in}}%
\pgfpathcurveto{\pgfqpoint{2.763144in}{2.183852in}}{\pgfqpoint{2.773743in}{2.188242in}}{\pgfqpoint{2.781556in}{2.196056in}}%
\pgfpathcurveto{\pgfqpoint{2.789370in}{2.203869in}}{\pgfqpoint{2.793760in}{2.214468in}}{\pgfqpoint{2.793760in}{2.225518in}}%
\pgfpathcurveto{\pgfqpoint{2.793760in}{2.236568in}}{\pgfqpoint{2.789370in}{2.247168in}}{\pgfqpoint{2.781556in}{2.254981in}}%
\pgfpathcurveto{\pgfqpoint{2.773743in}{2.262795in}}{\pgfqpoint{2.763144in}{2.267185in}}{\pgfqpoint{2.752094in}{2.267185in}}%
\pgfpathcurveto{\pgfqpoint{2.741044in}{2.267185in}}{\pgfqpoint{2.730445in}{2.262795in}}{\pgfqpoint{2.722631in}{2.254981in}}%
\pgfpathcurveto{\pgfqpoint{2.714817in}{2.247168in}}{\pgfqpoint{2.710427in}{2.236568in}}{\pgfqpoint{2.710427in}{2.225518in}}%
\pgfpathcurveto{\pgfqpoint{2.710427in}{2.214468in}}{\pgfqpoint{2.714817in}{2.203869in}}{\pgfqpoint{2.722631in}{2.196056in}}%
\pgfpathcurveto{\pgfqpoint{2.730445in}{2.188242in}}{\pgfqpoint{2.741044in}{2.183852in}}{\pgfqpoint{2.752094in}{2.183852in}}%
\pgfpathclose%
\pgfusepath{stroke,fill}%
\end{pgfscope}%
\begin{pgfscope}%
\pgfpathrectangle{\pgfqpoint{0.511823in}{0.504323in}}{\pgfqpoint{3.218177in}{3.225677in}} %
\pgfusepath{clip}%
\pgfsetbuttcap%
\pgfsetroundjoin%
\definecolor{currentfill}{rgb}{0.501961,0.000000,0.000000}%
\pgfsetfillcolor{currentfill}%
\pgfsetfillopacity{0.400000}%
\pgfsetlinewidth{0.501875pt}%
\definecolor{currentstroke}{rgb}{0.501961,0.000000,0.000000}%
\pgfsetstrokecolor{currentstroke}%
\pgfsetstrokeopacity{0.400000}%
\pgfsetdash{}{0pt}%
\pgfpathmoveto{\pgfqpoint{2.600538in}{2.079811in}}%
\pgfpathcurveto{\pgfqpoint{2.611589in}{2.079811in}}{\pgfqpoint{2.622188in}{2.084202in}}{\pgfqpoint{2.630001in}{2.092015in}}%
\pgfpathcurveto{\pgfqpoint{2.637815in}{2.099829in}}{\pgfqpoint{2.642205in}{2.110428in}}{\pgfqpoint{2.642205in}{2.121478in}}%
\pgfpathcurveto{\pgfqpoint{2.642205in}{2.132528in}}{\pgfqpoint{2.637815in}{2.143127in}}{\pgfqpoint{2.630001in}{2.150941in}}%
\pgfpathcurveto{\pgfqpoint{2.622188in}{2.158754in}}{\pgfqpoint{2.611589in}{2.163145in}}{\pgfqpoint{2.600538in}{2.163145in}}%
\pgfpathcurveto{\pgfqpoint{2.589488in}{2.163145in}}{\pgfqpoint{2.578889in}{2.158754in}}{\pgfqpoint{2.571076in}{2.150941in}}%
\pgfpathcurveto{\pgfqpoint{2.563262in}{2.143127in}}{\pgfqpoint{2.558872in}{2.132528in}}{\pgfqpoint{2.558872in}{2.121478in}}%
\pgfpathcurveto{\pgfqpoint{2.558872in}{2.110428in}}{\pgfqpoint{2.563262in}{2.099829in}}{\pgfqpoint{2.571076in}{2.092015in}}%
\pgfpathcurveto{\pgfqpoint{2.578889in}{2.084202in}}{\pgfqpoint{2.589488in}{2.079811in}}{\pgfqpoint{2.600538in}{2.079811in}}%
\pgfpathclose%
\pgfusepath{stroke,fill}%
\end{pgfscope}%
\begin{pgfscope}%
\pgfpathrectangle{\pgfqpoint{0.511823in}{0.504323in}}{\pgfqpoint{3.218177in}{3.225677in}} %
\pgfusepath{clip}%
\pgfsetbuttcap%
\pgfsetroundjoin%
\definecolor{currentfill}{rgb}{0.501961,0.000000,0.000000}%
\pgfsetfillcolor{currentfill}%
\pgfsetfillopacity{0.400000}%
\pgfsetlinewidth{0.501875pt}%
\definecolor{currentstroke}{rgb}{0.501961,0.000000,0.000000}%
\pgfsetstrokecolor{currentstroke}%
\pgfsetstrokeopacity{0.400000}%
\pgfsetdash{}{0pt}%
\pgfpathmoveto{\pgfqpoint{2.716182in}{2.176671in}}%
\pgfpathcurveto{\pgfqpoint{2.727232in}{2.176671in}}{\pgfqpoint{2.737831in}{2.181061in}}{\pgfqpoint{2.745645in}{2.188875in}}%
\pgfpathcurveto{\pgfqpoint{2.753458in}{2.196689in}}{\pgfqpoint{2.757848in}{2.207288in}}{\pgfqpoint{2.757848in}{2.218338in}}%
\pgfpathcurveto{\pgfqpoint{2.757848in}{2.229388in}}{\pgfqpoint{2.753458in}{2.239987in}}{\pgfqpoint{2.745645in}{2.247801in}}%
\pgfpathcurveto{\pgfqpoint{2.737831in}{2.255614in}}{\pgfqpoint{2.727232in}{2.260005in}}{\pgfqpoint{2.716182in}{2.260005in}}%
\pgfpathcurveto{\pgfqpoint{2.705132in}{2.260005in}}{\pgfqpoint{2.694533in}{2.255614in}}{\pgfqpoint{2.686719in}{2.247801in}}%
\pgfpathcurveto{\pgfqpoint{2.678905in}{2.239987in}}{\pgfqpoint{2.674515in}{2.229388in}}{\pgfqpoint{2.674515in}{2.218338in}}%
\pgfpathcurveto{\pgfqpoint{2.674515in}{2.207288in}}{\pgfqpoint{2.678905in}{2.196689in}}{\pgfqpoint{2.686719in}{2.188875in}}%
\pgfpathcurveto{\pgfqpoint{2.694533in}{2.181061in}}{\pgfqpoint{2.705132in}{2.176671in}}{\pgfqpoint{2.716182in}{2.176671in}}%
\pgfpathclose%
\pgfusepath{stroke,fill}%
\end{pgfscope}%
\begin{pgfscope}%
\pgfpathrectangle{\pgfqpoint{0.511823in}{0.504323in}}{\pgfqpoint{3.218177in}{3.225677in}} %
\pgfusepath{clip}%
\pgfsetbuttcap%
\pgfsetroundjoin%
\definecolor{currentfill}{rgb}{0.501961,0.000000,0.000000}%
\pgfsetfillcolor{currentfill}%
\pgfsetfillopacity{0.400000}%
\pgfsetlinewidth{0.501875pt}%
\definecolor{currentstroke}{rgb}{0.501961,0.000000,0.000000}%
\pgfsetstrokecolor{currentstroke}%
\pgfsetstrokeopacity{0.400000}%
\pgfsetdash{}{0pt}%
\pgfpathmoveto{\pgfqpoint{2.777187in}{2.233069in}}%
\pgfpathcurveto{\pgfqpoint{2.788237in}{2.233069in}}{\pgfqpoint{2.798836in}{2.237460in}}{\pgfqpoint{2.806650in}{2.245273in}}%
\pgfpathcurveto{\pgfqpoint{2.814463in}{2.253087in}}{\pgfqpoint{2.818854in}{2.263686in}}{\pgfqpoint{2.818854in}{2.274736in}}%
\pgfpathcurveto{\pgfqpoint{2.818854in}{2.285786in}}{\pgfqpoint{2.814463in}{2.296385in}}{\pgfqpoint{2.806650in}{2.304199in}}%
\pgfpathcurveto{\pgfqpoint{2.798836in}{2.312012in}}{\pgfqpoint{2.788237in}{2.316403in}}{\pgfqpoint{2.777187in}{2.316403in}}%
\pgfpathcurveto{\pgfqpoint{2.766137in}{2.316403in}}{\pgfqpoint{2.755538in}{2.312012in}}{\pgfqpoint{2.747724in}{2.304199in}}%
\pgfpathcurveto{\pgfqpoint{2.739911in}{2.296385in}}{\pgfqpoint{2.735520in}{2.285786in}}{\pgfqpoint{2.735520in}{2.274736in}}%
\pgfpathcurveto{\pgfqpoint{2.735520in}{2.263686in}}{\pgfqpoint{2.739911in}{2.253087in}}{\pgfqpoint{2.747724in}{2.245273in}}%
\pgfpathcurveto{\pgfqpoint{2.755538in}{2.237460in}}{\pgfqpoint{2.766137in}{2.233069in}}{\pgfqpoint{2.777187in}{2.233069in}}%
\pgfpathclose%
\pgfusepath{stroke,fill}%
\end{pgfscope}%
\begin{pgfscope}%
\pgfpathrectangle{\pgfqpoint{0.511823in}{0.504323in}}{\pgfqpoint{3.218177in}{3.225677in}} %
\pgfusepath{clip}%
\pgfsetbuttcap%
\pgfsetroundjoin%
\definecolor{currentfill}{rgb}{0.501961,0.000000,0.000000}%
\pgfsetfillcolor{currentfill}%
\pgfsetfillopacity{0.400000}%
\pgfsetlinewidth{0.501875pt}%
\definecolor{currentstroke}{rgb}{0.501961,0.000000,0.000000}%
\pgfsetstrokecolor{currentstroke}%
\pgfsetstrokeopacity{0.400000}%
\pgfsetdash{}{0pt}%
\pgfpathmoveto{\pgfqpoint{2.688789in}{2.175528in}}%
\pgfpathcurveto{\pgfqpoint{2.699839in}{2.175528in}}{\pgfqpoint{2.710438in}{2.179918in}}{\pgfqpoint{2.718252in}{2.187732in}}%
\pgfpathcurveto{\pgfqpoint{2.726066in}{2.195546in}}{\pgfqpoint{2.730456in}{2.206145in}}{\pgfqpoint{2.730456in}{2.217195in}}%
\pgfpathcurveto{\pgfqpoint{2.730456in}{2.228245in}}{\pgfqpoint{2.726066in}{2.238844in}}{\pgfqpoint{2.718252in}{2.246657in}}%
\pgfpathcurveto{\pgfqpoint{2.710438in}{2.254471in}}{\pgfqpoint{2.699839in}{2.258861in}}{\pgfqpoint{2.688789in}{2.258861in}}%
\pgfpathcurveto{\pgfqpoint{2.677739in}{2.258861in}}{\pgfqpoint{2.667140in}{2.254471in}}{\pgfqpoint{2.659326in}{2.246657in}}%
\pgfpathcurveto{\pgfqpoint{2.651513in}{2.238844in}}{\pgfqpoint{2.647123in}{2.228245in}}{\pgfqpoint{2.647123in}{2.217195in}}%
\pgfpathcurveto{\pgfqpoint{2.647123in}{2.206145in}}{\pgfqpoint{2.651513in}{2.195546in}}{\pgfqpoint{2.659326in}{2.187732in}}%
\pgfpathcurveto{\pgfqpoint{2.667140in}{2.179918in}}{\pgfqpoint{2.677739in}{2.175528in}}{\pgfqpoint{2.688789in}{2.175528in}}%
\pgfpathclose%
\pgfusepath{stroke,fill}%
\end{pgfscope}%
\begin{pgfscope}%
\pgfpathrectangle{\pgfqpoint{0.511823in}{0.504323in}}{\pgfqpoint{3.218177in}{3.225677in}} %
\pgfusepath{clip}%
\pgfsetbuttcap%
\pgfsetroundjoin%
\definecolor{currentfill}{rgb}{0.501961,0.000000,0.000000}%
\pgfsetfillcolor{currentfill}%
\pgfsetfillopacity{0.400000}%
\pgfsetlinewidth{0.501875pt}%
\definecolor{currentstroke}{rgb}{0.501961,0.000000,0.000000}%
\pgfsetstrokecolor{currentstroke}%
\pgfsetstrokeopacity{0.400000}%
\pgfsetdash{}{0pt}%
\pgfpathmoveto{\pgfqpoint{2.750742in}{2.233226in}}%
\pgfpathcurveto{\pgfqpoint{2.761792in}{2.233226in}}{\pgfqpoint{2.772391in}{2.237616in}}{\pgfqpoint{2.780205in}{2.245430in}}%
\pgfpathcurveto{\pgfqpoint{2.788019in}{2.253244in}}{\pgfqpoint{2.792409in}{2.263843in}}{\pgfqpoint{2.792409in}{2.274893in}}%
\pgfpathcurveto{\pgfqpoint{2.792409in}{2.285943in}}{\pgfqpoint{2.788019in}{2.296542in}}{\pgfqpoint{2.780205in}{2.304355in}}%
\pgfpathcurveto{\pgfqpoint{2.772391in}{2.312169in}}{\pgfqpoint{2.761792in}{2.316559in}}{\pgfqpoint{2.750742in}{2.316559in}}%
\pgfpathcurveto{\pgfqpoint{2.739692in}{2.316559in}}{\pgfqpoint{2.729093in}{2.312169in}}{\pgfqpoint{2.721280in}{2.304355in}}%
\pgfpathcurveto{\pgfqpoint{2.713466in}{2.296542in}}{\pgfqpoint{2.709076in}{2.285943in}}{\pgfqpoint{2.709076in}{2.274893in}}%
\pgfpathcurveto{\pgfqpoint{2.709076in}{2.263843in}}{\pgfqpoint{2.713466in}{2.253244in}}{\pgfqpoint{2.721280in}{2.245430in}}%
\pgfpathcurveto{\pgfqpoint{2.729093in}{2.237616in}}{\pgfqpoint{2.739692in}{2.233226in}}{\pgfqpoint{2.750742in}{2.233226in}}%
\pgfpathclose%
\pgfusepath{stroke,fill}%
\end{pgfscope}%
\begin{pgfscope}%
\pgfpathrectangle{\pgfqpoint{0.511823in}{0.504323in}}{\pgfqpoint{3.218177in}{3.225677in}} %
\pgfusepath{clip}%
\pgfsetbuttcap%
\pgfsetroundjoin%
\definecolor{currentfill}{rgb}{0.501961,0.000000,0.000000}%
\pgfsetfillcolor{currentfill}%
\pgfsetfillopacity{0.400000}%
\pgfsetlinewidth{0.501875pt}%
\definecolor{currentstroke}{rgb}{0.501961,0.000000,0.000000}%
\pgfsetstrokecolor{currentstroke}%
\pgfsetstrokeopacity{0.400000}%
\pgfsetdash{}{0pt}%
\pgfpathmoveto{\pgfqpoint{2.653388in}{2.167829in}}%
\pgfpathcurveto{\pgfqpoint{2.664439in}{2.167829in}}{\pgfqpoint{2.675038in}{2.172219in}}{\pgfqpoint{2.682851in}{2.180033in}}%
\pgfpathcurveto{\pgfqpoint{2.690665in}{2.187846in}}{\pgfqpoint{2.695055in}{2.198445in}}{\pgfqpoint{2.695055in}{2.209496in}}%
\pgfpathcurveto{\pgfqpoint{2.695055in}{2.220546in}}{\pgfqpoint{2.690665in}{2.231145in}}{\pgfqpoint{2.682851in}{2.238958in}}%
\pgfpathcurveto{\pgfqpoint{2.675038in}{2.246772in}}{\pgfqpoint{2.664439in}{2.251162in}}{\pgfqpoint{2.653388in}{2.251162in}}%
\pgfpathcurveto{\pgfqpoint{2.642338in}{2.251162in}}{\pgfqpoint{2.631739in}{2.246772in}}{\pgfqpoint{2.623926in}{2.238958in}}%
\pgfpathcurveto{\pgfqpoint{2.616112in}{2.231145in}}{\pgfqpoint{2.611722in}{2.220546in}}{\pgfqpoint{2.611722in}{2.209496in}}%
\pgfpathcurveto{\pgfqpoint{2.611722in}{2.198445in}}{\pgfqpoint{2.616112in}{2.187846in}}{\pgfqpoint{2.623926in}{2.180033in}}%
\pgfpathcurveto{\pgfqpoint{2.631739in}{2.172219in}}{\pgfqpoint{2.642338in}{2.167829in}}{\pgfqpoint{2.653388in}{2.167829in}}%
\pgfpathclose%
\pgfusepath{stroke,fill}%
\end{pgfscope}%
\begin{pgfscope}%
\pgfpathrectangle{\pgfqpoint{0.511823in}{0.504323in}}{\pgfqpoint{3.218177in}{3.225677in}} %
\pgfusepath{clip}%
\pgfsetbuttcap%
\pgfsetroundjoin%
\definecolor{currentfill}{rgb}{0.501961,0.000000,0.000000}%
\pgfsetfillcolor{currentfill}%
\pgfsetfillopacity{0.400000}%
\pgfsetlinewidth{0.501875pt}%
\definecolor{currentstroke}{rgb}{0.501961,0.000000,0.000000}%
\pgfsetstrokecolor{currentstroke}%
\pgfsetstrokeopacity{0.400000}%
\pgfsetdash{}{0pt}%
\pgfpathmoveto{\pgfqpoint{2.854170in}{2.334615in}}%
\pgfpathcurveto{\pgfqpoint{2.865220in}{2.334615in}}{\pgfqpoint{2.875819in}{2.339005in}}{\pgfqpoint{2.883633in}{2.346818in}}%
\pgfpathcurveto{\pgfqpoint{2.891447in}{2.354632in}}{\pgfqpoint{2.895837in}{2.365231in}}{\pgfqpoint{2.895837in}{2.376281in}}%
\pgfpathcurveto{\pgfqpoint{2.895837in}{2.387331in}}{\pgfqpoint{2.891447in}{2.397930in}}{\pgfqpoint{2.883633in}{2.405744in}}%
\pgfpathcurveto{\pgfqpoint{2.875819in}{2.413558in}}{\pgfqpoint{2.865220in}{2.417948in}}{\pgfqpoint{2.854170in}{2.417948in}}%
\pgfpathcurveto{\pgfqpoint{2.843120in}{2.417948in}}{\pgfqpoint{2.832521in}{2.413558in}}{\pgfqpoint{2.824708in}{2.405744in}}%
\pgfpathcurveto{\pgfqpoint{2.816894in}{2.397930in}}{\pgfqpoint{2.812504in}{2.387331in}}{\pgfqpoint{2.812504in}{2.376281in}}%
\pgfpathcurveto{\pgfqpoint{2.812504in}{2.365231in}}{\pgfqpoint{2.816894in}{2.354632in}}{\pgfqpoint{2.824708in}{2.346818in}}%
\pgfpathcurveto{\pgfqpoint{2.832521in}{2.339005in}}{\pgfqpoint{2.843120in}{2.334615in}}{\pgfqpoint{2.854170in}{2.334615in}}%
\pgfpathclose%
\pgfusepath{stroke,fill}%
\end{pgfscope}%
\begin{pgfscope}%
\pgfpathrectangle{\pgfqpoint{0.511823in}{0.504323in}}{\pgfqpoint{3.218177in}{3.225677in}} %
\pgfusepath{clip}%
\pgfsetbuttcap%
\pgfsetroundjoin%
\definecolor{currentfill}{rgb}{0.501961,0.000000,0.000000}%
\pgfsetfillcolor{currentfill}%
\pgfsetfillopacity{0.400000}%
\pgfsetlinewidth{0.501875pt}%
\definecolor{currentstroke}{rgb}{0.501961,0.000000,0.000000}%
\pgfsetstrokecolor{currentstroke}%
\pgfsetstrokeopacity{0.400000}%
\pgfsetdash{}{0pt}%
\pgfpathmoveto{\pgfqpoint{2.753293in}{2.266099in}}%
\pgfpathcurveto{\pgfqpoint{2.764344in}{2.266099in}}{\pgfqpoint{2.774943in}{2.270489in}}{\pgfqpoint{2.782756in}{2.278303in}}%
\pgfpathcurveto{\pgfqpoint{2.790570in}{2.286117in}}{\pgfqpoint{2.794960in}{2.296716in}}{\pgfqpoint{2.794960in}{2.307766in}}%
\pgfpathcurveto{\pgfqpoint{2.794960in}{2.318816in}}{\pgfqpoint{2.790570in}{2.329415in}}{\pgfqpoint{2.782756in}{2.337229in}}%
\pgfpathcurveto{\pgfqpoint{2.774943in}{2.345042in}}{\pgfqpoint{2.764344in}{2.349432in}}{\pgfqpoint{2.753293in}{2.349432in}}%
\pgfpathcurveto{\pgfqpoint{2.742243in}{2.349432in}}{\pgfqpoint{2.731644in}{2.345042in}}{\pgfqpoint{2.723831in}{2.337229in}}%
\pgfpathcurveto{\pgfqpoint{2.716017in}{2.329415in}}{\pgfqpoint{2.711627in}{2.318816in}}{\pgfqpoint{2.711627in}{2.307766in}}%
\pgfpathcurveto{\pgfqpoint{2.711627in}{2.296716in}}{\pgfqpoint{2.716017in}{2.286117in}}{\pgfqpoint{2.723831in}{2.278303in}}%
\pgfpathcurveto{\pgfqpoint{2.731644in}{2.270489in}}{\pgfqpoint{2.742243in}{2.266099in}}{\pgfqpoint{2.753293in}{2.266099in}}%
\pgfpathclose%
\pgfusepath{stroke,fill}%
\end{pgfscope}%
\begin{pgfscope}%
\pgfpathrectangle{\pgfqpoint{0.511823in}{0.504323in}}{\pgfqpoint{3.218177in}{3.225677in}} %
\pgfusepath{clip}%
\pgfsetbuttcap%
\pgfsetroundjoin%
\definecolor{currentfill}{rgb}{0.501961,0.000000,0.000000}%
\pgfsetfillcolor{currentfill}%
\pgfsetfillopacity{0.400000}%
\pgfsetlinewidth{0.501875pt}%
\definecolor{currentstroke}{rgb}{0.501961,0.000000,0.000000}%
\pgfsetstrokecolor{currentstroke}%
\pgfsetstrokeopacity{0.400000}%
\pgfsetdash{}{0pt}%
\pgfpathmoveto{\pgfqpoint{2.721116in}{2.251029in}}%
\pgfpathcurveto{\pgfqpoint{2.732166in}{2.251029in}}{\pgfqpoint{2.742765in}{2.255419in}}{\pgfqpoint{2.750579in}{2.263233in}}%
\pgfpathcurveto{\pgfqpoint{2.758393in}{2.271046in}}{\pgfqpoint{2.762783in}{2.281645in}}{\pgfqpoint{2.762783in}{2.292695in}}%
\pgfpathcurveto{\pgfqpoint{2.762783in}{2.303746in}}{\pgfqpoint{2.758393in}{2.314345in}}{\pgfqpoint{2.750579in}{2.322158in}}%
\pgfpathcurveto{\pgfqpoint{2.742765in}{2.329972in}}{\pgfqpoint{2.732166in}{2.334362in}}{\pgfqpoint{2.721116in}{2.334362in}}%
\pgfpathcurveto{\pgfqpoint{2.710066in}{2.334362in}}{\pgfqpoint{2.699467in}{2.329972in}}{\pgfqpoint{2.691653in}{2.322158in}}%
\pgfpathcurveto{\pgfqpoint{2.683840in}{2.314345in}}{\pgfqpoint{2.679449in}{2.303746in}}{\pgfqpoint{2.679449in}{2.292695in}}%
\pgfpathcurveto{\pgfqpoint{2.679449in}{2.281645in}}{\pgfqpoint{2.683840in}{2.271046in}}{\pgfqpoint{2.691653in}{2.263233in}}%
\pgfpathcurveto{\pgfqpoint{2.699467in}{2.255419in}}{\pgfqpoint{2.710066in}{2.251029in}}{\pgfqpoint{2.721116in}{2.251029in}}%
\pgfpathclose%
\pgfusepath{stroke,fill}%
\end{pgfscope}%
\begin{pgfscope}%
\pgfpathrectangle{\pgfqpoint{0.511823in}{0.504323in}}{\pgfqpoint{3.218177in}{3.225677in}} %
\pgfusepath{clip}%
\pgfsetbuttcap%
\pgfsetroundjoin%
\definecolor{currentfill}{rgb}{0.501961,0.000000,0.000000}%
\pgfsetfillcolor{currentfill}%
\pgfsetfillopacity{0.400000}%
\pgfsetlinewidth{0.501875pt}%
\definecolor{currentstroke}{rgb}{0.501961,0.000000,0.000000}%
\pgfsetstrokecolor{currentstroke}%
\pgfsetstrokeopacity{0.400000}%
\pgfsetdash{}{0pt}%
\pgfpathmoveto{\pgfqpoint{2.777823in}{2.306526in}}%
\pgfpathcurveto{\pgfqpoint{2.788873in}{2.306526in}}{\pgfqpoint{2.799472in}{2.310916in}}{\pgfqpoint{2.807286in}{2.318730in}}%
\pgfpathcurveto{\pgfqpoint{2.815099in}{2.326544in}}{\pgfqpoint{2.819490in}{2.337143in}}{\pgfqpoint{2.819490in}{2.348193in}}%
\pgfpathcurveto{\pgfqpoint{2.819490in}{2.359243in}}{\pgfqpoint{2.815099in}{2.369842in}}{\pgfqpoint{2.807286in}{2.377656in}}%
\pgfpathcurveto{\pgfqpoint{2.799472in}{2.385469in}}{\pgfqpoint{2.788873in}{2.389859in}}{\pgfqpoint{2.777823in}{2.389859in}}%
\pgfpathcurveto{\pgfqpoint{2.766773in}{2.389859in}}{\pgfqpoint{2.756174in}{2.385469in}}{\pgfqpoint{2.748360in}{2.377656in}}%
\pgfpathcurveto{\pgfqpoint{2.740546in}{2.369842in}}{\pgfqpoint{2.736156in}{2.359243in}}{\pgfqpoint{2.736156in}{2.348193in}}%
\pgfpathcurveto{\pgfqpoint{2.736156in}{2.337143in}}{\pgfqpoint{2.740546in}{2.326544in}}{\pgfqpoint{2.748360in}{2.318730in}}%
\pgfpathcurveto{\pgfqpoint{2.756174in}{2.310916in}}{\pgfqpoint{2.766773in}{2.306526in}}{\pgfqpoint{2.777823in}{2.306526in}}%
\pgfpathclose%
\pgfusepath{stroke,fill}%
\end{pgfscope}%
\begin{pgfscope}%
\pgfpathrectangle{\pgfqpoint{0.511823in}{0.504323in}}{\pgfqpoint{3.218177in}{3.225677in}} %
\pgfusepath{clip}%
\pgfsetbuttcap%
\pgfsetroundjoin%
\definecolor{currentfill}{rgb}{0.501961,0.000000,0.000000}%
\pgfsetfillcolor{currentfill}%
\pgfsetfillopacity{0.400000}%
\pgfsetlinewidth{0.501875pt}%
\definecolor{currentstroke}{rgb}{0.501961,0.000000,0.000000}%
\pgfsetstrokecolor{currentstroke}%
\pgfsetstrokeopacity{0.400000}%
\pgfsetdash{}{0pt}%
\pgfpathmoveto{\pgfqpoint{2.772828in}{2.313168in}}%
\pgfpathcurveto{\pgfqpoint{2.783878in}{2.313168in}}{\pgfqpoint{2.794477in}{2.317558in}}{\pgfqpoint{2.802290in}{2.325372in}}%
\pgfpathcurveto{\pgfqpoint{2.810104in}{2.333186in}}{\pgfqpoint{2.814494in}{2.343785in}}{\pgfqpoint{2.814494in}{2.354835in}}%
\pgfpathcurveto{\pgfqpoint{2.814494in}{2.365885in}}{\pgfqpoint{2.810104in}{2.376484in}}{\pgfqpoint{2.802290in}{2.384297in}}%
\pgfpathcurveto{\pgfqpoint{2.794477in}{2.392111in}}{\pgfqpoint{2.783878in}{2.396501in}}{\pgfqpoint{2.772828in}{2.396501in}}%
\pgfpathcurveto{\pgfqpoint{2.761777in}{2.396501in}}{\pgfqpoint{2.751178in}{2.392111in}}{\pgfqpoint{2.743365in}{2.384297in}}%
\pgfpathcurveto{\pgfqpoint{2.735551in}{2.376484in}}{\pgfqpoint{2.731161in}{2.365885in}}{\pgfqpoint{2.731161in}{2.354835in}}%
\pgfpathcurveto{\pgfqpoint{2.731161in}{2.343785in}}{\pgfqpoint{2.735551in}{2.333186in}}{\pgfqpoint{2.743365in}{2.325372in}}%
\pgfpathcurveto{\pgfqpoint{2.751178in}{2.317558in}}{\pgfqpoint{2.761777in}{2.313168in}}{\pgfqpoint{2.772828in}{2.313168in}}%
\pgfpathclose%
\pgfusepath{stroke,fill}%
\end{pgfscope}%
\begin{pgfscope}%
\pgfpathrectangle{\pgfqpoint{0.511823in}{0.504323in}}{\pgfqpoint{3.218177in}{3.225677in}} %
\pgfusepath{clip}%
\pgfsetbuttcap%
\pgfsetroundjoin%
\definecolor{currentfill}{rgb}{0.501961,0.000000,0.000000}%
\pgfsetfillcolor{currentfill}%
\pgfsetfillopacity{0.400000}%
\pgfsetlinewidth{0.501875pt}%
\definecolor{currentstroke}{rgb}{0.501961,0.000000,0.000000}%
\pgfsetstrokecolor{currentstroke}%
\pgfsetstrokeopacity{0.400000}%
\pgfsetdash{}{0pt}%
\pgfpathmoveto{\pgfqpoint{2.699684in}{2.264809in}}%
\pgfpathcurveto{\pgfqpoint{2.710734in}{2.264809in}}{\pgfqpoint{2.721333in}{2.269199in}}{\pgfqpoint{2.729147in}{2.277013in}}%
\pgfpathcurveto{\pgfqpoint{2.736960in}{2.284826in}}{\pgfqpoint{2.741351in}{2.295425in}}{\pgfqpoint{2.741351in}{2.306476in}}%
\pgfpathcurveto{\pgfqpoint{2.741351in}{2.317526in}}{\pgfqpoint{2.736960in}{2.328125in}}{\pgfqpoint{2.729147in}{2.335938in}}%
\pgfpathcurveto{\pgfqpoint{2.721333in}{2.343752in}}{\pgfqpoint{2.710734in}{2.348142in}}{\pgfqpoint{2.699684in}{2.348142in}}%
\pgfpathcurveto{\pgfqpoint{2.688634in}{2.348142in}}{\pgfqpoint{2.678035in}{2.343752in}}{\pgfqpoint{2.670221in}{2.335938in}}%
\pgfpathcurveto{\pgfqpoint{2.662408in}{2.328125in}}{\pgfqpoint{2.658017in}{2.317526in}}{\pgfqpoint{2.658017in}{2.306476in}}%
\pgfpathcurveto{\pgfqpoint{2.658017in}{2.295425in}}{\pgfqpoint{2.662408in}{2.284826in}}{\pgfqpoint{2.670221in}{2.277013in}}%
\pgfpathcurveto{\pgfqpoint{2.678035in}{2.269199in}}{\pgfqpoint{2.688634in}{2.264809in}}{\pgfqpoint{2.699684in}{2.264809in}}%
\pgfpathclose%
\pgfusepath{stroke,fill}%
\end{pgfscope}%
\begin{pgfscope}%
\pgfpathrectangle{\pgfqpoint{0.511823in}{0.504323in}}{\pgfqpoint{3.218177in}{3.225677in}} %
\pgfusepath{clip}%
\pgfsetbuttcap%
\pgfsetroundjoin%
\definecolor{currentfill}{rgb}{0.501961,0.000000,0.000000}%
\pgfsetfillcolor{currentfill}%
\pgfsetfillopacity{0.400000}%
\pgfsetlinewidth{0.501875pt}%
\definecolor{currentstroke}{rgb}{0.501961,0.000000,0.000000}%
\pgfsetstrokecolor{currentstroke}%
\pgfsetstrokeopacity{0.400000}%
\pgfsetdash{}{0pt}%
\pgfpathmoveto{\pgfqpoint{2.785204in}{2.344642in}}%
\pgfpathcurveto{\pgfqpoint{2.796254in}{2.344642in}}{\pgfqpoint{2.806853in}{2.349032in}}{\pgfqpoint{2.814667in}{2.356845in}}%
\pgfpathcurveto{\pgfqpoint{2.822480in}{2.364659in}}{\pgfqpoint{2.826870in}{2.375258in}}{\pgfqpoint{2.826870in}{2.386308in}}%
\pgfpathcurveto{\pgfqpoint{2.826870in}{2.397358in}}{\pgfqpoint{2.822480in}{2.407957in}}{\pgfqpoint{2.814667in}{2.415771in}}%
\pgfpathcurveto{\pgfqpoint{2.806853in}{2.423585in}}{\pgfqpoint{2.796254in}{2.427975in}}{\pgfqpoint{2.785204in}{2.427975in}}%
\pgfpathcurveto{\pgfqpoint{2.774154in}{2.427975in}}{\pgfqpoint{2.763555in}{2.423585in}}{\pgfqpoint{2.755741in}{2.415771in}}%
\pgfpathcurveto{\pgfqpoint{2.747927in}{2.407957in}}{\pgfqpoint{2.743537in}{2.397358in}}{\pgfqpoint{2.743537in}{2.386308in}}%
\pgfpathcurveto{\pgfqpoint{2.743537in}{2.375258in}}{\pgfqpoint{2.747927in}{2.364659in}}{\pgfqpoint{2.755741in}{2.356845in}}%
\pgfpathcurveto{\pgfqpoint{2.763555in}{2.349032in}}{\pgfqpoint{2.774154in}{2.344642in}}{\pgfqpoint{2.785204in}{2.344642in}}%
\pgfpathclose%
\pgfusepath{stroke,fill}%
\end{pgfscope}%
\begin{pgfscope}%
\pgfpathrectangle{\pgfqpoint{0.511823in}{0.504323in}}{\pgfqpoint{3.218177in}{3.225677in}} %
\pgfusepath{clip}%
\pgfsetbuttcap%
\pgfsetroundjoin%
\definecolor{currentfill}{rgb}{0.501961,0.000000,0.000000}%
\pgfsetfillcolor{currentfill}%
\pgfsetfillopacity{0.400000}%
\pgfsetlinewidth{0.501875pt}%
\definecolor{currentstroke}{rgb}{0.501961,0.000000,0.000000}%
\pgfsetstrokecolor{currentstroke}%
\pgfsetstrokeopacity{0.400000}%
\pgfsetdash{}{0pt}%
\pgfpathmoveto{\pgfqpoint{2.744976in}{2.322618in}}%
\pgfpathcurveto{\pgfqpoint{2.756026in}{2.322618in}}{\pgfqpoint{2.766625in}{2.327008in}}{\pgfqpoint{2.774439in}{2.334822in}}%
\pgfpathcurveto{\pgfqpoint{2.782253in}{2.342636in}}{\pgfqpoint{2.786643in}{2.353235in}}{\pgfqpoint{2.786643in}{2.364285in}}%
\pgfpathcurveto{\pgfqpoint{2.786643in}{2.375335in}}{\pgfqpoint{2.782253in}{2.385934in}}{\pgfqpoint{2.774439in}{2.393747in}}%
\pgfpathcurveto{\pgfqpoint{2.766625in}{2.401561in}}{\pgfqpoint{2.756026in}{2.405951in}}{\pgfqpoint{2.744976in}{2.405951in}}%
\pgfpathcurveto{\pgfqpoint{2.733926in}{2.405951in}}{\pgfqpoint{2.723327in}{2.401561in}}{\pgfqpoint{2.715513in}{2.393747in}}%
\pgfpathcurveto{\pgfqpoint{2.707700in}{2.385934in}}{\pgfqpoint{2.703310in}{2.375335in}}{\pgfqpoint{2.703310in}{2.364285in}}%
\pgfpathcurveto{\pgfqpoint{2.703310in}{2.353235in}}{\pgfqpoint{2.707700in}{2.342636in}}{\pgfqpoint{2.715513in}{2.334822in}}%
\pgfpathcurveto{\pgfqpoint{2.723327in}{2.327008in}}{\pgfqpoint{2.733926in}{2.322618in}}{\pgfqpoint{2.744976in}{2.322618in}}%
\pgfpathclose%
\pgfusepath{stroke,fill}%
\end{pgfscope}%
\begin{pgfscope}%
\pgfpathrectangle{\pgfqpoint{0.511823in}{0.504323in}}{\pgfqpoint{3.218177in}{3.225677in}} %
\pgfusepath{clip}%
\pgfsetbuttcap%
\pgfsetroundjoin%
\definecolor{currentfill}{rgb}{0.501961,0.000000,0.000000}%
\pgfsetfillcolor{currentfill}%
\pgfsetfillopacity{0.400000}%
\pgfsetlinewidth{0.501875pt}%
\definecolor{currentstroke}{rgb}{0.501961,0.000000,0.000000}%
\pgfsetstrokecolor{currentstroke}%
\pgfsetstrokeopacity{0.400000}%
\pgfsetdash{}{0pt}%
\pgfpathmoveto{\pgfqpoint{2.695444in}{2.292581in}}%
\pgfpathcurveto{\pgfqpoint{2.706494in}{2.292581in}}{\pgfqpoint{2.717093in}{2.296971in}}{\pgfqpoint{2.724907in}{2.304785in}}%
\pgfpathcurveto{\pgfqpoint{2.732721in}{2.312599in}}{\pgfqpoint{2.737111in}{2.323198in}}{\pgfqpoint{2.737111in}{2.334248in}}%
\pgfpathcurveto{\pgfqpoint{2.737111in}{2.345298in}}{\pgfqpoint{2.732721in}{2.355897in}}{\pgfqpoint{2.724907in}{2.363711in}}%
\pgfpathcurveto{\pgfqpoint{2.717093in}{2.371524in}}{\pgfqpoint{2.706494in}{2.375915in}}{\pgfqpoint{2.695444in}{2.375915in}}%
\pgfpathcurveto{\pgfqpoint{2.684394in}{2.375915in}}{\pgfqpoint{2.673795in}{2.371524in}}{\pgfqpoint{2.665981in}{2.363711in}}%
\pgfpathcurveto{\pgfqpoint{2.658168in}{2.355897in}}{\pgfqpoint{2.653778in}{2.345298in}}{\pgfqpoint{2.653778in}{2.334248in}}%
\pgfpathcurveto{\pgfqpoint{2.653778in}{2.323198in}}{\pgfqpoint{2.658168in}{2.312599in}}{\pgfqpoint{2.665981in}{2.304785in}}%
\pgfpathcurveto{\pgfqpoint{2.673795in}{2.296971in}}{\pgfqpoint{2.684394in}{2.292581in}}{\pgfqpoint{2.695444in}{2.292581in}}%
\pgfpathclose%
\pgfusepath{stroke,fill}%
\end{pgfscope}%
\begin{pgfscope}%
\pgfpathrectangle{\pgfqpoint{0.511823in}{0.504323in}}{\pgfqpoint{3.218177in}{3.225677in}} %
\pgfusepath{clip}%
\pgfsetbuttcap%
\pgfsetroundjoin%
\definecolor{currentfill}{rgb}{0.501961,0.000000,0.000000}%
\pgfsetfillcolor{currentfill}%
\pgfsetfillopacity{0.400000}%
\pgfsetlinewidth{0.501875pt}%
\definecolor{currentstroke}{rgb}{0.501961,0.000000,0.000000}%
\pgfsetstrokecolor{currentstroke}%
\pgfsetstrokeopacity{0.400000}%
\pgfsetdash{}{0pt}%
\pgfpathmoveto{\pgfqpoint{2.668870in}{2.281093in}}%
\pgfpathcurveto{\pgfqpoint{2.679920in}{2.281093in}}{\pgfqpoint{2.690519in}{2.285484in}}{\pgfqpoint{2.698333in}{2.293297in}}%
\pgfpathcurveto{\pgfqpoint{2.706147in}{2.301111in}}{\pgfqpoint{2.710537in}{2.311710in}}{\pgfqpoint{2.710537in}{2.322760in}}%
\pgfpathcurveto{\pgfqpoint{2.710537in}{2.333810in}}{\pgfqpoint{2.706147in}{2.344409in}}{\pgfqpoint{2.698333in}{2.352223in}}%
\pgfpathcurveto{\pgfqpoint{2.690519in}{2.360037in}}{\pgfqpoint{2.679920in}{2.364427in}}{\pgfqpoint{2.668870in}{2.364427in}}%
\pgfpathcurveto{\pgfqpoint{2.657820in}{2.364427in}}{\pgfqpoint{2.647221in}{2.360037in}}{\pgfqpoint{2.639407in}{2.352223in}}%
\pgfpathcurveto{\pgfqpoint{2.631594in}{2.344409in}}{\pgfqpoint{2.627203in}{2.333810in}}{\pgfqpoint{2.627203in}{2.322760in}}%
\pgfpathcurveto{\pgfqpoint{2.627203in}{2.311710in}}{\pgfqpoint{2.631594in}{2.301111in}}{\pgfqpoint{2.639407in}{2.293297in}}%
\pgfpathcurveto{\pgfqpoint{2.647221in}{2.285484in}}{\pgfqpoint{2.657820in}{2.281093in}}{\pgfqpoint{2.668870in}{2.281093in}}%
\pgfpathclose%
\pgfusepath{stroke,fill}%
\end{pgfscope}%
\begin{pgfscope}%
\pgfpathrectangle{\pgfqpoint{0.511823in}{0.504323in}}{\pgfqpoint{3.218177in}{3.225677in}} %
\pgfusepath{clip}%
\pgfsetbuttcap%
\pgfsetroundjoin%
\definecolor{currentfill}{rgb}{0.501961,0.000000,0.000000}%
\pgfsetfillcolor{currentfill}%
\pgfsetfillopacity{0.400000}%
\pgfsetlinewidth{0.501875pt}%
\definecolor{currentstroke}{rgb}{0.501961,0.000000,0.000000}%
\pgfsetstrokecolor{currentstroke}%
\pgfsetstrokeopacity{0.400000}%
\pgfsetdash{}{0pt}%
\pgfpathmoveto{\pgfqpoint{2.553504in}{2.195404in}}%
\pgfpathcurveto{\pgfqpoint{2.564554in}{2.195404in}}{\pgfqpoint{2.575153in}{2.199794in}}{\pgfqpoint{2.582967in}{2.207607in}}%
\pgfpathcurveto{\pgfqpoint{2.590781in}{2.215421in}}{\pgfqpoint{2.595171in}{2.226020in}}{\pgfqpoint{2.595171in}{2.237070in}}%
\pgfpathcurveto{\pgfqpoint{2.595171in}{2.248120in}}{\pgfqpoint{2.590781in}{2.258719in}}{\pgfqpoint{2.582967in}{2.266533in}}%
\pgfpathcurveto{\pgfqpoint{2.575153in}{2.274347in}}{\pgfqpoint{2.564554in}{2.278737in}}{\pgfqpoint{2.553504in}{2.278737in}}%
\pgfpathcurveto{\pgfqpoint{2.542454in}{2.278737in}}{\pgfqpoint{2.531855in}{2.274347in}}{\pgfqpoint{2.524041in}{2.266533in}}%
\pgfpathcurveto{\pgfqpoint{2.516228in}{2.258719in}}{\pgfqpoint{2.511837in}{2.248120in}}{\pgfqpoint{2.511837in}{2.237070in}}%
\pgfpathcurveto{\pgfqpoint{2.511837in}{2.226020in}}{\pgfqpoint{2.516228in}{2.215421in}}{\pgfqpoint{2.524041in}{2.207607in}}%
\pgfpathcurveto{\pgfqpoint{2.531855in}{2.199794in}}{\pgfqpoint{2.542454in}{2.195404in}}{\pgfqpoint{2.553504in}{2.195404in}}%
\pgfpathclose%
\pgfusepath{stroke,fill}%
\end{pgfscope}%
\begin{pgfscope}%
\pgfpathrectangle{\pgfqpoint{0.511823in}{0.504323in}}{\pgfqpoint{3.218177in}{3.225677in}} %
\pgfusepath{clip}%
\pgfsetbuttcap%
\pgfsetroundjoin%
\definecolor{currentfill}{rgb}{0.501961,0.000000,0.000000}%
\pgfsetfillcolor{currentfill}%
\pgfsetfillopacity{0.400000}%
\pgfsetlinewidth{0.501875pt}%
\definecolor{currentstroke}{rgb}{0.501961,0.000000,0.000000}%
\pgfsetstrokecolor{currentstroke}%
\pgfsetstrokeopacity{0.400000}%
\pgfsetdash{}{0pt}%
\pgfpathmoveto{\pgfqpoint{2.676485in}{2.308404in}}%
\pgfpathcurveto{\pgfqpoint{2.687535in}{2.308404in}}{\pgfqpoint{2.698134in}{2.312794in}}{\pgfqpoint{2.705948in}{2.320608in}}%
\pgfpathcurveto{\pgfqpoint{2.713762in}{2.328422in}}{\pgfqpoint{2.718152in}{2.339021in}}{\pgfqpoint{2.718152in}{2.350071in}}%
\pgfpathcurveto{\pgfqpoint{2.718152in}{2.361121in}}{\pgfqpoint{2.713762in}{2.371720in}}{\pgfqpoint{2.705948in}{2.379534in}}%
\pgfpathcurveto{\pgfqpoint{2.698134in}{2.387347in}}{\pgfqpoint{2.687535in}{2.391737in}}{\pgfqpoint{2.676485in}{2.391737in}}%
\pgfpathcurveto{\pgfqpoint{2.665435in}{2.391737in}}{\pgfqpoint{2.654836in}{2.387347in}}{\pgfqpoint{2.647023in}{2.379534in}}%
\pgfpathcurveto{\pgfqpoint{2.639209in}{2.371720in}}{\pgfqpoint{2.634819in}{2.361121in}}{\pgfqpoint{2.634819in}{2.350071in}}%
\pgfpathcurveto{\pgfqpoint{2.634819in}{2.339021in}}{\pgfqpoint{2.639209in}{2.328422in}}{\pgfqpoint{2.647023in}{2.320608in}}%
\pgfpathcurveto{\pgfqpoint{2.654836in}{2.312794in}}{\pgfqpoint{2.665435in}{2.308404in}}{\pgfqpoint{2.676485in}{2.308404in}}%
\pgfpathclose%
\pgfusepath{stroke,fill}%
\end{pgfscope}%
\begin{pgfscope}%
\pgfpathrectangle{\pgfqpoint{0.511823in}{0.504323in}}{\pgfqpoint{3.218177in}{3.225677in}} %
\pgfusepath{clip}%
\pgfsetbuttcap%
\pgfsetroundjoin%
\definecolor{currentfill}{rgb}{0.501961,0.000000,0.000000}%
\pgfsetfillcolor{currentfill}%
\pgfsetfillopacity{0.400000}%
\pgfsetlinewidth{0.501875pt}%
\definecolor{currentstroke}{rgb}{0.501961,0.000000,0.000000}%
\pgfsetstrokecolor{currentstroke}%
\pgfsetstrokeopacity{0.400000}%
\pgfsetdash{}{0pt}%
\pgfpathmoveto{\pgfqpoint{2.669282in}{2.312918in}}%
\pgfpathcurveto{\pgfqpoint{2.680332in}{2.312918in}}{\pgfqpoint{2.690931in}{2.317308in}}{\pgfqpoint{2.698745in}{2.325122in}}%
\pgfpathcurveto{\pgfqpoint{2.706558in}{2.332936in}}{\pgfqpoint{2.710949in}{2.343535in}}{\pgfqpoint{2.710949in}{2.354585in}}%
\pgfpathcurveto{\pgfqpoint{2.710949in}{2.365635in}}{\pgfqpoint{2.706558in}{2.376234in}}{\pgfqpoint{2.698745in}{2.384048in}}%
\pgfpathcurveto{\pgfqpoint{2.690931in}{2.391861in}}{\pgfqpoint{2.680332in}{2.396252in}}{\pgfqpoint{2.669282in}{2.396252in}}%
\pgfpathcurveto{\pgfqpoint{2.658232in}{2.396252in}}{\pgfqpoint{2.647633in}{2.391861in}}{\pgfqpoint{2.639819in}{2.384048in}}%
\pgfpathcurveto{\pgfqpoint{2.632006in}{2.376234in}}{\pgfqpoint{2.627615in}{2.365635in}}{\pgfqpoint{2.627615in}{2.354585in}}%
\pgfpathcurveto{\pgfqpoint{2.627615in}{2.343535in}}{\pgfqpoint{2.632006in}{2.332936in}}{\pgfqpoint{2.639819in}{2.325122in}}%
\pgfpathcurveto{\pgfqpoint{2.647633in}{2.317308in}}{\pgfqpoint{2.658232in}{2.312918in}}{\pgfqpoint{2.669282in}{2.312918in}}%
\pgfpathclose%
\pgfusepath{stroke,fill}%
\end{pgfscope}%
\begin{pgfscope}%
\pgfpathrectangle{\pgfqpoint{0.511823in}{0.504323in}}{\pgfqpoint{3.218177in}{3.225677in}} %
\pgfusepath{clip}%
\pgfsetbuttcap%
\pgfsetroundjoin%
\definecolor{currentfill}{rgb}{0.501961,0.000000,0.000000}%
\pgfsetfillcolor{currentfill}%
\pgfsetfillopacity{0.400000}%
\pgfsetlinewidth{0.501875pt}%
\definecolor{currentstroke}{rgb}{0.501961,0.000000,0.000000}%
\pgfsetstrokecolor{currentstroke}%
\pgfsetstrokeopacity{0.400000}%
\pgfsetdash{}{0pt}%
\pgfpathmoveto{\pgfqpoint{2.564552in}{2.234600in}}%
\pgfpathcurveto{\pgfqpoint{2.575603in}{2.234600in}}{\pgfqpoint{2.586202in}{2.238990in}}{\pgfqpoint{2.594015in}{2.246803in}}%
\pgfpathcurveto{\pgfqpoint{2.601829in}{2.254617in}}{\pgfqpoint{2.606219in}{2.265216in}}{\pgfqpoint{2.606219in}{2.276266in}}%
\pgfpathcurveto{\pgfqpoint{2.606219in}{2.287316in}}{\pgfqpoint{2.601829in}{2.297915in}}{\pgfqpoint{2.594015in}{2.305729in}}%
\pgfpathcurveto{\pgfqpoint{2.586202in}{2.313543in}}{\pgfqpoint{2.575603in}{2.317933in}}{\pgfqpoint{2.564552in}{2.317933in}}%
\pgfpathcurveto{\pgfqpoint{2.553502in}{2.317933in}}{\pgfqpoint{2.542903in}{2.313543in}}{\pgfqpoint{2.535090in}{2.305729in}}%
\pgfpathcurveto{\pgfqpoint{2.527276in}{2.297915in}}{\pgfqpoint{2.522886in}{2.287316in}}{\pgfqpoint{2.522886in}{2.276266in}}%
\pgfpathcurveto{\pgfqpoint{2.522886in}{2.265216in}}{\pgfqpoint{2.527276in}{2.254617in}}{\pgfqpoint{2.535090in}{2.246803in}}%
\pgfpathcurveto{\pgfqpoint{2.542903in}{2.238990in}}{\pgfqpoint{2.553502in}{2.234600in}}{\pgfqpoint{2.564552in}{2.234600in}}%
\pgfpathclose%
\pgfusepath{stroke,fill}%
\end{pgfscope}%
\begin{pgfscope}%
\pgfpathrectangle{\pgfqpoint{0.511823in}{0.504323in}}{\pgfqpoint{3.218177in}{3.225677in}} %
\pgfusepath{clip}%
\pgfsetbuttcap%
\pgfsetroundjoin%
\definecolor{currentfill}{rgb}{0.501961,0.000000,0.000000}%
\pgfsetfillcolor{currentfill}%
\pgfsetfillopacity{0.400000}%
\pgfsetlinewidth{0.501875pt}%
\definecolor{currentstroke}{rgb}{0.501961,0.000000,0.000000}%
\pgfsetstrokecolor{currentstroke}%
\pgfsetstrokeopacity{0.400000}%
\pgfsetdash{}{0pt}%
\pgfpathmoveto{\pgfqpoint{2.773610in}{2.423363in}}%
\pgfpathcurveto{\pgfqpoint{2.784660in}{2.423363in}}{\pgfqpoint{2.795259in}{2.427753in}}{\pgfqpoint{2.803073in}{2.435566in}}%
\pgfpathcurveto{\pgfqpoint{2.810887in}{2.443380in}}{\pgfqpoint{2.815277in}{2.453979in}}{\pgfqpoint{2.815277in}{2.465029in}}%
\pgfpathcurveto{\pgfqpoint{2.815277in}{2.476079in}}{\pgfqpoint{2.810887in}{2.486678in}}{\pgfqpoint{2.803073in}{2.494492in}}%
\pgfpathcurveto{\pgfqpoint{2.795259in}{2.502306in}}{\pgfqpoint{2.784660in}{2.506696in}}{\pgfqpoint{2.773610in}{2.506696in}}%
\pgfpathcurveto{\pgfqpoint{2.762560in}{2.506696in}}{\pgfqpoint{2.751961in}{2.502306in}}{\pgfqpoint{2.744147in}{2.494492in}}%
\pgfpathcurveto{\pgfqpoint{2.736334in}{2.486678in}}{\pgfqpoint{2.731943in}{2.476079in}}{\pgfqpoint{2.731943in}{2.465029in}}%
\pgfpathcurveto{\pgfqpoint{2.731943in}{2.453979in}}{\pgfqpoint{2.736334in}{2.443380in}}{\pgfqpoint{2.744147in}{2.435566in}}%
\pgfpathcurveto{\pgfqpoint{2.751961in}{2.427753in}}{\pgfqpoint{2.762560in}{2.423363in}}{\pgfqpoint{2.773610in}{2.423363in}}%
\pgfpathclose%
\pgfusepath{stroke,fill}%
\end{pgfscope}%
\begin{pgfscope}%
\pgfpathrectangle{\pgfqpoint{0.511823in}{0.504323in}}{\pgfqpoint{3.218177in}{3.225677in}} %
\pgfusepath{clip}%
\pgfsetbuttcap%
\pgfsetroundjoin%
\definecolor{currentfill}{rgb}{0.501961,0.000000,0.000000}%
\pgfsetfillcolor{currentfill}%
\pgfsetfillopacity{0.400000}%
\pgfsetlinewidth{0.501875pt}%
\definecolor{currentstroke}{rgb}{0.501961,0.000000,0.000000}%
\pgfsetstrokecolor{currentstroke}%
\pgfsetstrokeopacity{0.400000}%
\pgfsetdash{}{0pt}%
\pgfpathmoveto{\pgfqpoint{2.634514in}{2.315047in}}%
\pgfpathcurveto{\pgfqpoint{2.645564in}{2.315047in}}{\pgfqpoint{2.656163in}{2.319437in}}{\pgfqpoint{2.663977in}{2.327250in}}%
\pgfpathcurveto{\pgfqpoint{2.671790in}{2.335064in}}{\pgfqpoint{2.676181in}{2.345663in}}{\pgfqpoint{2.676181in}{2.356713in}}%
\pgfpathcurveto{\pgfqpoint{2.676181in}{2.367763in}}{\pgfqpoint{2.671790in}{2.378362in}}{\pgfqpoint{2.663977in}{2.386176in}}%
\pgfpathcurveto{\pgfqpoint{2.656163in}{2.393990in}}{\pgfqpoint{2.645564in}{2.398380in}}{\pgfqpoint{2.634514in}{2.398380in}}%
\pgfpathcurveto{\pgfqpoint{2.623464in}{2.398380in}}{\pgfqpoint{2.612865in}{2.393990in}}{\pgfqpoint{2.605051in}{2.386176in}}%
\pgfpathcurveto{\pgfqpoint{2.597237in}{2.378362in}}{\pgfqpoint{2.592847in}{2.367763in}}{\pgfqpoint{2.592847in}{2.356713in}}%
\pgfpathcurveto{\pgfqpoint{2.592847in}{2.345663in}}{\pgfqpoint{2.597237in}{2.335064in}}{\pgfqpoint{2.605051in}{2.327250in}}%
\pgfpathcurveto{\pgfqpoint{2.612865in}{2.319437in}}{\pgfqpoint{2.623464in}{2.315047in}}{\pgfqpoint{2.634514in}{2.315047in}}%
\pgfpathclose%
\pgfusepath{stroke,fill}%
\end{pgfscope}%
\begin{pgfscope}%
\pgfpathrectangle{\pgfqpoint{0.511823in}{0.504323in}}{\pgfqpoint{3.218177in}{3.225677in}} %
\pgfusepath{clip}%
\pgfsetbuttcap%
\pgfsetroundjoin%
\definecolor{currentfill}{rgb}{0.501961,0.000000,0.000000}%
\pgfsetfillcolor{currentfill}%
\pgfsetfillopacity{0.400000}%
\pgfsetlinewidth{0.501875pt}%
\definecolor{currentstroke}{rgb}{0.501961,0.000000,0.000000}%
\pgfsetstrokecolor{currentstroke}%
\pgfsetstrokeopacity{0.400000}%
\pgfsetdash{}{0pt}%
\pgfpathmoveto{\pgfqpoint{2.525095in}{2.230965in}}%
\pgfpathcurveto{\pgfqpoint{2.536146in}{2.230965in}}{\pgfqpoint{2.546745in}{2.235355in}}{\pgfqpoint{2.554558in}{2.243169in}}%
\pgfpathcurveto{\pgfqpoint{2.562372in}{2.250982in}}{\pgfqpoint{2.566762in}{2.261581in}}{\pgfqpoint{2.566762in}{2.272632in}}%
\pgfpathcurveto{\pgfqpoint{2.566762in}{2.283682in}}{\pgfqpoint{2.562372in}{2.294281in}}{\pgfqpoint{2.554558in}{2.302094in}}%
\pgfpathcurveto{\pgfqpoint{2.546745in}{2.309908in}}{\pgfqpoint{2.536146in}{2.314298in}}{\pgfqpoint{2.525095in}{2.314298in}}%
\pgfpathcurveto{\pgfqpoint{2.514045in}{2.314298in}}{\pgfqpoint{2.503446in}{2.309908in}}{\pgfqpoint{2.495633in}{2.302094in}}%
\pgfpathcurveto{\pgfqpoint{2.487819in}{2.294281in}}{\pgfqpoint{2.483429in}{2.283682in}}{\pgfqpoint{2.483429in}{2.272632in}}%
\pgfpathcurveto{\pgfqpoint{2.483429in}{2.261581in}}{\pgfqpoint{2.487819in}{2.250982in}}{\pgfqpoint{2.495633in}{2.243169in}}%
\pgfpathcurveto{\pgfqpoint{2.503446in}{2.235355in}}{\pgfqpoint{2.514045in}{2.230965in}}{\pgfqpoint{2.525095in}{2.230965in}}%
\pgfpathclose%
\pgfusepath{stroke,fill}%
\end{pgfscope}%
\begin{pgfscope}%
\pgfpathrectangle{\pgfqpoint{0.511823in}{0.504323in}}{\pgfqpoint{3.218177in}{3.225677in}} %
\pgfusepath{clip}%
\pgfsetbuttcap%
\pgfsetroundjoin%
\definecolor{currentfill}{rgb}{0.501961,0.000000,0.000000}%
\pgfsetfillcolor{currentfill}%
\pgfsetfillopacity{0.400000}%
\pgfsetlinewidth{0.501875pt}%
\definecolor{currentstroke}{rgb}{0.501961,0.000000,0.000000}%
\pgfsetstrokecolor{currentstroke}%
\pgfsetstrokeopacity{0.400000}%
\pgfsetdash{}{0pt}%
\pgfpathmoveto{\pgfqpoint{2.583079in}{2.291536in}}%
\pgfpathcurveto{\pgfqpoint{2.594129in}{2.291536in}}{\pgfqpoint{2.604728in}{2.295926in}}{\pgfqpoint{2.612542in}{2.303740in}}%
\pgfpathcurveto{\pgfqpoint{2.620355in}{2.311554in}}{\pgfqpoint{2.624746in}{2.322153in}}{\pgfqpoint{2.624746in}{2.333203in}}%
\pgfpathcurveto{\pgfqpoint{2.624746in}{2.344253in}}{\pgfqpoint{2.620355in}{2.354852in}}{\pgfqpoint{2.612542in}{2.362666in}}%
\pgfpathcurveto{\pgfqpoint{2.604728in}{2.370479in}}{\pgfqpoint{2.594129in}{2.374869in}}{\pgfqpoint{2.583079in}{2.374869in}}%
\pgfpathcurveto{\pgfqpoint{2.572029in}{2.374869in}}{\pgfqpoint{2.561430in}{2.370479in}}{\pgfqpoint{2.553616in}{2.362666in}}%
\pgfpathcurveto{\pgfqpoint{2.545802in}{2.354852in}}{\pgfqpoint{2.541412in}{2.344253in}}{\pgfqpoint{2.541412in}{2.333203in}}%
\pgfpathcurveto{\pgfqpoint{2.541412in}{2.322153in}}{\pgfqpoint{2.545802in}{2.311554in}}{\pgfqpoint{2.553616in}{2.303740in}}%
\pgfpathcurveto{\pgfqpoint{2.561430in}{2.295926in}}{\pgfqpoint{2.572029in}{2.291536in}}{\pgfqpoint{2.583079in}{2.291536in}}%
\pgfpathclose%
\pgfusepath{stroke,fill}%
\end{pgfscope}%
\begin{pgfscope}%
\pgfpathrectangle{\pgfqpoint{0.511823in}{0.504323in}}{\pgfqpoint{3.218177in}{3.225677in}} %
\pgfusepath{clip}%
\pgfsetbuttcap%
\pgfsetroundjoin%
\definecolor{currentfill}{rgb}{0.501961,0.000000,0.000000}%
\pgfsetfillcolor{currentfill}%
\pgfsetfillopacity{0.400000}%
\pgfsetlinewidth{0.501875pt}%
\definecolor{currentstroke}{rgb}{0.501961,0.000000,0.000000}%
\pgfsetstrokecolor{currentstroke}%
\pgfsetstrokeopacity{0.400000}%
\pgfsetdash{}{0pt}%
\pgfpathmoveto{\pgfqpoint{2.613758in}{2.328876in}}%
\pgfpathcurveto{\pgfqpoint{2.624808in}{2.328876in}}{\pgfqpoint{2.635407in}{2.333267in}}{\pgfqpoint{2.643221in}{2.341080in}}%
\pgfpathcurveto{\pgfqpoint{2.651035in}{2.348894in}}{\pgfqpoint{2.655425in}{2.359493in}}{\pgfqpoint{2.655425in}{2.370543in}}%
\pgfpathcurveto{\pgfqpoint{2.655425in}{2.381593in}}{\pgfqpoint{2.651035in}{2.392192in}}{\pgfqpoint{2.643221in}{2.400006in}}%
\pgfpathcurveto{\pgfqpoint{2.635407in}{2.407819in}}{\pgfqpoint{2.624808in}{2.412210in}}{\pgfqpoint{2.613758in}{2.412210in}}%
\pgfpathcurveto{\pgfqpoint{2.602708in}{2.412210in}}{\pgfqpoint{2.592109in}{2.407819in}}{\pgfqpoint{2.584295in}{2.400006in}}%
\pgfpathcurveto{\pgfqpoint{2.576482in}{2.392192in}}{\pgfqpoint{2.572091in}{2.381593in}}{\pgfqpoint{2.572091in}{2.370543in}}%
\pgfpathcurveto{\pgfqpoint{2.572091in}{2.359493in}}{\pgfqpoint{2.576482in}{2.348894in}}{\pgfqpoint{2.584295in}{2.341080in}}%
\pgfpathcurveto{\pgfqpoint{2.592109in}{2.333267in}}{\pgfqpoint{2.602708in}{2.328876in}}{\pgfqpoint{2.613758in}{2.328876in}}%
\pgfpathclose%
\pgfusepath{stroke,fill}%
\end{pgfscope}%
\begin{pgfscope}%
\pgfpathrectangle{\pgfqpoint{0.511823in}{0.504323in}}{\pgfqpoint{3.218177in}{3.225677in}} %
\pgfusepath{clip}%
\pgfsetbuttcap%
\pgfsetroundjoin%
\definecolor{currentfill}{rgb}{0.501961,0.000000,0.000000}%
\pgfsetfillcolor{currentfill}%
\pgfsetfillopacity{0.400000}%
\pgfsetlinewidth{0.501875pt}%
\definecolor{currentstroke}{rgb}{0.501961,0.000000,0.000000}%
\pgfsetstrokecolor{currentstroke}%
\pgfsetstrokeopacity{0.400000}%
\pgfsetdash{}{0pt}%
\pgfpathmoveto{\pgfqpoint{2.608684in}{2.335086in}}%
\pgfpathcurveto{\pgfqpoint{2.619734in}{2.335086in}}{\pgfqpoint{2.630333in}{2.339476in}}{\pgfqpoint{2.638146in}{2.347290in}}%
\pgfpathcurveto{\pgfqpoint{2.645960in}{2.355103in}}{\pgfqpoint{2.650350in}{2.365703in}}{\pgfqpoint{2.650350in}{2.376753in}}%
\pgfpathcurveto{\pgfqpoint{2.650350in}{2.387803in}}{\pgfqpoint{2.645960in}{2.398402in}}{\pgfqpoint{2.638146in}{2.406215in}}%
\pgfpathcurveto{\pgfqpoint{2.630333in}{2.414029in}}{\pgfqpoint{2.619734in}{2.418419in}}{\pgfqpoint{2.608684in}{2.418419in}}%
\pgfpathcurveto{\pgfqpoint{2.597633in}{2.418419in}}{\pgfqpoint{2.587034in}{2.414029in}}{\pgfqpoint{2.579221in}{2.406215in}}%
\pgfpathcurveto{\pgfqpoint{2.571407in}{2.398402in}}{\pgfqpoint{2.567017in}{2.387803in}}{\pgfqpoint{2.567017in}{2.376753in}}%
\pgfpathcurveto{\pgfqpoint{2.567017in}{2.365703in}}{\pgfqpoint{2.571407in}{2.355103in}}{\pgfqpoint{2.579221in}{2.347290in}}%
\pgfpathcurveto{\pgfqpoint{2.587034in}{2.339476in}}{\pgfqpoint{2.597633in}{2.335086in}}{\pgfqpoint{2.608684in}{2.335086in}}%
\pgfpathclose%
\pgfusepath{stroke,fill}%
\end{pgfscope}%
\begin{pgfscope}%
\pgfpathrectangle{\pgfqpoint{0.511823in}{0.504323in}}{\pgfqpoint{3.218177in}{3.225677in}} %
\pgfusepath{clip}%
\pgfsetbuttcap%
\pgfsetroundjoin%
\definecolor{currentfill}{rgb}{0.501961,0.000000,0.000000}%
\pgfsetfillcolor{currentfill}%
\pgfsetfillopacity{0.400000}%
\pgfsetlinewidth{0.501875pt}%
\definecolor{currentstroke}{rgb}{0.501961,0.000000,0.000000}%
\pgfsetstrokecolor{currentstroke}%
\pgfsetstrokeopacity{0.400000}%
\pgfsetdash{}{0pt}%
\pgfpathmoveto{\pgfqpoint{2.531438in}{2.277249in}}%
\pgfpathcurveto{\pgfqpoint{2.542488in}{2.277249in}}{\pgfqpoint{2.553088in}{2.281640in}}{\pgfqpoint{2.560901in}{2.289453in}}%
\pgfpathcurveto{\pgfqpoint{2.568715in}{2.297267in}}{\pgfqpoint{2.573105in}{2.307866in}}{\pgfqpoint{2.573105in}{2.318916in}}%
\pgfpathcurveto{\pgfqpoint{2.573105in}{2.329966in}}{\pgfqpoint{2.568715in}{2.340565in}}{\pgfqpoint{2.560901in}{2.348379in}}%
\pgfpathcurveto{\pgfqpoint{2.553088in}{2.356192in}}{\pgfqpoint{2.542488in}{2.360583in}}{\pgfqpoint{2.531438in}{2.360583in}}%
\pgfpathcurveto{\pgfqpoint{2.520388in}{2.360583in}}{\pgfqpoint{2.509789in}{2.356192in}}{\pgfqpoint{2.501976in}{2.348379in}}%
\pgfpathcurveto{\pgfqpoint{2.494162in}{2.340565in}}{\pgfqpoint{2.489772in}{2.329966in}}{\pgfqpoint{2.489772in}{2.318916in}}%
\pgfpathcurveto{\pgfqpoint{2.489772in}{2.307866in}}{\pgfqpoint{2.494162in}{2.297267in}}{\pgfqpoint{2.501976in}{2.289453in}}%
\pgfpathcurveto{\pgfqpoint{2.509789in}{2.281640in}}{\pgfqpoint{2.520388in}{2.277249in}}{\pgfqpoint{2.531438in}{2.277249in}}%
\pgfpathclose%
\pgfusepath{stroke,fill}%
\end{pgfscope}%
\begin{pgfscope}%
\pgfpathrectangle{\pgfqpoint{0.511823in}{0.504323in}}{\pgfqpoint{3.218177in}{3.225677in}} %
\pgfusepath{clip}%
\pgfsetbuttcap%
\pgfsetroundjoin%
\definecolor{currentfill}{rgb}{0.501961,0.000000,0.000000}%
\pgfsetfillcolor{currentfill}%
\pgfsetfillopacity{0.400000}%
\pgfsetlinewidth{0.501875pt}%
\definecolor{currentstroke}{rgb}{0.501961,0.000000,0.000000}%
\pgfsetstrokecolor{currentstroke}%
\pgfsetstrokeopacity{0.400000}%
\pgfsetdash{}{0pt}%
\pgfpathmoveto{\pgfqpoint{2.603933in}{2.352349in}}%
\pgfpathcurveto{\pgfqpoint{2.614983in}{2.352349in}}{\pgfqpoint{2.625582in}{2.356739in}}{\pgfqpoint{2.633396in}{2.364553in}}%
\pgfpathcurveto{\pgfqpoint{2.641210in}{2.372367in}}{\pgfqpoint{2.645600in}{2.382966in}}{\pgfqpoint{2.645600in}{2.394016in}}%
\pgfpathcurveto{\pgfqpoint{2.645600in}{2.405066in}}{\pgfqpoint{2.641210in}{2.415665in}}{\pgfqpoint{2.633396in}{2.423479in}}%
\pgfpathcurveto{\pgfqpoint{2.625582in}{2.431292in}}{\pgfqpoint{2.614983in}{2.435682in}}{\pgfqpoint{2.603933in}{2.435682in}}%
\pgfpathcurveto{\pgfqpoint{2.592883in}{2.435682in}}{\pgfqpoint{2.582284in}{2.431292in}}{\pgfqpoint{2.574470in}{2.423479in}}%
\pgfpathcurveto{\pgfqpoint{2.566657in}{2.415665in}}{\pgfqpoint{2.562266in}{2.405066in}}{\pgfqpoint{2.562266in}{2.394016in}}%
\pgfpathcurveto{\pgfqpoint{2.562266in}{2.382966in}}{\pgfqpoint{2.566657in}{2.372367in}}{\pgfqpoint{2.574470in}{2.364553in}}%
\pgfpathcurveto{\pgfqpoint{2.582284in}{2.356739in}}{\pgfqpoint{2.592883in}{2.352349in}}{\pgfqpoint{2.603933in}{2.352349in}}%
\pgfpathclose%
\pgfusepath{stroke,fill}%
\end{pgfscope}%
\begin{pgfscope}%
\pgfpathrectangle{\pgfqpoint{0.511823in}{0.504323in}}{\pgfqpoint{3.218177in}{3.225677in}} %
\pgfusepath{clip}%
\pgfsetbuttcap%
\pgfsetroundjoin%
\definecolor{currentfill}{rgb}{0.501961,0.000000,0.000000}%
\pgfsetfillcolor{currentfill}%
\pgfsetfillopacity{0.400000}%
\pgfsetlinewidth{0.501875pt}%
\definecolor{currentstroke}{rgb}{0.501961,0.000000,0.000000}%
\pgfsetstrokecolor{currentstroke}%
\pgfsetstrokeopacity{0.400000}%
\pgfsetdash{}{0pt}%
\pgfpathmoveto{\pgfqpoint{2.455734in}{2.229974in}}%
\pgfpathcurveto{\pgfqpoint{2.466784in}{2.229974in}}{\pgfqpoint{2.477383in}{2.234364in}}{\pgfqpoint{2.485196in}{2.242178in}}%
\pgfpathcurveto{\pgfqpoint{2.493010in}{2.249991in}}{\pgfqpoint{2.497400in}{2.260590in}}{\pgfqpoint{2.497400in}{2.271640in}}%
\pgfpathcurveto{\pgfqpoint{2.497400in}{2.282691in}}{\pgfqpoint{2.493010in}{2.293290in}}{\pgfqpoint{2.485196in}{2.301103in}}%
\pgfpathcurveto{\pgfqpoint{2.477383in}{2.308917in}}{\pgfqpoint{2.466784in}{2.313307in}}{\pgfqpoint{2.455734in}{2.313307in}}%
\pgfpathcurveto{\pgfqpoint{2.444684in}{2.313307in}}{\pgfqpoint{2.434084in}{2.308917in}}{\pgfqpoint{2.426271in}{2.301103in}}%
\pgfpathcurveto{\pgfqpoint{2.418457in}{2.293290in}}{\pgfqpoint{2.414067in}{2.282691in}}{\pgfqpoint{2.414067in}{2.271640in}}%
\pgfpathcurveto{\pgfqpoint{2.414067in}{2.260590in}}{\pgfqpoint{2.418457in}{2.249991in}}{\pgfqpoint{2.426271in}{2.242178in}}%
\pgfpathcurveto{\pgfqpoint{2.434084in}{2.234364in}}{\pgfqpoint{2.444684in}{2.229974in}}{\pgfqpoint{2.455734in}{2.229974in}}%
\pgfpathclose%
\pgfusepath{stroke,fill}%
\end{pgfscope}%
\begin{pgfscope}%
\pgfpathrectangle{\pgfqpoint{0.511823in}{0.504323in}}{\pgfqpoint{3.218177in}{3.225677in}} %
\pgfusepath{clip}%
\pgfsetbuttcap%
\pgfsetroundjoin%
\definecolor{currentfill}{rgb}{0.501961,0.000000,0.000000}%
\pgfsetfillcolor{currentfill}%
\pgfsetfillopacity{0.400000}%
\pgfsetlinewidth{0.501875pt}%
\definecolor{currentstroke}{rgb}{0.501961,0.000000,0.000000}%
\pgfsetstrokecolor{currentstroke}%
\pgfsetstrokeopacity{0.400000}%
\pgfsetdash{}{0pt}%
\pgfpathmoveto{\pgfqpoint{2.548090in}{2.323555in}}%
\pgfpathcurveto{\pgfqpoint{2.559140in}{2.323555in}}{\pgfqpoint{2.569739in}{2.327945in}}{\pgfqpoint{2.577553in}{2.335759in}}%
\pgfpathcurveto{\pgfqpoint{2.585366in}{2.343572in}}{\pgfqpoint{2.589757in}{2.354171in}}{\pgfqpoint{2.589757in}{2.365221in}}%
\pgfpathcurveto{\pgfqpoint{2.589757in}{2.376272in}}{\pgfqpoint{2.585366in}{2.386871in}}{\pgfqpoint{2.577553in}{2.394684in}}%
\pgfpathcurveto{\pgfqpoint{2.569739in}{2.402498in}}{\pgfqpoint{2.559140in}{2.406888in}}{\pgfqpoint{2.548090in}{2.406888in}}%
\pgfpathcurveto{\pgfqpoint{2.537040in}{2.406888in}}{\pgfqpoint{2.526441in}{2.402498in}}{\pgfqpoint{2.518627in}{2.394684in}}%
\pgfpathcurveto{\pgfqpoint{2.510813in}{2.386871in}}{\pgfqpoint{2.506423in}{2.376272in}}{\pgfqpoint{2.506423in}{2.365221in}}%
\pgfpathcurveto{\pgfqpoint{2.506423in}{2.354171in}}{\pgfqpoint{2.510813in}{2.343572in}}{\pgfqpoint{2.518627in}{2.335759in}}%
\pgfpathcurveto{\pgfqpoint{2.526441in}{2.327945in}}{\pgfqpoint{2.537040in}{2.323555in}}{\pgfqpoint{2.548090in}{2.323555in}}%
\pgfpathclose%
\pgfusepath{stroke,fill}%
\end{pgfscope}%
\begin{pgfscope}%
\pgfpathrectangle{\pgfqpoint{0.511823in}{0.504323in}}{\pgfqpoint{3.218177in}{3.225677in}} %
\pgfusepath{clip}%
\pgfsetbuttcap%
\pgfsetroundjoin%
\definecolor{currentfill}{rgb}{0.501961,0.000000,0.000000}%
\pgfsetfillcolor{currentfill}%
\pgfsetfillopacity{0.400000}%
\pgfsetlinewidth{0.501875pt}%
\definecolor{currentstroke}{rgb}{0.501961,0.000000,0.000000}%
\pgfsetstrokecolor{currentstroke}%
\pgfsetstrokeopacity{0.400000}%
\pgfsetdash{}{0pt}%
\pgfpathmoveto{\pgfqpoint{2.624356in}{2.403608in}}%
\pgfpathcurveto{\pgfqpoint{2.635406in}{2.403608in}}{\pgfqpoint{2.646005in}{2.407998in}}{\pgfqpoint{2.653819in}{2.415812in}}%
\pgfpathcurveto{\pgfqpoint{2.661632in}{2.423625in}}{\pgfqpoint{2.666023in}{2.434224in}}{\pgfqpoint{2.666023in}{2.445275in}}%
\pgfpathcurveto{\pgfqpoint{2.666023in}{2.456325in}}{\pgfqpoint{2.661632in}{2.466924in}}{\pgfqpoint{2.653819in}{2.474737in}}%
\pgfpathcurveto{\pgfqpoint{2.646005in}{2.482551in}}{\pgfqpoint{2.635406in}{2.486941in}}{\pgfqpoint{2.624356in}{2.486941in}}%
\pgfpathcurveto{\pgfqpoint{2.613306in}{2.486941in}}{\pgfqpoint{2.602707in}{2.482551in}}{\pgfqpoint{2.594893in}{2.474737in}}%
\pgfpathcurveto{\pgfqpoint{2.587079in}{2.466924in}}{\pgfqpoint{2.582689in}{2.456325in}}{\pgfqpoint{2.582689in}{2.445275in}}%
\pgfpathcurveto{\pgfqpoint{2.582689in}{2.434224in}}{\pgfqpoint{2.587079in}{2.423625in}}{\pgfqpoint{2.594893in}{2.415812in}}%
\pgfpathcurveto{\pgfqpoint{2.602707in}{2.407998in}}{\pgfqpoint{2.613306in}{2.403608in}}{\pgfqpoint{2.624356in}{2.403608in}}%
\pgfpathclose%
\pgfusepath{stroke,fill}%
\end{pgfscope}%
\begin{pgfscope}%
\pgfpathrectangle{\pgfqpoint{0.511823in}{0.504323in}}{\pgfqpoint{3.218177in}{3.225677in}} %
\pgfusepath{clip}%
\pgfsetbuttcap%
\pgfsetroundjoin%
\definecolor{currentfill}{rgb}{0.501961,0.000000,0.000000}%
\pgfsetfillcolor{currentfill}%
\pgfsetfillopacity{0.400000}%
\pgfsetlinewidth{0.501875pt}%
\definecolor{currentstroke}{rgb}{0.501961,0.000000,0.000000}%
\pgfsetstrokecolor{currentstroke}%
\pgfsetstrokeopacity{0.400000}%
\pgfsetdash{}{0pt}%
\pgfpathmoveto{\pgfqpoint{2.513609in}{2.313314in}}%
\pgfpathcurveto{\pgfqpoint{2.524659in}{2.313314in}}{\pgfqpoint{2.535258in}{2.317704in}}{\pgfqpoint{2.543072in}{2.325517in}}%
\pgfpathcurveto{\pgfqpoint{2.550885in}{2.333331in}}{\pgfqpoint{2.555275in}{2.343930in}}{\pgfqpoint{2.555275in}{2.354980in}}%
\pgfpathcurveto{\pgfqpoint{2.555275in}{2.366030in}}{\pgfqpoint{2.550885in}{2.376629in}}{\pgfqpoint{2.543072in}{2.384443in}}%
\pgfpathcurveto{\pgfqpoint{2.535258in}{2.392257in}}{\pgfqpoint{2.524659in}{2.396647in}}{\pgfqpoint{2.513609in}{2.396647in}}%
\pgfpathcurveto{\pgfqpoint{2.502559in}{2.396647in}}{\pgfqpoint{2.491960in}{2.392257in}}{\pgfqpoint{2.484146in}{2.384443in}}%
\pgfpathcurveto{\pgfqpoint{2.476332in}{2.376629in}}{\pgfqpoint{2.471942in}{2.366030in}}{\pgfqpoint{2.471942in}{2.354980in}}%
\pgfpathcurveto{\pgfqpoint{2.471942in}{2.343930in}}{\pgfqpoint{2.476332in}{2.333331in}}{\pgfqpoint{2.484146in}{2.325517in}}%
\pgfpathcurveto{\pgfqpoint{2.491960in}{2.317704in}}{\pgfqpoint{2.502559in}{2.313314in}}{\pgfqpoint{2.513609in}{2.313314in}}%
\pgfpathclose%
\pgfusepath{stroke,fill}%
\end{pgfscope}%
\begin{pgfscope}%
\pgfpathrectangle{\pgfqpoint{0.511823in}{0.504323in}}{\pgfqpoint{3.218177in}{3.225677in}} %
\pgfusepath{clip}%
\pgfsetbuttcap%
\pgfsetroundjoin%
\definecolor{currentfill}{rgb}{0.501961,0.000000,0.000000}%
\pgfsetfillcolor{currentfill}%
\pgfsetfillopacity{0.400000}%
\pgfsetlinewidth{0.501875pt}%
\definecolor{currentstroke}{rgb}{0.501961,0.000000,0.000000}%
\pgfsetstrokecolor{currentstroke}%
\pgfsetstrokeopacity{0.400000}%
\pgfsetdash{}{0pt}%
\pgfpathmoveto{\pgfqpoint{2.632348in}{2.433327in}}%
\pgfpathcurveto{\pgfqpoint{2.643398in}{2.433327in}}{\pgfqpoint{2.653997in}{2.437717in}}{\pgfqpoint{2.661811in}{2.445531in}}%
\pgfpathcurveto{\pgfqpoint{2.669624in}{2.453345in}}{\pgfqpoint{2.674014in}{2.463944in}}{\pgfqpoint{2.674014in}{2.474994in}}%
\pgfpathcurveto{\pgfqpoint{2.674014in}{2.486044in}}{\pgfqpoint{2.669624in}{2.496643in}}{\pgfqpoint{2.661811in}{2.504456in}}%
\pgfpathcurveto{\pgfqpoint{2.653997in}{2.512270in}}{\pgfqpoint{2.643398in}{2.516660in}}{\pgfqpoint{2.632348in}{2.516660in}}%
\pgfpathcurveto{\pgfqpoint{2.621298in}{2.516660in}}{\pgfqpoint{2.610699in}{2.512270in}}{\pgfqpoint{2.602885in}{2.504456in}}%
\pgfpathcurveto{\pgfqpoint{2.595071in}{2.496643in}}{\pgfqpoint{2.590681in}{2.486044in}}{\pgfqpoint{2.590681in}{2.474994in}}%
\pgfpathcurveto{\pgfqpoint{2.590681in}{2.463944in}}{\pgfqpoint{2.595071in}{2.453345in}}{\pgfqpoint{2.602885in}{2.445531in}}%
\pgfpathcurveto{\pgfqpoint{2.610699in}{2.437717in}}{\pgfqpoint{2.621298in}{2.433327in}}{\pgfqpoint{2.632348in}{2.433327in}}%
\pgfpathclose%
\pgfusepath{stroke,fill}%
\end{pgfscope}%
\begin{pgfscope}%
\pgfpathrectangle{\pgfqpoint{0.511823in}{0.504323in}}{\pgfqpoint{3.218177in}{3.225677in}} %
\pgfusepath{clip}%
\pgfsetbuttcap%
\pgfsetroundjoin%
\definecolor{currentfill}{rgb}{0.501961,0.000000,0.000000}%
\pgfsetfillcolor{currentfill}%
\pgfsetfillopacity{0.400000}%
\pgfsetlinewidth{0.501875pt}%
\definecolor{currentstroke}{rgb}{0.501961,0.000000,0.000000}%
\pgfsetstrokecolor{currentstroke}%
\pgfsetstrokeopacity{0.400000}%
\pgfsetdash{}{0pt}%
\pgfpathmoveto{\pgfqpoint{2.640767in}{2.452464in}}%
\pgfpathcurveto{\pgfqpoint{2.651817in}{2.452464in}}{\pgfqpoint{2.662416in}{2.456854in}}{\pgfqpoint{2.670229in}{2.464668in}}%
\pgfpathcurveto{\pgfqpoint{2.678043in}{2.472481in}}{\pgfqpoint{2.682433in}{2.483080in}}{\pgfqpoint{2.682433in}{2.494130in}}%
\pgfpathcurveto{\pgfqpoint{2.682433in}{2.505180in}}{\pgfqpoint{2.678043in}{2.515779in}}{\pgfqpoint{2.670229in}{2.523593in}}%
\pgfpathcurveto{\pgfqpoint{2.662416in}{2.531407in}}{\pgfqpoint{2.651817in}{2.535797in}}{\pgfqpoint{2.640767in}{2.535797in}}%
\pgfpathcurveto{\pgfqpoint{2.629717in}{2.535797in}}{\pgfqpoint{2.619118in}{2.531407in}}{\pgfqpoint{2.611304in}{2.523593in}}%
\pgfpathcurveto{\pgfqpoint{2.603490in}{2.515779in}}{\pgfqpoint{2.599100in}{2.505180in}}{\pgfqpoint{2.599100in}{2.494130in}}%
\pgfpathcurveto{\pgfqpoint{2.599100in}{2.483080in}}{\pgfqpoint{2.603490in}{2.472481in}}{\pgfqpoint{2.611304in}{2.464668in}}%
\pgfpathcurveto{\pgfqpoint{2.619118in}{2.456854in}}{\pgfqpoint{2.629717in}{2.452464in}}{\pgfqpoint{2.640767in}{2.452464in}}%
\pgfpathclose%
\pgfusepath{stroke,fill}%
\end{pgfscope}%
\begin{pgfscope}%
\pgfpathrectangle{\pgfqpoint{0.511823in}{0.504323in}}{\pgfqpoint{3.218177in}{3.225677in}} %
\pgfusepath{clip}%
\pgfsetbuttcap%
\pgfsetroundjoin%
\definecolor{currentfill}{rgb}{0.501961,0.000000,0.000000}%
\pgfsetfillcolor{currentfill}%
\pgfsetfillopacity{0.400000}%
\pgfsetlinewidth{0.501875pt}%
\definecolor{currentstroke}{rgb}{0.501961,0.000000,0.000000}%
\pgfsetstrokecolor{currentstroke}%
\pgfsetstrokeopacity{0.400000}%
\pgfsetdash{}{0pt}%
\pgfpathmoveto{\pgfqpoint{2.503882in}{2.336145in}}%
\pgfpathcurveto{\pgfqpoint{2.514932in}{2.336145in}}{\pgfqpoint{2.525531in}{2.340535in}}{\pgfqpoint{2.533345in}{2.348349in}}%
\pgfpathcurveto{\pgfqpoint{2.541158in}{2.356163in}}{\pgfqpoint{2.545549in}{2.366762in}}{\pgfqpoint{2.545549in}{2.377812in}}%
\pgfpathcurveto{\pgfqpoint{2.545549in}{2.388862in}}{\pgfqpoint{2.541158in}{2.399461in}}{\pgfqpoint{2.533345in}{2.407275in}}%
\pgfpathcurveto{\pgfqpoint{2.525531in}{2.415088in}}{\pgfqpoint{2.514932in}{2.419478in}}{\pgfqpoint{2.503882in}{2.419478in}}%
\pgfpathcurveto{\pgfqpoint{2.492832in}{2.419478in}}{\pgfqpoint{2.482233in}{2.415088in}}{\pgfqpoint{2.474419in}{2.407275in}}%
\pgfpathcurveto{\pgfqpoint{2.466606in}{2.399461in}}{\pgfqpoint{2.462215in}{2.388862in}}{\pgfqpoint{2.462215in}{2.377812in}}%
\pgfpathcurveto{\pgfqpoint{2.462215in}{2.366762in}}{\pgfqpoint{2.466606in}{2.356163in}}{\pgfqpoint{2.474419in}{2.348349in}}%
\pgfpathcurveto{\pgfqpoint{2.482233in}{2.340535in}}{\pgfqpoint{2.492832in}{2.336145in}}{\pgfqpoint{2.503882in}{2.336145in}}%
\pgfpathclose%
\pgfusepath{stroke,fill}%
\end{pgfscope}%
\begin{pgfscope}%
\pgfpathrectangle{\pgfqpoint{0.511823in}{0.504323in}}{\pgfqpoint{3.218177in}{3.225677in}} %
\pgfusepath{clip}%
\pgfsetbuttcap%
\pgfsetroundjoin%
\definecolor{currentfill}{rgb}{0.501961,0.000000,0.000000}%
\pgfsetfillcolor{currentfill}%
\pgfsetfillopacity{0.400000}%
\pgfsetlinewidth{0.501875pt}%
\definecolor{currentstroke}{rgb}{0.501961,0.000000,0.000000}%
\pgfsetstrokecolor{currentstroke}%
\pgfsetstrokeopacity{0.400000}%
\pgfsetdash{}{0pt}%
\pgfpathmoveto{\pgfqpoint{2.329691in}{2.183244in}}%
\pgfpathcurveto{\pgfqpoint{2.340741in}{2.183244in}}{\pgfqpoint{2.351340in}{2.187634in}}{\pgfqpoint{2.359153in}{2.195448in}}%
\pgfpathcurveto{\pgfqpoint{2.366967in}{2.203261in}}{\pgfqpoint{2.371357in}{2.213860in}}{\pgfqpoint{2.371357in}{2.224910in}}%
\pgfpathcurveto{\pgfqpoint{2.371357in}{2.235960in}}{\pgfqpoint{2.366967in}{2.246560in}}{\pgfqpoint{2.359153in}{2.254373in}}%
\pgfpathcurveto{\pgfqpoint{2.351340in}{2.262187in}}{\pgfqpoint{2.340741in}{2.266577in}}{\pgfqpoint{2.329691in}{2.266577in}}%
\pgfpathcurveto{\pgfqpoint{2.318641in}{2.266577in}}{\pgfqpoint{2.308041in}{2.262187in}}{\pgfqpoint{2.300228in}{2.254373in}}%
\pgfpathcurveto{\pgfqpoint{2.292414in}{2.246560in}}{\pgfqpoint{2.288024in}{2.235960in}}{\pgfqpoint{2.288024in}{2.224910in}}%
\pgfpathcurveto{\pgfqpoint{2.288024in}{2.213860in}}{\pgfqpoint{2.292414in}{2.203261in}}{\pgfqpoint{2.300228in}{2.195448in}}%
\pgfpathcurveto{\pgfqpoint{2.308041in}{2.187634in}}{\pgfqpoint{2.318641in}{2.183244in}}{\pgfqpoint{2.329691in}{2.183244in}}%
\pgfpathclose%
\pgfusepath{stroke,fill}%
\end{pgfscope}%
\begin{pgfscope}%
\pgfpathrectangle{\pgfqpoint{0.511823in}{0.504323in}}{\pgfqpoint{3.218177in}{3.225677in}} %
\pgfusepath{clip}%
\pgfsetbuttcap%
\pgfsetroundjoin%
\definecolor{currentfill}{rgb}{0.501961,0.000000,0.000000}%
\pgfsetfillcolor{currentfill}%
\pgfsetfillopacity{0.400000}%
\pgfsetlinewidth{0.501875pt}%
\definecolor{currentstroke}{rgb}{0.501961,0.000000,0.000000}%
\pgfsetstrokecolor{currentstroke}%
\pgfsetstrokeopacity{0.400000}%
\pgfsetdash{}{0pt}%
\pgfpathmoveto{\pgfqpoint{2.629704in}{2.476571in}}%
\pgfpathcurveto{\pgfqpoint{2.640754in}{2.476571in}}{\pgfqpoint{2.651353in}{2.480961in}}{\pgfqpoint{2.659167in}{2.488774in}}%
\pgfpathcurveto{\pgfqpoint{2.666980in}{2.496588in}}{\pgfqpoint{2.671371in}{2.507187in}}{\pgfqpoint{2.671371in}{2.518237in}}%
\pgfpathcurveto{\pgfqpoint{2.671371in}{2.529287in}}{\pgfqpoint{2.666980in}{2.539886in}}{\pgfqpoint{2.659167in}{2.547700in}}%
\pgfpathcurveto{\pgfqpoint{2.651353in}{2.555514in}}{\pgfqpoint{2.640754in}{2.559904in}}{\pgfqpoint{2.629704in}{2.559904in}}%
\pgfpathcurveto{\pgfqpoint{2.618654in}{2.559904in}}{\pgfqpoint{2.608055in}{2.555514in}}{\pgfqpoint{2.600241in}{2.547700in}}%
\pgfpathcurveto{\pgfqpoint{2.592428in}{2.539886in}}{\pgfqpoint{2.588037in}{2.529287in}}{\pgfqpoint{2.588037in}{2.518237in}}%
\pgfpathcurveto{\pgfqpoint{2.588037in}{2.507187in}}{\pgfqpoint{2.592428in}{2.496588in}}{\pgfqpoint{2.600241in}{2.488774in}}%
\pgfpathcurveto{\pgfqpoint{2.608055in}{2.480961in}}{\pgfqpoint{2.618654in}{2.476571in}}{\pgfqpoint{2.629704in}{2.476571in}}%
\pgfpathclose%
\pgfusepath{stroke,fill}%
\end{pgfscope}%
\begin{pgfscope}%
\pgfpathrectangle{\pgfqpoint{0.511823in}{0.504323in}}{\pgfqpoint{3.218177in}{3.225677in}} %
\pgfusepath{clip}%
\pgfsetbuttcap%
\pgfsetroundjoin%
\definecolor{currentfill}{rgb}{0.501961,0.000000,0.000000}%
\pgfsetfillcolor{currentfill}%
\pgfsetfillopacity{0.400000}%
\pgfsetlinewidth{0.501875pt}%
\definecolor{currentstroke}{rgb}{0.501961,0.000000,0.000000}%
\pgfsetstrokecolor{currentstroke}%
\pgfsetstrokeopacity{0.400000}%
\pgfsetdash{}{0pt}%
\pgfpathmoveto{\pgfqpoint{2.666787in}{2.523447in}}%
\pgfpathcurveto{\pgfqpoint{2.677838in}{2.523447in}}{\pgfqpoint{2.688437in}{2.527837in}}{\pgfqpoint{2.696250in}{2.535651in}}%
\pgfpathcurveto{\pgfqpoint{2.704064in}{2.543464in}}{\pgfqpoint{2.708454in}{2.554063in}}{\pgfqpoint{2.708454in}{2.565113in}}%
\pgfpathcurveto{\pgfqpoint{2.708454in}{2.576163in}}{\pgfqpoint{2.704064in}{2.586763in}}{\pgfqpoint{2.696250in}{2.594576in}}%
\pgfpathcurveto{\pgfqpoint{2.688437in}{2.602390in}}{\pgfqpoint{2.677838in}{2.606780in}}{\pgfqpoint{2.666787in}{2.606780in}}%
\pgfpathcurveto{\pgfqpoint{2.655737in}{2.606780in}}{\pgfqpoint{2.645138in}{2.602390in}}{\pgfqpoint{2.637325in}{2.594576in}}%
\pgfpathcurveto{\pgfqpoint{2.629511in}{2.586763in}}{\pgfqpoint{2.625121in}{2.576163in}}{\pgfqpoint{2.625121in}{2.565113in}}%
\pgfpathcurveto{\pgfqpoint{2.625121in}{2.554063in}}{\pgfqpoint{2.629511in}{2.543464in}}{\pgfqpoint{2.637325in}{2.535651in}}%
\pgfpathcurveto{\pgfqpoint{2.645138in}{2.527837in}}{\pgfqpoint{2.655737in}{2.523447in}}{\pgfqpoint{2.666787in}{2.523447in}}%
\pgfpathclose%
\pgfusepath{stroke,fill}%
\end{pgfscope}%
\begin{pgfscope}%
\pgfpathrectangle{\pgfqpoint{0.511823in}{0.504323in}}{\pgfqpoint{3.218177in}{3.225677in}} %
\pgfusepath{clip}%
\pgfsetbuttcap%
\pgfsetroundjoin%
\definecolor{currentfill}{rgb}{0.501961,0.000000,0.000000}%
\pgfsetfillcolor{currentfill}%
\pgfsetfillopacity{0.400000}%
\pgfsetlinewidth{0.501875pt}%
\definecolor{currentstroke}{rgb}{0.501961,0.000000,0.000000}%
\pgfsetstrokecolor{currentstroke}%
\pgfsetstrokeopacity{0.400000}%
\pgfsetdash{}{0pt}%
\pgfpathmoveto{\pgfqpoint{2.388711in}{2.269172in}}%
\pgfpathcurveto{\pgfqpoint{2.399761in}{2.269172in}}{\pgfqpoint{2.410360in}{2.273563in}}{\pgfqpoint{2.418173in}{2.281376in}}%
\pgfpathcurveto{\pgfqpoint{2.425987in}{2.289190in}}{\pgfqpoint{2.430377in}{2.299789in}}{\pgfqpoint{2.430377in}{2.310839in}}%
\pgfpathcurveto{\pgfqpoint{2.430377in}{2.321889in}}{\pgfqpoint{2.425987in}{2.332488in}}{\pgfqpoint{2.418173in}{2.340302in}}%
\pgfpathcurveto{\pgfqpoint{2.410360in}{2.348116in}}{\pgfqpoint{2.399761in}{2.352506in}}{\pgfqpoint{2.388711in}{2.352506in}}%
\pgfpathcurveto{\pgfqpoint{2.377660in}{2.352506in}}{\pgfqpoint{2.367061in}{2.348116in}}{\pgfqpoint{2.359248in}{2.340302in}}%
\pgfpathcurveto{\pgfqpoint{2.351434in}{2.332488in}}{\pgfqpoint{2.347044in}{2.321889in}}{\pgfqpoint{2.347044in}{2.310839in}}%
\pgfpathcurveto{\pgfqpoint{2.347044in}{2.299789in}}{\pgfqpoint{2.351434in}{2.289190in}}{\pgfqpoint{2.359248in}{2.281376in}}%
\pgfpathcurveto{\pgfqpoint{2.367061in}{2.273563in}}{\pgfqpoint{2.377660in}{2.269172in}}{\pgfqpoint{2.388711in}{2.269172in}}%
\pgfpathclose%
\pgfusepath{stroke,fill}%
\end{pgfscope}%
\begin{pgfscope}%
\pgfpathrectangle{\pgfqpoint{0.511823in}{0.504323in}}{\pgfqpoint{3.218177in}{3.225677in}} %
\pgfusepath{clip}%
\pgfsetbuttcap%
\pgfsetroundjoin%
\definecolor{currentfill}{rgb}{0.501961,0.000000,0.000000}%
\pgfsetfillcolor{currentfill}%
\pgfsetfillopacity{0.400000}%
\pgfsetlinewidth{0.501875pt}%
\definecolor{currentstroke}{rgb}{0.501961,0.000000,0.000000}%
\pgfsetstrokecolor{currentstroke}%
\pgfsetstrokeopacity{0.400000}%
\pgfsetdash{}{0pt}%
\pgfpathmoveto{\pgfqpoint{2.549166in}{2.434026in}}%
\pgfpathcurveto{\pgfqpoint{2.560216in}{2.434026in}}{\pgfqpoint{2.570815in}{2.438417in}}{\pgfqpoint{2.578629in}{2.446230in}}%
\pgfpathcurveto{\pgfqpoint{2.586442in}{2.454044in}}{\pgfqpoint{2.590832in}{2.464643in}}{\pgfqpoint{2.590832in}{2.475693in}}%
\pgfpathcurveto{\pgfqpoint{2.590832in}{2.486743in}}{\pgfqpoint{2.586442in}{2.497342in}}{\pgfqpoint{2.578629in}{2.505156in}}%
\pgfpathcurveto{\pgfqpoint{2.570815in}{2.512969in}}{\pgfqpoint{2.560216in}{2.517360in}}{\pgfqpoint{2.549166in}{2.517360in}}%
\pgfpathcurveto{\pgfqpoint{2.538116in}{2.517360in}}{\pgfqpoint{2.527517in}{2.512969in}}{\pgfqpoint{2.519703in}{2.505156in}}%
\pgfpathcurveto{\pgfqpoint{2.511889in}{2.497342in}}{\pgfqpoint{2.507499in}{2.486743in}}{\pgfqpoint{2.507499in}{2.475693in}}%
\pgfpathcurveto{\pgfqpoint{2.507499in}{2.464643in}}{\pgfqpoint{2.511889in}{2.454044in}}{\pgfqpoint{2.519703in}{2.446230in}}%
\pgfpathcurveto{\pgfqpoint{2.527517in}{2.438417in}}{\pgfqpoint{2.538116in}{2.434026in}}{\pgfqpoint{2.549166in}{2.434026in}}%
\pgfpathclose%
\pgfusepath{stroke,fill}%
\end{pgfscope}%
\begin{pgfscope}%
\pgfpathrectangle{\pgfqpoint{0.511823in}{0.504323in}}{\pgfqpoint{3.218177in}{3.225677in}} %
\pgfusepath{clip}%
\pgfsetbuttcap%
\pgfsetroundjoin%
\definecolor{currentfill}{rgb}{0.501961,0.000000,0.000000}%
\pgfsetfillcolor{currentfill}%
\pgfsetfillopacity{0.400000}%
\pgfsetlinewidth{0.501875pt}%
\definecolor{currentstroke}{rgb}{0.501961,0.000000,0.000000}%
\pgfsetstrokecolor{currentstroke}%
\pgfsetstrokeopacity{0.400000}%
\pgfsetdash{}{0pt}%
\pgfpathmoveto{\pgfqpoint{2.553790in}{2.449849in}}%
\pgfpathcurveto{\pgfqpoint{2.564840in}{2.449849in}}{\pgfqpoint{2.575439in}{2.454239in}}{\pgfqpoint{2.583253in}{2.462053in}}%
\pgfpathcurveto{\pgfqpoint{2.591067in}{2.469866in}}{\pgfqpoint{2.595457in}{2.480465in}}{\pgfqpoint{2.595457in}{2.491516in}}%
\pgfpathcurveto{\pgfqpoint{2.595457in}{2.502566in}}{\pgfqpoint{2.591067in}{2.513165in}}{\pgfqpoint{2.583253in}{2.520978in}}%
\pgfpathcurveto{\pgfqpoint{2.575439in}{2.528792in}}{\pgfqpoint{2.564840in}{2.533182in}}{\pgfqpoint{2.553790in}{2.533182in}}%
\pgfpathcurveto{\pgfqpoint{2.542740in}{2.533182in}}{\pgfqpoint{2.532141in}{2.528792in}}{\pgfqpoint{2.524327in}{2.520978in}}%
\pgfpathcurveto{\pgfqpoint{2.516514in}{2.513165in}}{\pgfqpoint{2.512123in}{2.502566in}}{\pgfqpoint{2.512123in}{2.491516in}}%
\pgfpathcurveto{\pgfqpoint{2.512123in}{2.480465in}}{\pgfqpoint{2.516514in}{2.469866in}}{\pgfqpoint{2.524327in}{2.462053in}}%
\pgfpathcurveto{\pgfqpoint{2.532141in}{2.454239in}}{\pgfqpoint{2.542740in}{2.449849in}}{\pgfqpoint{2.553790in}{2.449849in}}%
\pgfpathclose%
\pgfusepath{stroke,fill}%
\end{pgfscope}%
\begin{pgfscope}%
\pgfpathrectangle{\pgfqpoint{0.511823in}{0.504323in}}{\pgfqpoint{3.218177in}{3.225677in}} %
\pgfusepath{clip}%
\pgfsetbuttcap%
\pgfsetroundjoin%
\definecolor{currentfill}{rgb}{0.501961,0.000000,0.000000}%
\pgfsetfillcolor{currentfill}%
\pgfsetfillopacity{0.400000}%
\pgfsetlinewidth{0.501875pt}%
\definecolor{currentstroke}{rgb}{0.501961,0.000000,0.000000}%
\pgfsetstrokecolor{currentstroke}%
\pgfsetstrokeopacity{0.400000}%
\pgfsetdash{}{0pt}%
\pgfpathmoveto{\pgfqpoint{2.369888in}{2.281902in}}%
\pgfpathcurveto{\pgfqpoint{2.380938in}{2.281902in}}{\pgfqpoint{2.391537in}{2.286292in}}{\pgfqpoint{2.399350in}{2.294106in}}%
\pgfpathcurveto{\pgfqpoint{2.407164in}{2.301919in}}{\pgfqpoint{2.411554in}{2.312518in}}{\pgfqpoint{2.411554in}{2.323569in}}%
\pgfpathcurveto{\pgfqpoint{2.411554in}{2.334619in}}{\pgfqpoint{2.407164in}{2.345218in}}{\pgfqpoint{2.399350in}{2.353031in}}%
\pgfpathcurveto{\pgfqpoint{2.391537in}{2.360845in}}{\pgfqpoint{2.380938in}{2.365235in}}{\pgfqpoint{2.369888in}{2.365235in}}%
\pgfpathcurveto{\pgfqpoint{2.358837in}{2.365235in}}{\pgfqpoint{2.348238in}{2.360845in}}{\pgfqpoint{2.340425in}{2.353031in}}%
\pgfpathcurveto{\pgfqpoint{2.332611in}{2.345218in}}{\pgfqpoint{2.328221in}{2.334619in}}{\pgfqpoint{2.328221in}{2.323569in}}%
\pgfpathcurveto{\pgfqpoint{2.328221in}{2.312518in}}{\pgfqpoint{2.332611in}{2.301919in}}{\pgfqpoint{2.340425in}{2.294106in}}%
\pgfpathcurveto{\pgfqpoint{2.348238in}{2.286292in}}{\pgfqpoint{2.358837in}{2.281902in}}{\pgfqpoint{2.369888in}{2.281902in}}%
\pgfpathclose%
\pgfusepath{stroke,fill}%
\end{pgfscope}%
\begin{pgfscope}%
\pgfpathrectangle{\pgfqpoint{0.511823in}{0.504323in}}{\pgfqpoint{3.218177in}{3.225677in}} %
\pgfusepath{clip}%
\pgfsetbuttcap%
\pgfsetroundjoin%
\definecolor{currentfill}{rgb}{0.501961,0.000000,0.000000}%
\pgfsetfillcolor{currentfill}%
\pgfsetfillopacity{0.400000}%
\pgfsetlinewidth{0.501875pt}%
\definecolor{currentstroke}{rgb}{0.501961,0.000000,0.000000}%
\pgfsetstrokecolor{currentstroke}%
\pgfsetstrokeopacity{0.400000}%
\pgfsetdash{}{0pt}%
\pgfpathmoveto{\pgfqpoint{2.462865in}{2.383556in}}%
\pgfpathcurveto{\pgfqpoint{2.473915in}{2.383556in}}{\pgfqpoint{2.484514in}{2.387946in}}{\pgfqpoint{2.492328in}{2.395759in}}%
\pgfpathcurveto{\pgfqpoint{2.500141in}{2.403573in}}{\pgfqpoint{2.504531in}{2.414172in}}{\pgfqpoint{2.504531in}{2.425222in}}%
\pgfpathcurveto{\pgfqpoint{2.504531in}{2.436272in}}{\pgfqpoint{2.500141in}{2.446871in}}{\pgfqpoint{2.492328in}{2.454685in}}%
\pgfpathcurveto{\pgfqpoint{2.484514in}{2.462499in}}{\pgfqpoint{2.473915in}{2.466889in}}{\pgfqpoint{2.462865in}{2.466889in}}%
\pgfpathcurveto{\pgfqpoint{2.451815in}{2.466889in}}{\pgfqpoint{2.441216in}{2.462499in}}{\pgfqpoint{2.433402in}{2.454685in}}%
\pgfpathcurveto{\pgfqpoint{2.425588in}{2.446871in}}{\pgfqpoint{2.421198in}{2.436272in}}{\pgfqpoint{2.421198in}{2.425222in}}%
\pgfpathcurveto{\pgfqpoint{2.421198in}{2.414172in}}{\pgfqpoint{2.425588in}{2.403573in}}{\pgfqpoint{2.433402in}{2.395759in}}%
\pgfpathcurveto{\pgfqpoint{2.441216in}{2.387946in}}{\pgfqpoint{2.451815in}{2.383556in}}{\pgfqpoint{2.462865in}{2.383556in}}%
\pgfpathclose%
\pgfusepath{stroke,fill}%
\end{pgfscope}%
\begin{pgfscope}%
\pgfpathrectangle{\pgfqpoint{0.511823in}{0.504323in}}{\pgfqpoint{3.218177in}{3.225677in}} %
\pgfusepath{clip}%
\pgfsetbuttcap%
\pgfsetroundjoin%
\definecolor{currentfill}{rgb}{0.501961,0.000000,0.000000}%
\pgfsetfillcolor{currentfill}%
\pgfsetfillopacity{0.400000}%
\pgfsetlinewidth{0.501875pt}%
\definecolor{currentstroke}{rgb}{0.501961,0.000000,0.000000}%
\pgfsetstrokecolor{currentstroke}%
\pgfsetstrokeopacity{0.400000}%
\pgfsetdash{}{0pt}%
\pgfpathmoveto{\pgfqpoint{2.619981in}{2.549792in}}%
\pgfpathcurveto{\pgfqpoint{2.631031in}{2.549792in}}{\pgfqpoint{2.641630in}{2.554182in}}{\pgfqpoint{2.649444in}{2.561996in}}%
\pgfpathcurveto{\pgfqpoint{2.657257in}{2.569810in}}{\pgfqpoint{2.661648in}{2.580409in}}{\pgfqpoint{2.661648in}{2.591459in}}%
\pgfpathcurveto{\pgfqpoint{2.661648in}{2.602509in}}{\pgfqpoint{2.657257in}{2.613108in}}{\pgfqpoint{2.649444in}{2.620922in}}%
\pgfpathcurveto{\pgfqpoint{2.641630in}{2.628735in}}{\pgfqpoint{2.631031in}{2.633126in}}{\pgfqpoint{2.619981in}{2.633126in}}%
\pgfpathcurveto{\pgfqpoint{2.608931in}{2.633126in}}{\pgfqpoint{2.598332in}{2.628735in}}{\pgfqpoint{2.590518in}{2.620922in}}%
\pgfpathcurveto{\pgfqpoint{2.582704in}{2.613108in}}{\pgfqpoint{2.578314in}{2.602509in}}{\pgfqpoint{2.578314in}{2.591459in}}%
\pgfpathcurveto{\pgfqpoint{2.578314in}{2.580409in}}{\pgfqpoint{2.582704in}{2.569810in}}{\pgfqpoint{2.590518in}{2.561996in}}%
\pgfpathcurveto{\pgfqpoint{2.598332in}{2.554182in}}{\pgfqpoint{2.608931in}{2.549792in}}{\pgfqpoint{2.619981in}{2.549792in}}%
\pgfpathclose%
\pgfusepath{stroke,fill}%
\end{pgfscope}%
\begin{pgfscope}%
\pgfpathrectangle{\pgfqpoint{0.511823in}{0.504323in}}{\pgfqpoint{3.218177in}{3.225677in}} %
\pgfusepath{clip}%
\pgfsetbuttcap%
\pgfsetroundjoin%
\definecolor{currentfill}{rgb}{0.501961,0.000000,0.000000}%
\pgfsetfillcolor{currentfill}%
\pgfsetfillopacity{0.400000}%
\pgfsetlinewidth{0.501875pt}%
\definecolor{currentstroke}{rgb}{0.501961,0.000000,0.000000}%
\pgfsetstrokecolor{currentstroke}%
\pgfsetstrokeopacity{0.400000}%
\pgfsetdash{}{0pt}%
\pgfpathmoveto{\pgfqpoint{2.520494in}{2.462980in}}%
\pgfpathcurveto{\pgfqpoint{2.531544in}{2.462980in}}{\pgfqpoint{2.542143in}{2.467370in}}{\pgfqpoint{2.549956in}{2.475184in}}%
\pgfpathcurveto{\pgfqpoint{2.557770in}{2.482997in}}{\pgfqpoint{2.562160in}{2.493596in}}{\pgfqpoint{2.562160in}{2.504646in}}%
\pgfpathcurveto{\pgfqpoint{2.562160in}{2.515697in}}{\pgfqpoint{2.557770in}{2.526296in}}{\pgfqpoint{2.549956in}{2.534109in}}%
\pgfpathcurveto{\pgfqpoint{2.542143in}{2.541923in}}{\pgfqpoint{2.531544in}{2.546313in}}{\pgfqpoint{2.520494in}{2.546313in}}%
\pgfpathcurveto{\pgfqpoint{2.509443in}{2.546313in}}{\pgfqpoint{2.498844in}{2.541923in}}{\pgfqpoint{2.491031in}{2.534109in}}%
\pgfpathcurveto{\pgfqpoint{2.483217in}{2.526296in}}{\pgfqpoint{2.478827in}{2.515697in}}{\pgfqpoint{2.478827in}{2.504646in}}%
\pgfpathcurveto{\pgfqpoint{2.478827in}{2.493596in}}{\pgfqpoint{2.483217in}{2.482997in}}{\pgfqpoint{2.491031in}{2.475184in}}%
\pgfpathcurveto{\pgfqpoint{2.498844in}{2.467370in}}{\pgfqpoint{2.509443in}{2.462980in}}{\pgfqpoint{2.520494in}{2.462980in}}%
\pgfpathclose%
\pgfusepath{stroke,fill}%
\end{pgfscope}%
\begin{pgfscope}%
\pgfpathrectangle{\pgfqpoint{0.511823in}{0.504323in}}{\pgfqpoint{3.218177in}{3.225677in}} %
\pgfusepath{clip}%
\pgfsetbuttcap%
\pgfsetroundjoin%
\definecolor{currentfill}{rgb}{0.501961,0.000000,0.000000}%
\pgfsetfillcolor{currentfill}%
\pgfsetfillopacity{0.400000}%
\pgfsetlinewidth{0.501875pt}%
\definecolor{currentstroke}{rgb}{0.501961,0.000000,0.000000}%
\pgfsetstrokecolor{currentstroke}%
\pgfsetstrokeopacity{0.400000}%
\pgfsetdash{}{0pt}%
\pgfpathmoveto{\pgfqpoint{2.496126in}{2.450146in}}%
\pgfpathcurveto{\pgfqpoint{2.507177in}{2.450146in}}{\pgfqpoint{2.517776in}{2.454536in}}{\pgfqpoint{2.525589in}{2.462350in}}%
\pgfpathcurveto{\pgfqpoint{2.533403in}{2.470163in}}{\pgfqpoint{2.537793in}{2.480762in}}{\pgfqpoint{2.537793in}{2.491813in}}%
\pgfpathcurveto{\pgfqpoint{2.537793in}{2.502863in}}{\pgfqpoint{2.533403in}{2.513462in}}{\pgfqpoint{2.525589in}{2.521275in}}%
\pgfpathcurveto{\pgfqpoint{2.517776in}{2.529089in}}{\pgfqpoint{2.507177in}{2.533479in}}{\pgfqpoint{2.496126in}{2.533479in}}%
\pgfpathcurveto{\pgfqpoint{2.485076in}{2.533479in}}{\pgfqpoint{2.474477in}{2.529089in}}{\pgfqpoint{2.466664in}{2.521275in}}%
\pgfpathcurveto{\pgfqpoint{2.458850in}{2.513462in}}{\pgfqpoint{2.454460in}{2.502863in}}{\pgfqpoint{2.454460in}{2.491813in}}%
\pgfpathcurveto{\pgfqpoint{2.454460in}{2.480762in}}{\pgfqpoint{2.458850in}{2.470163in}}{\pgfqpoint{2.466664in}{2.462350in}}%
\pgfpathcurveto{\pgfqpoint{2.474477in}{2.454536in}}{\pgfqpoint{2.485076in}{2.450146in}}{\pgfqpoint{2.496126in}{2.450146in}}%
\pgfpathclose%
\pgfusepath{stroke,fill}%
\end{pgfscope}%
\begin{pgfscope}%
\pgfpathrectangle{\pgfqpoint{0.511823in}{0.504323in}}{\pgfqpoint{3.218177in}{3.225677in}} %
\pgfusepath{clip}%
\pgfsetbuttcap%
\pgfsetroundjoin%
\definecolor{currentfill}{rgb}{0.501961,0.000000,0.000000}%
\pgfsetfillcolor{currentfill}%
\pgfsetfillopacity{0.400000}%
\pgfsetlinewidth{0.501875pt}%
\definecolor{currentstroke}{rgb}{0.501961,0.000000,0.000000}%
\pgfsetstrokecolor{currentstroke}%
\pgfsetstrokeopacity{0.400000}%
\pgfsetdash{}{0pt}%
\pgfpathmoveto{\pgfqpoint{2.592792in}{2.558928in}}%
\pgfpathcurveto{\pgfqpoint{2.603842in}{2.558928in}}{\pgfqpoint{2.614441in}{2.563318in}}{\pgfqpoint{2.622255in}{2.571132in}}%
\pgfpathcurveto{\pgfqpoint{2.630069in}{2.578946in}}{\pgfqpoint{2.634459in}{2.589545in}}{\pgfqpoint{2.634459in}{2.600595in}}%
\pgfpathcurveto{\pgfqpoint{2.634459in}{2.611645in}}{\pgfqpoint{2.630069in}{2.622244in}}{\pgfqpoint{2.622255in}{2.630057in}}%
\pgfpathcurveto{\pgfqpoint{2.614441in}{2.637871in}}{\pgfqpoint{2.603842in}{2.642261in}}{\pgfqpoint{2.592792in}{2.642261in}}%
\pgfpathcurveto{\pgfqpoint{2.581742in}{2.642261in}}{\pgfqpoint{2.571143in}{2.637871in}}{\pgfqpoint{2.563330in}{2.630057in}}%
\pgfpathcurveto{\pgfqpoint{2.555516in}{2.622244in}}{\pgfqpoint{2.551126in}{2.611645in}}{\pgfqpoint{2.551126in}{2.600595in}}%
\pgfpathcurveto{\pgfqpoint{2.551126in}{2.589545in}}{\pgfqpoint{2.555516in}{2.578946in}}{\pgfqpoint{2.563330in}{2.571132in}}%
\pgfpathcurveto{\pgfqpoint{2.571143in}{2.563318in}}{\pgfqpoint{2.581742in}{2.558928in}}{\pgfqpoint{2.592792in}{2.558928in}}%
\pgfpathclose%
\pgfusepath{stroke,fill}%
\end{pgfscope}%
\begin{pgfscope}%
\pgfpathrectangle{\pgfqpoint{0.511823in}{0.504323in}}{\pgfqpoint{3.218177in}{3.225677in}} %
\pgfusepath{clip}%
\pgfsetbuttcap%
\pgfsetroundjoin%
\definecolor{currentfill}{rgb}{0.501961,0.000000,0.000000}%
\pgfsetfillcolor{currentfill}%
\pgfsetfillopacity{0.400000}%
\pgfsetlinewidth{0.501875pt}%
\definecolor{currentstroke}{rgb}{0.501961,0.000000,0.000000}%
\pgfsetstrokecolor{currentstroke}%
\pgfsetstrokeopacity{0.400000}%
\pgfsetdash{}{0pt}%
\pgfpathmoveto{\pgfqpoint{2.429685in}{2.405845in}}%
\pgfpathcurveto{\pgfqpoint{2.440735in}{2.405845in}}{\pgfqpoint{2.451334in}{2.410236in}}{\pgfqpoint{2.459148in}{2.418049in}}%
\pgfpathcurveto{\pgfqpoint{2.466961in}{2.425863in}}{\pgfqpoint{2.471352in}{2.436462in}}{\pgfqpoint{2.471352in}{2.447512in}}%
\pgfpathcurveto{\pgfqpoint{2.471352in}{2.458562in}}{\pgfqpoint{2.466961in}{2.469161in}}{\pgfqpoint{2.459148in}{2.476975in}}%
\pgfpathcurveto{\pgfqpoint{2.451334in}{2.484788in}}{\pgfqpoint{2.440735in}{2.489179in}}{\pgfqpoint{2.429685in}{2.489179in}}%
\pgfpathcurveto{\pgfqpoint{2.418635in}{2.489179in}}{\pgfqpoint{2.408036in}{2.484788in}}{\pgfqpoint{2.400222in}{2.476975in}}%
\pgfpathcurveto{\pgfqpoint{2.392408in}{2.469161in}}{\pgfqpoint{2.388018in}{2.458562in}}{\pgfqpoint{2.388018in}{2.447512in}}%
\pgfpathcurveto{\pgfqpoint{2.388018in}{2.436462in}}{\pgfqpoint{2.392408in}{2.425863in}}{\pgfqpoint{2.400222in}{2.418049in}}%
\pgfpathcurveto{\pgfqpoint{2.408036in}{2.410236in}}{\pgfqpoint{2.418635in}{2.405845in}}{\pgfqpoint{2.429685in}{2.405845in}}%
\pgfpathclose%
\pgfusepath{stroke,fill}%
\end{pgfscope}%
\begin{pgfscope}%
\pgfpathrectangle{\pgfqpoint{0.511823in}{0.504323in}}{\pgfqpoint{3.218177in}{3.225677in}} %
\pgfusepath{clip}%
\pgfsetbuttcap%
\pgfsetroundjoin%
\definecolor{currentfill}{rgb}{0.501961,0.000000,0.000000}%
\pgfsetfillcolor{currentfill}%
\pgfsetfillopacity{0.400000}%
\pgfsetlinewidth{0.501875pt}%
\definecolor{currentstroke}{rgb}{0.501961,0.000000,0.000000}%
\pgfsetstrokecolor{currentstroke}%
\pgfsetstrokeopacity{0.400000}%
\pgfsetdash{}{0pt}%
\pgfpathmoveto{\pgfqpoint{2.315413in}{2.300540in}}%
\pgfpathcurveto{\pgfqpoint{2.326463in}{2.300540in}}{\pgfqpoint{2.337062in}{2.304930in}}{\pgfqpoint{2.344876in}{2.312744in}}%
\pgfpathcurveto{\pgfqpoint{2.352689in}{2.320557in}}{\pgfqpoint{2.357079in}{2.331156in}}{\pgfqpoint{2.357079in}{2.342206in}}%
\pgfpathcurveto{\pgfqpoint{2.357079in}{2.353256in}}{\pgfqpoint{2.352689in}{2.363856in}}{\pgfqpoint{2.344876in}{2.371669in}}%
\pgfpathcurveto{\pgfqpoint{2.337062in}{2.379483in}}{\pgfqpoint{2.326463in}{2.383873in}}{\pgfqpoint{2.315413in}{2.383873in}}%
\pgfpathcurveto{\pgfqpoint{2.304363in}{2.383873in}}{\pgfqpoint{2.293764in}{2.379483in}}{\pgfqpoint{2.285950in}{2.371669in}}%
\pgfpathcurveto{\pgfqpoint{2.278136in}{2.363856in}}{\pgfqpoint{2.273746in}{2.353256in}}{\pgfqpoint{2.273746in}{2.342206in}}%
\pgfpathcurveto{\pgfqpoint{2.273746in}{2.331156in}}{\pgfqpoint{2.278136in}{2.320557in}}{\pgfqpoint{2.285950in}{2.312744in}}%
\pgfpathcurveto{\pgfqpoint{2.293764in}{2.304930in}}{\pgfqpoint{2.304363in}{2.300540in}}{\pgfqpoint{2.315413in}{2.300540in}}%
\pgfpathclose%
\pgfusepath{stroke,fill}%
\end{pgfscope}%
\begin{pgfscope}%
\pgfpathrectangle{\pgfqpoint{0.511823in}{0.504323in}}{\pgfqpoint{3.218177in}{3.225677in}} %
\pgfusepath{clip}%
\pgfsetbuttcap%
\pgfsetroundjoin%
\definecolor{currentfill}{rgb}{0.501961,0.000000,0.000000}%
\pgfsetfillcolor{currentfill}%
\pgfsetfillopacity{0.400000}%
\pgfsetlinewidth{0.501875pt}%
\definecolor{currentstroke}{rgb}{0.501961,0.000000,0.000000}%
\pgfsetstrokecolor{currentstroke}%
\pgfsetstrokeopacity{0.400000}%
\pgfsetdash{}{0pt}%
\pgfpathmoveto{\pgfqpoint{2.590881in}{2.593704in}}%
\pgfpathcurveto{\pgfqpoint{2.601931in}{2.593704in}}{\pgfqpoint{2.612531in}{2.598095in}}{\pgfqpoint{2.620344in}{2.605908in}}%
\pgfpathcurveto{\pgfqpoint{2.628158in}{2.613722in}}{\pgfqpoint{2.632548in}{2.624321in}}{\pgfqpoint{2.632548in}{2.635371in}}%
\pgfpathcurveto{\pgfqpoint{2.632548in}{2.646421in}}{\pgfqpoint{2.628158in}{2.657020in}}{\pgfqpoint{2.620344in}{2.664834in}}%
\pgfpathcurveto{\pgfqpoint{2.612531in}{2.672647in}}{\pgfqpoint{2.601931in}{2.677038in}}{\pgfqpoint{2.590881in}{2.677038in}}%
\pgfpathcurveto{\pgfqpoint{2.579831in}{2.677038in}}{\pgfqpoint{2.569232in}{2.672647in}}{\pgfqpoint{2.561419in}{2.664834in}}%
\pgfpathcurveto{\pgfqpoint{2.553605in}{2.657020in}}{\pgfqpoint{2.549215in}{2.646421in}}{\pgfqpoint{2.549215in}{2.635371in}}%
\pgfpathcurveto{\pgfqpoint{2.549215in}{2.624321in}}{\pgfqpoint{2.553605in}{2.613722in}}{\pgfqpoint{2.561419in}{2.605908in}}%
\pgfpathcurveto{\pgfqpoint{2.569232in}{2.598095in}}{\pgfqpoint{2.579831in}{2.593704in}}{\pgfqpoint{2.590881in}{2.593704in}}%
\pgfpathclose%
\pgfusepath{stroke,fill}%
\end{pgfscope}%
\begin{pgfscope}%
\pgfpathrectangle{\pgfqpoint{0.511823in}{0.504323in}}{\pgfqpoint{3.218177in}{3.225677in}} %
\pgfusepath{clip}%
\pgfsetbuttcap%
\pgfsetroundjoin%
\definecolor{currentfill}{rgb}{0.501961,0.000000,0.000000}%
\pgfsetfillcolor{currentfill}%
\pgfsetfillopacity{0.400000}%
\pgfsetlinewidth{0.501875pt}%
\definecolor{currentstroke}{rgb}{0.501961,0.000000,0.000000}%
\pgfsetstrokecolor{currentstroke}%
\pgfsetstrokeopacity{0.400000}%
\pgfsetdash{}{0pt}%
\pgfpathmoveto{\pgfqpoint{2.394989in}{2.403848in}}%
\pgfpathcurveto{\pgfqpoint{2.406039in}{2.403848in}}{\pgfqpoint{2.416638in}{2.408238in}}{\pgfqpoint{2.424452in}{2.416052in}}%
\pgfpathcurveto{\pgfqpoint{2.432265in}{2.423865in}}{\pgfqpoint{2.436656in}{2.434464in}}{\pgfqpoint{2.436656in}{2.445514in}}%
\pgfpathcurveto{\pgfqpoint{2.436656in}{2.456565in}}{\pgfqpoint{2.432265in}{2.467164in}}{\pgfqpoint{2.424452in}{2.474977in}}%
\pgfpathcurveto{\pgfqpoint{2.416638in}{2.482791in}}{\pgfqpoint{2.406039in}{2.487181in}}{\pgfqpoint{2.394989in}{2.487181in}}%
\pgfpathcurveto{\pgfqpoint{2.383939in}{2.487181in}}{\pgfqpoint{2.373340in}{2.482791in}}{\pgfqpoint{2.365526in}{2.474977in}}%
\pgfpathcurveto{\pgfqpoint{2.357712in}{2.467164in}}{\pgfqpoint{2.353322in}{2.456565in}}{\pgfqpoint{2.353322in}{2.445514in}}%
\pgfpathcurveto{\pgfqpoint{2.353322in}{2.434464in}}{\pgfqpoint{2.357712in}{2.423865in}}{\pgfqpoint{2.365526in}{2.416052in}}%
\pgfpathcurveto{\pgfqpoint{2.373340in}{2.408238in}}{\pgfqpoint{2.383939in}{2.403848in}}{\pgfqpoint{2.394989in}{2.403848in}}%
\pgfpathclose%
\pgfusepath{stroke,fill}%
\end{pgfscope}%
\begin{pgfscope}%
\pgfpathrectangle{\pgfqpoint{0.511823in}{0.504323in}}{\pgfqpoint{3.218177in}{3.225677in}} %
\pgfusepath{clip}%
\pgfsetbuttcap%
\pgfsetroundjoin%
\definecolor{currentfill}{rgb}{0.501961,0.000000,0.000000}%
\pgfsetfillcolor{currentfill}%
\pgfsetfillopacity{0.400000}%
\pgfsetlinewidth{0.501875pt}%
\definecolor{currentstroke}{rgb}{0.501961,0.000000,0.000000}%
\pgfsetstrokecolor{currentstroke}%
\pgfsetstrokeopacity{0.400000}%
\pgfsetdash{}{0pt}%
\pgfpathmoveto{\pgfqpoint{2.458004in}{2.480530in}}%
\pgfpathcurveto{\pgfqpoint{2.469054in}{2.480530in}}{\pgfqpoint{2.479653in}{2.484920in}}{\pgfqpoint{2.487467in}{2.492734in}}%
\pgfpathcurveto{\pgfqpoint{2.495280in}{2.500547in}}{\pgfqpoint{2.499670in}{2.511147in}}{\pgfqpoint{2.499670in}{2.522197in}}%
\pgfpathcurveto{\pgfqpoint{2.499670in}{2.533247in}}{\pgfqpoint{2.495280in}{2.543846in}}{\pgfqpoint{2.487467in}{2.551659in}}%
\pgfpathcurveto{\pgfqpoint{2.479653in}{2.559473in}}{\pgfqpoint{2.469054in}{2.563863in}}{\pgfqpoint{2.458004in}{2.563863in}}%
\pgfpathcurveto{\pgfqpoint{2.446954in}{2.563863in}}{\pgfqpoint{2.436355in}{2.559473in}}{\pgfqpoint{2.428541in}{2.551659in}}%
\pgfpathcurveto{\pgfqpoint{2.420727in}{2.543846in}}{\pgfqpoint{2.416337in}{2.533247in}}{\pgfqpoint{2.416337in}{2.522197in}}%
\pgfpathcurveto{\pgfqpoint{2.416337in}{2.511147in}}{\pgfqpoint{2.420727in}{2.500547in}}{\pgfqpoint{2.428541in}{2.492734in}}%
\pgfpathcurveto{\pgfqpoint{2.436355in}{2.484920in}}{\pgfqpoint{2.446954in}{2.480530in}}{\pgfqpoint{2.458004in}{2.480530in}}%
\pgfpathclose%
\pgfusepath{stroke,fill}%
\end{pgfscope}%
\begin{pgfscope}%
\pgfpathrectangle{\pgfqpoint{0.511823in}{0.504323in}}{\pgfqpoint{3.218177in}{3.225677in}} %
\pgfusepath{clip}%
\pgfsetbuttcap%
\pgfsetroundjoin%
\definecolor{currentfill}{rgb}{0.501961,0.000000,0.000000}%
\pgfsetfillcolor{currentfill}%
\pgfsetfillopacity{0.400000}%
\pgfsetlinewidth{0.501875pt}%
\definecolor{currentstroke}{rgb}{0.501961,0.000000,0.000000}%
\pgfsetstrokecolor{currentstroke}%
\pgfsetstrokeopacity{0.400000}%
\pgfsetdash{}{0pt}%
\pgfpathmoveto{\pgfqpoint{2.554847in}{2.593497in}}%
\pgfpathcurveto{\pgfqpoint{2.565897in}{2.593497in}}{\pgfqpoint{2.576496in}{2.597887in}}{\pgfqpoint{2.584310in}{2.605701in}}%
\pgfpathcurveto{\pgfqpoint{2.592123in}{2.613514in}}{\pgfqpoint{2.596514in}{2.624113in}}{\pgfqpoint{2.596514in}{2.635163in}}%
\pgfpathcurveto{\pgfqpoint{2.596514in}{2.646213in}}{\pgfqpoint{2.592123in}{2.656813in}}{\pgfqpoint{2.584310in}{2.664626in}}%
\pgfpathcurveto{\pgfqpoint{2.576496in}{2.672440in}}{\pgfqpoint{2.565897in}{2.676830in}}{\pgfqpoint{2.554847in}{2.676830in}}%
\pgfpathcurveto{\pgfqpoint{2.543797in}{2.676830in}}{\pgfqpoint{2.533198in}{2.672440in}}{\pgfqpoint{2.525384in}{2.664626in}}%
\pgfpathcurveto{\pgfqpoint{2.517570in}{2.656813in}}{\pgfqpoint{2.513180in}{2.646213in}}{\pgfqpoint{2.513180in}{2.635163in}}%
\pgfpathcurveto{\pgfqpoint{2.513180in}{2.624113in}}{\pgfqpoint{2.517570in}{2.613514in}}{\pgfqpoint{2.525384in}{2.605701in}}%
\pgfpathcurveto{\pgfqpoint{2.533198in}{2.597887in}}{\pgfqpoint{2.543797in}{2.593497in}}{\pgfqpoint{2.554847in}{2.593497in}}%
\pgfpathclose%
\pgfusepath{stroke,fill}%
\end{pgfscope}%
\begin{pgfscope}%
\pgfpathrectangle{\pgfqpoint{0.511823in}{0.504323in}}{\pgfqpoint{3.218177in}{3.225677in}} %
\pgfusepath{clip}%
\pgfsetbuttcap%
\pgfsetroundjoin%
\definecolor{currentfill}{rgb}{0.501961,0.000000,0.000000}%
\pgfsetfillcolor{currentfill}%
\pgfsetfillopacity{0.400000}%
\pgfsetlinewidth{0.501875pt}%
\definecolor{currentstroke}{rgb}{0.501961,0.000000,0.000000}%
\pgfsetstrokecolor{currentstroke}%
\pgfsetstrokeopacity{0.400000}%
\pgfsetdash{}{0pt}%
\pgfpathmoveto{\pgfqpoint{2.470900in}{2.517586in}}%
\pgfpathcurveto{\pgfqpoint{2.481950in}{2.517586in}}{\pgfqpoint{2.492549in}{2.521976in}}{\pgfqpoint{2.500363in}{2.529790in}}%
\pgfpathcurveto{\pgfqpoint{2.508176in}{2.537603in}}{\pgfqpoint{2.512567in}{2.548202in}}{\pgfqpoint{2.512567in}{2.559253in}}%
\pgfpathcurveto{\pgfqpoint{2.512567in}{2.570303in}}{\pgfqpoint{2.508176in}{2.580902in}}{\pgfqpoint{2.500363in}{2.588715in}}%
\pgfpathcurveto{\pgfqpoint{2.492549in}{2.596529in}}{\pgfqpoint{2.481950in}{2.600919in}}{\pgfqpoint{2.470900in}{2.600919in}}%
\pgfpathcurveto{\pgfqpoint{2.459850in}{2.600919in}}{\pgfqpoint{2.449251in}{2.596529in}}{\pgfqpoint{2.441437in}{2.588715in}}%
\pgfpathcurveto{\pgfqpoint{2.433624in}{2.580902in}}{\pgfqpoint{2.429233in}{2.570303in}}{\pgfqpoint{2.429233in}{2.559253in}}%
\pgfpathcurveto{\pgfqpoint{2.429233in}{2.548202in}}{\pgfqpoint{2.433624in}{2.537603in}}{\pgfqpoint{2.441437in}{2.529790in}}%
\pgfpathcurveto{\pgfqpoint{2.449251in}{2.521976in}}{\pgfqpoint{2.459850in}{2.517586in}}{\pgfqpoint{2.470900in}{2.517586in}}%
\pgfpathclose%
\pgfusepath{stroke,fill}%
\end{pgfscope}%
\begin{pgfscope}%
\pgfpathrectangle{\pgfqpoint{0.511823in}{0.504323in}}{\pgfqpoint{3.218177in}{3.225677in}} %
\pgfusepath{clip}%
\pgfsetbuttcap%
\pgfsetroundjoin%
\definecolor{currentfill}{rgb}{0.501961,0.000000,0.000000}%
\pgfsetfillcolor{currentfill}%
\pgfsetfillopacity{0.400000}%
\pgfsetlinewidth{0.501875pt}%
\definecolor{currentstroke}{rgb}{0.501961,0.000000,0.000000}%
\pgfsetstrokecolor{currentstroke}%
\pgfsetstrokeopacity{0.400000}%
\pgfsetdash{}{0pt}%
\pgfpathmoveto{\pgfqpoint{2.454883in}{2.512578in}}%
\pgfpathcurveto{\pgfqpoint{2.465933in}{2.512578in}}{\pgfqpoint{2.476532in}{2.516969in}}{\pgfqpoint{2.484346in}{2.524782in}}%
\pgfpathcurveto{\pgfqpoint{2.492160in}{2.532596in}}{\pgfqpoint{2.496550in}{2.543195in}}{\pgfqpoint{2.496550in}{2.554245in}}%
\pgfpathcurveto{\pgfqpoint{2.496550in}{2.565295in}}{\pgfqpoint{2.492160in}{2.575894in}}{\pgfqpoint{2.484346in}{2.583708in}}%
\pgfpathcurveto{\pgfqpoint{2.476532in}{2.591521in}}{\pgfqpoint{2.465933in}{2.595912in}}{\pgfqpoint{2.454883in}{2.595912in}}%
\pgfpathcurveto{\pgfqpoint{2.443833in}{2.595912in}}{\pgfqpoint{2.433234in}{2.591521in}}{\pgfqpoint{2.425420in}{2.583708in}}%
\pgfpathcurveto{\pgfqpoint{2.417607in}{2.575894in}}{\pgfqpoint{2.413216in}{2.565295in}}{\pgfqpoint{2.413216in}{2.554245in}}%
\pgfpathcurveto{\pgfqpoint{2.413216in}{2.543195in}}{\pgfqpoint{2.417607in}{2.532596in}}{\pgfqpoint{2.425420in}{2.524782in}}%
\pgfpathcurveto{\pgfqpoint{2.433234in}{2.516969in}}{\pgfqpoint{2.443833in}{2.512578in}}{\pgfqpoint{2.454883in}{2.512578in}}%
\pgfpathclose%
\pgfusepath{stroke,fill}%
\end{pgfscope}%
\begin{pgfscope}%
\pgfpathrectangle{\pgfqpoint{0.511823in}{0.504323in}}{\pgfqpoint{3.218177in}{3.225677in}} %
\pgfusepath{clip}%
\pgfsetbuttcap%
\pgfsetroundjoin%
\definecolor{currentfill}{rgb}{0.501961,0.000000,0.000000}%
\pgfsetfillcolor{currentfill}%
\pgfsetfillopacity{0.400000}%
\pgfsetlinewidth{0.501875pt}%
\definecolor{currentstroke}{rgb}{0.501961,0.000000,0.000000}%
\pgfsetstrokecolor{currentstroke}%
\pgfsetstrokeopacity{0.400000}%
\pgfsetdash{}{0pt}%
\pgfpathmoveto{\pgfqpoint{2.381119in}{2.445895in}}%
\pgfpathcurveto{\pgfqpoint{2.392169in}{2.445895in}}{\pgfqpoint{2.402768in}{2.450285in}}{\pgfqpoint{2.410581in}{2.458099in}}%
\pgfpathcurveto{\pgfqpoint{2.418395in}{2.465913in}}{\pgfqpoint{2.422785in}{2.476512in}}{\pgfqpoint{2.422785in}{2.487562in}}%
\pgfpathcurveto{\pgfqpoint{2.422785in}{2.498612in}}{\pgfqpoint{2.418395in}{2.509211in}}{\pgfqpoint{2.410581in}{2.517025in}}%
\pgfpathcurveto{\pgfqpoint{2.402768in}{2.524838in}}{\pgfqpoint{2.392169in}{2.529228in}}{\pgfqpoint{2.381119in}{2.529228in}}%
\pgfpathcurveto{\pgfqpoint{2.370069in}{2.529228in}}{\pgfqpoint{2.359470in}{2.524838in}}{\pgfqpoint{2.351656in}{2.517025in}}%
\pgfpathcurveto{\pgfqpoint{2.343842in}{2.509211in}}{\pgfqpoint{2.339452in}{2.498612in}}{\pgfqpoint{2.339452in}{2.487562in}}%
\pgfpathcurveto{\pgfqpoint{2.339452in}{2.476512in}}{\pgfqpoint{2.343842in}{2.465913in}}{\pgfqpoint{2.351656in}{2.458099in}}%
\pgfpathcurveto{\pgfqpoint{2.359470in}{2.450285in}}{\pgfqpoint{2.370069in}{2.445895in}}{\pgfqpoint{2.381119in}{2.445895in}}%
\pgfpathclose%
\pgfusepath{stroke,fill}%
\end{pgfscope}%
\begin{pgfscope}%
\pgfpathrectangle{\pgfqpoint{0.511823in}{0.504323in}}{\pgfqpoint{3.218177in}{3.225677in}} %
\pgfusepath{clip}%
\pgfsetbuttcap%
\pgfsetroundjoin%
\definecolor{currentfill}{rgb}{0.501961,0.000000,0.000000}%
\pgfsetfillcolor{currentfill}%
\pgfsetfillopacity{0.400000}%
\pgfsetlinewidth{0.501875pt}%
\definecolor{currentstroke}{rgb}{0.501961,0.000000,0.000000}%
\pgfsetstrokecolor{currentstroke}%
\pgfsetstrokeopacity{0.400000}%
\pgfsetdash{}{0pt}%
\pgfpathmoveto{\pgfqpoint{2.478369in}{2.561720in}}%
\pgfpathcurveto{\pgfqpoint{2.489419in}{2.561720in}}{\pgfqpoint{2.500018in}{2.566110in}}{\pgfqpoint{2.507832in}{2.573924in}}%
\pgfpathcurveto{\pgfqpoint{2.515645in}{2.581737in}}{\pgfqpoint{2.520035in}{2.592336in}}{\pgfqpoint{2.520035in}{2.603386in}}%
\pgfpathcurveto{\pgfqpoint{2.520035in}{2.614436in}}{\pgfqpoint{2.515645in}{2.625035in}}{\pgfqpoint{2.507832in}{2.632849in}}%
\pgfpathcurveto{\pgfqpoint{2.500018in}{2.640663in}}{\pgfqpoint{2.489419in}{2.645053in}}{\pgfqpoint{2.478369in}{2.645053in}}%
\pgfpathcurveto{\pgfqpoint{2.467319in}{2.645053in}}{\pgfqpoint{2.456720in}{2.640663in}}{\pgfqpoint{2.448906in}{2.632849in}}%
\pgfpathcurveto{\pgfqpoint{2.441092in}{2.625035in}}{\pgfqpoint{2.436702in}{2.614436in}}{\pgfqpoint{2.436702in}{2.603386in}}%
\pgfpathcurveto{\pgfqpoint{2.436702in}{2.592336in}}{\pgfqpoint{2.441092in}{2.581737in}}{\pgfqpoint{2.448906in}{2.573924in}}%
\pgfpathcurveto{\pgfqpoint{2.456720in}{2.566110in}}{\pgfqpoint{2.467319in}{2.561720in}}{\pgfqpoint{2.478369in}{2.561720in}}%
\pgfpathclose%
\pgfusepath{stroke,fill}%
\end{pgfscope}%
\begin{pgfscope}%
\pgfpathrectangle{\pgfqpoint{0.511823in}{0.504323in}}{\pgfqpoint{3.218177in}{3.225677in}} %
\pgfusepath{clip}%
\pgfsetbuttcap%
\pgfsetroundjoin%
\definecolor{currentfill}{rgb}{0.501961,0.000000,0.000000}%
\pgfsetfillcolor{currentfill}%
\pgfsetfillopacity{0.400000}%
\pgfsetlinewidth{0.501875pt}%
\definecolor{currentstroke}{rgb}{0.501961,0.000000,0.000000}%
\pgfsetstrokecolor{currentstroke}%
\pgfsetstrokeopacity{0.400000}%
\pgfsetdash{}{0pt}%
\pgfpathmoveto{\pgfqpoint{2.612112in}{2.718351in}}%
\pgfpathcurveto{\pgfqpoint{2.623162in}{2.718351in}}{\pgfqpoint{2.633761in}{2.722741in}}{\pgfqpoint{2.641575in}{2.730555in}}%
\pgfpathcurveto{\pgfqpoint{2.649389in}{2.738368in}}{\pgfqpoint{2.653779in}{2.748967in}}{\pgfqpoint{2.653779in}{2.760017in}}%
\pgfpathcurveto{\pgfqpoint{2.653779in}{2.771067in}}{\pgfqpoint{2.649389in}{2.781666in}}{\pgfqpoint{2.641575in}{2.789480in}}%
\pgfpathcurveto{\pgfqpoint{2.633761in}{2.797294in}}{\pgfqpoint{2.623162in}{2.801684in}}{\pgfqpoint{2.612112in}{2.801684in}}%
\pgfpathcurveto{\pgfqpoint{2.601062in}{2.801684in}}{\pgfqpoint{2.590463in}{2.797294in}}{\pgfqpoint{2.582649in}{2.789480in}}%
\pgfpathcurveto{\pgfqpoint{2.574836in}{2.781666in}}{\pgfqpoint{2.570446in}{2.771067in}}{\pgfqpoint{2.570446in}{2.760017in}}%
\pgfpathcurveto{\pgfqpoint{2.570446in}{2.748967in}}{\pgfqpoint{2.574836in}{2.738368in}}{\pgfqpoint{2.582649in}{2.730555in}}%
\pgfpathcurveto{\pgfqpoint{2.590463in}{2.722741in}}{\pgfqpoint{2.601062in}{2.718351in}}{\pgfqpoint{2.612112in}{2.718351in}}%
\pgfpathclose%
\pgfusepath{stroke,fill}%
\end{pgfscope}%
\begin{pgfscope}%
\pgfpathrectangle{\pgfqpoint{0.511823in}{0.504323in}}{\pgfqpoint{3.218177in}{3.225677in}} %
\pgfusepath{clip}%
\pgfsetbuttcap%
\pgfsetroundjoin%
\definecolor{currentfill}{rgb}{0.501961,0.000000,0.000000}%
\pgfsetfillcolor{currentfill}%
\pgfsetfillopacity{0.400000}%
\pgfsetlinewidth{0.501875pt}%
\definecolor{currentstroke}{rgb}{0.501961,0.000000,0.000000}%
\pgfsetstrokecolor{currentstroke}%
\pgfsetstrokeopacity{0.400000}%
\pgfsetdash{}{0pt}%
\pgfpathmoveto{\pgfqpoint{2.365758in}{2.464014in}}%
\pgfpathcurveto{\pgfqpoint{2.376808in}{2.464014in}}{\pgfqpoint{2.387407in}{2.468404in}}{\pgfqpoint{2.395220in}{2.476218in}}%
\pgfpathcurveto{\pgfqpoint{2.403034in}{2.484031in}}{\pgfqpoint{2.407424in}{2.494630in}}{\pgfqpoint{2.407424in}{2.505680in}}%
\pgfpathcurveto{\pgfqpoint{2.407424in}{2.516730in}}{\pgfqpoint{2.403034in}{2.527330in}}{\pgfqpoint{2.395220in}{2.535143in}}%
\pgfpathcurveto{\pgfqpoint{2.387407in}{2.542957in}}{\pgfqpoint{2.376808in}{2.547347in}}{\pgfqpoint{2.365758in}{2.547347in}}%
\pgfpathcurveto{\pgfqpoint{2.354707in}{2.547347in}}{\pgfqpoint{2.344108in}{2.542957in}}{\pgfqpoint{2.336295in}{2.535143in}}%
\pgfpathcurveto{\pgfqpoint{2.328481in}{2.527330in}}{\pgfqpoint{2.324091in}{2.516730in}}{\pgfqpoint{2.324091in}{2.505680in}}%
\pgfpathcurveto{\pgfqpoint{2.324091in}{2.494630in}}{\pgfqpoint{2.328481in}{2.484031in}}{\pgfqpoint{2.336295in}{2.476218in}}%
\pgfpathcurveto{\pgfqpoint{2.344108in}{2.468404in}}{\pgfqpoint{2.354707in}{2.464014in}}{\pgfqpoint{2.365758in}{2.464014in}}%
\pgfpathclose%
\pgfusepath{stroke,fill}%
\end{pgfscope}%
\begin{pgfscope}%
\pgfpathrectangle{\pgfqpoint{0.511823in}{0.504323in}}{\pgfqpoint{3.218177in}{3.225677in}} %
\pgfusepath{clip}%
\pgfsetbuttcap%
\pgfsetroundjoin%
\definecolor{currentfill}{rgb}{0.501961,0.000000,0.000000}%
\pgfsetfillcolor{currentfill}%
\pgfsetfillopacity{0.400000}%
\pgfsetlinewidth{0.501875pt}%
\definecolor{currentstroke}{rgb}{0.501961,0.000000,0.000000}%
\pgfsetstrokecolor{currentstroke}%
\pgfsetstrokeopacity{0.400000}%
\pgfsetdash{}{0pt}%
\pgfpathmoveto{\pgfqpoint{2.313379in}{2.418417in}}%
\pgfpathcurveto{\pgfqpoint{2.324429in}{2.418417in}}{\pgfqpoint{2.335028in}{2.422807in}}{\pgfqpoint{2.342842in}{2.430620in}}%
\pgfpathcurveto{\pgfqpoint{2.350655in}{2.438434in}}{\pgfqpoint{2.355046in}{2.449033in}}{\pgfqpoint{2.355046in}{2.460083in}}%
\pgfpathcurveto{\pgfqpoint{2.355046in}{2.471133in}}{\pgfqpoint{2.350655in}{2.481732in}}{\pgfqpoint{2.342842in}{2.489546in}}%
\pgfpathcurveto{\pgfqpoint{2.335028in}{2.497360in}}{\pgfqpoint{2.324429in}{2.501750in}}{\pgfqpoint{2.313379in}{2.501750in}}%
\pgfpathcurveto{\pgfqpoint{2.302329in}{2.501750in}}{\pgfqpoint{2.291730in}{2.497360in}}{\pgfqpoint{2.283916in}{2.489546in}}%
\pgfpathcurveto{\pgfqpoint{2.276102in}{2.481732in}}{\pgfqpoint{2.271712in}{2.471133in}}{\pgfqpoint{2.271712in}{2.460083in}}%
\pgfpathcurveto{\pgfqpoint{2.271712in}{2.449033in}}{\pgfqpoint{2.276102in}{2.438434in}}{\pgfqpoint{2.283916in}{2.430620in}}%
\pgfpathcurveto{\pgfqpoint{2.291730in}{2.422807in}}{\pgfqpoint{2.302329in}{2.418417in}}{\pgfqpoint{2.313379in}{2.418417in}}%
\pgfpathclose%
\pgfusepath{stroke,fill}%
\end{pgfscope}%
\begin{pgfscope}%
\pgfpathrectangle{\pgfqpoint{0.511823in}{0.504323in}}{\pgfqpoint{3.218177in}{3.225677in}} %
\pgfusepath{clip}%
\pgfsetbuttcap%
\pgfsetroundjoin%
\definecolor{currentfill}{rgb}{0.501961,0.000000,0.000000}%
\pgfsetfillcolor{currentfill}%
\pgfsetfillopacity{0.400000}%
\pgfsetlinewidth{0.501875pt}%
\definecolor{currentstroke}{rgb}{0.501961,0.000000,0.000000}%
\pgfsetstrokecolor{currentstroke}%
\pgfsetstrokeopacity{0.400000}%
\pgfsetdash{}{0pt}%
\pgfpathmoveto{\pgfqpoint{2.429008in}{2.557024in}}%
\pgfpathcurveto{\pgfqpoint{2.440058in}{2.557024in}}{\pgfqpoint{2.450657in}{2.561414in}}{\pgfqpoint{2.458471in}{2.569227in}}%
\pgfpathcurveto{\pgfqpoint{2.466284in}{2.577041in}}{\pgfqpoint{2.470675in}{2.587640in}}{\pgfqpoint{2.470675in}{2.598690in}}%
\pgfpathcurveto{\pgfqpoint{2.470675in}{2.609740in}}{\pgfqpoint{2.466284in}{2.620339in}}{\pgfqpoint{2.458471in}{2.628153in}}%
\pgfpathcurveto{\pgfqpoint{2.450657in}{2.635967in}}{\pgfqpoint{2.440058in}{2.640357in}}{\pgfqpoint{2.429008in}{2.640357in}}%
\pgfpathcurveto{\pgfqpoint{2.417958in}{2.640357in}}{\pgfqpoint{2.407359in}{2.635967in}}{\pgfqpoint{2.399545in}{2.628153in}}%
\pgfpathcurveto{\pgfqpoint{2.391731in}{2.620339in}}{\pgfqpoint{2.387341in}{2.609740in}}{\pgfqpoint{2.387341in}{2.598690in}}%
\pgfpathcurveto{\pgfqpoint{2.387341in}{2.587640in}}{\pgfqpoint{2.391731in}{2.577041in}}{\pgfqpoint{2.399545in}{2.569227in}}%
\pgfpathcurveto{\pgfqpoint{2.407359in}{2.561414in}}{\pgfqpoint{2.417958in}{2.557024in}}{\pgfqpoint{2.429008in}{2.557024in}}%
\pgfpathclose%
\pgfusepath{stroke,fill}%
\end{pgfscope}%
\begin{pgfscope}%
\pgfpathrectangle{\pgfqpoint{0.511823in}{0.504323in}}{\pgfqpoint{3.218177in}{3.225677in}} %
\pgfusepath{clip}%
\pgfsetbuttcap%
\pgfsetroundjoin%
\definecolor{currentfill}{rgb}{0.501961,0.000000,0.000000}%
\pgfsetfillcolor{currentfill}%
\pgfsetfillopacity{0.400000}%
\pgfsetlinewidth{0.501875pt}%
\definecolor{currentstroke}{rgb}{0.501961,0.000000,0.000000}%
\pgfsetstrokecolor{currentstroke}%
\pgfsetstrokeopacity{0.400000}%
\pgfsetdash{}{0pt}%
\pgfpathmoveto{\pgfqpoint{2.334370in}{2.464517in}}%
\pgfpathcurveto{\pgfqpoint{2.345420in}{2.464517in}}{\pgfqpoint{2.356019in}{2.468908in}}{\pgfqpoint{2.363833in}{2.476721in}}%
\pgfpathcurveto{\pgfqpoint{2.371647in}{2.484535in}}{\pgfqpoint{2.376037in}{2.495134in}}{\pgfqpoint{2.376037in}{2.506184in}}%
\pgfpathcurveto{\pgfqpoint{2.376037in}{2.517234in}}{\pgfqpoint{2.371647in}{2.527833in}}{\pgfqpoint{2.363833in}{2.535647in}}%
\pgfpathcurveto{\pgfqpoint{2.356019in}{2.543461in}}{\pgfqpoint{2.345420in}{2.547851in}}{\pgfqpoint{2.334370in}{2.547851in}}%
\pgfpathcurveto{\pgfqpoint{2.323320in}{2.547851in}}{\pgfqpoint{2.312721in}{2.543461in}}{\pgfqpoint{2.304907in}{2.535647in}}%
\pgfpathcurveto{\pgfqpoint{2.297094in}{2.527833in}}{\pgfqpoint{2.292703in}{2.517234in}}{\pgfqpoint{2.292703in}{2.506184in}}%
\pgfpathcurveto{\pgfqpoint{2.292703in}{2.495134in}}{\pgfqpoint{2.297094in}{2.484535in}}{\pgfqpoint{2.304907in}{2.476721in}}%
\pgfpathcurveto{\pgfqpoint{2.312721in}{2.468908in}}{\pgfqpoint{2.323320in}{2.464517in}}{\pgfqpoint{2.334370in}{2.464517in}}%
\pgfpathclose%
\pgfusepath{stroke,fill}%
\end{pgfscope}%
\begin{pgfscope}%
\pgfpathrectangle{\pgfqpoint{0.511823in}{0.504323in}}{\pgfqpoint{3.218177in}{3.225677in}} %
\pgfusepath{clip}%
\pgfsetbuttcap%
\pgfsetroundjoin%
\definecolor{currentfill}{rgb}{0.501961,0.000000,0.000000}%
\pgfsetfillcolor{currentfill}%
\pgfsetfillopacity{0.400000}%
\pgfsetlinewidth{0.501875pt}%
\definecolor{currentstroke}{rgb}{0.501961,0.000000,0.000000}%
\pgfsetstrokecolor{currentstroke}%
\pgfsetstrokeopacity{0.400000}%
\pgfsetdash{}{0pt}%
\pgfpathmoveto{\pgfqpoint{2.401083in}{2.550569in}}%
\pgfpathcurveto{\pgfqpoint{2.412133in}{2.550569in}}{\pgfqpoint{2.422732in}{2.554959in}}{\pgfqpoint{2.430546in}{2.562773in}}%
\pgfpathcurveto{\pgfqpoint{2.438359in}{2.570586in}}{\pgfqpoint{2.442749in}{2.581185in}}{\pgfqpoint{2.442749in}{2.592235in}}%
\pgfpathcurveto{\pgfqpoint{2.442749in}{2.603286in}}{\pgfqpoint{2.438359in}{2.613885in}}{\pgfqpoint{2.430546in}{2.621698in}}%
\pgfpathcurveto{\pgfqpoint{2.422732in}{2.629512in}}{\pgfqpoint{2.412133in}{2.633902in}}{\pgfqpoint{2.401083in}{2.633902in}}%
\pgfpathcurveto{\pgfqpoint{2.390033in}{2.633902in}}{\pgfqpoint{2.379434in}{2.629512in}}{\pgfqpoint{2.371620in}{2.621698in}}%
\pgfpathcurveto{\pgfqpoint{2.363806in}{2.613885in}}{\pgfqpoint{2.359416in}{2.603286in}}{\pgfqpoint{2.359416in}{2.592235in}}%
\pgfpathcurveto{\pgfqpoint{2.359416in}{2.581185in}}{\pgfqpoint{2.363806in}{2.570586in}}{\pgfqpoint{2.371620in}{2.562773in}}%
\pgfpathcurveto{\pgfqpoint{2.379434in}{2.554959in}}{\pgfqpoint{2.390033in}{2.550569in}}{\pgfqpoint{2.401083in}{2.550569in}}%
\pgfpathclose%
\pgfusepath{stroke,fill}%
\end{pgfscope}%
\begin{pgfscope}%
\pgfpathrectangle{\pgfqpoint{0.511823in}{0.504323in}}{\pgfqpoint{3.218177in}{3.225677in}} %
\pgfusepath{clip}%
\pgfsetbuttcap%
\pgfsetroundjoin%
\definecolor{currentfill}{rgb}{0.501961,0.000000,0.000000}%
\pgfsetfillcolor{currentfill}%
\pgfsetfillopacity{0.400000}%
\pgfsetlinewidth{0.501875pt}%
\definecolor{currentstroke}{rgb}{0.501961,0.000000,0.000000}%
\pgfsetstrokecolor{currentstroke}%
\pgfsetstrokeopacity{0.400000}%
\pgfsetdash{}{0pt}%
\pgfpathmoveto{\pgfqpoint{2.386017in}{2.545945in}}%
\pgfpathcurveto{\pgfqpoint{2.397068in}{2.545945in}}{\pgfqpoint{2.407667in}{2.550335in}}{\pgfqpoint{2.415480in}{2.558149in}}%
\pgfpathcurveto{\pgfqpoint{2.423294in}{2.565962in}}{\pgfqpoint{2.427684in}{2.576561in}}{\pgfqpoint{2.427684in}{2.587611in}}%
\pgfpathcurveto{\pgfqpoint{2.427684in}{2.598662in}}{\pgfqpoint{2.423294in}{2.609261in}}{\pgfqpoint{2.415480in}{2.617074in}}%
\pgfpathcurveto{\pgfqpoint{2.407667in}{2.624888in}}{\pgfqpoint{2.397068in}{2.629278in}}{\pgfqpoint{2.386017in}{2.629278in}}%
\pgfpathcurveto{\pgfqpoint{2.374967in}{2.629278in}}{\pgfqpoint{2.364368in}{2.624888in}}{\pgfqpoint{2.356555in}{2.617074in}}%
\pgfpathcurveto{\pgfqpoint{2.348741in}{2.609261in}}{\pgfqpoint{2.344351in}{2.598662in}}{\pgfqpoint{2.344351in}{2.587611in}}%
\pgfpathcurveto{\pgfqpoint{2.344351in}{2.576561in}}{\pgfqpoint{2.348741in}{2.565962in}}{\pgfqpoint{2.356555in}{2.558149in}}%
\pgfpathcurveto{\pgfqpoint{2.364368in}{2.550335in}}{\pgfqpoint{2.374967in}{2.545945in}}{\pgfqpoint{2.386017in}{2.545945in}}%
\pgfpathclose%
\pgfusepath{stroke,fill}%
\end{pgfscope}%
\begin{pgfscope}%
\pgfpathrectangle{\pgfqpoint{0.511823in}{0.504323in}}{\pgfqpoint{3.218177in}{3.225677in}} %
\pgfusepath{clip}%
\pgfsetbuttcap%
\pgfsetroundjoin%
\definecolor{currentfill}{rgb}{0.501961,0.000000,0.000000}%
\pgfsetfillcolor{currentfill}%
\pgfsetfillopacity{0.400000}%
\pgfsetlinewidth{0.501875pt}%
\definecolor{currentstroke}{rgb}{0.501961,0.000000,0.000000}%
\pgfsetstrokecolor{currentstroke}%
\pgfsetstrokeopacity{0.400000}%
\pgfsetdash{}{0pt}%
\pgfpathmoveto{\pgfqpoint{2.419487in}{2.595982in}}%
\pgfpathcurveto{\pgfqpoint{2.430537in}{2.595982in}}{\pgfqpoint{2.441136in}{2.600372in}}{\pgfqpoint{2.448950in}{2.608186in}}%
\pgfpathcurveto{\pgfqpoint{2.456764in}{2.616000in}}{\pgfqpoint{2.461154in}{2.626599in}}{\pgfqpoint{2.461154in}{2.637649in}}%
\pgfpathcurveto{\pgfqpoint{2.461154in}{2.648699in}}{\pgfqpoint{2.456764in}{2.659298in}}{\pgfqpoint{2.448950in}{2.667112in}}%
\pgfpathcurveto{\pgfqpoint{2.441136in}{2.674925in}}{\pgfqpoint{2.430537in}{2.679316in}}{\pgfqpoint{2.419487in}{2.679316in}}%
\pgfpathcurveto{\pgfqpoint{2.408437in}{2.679316in}}{\pgfqpoint{2.397838in}{2.674925in}}{\pgfqpoint{2.390025in}{2.667112in}}%
\pgfpathcurveto{\pgfqpoint{2.382211in}{2.659298in}}{\pgfqpoint{2.377821in}{2.648699in}}{\pgfqpoint{2.377821in}{2.637649in}}%
\pgfpathcurveto{\pgfqpoint{2.377821in}{2.626599in}}{\pgfqpoint{2.382211in}{2.616000in}}{\pgfqpoint{2.390025in}{2.608186in}}%
\pgfpathcurveto{\pgfqpoint{2.397838in}{2.600372in}}{\pgfqpoint{2.408437in}{2.595982in}}{\pgfqpoint{2.419487in}{2.595982in}}%
\pgfpathclose%
\pgfusepath{stroke,fill}%
\end{pgfscope}%
\begin{pgfscope}%
\pgfpathrectangle{\pgfqpoint{0.511823in}{0.504323in}}{\pgfqpoint{3.218177in}{3.225677in}} %
\pgfusepath{clip}%
\pgfsetbuttcap%
\pgfsetroundjoin%
\definecolor{currentfill}{rgb}{0.501961,0.000000,0.000000}%
\pgfsetfillcolor{currentfill}%
\pgfsetfillopacity{0.400000}%
\pgfsetlinewidth{0.501875pt}%
\definecolor{currentstroke}{rgb}{0.501961,0.000000,0.000000}%
\pgfsetstrokecolor{currentstroke}%
\pgfsetstrokeopacity{0.400000}%
\pgfsetdash{}{0pt}%
\pgfpathmoveto{\pgfqpoint{2.388006in}{2.572813in}}%
\pgfpathcurveto{\pgfqpoint{2.399056in}{2.572813in}}{\pgfqpoint{2.409655in}{2.577203in}}{\pgfqpoint{2.417469in}{2.585017in}}%
\pgfpathcurveto{\pgfqpoint{2.425283in}{2.592831in}}{\pgfqpoint{2.429673in}{2.603430in}}{\pgfqpoint{2.429673in}{2.614480in}}%
\pgfpathcurveto{\pgfqpoint{2.429673in}{2.625530in}}{\pgfqpoint{2.425283in}{2.636129in}}{\pgfqpoint{2.417469in}{2.643942in}}%
\pgfpathcurveto{\pgfqpoint{2.409655in}{2.651756in}}{\pgfqpoint{2.399056in}{2.656146in}}{\pgfqpoint{2.388006in}{2.656146in}}%
\pgfpathcurveto{\pgfqpoint{2.376956in}{2.656146in}}{\pgfqpoint{2.366357in}{2.651756in}}{\pgfqpoint{2.358543in}{2.643942in}}%
\pgfpathcurveto{\pgfqpoint{2.350730in}{2.636129in}}{\pgfqpoint{2.346340in}{2.625530in}}{\pgfqpoint{2.346340in}{2.614480in}}%
\pgfpathcurveto{\pgfqpoint{2.346340in}{2.603430in}}{\pgfqpoint{2.350730in}{2.592831in}}{\pgfqpoint{2.358543in}{2.585017in}}%
\pgfpathcurveto{\pgfqpoint{2.366357in}{2.577203in}}{\pgfqpoint{2.376956in}{2.572813in}}{\pgfqpoint{2.388006in}{2.572813in}}%
\pgfpathclose%
\pgfusepath{stroke,fill}%
\end{pgfscope}%
\begin{pgfscope}%
\pgfpathrectangle{\pgfqpoint{0.511823in}{0.504323in}}{\pgfqpoint{3.218177in}{3.225677in}} %
\pgfusepath{clip}%
\pgfsetbuttcap%
\pgfsetroundjoin%
\definecolor{currentfill}{rgb}{0.501961,0.000000,0.000000}%
\pgfsetfillcolor{currentfill}%
\pgfsetfillopacity{0.400000}%
\pgfsetlinewidth{0.501875pt}%
\definecolor{currentstroke}{rgb}{0.501961,0.000000,0.000000}%
\pgfsetstrokecolor{currentstroke}%
\pgfsetstrokeopacity{0.400000}%
\pgfsetdash{}{0pt}%
\pgfpathmoveto{\pgfqpoint{2.391178in}{2.588900in}}%
\pgfpathcurveto{\pgfqpoint{2.402228in}{2.588900in}}{\pgfqpoint{2.412827in}{2.593291in}}{\pgfqpoint{2.420640in}{2.601104in}}%
\pgfpathcurveto{\pgfqpoint{2.428454in}{2.608918in}}{\pgfqpoint{2.432844in}{2.619517in}}{\pgfqpoint{2.432844in}{2.630567in}}%
\pgfpathcurveto{\pgfqpoint{2.432844in}{2.641617in}}{\pgfqpoint{2.428454in}{2.652216in}}{\pgfqpoint{2.420640in}{2.660030in}}%
\pgfpathcurveto{\pgfqpoint{2.412827in}{2.667843in}}{\pgfqpoint{2.402228in}{2.672234in}}{\pgfqpoint{2.391178in}{2.672234in}}%
\pgfpathcurveto{\pgfqpoint{2.380128in}{2.672234in}}{\pgfqpoint{2.369528in}{2.667843in}}{\pgfqpoint{2.361715in}{2.660030in}}%
\pgfpathcurveto{\pgfqpoint{2.353901in}{2.652216in}}{\pgfqpoint{2.349511in}{2.641617in}}{\pgfqpoint{2.349511in}{2.630567in}}%
\pgfpathcurveto{\pgfqpoint{2.349511in}{2.619517in}}{\pgfqpoint{2.353901in}{2.608918in}}{\pgfqpoint{2.361715in}{2.601104in}}%
\pgfpathcurveto{\pgfqpoint{2.369528in}{2.593291in}}{\pgfqpoint{2.380128in}{2.588900in}}{\pgfqpoint{2.391178in}{2.588900in}}%
\pgfpathclose%
\pgfusepath{stroke,fill}%
\end{pgfscope}%
\begin{pgfscope}%
\pgfpathrectangle{\pgfqpoint{0.511823in}{0.504323in}}{\pgfqpoint{3.218177in}{3.225677in}} %
\pgfusepath{clip}%
\pgfsetbuttcap%
\pgfsetroundjoin%
\definecolor{currentfill}{rgb}{0.501961,0.000000,0.000000}%
\pgfsetfillcolor{currentfill}%
\pgfsetfillopacity{0.400000}%
\pgfsetlinewidth{0.501875pt}%
\definecolor{currentstroke}{rgb}{0.501961,0.000000,0.000000}%
\pgfsetstrokecolor{currentstroke}%
\pgfsetstrokeopacity{0.400000}%
\pgfsetdash{}{0pt}%
\pgfpathmoveto{\pgfqpoint{2.464562in}{2.685928in}}%
\pgfpathcurveto{\pgfqpoint{2.475612in}{2.685928in}}{\pgfqpoint{2.486211in}{2.690319in}}{\pgfqpoint{2.494025in}{2.698132in}}%
\pgfpathcurveto{\pgfqpoint{2.501838in}{2.705946in}}{\pgfqpoint{2.506228in}{2.716545in}}{\pgfqpoint{2.506228in}{2.727595in}}%
\pgfpathcurveto{\pgfqpoint{2.506228in}{2.738645in}}{\pgfqpoint{2.501838in}{2.749244in}}{\pgfqpoint{2.494025in}{2.757058in}}%
\pgfpathcurveto{\pgfqpoint{2.486211in}{2.764871in}}{\pgfqpoint{2.475612in}{2.769262in}}{\pgfqpoint{2.464562in}{2.769262in}}%
\pgfpathcurveto{\pgfqpoint{2.453512in}{2.769262in}}{\pgfqpoint{2.442913in}{2.764871in}}{\pgfqpoint{2.435099in}{2.757058in}}%
\pgfpathcurveto{\pgfqpoint{2.427285in}{2.749244in}}{\pgfqpoint{2.422895in}{2.738645in}}{\pgfqpoint{2.422895in}{2.727595in}}%
\pgfpathcurveto{\pgfqpoint{2.422895in}{2.716545in}}{\pgfqpoint{2.427285in}{2.705946in}}{\pgfqpoint{2.435099in}{2.698132in}}%
\pgfpathcurveto{\pgfqpoint{2.442913in}{2.690319in}}{\pgfqpoint{2.453512in}{2.685928in}}{\pgfqpoint{2.464562in}{2.685928in}}%
\pgfpathclose%
\pgfusepath{stroke,fill}%
\end{pgfscope}%
\begin{pgfscope}%
\pgfpathrectangle{\pgfqpoint{0.511823in}{0.504323in}}{\pgfqpoint{3.218177in}{3.225677in}} %
\pgfusepath{clip}%
\pgfsetbuttcap%
\pgfsetroundjoin%
\definecolor{currentfill}{rgb}{0.501961,0.000000,0.000000}%
\pgfsetfillcolor{currentfill}%
\pgfsetfillopacity{0.400000}%
\pgfsetlinewidth{0.501875pt}%
\definecolor{currentstroke}{rgb}{0.501961,0.000000,0.000000}%
\pgfsetstrokecolor{currentstroke}%
\pgfsetstrokeopacity{0.400000}%
\pgfsetdash{}{0pt}%
\pgfpathmoveto{\pgfqpoint{2.271522in}{2.475579in}}%
\pgfpathcurveto{\pgfqpoint{2.282572in}{2.475579in}}{\pgfqpoint{2.293171in}{2.479969in}}{\pgfqpoint{2.300985in}{2.487783in}}%
\pgfpathcurveto{\pgfqpoint{2.308799in}{2.495596in}}{\pgfqpoint{2.313189in}{2.506195in}}{\pgfqpoint{2.313189in}{2.517245in}}%
\pgfpathcurveto{\pgfqpoint{2.313189in}{2.528295in}}{\pgfqpoint{2.308799in}{2.538895in}}{\pgfqpoint{2.300985in}{2.546708in}}%
\pgfpathcurveto{\pgfqpoint{2.293171in}{2.554522in}}{\pgfqpoint{2.282572in}{2.558912in}}{\pgfqpoint{2.271522in}{2.558912in}}%
\pgfpathcurveto{\pgfqpoint{2.260472in}{2.558912in}}{\pgfqpoint{2.249873in}{2.554522in}}{\pgfqpoint{2.242059in}{2.546708in}}%
\pgfpathcurveto{\pgfqpoint{2.234246in}{2.538895in}}{\pgfqpoint{2.229856in}{2.528295in}}{\pgfqpoint{2.229856in}{2.517245in}}%
\pgfpathcurveto{\pgfqpoint{2.229856in}{2.506195in}}{\pgfqpoint{2.234246in}{2.495596in}}{\pgfqpoint{2.242059in}{2.487783in}}%
\pgfpathcurveto{\pgfqpoint{2.249873in}{2.479969in}}{\pgfqpoint{2.260472in}{2.475579in}}{\pgfqpoint{2.271522in}{2.475579in}}%
\pgfpathclose%
\pgfusepath{stroke,fill}%
\end{pgfscope}%
\begin{pgfscope}%
\pgfpathrectangle{\pgfqpoint{0.511823in}{0.504323in}}{\pgfqpoint{3.218177in}{3.225677in}} %
\pgfusepath{clip}%
\pgfsetbuttcap%
\pgfsetroundjoin%
\definecolor{currentfill}{rgb}{0.501961,0.000000,0.000000}%
\pgfsetfillcolor{currentfill}%
\pgfsetfillopacity{0.400000}%
\pgfsetlinewidth{0.501875pt}%
\definecolor{currentstroke}{rgb}{0.501961,0.000000,0.000000}%
\pgfsetstrokecolor{currentstroke}%
\pgfsetstrokeopacity{0.400000}%
\pgfsetdash{}{0pt}%
\pgfpathmoveto{\pgfqpoint{2.349552in}{2.578419in}}%
\pgfpathcurveto{\pgfqpoint{2.360602in}{2.578419in}}{\pgfqpoint{2.371201in}{2.582809in}}{\pgfqpoint{2.379015in}{2.590623in}}%
\pgfpathcurveto{\pgfqpoint{2.386828in}{2.598437in}}{\pgfqpoint{2.391219in}{2.609036in}}{\pgfqpoint{2.391219in}{2.620086in}}%
\pgfpathcurveto{\pgfqpoint{2.391219in}{2.631136in}}{\pgfqpoint{2.386828in}{2.641735in}}{\pgfqpoint{2.379015in}{2.649549in}}%
\pgfpathcurveto{\pgfqpoint{2.371201in}{2.657362in}}{\pgfqpoint{2.360602in}{2.661752in}}{\pgfqpoint{2.349552in}{2.661752in}}%
\pgfpathcurveto{\pgfqpoint{2.338502in}{2.661752in}}{\pgfqpoint{2.327903in}{2.657362in}}{\pgfqpoint{2.320089in}{2.649549in}}%
\pgfpathcurveto{\pgfqpoint{2.312276in}{2.641735in}}{\pgfqpoint{2.307885in}{2.631136in}}{\pgfqpoint{2.307885in}{2.620086in}}%
\pgfpathcurveto{\pgfqpoint{2.307885in}{2.609036in}}{\pgfqpoint{2.312276in}{2.598437in}}{\pgfqpoint{2.320089in}{2.590623in}}%
\pgfpathcurveto{\pgfqpoint{2.327903in}{2.582809in}}{\pgfqpoint{2.338502in}{2.578419in}}{\pgfqpoint{2.349552in}{2.578419in}}%
\pgfpathclose%
\pgfusepath{stroke,fill}%
\end{pgfscope}%
\begin{pgfscope}%
\pgfpathrectangle{\pgfqpoint{0.511823in}{0.504323in}}{\pgfqpoint{3.218177in}{3.225677in}} %
\pgfusepath{clip}%
\pgfsetbuttcap%
\pgfsetroundjoin%
\definecolor{currentfill}{rgb}{0.501961,0.000000,0.000000}%
\pgfsetfillcolor{currentfill}%
\pgfsetfillopacity{0.400000}%
\pgfsetlinewidth{0.501875pt}%
\definecolor{currentstroke}{rgb}{0.501961,0.000000,0.000000}%
\pgfsetstrokecolor{currentstroke}%
\pgfsetstrokeopacity{0.400000}%
\pgfsetdash{}{0pt}%
\pgfpathmoveto{\pgfqpoint{2.365596in}{2.609808in}}%
\pgfpathcurveto{\pgfqpoint{2.376646in}{2.609808in}}{\pgfqpoint{2.387245in}{2.614199in}}{\pgfqpoint{2.395058in}{2.622012in}}%
\pgfpathcurveto{\pgfqpoint{2.402872in}{2.629826in}}{\pgfqpoint{2.407262in}{2.640425in}}{\pgfqpoint{2.407262in}{2.651475in}}%
\pgfpathcurveto{\pgfqpoint{2.407262in}{2.662525in}}{\pgfqpoint{2.402872in}{2.673124in}}{\pgfqpoint{2.395058in}{2.680938in}}%
\pgfpathcurveto{\pgfqpoint{2.387245in}{2.688751in}}{\pgfqpoint{2.376646in}{2.693142in}}{\pgfqpoint{2.365596in}{2.693142in}}%
\pgfpathcurveto{\pgfqpoint{2.354545in}{2.693142in}}{\pgfqpoint{2.343946in}{2.688751in}}{\pgfqpoint{2.336133in}{2.680938in}}%
\pgfpathcurveto{\pgfqpoint{2.328319in}{2.673124in}}{\pgfqpoint{2.323929in}{2.662525in}}{\pgfqpoint{2.323929in}{2.651475in}}%
\pgfpathcurveto{\pgfqpoint{2.323929in}{2.640425in}}{\pgfqpoint{2.328319in}{2.629826in}}{\pgfqpoint{2.336133in}{2.622012in}}%
\pgfpathcurveto{\pgfqpoint{2.343946in}{2.614199in}}{\pgfqpoint{2.354545in}{2.609808in}}{\pgfqpoint{2.365596in}{2.609808in}}%
\pgfpathclose%
\pgfusepath{stroke,fill}%
\end{pgfscope}%
\begin{pgfscope}%
\pgfpathrectangle{\pgfqpoint{0.511823in}{0.504323in}}{\pgfqpoint{3.218177in}{3.225677in}} %
\pgfusepath{clip}%
\pgfsetbuttcap%
\pgfsetroundjoin%
\definecolor{currentfill}{rgb}{0.501961,0.000000,0.000000}%
\pgfsetfillcolor{currentfill}%
\pgfsetfillopacity{0.400000}%
\pgfsetlinewidth{0.501875pt}%
\definecolor{currentstroke}{rgb}{0.501961,0.000000,0.000000}%
\pgfsetstrokecolor{currentstroke}%
\pgfsetstrokeopacity{0.400000}%
\pgfsetdash{}{0pt}%
\pgfpathmoveto{\pgfqpoint{2.232612in}{2.465563in}}%
\pgfpathcurveto{\pgfqpoint{2.243662in}{2.465563in}}{\pgfqpoint{2.254261in}{2.469953in}}{\pgfqpoint{2.262075in}{2.477767in}}%
\pgfpathcurveto{\pgfqpoint{2.269888in}{2.485581in}}{\pgfqpoint{2.274279in}{2.496180in}}{\pgfqpoint{2.274279in}{2.507230in}}%
\pgfpathcurveto{\pgfqpoint{2.274279in}{2.518280in}}{\pgfqpoint{2.269888in}{2.528879in}}{\pgfqpoint{2.262075in}{2.536693in}}%
\pgfpathcurveto{\pgfqpoint{2.254261in}{2.544506in}}{\pgfqpoint{2.243662in}{2.548896in}}{\pgfqpoint{2.232612in}{2.548896in}}%
\pgfpathcurveto{\pgfqpoint{2.221562in}{2.548896in}}{\pgfqpoint{2.210963in}{2.544506in}}{\pgfqpoint{2.203149in}{2.536693in}}%
\pgfpathcurveto{\pgfqpoint{2.195336in}{2.528879in}}{\pgfqpoint{2.190945in}{2.518280in}}{\pgfqpoint{2.190945in}{2.507230in}}%
\pgfpathcurveto{\pgfqpoint{2.190945in}{2.496180in}}{\pgfqpoint{2.195336in}{2.485581in}}{\pgfqpoint{2.203149in}{2.477767in}}%
\pgfpathcurveto{\pgfqpoint{2.210963in}{2.469953in}}{\pgfqpoint{2.221562in}{2.465563in}}{\pgfqpoint{2.232612in}{2.465563in}}%
\pgfpathclose%
\pgfusepath{stroke,fill}%
\end{pgfscope}%
\begin{pgfscope}%
\pgfpathrectangle{\pgfqpoint{0.511823in}{0.504323in}}{\pgfqpoint{3.218177in}{3.225677in}} %
\pgfusepath{clip}%
\pgfsetbuttcap%
\pgfsetroundjoin%
\definecolor{currentfill}{rgb}{0.501961,0.000000,0.000000}%
\pgfsetfillcolor{currentfill}%
\pgfsetfillopacity{0.400000}%
\pgfsetlinewidth{0.501875pt}%
\definecolor{currentstroke}{rgb}{0.501961,0.000000,0.000000}%
\pgfsetstrokecolor{currentstroke}%
\pgfsetstrokeopacity{0.400000}%
\pgfsetdash{}{0pt}%
\pgfpathmoveto{\pgfqpoint{2.248624in}{2.496464in}}%
\pgfpathcurveto{\pgfqpoint{2.259674in}{2.496464in}}{\pgfqpoint{2.270273in}{2.500854in}}{\pgfqpoint{2.278087in}{2.508668in}}%
\pgfpathcurveto{\pgfqpoint{2.285900in}{2.516481in}}{\pgfqpoint{2.290291in}{2.527081in}}{\pgfqpoint{2.290291in}{2.538131in}}%
\pgfpathcurveto{\pgfqpoint{2.290291in}{2.549181in}}{\pgfqpoint{2.285900in}{2.559780in}}{\pgfqpoint{2.278087in}{2.567593in}}%
\pgfpathcurveto{\pgfqpoint{2.270273in}{2.575407in}}{\pgfqpoint{2.259674in}{2.579797in}}{\pgfqpoint{2.248624in}{2.579797in}}%
\pgfpathcurveto{\pgfqpoint{2.237574in}{2.579797in}}{\pgfqpoint{2.226975in}{2.575407in}}{\pgfqpoint{2.219161in}{2.567593in}}%
\pgfpathcurveto{\pgfqpoint{2.211348in}{2.559780in}}{\pgfqpoint{2.206957in}{2.549181in}}{\pgfqpoint{2.206957in}{2.538131in}}%
\pgfpathcurveto{\pgfqpoint{2.206957in}{2.527081in}}{\pgfqpoint{2.211348in}{2.516481in}}{\pgfqpoint{2.219161in}{2.508668in}}%
\pgfpathcurveto{\pgfqpoint{2.226975in}{2.500854in}}{\pgfqpoint{2.237574in}{2.496464in}}{\pgfqpoint{2.248624in}{2.496464in}}%
\pgfpathclose%
\pgfusepath{stroke,fill}%
\end{pgfscope}%
\begin{pgfscope}%
\pgfpathrectangle{\pgfqpoint{0.511823in}{0.504323in}}{\pgfqpoint{3.218177in}{3.225677in}} %
\pgfusepath{clip}%
\pgfsetbuttcap%
\pgfsetroundjoin%
\definecolor{currentfill}{rgb}{0.501961,0.000000,0.000000}%
\pgfsetfillcolor{currentfill}%
\pgfsetfillopacity{0.400000}%
\pgfsetlinewidth{0.501875pt}%
\definecolor{currentstroke}{rgb}{0.501961,0.000000,0.000000}%
\pgfsetstrokecolor{currentstroke}%
\pgfsetstrokeopacity{0.400000}%
\pgfsetdash{}{0pt}%
\pgfpathmoveto{\pgfqpoint{2.292295in}{2.560796in}}%
\pgfpathcurveto{\pgfqpoint{2.303345in}{2.560796in}}{\pgfqpoint{2.313944in}{2.565187in}}{\pgfqpoint{2.321758in}{2.573000in}}%
\pgfpathcurveto{\pgfqpoint{2.329572in}{2.580814in}}{\pgfqpoint{2.333962in}{2.591413in}}{\pgfqpoint{2.333962in}{2.602463in}}%
\pgfpathcurveto{\pgfqpoint{2.333962in}{2.613513in}}{\pgfqpoint{2.329572in}{2.624112in}}{\pgfqpoint{2.321758in}{2.631926in}}%
\pgfpathcurveto{\pgfqpoint{2.313944in}{2.639739in}}{\pgfqpoint{2.303345in}{2.644130in}}{\pgfqpoint{2.292295in}{2.644130in}}%
\pgfpathcurveto{\pgfqpoint{2.281245in}{2.644130in}}{\pgfqpoint{2.270646in}{2.639739in}}{\pgfqpoint{2.262832in}{2.631926in}}%
\pgfpathcurveto{\pgfqpoint{2.255019in}{2.624112in}}{\pgfqpoint{2.250628in}{2.613513in}}{\pgfqpoint{2.250628in}{2.602463in}}%
\pgfpathcurveto{\pgfqpoint{2.250628in}{2.591413in}}{\pgfqpoint{2.255019in}{2.580814in}}{\pgfqpoint{2.262832in}{2.573000in}}%
\pgfpathcurveto{\pgfqpoint{2.270646in}{2.565187in}}{\pgfqpoint{2.281245in}{2.560796in}}{\pgfqpoint{2.292295in}{2.560796in}}%
\pgfpathclose%
\pgfusepath{stroke,fill}%
\end{pgfscope}%
\begin{pgfscope}%
\pgfpathrectangle{\pgfqpoint{0.511823in}{0.504323in}}{\pgfqpoint{3.218177in}{3.225677in}} %
\pgfusepath{clip}%
\pgfsetbuttcap%
\pgfsetroundjoin%
\definecolor{currentfill}{rgb}{0.501961,0.000000,0.000000}%
\pgfsetfillcolor{currentfill}%
\pgfsetfillopacity{0.400000}%
\pgfsetlinewidth{0.501875pt}%
\definecolor{currentstroke}{rgb}{0.501961,0.000000,0.000000}%
\pgfsetstrokecolor{currentstroke}%
\pgfsetstrokeopacity{0.400000}%
\pgfsetdash{}{0pt}%
\pgfpathmoveto{\pgfqpoint{2.269343in}{2.545690in}}%
\pgfpathcurveto{\pgfqpoint{2.280394in}{2.545690in}}{\pgfqpoint{2.290993in}{2.550080in}}{\pgfqpoint{2.298806in}{2.557894in}}%
\pgfpathcurveto{\pgfqpoint{2.306620in}{2.565708in}}{\pgfqpoint{2.311010in}{2.576307in}}{\pgfqpoint{2.311010in}{2.587357in}}%
\pgfpathcurveto{\pgfqpoint{2.311010in}{2.598407in}}{\pgfqpoint{2.306620in}{2.609006in}}{\pgfqpoint{2.298806in}{2.616819in}}%
\pgfpathcurveto{\pgfqpoint{2.290993in}{2.624633in}}{\pgfqpoint{2.280394in}{2.629023in}}{\pgfqpoint{2.269343in}{2.629023in}}%
\pgfpathcurveto{\pgfqpoint{2.258293in}{2.629023in}}{\pgfqpoint{2.247694in}{2.624633in}}{\pgfqpoint{2.239881in}{2.616819in}}%
\pgfpathcurveto{\pgfqpoint{2.232067in}{2.609006in}}{\pgfqpoint{2.227677in}{2.598407in}}{\pgfqpoint{2.227677in}{2.587357in}}%
\pgfpathcurveto{\pgfqpoint{2.227677in}{2.576307in}}{\pgfqpoint{2.232067in}{2.565708in}}{\pgfqpoint{2.239881in}{2.557894in}}%
\pgfpathcurveto{\pgfqpoint{2.247694in}{2.550080in}}{\pgfqpoint{2.258293in}{2.545690in}}{\pgfqpoint{2.269343in}{2.545690in}}%
\pgfpathclose%
\pgfusepath{stroke,fill}%
\end{pgfscope}%
\begin{pgfscope}%
\pgfpathrectangle{\pgfqpoint{0.511823in}{0.504323in}}{\pgfqpoint{3.218177in}{3.225677in}} %
\pgfusepath{clip}%
\pgfsetbuttcap%
\pgfsetroundjoin%
\definecolor{currentfill}{rgb}{0.501961,0.000000,0.000000}%
\pgfsetfillcolor{currentfill}%
\pgfsetfillopacity{0.400000}%
\pgfsetlinewidth{0.501875pt}%
\definecolor{currentstroke}{rgb}{0.501961,0.000000,0.000000}%
\pgfsetstrokecolor{currentstroke}%
\pgfsetstrokeopacity{0.400000}%
\pgfsetdash{}{0pt}%
\pgfpathmoveto{\pgfqpoint{2.396574in}{2.712337in}}%
\pgfpathcurveto{\pgfqpoint{2.407624in}{2.712337in}}{\pgfqpoint{2.418223in}{2.716727in}}{\pgfqpoint{2.426037in}{2.724540in}}%
\pgfpathcurveto{\pgfqpoint{2.433851in}{2.732354in}}{\pgfqpoint{2.438241in}{2.742953in}}{\pgfqpoint{2.438241in}{2.754003in}}%
\pgfpathcurveto{\pgfqpoint{2.438241in}{2.765053in}}{\pgfqpoint{2.433851in}{2.775652in}}{\pgfqpoint{2.426037in}{2.783466in}}%
\pgfpathcurveto{\pgfqpoint{2.418223in}{2.791280in}}{\pgfqpoint{2.407624in}{2.795670in}}{\pgfqpoint{2.396574in}{2.795670in}}%
\pgfpathcurveto{\pgfqpoint{2.385524in}{2.795670in}}{\pgfqpoint{2.374925in}{2.791280in}}{\pgfqpoint{2.367111in}{2.783466in}}%
\pgfpathcurveto{\pgfqpoint{2.359298in}{2.775652in}}{\pgfqpoint{2.354908in}{2.765053in}}{\pgfqpoint{2.354908in}{2.754003in}}%
\pgfpathcurveto{\pgfqpoint{2.354908in}{2.742953in}}{\pgfqpoint{2.359298in}{2.732354in}}{\pgfqpoint{2.367111in}{2.724540in}}%
\pgfpathcurveto{\pgfqpoint{2.374925in}{2.716727in}}{\pgfqpoint{2.385524in}{2.712337in}}{\pgfqpoint{2.396574in}{2.712337in}}%
\pgfpathclose%
\pgfusepath{stroke,fill}%
\end{pgfscope}%
\begin{pgfscope}%
\pgfpathrectangle{\pgfqpoint{0.511823in}{0.504323in}}{\pgfqpoint{3.218177in}{3.225677in}} %
\pgfusepath{clip}%
\pgfsetbuttcap%
\pgfsetroundjoin%
\definecolor{currentfill}{rgb}{0.501961,0.000000,0.000000}%
\pgfsetfillcolor{currentfill}%
\pgfsetfillopacity{0.400000}%
\pgfsetlinewidth{0.501875pt}%
\definecolor{currentstroke}{rgb}{0.501961,0.000000,0.000000}%
\pgfsetstrokecolor{currentstroke}%
\pgfsetstrokeopacity{0.400000}%
\pgfsetdash{}{0pt}%
\pgfpathmoveto{\pgfqpoint{2.310192in}{2.620524in}}%
\pgfpathcurveto{\pgfqpoint{2.321242in}{2.620524in}}{\pgfqpoint{2.331841in}{2.624915in}}{\pgfqpoint{2.339655in}{2.632728in}}%
\pgfpathcurveto{\pgfqpoint{2.347468in}{2.640542in}}{\pgfqpoint{2.351859in}{2.651141in}}{\pgfqpoint{2.351859in}{2.662191in}}%
\pgfpathcurveto{\pgfqpoint{2.351859in}{2.673241in}}{\pgfqpoint{2.347468in}{2.683840in}}{\pgfqpoint{2.339655in}{2.691654in}}%
\pgfpathcurveto{\pgfqpoint{2.331841in}{2.699467in}}{\pgfqpoint{2.321242in}{2.703858in}}{\pgfqpoint{2.310192in}{2.703858in}}%
\pgfpathcurveto{\pgfqpoint{2.299142in}{2.703858in}}{\pgfqpoint{2.288543in}{2.699467in}}{\pgfqpoint{2.280729in}{2.691654in}}%
\pgfpathcurveto{\pgfqpoint{2.272916in}{2.683840in}}{\pgfqpoint{2.268525in}{2.673241in}}{\pgfqpoint{2.268525in}{2.662191in}}%
\pgfpathcurveto{\pgfqpoint{2.268525in}{2.651141in}}{\pgfqpoint{2.272916in}{2.640542in}}{\pgfqpoint{2.280729in}{2.632728in}}%
\pgfpathcurveto{\pgfqpoint{2.288543in}{2.624915in}}{\pgfqpoint{2.299142in}{2.620524in}}{\pgfqpoint{2.310192in}{2.620524in}}%
\pgfpathclose%
\pgfusepath{stroke,fill}%
\end{pgfscope}%
\begin{pgfscope}%
\pgfpathrectangle{\pgfqpoint{0.511823in}{0.504323in}}{\pgfqpoint{3.218177in}{3.225677in}} %
\pgfusepath{clip}%
\pgfsetbuttcap%
\pgfsetroundjoin%
\definecolor{currentfill}{rgb}{0.501961,0.000000,0.000000}%
\pgfsetfillcolor{currentfill}%
\pgfsetfillopacity{0.400000}%
\pgfsetlinewidth{0.501875pt}%
\definecolor{currentstroke}{rgb}{0.501961,0.000000,0.000000}%
\pgfsetstrokecolor{currentstroke}%
\pgfsetstrokeopacity{0.400000}%
\pgfsetdash{}{0pt}%
\pgfpathmoveto{\pgfqpoint{2.527793in}{2.900693in}}%
\pgfpathcurveto{\pgfqpoint{2.538843in}{2.900693in}}{\pgfqpoint{2.549442in}{2.905083in}}{\pgfqpoint{2.557256in}{2.912897in}}%
\pgfpathcurveto{\pgfqpoint{2.565069in}{2.920710in}}{\pgfqpoint{2.569460in}{2.931309in}}{\pgfqpoint{2.569460in}{2.942360in}}%
\pgfpathcurveto{\pgfqpoint{2.569460in}{2.953410in}}{\pgfqpoint{2.565069in}{2.964009in}}{\pgfqpoint{2.557256in}{2.971822in}}%
\pgfpathcurveto{\pgfqpoint{2.549442in}{2.979636in}}{\pgfqpoint{2.538843in}{2.984026in}}{\pgfqpoint{2.527793in}{2.984026in}}%
\pgfpathcurveto{\pgfqpoint{2.516743in}{2.984026in}}{\pgfqpoint{2.506144in}{2.979636in}}{\pgfqpoint{2.498330in}{2.971822in}}%
\pgfpathcurveto{\pgfqpoint{2.490517in}{2.964009in}}{\pgfqpoint{2.486126in}{2.953410in}}{\pgfqpoint{2.486126in}{2.942360in}}%
\pgfpathcurveto{\pgfqpoint{2.486126in}{2.931309in}}{\pgfqpoint{2.490517in}{2.920710in}}{\pgfqpoint{2.498330in}{2.912897in}}%
\pgfpathcurveto{\pgfqpoint{2.506144in}{2.905083in}}{\pgfqpoint{2.516743in}{2.900693in}}{\pgfqpoint{2.527793in}{2.900693in}}%
\pgfpathclose%
\pgfusepath{stroke,fill}%
\end{pgfscope}%
\begin{pgfscope}%
\pgfpathrectangle{\pgfqpoint{0.511823in}{0.504323in}}{\pgfqpoint{3.218177in}{3.225677in}} %
\pgfusepath{clip}%
\pgfsetbuttcap%
\pgfsetroundjoin%
\definecolor{currentfill}{rgb}{0.501961,0.000000,0.000000}%
\pgfsetfillcolor{currentfill}%
\pgfsetfillopacity{0.400000}%
\pgfsetlinewidth{0.501875pt}%
\definecolor{currentstroke}{rgb}{0.501961,0.000000,0.000000}%
\pgfsetstrokecolor{currentstroke}%
\pgfsetstrokeopacity{0.400000}%
\pgfsetdash{}{0pt}%
\pgfpathmoveto{\pgfqpoint{2.256332in}{2.580034in}}%
\pgfpathcurveto{\pgfqpoint{2.267382in}{2.580034in}}{\pgfqpoint{2.277982in}{2.584425in}}{\pgfqpoint{2.285795in}{2.592238in}}%
\pgfpathcurveto{\pgfqpoint{2.293609in}{2.600052in}}{\pgfqpoint{2.297999in}{2.610651in}}{\pgfqpoint{2.297999in}{2.621701in}}%
\pgfpathcurveto{\pgfqpoint{2.297999in}{2.632751in}}{\pgfqpoint{2.293609in}{2.643350in}}{\pgfqpoint{2.285795in}{2.651164in}}%
\pgfpathcurveto{\pgfqpoint{2.277982in}{2.658977in}}{\pgfqpoint{2.267382in}{2.663368in}}{\pgfqpoint{2.256332in}{2.663368in}}%
\pgfpathcurveto{\pgfqpoint{2.245282in}{2.663368in}}{\pgfqpoint{2.234683in}{2.658977in}}{\pgfqpoint{2.226870in}{2.651164in}}%
\pgfpathcurveto{\pgfqpoint{2.219056in}{2.643350in}}{\pgfqpoint{2.214666in}{2.632751in}}{\pgfqpoint{2.214666in}{2.621701in}}%
\pgfpathcurveto{\pgfqpoint{2.214666in}{2.610651in}}{\pgfqpoint{2.219056in}{2.600052in}}{\pgfqpoint{2.226870in}{2.592238in}}%
\pgfpathcurveto{\pgfqpoint{2.234683in}{2.584425in}}{\pgfqpoint{2.245282in}{2.580034in}}{\pgfqpoint{2.256332in}{2.580034in}}%
\pgfpathclose%
\pgfusepath{stroke,fill}%
\end{pgfscope}%
\begin{pgfscope}%
\pgfpathrectangle{\pgfqpoint{0.511823in}{0.504323in}}{\pgfqpoint{3.218177in}{3.225677in}} %
\pgfusepath{clip}%
\pgfsetbuttcap%
\pgfsetroundjoin%
\definecolor{currentfill}{rgb}{0.501961,0.000000,0.000000}%
\pgfsetfillcolor{currentfill}%
\pgfsetfillopacity{0.400000}%
\pgfsetlinewidth{0.501875pt}%
\definecolor{currentstroke}{rgb}{0.501961,0.000000,0.000000}%
\pgfsetstrokecolor{currentstroke}%
\pgfsetstrokeopacity{0.400000}%
\pgfsetdash{}{0pt}%
\pgfpathmoveto{\pgfqpoint{2.254838in}{2.590939in}}%
\pgfpathcurveto{\pgfqpoint{2.265888in}{2.590939in}}{\pgfqpoint{2.276487in}{2.595330in}}{\pgfqpoint{2.284300in}{2.603143in}}%
\pgfpathcurveto{\pgfqpoint{2.292114in}{2.610957in}}{\pgfqpoint{2.296504in}{2.621556in}}{\pgfqpoint{2.296504in}{2.632606in}}%
\pgfpathcurveto{\pgfqpoint{2.296504in}{2.643656in}}{\pgfqpoint{2.292114in}{2.654255in}}{\pgfqpoint{2.284300in}{2.662069in}}%
\pgfpathcurveto{\pgfqpoint{2.276487in}{2.669882in}}{\pgfqpoint{2.265888in}{2.674273in}}{\pgfqpoint{2.254838in}{2.674273in}}%
\pgfpathcurveto{\pgfqpoint{2.243787in}{2.674273in}}{\pgfqpoint{2.233188in}{2.669882in}}{\pgfqpoint{2.225375in}{2.662069in}}%
\pgfpathcurveto{\pgfqpoint{2.217561in}{2.654255in}}{\pgfqpoint{2.213171in}{2.643656in}}{\pgfqpoint{2.213171in}{2.632606in}}%
\pgfpathcurveto{\pgfqpoint{2.213171in}{2.621556in}}{\pgfqpoint{2.217561in}{2.610957in}}{\pgfqpoint{2.225375in}{2.603143in}}%
\pgfpathcurveto{\pgfqpoint{2.233188in}{2.595330in}}{\pgfqpoint{2.243787in}{2.590939in}}{\pgfqpoint{2.254838in}{2.590939in}}%
\pgfpathclose%
\pgfusepath{stroke,fill}%
\end{pgfscope}%
\begin{pgfscope}%
\pgfpathrectangle{\pgfqpoint{0.511823in}{0.504323in}}{\pgfqpoint{3.218177in}{3.225677in}} %
\pgfusepath{clip}%
\pgfsetbuttcap%
\pgfsetroundjoin%
\definecolor{currentfill}{rgb}{0.501961,0.000000,0.000000}%
\pgfsetfillcolor{currentfill}%
\pgfsetfillopacity{0.400000}%
\pgfsetlinewidth{0.501875pt}%
\definecolor{currentstroke}{rgb}{0.501961,0.000000,0.000000}%
\pgfsetstrokecolor{currentstroke}%
\pgfsetstrokeopacity{0.400000}%
\pgfsetdash{}{0pt}%
\pgfpathmoveto{\pgfqpoint{2.451393in}{2.849937in}}%
\pgfpathcurveto{\pgfqpoint{2.462443in}{2.849937in}}{\pgfqpoint{2.473042in}{2.854327in}}{\pgfqpoint{2.480855in}{2.862141in}}%
\pgfpathcurveto{\pgfqpoint{2.488669in}{2.869954in}}{\pgfqpoint{2.493059in}{2.880553in}}{\pgfqpoint{2.493059in}{2.891604in}}%
\pgfpathcurveto{\pgfqpoint{2.493059in}{2.902654in}}{\pgfqpoint{2.488669in}{2.913253in}}{\pgfqpoint{2.480855in}{2.921066in}}%
\pgfpathcurveto{\pgfqpoint{2.473042in}{2.928880in}}{\pgfqpoint{2.462443in}{2.933270in}}{\pgfqpoint{2.451393in}{2.933270in}}%
\pgfpathcurveto{\pgfqpoint{2.440342in}{2.933270in}}{\pgfqpoint{2.429743in}{2.928880in}}{\pgfqpoint{2.421930in}{2.921066in}}%
\pgfpathcurveto{\pgfqpoint{2.414116in}{2.913253in}}{\pgfqpoint{2.409726in}{2.902654in}}{\pgfqpoint{2.409726in}{2.891604in}}%
\pgfpathcurveto{\pgfqpoint{2.409726in}{2.880553in}}{\pgfqpoint{2.414116in}{2.869954in}}{\pgfqpoint{2.421930in}{2.862141in}}%
\pgfpathcurveto{\pgfqpoint{2.429743in}{2.854327in}}{\pgfqpoint{2.440342in}{2.849937in}}{\pgfqpoint{2.451393in}{2.849937in}}%
\pgfpathclose%
\pgfusepath{stroke,fill}%
\end{pgfscope}%
\begin{pgfscope}%
\pgfpathrectangle{\pgfqpoint{0.511823in}{0.504323in}}{\pgfqpoint{3.218177in}{3.225677in}} %
\pgfusepath{clip}%
\pgfsetbuttcap%
\pgfsetroundjoin%
\definecolor{currentfill}{rgb}{0.501961,0.000000,0.000000}%
\pgfsetfillcolor{currentfill}%
\pgfsetfillopacity{0.400000}%
\pgfsetlinewidth{0.501875pt}%
\definecolor{currentstroke}{rgb}{0.501961,0.000000,0.000000}%
\pgfsetstrokecolor{currentstroke}%
\pgfsetstrokeopacity{0.400000}%
\pgfsetdash{}{0pt}%
\pgfpathmoveto{\pgfqpoint{2.222116in}{2.575523in}}%
\pgfpathcurveto{\pgfqpoint{2.233166in}{2.575523in}}{\pgfqpoint{2.243765in}{2.579913in}}{\pgfqpoint{2.251579in}{2.587727in}}%
\pgfpathcurveto{\pgfqpoint{2.259392in}{2.595540in}}{\pgfqpoint{2.263783in}{2.606139in}}{\pgfqpoint{2.263783in}{2.617189in}}%
\pgfpathcurveto{\pgfqpoint{2.263783in}{2.628239in}}{\pgfqpoint{2.259392in}{2.638838in}}{\pgfqpoint{2.251579in}{2.646652in}}%
\pgfpathcurveto{\pgfqpoint{2.243765in}{2.654466in}}{\pgfqpoint{2.233166in}{2.658856in}}{\pgfqpoint{2.222116in}{2.658856in}}%
\pgfpathcurveto{\pgfqpoint{2.211066in}{2.658856in}}{\pgfqpoint{2.200467in}{2.654466in}}{\pgfqpoint{2.192653in}{2.646652in}}%
\pgfpathcurveto{\pgfqpoint{2.184840in}{2.638838in}}{\pgfqpoint{2.180449in}{2.628239in}}{\pgfqpoint{2.180449in}{2.617189in}}%
\pgfpathcurveto{\pgfqpoint{2.180449in}{2.606139in}}{\pgfqpoint{2.184840in}{2.595540in}}{\pgfqpoint{2.192653in}{2.587727in}}%
\pgfpathcurveto{\pgfqpoint{2.200467in}{2.579913in}}{\pgfqpoint{2.211066in}{2.575523in}}{\pgfqpoint{2.222116in}{2.575523in}}%
\pgfpathclose%
\pgfusepath{stroke,fill}%
\end{pgfscope}%
\begin{pgfscope}%
\pgfpathrectangle{\pgfqpoint{0.511823in}{0.504323in}}{\pgfqpoint{3.218177in}{3.225677in}} %
\pgfusepath{clip}%
\pgfsetbuttcap%
\pgfsetroundjoin%
\definecolor{currentfill}{rgb}{0.501961,0.000000,0.000000}%
\pgfsetfillcolor{currentfill}%
\pgfsetfillopacity{0.400000}%
\pgfsetlinewidth{0.501875pt}%
\definecolor{currentstroke}{rgb}{0.501961,0.000000,0.000000}%
\pgfsetstrokecolor{currentstroke}%
\pgfsetstrokeopacity{0.400000}%
\pgfsetdash{}{0pt}%
\pgfpathmoveto{\pgfqpoint{2.431771in}{2.854355in}}%
\pgfpathcurveto{\pgfqpoint{2.442821in}{2.854355in}}{\pgfqpoint{2.453420in}{2.858745in}}{\pgfqpoint{2.461234in}{2.866559in}}%
\pgfpathcurveto{\pgfqpoint{2.469047in}{2.874372in}}{\pgfqpoint{2.473438in}{2.884971in}}{\pgfqpoint{2.473438in}{2.896021in}}%
\pgfpathcurveto{\pgfqpoint{2.473438in}{2.907071in}}{\pgfqpoint{2.469047in}{2.917671in}}{\pgfqpoint{2.461234in}{2.925484in}}%
\pgfpathcurveto{\pgfqpoint{2.453420in}{2.933298in}}{\pgfqpoint{2.442821in}{2.937688in}}{\pgfqpoint{2.431771in}{2.937688in}}%
\pgfpathcurveto{\pgfqpoint{2.420721in}{2.937688in}}{\pgfqpoint{2.410122in}{2.933298in}}{\pgfqpoint{2.402308in}{2.925484in}}%
\pgfpathcurveto{\pgfqpoint{2.394495in}{2.917671in}}{\pgfqpoint{2.390104in}{2.907071in}}{\pgfqpoint{2.390104in}{2.896021in}}%
\pgfpathcurveto{\pgfqpoint{2.390104in}{2.884971in}}{\pgfqpoint{2.394495in}{2.874372in}}{\pgfqpoint{2.402308in}{2.866559in}}%
\pgfpathcurveto{\pgfqpoint{2.410122in}{2.858745in}}{\pgfqpoint{2.420721in}{2.854355in}}{\pgfqpoint{2.431771in}{2.854355in}}%
\pgfpathclose%
\pgfusepath{stroke,fill}%
\end{pgfscope}%
\begin{pgfscope}%
\pgfpathrectangle{\pgfqpoint{0.511823in}{0.504323in}}{\pgfqpoint{3.218177in}{3.225677in}} %
\pgfusepath{clip}%
\pgfsetbuttcap%
\pgfsetroundjoin%
\definecolor{currentfill}{rgb}{0.501961,0.000000,0.000000}%
\pgfsetfillcolor{currentfill}%
\pgfsetfillopacity{0.400000}%
\pgfsetlinewidth{0.501875pt}%
\definecolor{currentstroke}{rgb}{0.501961,0.000000,0.000000}%
\pgfsetstrokecolor{currentstroke}%
\pgfsetstrokeopacity{0.400000}%
\pgfsetdash{}{0pt}%
\pgfpathmoveto{\pgfqpoint{2.009524in}{2.329710in}}%
\pgfpathcurveto{\pgfqpoint{2.020574in}{2.329710in}}{\pgfqpoint{2.031173in}{2.334100in}}{\pgfqpoint{2.038987in}{2.341914in}}%
\pgfpathcurveto{\pgfqpoint{2.046801in}{2.349727in}}{\pgfqpoint{2.051191in}{2.360326in}}{\pgfqpoint{2.051191in}{2.371376in}}%
\pgfpathcurveto{\pgfqpoint{2.051191in}{2.382427in}}{\pgfqpoint{2.046801in}{2.393026in}}{\pgfqpoint{2.038987in}{2.400839in}}%
\pgfpathcurveto{\pgfqpoint{2.031173in}{2.408653in}}{\pgfqpoint{2.020574in}{2.413043in}}{\pgfqpoint{2.009524in}{2.413043in}}%
\pgfpathcurveto{\pgfqpoint{1.998474in}{2.413043in}}{\pgfqpoint{1.987875in}{2.408653in}}{\pgfqpoint{1.980061in}{2.400839in}}%
\pgfpathcurveto{\pgfqpoint{1.972248in}{2.393026in}}{\pgfqpoint{1.967858in}{2.382427in}}{\pgfqpoint{1.967858in}{2.371376in}}%
\pgfpathcurveto{\pgfqpoint{1.967858in}{2.360326in}}{\pgfqpoint{1.972248in}{2.349727in}}{\pgfqpoint{1.980061in}{2.341914in}}%
\pgfpathcurveto{\pgfqpoint{1.987875in}{2.334100in}}{\pgfqpoint{1.998474in}{2.329710in}}{\pgfqpoint{2.009524in}{2.329710in}}%
\pgfpathclose%
\pgfusepath{stroke,fill}%
\end{pgfscope}%
\begin{pgfscope}%
\pgfpathrectangle{\pgfqpoint{0.511823in}{0.504323in}}{\pgfqpoint{3.218177in}{3.225677in}} %
\pgfusepath{clip}%
\pgfsetbuttcap%
\pgfsetroundjoin%
\definecolor{currentfill}{rgb}{0.501961,0.000000,0.000000}%
\pgfsetfillcolor{currentfill}%
\pgfsetfillopacity{0.400000}%
\pgfsetlinewidth{0.501875pt}%
\definecolor{currentstroke}{rgb}{0.501961,0.000000,0.000000}%
\pgfsetstrokecolor{currentstroke}%
\pgfsetstrokeopacity{0.400000}%
\pgfsetdash{}{0pt}%
\pgfpathmoveto{\pgfqpoint{2.194525in}{2.578804in}}%
\pgfpathcurveto{\pgfqpoint{2.205575in}{2.578804in}}{\pgfqpoint{2.216174in}{2.583194in}}{\pgfqpoint{2.223988in}{2.591008in}}%
\pgfpathcurveto{\pgfqpoint{2.231801in}{2.598821in}}{\pgfqpoint{2.236192in}{2.609420in}}{\pgfqpoint{2.236192in}{2.620470in}}%
\pgfpathcurveto{\pgfqpoint{2.236192in}{2.631520in}}{\pgfqpoint{2.231801in}{2.642120in}}{\pgfqpoint{2.223988in}{2.649933in}}%
\pgfpathcurveto{\pgfqpoint{2.216174in}{2.657747in}}{\pgfqpoint{2.205575in}{2.662137in}}{\pgfqpoint{2.194525in}{2.662137in}}%
\pgfpathcurveto{\pgfqpoint{2.183475in}{2.662137in}}{\pgfqpoint{2.172876in}{2.657747in}}{\pgfqpoint{2.165062in}{2.649933in}}%
\pgfpathcurveto{\pgfqpoint{2.157249in}{2.642120in}}{\pgfqpoint{2.152858in}{2.631520in}}{\pgfqpoint{2.152858in}{2.620470in}}%
\pgfpathcurveto{\pgfqpoint{2.152858in}{2.609420in}}{\pgfqpoint{2.157249in}{2.598821in}}{\pgfqpoint{2.165062in}{2.591008in}}%
\pgfpathcurveto{\pgfqpoint{2.172876in}{2.583194in}}{\pgfqpoint{2.183475in}{2.578804in}}{\pgfqpoint{2.194525in}{2.578804in}}%
\pgfpathclose%
\pgfusepath{stroke,fill}%
\end{pgfscope}%
\begin{pgfscope}%
\pgfpathrectangle{\pgfqpoint{0.511823in}{0.504323in}}{\pgfqpoint{3.218177in}{3.225677in}} %
\pgfusepath{clip}%
\pgfsetbuttcap%
\pgfsetroundjoin%
\definecolor{currentfill}{rgb}{0.501961,0.000000,0.000000}%
\pgfsetfillcolor{currentfill}%
\pgfsetfillopacity{0.400000}%
\pgfsetlinewidth{0.501875pt}%
\definecolor{currentstroke}{rgb}{0.501961,0.000000,0.000000}%
\pgfsetstrokecolor{currentstroke}%
\pgfsetstrokeopacity{0.400000}%
\pgfsetdash{}{0pt}%
\pgfpathmoveto{\pgfqpoint{2.347823in}{2.790129in}}%
\pgfpathcurveto{\pgfqpoint{2.358873in}{2.790129in}}{\pgfqpoint{2.369473in}{2.794520in}}{\pgfqpoint{2.377286in}{2.802333in}}%
\pgfpathcurveto{\pgfqpoint{2.385100in}{2.810147in}}{\pgfqpoint{2.389490in}{2.820746in}}{\pgfqpoint{2.389490in}{2.831796in}}%
\pgfpathcurveto{\pgfqpoint{2.389490in}{2.842846in}}{\pgfqpoint{2.385100in}{2.853445in}}{\pgfqpoint{2.377286in}{2.861259in}}%
\pgfpathcurveto{\pgfqpoint{2.369473in}{2.869072in}}{\pgfqpoint{2.358873in}{2.873463in}}{\pgfqpoint{2.347823in}{2.873463in}}%
\pgfpathcurveto{\pgfqpoint{2.336773in}{2.873463in}}{\pgfqpoint{2.326174in}{2.869072in}}{\pgfqpoint{2.318361in}{2.861259in}}%
\pgfpathcurveto{\pgfqpoint{2.310547in}{2.853445in}}{\pgfqpoint{2.306157in}{2.842846in}}{\pgfqpoint{2.306157in}{2.831796in}}%
\pgfpathcurveto{\pgfqpoint{2.306157in}{2.820746in}}{\pgfqpoint{2.310547in}{2.810147in}}{\pgfqpoint{2.318361in}{2.802333in}}%
\pgfpathcurveto{\pgfqpoint{2.326174in}{2.794520in}}{\pgfqpoint{2.336773in}{2.790129in}}{\pgfqpoint{2.347823in}{2.790129in}}%
\pgfpathclose%
\pgfusepath{stroke,fill}%
\end{pgfscope}%
\begin{pgfscope}%
\pgfpathrectangle{\pgfqpoint{0.511823in}{0.504323in}}{\pgfqpoint{3.218177in}{3.225677in}} %
\pgfusepath{clip}%
\pgfsetbuttcap%
\pgfsetroundjoin%
\definecolor{currentfill}{rgb}{0.501961,0.000000,0.000000}%
\pgfsetfillcolor{currentfill}%
\pgfsetfillopacity{0.400000}%
\pgfsetlinewidth{0.501875pt}%
\definecolor{currentstroke}{rgb}{0.501961,0.000000,0.000000}%
\pgfsetstrokecolor{currentstroke}%
\pgfsetstrokeopacity{0.400000}%
\pgfsetdash{}{0pt}%
\pgfpathmoveto{\pgfqpoint{2.355439in}{2.814398in}}%
\pgfpathcurveto{\pgfqpoint{2.366490in}{2.814398in}}{\pgfqpoint{2.377089in}{2.818788in}}{\pgfqpoint{2.384902in}{2.826602in}}%
\pgfpathcurveto{\pgfqpoint{2.392716in}{2.834415in}}{\pgfqpoint{2.397106in}{2.845015in}}{\pgfqpoint{2.397106in}{2.856065in}}%
\pgfpathcurveto{\pgfqpoint{2.397106in}{2.867115in}}{\pgfqpoint{2.392716in}{2.877714in}}{\pgfqpoint{2.384902in}{2.885527in}}%
\pgfpathcurveto{\pgfqpoint{2.377089in}{2.893341in}}{\pgfqpoint{2.366490in}{2.897731in}}{\pgfqpoint{2.355439in}{2.897731in}}%
\pgfpathcurveto{\pgfqpoint{2.344389in}{2.897731in}}{\pgfqpoint{2.333790in}{2.893341in}}{\pgfqpoint{2.325977in}{2.885527in}}%
\pgfpathcurveto{\pgfqpoint{2.318163in}{2.877714in}}{\pgfqpoint{2.313773in}{2.867115in}}{\pgfqpoint{2.313773in}{2.856065in}}%
\pgfpathcurveto{\pgfqpoint{2.313773in}{2.845015in}}{\pgfqpoint{2.318163in}{2.834415in}}{\pgfqpoint{2.325977in}{2.826602in}}%
\pgfpathcurveto{\pgfqpoint{2.333790in}{2.818788in}}{\pgfqpoint{2.344389in}{2.814398in}}{\pgfqpoint{2.355439in}{2.814398in}}%
\pgfpathclose%
\pgfusepath{stroke,fill}%
\end{pgfscope}%
\begin{pgfscope}%
\pgfpathrectangle{\pgfqpoint{0.511823in}{0.504323in}}{\pgfqpoint{3.218177in}{3.225677in}} %
\pgfusepath{clip}%
\pgfsetbuttcap%
\pgfsetroundjoin%
\definecolor{currentfill}{rgb}{0.501961,0.000000,0.000000}%
\pgfsetfillcolor{currentfill}%
\pgfsetfillopacity{0.400000}%
\pgfsetlinewidth{0.501875pt}%
\definecolor{currentstroke}{rgb}{0.501961,0.000000,0.000000}%
\pgfsetstrokecolor{currentstroke}%
\pgfsetstrokeopacity{0.400000}%
\pgfsetdash{}{0pt}%
\pgfpathmoveto{\pgfqpoint{2.260969in}{2.705023in}}%
\pgfpathcurveto{\pgfqpoint{2.272019in}{2.705023in}}{\pgfqpoint{2.282618in}{2.709413in}}{\pgfqpoint{2.290432in}{2.717227in}}%
\pgfpathcurveto{\pgfqpoint{2.298245in}{2.725040in}}{\pgfqpoint{2.302636in}{2.735639in}}{\pgfqpoint{2.302636in}{2.746689in}}%
\pgfpathcurveto{\pgfqpoint{2.302636in}{2.757740in}}{\pgfqpoint{2.298245in}{2.768339in}}{\pgfqpoint{2.290432in}{2.776152in}}%
\pgfpathcurveto{\pgfqpoint{2.282618in}{2.783966in}}{\pgfqpoint{2.272019in}{2.788356in}}{\pgfqpoint{2.260969in}{2.788356in}}%
\pgfpathcurveto{\pgfqpoint{2.249919in}{2.788356in}}{\pgfqpoint{2.239320in}{2.783966in}}{\pgfqpoint{2.231506in}{2.776152in}}%
\pgfpathcurveto{\pgfqpoint{2.223693in}{2.768339in}}{\pgfqpoint{2.219302in}{2.757740in}}{\pgfqpoint{2.219302in}{2.746689in}}%
\pgfpathcurveto{\pgfqpoint{2.219302in}{2.735639in}}{\pgfqpoint{2.223693in}{2.725040in}}{\pgfqpoint{2.231506in}{2.717227in}}%
\pgfpathcurveto{\pgfqpoint{2.239320in}{2.709413in}}{\pgfqpoint{2.249919in}{2.705023in}}{\pgfqpoint{2.260969in}{2.705023in}}%
\pgfpathclose%
\pgfusepath{stroke,fill}%
\end{pgfscope}%
\begin{pgfscope}%
\pgfpathrectangle{\pgfqpoint{0.511823in}{0.504323in}}{\pgfqpoint{3.218177in}{3.225677in}} %
\pgfusepath{clip}%
\pgfsetbuttcap%
\pgfsetroundjoin%
\definecolor{currentfill}{rgb}{0.501961,0.000000,0.000000}%
\pgfsetfillcolor{currentfill}%
\pgfsetfillopacity{0.400000}%
\pgfsetlinewidth{0.501875pt}%
\definecolor{currentstroke}{rgb}{0.501961,0.000000,0.000000}%
\pgfsetstrokecolor{currentstroke}%
\pgfsetstrokeopacity{0.400000}%
\pgfsetdash{}{0pt}%
\pgfpathmoveto{\pgfqpoint{2.169429in}{2.597986in}}%
\pgfpathcurveto{\pgfqpoint{2.180479in}{2.597986in}}{\pgfqpoint{2.191078in}{2.602376in}}{\pgfqpoint{2.198891in}{2.610189in}}%
\pgfpathcurveto{\pgfqpoint{2.206705in}{2.618003in}}{\pgfqpoint{2.211095in}{2.628602in}}{\pgfqpoint{2.211095in}{2.639652in}}%
\pgfpathcurveto{\pgfqpoint{2.211095in}{2.650702in}}{\pgfqpoint{2.206705in}{2.661301in}}{\pgfqpoint{2.198891in}{2.669115in}}%
\pgfpathcurveto{\pgfqpoint{2.191078in}{2.676929in}}{\pgfqpoint{2.180479in}{2.681319in}}{\pgfqpoint{2.169429in}{2.681319in}}%
\pgfpathcurveto{\pgfqpoint{2.158378in}{2.681319in}}{\pgfqpoint{2.147779in}{2.676929in}}{\pgfqpoint{2.139966in}{2.669115in}}%
\pgfpathcurveto{\pgfqpoint{2.132152in}{2.661301in}}{\pgfqpoint{2.127762in}{2.650702in}}{\pgfqpoint{2.127762in}{2.639652in}}%
\pgfpathcurveto{\pgfqpoint{2.127762in}{2.628602in}}{\pgfqpoint{2.132152in}{2.618003in}}{\pgfqpoint{2.139966in}{2.610189in}}%
\pgfpathcurveto{\pgfqpoint{2.147779in}{2.602376in}}{\pgfqpoint{2.158378in}{2.597986in}}{\pgfqpoint{2.169429in}{2.597986in}}%
\pgfpathclose%
\pgfusepath{stroke,fill}%
\end{pgfscope}%
\begin{pgfscope}%
\pgfpathrectangle{\pgfqpoint{0.511823in}{0.504323in}}{\pgfqpoint{3.218177in}{3.225677in}} %
\pgfusepath{clip}%
\pgfsetbuttcap%
\pgfsetroundjoin%
\definecolor{currentfill}{rgb}{0.501961,0.000000,0.000000}%
\pgfsetfillcolor{currentfill}%
\pgfsetfillopacity{0.400000}%
\pgfsetlinewidth{0.501875pt}%
\definecolor{currentstroke}{rgb}{0.501961,0.000000,0.000000}%
\pgfsetstrokecolor{currentstroke}%
\pgfsetstrokeopacity{0.400000}%
\pgfsetdash{}{0pt}%
\pgfpathmoveto{\pgfqpoint{2.201660in}{2.653982in}}%
\pgfpathcurveto{\pgfqpoint{2.212710in}{2.653982in}}{\pgfqpoint{2.223309in}{2.658372in}}{\pgfqpoint{2.231123in}{2.666186in}}%
\pgfpathcurveto{\pgfqpoint{2.238936in}{2.673999in}}{\pgfqpoint{2.243326in}{2.684598in}}{\pgfqpoint{2.243326in}{2.695648in}}%
\pgfpathcurveto{\pgfqpoint{2.243326in}{2.706699in}}{\pgfqpoint{2.238936in}{2.717298in}}{\pgfqpoint{2.231123in}{2.725111in}}%
\pgfpathcurveto{\pgfqpoint{2.223309in}{2.732925in}}{\pgfqpoint{2.212710in}{2.737315in}}{\pgfqpoint{2.201660in}{2.737315in}}%
\pgfpathcurveto{\pgfqpoint{2.190610in}{2.737315in}}{\pgfqpoint{2.180011in}{2.732925in}}{\pgfqpoint{2.172197in}{2.725111in}}%
\pgfpathcurveto{\pgfqpoint{2.164383in}{2.717298in}}{\pgfqpoint{2.159993in}{2.706699in}}{\pgfqpoint{2.159993in}{2.695648in}}%
\pgfpathcurveto{\pgfqpoint{2.159993in}{2.684598in}}{\pgfqpoint{2.164383in}{2.673999in}}{\pgfqpoint{2.172197in}{2.666186in}}%
\pgfpathcurveto{\pgfqpoint{2.180011in}{2.658372in}}{\pgfqpoint{2.190610in}{2.653982in}}{\pgfqpoint{2.201660in}{2.653982in}}%
\pgfpathclose%
\pgfusepath{stroke,fill}%
\end{pgfscope}%
\begin{pgfscope}%
\pgfpathrectangle{\pgfqpoint{0.511823in}{0.504323in}}{\pgfqpoint{3.218177in}{3.225677in}} %
\pgfusepath{clip}%
\pgfsetbuttcap%
\pgfsetroundjoin%
\definecolor{currentfill}{rgb}{0.501961,0.000000,0.000000}%
\pgfsetfillcolor{currentfill}%
\pgfsetfillopacity{0.400000}%
\pgfsetlinewidth{0.501875pt}%
\definecolor{currentstroke}{rgb}{0.501961,0.000000,0.000000}%
\pgfsetstrokecolor{currentstroke}%
\pgfsetstrokeopacity{0.400000}%
\pgfsetdash{}{0pt}%
\pgfpathmoveto{\pgfqpoint{2.264478in}{2.751608in}}%
\pgfpathcurveto{\pgfqpoint{2.275529in}{2.751608in}}{\pgfqpoint{2.286128in}{2.755998in}}{\pgfqpoint{2.293941in}{2.763812in}}%
\pgfpathcurveto{\pgfqpoint{2.301755in}{2.771625in}}{\pgfqpoint{2.306145in}{2.782224in}}{\pgfqpoint{2.306145in}{2.793274in}}%
\pgfpathcurveto{\pgfqpoint{2.306145in}{2.804325in}}{\pgfqpoint{2.301755in}{2.814924in}}{\pgfqpoint{2.293941in}{2.822737in}}%
\pgfpathcurveto{\pgfqpoint{2.286128in}{2.830551in}}{\pgfqpoint{2.275529in}{2.834941in}}{\pgfqpoint{2.264478in}{2.834941in}}%
\pgfpathcurveto{\pgfqpoint{2.253428in}{2.834941in}}{\pgfqpoint{2.242829in}{2.830551in}}{\pgfqpoint{2.235016in}{2.822737in}}%
\pgfpathcurveto{\pgfqpoint{2.227202in}{2.814924in}}{\pgfqpoint{2.222812in}{2.804325in}}{\pgfqpoint{2.222812in}{2.793274in}}%
\pgfpathcurveto{\pgfqpoint{2.222812in}{2.782224in}}{\pgfqpoint{2.227202in}{2.771625in}}{\pgfqpoint{2.235016in}{2.763812in}}%
\pgfpathcurveto{\pgfqpoint{2.242829in}{2.755998in}}{\pgfqpoint{2.253428in}{2.751608in}}{\pgfqpoint{2.264478in}{2.751608in}}%
\pgfpathclose%
\pgfusepath{stroke,fill}%
\end{pgfscope}%
\begin{pgfscope}%
\pgfpathrectangle{\pgfqpoint{0.511823in}{0.504323in}}{\pgfqpoint{3.218177in}{3.225677in}} %
\pgfusepath{clip}%
\pgfsetbuttcap%
\pgfsetroundjoin%
\definecolor{currentfill}{rgb}{0.501961,0.000000,0.000000}%
\pgfsetfillcolor{currentfill}%
\pgfsetfillopacity{0.400000}%
\pgfsetlinewidth{0.501875pt}%
\definecolor{currentstroke}{rgb}{0.501961,0.000000,0.000000}%
\pgfsetstrokecolor{currentstroke}%
\pgfsetstrokeopacity{0.400000}%
\pgfsetdash{}{0pt}%
\pgfpathmoveto{\pgfqpoint{2.222500in}{2.709298in}}%
\pgfpathcurveto{\pgfqpoint{2.233550in}{2.709298in}}{\pgfqpoint{2.244149in}{2.713689in}}{\pgfqpoint{2.251963in}{2.721502in}}%
\pgfpathcurveto{\pgfqpoint{2.259776in}{2.729316in}}{\pgfqpoint{2.264167in}{2.739915in}}{\pgfqpoint{2.264167in}{2.750965in}}%
\pgfpathcurveto{\pgfqpoint{2.264167in}{2.762015in}}{\pgfqpoint{2.259776in}{2.772614in}}{\pgfqpoint{2.251963in}{2.780428in}}%
\pgfpathcurveto{\pgfqpoint{2.244149in}{2.788241in}}{\pgfqpoint{2.233550in}{2.792632in}}{\pgfqpoint{2.222500in}{2.792632in}}%
\pgfpathcurveto{\pgfqpoint{2.211450in}{2.792632in}}{\pgfqpoint{2.200851in}{2.788241in}}{\pgfqpoint{2.193037in}{2.780428in}}%
\pgfpathcurveto{\pgfqpoint{2.185224in}{2.772614in}}{\pgfqpoint{2.180833in}{2.762015in}}{\pgfqpoint{2.180833in}{2.750965in}}%
\pgfpathcurveto{\pgfqpoint{2.180833in}{2.739915in}}{\pgfqpoint{2.185224in}{2.729316in}}{\pgfqpoint{2.193037in}{2.721502in}}%
\pgfpathcurveto{\pgfqpoint{2.200851in}{2.713689in}}{\pgfqpoint{2.211450in}{2.709298in}}{\pgfqpoint{2.222500in}{2.709298in}}%
\pgfpathclose%
\pgfusepath{stroke,fill}%
\end{pgfscope}%
\begin{pgfscope}%
\pgfpathrectangle{\pgfqpoint{0.511823in}{0.504323in}}{\pgfqpoint{3.218177in}{3.225677in}} %
\pgfusepath{clip}%
\pgfsetbuttcap%
\pgfsetroundjoin%
\definecolor{currentfill}{rgb}{0.501961,0.000000,0.000000}%
\pgfsetfillcolor{currentfill}%
\pgfsetfillopacity{0.400000}%
\pgfsetlinewidth{0.501875pt}%
\definecolor{currentstroke}{rgb}{0.501961,0.000000,0.000000}%
\pgfsetstrokecolor{currentstroke}%
\pgfsetstrokeopacity{0.400000}%
\pgfsetdash{}{0pt}%
\pgfpathmoveto{\pgfqpoint{2.263287in}{2.778619in}}%
\pgfpathcurveto{\pgfqpoint{2.274337in}{2.778619in}}{\pgfqpoint{2.284936in}{2.783009in}}{\pgfqpoint{2.292750in}{2.790823in}}%
\pgfpathcurveto{\pgfqpoint{2.300564in}{2.798636in}}{\pgfqpoint{2.304954in}{2.809235in}}{\pgfqpoint{2.304954in}{2.820286in}}%
\pgfpathcurveto{\pgfqpoint{2.304954in}{2.831336in}}{\pgfqpoint{2.300564in}{2.841935in}}{\pgfqpoint{2.292750in}{2.849748in}}%
\pgfpathcurveto{\pgfqpoint{2.284936in}{2.857562in}}{\pgfqpoint{2.274337in}{2.861952in}}{\pgfqpoint{2.263287in}{2.861952in}}%
\pgfpathcurveto{\pgfqpoint{2.252237in}{2.861952in}}{\pgfqpoint{2.241638in}{2.857562in}}{\pgfqpoint{2.233824in}{2.849748in}}%
\pgfpathcurveto{\pgfqpoint{2.226011in}{2.841935in}}{\pgfqpoint{2.221620in}{2.831336in}}{\pgfqpoint{2.221620in}{2.820286in}}%
\pgfpathcurveto{\pgfqpoint{2.221620in}{2.809235in}}{\pgfqpoint{2.226011in}{2.798636in}}{\pgfqpoint{2.233824in}{2.790823in}}%
\pgfpathcurveto{\pgfqpoint{2.241638in}{2.783009in}}{\pgfqpoint{2.252237in}{2.778619in}}{\pgfqpoint{2.263287in}{2.778619in}}%
\pgfpathclose%
\pgfusepath{stroke,fill}%
\end{pgfscope}%
\begin{pgfscope}%
\pgfpathrectangle{\pgfqpoint{0.511823in}{0.504323in}}{\pgfqpoint{3.218177in}{3.225677in}} %
\pgfusepath{clip}%
\pgfsetbuttcap%
\pgfsetroundjoin%
\definecolor{currentfill}{rgb}{0.501961,0.000000,0.000000}%
\pgfsetfillcolor{currentfill}%
\pgfsetfillopacity{0.400000}%
\pgfsetlinewidth{0.501875pt}%
\definecolor{currentstroke}{rgb}{0.501961,0.000000,0.000000}%
\pgfsetstrokecolor{currentstroke}%
\pgfsetstrokeopacity{0.400000}%
\pgfsetdash{}{0pt}%
\pgfpathmoveto{\pgfqpoint{2.131072in}{2.612564in}}%
\pgfpathcurveto{\pgfqpoint{2.142122in}{2.612564in}}{\pgfqpoint{2.152721in}{2.616955in}}{\pgfqpoint{2.160534in}{2.624768in}}%
\pgfpathcurveto{\pgfqpoint{2.168348in}{2.632582in}}{\pgfqpoint{2.172738in}{2.643181in}}{\pgfqpoint{2.172738in}{2.654231in}}%
\pgfpathcurveto{\pgfqpoint{2.172738in}{2.665281in}}{\pgfqpoint{2.168348in}{2.675880in}}{\pgfqpoint{2.160534in}{2.683694in}}%
\pgfpathcurveto{\pgfqpoint{2.152721in}{2.691508in}}{\pgfqpoint{2.142122in}{2.695898in}}{\pgfqpoint{2.131072in}{2.695898in}}%
\pgfpathcurveto{\pgfqpoint{2.120022in}{2.695898in}}{\pgfqpoint{2.109422in}{2.691508in}}{\pgfqpoint{2.101609in}{2.683694in}}%
\pgfpathcurveto{\pgfqpoint{2.093795in}{2.675880in}}{\pgfqpoint{2.089405in}{2.665281in}}{\pgfqpoint{2.089405in}{2.654231in}}%
\pgfpathcurveto{\pgfqpoint{2.089405in}{2.643181in}}{\pgfqpoint{2.093795in}{2.632582in}}{\pgfqpoint{2.101609in}{2.624768in}}%
\pgfpathcurveto{\pgfqpoint{2.109422in}{2.616955in}}{\pgfqpoint{2.120022in}{2.612564in}}{\pgfqpoint{2.131072in}{2.612564in}}%
\pgfpathclose%
\pgfusepath{stroke,fill}%
\end{pgfscope}%
\begin{pgfscope}%
\pgfpathrectangle{\pgfqpoint{0.511823in}{0.504323in}}{\pgfqpoint{3.218177in}{3.225677in}} %
\pgfusepath{clip}%
\pgfsetbuttcap%
\pgfsetroundjoin%
\definecolor{currentfill}{rgb}{0.501961,0.000000,0.000000}%
\pgfsetfillcolor{currentfill}%
\pgfsetfillopacity{0.400000}%
\pgfsetlinewidth{0.501875pt}%
\definecolor{currentstroke}{rgb}{0.501961,0.000000,0.000000}%
\pgfsetstrokecolor{currentstroke}%
\pgfsetstrokeopacity{0.400000}%
\pgfsetdash{}{0pt}%
\pgfpathmoveto{\pgfqpoint{2.240156in}{2.775952in}}%
\pgfpathcurveto{\pgfqpoint{2.251207in}{2.775952in}}{\pgfqpoint{2.261806in}{2.780342in}}{\pgfqpoint{2.269619in}{2.788156in}}%
\pgfpathcurveto{\pgfqpoint{2.277433in}{2.795969in}}{\pgfqpoint{2.281823in}{2.806568in}}{\pgfqpoint{2.281823in}{2.817619in}}%
\pgfpathcurveto{\pgfqpoint{2.281823in}{2.828669in}}{\pgfqpoint{2.277433in}{2.839268in}}{\pgfqpoint{2.269619in}{2.847081in}}%
\pgfpathcurveto{\pgfqpoint{2.261806in}{2.854895in}}{\pgfqpoint{2.251207in}{2.859285in}}{\pgfqpoint{2.240156in}{2.859285in}}%
\pgfpathcurveto{\pgfqpoint{2.229106in}{2.859285in}}{\pgfqpoint{2.218507in}{2.854895in}}{\pgfqpoint{2.210694in}{2.847081in}}%
\pgfpathcurveto{\pgfqpoint{2.202880in}{2.839268in}}{\pgfqpoint{2.198490in}{2.828669in}}{\pgfqpoint{2.198490in}{2.817619in}}%
\pgfpathcurveto{\pgfqpoint{2.198490in}{2.806568in}}{\pgfqpoint{2.202880in}{2.795969in}}{\pgfqpoint{2.210694in}{2.788156in}}%
\pgfpathcurveto{\pgfqpoint{2.218507in}{2.780342in}}{\pgfqpoint{2.229106in}{2.775952in}}{\pgfqpoint{2.240156in}{2.775952in}}%
\pgfpathclose%
\pgfusepath{stroke,fill}%
\end{pgfscope}%
\begin{pgfscope}%
\pgfpathrectangle{\pgfqpoint{0.511823in}{0.504323in}}{\pgfqpoint{3.218177in}{3.225677in}} %
\pgfusepath{clip}%
\pgfsetbuttcap%
\pgfsetroundjoin%
\definecolor{currentfill}{rgb}{0.501961,0.000000,0.000000}%
\pgfsetfillcolor{currentfill}%
\pgfsetfillopacity{0.400000}%
\pgfsetlinewidth{0.501875pt}%
\definecolor{currentstroke}{rgb}{0.501961,0.000000,0.000000}%
\pgfsetstrokecolor{currentstroke}%
\pgfsetstrokeopacity{0.400000}%
\pgfsetdash{}{0pt}%
\pgfpathmoveto{\pgfqpoint{2.266336in}{2.826747in}}%
\pgfpathcurveto{\pgfqpoint{2.277387in}{2.826747in}}{\pgfqpoint{2.287986in}{2.831137in}}{\pgfqpoint{2.295799in}{2.838951in}}%
\pgfpathcurveto{\pgfqpoint{2.303613in}{2.846764in}}{\pgfqpoint{2.308003in}{2.857363in}}{\pgfqpoint{2.308003in}{2.868413in}}%
\pgfpathcurveto{\pgfqpoint{2.308003in}{2.879463in}}{\pgfqpoint{2.303613in}{2.890063in}}{\pgfqpoint{2.295799in}{2.897876in}}%
\pgfpathcurveto{\pgfqpoint{2.287986in}{2.905690in}}{\pgfqpoint{2.277387in}{2.910080in}}{\pgfqpoint{2.266336in}{2.910080in}}%
\pgfpathcurveto{\pgfqpoint{2.255286in}{2.910080in}}{\pgfqpoint{2.244687in}{2.905690in}}{\pgfqpoint{2.236874in}{2.897876in}}%
\pgfpathcurveto{\pgfqpoint{2.229060in}{2.890063in}}{\pgfqpoint{2.224670in}{2.879463in}}{\pgfqpoint{2.224670in}{2.868413in}}%
\pgfpathcurveto{\pgfqpoint{2.224670in}{2.857363in}}{\pgfqpoint{2.229060in}{2.846764in}}{\pgfqpoint{2.236874in}{2.838951in}}%
\pgfpathcurveto{\pgfqpoint{2.244687in}{2.831137in}}{\pgfqpoint{2.255286in}{2.826747in}}{\pgfqpoint{2.266336in}{2.826747in}}%
\pgfpathclose%
\pgfusepath{stroke,fill}%
\end{pgfscope}%
\begin{pgfscope}%
\pgfpathrectangle{\pgfqpoint{0.511823in}{0.504323in}}{\pgfqpoint{3.218177in}{3.225677in}} %
\pgfusepath{clip}%
\pgfsetbuttcap%
\pgfsetroundjoin%
\definecolor{currentfill}{rgb}{0.501961,0.000000,0.000000}%
\pgfsetfillcolor{currentfill}%
\pgfsetfillopacity{0.400000}%
\pgfsetlinewidth{0.501875pt}%
\definecolor{currentstroke}{rgb}{0.501961,0.000000,0.000000}%
\pgfsetstrokecolor{currentstroke}%
\pgfsetstrokeopacity{0.400000}%
\pgfsetdash{}{0pt}%
\pgfpathmoveto{\pgfqpoint{2.392059in}{3.016907in}}%
\pgfpathcurveto{\pgfqpoint{2.403109in}{3.016907in}}{\pgfqpoint{2.413708in}{3.021298in}}{\pgfqpoint{2.421522in}{3.029111in}}%
\pgfpathcurveto{\pgfqpoint{2.429335in}{3.036925in}}{\pgfqpoint{2.433725in}{3.047524in}}{\pgfqpoint{2.433725in}{3.058574in}}%
\pgfpathcurveto{\pgfqpoint{2.433725in}{3.069624in}}{\pgfqpoint{2.429335in}{3.080223in}}{\pgfqpoint{2.421522in}{3.088037in}}%
\pgfpathcurveto{\pgfqpoint{2.413708in}{3.095850in}}{\pgfqpoint{2.403109in}{3.100241in}}{\pgfqpoint{2.392059in}{3.100241in}}%
\pgfpathcurveto{\pgfqpoint{2.381009in}{3.100241in}}{\pgfqpoint{2.370410in}{3.095850in}}{\pgfqpoint{2.362596in}{3.088037in}}%
\pgfpathcurveto{\pgfqpoint{2.354782in}{3.080223in}}{\pgfqpoint{2.350392in}{3.069624in}}{\pgfqpoint{2.350392in}{3.058574in}}%
\pgfpathcurveto{\pgfqpoint{2.350392in}{3.047524in}}{\pgfqpoint{2.354782in}{3.036925in}}{\pgfqpoint{2.362596in}{3.029111in}}%
\pgfpathcurveto{\pgfqpoint{2.370410in}{3.021298in}}{\pgfqpoint{2.381009in}{3.016907in}}{\pgfqpoint{2.392059in}{3.016907in}}%
\pgfpathclose%
\pgfusepath{stroke,fill}%
\end{pgfscope}%
\begin{pgfscope}%
\pgfpathrectangle{\pgfqpoint{0.511823in}{0.504323in}}{\pgfqpoint{3.218177in}{3.225677in}} %
\pgfusepath{clip}%
\pgfsetbuttcap%
\pgfsetroundjoin%
\definecolor{currentfill}{rgb}{0.501961,0.000000,0.000000}%
\pgfsetfillcolor{currentfill}%
\pgfsetfillopacity{0.400000}%
\pgfsetlinewidth{0.501875pt}%
\definecolor{currentstroke}{rgb}{0.501961,0.000000,0.000000}%
\pgfsetstrokecolor{currentstroke}%
\pgfsetstrokeopacity{0.400000}%
\pgfsetdash{}{0pt}%
\pgfpathmoveto{\pgfqpoint{2.210682in}{2.778669in}}%
\pgfpathcurveto{\pgfqpoint{2.221732in}{2.778669in}}{\pgfqpoint{2.232331in}{2.783059in}}{\pgfqpoint{2.240144in}{2.790873in}}%
\pgfpathcurveto{\pgfqpoint{2.247958in}{2.798687in}}{\pgfqpoint{2.252348in}{2.809286in}}{\pgfqpoint{2.252348in}{2.820336in}}%
\pgfpathcurveto{\pgfqpoint{2.252348in}{2.831386in}}{\pgfqpoint{2.247958in}{2.841985in}}{\pgfqpoint{2.240144in}{2.849798in}}%
\pgfpathcurveto{\pgfqpoint{2.232331in}{2.857612in}}{\pgfqpoint{2.221732in}{2.862002in}}{\pgfqpoint{2.210682in}{2.862002in}}%
\pgfpathcurveto{\pgfqpoint{2.199631in}{2.862002in}}{\pgfqpoint{2.189032in}{2.857612in}}{\pgfqpoint{2.181219in}{2.849798in}}%
\pgfpathcurveto{\pgfqpoint{2.173405in}{2.841985in}}{\pgfqpoint{2.169015in}{2.831386in}}{\pgfqpoint{2.169015in}{2.820336in}}%
\pgfpathcurveto{\pgfqpoint{2.169015in}{2.809286in}}{\pgfqpoint{2.173405in}{2.798687in}}{\pgfqpoint{2.181219in}{2.790873in}}%
\pgfpathcurveto{\pgfqpoint{2.189032in}{2.783059in}}{\pgfqpoint{2.199631in}{2.778669in}}{\pgfqpoint{2.210682in}{2.778669in}}%
\pgfpathclose%
\pgfusepath{stroke,fill}%
\end{pgfscope}%
\begin{pgfscope}%
\pgfpathrectangle{\pgfqpoint{0.511823in}{0.504323in}}{\pgfqpoint{3.218177in}{3.225677in}} %
\pgfusepath{clip}%
\pgfsetbuttcap%
\pgfsetroundjoin%
\definecolor{currentfill}{rgb}{0.501961,0.000000,0.000000}%
\pgfsetfillcolor{currentfill}%
\pgfsetfillopacity{0.400000}%
\pgfsetlinewidth{0.501875pt}%
\definecolor{currentstroke}{rgb}{0.501961,0.000000,0.000000}%
\pgfsetstrokecolor{currentstroke}%
\pgfsetstrokeopacity{0.400000}%
\pgfsetdash{}{0pt}%
\pgfpathmoveto{\pgfqpoint{2.202116in}{2.781256in}}%
\pgfpathcurveto{\pgfqpoint{2.213166in}{2.781256in}}{\pgfqpoint{2.223765in}{2.785646in}}{\pgfqpoint{2.231579in}{2.793460in}}%
\pgfpathcurveto{\pgfqpoint{2.239393in}{2.801273in}}{\pgfqpoint{2.243783in}{2.811872in}}{\pgfqpoint{2.243783in}{2.822923in}}%
\pgfpathcurveto{\pgfqpoint{2.243783in}{2.833973in}}{\pgfqpoint{2.239393in}{2.844572in}}{\pgfqpoint{2.231579in}{2.852385in}}%
\pgfpathcurveto{\pgfqpoint{2.223765in}{2.860199in}}{\pgfqpoint{2.213166in}{2.864589in}}{\pgfqpoint{2.202116in}{2.864589in}}%
\pgfpathcurveto{\pgfqpoint{2.191066in}{2.864589in}}{\pgfqpoint{2.180467in}{2.860199in}}{\pgfqpoint{2.172654in}{2.852385in}}%
\pgfpathcurveto{\pgfqpoint{2.164840in}{2.844572in}}{\pgfqpoint{2.160450in}{2.833973in}}{\pgfqpoint{2.160450in}{2.822923in}}%
\pgfpathcurveto{\pgfqpoint{2.160450in}{2.811872in}}{\pgfqpoint{2.164840in}{2.801273in}}{\pgfqpoint{2.172654in}{2.793460in}}%
\pgfpathcurveto{\pgfqpoint{2.180467in}{2.785646in}}{\pgfqpoint{2.191066in}{2.781256in}}{\pgfqpoint{2.202116in}{2.781256in}}%
\pgfpathclose%
\pgfusepath{stroke,fill}%
\end{pgfscope}%
\begin{pgfscope}%
\pgfpathrectangle{\pgfqpoint{0.511823in}{0.504323in}}{\pgfqpoint{3.218177in}{3.225677in}} %
\pgfusepath{clip}%
\pgfsetbuttcap%
\pgfsetroundjoin%
\definecolor{currentfill}{rgb}{0.501961,0.000000,0.000000}%
\pgfsetfillcolor{currentfill}%
\pgfsetfillopacity{0.400000}%
\pgfsetlinewidth{0.501875pt}%
\definecolor{currentstroke}{rgb}{0.501961,0.000000,0.000000}%
\pgfsetstrokecolor{currentstroke}%
\pgfsetstrokeopacity{0.400000}%
\pgfsetdash{}{0pt}%
\pgfpathmoveto{\pgfqpoint{2.082419in}{2.625718in}}%
\pgfpathcurveto{\pgfqpoint{2.093469in}{2.625718in}}{\pgfqpoint{2.104068in}{2.630108in}}{\pgfqpoint{2.111882in}{2.637922in}}%
\pgfpathcurveto{\pgfqpoint{2.119695in}{2.645735in}}{\pgfqpoint{2.124086in}{2.656334in}}{\pgfqpoint{2.124086in}{2.667384in}}%
\pgfpathcurveto{\pgfqpoint{2.124086in}{2.678434in}}{\pgfqpoint{2.119695in}{2.689034in}}{\pgfqpoint{2.111882in}{2.696847in}}%
\pgfpathcurveto{\pgfqpoint{2.104068in}{2.704661in}}{\pgfqpoint{2.093469in}{2.709051in}}{\pgfqpoint{2.082419in}{2.709051in}}%
\pgfpathcurveto{\pgfqpoint{2.071369in}{2.709051in}}{\pgfqpoint{2.060770in}{2.704661in}}{\pgfqpoint{2.052956in}{2.696847in}}%
\pgfpathcurveto{\pgfqpoint{2.045142in}{2.689034in}}{\pgfqpoint{2.040752in}{2.678434in}}{\pgfqpoint{2.040752in}{2.667384in}}%
\pgfpathcurveto{\pgfqpoint{2.040752in}{2.656334in}}{\pgfqpoint{2.045142in}{2.645735in}}{\pgfqpoint{2.052956in}{2.637922in}}%
\pgfpathcurveto{\pgfqpoint{2.060770in}{2.630108in}}{\pgfqpoint{2.071369in}{2.625718in}}{\pgfqpoint{2.082419in}{2.625718in}}%
\pgfpathclose%
\pgfusepath{stroke,fill}%
\end{pgfscope}%
\begin{pgfscope}%
\pgfpathrectangle{\pgfqpoint{0.511823in}{0.504323in}}{\pgfqpoint{3.218177in}{3.225677in}} %
\pgfusepath{clip}%
\pgfsetbuttcap%
\pgfsetroundjoin%
\definecolor{currentfill}{rgb}{0.501961,0.000000,0.000000}%
\pgfsetfillcolor{currentfill}%
\pgfsetfillopacity{0.400000}%
\pgfsetlinewidth{0.501875pt}%
\definecolor{currentstroke}{rgb}{0.501961,0.000000,0.000000}%
\pgfsetstrokecolor{currentstroke}%
\pgfsetstrokeopacity{0.400000}%
\pgfsetdash{}{0pt}%
\pgfpathmoveto{\pgfqpoint{2.039758in}{2.578321in}}%
\pgfpathcurveto{\pgfqpoint{2.050808in}{2.578321in}}{\pgfqpoint{2.061407in}{2.582711in}}{\pgfqpoint{2.069220in}{2.590525in}}%
\pgfpathcurveto{\pgfqpoint{2.077034in}{2.598338in}}{\pgfqpoint{2.081424in}{2.608937in}}{\pgfqpoint{2.081424in}{2.619988in}}%
\pgfpathcurveto{\pgfqpoint{2.081424in}{2.631038in}}{\pgfqpoint{2.077034in}{2.641637in}}{\pgfqpoint{2.069220in}{2.649450in}}%
\pgfpathcurveto{\pgfqpoint{2.061407in}{2.657264in}}{\pgfqpoint{2.050808in}{2.661654in}}{\pgfqpoint{2.039758in}{2.661654in}}%
\pgfpathcurveto{\pgfqpoint{2.028708in}{2.661654in}}{\pgfqpoint{2.018109in}{2.657264in}}{\pgfqpoint{2.010295in}{2.649450in}}%
\pgfpathcurveto{\pgfqpoint{2.002481in}{2.641637in}}{\pgfqpoint{1.998091in}{2.631038in}}{\pgfqpoint{1.998091in}{2.619988in}}%
\pgfpathcurveto{\pgfqpoint{1.998091in}{2.608937in}}{\pgfqpoint{2.002481in}{2.598338in}}{\pgfqpoint{2.010295in}{2.590525in}}%
\pgfpathcurveto{\pgfqpoint{2.018109in}{2.582711in}}{\pgfqpoint{2.028708in}{2.578321in}}{\pgfqpoint{2.039758in}{2.578321in}}%
\pgfpathclose%
\pgfusepath{stroke,fill}%
\end{pgfscope}%
\begin{pgfscope}%
\pgfpathrectangle{\pgfqpoint{0.511823in}{0.504323in}}{\pgfqpoint{3.218177in}{3.225677in}} %
\pgfusepath{clip}%
\pgfsetbuttcap%
\pgfsetroundjoin%
\definecolor{currentfill}{rgb}{0.501961,0.000000,0.000000}%
\pgfsetfillcolor{currentfill}%
\pgfsetfillopacity{0.400000}%
\pgfsetlinewidth{0.501875pt}%
\definecolor{currentstroke}{rgb}{0.501961,0.000000,0.000000}%
\pgfsetstrokecolor{currentstroke}%
\pgfsetstrokeopacity{0.400000}%
\pgfsetdash{}{0pt}%
\pgfpathmoveto{\pgfqpoint{1.964944in}{2.483835in}}%
\pgfpathcurveto{\pgfqpoint{1.975994in}{2.483835in}}{\pgfqpoint{1.986593in}{2.488225in}}{\pgfqpoint{1.994406in}{2.496039in}}%
\pgfpathcurveto{\pgfqpoint{2.002220in}{2.503852in}}{\pgfqpoint{2.006610in}{2.514452in}}{\pgfqpoint{2.006610in}{2.525502in}}%
\pgfpathcurveto{\pgfqpoint{2.006610in}{2.536552in}}{\pgfqpoint{2.002220in}{2.547151in}}{\pgfqpoint{1.994406in}{2.554964in}}%
\pgfpathcurveto{\pgfqpoint{1.986593in}{2.562778in}}{\pgfqpoint{1.975994in}{2.567168in}}{\pgfqpoint{1.964944in}{2.567168in}}%
\pgfpathcurveto{\pgfqpoint{1.953893in}{2.567168in}}{\pgfqpoint{1.943294in}{2.562778in}}{\pgfqpoint{1.935481in}{2.554964in}}%
\pgfpathcurveto{\pgfqpoint{1.927667in}{2.547151in}}{\pgfqpoint{1.923277in}{2.536552in}}{\pgfqpoint{1.923277in}{2.525502in}}%
\pgfpathcurveto{\pgfqpoint{1.923277in}{2.514452in}}{\pgfqpoint{1.927667in}{2.503852in}}{\pgfqpoint{1.935481in}{2.496039in}}%
\pgfpathcurveto{\pgfqpoint{1.943294in}{2.488225in}}{\pgfqpoint{1.953893in}{2.483835in}}{\pgfqpoint{1.964944in}{2.483835in}}%
\pgfpathclose%
\pgfusepath{stroke,fill}%
\end{pgfscope}%
\begin{pgfscope}%
\pgfpathrectangle{\pgfqpoint{0.511823in}{0.504323in}}{\pgfqpoint{3.218177in}{3.225677in}} %
\pgfusepath{clip}%
\pgfsetbuttcap%
\pgfsetroundjoin%
\definecolor{currentfill}{rgb}{0.501961,0.000000,0.000000}%
\pgfsetfillcolor{currentfill}%
\pgfsetfillopacity{0.400000}%
\pgfsetlinewidth{0.501875pt}%
\definecolor{currentstroke}{rgb}{0.501961,0.000000,0.000000}%
\pgfsetstrokecolor{currentstroke}%
\pgfsetstrokeopacity{0.400000}%
\pgfsetdash{}{0pt}%
\pgfpathmoveto{\pgfqpoint{2.155761in}{2.773782in}}%
\pgfpathcurveto{\pgfqpoint{2.166812in}{2.773782in}}{\pgfqpoint{2.177411in}{2.778172in}}{\pgfqpoint{2.185224in}{2.785986in}}%
\pgfpathcurveto{\pgfqpoint{2.193038in}{2.793799in}}{\pgfqpoint{2.197428in}{2.804398in}}{\pgfqpoint{2.197428in}{2.815448in}}%
\pgfpathcurveto{\pgfqpoint{2.197428in}{2.826498in}}{\pgfqpoint{2.193038in}{2.837098in}}{\pgfqpoint{2.185224in}{2.844911in}}%
\pgfpathcurveto{\pgfqpoint{2.177411in}{2.852725in}}{\pgfqpoint{2.166812in}{2.857115in}}{\pgfqpoint{2.155761in}{2.857115in}}%
\pgfpathcurveto{\pgfqpoint{2.144711in}{2.857115in}}{\pgfqpoint{2.134112in}{2.852725in}}{\pgfqpoint{2.126299in}{2.844911in}}%
\pgfpathcurveto{\pgfqpoint{2.118485in}{2.837098in}}{\pgfqpoint{2.114095in}{2.826498in}}{\pgfqpoint{2.114095in}{2.815448in}}%
\pgfpathcurveto{\pgfqpoint{2.114095in}{2.804398in}}{\pgfqpoint{2.118485in}{2.793799in}}{\pgfqpoint{2.126299in}{2.785986in}}%
\pgfpathcurveto{\pgfqpoint{2.134112in}{2.778172in}}{\pgfqpoint{2.144711in}{2.773782in}}{\pgfqpoint{2.155761in}{2.773782in}}%
\pgfpathclose%
\pgfusepath{stroke,fill}%
\end{pgfscope}%
\begin{pgfscope}%
\pgfpathrectangle{\pgfqpoint{0.511823in}{0.504323in}}{\pgfqpoint{3.218177in}{3.225677in}} %
\pgfusepath{clip}%
\pgfsetbuttcap%
\pgfsetroundjoin%
\definecolor{currentfill}{rgb}{0.501961,0.000000,0.000000}%
\pgfsetfillcolor{currentfill}%
\pgfsetfillopacity{0.400000}%
\pgfsetlinewidth{0.501875pt}%
\definecolor{currentstroke}{rgb}{0.501961,0.000000,0.000000}%
\pgfsetstrokecolor{currentstroke}%
\pgfsetstrokeopacity{0.400000}%
\pgfsetdash{}{0pt}%
\pgfpathmoveto{\pgfqpoint{2.259886in}{2.940926in}}%
\pgfpathcurveto{\pgfqpoint{2.270936in}{2.940926in}}{\pgfqpoint{2.281535in}{2.945316in}}{\pgfqpoint{2.289348in}{2.953130in}}%
\pgfpathcurveto{\pgfqpoint{2.297162in}{2.960943in}}{\pgfqpoint{2.301552in}{2.971542in}}{\pgfqpoint{2.301552in}{2.982593in}}%
\pgfpathcurveto{\pgfqpoint{2.301552in}{2.993643in}}{\pgfqpoint{2.297162in}{3.004242in}}{\pgfqpoint{2.289348in}{3.012055in}}%
\pgfpathcurveto{\pgfqpoint{2.281535in}{3.019869in}}{\pgfqpoint{2.270936in}{3.024259in}}{\pgfqpoint{2.259886in}{3.024259in}}%
\pgfpathcurveto{\pgfqpoint{2.248836in}{3.024259in}}{\pgfqpoint{2.238237in}{3.019869in}}{\pgfqpoint{2.230423in}{3.012055in}}%
\pgfpathcurveto{\pgfqpoint{2.222609in}{3.004242in}}{\pgfqpoint{2.218219in}{2.993643in}}{\pgfqpoint{2.218219in}{2.982593in}}%
\pgfpathcurveto{\pgfqpoint{2.218219in}{2.971542in}}{\pgfqpoint{2.222609in}{2.960943in}}{\pgfqpoint{2.230423in}{2.953130in}}%
\pgfpathcurveto{\pgfqpoint{2.238237in}{2.945316in}}{\pgfqpoint{2.248836in}{2.940926in}}{\pgfqpoint{2.259886in}{2.940926in}}%
\pgfpathclose%
\pgfusepath{stroke,fill}%
\end{pgfscope}%
\begin{pgfscope}%
\pgfpathrectangle{\pgfqpoint{0.511823in}{0.504323in}}{\pgfqpoint{3.218177in}{3.225677in}} %
\pgfusepath{clip}%
\pgfsetbuttcap%
\pgfsetroundjoin%
\definecolor{currentfill}{rgb}{0.501961,0.000000,0.000000}%
\pgfsetfillcolor{currentfill}%
\pgfsetfillopacity{0.400000}%
\pgfsetlinewidth{0.501875pt}%
\definecolor{currentstroke}{rgb}{0.501961,0.000000,0.000000}%
\pgfsetstrokecolor{currentstroke}%
\pgfsetstrokeopacity{0.400000}%
\pgfsetdash{}{0pt}%
\pgfpathmoveto{\pgfqpoint{2.101518in}{2.723769in}}%
\pgfpathcurveto{\pgfqpoint{2.112568in}{2.723769in}}{\pgfqpoint{2.123167in}{2.728160in}}{\pgfqpoint{2.130981in}{2.735973in}}%
\pgfpathcurveto{\pgfqpoint{2.138794in}{2.743787in}}{\pgfqpoint{2.143185in}{2.754386in}}{\pgfqpoint{2.143185in}{2.765436in}}%
\pgfpathcurveto{\pgfqpoint{2.143185in}{2.776486in}}{\pgfqpoint{2.138794in}{2.787085in}}{\pgfqpoint{2.130981in}{2.794899in}}%
\pgfpathcurveto{\pgfqpoint{2.123167in}{2.802712in}}{\pgfqpoint{2.112568in}{2.807103in}}{\pgfqpoint{2.101518in}{2.807103in}}%
\pgfpathcurveto{\pgfqpoint{2.090468in}{2.807103in}}{\pgfqpoint{2.079869in}{2.802712in}}{\pgfqpoint{2.072055in}{2.794899in}}%
\pgfpathcurveto{\pgfqpoint{2.064241in}{2.787085in}}{\pgfqpoint{2.059851in}{2.776486in}}{\pgfqpoint{2.059851in}{2.765436in}}%
\pgfpathcurveto{\pgfqpoint{2.059851in}{2.754386in}}{\pgfqpoint{2.064241in}{2.743787in}}{\pgfqpoint{2.072055in}{2.735973in}}%
\pgfpathcurveto{\pgfqpoint{2.079869in}{2.728160in}}{\pgfqpoint{2.090468in}{2.723769in}}{\pgfqpoint{2.101518in}{2.723769in}}%
\pgfpathclose%
\pgfusepath{stroke,fill}%
\end{pgfscope}%
\begin{pgfscope}%
\pgfpathrectangle{\pgfqpoint{0.511823in}{0.504323in}}{\pgfqpoint{3.218177in}{3.225677in}} %
\pgfusepath{clip}%
\pgfsetbuttcap%
\pgfsetroundjoin%
\definecolor{currentfill}{rgb}{0.501961,0.000000,0.000000}%
\pgfsetfillcolor{currentfill}%
\pgfsetfillopacity{0.400000}%
\pgfsetlinewidth{0.501875pt}%
\definecolor{currentstroke}{rgb}{0.501961,0.000000,0.000000}%
\pgfsetstrokecolor{currentstroke}%
\pgfsetstrokeopacity{0.400000}%
\pgfsetdash{}{0pt}%
\pgfpathmoveto{\pgfqpoint{1.925996in}{2.478066in}}%
\pgfpathcurveto{\pgfqpoint{1.937046in}{2.478066in}}{\pgfqpoint{1.947645in}{2.482456in}}{\pgfqpoint{1.955459in}{2.490269in}}%
\pgfpathcurveto{\pgfqpoint{1.963272in}{2.498083in}}{\pgfqpoint{1.967663in}{2.508682in}}{\pgfqpoint{1.967663in}{2.519732in}}%
\pgfpathcurveto{\pgfqpoint{1.967663in}{2.530782in}}{\pgfqpoint{1.963272in}{2.541381in}}{\pgfqpoint{1.955459in}{2.549195in}}%
\pgfpathcurveto{\pgfqpoint{1.947645in}{2.557009in}}{\pgfqpoint{1.937046in}{2.561399in}}{\pgfqpoint{1.925996in}{2.561399in}}%
\pgfpathcurveto{\pgfqpoint{1.914946in}{2.561399in}}{\pgfqpoint{1.904347in}{2.557009in}}{\pgfqpoint{1.896533in}{2.549195in}}%
\pgfpathcurveto{\pgfqpoint{1.888720in}{2.541381in}}{\pgfqpoint{1.884329in}{2.530782in}}{\pgfqpoint{1.884329in}{2.519732in}}%
\pgfpathcurveto{\pgfqpoint{1.884329in}{2.508682in}}{\pgfqpoint{1.888720in}{2.498083in}}{\pgfqpoint{1.896533in}{2.490269in}}%
\pgfpathcurveto{\pgfqpoint{1.904347in}{2.482456in}}{\pgfqpoint{1.914946in}{2.478066in}}{\pgfqpoint{1.925996in}{2.478066in}}%
\pgfpathclose%
\pgfusepath{stroke,fill}%
\end{pgfscope}%
\begin{pgfscope}%
\pgfpathrectangle{\pgfqpoint{0.511823in}{0.504323in}}{\pgfqpoint{3.218177in}{3.225677in}} %
\pgfusepath{clip}%
\pgfsetbuttcap%
\pgfsetroundjoin%
\definecolor{currentfill}{rgb}{0.501961,0.000000,0.000000}%
\pgfsetfillcolor{currentfill}%
\pgfsetfillopacity{0.400000}%
\pgfsetlinewidth{0.501875pt}%
\definecolor{currentstroke}{rgb}{0.501961,0.000000,0.000000}%
\pgfsetstrokecolor{currentstroke}%
\pgfsetstrokeopacity{0.400000}%
\pgfsetdash{}{0pt}%
\pgfpathmoveto{\pgfqpoint{2.107845in}{2.762568in}}%
\pgfpathcurveto{\pgfqpoint{2.118895in}{2.762568in}}{\pgfqpoint{2.129494in}{2.766958in}}{\pgfqpoint{2.137308in}{2.774772in}}%
\pgfpathcurveto{\pgfqpoint{2.145121in}{2.782585in}}{\pgfqpoint{2.149512in}{2.793184in}}{\pgfqpoint{2.149512in}{2.804234in}}%
\pgfpathcurveto{\pgfqpoint{2.149512in}{2.815285in}}{\pgfqpoint{2.145121in}{2.825884in}}{\pgfqpoint{2.137308in}{2.833697in}}%
\pgfpathcurveto{\pgfqpoint{2.129494in}{2.841511in}}{\pgfqpoint{2.118895in}{2.845901in}}{\pgfqpoint{2.107845in}{2.845901in}}%
\pgfpathcurveto{\pgfqpoint{2.096795in}{2.845901in}}{\pgfqpoint{2.086196in}{2.841511in}}{\pgfqpoint{2.078382in}{2.833697in}}%
\pgfpathcurveto{\pgfqpoint{2.070569in}{2.825884in}}{\pgfqpoint{2.066178in}{2.815285in}}{\pgfqpoint{2.066178in}{2.804234in}}%
\pgfpathcurveto{\pgfqpoint{2.066178in}{2.793184in}}{\pgfqpoint{2.070569in}{2.782585in}}{\pgfqpoint{2.078382in}{2.774772in}}%
\pgfpathcurveto{\pgfqpoint{2.086196in}{2.766958in}}{\pgfqpoint{2.096795in}{2.762568in}}{\pgfqpoint{2.107845in}{2.762568in}}%
\pgfpathclose%
\pgfusepath{stroke,fill}%
\end{pgfscope}%
\begin{pgfscope}%
\pgfpathrectangle{\pgfqpoint{0.511823in}{0.504323in}}{\pgfqpoint{3.218177in}{3.225677in}} %
\pgfusepath{clip}%
\pgfsetbuttcap%
\pgfsetroundjoin%
\definecolor{currentfill}{rgb}{0.501961,0.000000,0.000000}%
\pgfsetfillcolor{currentfill}%
\pgfsetfillopacity{0.400000}%
\pgfsetlinewidth{0.501875pt}%
\definecolor{currentstroke}{rgb}{0.501961,0.000000,0.000000}%
\pgfsetstrokecolor{currentstroke}%
\pgfsetstrokeopacity{0.400000}%
\pgfsetdash{}{0pt}%
\pgfpathmoveto{\pgfqpoint{2.072754in}{2.724755in}}%
\pgfpathcurveto{\pgfqpoint{2.083804in}{2.724755in}}{\pgfqpoint{2.094403in}{2.729146in}}{\pgfqpoint{2.102216in}{2.736959in}}%
\pgfpathcurveto{\pgfqpoint{2.110030in}{2.744773in}}{\pgfqpoint{2.114420in}{2.755372in}}{\pgfqpoint{2.114420in}{2.766422in}}%
\pgfpathcurveto{\pgfqpoint{2.114420in}{2.777472in}}{\pgfqpoint{2.110030in}{2.788071in}}{\pgfqpoint{2.102216in}{2.795885in}}%
\pgfpathcurveto{\pgfqpoint{2.094403in}{2.803698in}}{\pgfqpoint{2.083804in}{2.808089in}}{\pgfqpoint{2.072754in}{2.808089in}}%
\pgfpathcurveto{\pgfqpoint{2.061703in}{2.808089in}}{\pgfqpoint{2.051104in}{2.803698in}}{\pgfqpoint{2.043291in}{2.795885in}}%
\pgfpathcurveto{\pgfqpoint{2.035477in}{2.788071in}}{\pgfqpoint{2.031087in}{2.777472in}}{\pgfqpoint{2.031087in}{2.766422in}}%
\pgfpathcurveto{\pgfqpoint{2.031087in}{2.755372in}}{\pgfqpoint{2.035477in}{2.744773in}}{\pgfqpoint{2.043291in}{2.736959in}}%
\pgfpathcurveto{\pgfqpoint{2.051104in}{2.729146in}}{\pgfqpoint{2.061703in}{2.724755in}}{\pgfqpoint{2.072754in}{2.724755in}}%
\pgfpathclose%
\pgfusepath{stroke,fill}%
\end{pgfscope}%
\begin{pgfscope}%
\pgfpathrectangle{\pgfqpoint{0.511823in}{0.504323in}}{\pgfqpoint{3.218177in}{3.225677in}} %
\pgfusepath{clip}%
\pgfsetbuttcap%
\pgfsetroundjoin%
\definecolor{currentfill}{rgb}{0.501961,0.000000,0.000000}%
\pgfsetfillcolor{currentfill}%
\pgfsetfillopacity{0.400000}%
\pgfsetlinewidth{0.501875pt}%
\definecolor{currentstroke}{rgb}{0.501961,0.000000,0.000000}%
\pgfsetstrokecolor{currentstroke}%
\pgfsetstrokeopacity{0.400000}%
\pgfsetdash{}{0pt}%
\pgfpathmoveto{\pgfqpoint{2.054111in}{2.711264in}}%
\pgfpathcurveto{\pgfqpoint{2.065161in}{2.711264in}}{\pgfqpoint{2.075760in}{2.715654in}}{\pgfqpoint{2.083574in}{2.723467in}}%
\pgfpathcurveto{\pgfqpoint{2.091387in}{2.731281in}}{\pgfqpoint{2.095778in}{2.741880in}}{\pgfqpoint{2.095778in}{2.752930in}}%
\pgfpathcurveto{\pgfqpoint{2.095778in}{2.763980in}}{\pgfqpoint{2.091387in}{2.774579in}}{\pgfqpoint{2.083574in}{2.782393in}}%
\pgfpathcurveto{\pgfqpoint{2.075760in}{2.790207in}}{\pgfqpoint{2.065161in}{2.794597in}}{\pgfqpoint{2.054111in}{2.794597in}}%
\pgfpathcurveto{\pgfqpoint{2.043061in}{2.794597in}}{\pgfqpoint{2.032462in}{2.790207in}}{\pgfqpoint{2.024648in}{2.782393in}}%
\pgfpathcurveto{\pgfqpoint{2.016835in}{2.774579in}}{\pgfqpoint{2.012444in}{2.763980in}}{\pgfqpoint{2.012444in}{2.752930in}}%
\pgfpathcurveto{\pgfqpoint{2.012444in}{2.741880in}}{\pgfqpoint{2.016835in}{2.731281in}}{\pgfqpoint{2.024648in}{2.723467in}}%
\pgfpathcurveto{\pgfqpoint{2.032462in}{2.715654in}}{\pgfqpoint{2.043061in}{2.711264in}}{\pgfqpoint{2.054111in}{2.711264in}}%
\pgfpathclose%
\pgfusepath{stroke,fill}%
\end{pgfscope}%
\begin{pgfscope}%
\pgfpathrectangle{\pgfqpoint{0.511823in}{0.504323in}}{\pgfqpoint{3.218177in}{3.225677in}} %
\pgfusepath{clip}%
\pgfsetbuttcap%
\pgfsetroundjoin%
\definecolor{currentfill}{rgb}{0.501961,0.000000,0.000000}%
\pgfsetfillcolor{currentfill}%
\pgfsetfillopacity{0.400000}%
\pgfsetlinewidth{0.501875pt}%
\definecolor{currentstroke}{rgb}{0.501961,0.000000,0.000000}%
\pgfsetstrokecolor{currentstroke}%
\pgfsetstrokeopacity{0.400000}%
\pgfsetdash{}{0pt}%
\pgfpathmoveto{\pgfqpoint{2.194638in}{2.940314in}}%
\pgfpathcurveto{\pgfqpoint{2.205688in}{2.940314in}}{\pgfqpoint{2.216287in}{2.944704in}}{\pgfqpoint{2.224101in}{2.952518in}}%
\pgfpathcurveto{\pgfqpoint{2.231914in}{2.960332in}}{\pgfqpoint{2.236304in}{2.970931in}}{\pgfqpoint{2.236304in}{2.981981in}}%
\pgfpathcurveto{\pgfqpoint{2.236304in}{2.993031in}}{\pgfqpoint{2.231914in}{3.003630in}}{\pgfqpoint{2.224101in}{3.011443in}}%
\pgfpathcurveto{\pgfqpoint{2.216287in}{3.019257in}}{\pgfqpoint{2.205688in}{3.023647in}}{\pgfqpoint{2.194638in}{3.023647in}}%
\pgfpathcurveto{\pgfqpoint{2.183588in}{3.023647in}}{\pgfqpoint{2.172989in}{3.019257in}}{\pgfqpoint{2.165175in}{3.011443in}}%
\pgfpathcurveto{\pgfqpoint{2.157361in}{3.003630in}}{\pgfqpoint{2.152971in}{2.993031in}}{\pgfqpoint{2.152971in}{2.981981in}}%
\pgfpathcurveto{\pgfqpoint{2.152971in}{2.970931in}}{\pgfqpoint{2.157361in}{2.960332in}}{\pgfqpoint{2.165175in}{2.952518in}}%
\pgfpathcurveto{\pgfqpoint{2.172989in}{2.944704in}}{\pgfqpoint{2.183588in}{2.940314in}}{\pgfqpoint{2.194638in}{2.940314in}}%
\pgfpathclose%
\pgfusepath{stroke,fill}%
\end{pgfscope}%
\begin{pgfscope}%
\pgfpathrectangle{\pgfqpoint{0.511823in}{0.504323in}}{\pgfqpoint{3.218177in}{3.225677in}} %
\pgfusepath{clip}%
\pgfsetbuttcap%
\pgfsetroundjoin%
\definecolor{currentfill}{rgb}{0.501961,0.000000,0.000000}%
\pgfsetfillcolor{currentfill}%
\pgfsetfillopacity{0.400000}%
\pgfsetlinewidth{0.501875pt}%
\definecolor{currentstroke}{rgb}{0.501961,0.000000,0.000000}%
\pgfsetstrokecolor{currentstroke}%
\pgfsetstrokeopacity{0.400000}%
\pgfsetdash{}{0pt}%
\pgfpathmoveto{\pgfqpoint{2.151136in}{2.889860in}}%
\pgfpathcurveto{\pgfqpoint{2.162186in}{2.889860in}}{\pgfqpoint{2.172785in}{2.894250in}}{\pgfqpoint{2.180599in}{2.902064in}}%
\pgfpathcurveto{\pgfqpoint{2.188412in}{2.909877in}}{\pgfqpoint{2.192803in}{2.920476in}}{\pgfqpoint{2.192803in}{2.931527in}}%
\pgfpathcurveto{\pgfqpoint{2.192803in}{2.942577in}}{\pgfqpoint{2.188412in}{2.953176in}}{\pgfqpoint{2.180599in}{2.960989in}}%
\pgfpathcurveto{\pgfqpoint{2.172785in}{2.968803in}}{\pgfqpoint{2.162186in}{2.973193in}}{\pgfqpoint{2.151136in}{2.973193in}}%
\pgfpathcurveto{\pgfqpoint{2.140086in}{2.973193in}}{\pgfqpoint{2.129487in}{2.968803in}}{\pgfqpoint{2.121673in}{2.960989in}}%
\pgfpathcurveto{\pgfqpoint{2.113860in}{2.953176in}}{\pgfqpoint{2.109469in}{2.942577in}}{\pgfqpoint{2.109469in}{2.931527in}}%
\pgfpathcurveto{\pgfqpoint{2.109469in}{2.920476in}}{\pgfqpoint{2.113860in}{2.909877in}}{\pgfqpoint{2.121673in}{2.902064in}}%
\pgfpathcurveto{\pgfqpoint{2.129487in}{2.894250in}}{\pgfqpoint{2.140086in}{2.889860in}}{\pgfqpoint{2.151136in}{2.889860in}}%
\pgfpathclose%
\pgfusepath{stroke,fill}%
\end{pgfscope}%
\begin{pgfscope}%
\pgfpathrectangle{\pgfqpoint{0.511823in}{0.504323in}}{\pgfqpoint{3.218177in}{3.225677in}} %
\pgfusepath{clip}%
\pgfsetbuttcap%
\pgfsetroundjoin%
\definecolor{currentfill}{rgb}{0.501961,0.000000,0.000000}%
\pgfsetfillcolor{currentfill}%
\pgfsetfillopacity{0.400000}%
\pgfsetlinewidth{0.501875pt}%
\definecolor{currentstroke}{rgb}{0.501961,0.000000,0.000000}%
\pgfsetstrokecolor{currentstroke}%
\pgfsetstrokeopacity{0.400000}%
\pgfsetdash{}{0pt}%
\pgfpathmoveto{\pgfqpoint{2.161145in}{2.921414in}}%
\pgfpathcurveto{\pgfqpoint{2.172195in}{2.921414in}}{\pgfqpoint{2.182794in}{2.925804in}}{\pgfqpoint{2.190608in}{2.933618in}}%
\pgfpathcurveto{\pgfqpoint{2.198421in}{2.941431in}}{\pgfqpoint{2.202812in}{2.952030in}}{\pgfqpoint{2.202812in}{2.963080in}}%
\pgfpathcurveto{\pgfqpoint{2.202812in}{2.974131in}}{\pgfqpoint{2.198421in}{2.984730in}}{\pgfqpoint{2.190608in}{2.992543in}}%
\pgfpathcurveto{\pgfqpoint{2.182794in}{3.000357in}}{\pgfqpoint{2.172195in}{3.004747in}}{\pgfqpoint{2.161145in}{3.004747in}}%
\pgfpathcurveto{\pgfqpoint{2.150095in}{3.004747in}}{\pgfqpoint{2.139496in}{3.000357in}}{\pgfqpoint{2.131682in}{2.992543in}}%
\pgfpathcurveto{\pgfqpoint{2.123869in}{2.984730in}}{\pgfqpoint{2.119478in}{2.974131in}}{\pgfqpoint{2.119478in}{2.963080in}}%
\pgfpathcurveto{\pgfqpoint{2.119478in}{2.952030in}}{\pgfqpoint{2.123869in}{2.941431in}}{\pgfqpoint{2.131682in}{2.933618in}}%
\pgfpathcurveto{\pgfqpoint{2.139496in}{2.925804in}}{\pgfqpoint{2.150095in}{2.921414in}}{\pgfqpoint{2.161145in}{2.921414in}}%
\pgfpathclose%
\pgfusepath{stroke,fill}%
\end{pgfscope}%
\begin{pgfscope}%
\pgfpathrectangle{\pgfqpoint{0.511823in}{0.504323in}}{\pgfqpoint{3.218177in}{3.225677in}} %
\pgfusepath{clip}%
\pgfsetbuttcap%
\pgfsetroundjoin%
\definecolor{currentfill}{rgb}{0.501961,0.000000,0.000000}%
\pgfsetfillcolor{currentfill}%
\pgfsetfillopacity{0.400000}%
\pgfsetlinewidth{0.501875pt}%
\definecolor{currentstroke}{rgb}{0.501961,0.000000,0.000000}%
\pgfsetstrokecolor{currentstroke}%
\pgfsetstrokeopacity{0.400000}%
\pgfsetdash{}{0pt}%
\pgfpathmoveto{\pgfqpoint{1.997869in}{2.683332in}}%
\pgfpathcurveto{\pgfqpoint{2.008919in}{2.683332in}}{\pgfqpoint{2.019518in}{2.687722in}}{\pgfqpoint{2.027331in}{2.695536in}}%
\pgfpathcurveto{\pgfqpoint{2.035145in}{2.703350in}}{\pgfqpoint{2.039535in}{2.713949in}}{\pgfqpoint{2.039535in}{2.724999in}}%
\pgfpathcurveto{\pgfqpoint{2.039535in}{2.736049in}}{\pgfqpoint{2.035145in}{2.746648in}}{\pgfqpoint{2.027331in}{2.754462in}}%
\pgfpathcurveto{\pgfqpoint{2.019518in}{2.762275in}}{\pgfqpoint{2.008919in}{2.766665in}}{\pgfqpoint{1.997869in}{2.766665in}}%
\pgfpathcurveto{\pgfqpoint{1.986819in}{2.766665in}}{\pgfqpoint{1.976219in}{2.762275in}}{\pgfqpoint{1.968406in}{2.754462in}}%
\pgfpathcurveto{\pgfqpoint{1.960592in}{2.746648in}}{\pgfqpoint{1.956202in}{2.736049in}}{\pgfqpoint{1.956202in}{2.724999in}}%
\pgfpathcurveto{\pgfqpoint{1.956202in}{2.713949in}}{\pgfqpoint{1.960592in}{2.703350in}}{\pgfqpoint{1.968406in}{2.695536in}}%
\pgfpathcurveto{\pgfqpoint{1.976219in}{2.687722in}}{\pgfqpoint{1.986819in}{2.683332in}}{\pgfqpoint{1.997869in}{2.683332in}}%
\pgfpathclose%
\pgfusepath{stroke,fill}%
\end{pgfscope}%
\begin{pgfscope}%
\pgfpathrectangle{\pgfqpoint{0.511823in}{0.504323in}}{\pgfqpoint{3.218177in}{3.225677in}} %
\pgfusepath{clip}%
\pgfsetbuttcap%
\pgfsetroundjoin%
\definecolor{currentfill}{rgb}{0.501961,0.000000,0.000000}%
\pgfsetfillcolor{currentfill}%
\pgfsetfillopacity{0.400000}%
\pgfsetlinewidth{0.501875pt}%
\definecolor{currentstroke}{rgb}{0.501961,0.000000,0.000000}%
\pgfsetstrokecolor{currentstroke}%
\pgfsetstrokeopacity{0.400000}%
\pgfsetdash{}{0pt}%
\pgfpathmoveto{\pgfqpoint{2.069185in}{2.809954in}}%
\pgfpathcurveto{\pgfqpoint{2.080235in}{2.809954in}}{\pgfqpoint{2.090834in}{2.814344in}}{\pgfqpoint{2.098648in}{2.822158in}}%
\pgfpathcurveto{\pgfqpoint{2.106461in}{2.829971in}}{\pgfqpoint{2.110852in}{2.840570in}}{\pgfqpoint{2.110852in}{2.851621in}}%
\pgfpathcurveto{\pgfqpoint{2.110852in}{2.862671in}}{\pgfqpoint{2.106461in}{2.873270in}}{\pgfqpoint{2.098648in}{2.881083in}}%
\pgfpathcurveto{\pgfqpoint{2.090834in}{2.888897in}}{\pgfqpoint{2.080235in}{2.893287in}}{\pgfqpoint{2.069185in}{2.893287in}}%
\pgfpathcurveto{\pgfqpoint{2.058135in}{2.893287in}}{\pgfqpoint{2.047536in}{2.888897in}}{\pgfqpoint{2.039722in}{2.881083in}}%
\pgfpathcurveto{\pgfqpoint{2.031909in}{2.873270in}}{\pgfqpoint{2.027518in}{2.862671in}}{\pgfqpoint{2.027518in}{2.851621in}}%
\pgfpathcurveto{\pgfqpoint{2.027518in}{2.840570in}}{\pgfqpoint{2.031909in}{2.829971in}}{\pgfqpoint{2.039722in}{2.822158in}}%
\pgfpathcurveto{\pgfqpoint{2.047536in}{2.814344in}}{\pgfqpoint{2.058135in}{2.809954in}}{\pgfqpoint{2.069185in}{2.809954in}}%
\pgfpathclose%
\pgfusepath{stroke,fill}%
\end{pgfscope}%
\begin{pgfscope}%
\pgfpathrectangle{\pgfqpoint{0.511823in}{0.504323in}}{\pgfqpoint{3.218177in}{3.225677in}} %
\pgfusepath{clip}%
\pgfsetbuttcap%
\pgfsetroundjoin%
\definecolor{currentfill}{rgb}{0.501961,0.000000,0.000000}%
\pgfsetfillcolor{currentfill}%
\pgfsetfillopacity{0.400000}%
\pgfsetlinewidth{0.501875pt}%
\definecolor{currentstroke}{rgb}{0.501961,0.000000,0.000000}%
\pgfsetstrokecolor{currentstroke}%
\pgfsetstrokeopacity{0.400000}%
\pgfsetdash{}{0pt}%
\pgfpathmoveto{\pgfqpoint{1.963447in}{2.658460in}}%
\pgfpathcurveto{\pgfqpoint{1.974497in}{2.658460in}}{\pgfqpoint{1.985096in}{2.662850in}}{\pgfqpoint{1.992910in}{2.670664in}}%
\pgfpathcurveto{\pgfqpoint{2.000724in}{2.678477in}}{\pgfqpoint{2.005114in}{2.689076in}}{\pgfqpoint{2.005114in}{2.700127in}}%
\pgfpathcurveto{\pgfqpoint{2.005114in}{2.711177in}}{\pgfqpoint{2.000724in}{2.721776in}}{\pgfqpoint{1.992910in}{2.729589in}}%
\pgfpathcurveto{\pgfqpoint{1.985096in}{2.737403in}}{\pgfqpoint{1.974497in}{2.741793in}}{\pgfqpoint{1.963447in}{2.741793in}}%
\pgfpathcurveto{\pgfqpoint{1.952397in}{2.741793in}}{\pgfqpoint{1.941798in}{2.737403in}}{\pgfqpoint{1.933985in}{2.729589in}}%
\pgfpathcurveto{\pgfqpoint{1.926171in}{2.721776in}}{\pgfqpoint{1.921781in}{2.711177in}}{\pgfqpoint{1.921781in}{2.700127in}}%
\pgfpathcurveto{\pgfqpoint{1.921781in}{2.689076in}}{\pgfqpoint{1.926171in}{2.678477in}}{\pgfqpoint{1.933985in}{2.670664in}}%
\pgfpathcurveto{\pgfqpoint{1.941798in}{2.662850in}}{\pgfqpoint{1.952397in}{2.658460in}}{\pgfqpoint{1.963447in}{2.658460in}}%
\pgfpathclose%
\pgfusepath{stroke,fill}%
\end{pgfscope}%
\begin{pgfscope}%
\pgfpathrectangle{\pgfqpoint{0.511823in}{0.504323in}}{\pgfqpoint{3.218177in}{3.225677in}} %
\pgfusepath{clip}%
\pgfsetbuttcap%
\pgfsetroundjoin%
\definecolor{currentfill}{rgb}{0.501961,0.000000,0.000000}%
\pgfsetfillcolor{currentfill}%
\pgfsetfillopacity{0.400000}%
\pgfsetlinewidth{0.501875pt}%
\definecolor{currentstroke}{rgb}{0.501961,0.000000,0.000000}%
\pgfsetstrokecolor{currentstroke}%
\pgfsetstrokeopacity{0.400000}%
\pgfsetdash{}{0pt}%
\pgfpathmoveto{\pgfqpoint{2.106944in}{2.901556in}}%
\pgfpathcurveto{\pgfqpoint{2.117994in}{2.901556in}}{\pgfqpoint{2.128593in}{2.905946in}}{\pgfqpoint{2.136406in}{2.913760in}}%
\pgfpathcurveto{\pgfqpoint{2.144220in}{2.921574in}}{\pgfqpoint{2.148610in}{2.932173in}}{\pgfqpoint{2.148610in}{2.943223in}}%
\pgfpathcurveto{\pgfqpoint{2.148610in}{2.954273in}}{\pgfqpoint{2.144220in}{2.964872in}}{\pgfqpoint{2.136406in}{2.972686in}}%
\pgfpathcurveto{\pgfqpoint{2.128593in}{2.980499in}}{\pgfqpoint{2.117994in}{2.984889in}}{\pgfqpoint{2.106944in}{2.984889in}}%
\pgfpathcurveto{\pgfqpoint{2.095893in}{2.984889in}}{\pgfqpoint{2.085294in}{2.980499in}}{\pgfqpoint{2.077481in}{2.972686in}}%
\pgfpathcurveto{\pgfqpoint{2.069667in}{2.964872in}}{\pgfqpoint{2.065277in}{2.954273in}}{\pgfqpoint{2.065277in}{2.943223in}}%
\pgfpathcurveto{\pgfqpoint{2.065277in}{2.932173in}}{\pgfqpoint{2.069667in}{2.921574in}}{\pgfqpoint{2.077481in}{2.913760in}}%
\pgfpathcurveto{\pgfqpoint{2.085294in}{2.905946in}}{\pgfqpoint{2.095893in}{2.901556in}}{\pgfqpoint{2.106944in}{2.901556in}}%
\pgfpathclose%
\pgfusepath{stroke,fill}%
\end{pgfscope}%
\begin{pgfscope}%
\pgfpathrectangle{\pgfqpoint{0.511823in}{0.504323in}}{\pgfqpoint{3.218177in}{3.225677in}} %
\pgfusepath{clip}%
\pgfsetbuttcap%
\pgfsetroundjoin%
\definecolor{currentfill}{rgb}{0.501961,0.000000,0.000000}%
\pgfsetfillcolor{currentfill}%
\pgfsetfillopacity{0.400000}%
\pgfsetlinewidth{0.501875pt}%
\definecolor{currentstroke}{rgb}{0.501961,0.000000,0.000000}%
\pgfsetstrokecolor{currentstroke}%
\pgfsetstrokeopacity{0.400000}%
\pgfsetdash{}{0pt}%
\pgfpathmoveto{\pgfqpoint{1.948483in}{2.663848in}}%
\pgfpathcurveto{\pgfqpoint{1.959533in}{2.663848in}}{\pgfqpoint{1.970132in}{2.668238in}}{\pgfqpoint{1.977946in}{2.676052in}}%
\pgfpathcurveto{\pgfqpoint{1.985759in}{2.683865in}}{\pgfqpoint{1.990150in}{2.694464in}}{\pgfqpoint{1.990150in}{2.705514in}}%
\pgfpathcurveto{\pgfqpoint{1.990150in}{2.716564in}}{\pgfqpoint{1.985759in}{2.727164in}}{\pgfqpoint{1.977946in}{2.734977in}}%
\pgfpathcurveto{\pgfqpoint{1.970132in}{2.742791in}}{\pgfqpoint{1.959533in}{2.747181in}}{\pgfqpoint{1.948483in}{2.747181in}}%
\pgfpathcurveto{\pgfqpoint{1.937433in}{2.747181in}}{\pgfqpoint{1.926834in}{2.742791in}}{\pgfqpoint{1.919020in}{2.734977in}}%
\pgfpathcurveto{\pgfqpoint{1.911207in}{2.727164in}}{\pgfqpoint{1.906816in}{2.716564in}}{\pgfqpoint{1.906816in}{2.705514in}}%
\pgfpathcurveto{\pgfqpoint{1.906816in}{2.694464in}}{\pgfqpoint{1.911207in}{2.683865in}}{\pgfqpoint{1.919020in}{2.676052in}}%
\pgfpathcurveto{\pgfqpoint{1.926834in}{2.668238in}}{\pgfqpoint{1.937433in}{2.663848in}}{\pgfqpoint{1.948483in}{2.663848in}}%
\pgfpathclose%
\pgfusepath{stroke,fill}%
\end{pgfscope}%
\begin{pgfscope}%
\pgfpathrectangle{\pgfqpoint{0.511823in}{0.504323in}}{\pgfqpoint{3.218177in}{3.225677in}} %
\pgfusepath{clip}%
\pgfsetbuttcap%
\pgfsetroundjoin%
\definecolor{currentfill}{rgb}{0.501961,0.000000,0.000000}%
\pgfsetfillcolor{currentfill}%
\pgfsetfillopacity{0.400000}%
\pgfsetlinewidth{0.501875pt}%
\definecolor{currentstroke}{rgb}{0.501961,0.000000,0.000000}%
\pgfsetstrokecolor{currentstroke}%
\pgfsetstrokeopacity{0.400000}%
\pgfsetdash{}{0pt}%
\pgfpathmoveto{\pgfqpoint{2.099376in}{2.922376in}}%
\pgfpathcurveto{\pgfqpoint{2.110426in}{2.922376in}}{\pgfqpoint{2.121025in}{2.926766in}}{\pgfqpoint{2.128839in}{2.934580in}}%
\pgfpathcurveto{\pgfqpoint{2.136652in}{2.942394in}}{\pgfqpoint{2.141042in}{2.952993in}}{\pgfqpoint{2.141042in}{2.964043in}}%
\pgfpathcurveto{\pgfqpoint{2.141042in}{2.975093in}}{\pgfqpoint{2.136652in}{2.985692in}}{\pgfqpoint{2.128839in}{2.993505in}}%
\pgfpathcurveto{\pgfqpoint{2.121025in}{3.001319in}}{\pgfqpoint{2.110426in}{3.005709in}}{\pgfqpoint{2.099376in}{3.005709in}}%
\pgfpathcurveto{\pgfqpoint{2.088326in}{3.005709in}}{\pgfqpoint{2.077727in}{3.001319in}}{\pgfqpoint{2.069913in}{2.993505in}}%
\pgfpathcurveto{\pgfqpoint{2.062099in}{2.985692in}}{\pgfqpoint{2.057709in}{2.975093in}}{\pgfqpoint{2.057709in}{2.964043in}}%
\pgfpathcurveto{\pgfqpoint{2.057709in}{2.952993in}}{\pgfqpoint{2.062099in}{2.942394in}}{\pgfqpoint{2.069913in}{2.934580in}}%
\pgfpathcurveto{\pgfqpoint{2.077727in}{2.926766in}}{\pgfqpoint{2.088326in}{2.922376in}}{\pgfqpoint{2.099376in}{2.922376in}}%
\pgfpathclose%
\pgfusepath{stroke,fill}%
\end{pgfscope}%
\begin{pgfscope}%
\pgfpathrectangle{\pgfqpoint{0.511823in}{0.504323in}}{\pgfqpoint{3.218177in}{3.225677in}} %
\pgfusepath{clip}%
\pgfsetbuttcap%
\pgfsetroundjoin%
\definecolor{currentfill}{rgb}{0.501961,0.000000,0.000000}%
\pgfsetfillcolor{currentfill}%
\pgfsetfillopacity{0.400000}%
\pgfsetlinewidth{0.501875pt}%
\definecolor{currentstroke}{rgb}{0.501961,0.000000,0.000000}%
\pgfsetstrokecolor{currentstroke}%
\pgfsetstrokeopacity{0.400000}%
\pgfsetdash{}{0pt}%
\pgfpathmoveto{\pgfqpoint{2.025117in}{2.818238in}}%
\pgfpathcurveto{\pgfqpoint{2.036168in}{2.818238in}}{\pgfqpoint{2.046767in}{2.822629in}}{\pgfqpoint{2.054580in}{2.830442in}}%
\pgfpathcurveto{\pgfqpoint{2.062394in}{2.838256in}}{\pgfqpoint{2.066784in}{2.848855in}}{\pgfqpoint{2.066784in}{2.859905in}}%
\pgfpathcurveto{\pgfqpoint{2.066784in}{2.870955in}}{\pgfqpoint{2.062394in}{2.881554in}}{\pgfqpoint{2.054580in}{2.889368in}}%
\pgfpathcurveto{\pgfqpoint{2.046767in}{2.897181in}}{\pgfqpoint{2.036168in}{2.901572in}}{\pgfqpoint{2.025117in}{2.901572in}}%
\pgfpathcurveto{\pgfqpoint{2.014067in}{2.901572in}}{\pgfqpoint{2.003468in}{2.897181in}}{\pgfqpoint{1.995655in}{2.889368in}}%
\pgfpathcurveto{\pgfqpoint{1.987841in}{2.881554in}}{\pgfqpoint{1.983451in}{2.870955in}}{\pgfqpoint{1.983451in}{2.859905in}}%
\pgfpathcurveto{\pgfqpoint{1.983451in}{2.848855in}}{\pgfqpoint{1.987841in}{2.838256in}}{\pgfqpoint{1.995655in}{2.830442in}}%
\pgfpathcurveto{\pgfqpoint{2.003468in}{2.822629in}}{\pgfqpoint{2.014067in}{2.818238in}}{\pgfqpoint{2.025117in}{2.818238in}}%
\pgfpathclose%
\pgfusepath{stroke,fill}%
\end{pgfscope}%
\begin{pgfscope}%
\pgfpathrectangle{\pgfqpoint{0.511823in}{0.504323in}}{\pgfqpoint{3.218177in}{3.225677in}} %
\pgfusepath{clip}%
\pgfsetbuttcap%
\pgfsetroundjoin%
\definecolor{currentfill}{rgb}{0.501961,0.000000,0.000000}%
\pgfsetfillcolor{currentfill}%
\pgfsetfillopacity{0.400000}%
\pgfsetlinewidth{0.501875pt}%
\definecolor{currentstroke}{rgb}{0.501961,0.000000,0.000000}%
\pgfsetstrokecolor{currentstroke}%
\pgfsetstrokeopacity{0.400000}%
\pgfsetdash{}{0pt}%
\pgfpathmoveto{\pgfqpoint{2.016419in}{2.819982in}}%
\pgfpathcurveto{\pgfqpoint{2.027469in}{2.819982in}}{\pgfqpoint{2.038068in}{2.824372in}}{\pgfqpoint{2.045882in}{2.832186in}}%
\pgfpathcurveto{\pgfqpoint{2.053695in}{2.839999in}}{\pgfqpoint{2.058086in}{2.850598in}}{\pgfqpoint{2.058086in}{2.861649in}}%
\pgfpathcurveto{\pgfqpoint{2.058086in}{2.872699in}}{\pgfqpoint{2.053695in}{2.883298in}}{\pgfqpoint{2.045882in}{2.891111in}}%
\pgfpathcurveto{\pgfqpoint{2.038068in}{2.898925in}}{\pgfqpoint{2.027469in}{2.903315in}}{\pgfqpoint{2.016419in}{2.903315in}}%
\pgfpathcurveto{\pgfqpoint{2.005369in}{2.903315in}}{\pgfqpoint{1.994770in}{2.898925in}}{\pgfqpoint{1.986956in}{2.891111in}}%
\pgfpathcurveto{\pgfqpoint{1.979143in}{2.883298in}}{\pgfqpoint{1.974752in}{2.872699in}}{\pgfqpoint{1.974752in}{2.861649in}}%
\pgfpathcurveto{\pgfqpoint{1.974752in}{2.850598in}}{\pgfqpoint{1.979143in}{2.839999in}}{\pgfqpoint{1.986956in}{2.832186in}}%
\pgfpathcurveto{\pgfqpoint{1.994770in}{2.824372in}}{\pgfqpoint{2.005369in}{2.819982in}}{\pgfqpoint{2.016419in}{2.819982in}}%
\pgfpathclose%
\pgfusepath{stroke,fill}%
\end{pgfscope}%
\begin{pgfscope}%
\pgfpathrectangle{\pgfqpoint{0.511823in}{0.504323in}}{\pgfqpoint{3.218177in}{3.225677in}} %
\pgfusepath{clip}%
\pgfsetbuttcap%
\pgfsetroundjoin%
\definecolor{currentfill}{rgb}{0.501961,0.000000,0.000000}%
\pgfsetfillcolor{currentfill}%
\pgfsetfillopacity{0.400000}%
\pgfsetlinewidth{0.501875pt}%
\definecolor{currentstroke}{rgb}{0.501961,0.000000,0.000000}%
\pgfsetstrokecolor{currentstroke}%
\pgfsetstrokeopacity{0.400000}%
\pgfsetdash{}{0pt}%
\pgfpathmoveto{\pgfqpoint{1.900151in}{2.644056in}}%
\pgfpathcurveto{\pgfqpoint{1.911201in}{2.644056in}}{\pgfqpoint{1.921800in}{2.648446in}}{\pgfqpoint{1.929614in}{2.656260in}}%
\pgfpathcurveto{\pgfqpoint{1.937427in}{2.664073in}}{\pgfqpoint{1.941818in}{2.674672in}}{\pgfqpoint{1.941818in}{2.685723in}}%
\pgfpathcurveto{\pgfqpoint{1.941818in}{2.696773in}}{\pgfqpoint{1.937427in}{2.707372in}}{\pgfqpoint{1.929614in}{2.715185in}}%
\pgfpathcurveto{\pgfqpoint{1.921800in}{2.722999in}}{\pgfqpoint{1.911201in}{2.727389in}}{\pgfqpoint{1.900151in}{2.727389in}}%
\pgfpathcurveto{\pgfqpoint{1.889101in}{2.727389in}}{\pgfqpoint{1.878502in}{2.722999in}}{\pgfqpoint{1.870688in}{2.715185in}}%
\pgfpathcurveto{\pgfqpoint{1.862875in}{2.707372in}}{\pgfqpoint{1.858484in}{2.696773in}}{\pgfqpoint{1.858484in}{2.685723in}}%
\pgfpathcurveto{\pgfqpoint{1.858484in}{2.674672in}}{\pgfqpoint{1.862875in}{2.664073in}}{\pgfqpoint{1.870688in}{2.656260in}}%
\pgfpathcurveto{\pgfqpoint{1.878502in}{2.648446in}}{\pgfqpoint{1.889101in}{2.644056in}}{\pgfqpoint{1.900151in}{2.644056in}}%
\pgfpathclose%
\pgfusepath{stroke,fill}%
\end{pgfscope}%
\begin{pgfscope}%
\pgfpathrectangle{\pgfqpoint{0.511823in}{0.504323in}}{\pgfqpoint{3.218177in}{3.225677in}} %
\pgfusepath{clip}%
\pgfsetbuttcap%
\pgfsetroundjoin%
\definecolor{currentfill}{rgb}{0.501961,0.000000,0.000000}%
\pgfsetfillcolor{currentfill}%
\pgfsetfillopacity{0.400000}%
\pgfsetlinewidth{0.501875pt}%
\definecolor{currentstroke}{rgb}{0.501961,0.000000,0.000000}%
\pgfsetstrokecolor{currentstroke}%
\pgfsetstrokeopacity{0.400000}%
\pgfsetdash{}{0pt}%
\pgfpathmoveto{\pgfqpoint{2.032609in}{2.879248in}}%
\pgfpathcurveto{\pgfqpoint{2.043659in}{2.879248in}}{\pgfqpoint{2.054258in}{2.883638in}}{\pgfqpoint{2.062071in}{2.891452in}}%
\pgfpathcurveto{\pgfqpoint{2.069885in}{2.899266in}}{\pgfqpoint{2.074275in}{2.909865in}}{\pgfqpoint{2.074275in}{2.920915in}}%
\pgfpathcurveto{\pgfqpoint{2.074275in}{2.931965in}}{\pgfqpoint{2.069885in}{2.942564in}}{\pgfqpoint{2.062071in}{2.950377in}}%
\pgfpathcurveto{\pgfqpoint{2.054258in}{2.958191in}}{\pgfqpoint{2.043659in}{2.962581in}}{\pgfqpoint{2.032609in}{2.962581in}}%
\pgfpathcurveto{\pgfqpoint{2.021559in}{2.962581in}}{\pgfqpoint{2.010960in}{2.958191in}}{\pgfqpoint{2.003146in}{2.950377in}}%
\pgfpathcurveto{\pgfqpoint{1.995332in}{2.942564in}}{\pgfqpoint{1.990942in}{2.931965in}}{\pgfqpoint{1.990942in}{2.920915in}}%
\pgfpathcurveto{\pgfqpoint{1.990942in}{2.909865in}}{\pgfqpoint{1.995332in}{2.899266in}}{\pgfqpoint{2.003146in}{2.891452in}}%
\pgfpathcurveto{\pgfqpoint{2.010960in}{2.883638in}}{\pgfqpoint{2.021559in}{2.879248in}}{\pgfqpoint{2.032609in}{2.879248in}}%
\pgfpathclose%
\pgfusepath{stroke,fill}%
\end{pgfscope}%
\begin{pgfscope}%
\pgfpathrectangle{\pgfqpoint{0.511823in}{0.504323in}}{\pgfqpoint{3.218177in}{3.225677in}} %
\pgfusepath{clip}%
\pgfsetbuttcap%
\pgfsetroundjoin%
\definecolor{currentfill}{rgb}{0.501961,0.000000,0.000000}%
\pgfsetfillcolor{currentfill}%
\pgfsetfillopacity{0.400000}%
\pgfsetlinewidth{0.501875pt}%
\definecolor{currentstroke}{rgb}{0.501961,0.000000,0.000000}%
\pgfsetstrokecolor{currentstroke}%
\pgfsetstrokeopacity{0.400000}%
\pgfsetdash{}{0pt}%
\pgfpathmoveto{\pgfqpoint{1.861774in}{2.609576in}}%
\pgfpathcurveto{\pgfqpoint{1.872824in}{2.609576in}}{\pgfqpoint{1.883423in}{2.613967in}}{\pgfqpoint{1.891237in}{2.621780in}}%
\pgfpathcurveto{\pgfqpoint{1.899050in}{2.629594in}}{\pgfqpoint{1.903440in}{2.640193in}}{\pgfqpoint{1.903440in}{2.651243in}}%
\pgfpathcurveto{\pgfqpoint{1.903440in}{2.662293in}}{\pgfqpoint{1.899050in}{2.672892in}}{\pgfqpoint{1.891237in}{2.680706in}}%
\pgfpathcurveto{\pgfqpoint{1.883423in}{2.688519in}}{\pgfqpoint{1.872824in}{2.692910in}}{\pgfqpoint{1.861774in}{2.692910in}}%
\pgfpathcurveto{\pgfqpoint{1.850724in}{2.692910in}}{\pgfqpoint{1.840125in}{2.688519in}}{\pgfqpoint{1.832311in}{2.680706in}}%
\pgfpathcurveto{\pgfqpoint{1.824497in}{2.672892in}}{\pgfqpoint{1.820107in}{2.662293in}}{\pgfqpoint{1.820107in}{2.651243in}}%
\pgfpathcurveto{\pgfqpoint{1.820107in}{2.640193in}}{\pgfqpoint{1.824497in}{2.629594in}}{\pgfqpoint{1.832311in}{2.621780in}}%
\pgfpathcurveto{\pgfqpoint{1.840125in}{2.613967in}}{\pgfqpoint{1.850724in}{2.609576in}}{\pgfqpoint{1.861774in}{2.609576in}}%
\pgfpathclose%
\pgfusepath{stroke,fill}%
\end{pgfscope}%
\begin{pgfscope}%
\pgfpathrectangle{\pgfqpoint{0.511823in}{0.504323in}}{\pgfqpoint{3.218177in}{3.225677in}} %
\pgfusepath{clip}%
\pgfsetbuttcap%
\pgfsetroundjoin%
\definecolor{currentfill}{rgb}{0.501961,0.000000,0.000000}%
\pgfsetfillcolor{currentfill}%
\pgfsetfillopacity{0.400000}%
\pgfsetlinewidth{0.501875pt}%
\definecolor{currentstroke}{rgb}{0.501961,0.000000,0.000000}%
\pgfsetstrokecolor{currentstroke}%
\pgfsetstrokeopacity{0.400000}%
\pgfsetdash{}{0pt}%
\pgfpathmoveto{\pgfqpoint{1.892031in}{2.675359in}}%
\pgfpathcurveto{\pgfqpoint{1.903081in}{2.675359in}}{\pgfqpoint{1.913680in}{2.679749in}}{\pgfqpoint{1.921493in}{2.687563in}}%
\pgfpathcurveto{\pgfqpoint{1.929307in}{2.695376in}}{\pgfqpoint{1.933697in}{2.705975in}}{\pgfqpoint{1.933697in}{2.717025in}}%
\pgfpathcurveto{\pgfqpoint{1.933697in}{2.728076in}}{\pgfqpoint{1.929307in}{2.738675in}}{\pgfqpoint{1.921493in}{2.746488in}}%
\pgfpathcurveto{\pgfqpoint{1.913680in}{2.754302in}}{\pgfqpoint{1.903081in}{2.758692in}}{\pgfqpoint{1.892031in}{2.758692in}}%
\pgfpathcurveto{\pgfqpoint{1.880980in}{2.758692in}}{\pgfqpoint{1.870381in}{2.754302in}}{\pgfqpoint{1.862568in}{2.746488in}}%
\pgfpathcurveto{\pgfqpoint{1.854754in}{2.738675in}}{\pgfqpoint{1.850364in}{2.728076in}}{\pgfqpoint{1.850364in}{2.717025in}}%
\pgfpathcurveto{\pgfqpoint{1.850364in}{2.705975in}}{\pgfqpoint{1.854754in}{2.695376in}}{\pgfqpoint{1.862568in}{2.687563in}}%
\pgfpathcurveto{\pgfqpoint{1.870381in}{2.679749in}}{\pgfqpoint{1.880980in}{2.675359in}}{\pgfqpoint{1.892031in}{2.675359in}}%
\pgfpathclose%
\pgfusepath{stroke,fill}%
\end{pgfscope}%
\begin{pgfscope}%
\pgfpathrectangle{\pgfqpoint{0.511823in}{0.504323in}}{\pgfqpoint{3.218177in}{3.225677in}} %
\pgfusepath{clip}%
\pgfsetbuttcap%
\pgfsetroundjoin%
\definecolor{currentfill}{rgb}{0.501961,0.000000,0.000000}%
\pgfsetfillcolor{currentfill}%
\pgfsetfillopacity{0.400000}%
\pgfsetlinewidth{0.501875pt}%
\definecolor{currentstroke}{rgb}{0.501961,0.000000,0.000000}%
\pgfsetstrokecolor{currentstroke}%
\pgfsetstrokeopacity{0.400000}%
\pgfsetdash{}{0pt}%
\pgfpathmoveto{\pgfqpoint{1.904801in}{2.712329in}}%
\pgfpathcurveto{\pgfqpoint{1.915851in}{2.712329in}}{\pgfqpoint{1.926450in}{2.716719in}}{\pgfqpoint{1.934264in}{2.724533in}}%
\pgfpathcurveto{\pgfqpoint{1.942078in}{2.732347in}}{\pgfqpoint{1.946468in}{2.742946in}}{\pgfqpoint{1.946468in}{2.753996in}}%
\pgfpathcurveto{\pgfqpoint{1.946468in}{2.765046in}}{\pgfqpoint{1.942078in}{2.775645in}}{\pgfqpoint{1.934264in}{2.783459in}}%
\pgfpathcurveto{\pgfqpoint{1.926450in}{2.791272in}}{\pgfqpoint{1.915851in}{2.795662in}}{\pgfqpoint{1.904801in}{2.795662in}}%
\pgfpathcurveto{\pgfqpoint{1.893751in}{2.795662in}}{\pgfqpoint{1.883152in}{2.791272in}}{\pgfqpoint{1.875338in}{2.783459in}}%
\pgfpathcurveto{\pgfqpoint{1.867525in}{2.775645in}}{\pgfqpoint{1.863134in}{2.765046in}}{\pgfqpoint{1.863134in}{2.753996in}}%
\pgfpathcurveto{\pgfqpoint{1.863134in}{2.742946in}}{\pgfqpoint{1.867525in}{2.732347in}}{\pgfqpoint{1.875338in}{2.724533in}}%
\pgfpathcurveto{\pgfqpoint{1.883152in}{2.716719in}}{\pgfqpoint{1.893751in}{2.712329in}}{\pgfqpoint{1.904801in}{2.712329in}}%
\pgfpathclose%
\pgfusepath{stroke,fill}%
\end{pgfscope}%
\begin{pgfscope}%
\pgfpathrectangle{\pgfqpoint{0.511823in}{0.504323in}}{\pgfqpoint{3.218177in}{3.225677in}} %
\pgfusepath{clip}%
\pgfsetbuttcap%
\pgfsetroundjoin%
\definecolor{currentfill}{rgb}{0.501961,0.000000,0.000000}%
\pgfsetfillcolor{currentfill}%
\pgfsetfillopacity{0.400000}%
\pgfsetlinewidth{0.501875pt}%
\definecolor{currentstroke}{rgb}{0.501961,0.000000,0.000000}%
\pgfsetstrokecolor{currentstroke}%
\pgfsetstrokeopacity{0.400000}%
\pgfsetdash{}{0pt}%
\pgfpathmoveto{\pgfqpoint{2.088239in}{3.042310in}}%
\pgfpathcurveto{\pgfqpoint{2.099289in}{3.042310in}}{\pgfqpoint{2.109888in}{3.046701in}}{\pgfqpoint{2.117702in}{3.054514in}}%
\pgfpathcurveto{\pgfqpoint{2.125516in}{3.062328in}}{\pgfqpoint{2.129906in}{3.072927in}}{\pgfqpoint{2.129906in}{3.083977in}}%
\pgfpathcurveto{\pgfqpoint{2.129906in}{3.095027in}}{\pgfqpoint{2.125516in}{3.105626in}}{\pgfqpoint{2.117702in}{3.113440in}}%
\pgfpathcurveto{\pgfqpoint{2.109888in}{3.121254in}}{\pgfqpoint{2.099289in}{3.125644in}}{\pgfqpoint{2.088239in}{3.125644in}}%
\pgfpathcurveto{\pgfqpoint{2.077189in}{3.125644in}}{\pgfqpoint{2.066590in}{3.121254in}}{\pgfqpoint{2.058777in}{3.113440in}}%
\pgfpathcurveto{\pgfqpoint{2.050963in}{3.105626in}}{\pgfqpoint{2.046573in}{3.095027in}}{\pgfqpoint{2.046573in}{3.083977in}}%
\pgfpathcurveto{\pgfqpoint{2.046573in}{3.072927in}}{\pgfqpoint{2.050963in}{3.062328in}}{\pgfqpoint{2.058777in}{3.054514in}}%
\pgfpathcurveto{\pgfqpoint{2.066590in}{3.046701in}}{\pgfqpoint{2.077189in}{3.042310in}}{\pgfqpoint{2.088239in}{3.042310in}}%
\pgfpathclose%
\pgfusepath{stroke,fill}%
\end{pgfscope}%
\begin{pgfscope}%
\pgfpathrectangle{\pgfqpoint{0.511823in}{0.504323in}}{\pgfqpoint{3.218177in}{3.225677in}} %
\pgfusepath{clip}%
\pgfsetbuttcap%
\pgfsetroundjoin%
\definecolor{currentfill}{rgb}{0.501961,0.000000,0.000000}%
\pgfsetfillcolor{currentfill}%
\pgfsetfillopacity{0.400000}%
\pgfsetlinewidth{0.501875pt}%
\definecolor{currentstroke}{rgb}{0.501961,0.000000,0.000000}%
\pgfsetstrokecolor{currentstroke}%
\pgfsetstrokeopacity{0.400000}%
\pgfsetdash{}{0pt}%
\pgfpathmoveto{\pgfqpoint{1.956495in}{2.832950in}}%
\pgfpathcurveto{\pgfqpoint{1.967545in}{2.832950in}}{\pgfqpoint{1.978144in}{2.837341in}}{\pgfqpoint{1.985958in}{2.845154in}}%
\pgfpathcurveto{\pgfqpoint{1.993771in}{2.852968in}}{\pgfqpoint{1.998162in}{2.863567in}}{\pgfqpoint{1.998162in}{2.874617in}}%
\pgfpathcurveto{\pgfqpoint{1.998162in}{2.885667in}}{\pgfqpoint{1.993771in}{2.896266in}}{\pgfqpoint{1.985958in}{2.904080in}}%
\pgfpathcurveto{\pgfqpoint{1.978144in}{2.911893in}}{\pgfqpoint{1.967545in}{2.916284in}}{\pgfqpoint{1.956495in}{2.916284in}}%
\pgfpathcurveto{\pgfqpoint{1.945445in}{2.916284in}}{\pgfqpoint{1.934846in}{2.911893in}}{\pgfqpoint{1.927032in}{2.904080in}}%
\pgfpathcurveto{\pgfqpoint{1.919218in}{2.896266in}}{\pgfqpoint{1.914828in}{2.885667in}}{\pgfqpoint{1.914828in}{2.874617in}}%
\pgfpathcurveto{\pgfqpoint{1.914828in}{2.863567in}}{\pgfqpoint{1.919218in}{2.852968in}}{\pgfqpoint{1.927032in}{2.845154in}}%
\pgfpathcurveto{\pgfqpoint{1.934846in}{2.837341in}}{\pgfqpoint{1.945445in}{2.832950in}}{\pgfqpoint{1.956495in}{2.832950in}}%
\pgfpathclose%
\pgfusepath{stroke,fill}%
\end{pgfscope}%
\begin{pgfscope}%
\pgfpathrectangle{\pgfqpoint{0.511823in}{0.504323in}}{\pgfqpoint{3.218177in}{3.225677in}} %
\pgfusepath{clip}%
\pgfsetbuttcap%
\pgfsetroundjoin%
\definecolor{currentfill}{rgb}{0.501961,0.000000,0.000000}%
\pgfsetfillcolor{currentfill}%
\pgfsetfillopacity{0.400000}%
\pgfsetlinewidth{0.501875pt}%
\definecolor{currentstroke}{rgb}{0.501961,0.000000,0.000000}%
\pgfsetstrokecolor{currentstroke}%
\pgfsetstrokeopacity{0.400000}%
\pgfsetdash{}{0pt}%
\pgfpathmoveto{\pgfqpoint{1.899551in}{2.750501in}}%
\pgfpathcurveto{\pgfqpoint{1.910601in}{2.750501in}}{\pgfqpoint{1.921200in}{2.754892in}}{\pgfqpoint{1.929014in}{2.762705in}}%
\pgfpathcurveto{\pgfqpoint{1.936827in}{2.770519in}}{\pgfqpoint{1.941218in}{2.781118in}}{\pgfqpoint{1.941218in}{2.792168in}}%
\pgfpathcurveto{\pgfqpoint{1.941218in}{2.803218in}}{\pgfqpoint{1.936827in}{2.813817in}}{\pgfqpoint{1.929014in}{2.821631in}}%
\pgfpathcurveto{\pgfqpoint{1.921200in}{2.829445in}}{\pgfqpoint{1.910601in}{2.833835in}}{\pgfqpoint{1.899551in}{2.833835in}}%
\pgfpathcurveto{\pgfqpoint{1.888501in}{2.833835in}}{\pgfqpoint{1.877902in}{2.829445in}}{\pgfqpoint{1.870088in}{2.821631in}}%
\pgfpathcurveto{\pgfqpoint{1.862274in}{2.813817in}}{\pgfqpoint{1.857884in}{2.803218in}}{\pgfqpoint{1.857884in}{2.792168in}}%
\pgfpathcurveto{\pgfqpoint{1.857884in}{2.781118in}}{\pgfqpoint{1.862274in}{2.770519in}}{\pgfqpoint{1.870088in}{2.762705in}}%
\pgfpathcurveto{\pgfqpoint{1.877902in}{2.754892in}}{\pgfqpoint{1.888501in}{2.750501in}}{\pgfqpoint{1.899551in}{2.750501in}}%
\pgfpathclose%
\pgfusepath{stroke,fill}%
\end{pgfscope}%
\begin{pgfscope}%
\pgfpathrectangle{\pgfqpoint{0.511823in}{0.504323in}}{\pgfqpoint{3.218177in}{3.225677in}} %
\pgfusepath{clip}%
\pgfsetbuttcap%
\pgfsetroundjoin%
\definecolor{currentfill}{rgb}{0.501961,0.000000,0.000000}%
\pgfsetfillcolor{currentfill}%
\pgfsetfillopacity{0.400000}%
\pgfsetlinewidth{0.501875pt}%
\definecolor{currentstroke}{rgb}{0.501961,0.000000,0.000000}%
\pgfsetstrokecolor{currentstroke}%
\pgfsetstrokeopacity{0.400000}%
\pgfsetdash{}{0pt}%
\pgfpathmoveto{\pgfqpoint{1.967510in}{2.885720in}}%
\pgfpathcurveto{\pgfqpoint{1.978560in}{2.885720in}}{\pgfqpoint{1.989159in}{2.890110in}}{\pgfqpoint{1.996973in}{2.897924in}}%
\pgfpathcurveto{\pgfqpoint{2.004787in}{2.905737in}}{\pgfqpoint{2.009177in}{2.916336in}}{\pgfqpoint{2.009177in}{2.927386in}}%
\pgfpathcurveto{\pgfqpoint{2.009177in}{2.938437in}}{\pgfqpoint{2.004787in}{2.949036in}}{\pgfqpoint{1.996973in}{2.956849in}}%
\pgfpathcurveto{\pgfqpoint{1.989159in}{2.964663in}}{\pgfqpoint{1.978560in}{2.969053in}}{\pgfqpoint{1.967510in}{2.969053in}}%
\pgfpathcurveto{\pgfqpoint{1.956460in}{2.969053in}}{\pgfqpoint{1.945861in}{2.964663in}}{\pgfqpoint{1.938047in}{2.956849in}}%
\pgfpathcurveto{\pgfqpoint{1.930234in}{2.949036in}}{\pgfqpoint{1.925844in}{2.938437in}}{\pgfqpoint{1.925844in}{2.927386in}}%
\pgfpathcurveto{\pgfqpoint{1.925844in}{2.916336in}}{\pgfqpoint{1.930234in}{2.905737in}}{\pgfqpoint{1.938047in}{2.897924in}}%
\pgfpathcurveto{\pgfqpoint{1.945861in}{2.890110in}}{\pgfqpoint{1.956460in}{2.885720in}}{\pgfqpoint{1.967510in}{2.885720in}}%
\pgfpathclose%
\pgfusepath{stroke,fill}%
\end{pgfscope}%
\begin{pgfscope}%
\pgfpathrectangle{\pgfqpoint{0.511823in}{0.504323in}}{\pgfqpoint{3.218177in}{3.225677in}} %
\pgfusepath{clip}%
\pgfsetbuttcap%
\pgfsetroundjoin%
\definecolor{currentfill}{rgb}{0.501961,0.000000,0.000000}%
\pgfsetfillcolor{currentfill}%
\pgfsetfillopacity{0.400000}%
\pgfsetlinewidth{0.501875pt}%
\definecolor{currentstroke}{rgb}{0.501961,0.000000,0.000000}%
\pgfsetstrokecolor{currentstroke}%
\pgfsetstrokeopacity{0.400000}%
\pgfsetdash{}{0pt}%
\pgfpathmoveto{\pgfqpoint{2.074573in}{3.092032in}}%
\pgfpathcurveto{\pgfqpoint{2.085623in}{3.092032in}}{\pgfqpoint{2.096222in}{3.096423in}}{\pgfqpoint{2.104036in}{3.104236in}}%
\pgfpathcurveto{\pgfqpoint{2.111850in}{3.112050in}}{\pgfqpoint{2.116240in}{3.122649in}}{\pgfqpoint{2.116240in}{3.133699in}}%
\pgfpathcurveto{\pgfqpoint{2.116240in}{3.144749in}}{\pgfqpoint{2.111850in}{3.155348in}}{\pgfqpoint{2.104036in}{3.163162in}}%
\pgfpathcurveto{\pgfqpoint{2.096222in}{3.170975in}}{\pgfqpoint{2.085623in}{3.175366in}}{\pgfqpoint{2.074573in}{3.175366in}}%
\pgfpathcurveto{\pgfqpoint{2.063523in}{3.175366in}}{\pgfqpoint{2.052924in}{3.170975in}}{\pgfqpoint{2.045110in}{3.163162in}}%
\pgfpathcurveto{\pgfqpoint{2.037297in}{3.155348in}}{\pgfqpoint{2.032906in}{3.144749in}}{\pgfqpoint{2.032906in}{3.133699in}}%
\pgfpathcurveto{\pgfqpoint{2.032906in}{3.122649in}}{\pgfqpoint{2.037297in}{3.112050in}}{\pgfqpoint{2.045110in}{3.104236in}}%
\pgfpathcurveto{\pgfqpoint{2.052924in}{3.096423in}}{\pgfqpoint{2.063523in}{3.092032in}}{\pgfqpoint{2.074573in}{3.092032in}}%
\pgfpathclose%
\pgfusepath{stroke,fill}%
\end{pgfscope}%
\begin{pgfscope}%
\pgfpathrectangle{\pgfqpoint{0.511823in}{0.504323in}}{\pgfqpoint{3.218177in}{3.225677in}} %
\pgfusepath{clip}%
\pgfsetbuttcap%
\pgfsetroundjoin%
\definecolor{currentfill}{rgb}{0.501961,0.000000,0.000000}%
\pgfsetfillcolor{currentfill}%
\pgfsetfillopacity{0.400000}%
\pgfsetlinewidth{0.501875pt}%
\definecolor{currentstroke}{rgb}{0.501961,0.000000,0.000000}%
\pgfsetstrokecolor{currentstroke}%
\pgfsetstrokeopacity{0.400000}%
\pgfsetdash{}{0pt}%
\pgfpathmoveto{\pgfqpoint{1.910273in}{2.818387in}}%
\pgfpathcurveto{\pgfqpoint{1.921323in}{2.818387in}}{\pgfqpoint{1.931923in}{2.822777in}}{\pgfqpoint{1.939736in}{2.830591in}}%
\pgfpathcurveto{\pgfqpoint{1.947550in}{2.838404in}}{\pgfqpoint{1.951940in}{2.849003in}}{\pgfqpoint{1.951940in}{2.860053in}}%
\pgfpathcurveto{\pgfqpoint{1.951940in}{2.871103in}}{\pgfqpoint{1.947550in}{2.881702in}}{\pgfqpoint{1.939736in}{2.889516in}}%
\pgfpathcurveto{\pgfqpoint{1.931923in}{2.897330in}}{\pgfqpoint{1.921323in}{2.901720in}}{\pgfqpoint{1.910273in}{2.901720in}}%
\pgfpathcurveto{\pgfqpoint{1.899223in}{2.901720in}}{\pgfqpoint{1.888624in}{2.897330in}}{\pgfqpoint{1.880811in}{2.889516in}}%
\pgfpathcurveto{\pgfqpoint{1.872997in}{2.881702in}}{\pgfqpoint{1.868607in}{2.871103in}}{\pgfqpoint{1.868607in}{2.860053in}}%
\pgfpathcurveto{\pgfqpoint{1.868607in}{2.849003in}}{\pgfqpoint{1.872997in}{2.838404in}}{\pgfqpoint{1.880811in}{2.830591in}}%
\pgfpathcurveto{\pgfqpoint{1.888624in}{2.822777in}}{\pgfqpoint{1.899223in}{2.818387in}}{\pgfqpoint{1.910273in}{2.818387in}}%
\pgfpathclose%
\pgfusepath{stroke,fill}%
\end{pgfscope}%
\begin{pgfscope}%
\pgfpathrectangle{\pgfqpoint{0.511823in}{0.504323in}}{\pgfqpoint{3.218177in}{3.225677in}} %
\pgfusepath{clip}%
\pgfsetbuttcap%
\pgfsetroundjoin%
\definecolor{currentfill}{rgb}{0.501961,0.000000,0.000000}%
\pgfsetfillcolor{currentfill}%
\pgfsetfillopacity{0.400000}%
\pgfsetlinewidth{0.501875pt}%
\definecolor{currentstroke}{rgb}{0.501961,0.000000,0.000000}%
\pgfsetstrokecolor{currentstroke}%
\pgfsetstrokeopacity{0.400000}%
\pgfsetdash{}{0pt}%
\pgfpathmoveto{\pgfqpoint{1.847792in}{2.723109in}}%
\pgfpathcurveto{\pgfqpoint{1.858842in}{2.723109in}}{\pgfqpoint{1.869442in}{2.727499in}}{\pgfqpoint{1.877255in}{2.735313in}}%
\pgfpathcurveto{\pgfqpoint{1.885069in}{2.743126in}}{\pgfqpoint{1.889459in}{2.753726in}}{\pgfqpoint{1.889459in}{2.764776in}}%
\pgfpathcurveto{\pgfqpoint{1.889459in}{2.775826in}}{\pgfqpoint{1.885069in}{2.786425in}}{\pgfqpoint{1.877255in}{2.794238in}}%
\pgfpathcurveto{\pgfqpoint{1.869442in}{2.802052in}}{\pgfqpoint{1.858842in}{2.806442in}}{\pgfqpoint{1.847792in}{2.806442in}}%
\pgfpathcurveto{\pgfqpoint{1.836742in}{2.806442in}}{\pgfqpoint{1.826143in}{2.802052in}}{\pgfqpoint{1.818330in}{2.794238in}}%
\pgfpathcurveto{\pgfqpoint{1.810516in}{2.786425in}}{\pgfqpoint{1.806126in}{2.775826in}}{\pgfqpoint{1.806126in}{2.764776in}}%
\pgfpathcurveto{\pgfqpoint{1.806126in}{2.753726in}}{\pgfqpoint{1.810516in}{2.743126in}}{\pgfqpoint{1.818330in}{2.735313in}}%
\pgfpathcurveto{\pgfqpoint{1.826143in}{2.727499in}}{\pgfqpoint{1.836742in}{2.723109in}}{\pgfqpoint{1.847792in}{2.723109in}}%
\pgfpathclose%
\pgfusepath{stroke,fill}%
\end{pgfscope}%
\begin{pgfscope}%
\pgfpathrectangle{\pgfqpoint{0.511823in}{0.504323in}}{\pgfqpoint{3.218177in}{3.225677in}} %
\pgfusepath{clip}%
\pgfsetbuttcap%
\pgfsetroundjoin%
\definecolor{currentfill}{rgb}{0.501961,0.000000,0.000000}%
\pgfsetfillcolor{currentfill}%
\pgfsetfillopacity{0.400000}%
\pgfsetlinewidth{0.501875pt}%
\definecolor{currentstroke}{rgb}{0.501961,0.000000,0.000000}%
\pgfsetstrokecolor{currentstroke}%
\pgfsetstrokeopacity{0.400000}%
\pgfsetdash{}{0pt}%
\pgfpathmoveto{\pgfqpoint{1.837278in}{2.720244in}}%
\pgfpathcurveto{\pgfqpoint{1.848328in}{2.720244in}}{\pgfqpoint{1.858927in}{2.724635in}}{\pgfqpoint{1.866741in}{2.732448in}}%
\pgfpathcurveto{\pgfqpoint{1.874554in}{2.740262in}}{\pgfqpoint{1.878944in}{2.750861in}}{\pgfqpoint{1.878944in}{2.761911in}}%
\pgfpathcurveto{\pgfqpoint{1.878944in}{2.772961in}}{\pgfqpoint{1.874554in}{2.783560in}}{\pgfqpoint{1.866741in}{2.791374in}}%
\pgfpathcurveto{\pgfqpoint{1.858927in}{2.799187in}}{\pgfqpoint{1.848328in}{2.803578in}}{\pgfqpoint{1.837278in}{2.803578in}}%
\pgfpathcurveto{\pgfqpoint{1.826228in}{2.803578in}}{\pgfqpoint{1.815629in}{2.799187in}}{\pgfqpoint{1.807815in}{2.791374in}}%
\pgfpathcurveto{\pgfqpoint{1.800001in}{2.783560in}}{\pgfqpoint{1.795611in}{2.772961in}}{\pgfqpoint{1.795611in}{2.761911in}}%
\pgfpathcurveto{\pgfqpoint{1.795611in}{2.750861in}}{\pgfqpoint{1.800001in}{2.740262in}}{\pgfqpoint{1.807815in}{2.732448in}}%
\pgfpathcurveto{\pgfqpoint{1.815629in}{2.724635in}}{\pgfqpoint{1.826228in}{2.720244in}}{\pgfqpoint{1.837278in}{2.720244in}}%
\pgfpathclose%
\pgfusepath{stroke,fill}%
\end{pgfscope}%
\begin{pgfscope}%
\pgfpathrectangle{\pgfqpoint{0.511823in}{0.504323in}}{\pgfqpoint{3.218177in}{3.225677in}} %
\pgfusepath{clip}%
\pgfsetbuttcap%
\pgfsetroundjoin%
\definecolor{currentfill}{rgb}{0.501961,0.000000,0.000000}%
\pgfsetfillcolor{currentfill}%
\pgfsetfillopacity{0.400000}%
\pgfsetlinewidth{0.501875pt}%
\definecolor{currentstroke}{rgb}{0.501961,0.000000,0.000000}%
\pgfsetstrokecolor{currentstroke}%
\pgfsetstrokeopacity{0.400000}%
\pgfsetdash{}{0pt}%
\pgfpathmoveto{\pgfqpoint{1.894124in}{2.840004in}}%
\pgfpathcurveto{\pgfqpoint{1.905174in}{2.840004in}}{\pgfqpoint{1.915773in}{2.844395in}}{\pgfqpoint{1.923586in}{2.852208in}}%
\pgfpathcurveto{\pgfqpoint{1.931400in}{2.860022in}}{\pgfqpoint{1.935790in}{2.870621in}}{\pgfqpoint{1.935790in}{2.881671in}}%
\pgfpathcurveto{\pgfqpoint{1.935790in}{2.892721in}}{\pgfqpoint{1.931400in}{2.903320in}}{\pgfqpoint{1.923586in}{2.911134in}}%
\pgfpathcurveto{\pgfqpoint{1.915773in}{2.918947in}}{\pgfqpoint{1.905174in}{2.923338in}}{\pgfqpoint{1.894124in}{2.923338in}}%
\pgfpathcurveto{\pgfqpoint{1.883074in}{2.923338in}}{\pgfqpoint{1.872475in}{2.918947in}}{\pgfqpoint{1.864661in}{2.911134in}}%
\pgfpathcurveto{\pgfqpoint{1.856847in}{2.903320in}}{\pgfqpoint{1.852457in}{2.892721in}}{\pgfqpoint{1.852457in}{2.881671in}}%
\pgfpathcurveto{\pgfqpoint{1.852457in}{2.870621in}}{\pgfqpoint{1.856847in}{2.860022in}}{\pgfqpoint{1.864661in}{2.852208in}}%
\pgfpathcurveto{\pgfqpoint{1.872475in}{2.844395in}}{\pgfqpoint{1.883074in}{2.840004in}}{\pgfqpoint{1.894124in}{2.840004in}}%
\pgfpathclose%
\pgfusepath{stroke,fill}%
\end{pgfscope}%
\begin{pgfscope}%
\pgfpathrectangle{\pgfqpoint{0.511823in}{0.504323in}}{\pgfqpoint{3.218177in}{3.225677in}} %
\pgfusepath{clip}%
\pgfsetbuttcap%
\pgfsetroundjoin%
\definecolor{currentfill}{rgb}{0.501961,0.000000,0.000000}%
\pgfsetfillcolor{currentfill}%
\pgfsetfillopacity{0.400000}%
\pgfsetlinewidth{0.501875pt}%
\definecolor{currentstroke}{rgb}{0.501961,0.000000,0.000000}%
\pgfsetstrokecolor{currentstroke}%
\pgfsetstrokeopacity{0.400000}%
\pgfsetdash{}{0pt}%
\pgfpathmoveto{\pgfqpoint{1.990749in}{3.034624in}}%
\pgfpathcurveto{\pgfqpoint{2.001799in}{3.034624in}}{\pgfqpoint{2.012398in}{3.039014in}}{\pgfqpoint{2.020212in}{3.046828in}}%
\pgfpathcurveto{\pgfqpoint{2.028025in}{3.054642in}}{\pgfqpoint{2.032415in}{3.065241in}}{\pgfqpoint{2.032415in}{3.076291in}}%
\pgfpathcurveto{\pgfqpoint{2.032415in}{3.087341in}}{\pgfqpoint{2.028025in}{3.097940in}}{\pgfqpoint{2.020212in}{3.105754in}}%
\pgfpathcurveto{\pgfqpoint{2.012398in}{3.113567in}}{\pgfqpoint{2.001799in}{3.117958in}}{\pgfqpoint{1.990749in}{3.117958in}}%
\pgfpathcurveto{\pgfqpoint{1.979699in}{3.117958in}}{\pgfqpoint{1.969100in}{3.113567in}}{\pgfqpoint{1.961286in}{3.105754in}}%
\pgfpathcurveto{\pgfqpoint{1.953472in}{3.097940in}}{\pgfqpoint{1.949082in}{3.087341in}}{\pgfqpoint{1.949082in}{3.076291in}}%
\pgfpathcurveto{\pgfqpoint{1.949082in}{3.065241in}}{\pgfqpoint{1.953472in}{3.054642in}}{\pgfqpoint{1.961286in}{3.046828in}}%
\pgfpathcurveto{\pgfqpoint{1.969100in}{3.039014in}}{\pgfqpoint{1.979699in}{3.034624in}}{\pgfqpoint{1.990749in}{3.034624in}}%
\pgfpathclose%
\pgfusepath{stroke,fill}%
\end{pgfscope}%
\begin{pgfscope}%
\pgfpathrectangle{\pgfqpoint{0.511823in}{0.504323in}}{\pgfqpoint{3.218177in}{3.225677in}} %
\pgfusepath{clip}%
\pgfsetbuttcap%
\pgfsetroundjoin%
\definecolor{currentfill}{rgb}{0.501961,0.000000,0.000000}%
\pgfsetfillcolor{currentfill}%
\pgfsetfillopacity{0.400000}%
\pgfsetlinewidth{0.501875pt}%
\definecolor{currentstroke}{rgb}{0.501961,0.000000,0.000000}%
\pgfsetstrokecolor{currentstroke}%
\pgfsetstrokeopacity{0.400000}%
\pgfsetdash{}{0pt}%
\pgfpathmoveto{\pgfqpoint{1.874024in}{2.837435in}}%
\pgfpathcurveto{\pgfqpoint{1.885075in}{2.837435in}}{\pgfqpoint{1.895674in}{2.841825in}}{\pgfqpoint{1.903487in}{2.849639in}}%
\pgfpathcurveto{\pgfqpoint{1.911301in}{2.857452in}}{\pgfqpoint{1.915691in}{2.868052in}}{\pgfqpoint{1.915691in}{2.879102in}}%
\pgfpathcurveto{\pgfqpoint{1.915691in}{2.890152in}}{\pgfqpoint{1.911301in}{2.900751in}}{\pgfqpoint{1.903487in}{2.908564in}}%
\pgfpathcurveto{\pgfqpoint{1.895674in}{2.916378in}}{\pgfqpoint{1.885075in}{2.920768in}}{\pgfqpoint{1.874024in}{2.920768in}}%
\pgfpathcurveto{\pgfqpoint{1.862974in}{2.920768in}}{\pgfqpoint{1.852375in}{2.916378in}}{\pgfqpoint{1.844562in}{2.908564in}}%
\pgfpathcurveto{\pgfqpoint{1.836748in}{2.900751in}}{\pgfqpoint{1.832358in}{2.890152in}}{\pgfqpoint{1.832358in}{2.879102in}}%
\pgfpathcurveto{\pgfqpoint{1.832358in}{2.868052in}}{\pgfqpoint{1.836748in}{2.857452in}}{\pgfqpoint{1.844562in}{2.849639in}}%
\pgfpathcurveto{\pgfqpoint{1.852375in}{2.841825in}}{\pgfqpoint{1.862974in}{2.837435in}}{\pgfqpoint{1.874024in}{2.837435in}}%
\pgfpathclose%
\pgfusepath{stroke,fill}%
\end{pgfscope}%
\begin{pgfscope}%
\pgfpathrectangle{\pgfqpoint{0.511823in}{0.504323in}}{\pgfqpoint{3.218177in}{3.225677in}} %
\pgfusepath{clip}%
\pgfsetbuttcap%
\pgfsetroundjoin%
\definecolor{currentfill}{rgb}{0.501961,0.000000,0.000000}%
\pgfsetfillcolor{currentfill}%
\pgfsetfillopacity{0.400000}%
\pgfsetlinewidth{0.501875pt}%
\definecolor{currentstroke}{rgb}{0.501961,0.000000,0.000000}%
\pgfsetstrokecolor{currentstroke}%
\pgfsetstrokeopacity{0.400000}%
\pgfsetdash{}{0pt}%
\pgfpathmoveto{\pgfqpoint{1.860691in}{2.829913in}}%
\pgfpathcurveto{\pgfqpoint{1.871742in}{2.829913in}}{\pgfqpoint{1.882341in}{2.834303in}}{\pgfqpoint{1.890154in}{2.842117in}}%
\pgfpathcurveto{\pgfqpoint{1.897968in}{2.849931in}}{\pgfqpoint{1.902358in}{2.860530in}}{\pgfqpoint{1.902358in}{2.871580in}}%
\pgfpathcurveto{\pgfqpoint{1.902358in}{2.882630in}}{\pgfqpoint{1.897968in}{2.893229in}}{\pgfqpoint{1.890154in}{2.901042in}}%
\pgfpathcurveto{\pgfqpoint{1.882341in}{2.908856in}}{\pgfqpoint{1.871742in}{2.913246in}}{\pgfqpoint{1.860691in}{2.913246in}}%
\pgfpathcurveto{\pgfqpoint{1.849641in}{2.913246in}}{\pgfqpoint{1.839042in}{2.908856in}}{\pgfqpoint{1.831229in}{2.901042in}}%
\pgfpathcurveto{\pgfqpoint{1.823415in}{2.893229in}}{\pgfqpoint{1.819025in}{2.882630in}}{\pgfqpoint{1.819025in}{2.871580in}}%
\pgfpathcurveto{\pgfqpoint{1.819025in}{2.860530in}}{\pgfqpoint{1.823415in}{2.849931in}}{\pgfqpoint{1.831229in}{2.842117in}}%
\pgfpathcurveto{\pgfqpoint{1.839042in}{2.834303in}}{\pgfqpoint{1.849641in}{2.829913in}}{\pgfqpoint{1.860691in}{2.829913in}}%
\pgfpathclose%
\pgfusepath{stroke,fill}%
\end{pgfscope}%
\begin{pgfscope}%
\pgfpathrectangle{\pgfqpoint{0.511823in}{0.504323in}}{\pgfqpoint{3.218177in}{3.225677in}} %
\pgfusepath{clip}%
\pgfsetbuttcap%
\pgfsetroundjoin%
\definecolor{currentfill}{rgb}{0.501961,0.000000,0.000000}%
\pgfsetfillcolor{currentfill}%
\pgfsetfillopacity{0.400000}%
\pgfsetlinewidth{0.501875pt}%
\definecolor{currentstroke}{rgb}{0.501961,0.000000,0.000000}%
\pgfsetstrokecolor{currentstroke}%
\pgfsetstrokeopacity{0.400000}%
\pgfsetdash{}{0pt}%
\pgfpathmoveto{\pgfqpoint{1.808898in}{2.749914in}}%
\pgfpathcurveto{\pgfqpoint{1.819948in}{2.749914in}}{\pgfqpoint{1.830547in}{2.754304in}}{\pgfqpoint{1.838361in}{2.762118in}}%
\pgfpathcurveto{\pgfqpoint{1.846175in}{2.769932in}}{\pgfqpoint{1.850565in}{2.780531in}}{\pgfqpoint{1.850565in}{2.791581in}}%
\pgfpathcurveto{\pgfqpoint{1.850565in}{2.802631in}}{\pgfqpoint{1.846175in}{2.813230in}}{\pgfqpoint{1.838361in}{2.821043in}}%
\pgfpathcurveto{\pgfqpoint{1.830547in}{2.828857in}}{\pgfqpoint{1.819948in}{2.833247in}}{\pgfqpoint{1.808898in}{2.833247in}}%
\pgfpathcurveto{\pgfqpoint{1.797848in}{2.833247in}}{\pgfqpoint{1.787249in}{2.828857in}}{\pgfqpoint{1.779435in}{2.821043in}}%
\pgfpathcurveto{\pgfqpoint{1.771622in}{2.813230in}}{\pgfqpoint{1.767232in}{2.802631in}}{\pgfqpoint{1.767232in}{2.791581in}}%
\pgfpathcurveto{\pgfqpoint{1.767232in}{2.780531in}}{\pgfqpoint{1.771622in}{2.769932in}}{\pgfqpoint{1.779435in}{2.762118in}}%
\pgfpathcurveto{\pgfqpoint{1.787249in}{2.754304in}}{\pgfqpoint{1.797848in}{2.749914in}}{\pgfqpoint{1.808898in}{2.749914in}}%
\pgfpathclose%
\pgfusepath{stroke,fill}%
\end{pgfscope}%
\begin{pgfscope}%
\pgfpathrectangle{\pgfqpoint{0.511823in}{0.504323in}}{\pgfqpoint{3.218177in}{3.225677in}} %
\pgfusepath{clip}%
\pgfsetbuttcap%
\pgfsetroundjoin%
\definecolor{currentfill}{rgb}{0.501961,0.000000,0.000000}%
\pgfsetfillcolor{currentfill}%
\pgfsetfillopacity{0.400000}%
\pgfsetlinewidth{0.501875pt}%
\definecolor{currentstroke}{rgb}{0.501961,0.000000,0.000000}%
\pgfsetstrokecolor{currentstroke}%
\pgfsetstrokeopacity{0.400000}%
\pgfsetdash{}{0pt}%
\pgfpathmoveto{\pgfqpoint{1.792922in}{2.736455in}}%
\pgfpathcurveto{\pgfqpoint{1.803972in}{2.736455in}}{\pgfqpoint{1.814571in}{2.740845in}}{\pgfqpoint{1.822384in}{2.748659in}}%
\pgfpathcurveto{\pgfqpoint{1.830198in}{2.756472in}}{\pgfqpoint{1.834588in}{2.767072in}}{\pgfqpoint{1.834588in}{2.778122in}}%
\pgfpathcurveto{\pgfqpoint{1.834588in}{2.789172in}}{\pgfqpoint{1.830198in}{2.799771in}}{\pgfqpoint{1.822384in}{2.807584in}}%
\pgfpathcurveto{\pgfqpoint{1.814571in}{2.815398in}}{\pgfqpoint{1.803972in}{2.819788in}}{\pgfqpoint{1.792922in}{2.819788in}}%
\pgfpathcurveto{\pgfqpoint{1.781871in}{2.819788in}}{\pgfqpoint{1.771272in}{2.815398in}}{\pgfqpoint{1.763459in}{2.807584in}}%
\pgfpathcurveto{\pgfqpoint{1.755645in}{2.799771in}}{\pgfqpoint{1.751255in}{2.789172in}}{\pgfqpoint{1.751255in}{2.778122in}}%
\pgfpathcurveto{\pgfqpoint{1.751255in}{2.767072in}}{\pgfqpoint{1.755645in}{2.756472in}}{\pgfqpoint{1.763459in}{2.748659in}}%
\pgfpathcurveto{\pgfqpoint{1.771272in}{2.740845in}}{\pgfqpoint{1.781871in}{2.736455in}}{\pgfqpoint{1.792922in}{2.736455in}}%
\pgfpathclose%
\pgfusepath{stroke,fill}%
\end{pgfscope}%
\begin{pgfscope}%
\pgfpathrectangle{\pgfqpoint{0.511823in}{0.504323in}}{\pgfqpoint{3.218177in}{3.225677in}} %
\pgfusepath{clip}%
\pgfsetbuttcap%
\pgfsetroundjoin%
\definecolor{currentfill}{rgb}{0.501961,0.000000,0.000000}%
\pgfsetfillcolor{currentfill}%
\pgfsetfillopacity{0.400000}%
\pgfsetlinewidth{0.501875pt}%
\definecolor{currentstroke}{rgb}{0.501961,0.000000,0.000000}%
\pgfsetstrokecolor{currentstroke}%
\pgfsetstrokeopacity{0.400000}%
\pgfsetdash{}{0pt}%
\pgfpathmoveto{\pgfqpoint{1.912962in}{2.982534in}}%
\pgfpathcurveto{\pgfqpoint{1.924012in}{2.982534in}}{\pgfqpoint{1.934612in}{2.986924in}}{\pgfqpoint{1.942425in}{2.994738in}}%
\pgfpathcurveto{\pgfqpoint{1.950239in}{3.002552in}}{\pgfqpoint{1.954629in}{3.013151in}}{\pgfqpoint{1.954629in}{3.024201in}}%
\pgfpathcurveto{\pgfqpoint{1.954629in}{3.035251in}}{\pgfqpoint{1.950239in}{3.045850in}}{\pgfqpoint{1.942425in}{3.053664in}}%
\pgfpathcurveto{\pgfqpoint{1.934612in}{3.061477in}}{\pgfqpoint{1.924012in}{3.065868in}}{\pgfqpoint{1.912962in}{3.065868in}}%
\pgfpathcurveto{\pgfqpoint{1.901912in}{3.065868in}}{\pgfqpoint{1.891313in}{3.061477in}}{\pgfqpoint{1.883500in}{3.053664in}}%
\pgfpathcurveto{\pgfqpoint{1.875686in}{3.045850in}}{\pgfqpoint{1.871296in}{3.035251in}}{\pgfqpoint{1.871296in}{3.024201in}}%
\pgfpathcurveto{\pgfqpoint{1.871296in}{3.013151in}}{\pgfqpoint{1.875686in}{3.002552in}}{\pgfqpoint{1.883500in}{2.994738in}}%
\pgfpathcurveto{\pgfqpoint{1.891313in}{2.986924in}}{\pgfqpoint{1.901912in}{2.982534in}}{\pgfqpoint{1.912962in}{2.982534in}}%
\pgfpathclose%
\pgfusepath{stroke,fill}%
\end{pgfscope}%
\begin{pgfscope}%
\pgfpathrectangle{\pgfqpoint{0.511823in}{0.504323in}}{\pgfqpoint{3.218177in}{3.225677in}} %
\pgfusepath{clip}%
\pgfsetbuttcap%
\pgfsetroundjoin%
\definecolor{currentfill}{rgb}{0.501961,0.000000,0.000000}%
\pgfsetfillcolor{currentfill}%
\pgfsetfillopacity{0.400000}%
\pgfsetlinewidth{0.501875pt}%
\definecolor{currentstroke}{rgb}{0.501961,0.000000,0.000000}%
\pgfsetstrokecolor{currentstroke}%
\pgfsetstrokeopacity{0.400000}%
\pgfsetdash{}{0pt}%
\pgfpathmoveto{\pgfqpoint{1.864690in}{2.908441in}}%
\pgfpathcurveto{\pgfqpoint{1.875740in}{2.908441in}}{\pgfqpoint{1.886339in}{2.912831in}}{\pgfqpoint{1.894153in}{2.920645in}}%
\pgfpathcurveto{\pgfqpoint{1.901966in}{2.928458in}}{\pgfqpoint{1.906356in}{2.939057in}}{\pgfqpoint{1.906356in}{2.950107in}}%
\pgfpathcurveto{\pgfqpoint{1.906356in}{2.961158in}}{\pgfqpoint{1.901966in}{2.971757in}}{\pgfqpoint{1.894153in}{2.979570in}}%
\pgfpathcurveto{\pgfqpoint{1.886339in}{2.987384in}}{\pgfqpoint{1.875740in}{2.991774in}}{\pgfqpoint{1.864690in}{2.991774in}}%
\pgfpathcurveto{\pgfqpoint{1.853640in}{2.991774in}}{\pgfqpoint{1.843041in}{2.987384in}}{\pgfqpoint{1.835227in}{2.979570in}}%
\pgfpathcurveto{\pgfqpoint{1.827413in}{2.971757in}}{\pgfqpoint{1.823023in}{2.961158in}}{\pgfqpoint{1.823023in}{2.950107in}}%
\pgfpathcurveto{\pgfqpoint{1.823023in}{2.939057in}}{\pgfqpoint{1.827413in}{2.928458in}}{\pgfqpoint{1.835227in}{2.920645in}}%
\pgfpathcurveto{\pgfqpoint{1.843041in}{2.912831in}}{\pgfqpoint{1.853640in}{2.908441in}}{\pgfqpoint{1.864690in}{2.908441in}}%
\pgfpathclose%
\pgfusepath{stroke,fill}%
\end{pgfscope}%
\begin{pgfscope}%
\pgfpathrectangle{\pgfqpoint{0.511823in}{0.504323in}}{\pgfqpoint{3.218177in}{3.225677in}} %
\pgfusepath{clip}%
\pgfsetbuttcap%
\pgfsetroundjoin%
\definecolor{currentfill}{rgb}{0.501961,0.000000,0.000000}%
\pgfsetfillcolor{currentfill}%
\pgfsetfillopacity{0.400000}%
\pgfsetlinewidth{0.501875pt}%
\definecolor{currentstroke}{rgb}{0.501961,0.000000,0.000000}%
\pgfsetstrokecolor{currentstroke}%
\pgfsetstrokeopacity{0.400000}%
\pgfsetdash{}{0pt}%
\pgfpathmoveto{\pgfqpoint{1.860868in}{2.919371in}}%
\pgfpathcurveto{\pgfqpoint{1.871918in}{2.919371in}}{\pgfqpoint{1.882517in}{2.923761in}}{\pgfqpoint{1.890330in}{2.931575in}}%
\pgfpathcurveto{\pgfqpoint{1.898144in}{2.939389in}}{\pgfqpoint{1.902534in}{2.949988in}}{\pgfqpoint{1.902534in}{2.961038in}}%
\pgfpathcurveto{\pgfqpoint{1.902534in}{2.972088in}}{\pgfqpoint{1.898144in}{2.982687in}}{\pgfqpoint{1.890330in}{2.990501in}}%
\pgfpathcurveto{\pgfqpoint{1.882517in}{2.998314in}}{\pgfqpoint{1.871918in}{3.002704in}}{\pgfqpoint{1.860868in}{3.002704in}}%
\pgfpathcurveto{\pgfqpoint{1.849818in}{3.002704in}}{\pgfqpoint{1.839219in}{2.998314in}}{\pgfqpoint{1.831405in}{2.990501in}}%
\pgfpathcurveto{\pgfqpoint{1.823591in}{2.982687in}}{\pgfqpoint{1.819201in}{2.972088in}}{\pgfqpoint{1.819201in}{2.961038in}}%
\pgfpathcurveto{\pgfqpoint{1.819201in}{2.949988in}}{\pgfqpoint{1.823591in}{2.939389in}}{\pgfqpoint{1.831405in}{2.931575in}}%
\pgfpathcurveto{\pgfqpoint{1.839219in}{2.923761in}}{\pgfqpoint{1.849818in}{2.919371in}}{\pgfqpoint{1.860868in}{2.919371in}}%
\pgfpathclose%
\pgfusepath{stroke,fill}%
\end{pgfscope}%
\begin{pgfscope}%
\pgfpathrectangle{\pgfqpoint{0.511823in}{0.504323in}}{\pgfqpoint{3.218177in}{3.225677in}} %
\pgfusepath{clip}%
\pgfsetbuttcap%
\pgfsetroundjoin%
\definecolor{currentfill}{rgb}{0.501961,0.000000,0.000000}%
\pgfsetfillcolor{currentfill}%
\pgfsetfillopacity{0.400000}%
\pgfsetlinewidth{0.501875pt}%
\definecolor{currentstroke}{rgb}{0.501961,0.000000,0.000000}%
\pgfsetstrokecolor{currentstroke}%
\pgfsetstrokeopacity{0.400000}%
\pgfsetdash{}{0pt}%
\pgfpathmoveto{\pgfqpoint{1.750897in}{2.722710in}}%
\pgfpathcurveto{\pgfqpoint{1.761947in}{2.722710in}}{\pgfqpoint{1.772546in}{2.727101in}}{\pgfqpoint{1.780360in}{2.734914in}}%
\pgfpathcurveto{\pgfqpoint{1.788173in}{2.742728in}}{\pgfqpoint{1.792564in}{2.753327in}}{\pgfqpoint{1.792564in}{2.764377in}}%
\pgfpathcurveto{\pgfqpoint{1.792564in}{2.775427in}}{\pgfqpoint{1.788173in}{2.786026in}}{\pgfqpoint{1.780360in}{2.793840in}}%
\pgfpathcurveto{\pgfqpoint{1.772546in}{2.801653in}}{\pgfqpoint{1.761947in}{2.806044in}}{\pgfqpoint{1.750897in}{2.806044in}}%
\pgfpathcurveto{\pgfqpoint{1.739847in}{2.806044in}}{\pgfqpoint{1.729248in}{2.801653in}}{\pgfqpoint{1.721434in}{2.793840in}}%
\pgfpathcurveto{\pgfqpoint{1.713621in}{2.786026in}}{\pgfqpoint{1.709230in}{2.775427in}}{\pgfqpoint{1.709230in}{2.764377in}}%
\pgfpathcurveto{\pgfqpoint{1.709230in}{2.753327in}}{\pgfqpoint{1.713621in}{2.742728in}}{\pgfqpoint{1.721434in}{2.734914in}}%
\pgfpathcurveto{\pgfqpoint{1.729248in}{2.727101in}}{\pgfqpoint{1.739847in}{2.722710in}}{\pgfqpoint{1.750897in}{2.722710in}}%
\pgfpathclose%
\pgfusepath{stroke,fill}%
\end{pgfscope}%
\begin{pgfscope}%
\pgfpathrectangle{\pgfqpoint{0.511823in}{0.504323in}}{\pgfqpoint{3.218177in}{3.225677in}} %
\pgfusepath{clip}%
\pgfsetbuttcap%
\pgfsetroundjoin%
\definecolor{currentfill}{rgb}{0.501961,0.000000,0.000000}%
\pgfsetfillcolor{currentfill}%
\pgfsetfillopacity{0.400000}%
\pgfsetlinewidth{0.501875pt}%
\definecolor{currentstroke}{rgb}{0.501961,0.000000,0.000000}%
\pgfsetstrokecolor{currentstroke}%
\pgfsetstrokeopacity{0.400000}%
\pgfsetdash{}{0pt}%
\pgfpathmoveto{\pgfqpoint{1.853625in}{2.942394in}}%
\pgfpathcurveto{\pgfqpoint{1.864676in}{2.942394in}}{\pgfqpoint{1.875275in}{2.946784in}}{\pgfqpoint{1.883088in}{2.954597in}}%
\pgfpathcurveto{\pgfqpoint{1.890902in}{2.962411in}}{\pgfqpoint{1.895292in}{2.973010in}}{\pgfqpoint{1.895292in}{2.984060in}}%
\pgfpathcurveto{\pgfqpoint{1.895292in}{2.995110in}}{\pgfqpoint{1.890902in}{3.005709in}}{\pgfqpoint{1.883088in}{3.013523in}}%
\pgfpathcurveto{\pgfqpoint{1.875275in}{3.021337in}}{\pgfqpoint{1.864676in}{3.025727in}}{\pgfqpoint{1.853625in}{3.025727in}}%
\pgfpathcurveto{\pgfqpoint{1.842575in}{3.025727in}}{\pgfqpoint{1.831976in}{3.021337in}}{\pgfqpoint{1.824163in}{3.013523in}}%
\pgfpathcurveto{\pgfqpoint{1.816349in}{3.005709in}}{\pgfqpoint{1.811959in}{2.995110in}}{\pgfqpoint{1.811959in}{2.984060in}}%
\pgfpathcurveto{\pgfqpoint{1.811959in}{2.973010in}}{\pgfqpoint{1.816349in}{2.962411in}}{\pgfqpoint{1.824163in}{2.954597in}}%
\pgfpathcurveto{\pgfqpoint{1.831976in}{2.946784in}}{\pgfqpoint{1.842575in}{2.942394in}}{\pgfqpoint{1.853625in}{2.942394in}}%
\pgfpathclose%
\pgfusepath{stroke,fill}%
\end{pgfscope}%
\begin{pgfscope}%
\pgfpathrectangle{\pgfqpoint{0.511823in}{0.504323in}}{\pgfqpoint{3.218177in}{3.225677in}} %
\pgfusepath{clip}%
\pgfsetbuttcap%
\pgfsetroundjoin%
\definecolor{currentfill}{rgb}{0.501961,0.000000,0.000000}%
\pgfsetfillcolor{currentfill}%
\pgfsetfillopacity{0.400000}%
\pgfsetlinewidth{0.501875pt}%
\definecolor{currentstroke}{rgb}{0.501961,0.000000,0.000000}%
\pgfsetstrokecolor{currentstroke}%
\pgfsetstrokeopacity{0.400000}%
\pgfsetdash{}{0pt}%
\pgfpathmoveto{\pgfqpoint{1.832443in}{2.919174in}}%
\pgfpathcurveto{\pgfqpoint{1.843493in}{2.919174in}}{\pgfqpoint{1.854092in}{2.923565in}}{\pgfqpoint{1.861905in}{2.931378in}}%
\pgfpathcurveto{\pgfqpoint{1.869719in}{2.939192in}}{\pgfqpoint{1.874109in}{2.949791in}}{\pgfqpoint{1.874109in}{2.960841in}}%
\pgfpathcurveto{\pgfqpoint{1.874109in}{2.971891in}}{\pgfqpoint{1.869719in}{2.982490in}}{\pgfqpoint{1.861905in}{2.990304in}}%
\pgfpathcurveto{\pgfqpoint{1.854092in}{2.998117in}}{\pgfqpoint{1.843493in}{3.002508in}}{\pgfqpoint{1.832443in}{3.002508in}}%
\pgfpathcurveto{\pgfqpoint{1.821393in}{3.002508in}}{\pgfqpoint{1.810794in}{2.998117in}}{\pgfqpoint{1.802980in}{2.990304in}}%
\pgfpathcurveto{\pgfqpoint{1.795166in}{2.982490in}}{\pgfqpoint{1.790776in}{2.971891in}}{\pgfqpoint{1.790776in}{2.960841in}}%
\pgfpathcurveto{\pgfqpoint{1.790776in}{2.949791in}}{\pgfqpoint{1.795166in}{2.939192in}}{\pgfqpoint{1.802980in}{2.931378in}}%
\pgfpathcurveto{\pgfqpoint{1.810794in}{2.923565in}}{\pgfqpoint{1.821393in}{2.919174in}}{\pgfqpoint{1.832443in}{2.919174in}}%
\pgfpathclose%
\pgfusepath{stroke,fill}%
\end{pgfscope}%
\begin{pgfscope}%
\pgfpathrectangle{\pgfqpoint{0.511823in}{0.504323in}}{\pgfqpoint{3.218177in}{3.225677in}} %
\pgfusepath{clip}%
\pgfsetbuttcap%
\pgfsetroundjoin%
\definecolor{currentfill}{rgb}{0.501961,0.000000,0.000000}%
\pgfsetfillcolor{currentfill}%
\pgfsetfillopacity{0.400000}%
\pgfsetlinewidth{0.501875pt}%
\definecolor{currentstroke}{rgb}{0.501961,0.000000,0.000000}%
\pgfsetstrokecolor{currentstroke}%
\pgfsetstrokeopacity{0.400000}%
\pgfsetdash{}{0pt}%
\pgfpathmoveto{\pgfqpoint{1.891518in}{3.056469in}}%
\pgfpathcurveto{\pgfqpoint{1.902569in}{3.056469in}}{\pgfqpoint{1.913168in}{3.060859in}}{\pgfqpoint{1.920981in}{3.068673in}}%
\pgfpathcurveto{\pgfqpoint{1.928795in}{3.076486in}}{\pgfqpoint{1.933185in}{3.087085in}}{\pgfqpoint{1.933185in}{3.098135in}}%
\pgfpathcurveto{\pgfqpoint{1.933185in}{3.109186in}}{\pgfqpoint{1.928795in}{3.119785in}}{\pgfqpoint{1.920981in}{3.127598in}}%
\pgfpathcurveto{\pgfqpoint{1.913168in}{3.135412in}}{\pgfqpoint{1.902569in}{3.139802in}}{\pgfqpoint{1.891518in}{3.139802in}}%
\pgfpathcurveto{\pgfqpoint{1.880468in}{3.139802in}}{\pgfqpoint{1.869869in}{3.135412in}}{\pgfqpoint{1.862056in}{3.127598in}}%
\pgfpathcurveto{\pgfqpoint{1.854242in}{3.119785in}}{\pgfqpoint{1.849852in}{3.109186in}}{\pgfqpoint{1.849852in}{3.098135in}}%
\pgfpathcurveto{\pgfqpoint{1.849852in}{3.087085in}}{\pgfqpoint{1.854242in}{3.076486in}}{\pgfqpoint{1.862056in}{3.068673in}}%
\pgfpathcurveto{\pgfqpoint{1.869869in}{3.060859in}}{\pgfqpoint{1.880468in}{3.056469in}}{\pgfqpoint{1.891518in}{3.056469in}}%
\pgfpathclose%
\pgfusepath{stroke,fill}%
\end{pgfscope}%
\begin{pgfscope}%
\pgfpathrectangle{\pgfqpoint{0.511823in}{0.504323in}}{\pgfqpoint{3.218177in}{3.225677in}} %
\pgfusepath{clip}%
\pgfsetbuttcap%
\pgfsetroundjoin%
\definecolor{currentfill}{rgb}{0.501961,0.000000,0.000000}%
\pgfsetfillcolor{currentfill}%
\pgfsetfillopacity{0.400000}%
\pgfsetlinewidth{0.501875pt}%
\definecolor{currentstroke}{rgb}{0.501961,0.000000,0.000000}%
\pgfsetstrokecolor{currentstroke}%
\pgfsetstrokeopacity{0.400000}%
\pgfsetdash{}{0pt}%
\pgfpathmoveto{\pgfqpoint{1.857943in}{3.008639in}}%
\pgfpathcurveto{\pgfqpoint{1.868993in}{3.008639in}}{\pgfqpoint{1.879592in}{3.013029in}}{\pgfqpoint{1.887406in}{3.020843in}}%
\pgfpathcurveto{\pgfqpoint{1.895220in}{3.028657in}}{\pgfqpoint{1.899610in}{3.039256in}}{\pgfqpoint{1.899610in}{3.050306in}}%
\pgfpathcurveto{\pgfqpoint{1.899610in}{3.061356in}}{\pgfqpoint{1.895220in}{3.071955in}}{\pgfqpoint{1.887406in}{3.079769in}}%
\pgfpathcurveto{\pgfqpoint{1.879592in}{3.087582in}}{\pgfqpoint{1.868993in}{3.091973in}}{\pgfqpoint{1.857943in}{3.091973in}}%
\pgfpathcurveto{\pgfqpoint{1.846893in}{3.091973in}}{\pgfqpoint{1.836294in}{3.087582in}}{\pgfqpoint{1.828480in}{3.079769in}}%
\pgfpathcurveto{\pgfqpoint{1.820667in}{3.071955in}}{\pgfqpoint{1.816276in}{3.061356in}}{\pgfqpoint{1.816276in}{3.050306in}}%
\pgfpathcurveto{\pgfqpoint{1.816276in}{3.039256in}}{\pgfqpoint{1.820667in}{3.028657in}}{\pgfqpoint{1.828480in}{3.020843in}}%
\pgfpathcurveto{\pgfqpoint{1.836294in}{3.013029in}}{\pgfqpoint{1.846893in}{3.008639in}}{\pgfqpoint{1.857943in}{3.008639in}}%
\pgfpathclose%
\pgfusepath{stroke,fill}%
\end{pgfscope}%
\begin{pgfscope}%
\pgfpathrectangle{\pgfqpoint{0.511823in}{0.504323in}}{\pgfqpoint{3.218177in}{3.225677in}} %
\pgfusepath{clip}%
\pgfsetbuttcap%
\pgfsetroundjoin%
\definecolor{currentfill}{rgb}{0.501961,0.000000,0.000000}%
\pgfsetfillcolor{currentfill}%
\pgfsetfillopacity{0.400000}%
\pgfsetlinewidth{0.501875pt}%
\definecolor{currentstroke}{rgb}{0.501961,0.000000,0.000000}%
\pgfsetstrokecolor{currentstroke}%
\pgfsetstrokeopacity{0.400000}%
\pgfsetdash{}{0pt}%
\pgfpathmoveto{\pgfqpoint{1.778181in}{2.865779in}}%
\pgfpathcurveto{\pgfqpoint{1.789231in}{2.865779in}}{\pgfqpoint{1.799830in}{2.870169in}}{\pgfqpoint{1.807644in}{2.877983in}}%
\pgfpathcurveto{\pgfqpoint{1.815457in}{2.885797in}}{\pgfqpoint{1.819847in}{2.896396in}}{\pgfqpoint{1.819847in}{2.907446in}}%
\pgfpathcurveto{\pgfqpoint{1.819847in}{2.918496in}}{\pgfqpoint{1.815457in}{2.929095in}}{\pgfqpoint{1.807644in}{2.936908in}}%
\pgfpathcurveto{\pgfqpoint{1.799830in}{2.944722in}}{\pgfqpoint{1.789231in}{2.949112in}}{\pgfqpoint{1.778181in}{2.949112in}}%
\pgfpathcurveto{\pgfqpoint{1.767131in}{2.949112in}}{\pgfqpoint{1.756532in}{2.944722in}}{\pgfqpoint{1.748718in}{2.936908in}}%
\pgfpathcurveto{\pgfqpoint{1.740904in}{2.929095in}}{\pgfqpoint{1.736514in}{2.918496in}}{\pgfqpoint{1.736514in}{2.907446in}}%
\pgfpathcurveto{\pgfqpoint{1.736514in}{2.896396in}}{\pgfqpoint{1.740904in}{2.885797in}}{\pgfqpoint{1.748718in}{2.877983in}}%
\pgfpathcurveto{\pgfqpoint{1.756532in}{2.870169in}}{\pgfqpoint{1.767131in}{2.865779in}}{\pgfqpoint{1.778181in}{2.865779in}}%
\pgfpathclose%
\pgfusepath{stroke,fill}%
\end{pgfscope}%
\begin{pgfscope}%
\pgfpathrectangle{\pgfqpoint{0.511823in}{0.504323in}}{\pgfqpoint{3.218177in}{3.225677in}} %
\pgfusepath{clip}%
\pgfsetbuttcap%
\pgfsetroundjoin%
\definecolor{currentfill}{rgb}{0.501961,0.000000,0.000000}%
\pgfsetfillcolor{currentfill}%
\pgfsetfillopacity{0.400000}%
\pgfsetlinewidth{0.501875pt}%
\definecolor{currentstroke}{rgb}{0.501961,0.000000,0.000000}%
\pgfsetstrokecolor{currentstroke}%
\pgfsetstrokeopacity{0.400000}%
\pgfsetdash{}{0pt}%
\pgfpathmoveto{\pgfqpoint{1.801746in}{2.932912in}}%
\pgfpathcurveto{\pgfqpoint{1.812796in}{2.932912in}}{\pgfqpoint{1.823395in}{2.937302in}}{\pgfqpoint{1.831208in}{2.945116in}}%
\pgfpathcurveto{\pgfqpoint{1.839022in}{2.952930in}}{\pgfqpoint{1.843412in}{2.963529in}}{\pgfqpoint{1.843412in}{2.974579in}}%
\pgfpathcurveto{\pgfqpoint{1.843412in}{2.985629in}}{\pgfqpoint{1.839022in}{2.996228in}}{\pgfqpoint{1.831208in}{3.004042in}}%
\pgfpathcurveto{\pgfqpoint{1.823395in}{3.011855in}}{\pgfqpoint{1.812796in}{3.016245in}}{\pgfqpoint{1.801746in}{3.016245in}}%
\pgfpathcurveto{\pgfqpoint{1.790695in}{3.016245in}}{\pgfqpoint{1.780096in}{3.011855in}}{\pgfqpoint{1.772283in}{3.004042in}}%
\pgfpathcurveto{\pgfqpoint{1.764469in}{2.996228in}}{\pgfqpoint{1.760079in}{2.985629in}}{\pgfqpoint{1.760079in}{2.974579in}}%
\pgfpathcurveto{\pgfqpoint{1.760079in}{2.963529in}}{\pgfqpoint{1.764469in}{2.952930in}}{\pgfqpoint{1.772283in}{2.945116in}}%
\pgfpathcurveto{\pgfqpoint{1.780096in}{2.937302in}}{\pgfqpoint{1.790695in}{2.932912in}}{\pgfqpoint{1.801746in}{2.932912in}}%
\pgfpathclose%
\pgfusepath{stroke,fill}%
\end{pgfscope}%
\begin{pgfscope}%
\pgfpathrectangle{\pgfqpoint{0.511823in}{0.504323in}}{\pgfqpoint{3.218177in}{3.225677in}} %
\pgfusepath{clip}%
\pgfsetbuttcap%
\pgfsetroundjoin%
\definecolor{currentfill}{rgb}{0.501961,0.000000,0.000000}%
\pgfsetfillcolor{currentfill}%
\pgfsetfillopacity{0.400000}%
\pgfsetlinewidth{0.501875pt}%
\definecolor{currentstroke}{rgb}{0.501961,0.000000,0.000000}%
\pgfsetstrokecolor{currentstroke}%
\pgfsetstrokeopacity{0.400000}%
\pgfsetdash{}{0pt}%
\pgfpathmoveto{\pgfqpoint{1.793303in}{2.934759in}}%
\pgfpathcurveto{\pgfqpoint{1.804353in}{2.934759in}}{\pgfqpoint{1.814952in}{2.939149in}}{\pgfqpoint{1.822765in}{2.946963in}}%
\pgfpathcurveto{\pgfqpoint{1.830579in}{2.954776in}}{\pgfqpoint{1.834969in}{2.965375in}}{\pgfqpoint{1.834969in}{2.976426in}}%
\pgfpathcurveto{\pgfqpoint{1.834969in}{2.987476in}}{\pgfqpoint{1.830579in}{2.998075in}}{\pgfqpoint{1.822765in}{3.005888in}}%
\pgfpathcurveto{\pgfqpoint{1.814952in}{3.013702in}}{\pgfqpoint{1.804353in}{3.018092in}}{\pgfqpoint{1.793303in}{3.018092in}}%
\pgfpathcurveto{\pgfqpoint{1.782253in}{3.018092in}}{\pgfqpoint{1.771653in}{3.013702in}}{\pgfqpoint{1.763840in}{3.005888in}}%
\pgfpathcurveto{\pgfqpoint{1.756026in}{2.998075in}}{\pgfqpoint{1.751636in}{2.987476in}}{\pgfqpoint{1.751636in}{2.976426in}}%
\pgfpathcurveto{\pgfqpoint{1.751636in}{2.965375in}}{\pgfqpoint{1.756026in}{2.954776in}}{\pgfqpoint{1.763840in}{2.946963in}}%
\pgfpathcurveto{\pgfqpoint{1.771653in}{2.939149in}}{\pgfqpoint{1.782253in}{2.934759in}}{\pgfqpoint{1.793303in}{2.934759in}}%
\pgfpathclose%
\pgfusepath{stroke,fill}%
\end{pgfscope}%
\begin{pgfscope}%
\pgfpathrectangle{\pgfqpoint{0.511823in}{0.504323in}}{\pgfqpoint{3.218177in}{3.225677in}} %
\pgfusepath{clip}%
\pgfsetbuttcap%
\pgfsetroundjoin%
\definecolor{currentfill}{rgb}{0.501961,0.000000,0.000000}%
\pgfsetfillcolor{currentfill}%
\pgfsetfillopacity{0.400000}%
\pgfsetlinewidth{0.501875pt}%
\definecolor{currentstroke}{rgb}{0.501961,0.000000,0.000000}%
\pgfsetstrokecolor{currentstroke}%
\pgfsetstrokeopacity{0.400000}%
\pgfsetdash{}{0pt}%
\pgfpathmoveto{\pgfqpoint{1.762875in}{2.890612in}}%
\pgfpathcurveto{\pgfqpoint{1.773926in}{2.890612in}}{\pgfqpoint{1.784525in}{2.895003in}}{\pgfqpoint{1.792338in}{2.902816in}}%
\pgfpathcurveto{\pgfqpoint{1.800152in}{2.910630in}}{\pgfqpoint{1.804542in}{2.921229in}}{\pgfqpoint{1.804542in}{2.932279in}}%
\pgfpathcurveto{\pgfqpoint{1.804542in}{2.943329in}}{\pgfqpoint{1.800152in}{2.953928in}}{\pgfqpoint{1.792338in}{2.961742in}}%
\pgfpathcurveto{\pgfqpoint{1.784525in}{2.969555in}}{\pgfqpoint{1.773926in}{2.973946in}}{\pgfqpoint{1.762875in}{2.973946in}}%
\pgfpathcurveto{\pgfqpoint{1.751825in}{2.973946in}}{\pgfqpoint{1.741226in}{2.969555in}}{\pgfqpoint{1.733413in}{2.961742in}}%
\pgfpathcurveto{\pgfqpoint{1.725599in}{2.953928in}}{\pgfqpoint{1.721209in}{2.943329in}}{\pgfqpoint{1.721209in}{2.932279in}}%
\pgfpathcurveto{\pgfqpoint{1.721209in}{2.921229in}}{\pgfqpoint{1.725599in}{2.910630in}}{\pgfqpoint{1.733413in}{2.902816in}}%
\pgfpathcurveto{\pgfqpoint{1.741226in}{2.895003in}}{\pgfqpoint{1.751825in}{2.890612in}}{\pgfqpoint{1.762875in}{2.890612in}}%
\pgfpathclose%
\pgfusepath{stroke,fill}%
\end{pgfscope}%
\begin{pgfscope}%
\pgfpathrectangle{\pgfqpoint{0.511823in}{0.504323in}}{\pgfqpoint{3.218177in}{3.225677in}} %
\pgfusepath{clip}%
\pgfsetbuttcap%
\pgfsetroundjoin%
\definecolor{currentfill}{rgb}{0.501961,0.000000,0.000000}%
\pgfsetfillcolor{currentfill}%
\pgfsetfillopacity{0.400000}%
\pgfsetlinewidth{0.501875pt}%
\definecolor{currentstroke}{rgb}{0.501961,0.000000,0.000000}%
\pgfsetstrokecolor{currentstroke}%
\pgfsetstrokeopacity{0.400000}%
\pgfsetdash{}{0pt}%
\pgfpathmoveto{\pgfqpoint{1.713577in}{2.805860in}}%
\pgfpathcurveto{\pgfqpoint{1.724627in}{2.805860in}}{\pgfqpoint{1.735226in}{2.810250in}}{\pgfqpoint{1.743040in}{2.818064in}}%
\pgfpathcurveto{\pgfqpoint{1.750853in}{2.825878in}}{\pgfqpoint{1.755243in}{2.836477in}}{\pgfqpoint{1.755243in}{2.847527in}}%
\pgfpathcurveto{\pgfqpoint{1.755243in}{2.858577in}}{\pgfqpoint{1.750853in}{2.869176in}}{\pgfqpoint{1.743040in}{2.876989in}}%
\pgfpathcurveto{\pgfqpoint{1.735226in}{2.884803in}}{\pgfqpoint{1.724627in}{2.889193in}}{\pgfqpoint{1.713577in}{2.889193in}}%
\pgfpathcurveto{\pgfqpoint{1.702527in}{2.889193in}}{\pgfqpoint{1.691928in}{2.884803in}}{\pgfqpoint{1.684114in}{2.876989in}}%
\pgfpathcurveto{\pgfqpoint{1.676300in}{2.869176in}}{\pgfqpoint{1.671910in}{2.858577in}}{\pgfqpoint{1.671910in}{2.847527in}}%
\pgfpathcurveto{\pgfqpoint{1.671910in}{2.836477in}}{\pgfqpoint{1.676300in}{2.825878in}}{\pgfqpoint{1.684114in}{2.818064in}}%
\pgfpathcurveto{\pgfqpoint{1.691928in}{2.810250in}}{\pgfqpoint{1.702527in}{2.805860in}}{\pgfqpoint{1.713577in}{2.805860in}}%
\pgfpathclose%
\pgfusepath{stroke,fill}%
\end{pgfscope}%
\begin{pgfscope}%
\pgfpathrectangle{\pgfqpoint{0.511823in}{0.504323in}}{\pgfqpoint{3.218177in}{3.225677in}} %
\pgfusepath{clip}%
\pgfsetbuttcap%
\pgfsetroundjoin%
\definecolor{currentfill}{rgb}{0.501961,0.000000,0.000000}%
\pgfsetfillcolor{currentfill}%
\pgfsetfillopacity{0.400000}%
\pgfsetlinewidth{0.501875pt}%
\definecolor{currentstroke}{rgb}{0.501961,0.000000,0.000000}%
\pgfsetstrokecolor{currentstroke}%
\pgfsetstrokeopacity{0.400000}%
\pgfsetdash{}{0pt}%
\pgfpathmoveto{\pgfqpoint{1.800716in}{3.009799in}}%
\pgfpathcurveto{\pgfqpoint{1.811766in}{3.009799in}}{\pgfqpoint{1.822365in}{3.014189in}}{\pgfqpoint{1.830179in}{3.022003in}}%
\pgfpathcurveto{\pgfqpoint{1.837993in}{3.029816in}}{\pgfqpoint{1.842383in}{3.040415in}}{\pgfqpoint{1.842383in}{3.051465in}}%
\pgfpathcurveto{\pgfqpoint{1.842383in}{3.062516in}}{\pgfqpoint{1.837993in}{3.073115in}}{\pgfqpoint{1.830179in}{3.080928in}}%
\pgfpathcurveto{\pgfqpoint{1.822365in}{3.088742in}}{\pgfqpoint{1.811766in}{3.093132in}}{\pgfqpoint{1.800716in}{3.093132in}}%
\pgfpathcurveto{\pgfqpoint{1.789666in}{3.093132in}}{\pgfqpoint{1.779067in}{3.088742in}}{\pgfqpoint{1.771253in}{3.080928in}}%
\pgfpathcurveto{\pgfqpoint{1.763440in}{3.073115in}}{\pgfqpoint{1.759050in}{3.062516in}}{\pgfqpoint{1.759050in}{3.051465in}}%
\pgfpathcurveto{\pgfqpoint{1.759050in}{3.040415in}}{\pgfqpoint{1.763440in}{3.029816in}}{\pgfqpoint{1.771253in}{3.022003in}}%
\pgfpathcurveto{\pgfqpoint{1.779067in}{3.014189in}}{\pgfqpoint{1.789666in}{3.009799in}}{\pgfqpoint{1.800716in}{3.009799in}}%
\pgfpathclose%
\pgfusepath{stroke,fill}%
\end{pgfscope}%
\begin{pgfscope}%
\pgfpathrectangle{\pgfqpoint{0.511823in}{0.504323in}}{\pgfqpoint{3.218177in}{3.225677in}} %
\pgfusepath{clip}%
\pgfsetbuttcap%
\pgfsetroundjoin%
\definecolor{currentfill}{rgb}{0.501961,0.000000,0.000000}%
\pgfsetfillcolor{currentfill}%
\pgfsetfillopacity{0.400000}%
\pgfsetlinewidth{0.501875pt}%
\definecolor{currentstroke}{rgb}{0.501961,0.000000,0.000000}%
\pgfsetstrokecolor{currentstroke}%
\pgfsetstrokeopacity{0.400000}%
\pgfsetdash{}{0pt}%
\pgfpathmoveto{\pgfqpoint{1.864602in}{3.167320in}}%
\pgfpathcurveto{\pgfqpoint{1.875652in}{3.167320in}}{\pgfqpoint{1.886251in}{3.171710in}}{\pgfqpoint{1.894064in}{3.179524in}}%
\pgfpathcurveto{\pgfqpoint{1.901878in}{3.187338in}}{\pgfqpoint{1.906268in}{3.197937in}}{\pgfqpoint{1.906268in}{3.208987in}}%
\pgfpathcurveto{\pgfqpoint{1.906268in}{3.220037in}}{\pgfqpoint{1.901878in}{3.230636in}}{\pgfqpoint{1.894064in}{3.238449in}}%
\pgfpathcurveto{\pgfqpoint{1.886251in}{3.246263in}}{\pgfqpoint{1.875652in}{3.250653in}}{\pgfqpoint{1.864602in}{3.250653in}}%
\pgfpathcurveto{\pgfqpoint{1.853551in}{3.250653in}}{\pgfqpoint{1.842952in}{3.246263in}}{\pgfqpoint{1.835139in}{3.238449in}}%
\pgfpathcurveto{\pgfqpoint{1.827325in}{3.230636in}}{\pgfqpoint{1.822935in}{3.220037in}}{\pgfqpoint{1.822935in}{3.208987in}}%
\pgfpathcurveto{\pgfqpoint{1.822935in}{3.197937in}}{\pgfqpoint{1.827325in}{3.187338in}}{\pgfqpoint{1.835139in}{3.179524in}}%
\pgfpathcurveto{\pgfqpoint{1.842952in}{3.171710in}}{\pgfqpoint{1.853551in}{3.167320in}}{\pgfqpoint{1.864602in}{3.167320in}}%
\pgfpathclose%
\pgfusepath{stroke,fill}%
\end{pgfscope}%
\begin{pgfscope}%
\pgfpathrectangle{\pgfqpoint{0.511823in}{0.504323in}}{\pgfqpoint{3.218177in}{3.225677in}} %
\pgfusepath{clip}%
\pgfsetbuttcap%
\pgfsetroundjoin%
\definecolor{currentfill}{rgb}{0.501961,0.000000,0.000000}%
\pgfsetfillcolor{currentfill}%
\pgfsetfillopacity{0.400000}%
\pgfsetlinewidth{0.501875pt}%
\definecolor{currentstroke}{rgb}{0.501961,0.000000,0.000000}%
\pgfsetstrokecolor{currentstroke}%
\pgfsetstrokeopacity{0.400000}%
\pgfsetdash{}{0pt}%
\pgfpathmoveto{\pgfqpoint{1.732359in}{2.903106in}}%
\pgfpathcurveto{\pgfqpoint{1.743409in}{2.903106in}}{\pgfqpoint{1.754008in}{2.907496in}}{\pgfqpoint{1.761822in}{2.915310in}}%
\pgfpathcurveto{\pgfqpoint{1.769635in}{2.923123in}}{\pgfqpoint{1.774026in}{2.933722in}}{\pgfqpoint{1.774026in}{2.944773in}}%
\pgfpathcurveto{\pgfqpoint{1.774026in}{2.955823in}}{\pgfqpoint{1.769635in}{2.966422in}}{\pgfqpoint{1.761822in}{2.974235in}}%
\pgfpathcurveto{\pgfqpoint{1.754008in}{2.982049in}}{\pgfqpoint{1.743409in}{2.986439in}}{\pgfqpoint{1.732359in}{2.986439in}}%
\pgfpathcurveto{\pgfqpoint{1.721309in}{2.986439in}}{\pgfqpoint{1.710710in}{2.982049in}}{\pgfqpoint{1.702896in}{2.974235in}}%
\pgfpathcurveto{\pgfqpoint{1.695083in}{2.966422in}}{\pgfqpoint{1.690692in}{2.955823in}}{\pgfqpoint{1.690692in}{2.944773in}}%
\pgfpathcurveto{\pgfqpoint{1.690692in}{2.933722in}}{\pgfqpoint{1.695083in}{2.923123in}}{\pgfqpoint{1.702896in}{2.915310in}}%
\pgfpathcurveto{\pgfqpoint{1.710710in}{2.907496in}}{\pgfqpoint{1.721309in}{2.903106in}}{\pgfqpoint{1.732359in}{2.903106in}}%
\pgfpathclose%
\pgfusepath{stroke,fill}%
\end{pgfscope}%
\begin{pgfscope}%
\pgfpathrectangle{\pgfqpoint{0.511823in}{0.504323in}}{\pgfqpoint{3.218177in}{3.225677in}} %
\pgfusepath{clip}%
\pgfsetbuttcap%
\pgfsetroundjoin%
\definecolor{currentfill}{rgb}{0.501961,0.000000,0.000000}%
\pgfsetfillcolor{currentfill}%
\pgfsetfillopacity{0.400000}%
\pgfsetlinewidth{0.501875pt}%
\definecolor{currentstroke}{rgb}{0.501961,0.000000,0.000000}%
\pgfsetstrokecolor{currentstroke}%
\pgfsetstrokeopacity{0.400000}%
\pgfsetdash{}{0pt}%
\pgfpathmoveto{\pgfqpoint{1.815331in}{3.104002in}}%
\pgfpathcurveto{\pgfqpoint{1.826381in}{3.104002in}}{\pgfqpoint{1.836980in}{3.108392in}}{\pgfqpoint{1.844793in}{3.116206in}}%
\pgfpathcurveto{\pgfqpoint{1.852607in}{3.124019in}}{\pgfqpoint{1.856997in}{3.134618in}}{\pgfqpoint{1.856997in}{3.145669in}}%
\pgfpathcurveto{\pgfqpoint{1.856997in}{3.156719in}}{\pgfqpoint{1.852607in}{3.167318in}}{\pgfqpoint{1.844793in}{3.175131in}}%
\pgfpathcurveto{\pgfqpoint{1.836980in}{3.182945in}}{\pgfqpoint{1.826381in}{3.187335in}}{\pgfqpoint{1.815331in}{3.187335in}}%
\pgfpathcurveto{\pgfqpoint{1.804281in}{3.187335in}}{\pgfqpoint{1.793681in}{3.182945in}}{\pgfqpoint{1.785868in}{3.175131in}}%
\pgfpathcurveto{\pgfqpoint{1.778054in}{3.167318in}}{\pgfqpoint{1.773664in}{3.156719in}}{\pgfqpoint{1.773664in}{3.145669in}}%
\pgfpathcurveto{\pgfqpoint{1.773664in}{3.134618in}}{\pgfqpoint{1.778054in}{3.124019in}}{\pgfqpoint{1.785868in}{3.116206in}}%
\pgfpathcurveto{\pgfqpoint{1.793681in}{3.108392in}}{\pgfqpoint{1.804281in}{3.104002in}}{\pgfqpoint{1.815331in}{3.104002in}}%
\pgfpathclose%
\pgfusepath{stroke,fill}%
\end{pgfscope}%
\begin{pgfscope}%
\pgfpathrectangle{\pgfqpoint{0.511823in}{0.504323in}}{\pgfqpoint{3.218177in}{3.225677in}} %
\pgfusepath{clip}%
\pgfsetbuttcap%
\pgfsetroundjoin%
\definecolor{currentfill}{rgb}{0.501961,0.000000,0.000000}%
\pgfsetfillcolor{currentfill}%
\pgfsetfillopacity{0.400000}%
\pgfsetlinewidth{0.501875pt}%
\definecolor{currentstroke}{rgb}{0.501961,0.000000,0.000000}%
\pgfsetstrokecolor{currentstroke}%
\pgfsetstrokeopacity{0.400000}%
\pgfsetdash{}{0pt}%
\pgfpathmoveto{\pgfqpoint{1.705111in}{2.882955in}}%
\pgfpathcurveto{\pgfqpoint{1.716161in}{2.882955in}}{\pgfqpoint{1.726760in}{2.887345in}}{\pgfqpoint{1.734574in}{2.895159in}}%
\pgfpathcurveto{\pgfqpoint{1.742387in}{2.902972in}}{\pgfqpoint{1.746778in}{2.913571in}}{\pgfqpoint{1.746778in}{2.924622in}}%
\pgfpathcurveto{\pgfqpoint{1.746778in}{2.935672in}}{\pgfqpoint{1.742387in}{2.946271in}}{\pgfqpoint{1.734574in}{2.954084in}}%
\pgfpathcurveto{\pgfqpoint{1.726760in}{2.961898in}}{\pgfqpoint{1.716161in}{2.966288in}}{\pgfqpoint{1.705111in}{2.966288in}}%
\pgfpathcurveto{\pgfqpoint{1.694061in}{2.966288in}}{\pgfqpoint{1.683462in}{2.961898in}}{\pgfqpoint{1.675648in}{2.954084in}}%
\pgfpathcurveto{\pgfqpoint{1.667835in}{2.946271in}}{\pgfqpoint{1.663444in}{2.935672in}}{\pgfqpoint{1.663444in}{2.924622in}}%
\pgfpathcurveto{\pgfqpoint{1.663444in}{2.913571in}}{\pgfqpoint{1.667835in}{2.902972in}}{\pgfqpoint{1.675648in}{2.895159in}}%
\pgfpathcurveto{\pgfqpoint{1.683462in}{2.887345in}}{\pgfqpoint{1.694061in}{2.882955in}}{\pgfqpoint{1.705111in}{2.882955in}}%
\pgfpathclose%
\pgfusepath{stroke,fill}%
\end{pgfscope}%
\begin{pgfscope}%
\pgfpathrectangle{\pgfqpoint{0.511823in}{0.504323in}}{\pgfqpoint{3.218177in}{3.225677in}} %
\pgfusepath{clip}%
\pgfsetbuttcap%
\pgfsetroundjoin%
\definecolor{currentfill}{rgb}{0.501961,0.000000,0.000000}%
\pgfsetfillcolor{currentfill}%
\pgfsetfillopacity{0.400000}%
\pgfsetlinewidth{0.501875pt}%
\definecolor{currentstroke}{rgb}{0.501961,0.000000,0.000000}%
\pgfsetstrokecolor{currentstroke}%
\pgfsetstrokeopacity{0.400000}%
\pgfsetdash{}{0pt}%
\pgfpathmoveto{\pgfqpoint{1.696566in}{2.883815in}}%
\pgfpathcurveto{\pgfqpoint{1.707616in}{2.883815in}}{\pgfqpoint{1.718215in}{2.888206in}}{\pgfqpoint{1.726029in}{2.896019in}}%
\pgfpathcurveto{\pgfqpoint{1.733843in}{2.903833in}}{\pgfqpoint{1.738233in}{2.914432in}}{\pgfqpoint{1.738233in}{2.925482in}}%
\pgfpathcurveto{\pgfqpoint{1.738233in}{2.936532in}}{\pgfqpoint{1.733843in}{2.947131in}}{\pgfqpoint{1.726029in}{2.954945in}}%
\pgfpathcurveto{\pgfqpoint{1.718215in}{2.962759in}}{\pgfqpoint{1.707616in}{2.967149in}}{\pgfqpoint{1.696566in}{2.967149in}}%
\pgfpathcurveto{\pgfqpoint{1.685516in}{2.967149in}}{\pgfqpoint{1.674917in}{2.962759in}}{\pgfqpoint{1.667103in}{2.954945in}}%
\pgfpathcurveto{\pgfqpoint{1.659290in}{2.947131in}}{\pgfqpoint{1.654900in}{2.936532in}}{\pgfqpoint{1.654900in}{2.925482in}}%
\pgfpathcurveto{\pgfqpoint{1.654900in}{2.914432in}}{\pgfqpoint{1.659290in}{2.903833in}}{\pgfqpoint{1.667103in}{2.896019in}}%
\pgfpathcurveto{\pgfqpoint{1.674917in}{2.888206in}}{\pgfqpoint{1.685516in}{2.883815in}}{\pgfqpoint{1.696566in}{2.883815in}}%
\pgfpathclose%
\pgfusepath{stroke,fill}%
\end{pgfscope}%
\begin{pgfscope}%
\pgfpathrectangle{\pgfqpoint{0.511823in}{0.504323in}}{\pgfqpoint{3.218177in}{3.225677in}} %
\pgfusepath{clip}%
\pgfsetbuttcap%
\pgfsetroundjoin%
\definecolor{currentfill}{rgb}{0.501961,0.000000,0.000000}%
\pgfsetfillcolor{currentfill}%
\pgfsetfillopacity{0.400000}%
\pgfsetlinewidth{0.501875pt}%
\definecolor{currentstroke}{rgb}{0.501961,0.000000,0.000000}%
\pgfsetstrokecolor{currentstroke}%
\pgfsetstrokeopacity{0.400000}%
\pgfsetdash{}{0pt}%
\pgfpathmoveto{\pgfqpoint{1.705012in}{2.922720in}}%
\pgfpathcurveto{\pgfqpoint{1.716062in}{2.922720in}}{\pgfqpoint{1.726661in}{2.927111in}}{\pgfqpoint{1.734475in}{2.934924in}}%
\pgfpathcurveto{\pgfqpoint{1.742288in}{2.942738in}}{\pgfqpoint{1.746679in}{2.953337in}}{\pgfqpoint{1.746679in}{2.964387in}}%
\pgfpathcurveto{\pgfqpoint{1.746679in}{2.975437in}}{\pgfqpoint{1.742288in}{2.986036in}}{\pgfqpoint{1.734475in}{2.993850in}}%
\pgfpathcurveto{\pgfqpoint{1.726661in}{3.001663in}}{\pgfqpoint{1.716062in}{3.006054in}}{\pgfqpoint{1.705012in}{3.006054in}}%
\pgfpathcurveto{\pgfqpoint{1.693962in}{3.006054in}}{\pgfqpoint{1.683363in}{3.001663in}}{\pgfqpoint{1.675549in}{2.993850in}}%
\pgfpathcurveto{\pgfqpoint{1.667736in}{2.986036in}}{\pgfqpoint{1.663345in}{2.975437in}}{\pgfqpoint{1.663345in}{2.964387in}}%
\pgfpathcurveto{\pgfqpoint{1.663345in}{2.953337in}}{\pgfqpoint{1.667736in}{2.942738in}}{\pgfqpoint{1.675549in}{2.934924in}}%
\pgfpathcurveto{\pgfqpoint{1.683363in}{2.927111in}}{\pgfqpoint{1.693962in}{2.922720in}}{\pgfqpoint{1.705012in}{2.922720in}}%
\pgfpathclose%
\pgfusepath{stroke,fill}%
\end{pgfscope}%
\begin{pgfscope}%
\pgfpathrectangle{\pgfqpoint{0.511823in}{0.504323in}}{\pgfqpoint{3.218177in}{3.225677in}} %
\pgfusepath{clip}%
\pgfsetbuttcap%
\pgfsetroundjoin%
\definecolor{currentfill}{rgb}{0.501961,0.000000,0.000000}%
\pgfsetfillcolor{currentfill}%
\pgfsetfillopacity{0.400000}%
\pgfsetlinewidth{0.501875pt}%
\definecolor{currentstroke}{rgb}{0.501961,0.000000,0.000000}%
\pgfsetstrokecolor{currentstroke}%
\pgfsetstrokeopacity{0.400000}%
\pgfsetdash{}{0pt}%
\pgfpathmoveto{\pgfqpoint{1.683559in}{2.894655in}}%
\pgfpathcurveto{\pgfqpoint{1.694609in}{2.894655in}}{\pgfqpoint{1.705208in}{2.899046in}}{\pgfqpoint{1.713022in}{2.906859in}}%
\pgfpathcurveto{\pgfqpoint{1.720835in}{2.914673in}}{\pgfqpoint{1.725225in}{2.925272in}}{\pgfqpoint{1.725225in}{2.936322in}}%
\pgfpathcurveto{\pgfqpoint{1.725225in}{2.947372in}}{\pgfqpoint{1.720835in}{2.957971in}}{\pgfqpoint{1.713022in}{2.965785in}}%
\pgfpathcurveto{\pgfqpoint{1.705208in}{2.973598in}}{\pgfqpoint{1.694609in}{2.977989in}}{\pgfqpoint{1.683559in}{2.977989in}}%
\pgfpathcurveto{\pgfqpoint{1.672509in}{2.977989in}}{\pgfqpoint{1.661910in}{2.973598in}}{\pgfqpoint{1.654096in}{2.965785in}}%
\pgfpathcurveto{\pgfqpoint{1.646282in}{2.957971in}}{\pgfqpoint{1.641892in}{2.947372in}}{\pgfqpoint{1.641892in}{2.936322in}}%
\pgfpathcurveto{\pgfqpoint{1.641892in}{2.925272in}}{\pgfqpoint{1.646282in}{2.914673in}}{\pgfqpoint{1.654096in}{2.906859in}}%
\pgfpathcurveto{\pgfqpoint{1.661910in}{2.899046in}}{\pgfqpoint{1.672509in}{2.894655in}}{\pgfqpoint{1.683559in}{2.894655in}}%
\pgfpathclose%
\pgfusepath{stroke,fill}%
\end{pgfscope}%
\begin{pgfscope}%
\pgfpathrectangle{\pgfqpoint{0.511823in}{0.504323in}}{\pgfqpoint{3.218177in}{3.225677in}} %
\pgfusepath{clip}%
\pgfsetbuttcap%
\pgfsetroundjoin%
\definecolor{currentfill}{rgb}{0.501961,0.000000,0.000000}%
\pgfsetfillcolor{currentfill}%
\pgfsetfillopacity{0.400000}%
\pgfsetlinewidth{0.501875pt}%
\definecolor{currentstroke}{rgb}{0.501961,0.000000,0.000000}%
\pgfsetstrokecolor{currentstroke}%
\pgfsetstrokeopacity{0.400000}%
\pgfsetdash{}{0pt}%
\pgfpathmoveto{\pgfqpoint{1.739747in}{3.043192in}}%
\pgfpathcurveto{\pgfqpoint{1.750797in}{3.043192in}}{\pgfqpoint{1.761396in}{3.047582in}}{\pgfqpoint{1.769209in}{3.055396in}}%
\pgfpathcurveto{\pgfqpoint{1.777023in}{3.063210in}}{\pgfqpoint{1.781413in}{3.073809in}}{\pgfqpoint{1.781413in}{3.084859in}}%
\pgfpathcurveto{\pgfqpoint{1.781413in}{3.095909in}}{\pgfqpoint{1.777023in}{3.106508in}}{\pgfqpoint{1.769209in}{3.114322in}}%
\pgfpathcurveto{\pgfqpoint{1.761396in}{3.122135in}}{\pgfqpoint{1.750797in}{3.126525in}}{\pgfqpoint{1.739747in}{3.126525in}}%
\pgfpathcurveto{\pgfqpoint{1.728696in}{3.126525in}}{\pgfqpoint{1.718097in}{3.122135in}}{\pgfqpoint{1.710284in}{3.114322in}}%
\pgfpathcurveto{\pgfqpoint{1.702470in}{3.106508in}}{\pgfqpoint{1.698080in}{3.095909in}}{\pgfqpoint{1.698080in}{3.084859in}}%
\pgfpathcurveto{\pgfqpoint{1.698080in}{3.073809in}}{\pgfqpoint{1.702470in}{3.063210in}}{\pgfqpoint{1.710284in}{3.055396in}}%
\pgfpathcurveto{\pgfqpoint{1.718097in}{3.047582in}}{\pgfqpoint{1.728696in}{3.043192in}}{\pgfqpoint{1.739747in}{3.043192in}}%
\pgfpathclose%
\pgfusepath{stroke,fill}%
\end{pgfscope}%
\begin{pgfscope}%
\pgfpathrectangle{\pgfqpoint{0.511823in}{0.504323in}}{\pgfqpoint{3.218177in}{3.225677in}} %
\pgfusepath{clip}%
\pgfsetbuttcap%
\pgfsetroundjoin%
\definecolor{currentfill}{rgb}{0.501961,0.000000,0.000000}%
\pgfsetfillcolor{currentfill}%
\pgfsetfillopacity{0.400000}%
\pgfsetlinewidth{0.501875pt}%
\definecolor{currentstroke}{rgb}{0.501961,0.000000,0.000000}%
\pgfsetstrokecolor{currentstroke}%
\pgfsetstrokeopacity{0.400000}%
\pgfsetdash{}{0pt}%
\pgfpathmoveto{\pgfqpoint{1.741032in}{3.067958in}}%
\pgfpathcurveto{\pgfqpoint{1.752082in}{3.067958in}}{\pgfqpoint{1.762681in}{3.072348in}}{\pgfqpoint{1.770494in}{3.080162in}}%
\pgfpathcurveto{\pgfqpoint{1.778308in}{3.087976in}}{\pgfqpoint{1.782698in}{3.098575in}}{\pgfqpoint{1.782698in}{3.109625in}}%
\pgfpathcurveto{\pgfqpoint{1.782698in}{3.120675in}}{\pgfqpoint{1.778308in}{3.131274in}}{\pgfqpoint{1.770494in}{3.139088in}}%
\pgfpathcurveto{\pgfqpoint{1.762681in}{3.146901in}}{\pgfqpoint{1.752082in}{3.151291in}}{\pgfqpoint{1.741032in}{3.151291in}}%
\pgfpathcurveto{\pgfqpoint{1.729981in}{3.151291in}}{\pgfqpoint{1.719382in}{3.146901in}}{\pgfqpoint{1.711569in}{3.139088in}}%
\pgfpathcurveto{\pgfqpoint{1.703755in}{3.131274in}}{\pgfqpoint{1.699365in}{3.120675in}}{\pgfqpoint{1.699365in}{3.109625in}}%
\pgfpathcurveto{\pgfqpoint{1.699365in}{3.098575in}}{\pgfqpoint{1.703755in}{3.087976in}}{\pgfqpoint{1.711569in}{3.080162in}}%
\pgfpathcurveto{\pgfqpoint{1.719382in}{3.072348in}}{\pgfqpoint{1.729981in}{3.067958in}}{\pgfqpoint{1.741032in}{3.067958in}}%
\pgfpathclose%
\pgfusepath{stroke,fill}%
\end{pgfscope}%
\begin{pgfscope}%
\pgfpathrectangle{\pgfqpoint{0.511823in}{0.504323in}}{\pgfqpoint{3.218177in}{3.225677in}} %
\pgfusepath{clip}%
\pgfsetbuttcap%
\pgfsetroundjoin%
\definecolor{currentfill}{rgb}{0.501961,0.000000,0.000000}%
\pgfsetfillcolor{currentfill}%
\pgfsetfillopacity{0.400000}%
\pgfsetlinewidth{0.501875pt}%
\definecolor{currentstroke}{rgb}{0.501961,0.000000,0.000000}%
\pgfsetstrokecolor{currentstroke}%
\pgfsetstrokeopacity{0.400000}%
\pgfsetdash{}{0pt}%
\pgfpathmoveto{\pgfqpoint{1.548406in}{2.642565in}}%
\pgfpathcurveto{\pgfqpoint{1.559457in}{2.642565in}}{\pgfqpoint{1.570056in}{2.646955in}}{\pgfqpoint{1.577869in}{2.654769in}}%
\pgfpathcurveto{\pgfqpoint{1.585683in}{2.662583in}}{\pgfqpoint{1.590073in}{2.673182in}}{\pgfqpoint{1.590073in}{2.684232in}}%
\pgfpathcurveto{\pgfqpoint{1.590073in}{2.695282in}}{\pgfqpoint{1.585683in}{2.705881in}}{\pgfqpoint{1.577869in}{2.713695in}}%
\pgfpathcurveto{\pgfqpoint{1.570056in}{2.721508in}}{\pgfqpoint{1.559457in}{2.725898in}}{\pgfqpoint{1.548406in}{2.725898in}}%
\pgfpathcurveto{\pgfqpoint{1.537356in}{2.725898in}}{\pgfqpoint{1.526757in}{2.721508in}}{\pgfqpoint{1.518944in}{2.713695in}}%
\pgfpathcurveto{\pgfqpoint{1.511130in}{2.705881in}}{\pgfqpoint{1.506740in}{2.695282in}}{\pgfqpoint{1.506740in}{2.684232in}}%
\pgfpathcurveto{\pgfqpoint{1.506740in}{2.673182in}}{\pgfqpoint{1.511130in}{2.662583in}}{\pgfqpoint{1.518944in}{2.654769in}}%
\pgfpathcurveto{\pgfqpoint{1.526757in}{2.646955in}}{\pgfqpoint{1.537356in}{2.642565in}}{\pgfqpoint{1.548406in}{2.642565in}}%
\pgfpathclose%
\pgfusepath{stroke,fill}%
\end{pgfscope}%
\begin{pgfscope}%
\pgfpathrectangle{\pgfqpoint{0.511823in}{0.504323in}}{\pgfqpoint{3.218177in}{3.225677in}} %
\pgfusepath{clip}%
\pgfsetbuttcap%
\pgfsetroundjoin%
\definecolor{currentfill}{rgb}{0.501961,0.000000,0.000000}%
\pgfsetfillcolor{currentfill}%
\pgfsetfillopacity{0.400000}%
\pgfsetlinewidth{0.501875pt}%
\definecolor{currentstroke}{rgb}{0.501961,0.000000,0.000000}%
\pgfsetstrokecolor{currentstroke}%
\pgfsetstrokeopacity{0.400000}%
\pgfsetdash{}{0pt}%
\pgfpathmoveto{\pgfqpoint{1.707595in}{3.034258in}}%
\pgfpathcurveto{\pgfqpoint{1.718645in}{3.034258in}}{\pgfqpoint{1.729244in}{3.038648in}}{\pgfqpoint{1.737058in}{3.046462in}}%
\pgfpathcurveto{\pgfqpoint{1.744872in}{3.054276in}}{\pgfqpoint{1.749262in}{3.064875in}}{\pgfqpoint{1.749262in}{3.075925in}}%
\pgfpathcurveto{\pgfqpoint{1.749262in}{3.086975in}}{\pgfqpoint{1.744872in}{3.097574in}}{\pgfqpoint{1.737058in}{3.105388in}}%
\pgfpathcurveto{\pgfqpoint{1.729244in}{3.113201in}}{\pgfqpoint{1.718645in}{3.117592in}}{\pgfqpoint{1.707595in}{3.117592in}}%
\pgfpathcurveto{\pgfqpoint{1.696545in}{3.117592in}}{\pgfqpoint{1.685946in}{3.113201in}}{\pgfqpoint{1.678132in}{3.105388in}}%
\pgfpathcurveto{\pgfqpoint{1.670319in}{3.097574in}}{\pgfqpoint{1.665928in}{3.086975in}}{\pgfqpoint{1.665928in}{3.075925in}}%
\pgfpathcurveto{\pgfqpoint{1.665928in}{3.064875in}}{\pgfqpoint{1.670319in}{3.054276in}}{\pgfqpoint{1.678132in}{3.046462in}}%
\pgfpathcurveto{\pgfqpoint{1.685946in}{3.038648in}}{\pgfqpoint{1.696545in}{3.034258in}}{\pgfqpoint{1.707595in}{3.034258in}}%
\pgfpathclose%
\pgfusepath{stroke,fill}%
\end{pgfscope}%
\begin{pgfscope}%
\pgfpathrectangle{\pgfqpoint{0.511823in}{0.504323in}}{\pgfqpoint{3.218177in}{3.225677in}} %
\pgfusepath{clip}%
\pgfsetbuttcap%
\pgfsetroundjoin%
\definecolor{currentfill}{rgb}{0.501961,0.000000,0.000000}%
\pgfsetfillcolor{currentfill}%
\pgfsetfillopacity{0.400000}%
\pgfsetlinewidth{0.501875pt}%
\definecolor{currentstroke}{rgb}{0.501961,0.000000,0.000000}%
\pgfsetstrokecolor{currentstroke}%
\pgfsetstrokeopacity{0.400000}%
\pgfsetdash{}{0pt}%
\pgfpathmoveto{\pgfqpoint{1.679045in}{2.988855in}}%
\pgfpathcurveto{\pgfqpoint{1.690096in}{2.988855in}}{\pgfqpoint{1.700695in}{2.993246in}}{\pgfqpoint{1.708508in}{3.001059in}}%
\pgfpathcurveto{\pgfqpoint{1.716322in}{3.008873in}}{\pgfqpoint{1.720712in}{3.019472in}}{\pgfqpoint{1.720712in}{3.030522in}}%
\pgfpathcurveto{\pgfqpoint{1.720712in}{3.041572in}}{\pgfqpoint{1.716322in}{3.052171in}}{\pgfqpoint{1.708508in}{3.059985in}}%
\pgfpathcurveto{\pgfqpoint{1.700695in}{3.067798in}}{\pgfqpoint{1.690096in}{3.072189in}}{\pgfqpoint{1.679045in}{3.072189in}}%
\pgfpathcurveto{\pgfqpoint{1.667995in}{3.072189in}}{\pgfqpoint{1.657396in}{3.067798in}}{\pgfqpoint{1.649583in}{3.059985in}}%
\pgfpathcurveto{\pgfqpoint{1.641769in}{3.052171in}}{\pgfqpoint{1.637379in}{3.041572in}}{\pgfqpoint{1.637379in}{3.030522in}}%
\pgfpathcurveto{\pgfqpoint{1.637379in}{3.019472in}}{\pgfqpoint{1.641769in}{3.008873in}}{\pgfqpoint{1.649583in}{3.001059in}}%
\pgfpathcurveto{\pgfqpoint{1.657396in}{2.993246in}}{\pgfqpoint{1.667995in}{2.988855in}}{\pgfqpoint{1.679045in}{2.988855in}}%
\pgfpathclose%
\pgfusepath{stroke,fill}%
\end{pgfscope}%
\begin{pgfscope}%
\pgfpathrectangle{\pgfqpoint{0.511823in}{0.504323in}}{\pgfqpoint{3.218177in}{3.225677in}} %
\pgfusepath{clip}%
\pgfsetbuttcap%
\pgfsetroundjoin%
\definecolor{currentfill}{rgb}{0.501961,0.000000,0.000000}%
\pgfsetfillcolor{currentfill}%
\pgfsetfillopacity{0.400000}%
\pgfsetlinewidth{0.501875pt}%
\definecolor{currentstroke}{rgb}{0.501961,0.000000,0.000000}%
\pgfsetstrokecolor{currentstroke}%
\pgfsetstrokeopacity{0.400000}%
\pgfsetdash{}{0pt}%
\pgfpathmoveto{\pgfqpoint{1.736390in}{3.147721in}}%
\pgfpathcurveto{\pgfqpoint{1.747440in}{3.147721in}}{\pgfqpoint{1.758039in}{3.152111in}}{\pgfqpoint{1.765852in}{3.159925in}}%
\pgfpathcurveto{\pgfqpoint{1.773666in}{3.167738in}}{\pgfqpoint{1.778056in}{3.178337in}}{\pgfqpoint{1.778056in}{3.189388in}}%
\pgfpathcurveto{\pgfqpoint{1.778056in}{3.200438in}}{\pgfqpoint{1.773666in}{3.211037in}}{\pgfqpoint{1.765852in}{3.218850in}}%
\pgfpathcurveto{\pgfqpoint{1.758039in}{3.226664in}}{\pgfqpoint{1.747440in}{3.231054in}}{\pgfqpoint{1.736390in}{3.231054in}}%
\pgfpathcurveto{\pgfqpoint{1.725339in}{3.231054in}}{\pgfqpoint{1.714740in}{3.226664in}}{\pgfqpoint{1.706927in}{3.218850in}}%
\pgfpathcurveto{\pgfqpoint{1.699113in}{3.211037in}}{\pgfqpoint{1.694723in}{3.200438in}}{\pgfqpoint{1.694723in}{3.189388in}}%
\pgfpathcurveto{\pgfqpoint{1.694723in}{3.178337in}}{\pgfqpoint{1.699113in}{3.167738in}}{\pgfqpoint{1.706927in}{3.159925in}}%
\pgfpathcurveto{\pgfqpoint{1.714740in}{3.152111in}}{\pgfqpoint{1.725339in}{3.147721in}}{\pgfqpoint{1.736390in}{3.147721in}}%
\pgfpathclose%
\pgfusepath{stroke,fill}%
\end{pgfscope}%
\begin{pgfscope}%
\pgfpathrectangle{\pgfqpoint{0.511823in}{0.504323in}}{\pgfqpoint{3.218177in}{3.225677in}} %
\pgfusepath{clip}%
\pgfsetbuttcap%
\pgfsetroundjoin%
\definecolor{currentfill}{rgb}{0.501961,0.000000,0.000000}%
\pgfsetfillcolor{currentfill}%
\pgfsetfillopacity{0.400000}%
\pgfsetlinewidth{0.501875pt}%
\definecolor{currentstroke}{rgb}{0.501961,0.000000,0.000000}%
\pgfsetstrokecolor{currentstroke}%
\pgfsetstrokeopacity{0.400000}%
\pgfsetdash{}{0pt}%
\pgfpathmoveto{\pgfqpoint{1.601352in}{2.845671in}}%
\pgfpathcurveto{\pgfqpoint{1.612402in}{2.845671in}}{\pgfqpoint{1.623001in}{2.850061in}}{\pgfqpoint{1.630815in}{2.857875in}}%
\pgfpathcurveto{\pgfqpoint{1.638629in}{2.865689in}}{\pgfqpoint{1.643019in}{2.876288in}}{\pgfqpoint{1.643019in}{2.887338in}}%
\pgfpathcurveto{\pgfqpoint{1.643019in}{2.898388in}}{\pgfqpoint{1.638629in}{2.908987in}}{\pgfqpoint{1.630815in}{2.916800in}}%
\pgfpathcurveto{\pgfqpoint{1.623001in}{2.924614in}}{\pgfqpoint{1.612402in}{2.929004in}}{\pgfqpoint{1.601352in}{2.929004in}}%
\pgfpathcurveto{\pgfqpoint{1.590302in}{2.929004in}}{\pgfqpoint{1.579703in}{2.924614in}}{\pgfqpoint{1.571890in}{2.916800in}}%
\pgfpathcurveto{\pgfqpoint{1.564076in}{2.908987in}}{\pgfqpoint{1.559686in}{2.898388in}}{\pgfqpoint{1.559686in}{2.887338in}}%
\pgfpathcurveto{\pgfqpoint{1.559686in}{2.876288in}}{\pgfqpoint{1.564076in}{2.865689in}}{\pgfqpoint{1.571890in}{2.857875in}}%
\pgfpathcurveto{\pgfqpoint{1.579703in}{2.850061in}}{\pgfqpoint{1.590302in}{2.845671in}}{\pgfqpoint{1.601352in}{2.845671in}}%
\pgfpathclose%
\pgfusepath{stroke,fill}%
\end{pgfscope}%
\begin{pgfscope}%
\pgfpathrectangle{\pgfqpoint{0.511823in}{0.504323in}}{\pgfqpoint{3.218177in}{3.225677in}} %
\pgfusepath{clip}%
\pgfsetbuttcap%
\pgfsetroundjoin%
\definecolor{currentfill}{rgb}{0.501961,0.000000,0.000000}%
\pgfsetfillcolor{currentfill}%
\pgfsetfillopacity{0.400000}%
\pgfsetlinewidth{0.501875pt}%
\definecolor{currentstroke}{rgb}{0.501961,0.000000,0.000000}%
\pgfsetstrokecolor{currentstroke}%
\pgfsetstrokeopacity{0.400000}%
\pgfsetdash{}{0pt}%
\pgfpathmoveto{\pgfqpoint{1.642601in}{2.966907in}}%
\pgfpathcurveto{\pgfqpoint{1.653651in}{2.966907in}}{\pgfqpoint{1.664250in}{2.971297in}}{\pgfqpoint{1.672064in}{2.979111in}}%
\pgfpathcurveto{\pgfqpoint{1.679877in}{2.986925in}}{\pgfqpoint{1.684268in}{2.997524in}}{\pgfqpoint{1.684268in}{3.008574in}}%
\pgfpathcurveto{\pgfqpoint{1.684268in}{3.019624in}}{\pgfqpoint{1.679877in}{3.030223in}}{\pgfqpoint{1.672064in}{3.038037in}}%
\pgfpathcurveto{\pgfqpoint{1.664250in}{3.045850in}}{\pgfqpoint{1.653651in}{3.050240in}}{\pgfqpoint{1.642601in}{3.050240in}}%
\pgfpathcurveto{\pgfqpoint{1.631551in}{3.050240in}}{\pgfqpoint{1.620952in}{3.045850in}}{\pgfqpoint{1.613138in}{3.038037in}}%
\pgfpathcurveto{\pgfqpoint{1.605325in}{3.030223in}}{\pgfqpoint{1.600934in}{3.019624in}}{\pgfqpoint{1.600934in}{3.008574in}}%
\pgfpathcurveto{\pgfqpoint{1.600934in}{2.997524in}}{\pgfqpoint{1.605325in}{2.986925in}}{\pgfqpoint{1.613138in}{2.979111in}}%
\pgfpathcurveto{\pgfqpoint{1.620952in}{2.971297in}}{\pgfqpoint{1.631551in}{2.966907in}}{\pgfqpoint{1.642601in}{2.966907in}}%
\pgfpathclose%
\pgfusepath{stroke,fill}%
\end{pgfscope}%
\begin{pgfscope}%
\pgfpathrectangle{\pgfqpoint{0.511823in}{0.504323in}}{\pgfqpoint{3.218177in}{3.225677in}} %
\pgfusepath{clip}%
\pgfsetbuttcap%
\pgfsetroundjoin%
\definecolor{currentfill}{rgb}{0.501961,0.000000,0.000000}%
\pgfsetfillcolor{currentfill}%
\pgfsetfillopacity{0.400000}%
\pgfsetlinewidth{0.501875pt}%
\definecolor{currentstroke}{rgb}{0.501961,0.000000,0.000000}%
\pgfsetstrokecolor{currentstroke}%
\pgfsetstrokeopacity{0.400000}%
\pgfsetdash{}{0pt}%
\pgfpathmoveto{\pgfqpoint{1.685987in}{3.095651in}}%
\pgfpathcurveto{\pgfqpoint{1.697037in}{3.095651in}}{\pgfqpoint{1.707636in}{3.100041in}}{\pgfqpoint{1.715450in}{3.107855in}}%
\pgfpathcurveto{\pgfqpoint{1.723263in}{3.115669in}}{\pgfqpoint{1.727653in}{3.126268in}}{\pgfqpoint{1.727653in}{3.137318in}}%
\pgfpathcurveto{\pgfqpoint{1.727653in}{3.148368in}}{\pgfqpoint{1.723263in}{3.158967in}}{\pgfqpoint{1.715450in}{3.166781in}}%
\pgfpathcurveto{\pgfqpoint{1.707636in}{3.174594in}}{\pgfqpoint{1.697037in}{3.178985in}}{\pgfqpoint{1.685987in}{3.178985in}}%
\pgfpathcurveto{\pgfqpoint{1.674937in}{3.178985in}}{\pgfqpoint{1.664338in}{3.174594in}}{\pgfqpoint{1.656524in}{3.166781in}}%
\pgfpathcurveto{\pgfqpoint{1.648710in}{3.158967in}}{\pgfqpoint{1.644320in}{3.148368in}}{\pgfqpoint{1.644320in}{3.137318in}}%
\pgfpathcurveto{\pgfqpoint{1.644320in}{3.126268in}}{\pgfqpoint{1.648710in}{3.115669in}}{\pgfqpoint{1.656524in}{3.107855in}}%
\pgfpathcurveto{\pgfqpoint{1.664338in}{3.100041in}}{\pgfqpoint{1.674937in}{3.095651in}}{\pgfqpoint{1.685987in}{3.095651in}}%
\pgfpathclose%
\pgfusepath{stroke,fill}%
\end{pgfscope}%
\begin{pgfscope}%
\pgfpathrectangle{\pgfqpoint{0.511823in}{0.504323in}}{\pgfqpoint{3.218177in}{3.225677in}} %
\pgfusepath{clip}%
\pgfsetbuttcap%
\pgfsetroundjoin%
\definecolor{currentfill}{rgb}{0.501961,0.000000,0.000000}%
\pgfsetfillcolor{currentfill}%
\pgfsetfillopacity{0.400000}%
\pgfsetlinewidth{0.501875pt}%
\definecolor{currentstroke}{rgb}{0.501961,0.000000,0.000000}%
\pgfsetstrokecolor{currentstroke}%
\pgfsetstrokeopacity{0.400000}%
\pgfsetdash{}{0pt}%
\pgfpathmoveto{\pgfqpoint{1.662629in}{3.061249in}}%
\pgfpathcurveto{\pgfqpoint{1.673679in}{3.061249in}}{\pgfqpoint{1.684278in}{3.065640in}}{\pgfqpoint{1.692092in}{3.073453in}}%
\pgfpathcurveto{\pgfqpoint{1.699906in}{3.081267in}}{\pgfqpoint{1.704296in}{3.091866in}}{\pgfqpoint{1.704296in}{3.102916in}}%
\pgfpathcurveto{\pgfqpoint{1.704296in}{3.113966in}}{\pgfqpoint{1.699906in}{3.124565in}}{\pgfqpoint{1.692092in}{3.132379in}}%
\pgfpathcurveto{\pgfqpoint{1.684278in}{3.140193in}}{\pgfqpoint{1.673679in}{3.144583in}}{\pgfqpoint{1.662629in}{3.144583in}}%
\pgfpathcurveto{\pgfqpoint{1.651579in}{3.144583in}}{\pgfqpoint{1.640980in}{3.140193in}}{\pgfqpoint{1.633166in}{3.132379in}}%
\pgfpathcurveto{\pgfqpoint{1.625353in}{3.124565in}}{\pgfqpoint{1.620963in}{3.113966in}}{\pgfqpoint{1.620963in}{3.102916in}}%
\pgfpathcurveto{\pgfqpoint{1.620963in}{3.091866in}}{\pgfqpoint{1.625353in}{3.081267in}}{\pgfqpoint{1.633166in}{3.073453in}}%
\pgfpathcurveto{\pgfqpoint{1.640980in}{3.065640in}}{\pgfqpoint{1.651579in}{3.061249in}}{\pgfqpoint{1.662629in}{3.061249in}}%
\pgfpathclose%
\pgfusepath{stroke,fill}%
\end{pgfscope}%
\begin{pgfscope}%
\pgfpathrectangle{\pgfqpoint{0.511823in}{0.504323in}}{\pgfqpoint{3.218177in}{3.225677in}} %
\pgfusepath{clip}%
\pgfsetbuttcap%
\pgfsetroundjoin%
\definecolor{currentfill}{rgb}{0.501961,0.000000,0.000000}%
\pgfsetfillcolor{currentfill}%
\pgfsetfillopacity{0.400000}%
\pgfsetlinewidth{0.501875pt}%
\definecolor{currentstroke}{rgb}{0.501961,0.000000,0.000000}%
\pgfsetstrokecolor{currentstroke}%
\pgfsetstrokeopacity{0.400000}%
\pgfsetdash{}{0pt}%
\pgfpathmoveto{\pgfqpoint{1.683976in}{3.138088in}}%
\pgfpathcurveto{\pgfqpoint{1.695026in}{3.138088in}}{\pgfqpoint{1.705625in}{3.142479in}}{\pgfqpoint{1.713438in}{3.150292in}}%
\pgfpathcurveto{\pgfqpoint{1.721252in}{3.158106in}}{\pgfqpoint{1.725642in}{3.168705in}}{\pgfqpoint{1.725642in}{3.179755in}}%
\pgfpathcurveto{\pgfqpoint{1.725642in}{3.190805in}}{\pgfqpoint{1.721252in}{3.201404in}}{\pgfqpoint{1.713438in}{3.209218in}}%
\pgfpathcurveto{\pgfqpoint{1.705625in}{3.217031in}}{\pgfqpoint{1.695026in}{3.221422in}}{\pgfqpoint{1.683976in}{3.221422in}}%
\pgfpathcurveto{\pgfqpoint{1.672926in}{3.221422in}}{\pgfqpoint{1.662327in}{3.217031in}}{\pgfqpoint{1.654513in}{3.209218in}}%
\pgfpathcurveto{\pgfqpoint{1.646699in}{3.201404in}}{\pgfqpoint{1.642309in}{3.190805in}}{\pgfqpoint{1.642309in}{3.179755in}}%
\pgfpathcurveto{\pgfqpoint{1.642309in}{3.168705in}}{\pgfqpoint{1.646699in}{3.158106in}}{\pgfqpoint{1.654513in}{3.150292in}}%
\pgfpathcurveto{\pgfqpoint{1.662327in}{3.142479in}}{\pgfqpoint{1.672926in}{3.138088in}}{\pgfqpoint{1.683976in}{3.138088in}}%
\pgfpathclose%
\pgfusepath{stroke,fill}%
\end{pgfscope}%
\begin{pgfscope}%
\pgfpathrectangle{\pgfqpoint{0.511823in}{0.504323in}}{\pgfqpoint{3.218177in}{3.225677in}} %
\pgfusepath{clip}%
\pgfsetbuttcap%
\pgfsetroundjoin%
\definecolor{currentfill}{rgb}{0.501961,0.000000,0.000000}%
\pgfsetfillcolor{currentfill}%
\pgfsetfillopacity{0.400000}%
\pgfsetlinewidth{0.501875pt}%
\definecolor{currentstroke}{rgb}{0.501961,0.000000,0.000000}%
\pgfsetstrokecolor{currentstroke}%
\pgfsetstrokeopacity{0.400000}%
\pgfsetdash{}{0pt}%
\pgfpathmoveto{\pgfqpoint{1.569164in}{2.872044in}}%
\pgfpathcurveto{\pgfqpoint{1.580215in}{2.872044in}}{\pgfqpoint{1.590814in}{2.876434in}}{\pgfqpoint{1.598627in}{2.884248in}}%
\pgfpathcurveto{\pgfqpoint{1.606441in}{2.892061in}}{\pgfqpoint{1.610831in}{2.902660in}}{\pgfqpoint{1.610831in}{2.913710in}}%
\pgfpathcurveto{\pgfqpoint{1.610831in}{2.924761in}}{\pgfqpoint{1.606441in}{2.935360in}}{\pgfqpoint{1.598627in}{2.943173in}}%
\pgfpathcurveto{\pgfqpoint{1.590814in}{2.950987in}}{\pgfqpoint{1.580215in}{2.955377in}}{\pgfqpoint{1.569164in}{2.955377in}}%
\pgfpathcurveto{\pgfqpoint{1.558114in}{2.955377in}}{\pgfqpoint{1.547515in}{2.950987in}}{\pgfqpoint{1.539702in}{2.943173in}}%
\pgfpathcurveto{\pgfqpoint{1.531888in}{2.935360in}}{\pgfqpoint{1.527498in}{2.924761in}}{\pgfqpoint{1.527498in}{2.913710in}}%
\pgfpathcurveto{\pgfqpoint{1.527498in}{2.902660in}}{\pgfqpoint{1.531888in}{2.892061in}}{\pgfqpoint{1.539702in}{2.884248in}}%
\pgfpathcurveto{\pgfqpoint{1.547515in}{2.876434in}}{\pgfqpoint{1.558114in}{2.872044in}}{\pgfqpoint{1.569164in}{2.872044in}}%
\pgfpathclose%
\pgfusepath{stroke,fill}%
\end{pgfscope}%
\begin{pgfscope}%
\pgfpathrectangle{\pgfqpoint{0.511823in}{0.504323in}}{\pgfqpoint{3.218177in}{3.225677in}} %
\pgfusepath{clip}%
\pgfsetbuttcap%
\pgfsetroundjoin%
\definecolor{currentfill}{rgb}{0.501961,0.000000,0.000000}%
\pgfsetfillcolor{currentfill}%
\pgfsetfillopacity{0.400000}%
\pgfsetlinewidth{0.501875pt}%
\definecolor{currentstroke}{rgb}{0.501961,0.000000,0.000000}%
\pgfsetstrokecolor{currentstroke}%
\pgfsetstrokeopacity{0.400000}%
\pgfsetdash{}{0pt}%
\pgfpathmoveto{\pgfqpoint{1.566561in}{2.887265in}}%
\pgfpathcurveto{\pgfqpoint{1.577611in}{2.887265in}}{\pgfqpoint{1.588210in}{2.891656in}}{\pgfqpoint{1.596023in}{2.899469in}}%
\pgfpathcurveto{\pgfqpoint{1.603837in}{2.907283in}}{\pgfqpoint{1.608227in}{2.917882in}}{\pgfqpoint{1.608227in}{2.928932in}}%
\pgfpathcurveto{\pgfqpoint{1.608227in}{2.939982in}}{\pgfqpoint{1.603837in}{2.950581in}}{\pgfqpoint{1.596023in}{2.958395in}}%
\pgfpathcurveto{\pgfqpoint{1.588210in}{2.966209in}}{\pgfqpoint{1.577611in}{2.970599in}}{\pgfqpoint{1.566561in}{2.970599in}}%
\pgfpathcurveto{\pgfqpoint{1.555510in}{2.970599in}}{\pgfqpoint{1.544911in}{2.966209in}}{\pgfqpoint{1.537098in}{2.958395in}}%
\pgfpathcurveto{\pgfqpoint{1.529284in}{2.950581in}}{\pgfqpoint{1.524894in}{2.939982in}}{\pgfqpoint{1.524894in}{2.928932in}}%
\pgfpathcurveto{\pgfqpoint{1.524894in}{2.917882in}}{\pgfqpoint{1.529284in}{2.907283in}}{\pgfqpoint{1.537098in}{2.899469in}}%
\pgfpathcurveto{\pgfqpoint{1.544911in}{2.891656in}}{\pgfqpoint{1.555510in}{2.887265in}}{\pgfqpoint{1.566561in}{2.887265in}}%
\pgfpathclose%
\pgfusepath{stroke,fill}%
\end{pgfscope}%
\begin{pgfscope}%
\pgfpathrectangle{\pgfqpoint{0.511823in}{0.504323in}}{\pgfqpoint{3.218177in}{3.225677in}} %
\pgfusepath{clip}%
\pgfsetbuttcap%
\pgfsetroundjoin%
\definecolor{currentfill}{rgb}{0.501961,0.000000,0.000000}%
\pgfsetfillcolor{currentfill}%
\pgfsetfillopacity{0.400000}%
\pgfsetlinewidth{0.501875pt}%
\definecolor{currentstroke}{rgb}{0.501961,0.000000,0.000000}%
\pgfsetstrokecolor{currentstroke}%
\pgfsetstrokeopacity{0.400000}%
\pgfsetdash{}{0pt}%
\pgfpathmoveto{\pgfqpoint{1.616016in}{3.036989in}}%
\pgfpathcurveto{\pgfqpoint{1.627066in}{3.036989in}}{\pgfqpoint{1.637665in}{3.041379in}}{\pgfqpoint{1.645479in}{3.049193in}}%
\pgfpathcurveto{\pgfqpoint{1.653292in}{3.057007in}}{\pgfqpoint{1.657682in}{3.067606in}}{\pgfqpoint{1.657682in}{3.078656in}}%
\pgfpathcurveto{\pgfqpoint{1.657682in}{3.089706in}}{\pgfqpoint{1.653292in}{3.100305in}}{\pgfqpoint{1.645479in}{3.108119in}}%
\pgfpathcurveto{\pgfqpoint{1.637665in}{3.115932in}}{\pgfqpoint{1.627066in}{3.120322in}}{\pgfqpoint{1.616016in}{3.120322in}}%
\pgfpathcurveto{\pgfqpoint{1.604966in}{3.120322in}}{\pgfqpoint{1.594367in}{3.115932in}}{\pgfqpoint{1.586553in}{3.108119in}}%
\pgfpathcurveto{\pgfqpoint{1.578739in}{3.100305in}}{\pgfqpoint{1.574349in}{3.089706in}}{\pgfqpoint{1.574349in}{3.078656in}}%
\pgfpathcurveto{\pgfqpoint{1.574349in}{3.067606in}}{\pgfqpoint{1.578739in}{3.057007in}}{\pgfqpoint{1.586553in}{3.049193in}}%
\pgfpathcurveto{\pgfqpoint{1.594367in}{3.041379in}}{\pgfqpoint{1.604966in}{3.036989in}}{\pgfqpoint{1.616016in}{3.036989in}}%
\pgfpathclose%
\pgfusepath{stroke,fill}%
\end{pgfscope}%
\begin{pgfscope}%
\pgfpathrectangle{\pgfqpoint{0.511823in}{0.504323in}}{\pgfqpoint{3.218177in}{3.225677in}} %
\pgfusepath{clip}%
\pgfsetbuttcap%
\pgfsetroundjoin%
\definecolor{currentfill}{rgb}{0.501961,0.000000,0.000000}%
\pgfsetfillcolor{currentfill}%
\pgfsetfillopacity{0.400000}%
\pgfsetlinewidth{0.501875pt}%
\definecolor{currentstroke}{rgb}{0.501961,0.000000,0.000000}%
\pgfsetstrokecolor{currentstroke}%
\pgfsetstrokeopacity{0.400000}%
\pgfsetdash{}{0pt}%
\pgfpathmoveto{\pgfqpoint{1.520033in}{2.810849in}}%
\pgfpathcurveto{\pgfqpoint{1.531083in}{2.810849in}}{\pgfqpoint{1.541682in}{2.815240in}}{\pgfqpoint{1.549496in}{2.823053in}}%
\pgfpathcurveto{\pgfqpoint{1.557310in}{2.830867in}}{\pgfqpoint{1.561700in}{2.841466in}}{\pgfqpoint{1.561700in}{2.852516in}}%
\pgfpathcurveto{\pgfqpoint{1.561700in}{2.863566in}}{\pgfqpoint{1.557310in}{2.874165in}}{\pgfqpoint{1.549496in}{2.881979in}}%
\pgfpathcurveto{\pgfqpoint{1.541682in}{2.889793in}}{\pgfqpoint{1.531083in}{2.894183in}}{\pgfqpoint{1.520033in}{2.894183in}}%
\pgfpathcurveto{\pgfqpoint{1.508983in}{2.894183in}}{\pgfqpoint{1.498384in}{2.889793in}}{\pgfqpoint{1.490571in}{2.881979in}}%
\pgfpathcurveto{\pgfqpoint{1.482757in}{2.874165in}}{\pgfqpoint{1.478367in}{2.863566in}}{\pgfqpoint{1.478367in}{2.852516in}}%
\pgfpathcurveto{\pgfqpoint{1.478367in}{2.841466in}}{\pgfqpoint{1.482757in}{2.830867in}}{\pgfqpoint{1.490571in}{2.823053in}}%
\pgfpathcurveto{\pgfqpoint{1.498384in}{2.815240in}}{\pgfqpoint{1.508983in}{2.810849in}}{\pgfqpoint{1.520033in}{2.810849in}}%
\pgfpathclose%
\pgfusepath{stroke,fill}%
\end{pgfscope}%
\begin{pgfscope}%
\pgfpathrectangle{\pgfqpoint{0.511823in}{0.504323in}}{\pgfqpoint{3.218177in}{3.225677in}} %
\pgfusepath{clip}%
\pgfsetbuttcap%
\pgfsetroundjoin%
\definecolor{currentfill}{rgb}{0.501961,0.000000,0.000000}%
\pgfsetfillcolor{currentfill}%
\pgfsetfillopacity{0.400000}%
\pgfsetlinewidth{0.501875pt}%
\definecolor{currentstroke}{rgb}{0.501961,0.000000,0.000000}%
\pgfsetstrokecolor{currentstroke}%
\pgfsetstrokeopacity{0.400000}%
\pgfsetdash{}{0pt}%
\pgfpathmoveto{\pgfqpoint{1.593377in}{3.025551in}}%
\pgfpathcurveto{\pgfqpoint{1.604427in}{3.025551in}}{\pgfqpoint{1.615026in}{3.029942in}}{\pgfqpoint{1.622840in}{3.037755in}}%
\pgfpathcurveto{\pgfqpoint{1.630653in}{3.045569in}}{\pgfqpoint{1.635044in}{3.056168in}}{\pgfqpoint{1.635044in}{3.067218in}}%
\pgfpathcurveto{\pgfqpoint{1.635044in}{3.078268in}}{\pgfqpoint{1.630653in}{3.088867in}}{\pgfqpoint{1.622840in}{3.096681in}}%
\pgfpathcurveto{\pgfqpoint{1.615026in}{3.104494in}}{\pgfqpoint{1.604427in}{3.108885in}}{\pgfqpoint{1.593377in}{3.108885in}}%
\pgfpathcurveto{\pgfqpoint{1.582327in}{3.108885in}}{\pgfqpoint{1.571728in}{3.104494in}}{\pgfqpoint{1.563914in}{3.096681in}}%
\pgfpathcurveto{\pgfqpoint{1.556101in}{3.088867in}}{\pgfqpoint{1.551710in}{3.078268in}}{\pgfqpoint{1.551710in}{3.067218in}}%
\pgfpathcurveto{\pgfqpoint{1.551710in}{3.056168in}}{\pgfqpoint{1.556101in}{3.045569in}}{\pgfqpoint{1.563914in}{3.037755in}}%
\pgfpathcurveto{\pgfqpoint{1.571728in}{3.029942in}}{\pgfqpoint{1.582327in}{3.025551in}}{\pgfqpoint{1.593377in}{3.025551in}}%
\pgfpathclose%
\pgfusepath{stroke,fill}%
\end{pgfscope}%
\begin{pgfscope}%
\pgfpathrectangle{\pgfqpoint{0.511823in}{0.504323in}}{\pgfqpoint{3.218177in}{3.225677in}} %
\pgfusepath{clip}%
\pgfsetbuttcap%
\pgfsetroundjoin%
\definecolor{currentfill}{rgb}{0.501961,0.000000,0.000000}%
\pgfsetfillcolor{currentfill}%
\pgfsetfillopacity{0.400000}%
\pgfsetlinewidth{0.501875pt}%
\definecolor{currentstroke}{rgb}{0.501961,0.000000,0.000000}%
\pgfsetstrokecolor{currentstroke}%
\pgfsetstrokeopacity{0.400000}%
\pgfsetdash{}{0pt}%
\pgfpathmoveto{\pgfqpoint{1.606779in}{3.085266in}}%
\pgfpathcurveto{\pgfqpoint{1.617829in}{3.085266in}}{\pgfqpoint{1.628428in}{3.089656in}}{\pgfqpoint{1.636241in}{3.097470in}}%
\pgfpathcurveto{\pgfqpoint{1.644055in}{3.105283in}}{\pgfqpoint{1.648445in}{3.115882in}}{\pgfqpoint{1.648445in}{3.126932in}}%
\pgfpathcurveto{\pgfqpoint{1.648445in}{3.137983in}}{\pgfqpoint{1.644055in}{3.148582in}}{\pgfqpoint{1.636241in}{3.156395in}}%
\pgfpathcurveto{\pgfqpoint{1.628428in}{3.164209in}}{\pgfqpoint{1.617829in}{3.168599in}}{\pgfqpoint{1.606779in}{3.168599in}}%
\pgfpathcurveto{\pgfqpoint{1.595729in}{3.168599in}}{\pgfqpoint{1.585130in}{3.164209in}}{\pgfqpoint{1.577316in}{3.156395in}}%
\pgfpathcurveto{\pgfqpoint{1.569502in}{3.148582in}}{\pgfqpoint{1.565112in}{3.137983in}}{\pgfqpoint{1.565112in}{3.126932in}}%
\pgfpathcurveto{\pgfqpoint{1.565112in}{3.115882in}}{\pgfqpoint{1.569502in}{3.105283in}}{\pgfqpoint{1.577316in}{3.097470in}}%
\pgfpathcurveto{\pgfqpoint{1.585130in}{3.089656in}}{\pgfqpoint{1.595729in}{3.085266in}}{\pgfqpoint{1.606779in}{3.085266in}}%
\pgfpathclose%
\pgfusepath{stroke,fill}%
\end{pgfscope}%
\begin{pgfscope}%
\pgfpathrectangle{\pgfqpoint{0.511823in}{0.504323in}}{\pgfqpoint{3.218177in}{3.225677in}} %
\pgfusepath{clip}%
\pgfsetbuttcap%
\pgfsetroundjoin%
\definecolor{currentfill}{rgb}{0.501961,0.000000,0.000000}%
\pgfsetfillcolor{currentfill}%
\pgfsetfillopacity{0.400000}%
\pgfsetlinewidth{0.501875pt}%
\definecolor{currentstroke}{rgb}{0.501961,0.000000,0.000000}%
\pgfsetstrokecolor{currentstroke}%
\pgfsetstrokeopacity{0.400000}%
\pgfsetdash{}{0pt}%
\pgfpathmoveto{\pgfqpoint{1.572739in}{3.018800in}}%
\pgfpathcurveto{\pgfqpoint{1.583789in}{3.018800in}}{\pgfqpoint{1.594388in}{3.023190in}}{\pgfqpoint{1.602202in}{3.031004in}}%
\pgfpathcurveto{\pgfqpoint{1.610016in}{3.038817in}}{\pgfqpoint{1.614406in}{3.049416in}}{\pgfqpoint{1.614406in}{3.060466in}}%
\pgfpathcurveto{\pgfqpoint{1.614406in}{3.071516in}}{\pgfqpoint{1.610016in}{3.082115in}}{\pgfqpoint{1.602202in}{3.089929in}}%
\pgfpathcurveto{\pgfqpoint{1.594388in}{3.097743in}}{\pgfqpoint{1.583789in}{3.102133in}}{\pgfqpoint{1.572739in}{3.102133in}}%
\pgfpathcurveto{\pgfqpoint{1.561689in}{3.102133in}}{\pgfqpoint{1.551090in}{3.097743in}}{\pgfqpoint{1.543276in}{3.089929in}}%
\pgfpathcurveto{\pgfqpoint{1.535463in}{3.082115in}}{\pgfqpoint{1.531073in}{3.071516in}}{\pgfqpoint{1.531073in}{3.060466in}}%
\pgfpathcurveto{\pgfqpoint{1.531073in}{3.049416in}}{\pgfqpoint{1.535463in}{3.038817in}}{\pgfqpoint{1.543276in}{3.031004in}}%
\pgfpathcurveto{\pgfqpoint{1.551090in}{3.023190in}}{\pgfqpoint{1.561689in}{3.018800in}}{\pgfqpoint{1.572739in}{3.018800in}}%
\pgfpathclose%
\pgfusepath{stroke,fill}%
\end{pgfscope}%
\begin{pgfscope}%
\pgfpathrectangle{\pgfqpoint{0.511823in}{0.504323in}}{\pgfqpoint{3.218177in}{3.225677in}} %
\pgfusepath{clip}%
\pgfsetbuttcap%
\pgfsetroundjoin%
\definecolor{currentfill}{rgb}{0.501961,0.000000,0.000000}%
\pgfsetfillcolor{currentfill}%
\pgfsetfillopacity{0.400000}%
\pgfsetlinewidth{0.501875pt}%
\definecolor{currentstroke}{rgb}{0.501961,0.000000,0.000000}%
\pgfsetstrokecolor{currentstroke}%
\pgfsetstrokeopacity{0.400000}%
\pgfsetdash{}{0pt}%
\pgfpathmoveto{\pgfqpoint{1.491517in}{2.822963in}}%
\pgfpathcurveto{\pgfqpoint{1.502567in}{2.822963in}}{\pgfqpoint{1.513166in}{2.827353in}}{\pgfqpoint{1.520979in}{2.835167in}}%
\pgfpathcurveto{\pgfqpoint{1.528793in}{2.842980in}}{\pgfqpoint{1.533183in}{2.853580in}}{\pgfqpoint{1.533183in}{2.864630in}}%
\pgfpathcurveto{\pgfqpoint{1.533183in}{2.875680in}}{\pgfqpoint{1.528793in}{2.886279in}}{\pgfqpoint{1.520979in}{2.894092in}}%
\pgfpathcurveto{\pgfqpoint{1.513166in}{2.901906in}}{\pgfqpoint{1.502567in}{2.906296in}}{\pgfqpoint{1.491517in}{2.906296in}}%
\pgfpathcurveto{\pgfqpoint{1.480466in}{2.906296in}}{\pgfqpoint{1.469867in}{2.901906in}}{\pgfqpoint{1.462054in}{2.894092in}}%
\pgfpathcurveto{\pgfqpoint{1.454240in}{2.886279in}}{\pgfqpoint{1.449850in}{2.875680in}}{\pgfqpoint{1.449850in}{2.864630in}}%
\pgfpathcurveto{\pgfqpoint{1.449850in}{2.853580in}}{\pgfqpoint{1.454240in}{2.842980in}}{\pgfqpoint{1.462054in}{2.835167in}}%
\pgfpathcurveto{\pgfqpoint{1.469867in}{2.827353in}}{\pgfqpoint{1.480466in}{2.822963in}}{\pgfqpoint{1.491517in}{2.822963in}}%
\pgfpathclose%
\pgfusepath{stroke,fill}%
\end{pgfscope}%
\begin{pgfscope}%
\pgfpathrectangle{\pgfqpoint{0.511823in}{0.504323in}}{\pgfqpoint{3.218177in}{3.225677in}} %
\pgfusepath{clip}%
\pgfsetbuttcap%
\pgfsetroundjoin%
\definecolor{currentfill}{rgb}{0.501961,0.000000,0.000000}%
\pgfsetfillcolor{currentfill}%
\pgfsetfillopacity{0.400000}%
\pgfsetlinewidth{0.501875pt}%
\definecolor{currentstroke}{rgb}{0.501961,0.000000,0.000000}%
\pgfsetstrokecolor{currentstroke}%
\pgfsetstrokeopacity{0.400000}%
\pgfsetdash{}{0pt}%
\pgfpathmoveto{\pgfqpoint{1.517142in}{2.915768in}}%
\pgfpathcurveto{\pgfqpoint{1.528192in}{2.915768in}}{\pgfqpoint{1.538791in}{2.920158in}}{\pgfqpoint{1.546605in}{2.927972in}}%
\pgfpathcurveto{\pgfqpoint{1.554418in}{2.935786in}}{\pgfqpoint{1.558809in}{2.946385in}}{\pgfqpoint{1.558809in}{2.957435in}}%
\pgfpathcurveto{\pgfqpoint{1.558809in}{2.968485in}}{\pgfqpoint{1.554418in}{2.979084in}}{\pgfqpoint{1.546605in}{2.986898in}}%
\pgfpathcurveto{\pgfqpoint{1.538791in}{2.994711in}}{\pgfqpoint{1.528192in}{2.999102in}}{\pgfqpoint{1.517142in}{2.999102in}}%
\pgfpathcurveto{\pgfqpoint{1.506092in}{2.999102in}}{\pgfqpoint{1.495493in}{2.994711in}}{\pgfqpoint{1.487679in}{2.986898in}}%
\pgfpathcurveto{\pgfqpoint{1.479866in}{2.979084in}}{\pgfqpoint{1.475475in}{2.968485in}}{\pgfqpoint{1.475475in}{2.957435in}}%
\pgfpathcurveto{\pgfqpoint{1.475475in}{2.946385in}}{\pgfqpoint{1.479866in}{2.935786in}}{\pgfqpoint{1.487679in}{2.927972in}}%
\pgfpathcurveto{\pgfqpoint{1.495493in}{2.920158in}}{\pgfqpoint{1.506092in}{2.915768in}}{\pgfqpoint{1.517142in}{2.915768in}}%
\pgfpathclose%
\pgfusepath{stroke,fill}%
\end{pgfscope}%
\begin{pgfscope}%
\pgfpathrectangle{\pgfqpoint{0.511823in}{0.504323in}}{\pgfqpoint{3.218177in}{3.225677in}} %
\pgfusepath{clip}%
\pgfsetbuttcap%
\pgfsetroundjoin%
\definecolor{currentfill}{rgb}{0.501961,0.000000,0.000000}%
\pgfsetfillcolor{currentfill}%
\pgfsetfillopacity{0.400000}%
\pgfsetlinewidth{0.501875pt}%
\definecolor{currentstroke}{rgb}{0.501961,0.000000,0.000000}%
\pgfsetstrokecolor{currentstroke}%
\pgfsetstrokeopacity{0.400000}%
\pgfsetdash{}{0pt}%
\pgfpathmoveto{\pgfqpoint{1.520375in}{2.948468in}}%
\pgfpathcurveto{\pgfqpoint{1.531425in}{2.948468in}}{\pgfqpoint{1.542024in}{2.952858in}}{\pgfqpoint{1.549837in}{2.960672in}}%
\pgfpathcurveto{\pgfqpoint{1.557651in}{2.968486in}}{\pgfqpoint{1.562041in}{2.979085in}}{\pgfqpoint{1.562041in}{2.990135in}}%
\pgfpathcurveto{\pgfqpoint{1.562041in}{3.001185in}}{\pgfqpoint{1.557651in}{3.011784in}}{\pgfqpoint{1.549837in}{3.019598in}}%
\pgfpathcurveto{\pgfqpoint{1.542024in}{3.027411in}}{\pgfqpoint{1.531425in}{3.031802in}}{\pgfqpoint{1.520375in}{3.031802in}}%
\pgfpathcurveto{\pgfqpoint{1.509325in}{3.031802in}}{\pgfqpoint{1.498726in}{3.027411in}}{\pgfqpoint{1.490912in}{3.019598in}}%
\pgfpathcurveto{\pgfqpoint{1.483098in}{3.011784in}}{\pgfqpoint{1.478708in}{3.001185in}}{\pgfqpoint{1.478708in}{2.990135in}}%
\pgfpathcurveto{\pgfqpoint{1.478708in}{2.979085in}}{\pgfqpoint{1.483098in}{2.968486in}}{\pgfqpoint{1.490912in}{2.960672in}}%
\pgfpathcurveto{\pgfqpoint{1.498726in}{2.952858in}}{\pgfqpoint{1.509325in}{2.948468in}}{\pgfqpoint{1.520375in}{2.948468in}}%
\pgfpathclose%
\pgfusepath{stroke,fill}%
\end{pgfscope}%
\begin{pgfscope}%
\pgfpathrectangle{\pgfqpoint{0.511823in}{0.504323in}}{\pgfqpoint{3.218177in}{3.225677in}} %
\pgfusepath{clip}%
\pgfsetbuttcap%
\pgfsetroundjoin%
\definecolor{currentfill}{rgb}{0.501961,0.000000,0.000000}%
\pgfsetfillcolor{currentfill}%
\pgfsetfillopacity{0.400000}%
\pgfsetlinewidth{0.501875pt}%
\definecolor{currentstroke}{rgb}{0.501961,0.000000,0.000000}%
\pgfsetstrokecolor{currentstroke}%
\pgfsetstrokeopacity{0.400000}%
\pgfsetdash{}{0pt}%
\pgfpathmoveto{\pgfqpoint{1.543373in}{3.037100in}}%
\pgfpathcurveto{\pgfqpoint{1.554423in}{3.037100in}}{\pgfqpoint{1.565022in}{3.041491in}}{\pgfqpoint{1.572835in}{3.049304in}}%
\pgfpathcurveto{\pgfqpoint{1.580649in}{3.057118in}}{\pgfqpoint{1.585039in}{3.067717in}}{\pgfqpoint{1.585039in}{3.078767in}}%
\pgfpathcurveto{\pgfqpoint{1.585039in}{3.089817in}}{\pgfqpoint{1.580649in}{3.100416in}}{\pgfqpoint{1.572835in}{3.108230in}}%
\pgfpathcurveto{\pgfqpoint{1.565022in}{3.116043in}}{\pgfqpoint{1.554423in}{3.120434in}}{\pgfqpoint{1.543373in}{3.120434in}}%
\pgfpathcurveto{\pgfqpoint{1.532322in}{3.120434in}}{\pgfqpoint{1.521723in}{3.116043in}}{\pgfqpoint{1.513910in}{3.108230in}}%
\pgfpathcurveto{\pgfqpoint{1.506096in}{3.100416in}}{\pgfqpoint{1.501706in}{3.089817in}}{\pgfqpoint{1.501706in}{3.078767in}}%
\pgfpathcurveto{\pgfqpoint{1.501706in}{3.067717in}}{\pgfqpoint{1.506096in}{3.057118in}}{\pgfqpoint{1.513910in}{3.049304in}}%
\pgfpathcurveto{\pgfqpoint{1.521723in}{3.041491in}}{\pgfqpoint{1.532322in}{3.037100in}}{\pgfqpoint{1.543373in}{3.037100in}}%
\pgfpathclose%
\pgfusepath{stroke,fill}%
\end{pgfscope}%
\begin{pgfscope}%
\pgfpathrectangle{\pgfqpoint{0.511823in}{0.504323in}}{\pgfqpoint{3.218177in}{3.225677in}} %
\pgfusepath{clip}%
\pgfsetbuttcap%
\pgfsetroundjoin%
\definecolor{currentfill}{rgb}{0.501961,0.000000,0.000000}%
\pgfsetfillcolor{currentfill}%
\pgfsetfillopacity{0.400000}%
\pgfsetlinewidth{0.501875pt}%
\definecolor{currentstroke}{rgb}{0.501961,0.000000,0.000000}%
\pgfsetstrokecolor{currentstroke}%
\pgfsetstrokeopacity{0.400000}%
\pgfsetdash{}{0pt}%
\pgfpathmoveto{\pgfqpoint{1.528060in}{3.019250in}}%
\pgfpathcurveto{\pgfqpoint{1.539111in}{3.019250in}}{\pgfqpoint{1.549710in}{3.023640in}}{\pgfqpoint{1.557523in}{3.031454in}}%
\pgfpathcurveto{\pgfqpoint{1.565337in}{3.039267in}}{\pgfqpoint{1.569727in}{3.049866in}}{\pgfqpoint{1.569727in}{3.060916in}}%
\pgfpathcurveto{\pgfqpoint{1.569727in}{3.071967in}}{\pgfqpoint{1.565337in}{3.082566in}}{\pgfqpoint{1.557523in}{3.090379in}}%
\pgfpathcurveto{\pgfqpoint{1.549710in}{3.098193in}}{\pgfqpoint{1.539111in}{3.102583in}}{\pgfqpoint{1.528060in}{3.102583in}}%
\pgfpathcurveto{\pgfqpoint{1.517010in}{3.102583in}}{\pgfqpoint{1.506411in}{3.098193in}}{\pgfqpoint{1.498598in}{3.090379in}}%
\pgfpathcurveto{\pgfqpoint{1.490784in}{3.082566in}}{\pgfqpoint{1.486394in}{3.071967in}}{\pgfqpoint{1.486394in}{3.060916in}}%
\pgfpathcurveto{\pgfqpoint{1.486394in}{3.049866in}}{\pgfqpoint{1.490784in}{3.039267in}}{\pgfqpoint{1.498598in}{3.031454in}}%
\pgfpathcurveto{\pgfqpoint{1.506411in}{3.023640in}}{\pgfqpoint{1.517010in}{3.019250in}}{\pgfqpoint{1.528060in}{3.019250in}}%
\pgfpathclose%
\pgfusepath{stroke,fill}%
\end{pgfscope}%
\begin{pgfscope}%
\pgfpathrectangle{\pgfqpoint{0.511823in}{0.504323in}}{\pgfqpoint{3.218177in}{3.225677in}} %
\pgfusepath{clip}%
\pgfsetbuttcap%
\pgfsetroundjoin%
\definecolor{currentfill}{rgb}{0.501961,0.000000,0.000000}%
\pgfsetfillcolor{currentfill}%
\pgfsetfillopacity{0.400000}%
\pgfsetlinewidth{0.501875pt}%
\definecolor{currentstroke}{rgb}{0.501961,0.000000,0.000000}%
\pgfsetstrokecolor{currentstroke}%
\pgfsetstrokeopacity{0.400000}%
\pgfsetdash{}{0pt}%
\pgfpathmoveto{\pgfqpoint{1.491894in}{2.941371in}}%
\pgfpathcurveto{\pgfqpoint{1.502944in}{2.941371in}}{\pgfqpoint{1.513543in}{2.945761in}}{\pgfqpoint{1.521356in}{2.953575in}}%
\pgfpathcurveto{\pgfqpoint{1.529170in}{2.961388in}}{\pgfqpoint{1.533560in}{2.971987in}}{\pgfqpoint{1.533560in}{2.983037in}}%
\pgfpathcurveto{\pgfqpoint{1.533560in}{2.994087in}}{\pgfqpoint{1.529170in}{3.004686in}}{\pgfqpoint{1.521356in}{3.012500in}}%
\pgfpathcurveto{\pgfqpoint{1.513543in}{3.020314in}}{\pgfqpoint{1.502944in}{3.024704in}}{\pgfqpoint{1.491894in}{3.024704in}}%
\pgfpathcurveto{\pgfqpoint{1.480843in}{3.024704in}}{\pgfqpoint{1.470244in}{3.020314in}}{\pgfqpoint{1.462431in}{3.012500in}}%
\pgfpathcurveto{\pgfqpoint{1.454617in}{3.004686in}}{\pgfqpoint{1.450227in}{2.994087in}}{\pgfqpoint{1.450227in}{2.983037in}}%
\pgfpathcurveto{\pgfqpoint{1.450227in}{2.971987in}}{\pgfqpoint{1.454617in}{2.961388in}}{\pgfqpoint{1.462431in}{2.953575in}}%
\pgfpathcurveto{\pgfqpoint{1.470244in}{2.945761in}}{\pgfqpoint{1.480843in}{2.941371in}}{\pgfqpoint{1.491894in}{2.941371in}}%
\pgfpathclose%
\pgfusepath{stroke,fill}%
\end{pgfscope}%
\begin{pgfscope}%
\pgfpathrectangle{\pgfqpoint{0.511823in}{0.504323in}}{\pgfqpoint{3.218177in}{3.225677in}} %
\pgfusepath{clip}%
\pgfsetbuttcap%
\pgfsetroundjoin%
\definecolor{currentfill}{rgb}{0.501961,0.000000,0.000000}%
\pgfsetfillcolor{currentfill}%
\pgfsetfillopacity{0.400000}%
\pgfsetlinewidth{0.501875pt}%
\definecolor{currentstroke}{rgb}{0.501961,0.000000,0.000000}%
\pgfsetstrokecolor{currentstroke}%
\pgfsetstrokeopacity{0.400000}%
\pgfsetdash{}{0pt}%
\pgfpathmoveto{\pgfqpoint{1.433029in}{2.796198in}}%
\pgfpathcurveto{\pgfqpoint{1.444079in}{2.796198in}}{\pgfqpoint{1.454678in}{2.800588in}}{\pgfqpoint{1.462492in}{2.808402in}}%
\pgfpathcurveto{\pgfqpoint{1.470306in}{2.816216in}}{\pgfqpoint{1.474696in}{2.826815in}}{\pgfqpoint{1.474696in}{2.837865in}}%
\pgfpathcurveto{\pgfqpoint{1.474696in}{2.848915in}}{\pgfqpoint{1.470306in}{2.859514in}}{\pgfqpoint{1.462492in}{2.867327in}}%
\pgfpathcurveto{\pgfqpoint{1.454678in}{2.875141in}}{\pgfqpoint{1.444079in}{2.879531in}}{\pgfqpoint{1.433029in}{2.879531in}}%
\pgfpathcurveto{\pgfqpoint{1.421979in}{2.879531in}}{\pgfqpoint{1.411380in}{2.875141in}}{\pgfqpoint{1.403566in}{2.867327in}}%
\pgfpathcurveto{\pgfqpoint{1.395753in}{2.859514in}}{\pgfqpoint{1.391362in}{2.848915in}}{\pgfqpoint{1.391362in}{2.837865in}}%
\pgfpathcurveto{\pgfqpoint{1.391362in}{2.826815in}}{\pgfqpoint{1.395753in}{2.816216in}}{\pgfqpoint{1.403566in}{2.808402in}}%
\pgfpathcurveto{\pgfqpoint{1.411380in}{2.800588in}}{\pgfqpoint{1.421979in}{2.796198in}}{\pgfqpoint{1.433029in}{2.796198in}}%
\pgfpathclose%
\pgfusepath{stroke,fill}%
\end{pgfscope}%
\begin{pgfscope}%
\pgfpathrectangle{\pgfqpoint{0.511823in}{0.504323in}}{\pgfqpoint{3.218177in}{3.225677in}} %
\pgfusepath{clip}%
\pgfsetbuttcap%
\pgfsetroundjoin%
\definecolor{currentfill}{rgb}{0.501961,0.000000,0.000000}%
\pgfsetfillcolor{currentfill}%
\pgfsetfillopacity{0.400000}%
\pgfsetlinewidth{0.501875pt}%
\definecolor{currentstroke}{rgb}{0.501961,0.000000,0.000000}%
\pgfsetstrokecolor{currentstroke}%
\pgfsetstrokeopacity{0.400000}%
\pgfsetdash{}{0pt}%
\pgfpathmoveto{\pgfqpoint{1.419783in}{2.781009in}}%
\pgfpathcurveto{\pgfqpoint{1.430833in}{2.781009in}}{\pgfqpoint{1.441432in}{2.785399in}}{\pgfqpoint{1.449245in}{2.793213in}}%
\pgfpathcurveto{\pgfqpoint{1.457059in}{2.801027in}}{\pgfqpoint{1.461449in}{2.811626in}}{\pgfqpoint{1.461449in}{2.822676in}}%
\pgfpathcurveto{\pgfqpoint{1.461449in}{2.833726in}}{\pgfqpoint{1.457059in}{2.844325in}}{\pgfqpoint{1.449245in}{2.852139in}}%
\pgfpathcurveto{\pgfqpoint{1.441432in}{2.859952in}}{\pgfqpoint{1.430833in}{2.864343in}}{\pgfqpoint{1.419783in}{2.864343in}}%
\pgfpathcurveto{\pgfqpoint{1.408733in}{2.864343in}}{\pgfqpoint{1.398133in}{2.859952in}}{\pgfqpoint{1.390320in}{2.852139in}}%
\pgfpathcurveto{\pgfqpoint{1.382506in}{2.844325in}}{\pgfqpoint{1.378116in}{2.833726in}}{\pgfqpoint{1.378116in}{2.822676in}}%
\pgfpathcurveto{\pgfqpoint{1.378116in}{2.811626in}}{\pgfqpoint{1.382506in}{2.801027in}}{\pgfqpoint{1.390320in}{2.793213in}}%
\pgfpathcurveto{\pgfqpoint{1.398133in}{2.785399in}}{\pgfqpoint{1.408733in}{2.781009in}}{\pgfqpoint{1.419783in}{2.781009in}}%
\pgfpathclose%
\pgfusepath{stroke,fill}%
\end{pgfscope}%
\begin{pgfscope}%
\pgfpathrectangle{\pgfqpoint{0.511823in}{0.504323in}}{\pgfqpoint{3.218177in}{3.225677in}} %
\pgfusepath{clip}%
\pgfsetbuttcap%
\pgfsetroundjoin%
\definecolor{currentfill}{rgb}{0.501961,0.000000,0.000000}%
\pgfsetfillcolor{currentfill}%
\pgfsetfillopacity{0.400000}%
\pgfsetlinewidth{0.501875pt}%
\definecolor{currentstroke}{rgb}{0.501961,0.000000,0.000000}%
\pgfsetstrokecolor{currentstroke}%
\pgfsetstrokeopacity{0.400000}%
\pgfsetdash{}{0pt}%
\pgfpathmoveto{\pgfqpoint{1.445011in}{2.879014in}}%
\pgfpathcurveto{\pgfqpoint{1.456061in}{2.879014in}}{\pgfqpoint{1.466660in}{2.883405in}}{\pgfqpoint{1.474474in}{2.891218in}}%
\pgfpathcurveto{\pgfqpoint{1.482288in}{2.899032in}}{\pgfqpoint{1.486678in}{2.909631in}}{\pgfqpoint{1.486678in}{2.920681in}}%
\pgfpathcurveto{\pgfqpoint{1.486678in}{2.931731in}}{\pgfqpoint{1.482288in}{2.942330in}}{\pgfqpoint{1.474474in}{2.950144in}}%
\pgfpathcurveto{\pgfqpoint{1.466660in}{2.957957in}}{\pgfqpoint{1.456061in}{2.962348in}}{\pgfqpoint{1.445011in}{2.962348in}}%
\pgfpathcurveto{\pgfqpoint{1.433961in}{2.962348in}}{\pgfqpoint{1.423362in}{2.957957in}}{\pgfqpoint{1.415548in}{2.950144in}}%
\pgfpathcurveto{\pgfqpoint{1.407735in}{2.942330in}}{\pgfqpoint{1.403344in}{2.931731in}}{\pgfqpoint{1.403344in}{2.920681in}}%
\pgfpathcurveto{\pgfqpoint{1.403344in}{2.909631in}}{\pgfqpoint{1.407735in}{2.899032in}}{\pgfqpoint{1.415548in}{2.891218in}}%
\pgfpathcurveto{\pgfqpoint{1.423362in}{2.883405in}}{\pgfqpoint{1.433961in}{2.879014in}}{\pgfqpoint{1.445011in}{2.879014in}}%
\pgfpathclose%
\pgfusepath{stroke,fill}%
\end{pgfscope}%
\begin{pgfscope}%
\pgfpathrectangle{\pgfqpoint{0.511823in}{0.504323in}}{\pgfqpoint{3.218177in}{3.225677in}} %
\pgfusepath{clip}%
\pgfsetbuttcap%
\pgfsetroundjoin%
\definecolor{currentfill}{rgb}{0.501961,0.000000,0.000000}%
\pgfsetfillcolor{currentfill}%
\pgfsetfillopacity{0.400000}%
\pgfsetlinewidth{0.501875pt}%
\definecolor{currentstroke}{rgb}{0.501961,0.000000,0.000000}%
\pgfsetstrokecolor{currentstroke}%
\pgfsetstrokeopacity{0.400000}%
\pgfsetdash{}{0pt}%
\pgfpathmoveto{\pgfqpoint{1.436197in}{2.877540in}}%
\pgfpathcurveto{\pgfqpoint{1.447247in}{2.877540in}}{\pgfqpoint{1.457846in}{2.881931in}}{\pgfqpoint{1.465660in}{2.889744in}}%
\pgfpathcurveto{\pgfqpoint{1.473473in}{2.897558in}}{\pgfqpoint{1.477864in}{2.908157in}}{\pgfqpoint{1.477864in}{2.919207in}}%
\pgfpathcurveto{\pgfqpoint{1.477864in}{2.930257in}}{\pgfqpoint{1.473473in}{2.940856in}}{\pgfqpoint{1.465660in}{2.948670in}}%
\pgfpathcurveto{\pgfqpoint{1.457846in}{2.956483in}}{\pgfqpoint{1.447247in}{2.960874in}}{\pgfqpoint{1.436197in}{2.960874in}}%
\pgfpathcurveto{\pgfqpoint{1.425147in}{2.960874in}}{\pgfqpoint{1.414548in}{2.956483in}}{\pgfqpoint{1.406734in}{2.948670in}}%
\pgfpathcurveto{\pgfqpoint{1.398921in}{2.940856in}}{\pgfqpoint{1.394530in}{2.930257in}}{\pgfqpoint{1.394530in}{2.919207in}}%
\pgfpathcurveto{\pgfqpoint{1.394530in}{2.908157in}}{\pgfqpoint{1.398921in}{2.897558in}}{\pgfqpoint{1.406734in}{2.889744in}}%
\pgfpathcurveto{\pgfqpoint{1.414548in}{2.881931in}}{\pgfqpoint{1.425147in}{2.877540in}}{\pgfqpoint{1.436197in}{2.877540in}}%
\pgfpathclose%
\pgfusepath{stroke,fill}%
\end{pgfscope}%
\begin{pgfscope}%
\pgfpathrectangle{\pgfqpoint{0.511823in}{0.504323in}}{\pgfqpoint{3.218177in}{3.225677in}} %
\pgfusepath{clip}%
\pgfsetbuttcap%
\pgfsetroundjoin%
\definecolor{currentfill}{rgb}{0.501961,0.000000,0.000000}%
\pgfsetfillcolor{currentfill}%
\pgfsetfillopacity{0.400000}%
\pgfsetlinewidth{0.501875pt}%
\definecolor{currentstroke}{rgb}{0.501961,0.000000,0.000000}%
\pgfsetstrokecolor{currentstroke}%
\pgfsetstrokeopacity{0.400000}%
\pgfsetdash{}{0pt}%
\pgfpathmoveto{\pgfqpoint{1.444238in}{2.926852in}}%
\pgfpathcurveto{\pgfqpoint{1.455288in}{2.926852in}}{\pgfqpoint{1.465887in}{2.931243in}}{\pgfqpoint{1.473700in}{2.939056in}}%
\pgfpathcurveto{\pgfqpoint{1.481514in}{2.946870in}}{\pgfqpoint{1.485904in}{2.957469in}}{\pgfqpoint{1.485904in}{2.968519in}}%
\pgfpathcurveto{\pgfqpoint{1.485904in}{2.979569in}}{\pgfqpoint{1.481514in}{2.990168in}}{\pgfqpoint{1.473700in}{2.997982in}}%
\pgfpathcurveto{\pgfqpoint{1.465887in}{3.005796in}}{\pgfqpoint{1.455288in}{3.010186in}}{\pgfqpoint{1.444238in}{3.010186in}}%
\pgfpathcurveto{\pgfqpoint{1.433187in}{3.010186in}}{\pgfqpoint{1.422588in}{3.005796in}}{\pgfqpoint{1.414775in}{2.997982in}}%
\pgfpathcurveto{\pgfqpoint{1.406961in}{2.990168in}}{\pgfqpoint{1.402571in}{2.979569in}}{\pgfqpoint{1.402571in}{2.968519in}}%
\pgfpathcurveto{\pgfqpoint{1.402571in}{2.957469in}}{\pgfqpoint{1.406961in}{2.946870in}}{\pgfqpoint{1.414775in}{2.939056in}}%
\pgfpathcurveto{\pgfqpoint{1.422588in}{2.931243in}}{\pgfqpoint{1.433187in}{2.926852in}}{\pgfqpoint{1.444238in}{2.926852in}}%
\pgfpathclose%
\pgfusepath{stroke,fill}%
\end{pgfscope}%
\begin{pgfscope}%
\pgfpathrectangle{\pgfqpoint{0.511823in}{0.504323in}}{\pgfqpoint{3.218177in}{3.225677in}} %
\pgfusepath{clip}%
\pgfsetbuttcap%
\pgfsetroundjoin%
\definecolor{currentfill}{rgb}{0.501961,0.000000,0.000000}%
\pgfsetfillcolor{currentfill}%
\pgfsetfillopacity{0.400000}%
\pgfsetlinewidth{0.501875pt}%
\definecolor{currentstroke}{rgb}{0.501961,0.000000,0.000000}%
\pgfsetstrokecolor{currentstroke}%
\pgfsetstrokeopacity{0.400000}%
\pgfsetdash{}{0pt}%
\pgfpathmoveto{\pgfqpoint{1.419138in}{2.876080in}}%
\pgfpathcurveto{\pgfqpoint{1.430188in}{2.876080in}}{\pgfqpoint{1.440787in}{2.880470in}}{\pgfqpoint{1.448601in}{2.888284in}}%
\pgfpathcurveto{\pgfqpoint{1.456414in}{2.896098in}}{\pgfqpoint{1.460805in}{2.906697in}}{\pgfqpoint{1.460805in}{2.917747in}}%
\pgfpathcurveto{\pgfqpoint{1.460805in}{2.928797in}}{\pgfqpoint{1.456414in}{2.939396in}}{\pgfqpoint{1.448601in}{2.947210in}}%
\pgfpathcurveto{\pgfqpoint{1.440787in}{2.955023in}}{\pgfqpoint{1.430188in}{2.959414in}}{\pgfqpoint{1.419138in}{2.959414in}}%
\pgfpathcurveto{\pgfqpoint{1.408088in}{2.959414in}}{\pgfqpoint{1.397489in}{2.955023in}}{\pgfqpoint{1.389675in}{2.947210in}}%
\pgfpathcurveto{\pgfqpoint{1.381862in}{2.939396in}}{\pgfqpoint{1.377471in}{2.928797in}}{\pgfqpoint{1.377471in}{2.917747in}}%
\pgfpathcurveto{\pgfqpoint{1.377471in}{2.906697in}}{\pgfqpoint{1.381862in}{2.896098in}}{\pgfqpoint{1.389675in}{2.888284in}}%
\pgfpathcurveto{\pgfqpoint{1.397489in}{2.880470in}}{\pgfqpoint{1.408088in}{2.876080in}}{\pgfqpoint{1.419138in}{2.876080in}}%
\pgfpathclose%
\pgfusepath{stroke,fill}%
\end{pgfscope}%
\begin{pgfscope}%
\pgfpathrectangle{\pgfqpoint{0.511823in}{0.504323in}}{\pgfqpoint{3.218177in}{3.225677in}} %
\pgfusepath{clip}%
\pgfsetbuttcap%
\pgfsetroundjoin%
\definecolor{currentfill}{rgb}{0.501961,0.000000,0.000000}%
\pgfsetfillcolor{currentfill}%
\pgfsetfillopacity{0.400000}%
\pgfsetlinewidth{0.501875pt}%
\definecolor{currentstroke}{rgb}{0.501961,0.000000,0.000000}%
\pgfsetstrokecolor{currentstroke}%
\pgfsetstrokeopacity{0.400000}%
\pgfsetdash{}{0pt}%
\pgfpathmoveto{\pgfqpoint{1.446924in}{2.987356in}}%
\pgfpathcurveto{\pgfqpoint{1.457974in}{2.987356in}}{\pgfqpoint{1.468573in}{2.991746in}}{\pgfqpoint{1.476387in}{2.999560in}}%
\pgfpathcurveto{\pgfqpoint{1.484200in}{3.007374in}}{\pgfqpoint{1.488590in}{3.017973in}}{\pgfqpoint{1.488590in}{3.029023in}}%
\pgfpathcurveto{\pgfqpoint{1.488590in}{3.040073in}}{\pgfqpoint{1.484200in}{3.050672in}}{\pgfqpoint{1.476387in}{3.058485in}}%
\pgfpathcurveto{\pgfqpoint{1.468573in}{3.066299in}}{\pgfqpoint{1.457974in}{3.070689in}}{\pgfqpoint{1.446924in}{3.070689in}}%
\pgfpathcurveto{\pgfqpoint{1.435874in}{3.070689in}}{\pgfqpoint{1.425275in}{3.066299in}}{\pgfqpoint{1.417461in}{3.058485in}}%
\pgfpathcurveto{\pgfqpoint{1.409647in}{3.050672in}}{\pgfqpoint{1.405257in}{3.040073in}}{\pgfqpoint{1.405257in}{3.029023in}}%
\pgfpathcurveto{\pgfqpoint{1.405257in}{3.017973in}}{\pgfqpoint{1.409647in}{3.007374in}}{\pgfqpoint{1.417461in}{2.999560in}}%
\pgfpathcurveto{\pgfqpoint{1.425275in}{2.991746in}}{\pgfqpoint{1.435874in}{2.987356in}}{\pgfqpoint{1.446924in}{2.987356in}}%
\pgfpathclose%
\pgfusepath{stroke,fill}%
\end{pgfscope}%
\begin{pgfscope}%
\pgfpathrectangle{\pgfqpoint{0.511823in}{0.504323in}}{\pgfqpoint{3.218177in}{3.225677in}} %
\pgfusepath{clip}%
\pgfsetbuttcap%
\pgfsetroundjoin%
\definecolor{currentfill}{rgb}{0.501961,0.000000,0.000000}%
\pgfsetfillcolor{currentfill}%
\pgfsetfillopacity{0.400000}%
\pgfsetlinewidth{0.501875pt}%
\definecolor{currentstroke}{rgb}{0.501961,0.000000,0.000000}%
\pgfsetstrokecolor{currentstroke}%
\pgfsetstrokeopacity{0.400000}%
\pgfsetdash{}{0pt}%
\pgfpathmoveto{\pgfqpoint{1.421505in}{2.935012in}}%
\pgfpathcurveto{\pgfqpoint{1.432555in}{2.935012in}}{\pgfqpoint{1.443154in}{2.939402in}}{\pgfqpoint{1.450967in}{2.947216in}}%
\pgfpathcurveto{\pgfqpoint{1.458781in}{2.955030in}}{\pgfqpoint{1.463171in}{2.965629in}}{\pgfqpoint{1.463171in}{2.976679in}}%
\pgfpathcurveto{\pgfqpoint{1.463171in}{2.987729in}}{\pgfqpoint{1.458781in}{2.998328in}}{\pgfqpoint{1.450967in}{3.006142in}}%
\pgfpathcurveto{\pgfqpoint{1.443154in}{3.013955in}}{\pgfqpoint{1.432555in}{3.018345in}}{\pgfqpoint{1.421505in}{3.018345in}}%
\pgfpathcurveto{\pgfqpoint{1.410455in}{3.018345in}}{\pgfqpoint{1.399856in}{3.013955in}}{\pgfqpoint{1.392042in}{3.006142in}}%
\pgfpathcurveto{\pgfqpoint{1.384228in}{2.998328in}}{\pgfqpoint{1.379838in}{2.987729in}}{\pgfqpoint{1.379838in}{2.976679in}}%
\pgfpathcurveto{\pgfqpoint{1.379838in}{2.965629in}}{\pgfqpoint{1.384228in}{2.955030in}}{\pgfqpoint{1.392042in}{2.947216in}}%
\pgfpathcurveto{\pgfqpoint{1.399856in}{2.939402in}}{\pgfqpoint{1.410455in}{2.935012in}}{\pgfqpoint{1.421505in}{2.935012in}}%
\pgfpathclose%
\pgfusepath{stroke,fill}%
\end{pgfscope}%
\begin{pgfscope}%
\pgfpathrectangle{\pgfqpoint{0.511823in}{0.504323in}}{\pgfqpoint{3.218177in}{3.225677in}} %
\pgfusepath{clip}%
\pgfsetbuttcap%
\pgfsetroundjoin%
\definecolor{currentfill}{rgb}{0.501961,0.000000,0.000000}%
\pgfsetfillcolor{currentfill}%
\pgfsetfillopacity{0.400000}%
\pgfsetlinewidth{0.501875pt}%
\definecolor{currentstroke}{rgb}{0.501961,0.000000,0.000000}%
\pgfsetstrokecolor{currentstroke}%
\pgfsetstrokeopacity{0.400000}%
\pgfsetdash{}{0pt}%
\pgfpathmoveto{\pgfqpoint{1.443874in}{3.032355in}}%
\pgfpathcurveto{\pgfqpoint{1.454924in}{3.032355in}}{\pgfqpoint{1.465523in}{3.036745in}}{\pgfqpoint{1.473337in}{3.044559in}}%
\pgfpathcurveto{\pgfqpoint{1.481151in}{3.052373in}}{\pgfqpoint{1.485541in}{3.062972in}}{\pgfqpoint{1.485541in}{3.074022in}}%
\pgfpathcurveto{\pgfqpoint{1.485541in}{3.085072in}}{\pgfqpoint{1.481151in}{3.095671in}}{\pgfqpoint{1.473337in}{3.103485in}}%
\pgfpathcurveto{\pgfqpoint{1.465523in}{3.111298in}}{\pgfqpoint{1.454924in}{3.115689in}}{\pgfqpoint{1.443874in}{3.115689in}}%
\pgfpathcurveto{\pgfqpoint{1.432824in}{3.115689in}}{\pgfqpoint{1.422225in}{3.111298in}}{\pgfqpoint{1.414411in}{3.103485in}}%
\pgfpathcurveto{\pgfqpoint{1.406598in}{3.095671in}}{\pgfqpoint{1.402208in}{3.085072in}}{\pgfqpoint{1.402208in}{3.074022in}}%
\pgfpathcurveto{\pgfqpoint{1.402208in}{3.062972in}}{\pgfqpoint{1.406598in}{3.052373in}}{\pgfqpoint{1.414411in}{3.044559in}}%
\pgfpathcurveto{\pgfqpoint{1.422225in}{3.036745in}}{\pgfqpoint{1.432824in}{3.032355in}}{\pgfqpoint{1.443874in}{3.032355in}}%
\pgfpathclose%
\pgfusepath{stroke,fill}%
\end{pgfscope}%
\begin{pgfscope}%
\pgfpathrectangle{\pgfqpoint{0.511823in}{0.504323in}}{\pgfqpoint{3.218177in}{3.225677in}} %
\pgfusepath{clip}%
\pgfsetbuttcap%
\pgfsetroundjoin%
\definecolor{currentfill}{rgb}{0.501961,0.000000,0.000000}%
\pgfsetfillcolor{currentfill}%
\pgfsetfillopacity{0.400000}%
\pgfsetlinewidth{0.501875pt}%
\definecolor{currentstroke}{rgb}{0.501961,0.000000,0.000000}%
\pgfsetstrokecolor{currentstroke}%
\pgfsetstrokeopacity{0.400000}%
\pgfsetdash{}{0pt}%
\pgfpathmoveto{\pgfqpoint{1.428245in}{3.010493in}}%
\pgfpathcurveto{\pgfqpoint{1.439295in}{3.010493in}}{\pgfqpoint{1.449894in}{3.014883in}}{\pgfqpoint{1.457708in}{3.022697in}}%
\pgfpathcurveto{\pgfqpoint{1.465521in}{3.030510in}}{\pgfqpoint{1.469911in}{3.041109in}}{\pgfqpoint{1.469911in}{3.052160in}}%
\pgfpathcurveto{\pgfqpoint{1.469911in}{3.063210in}}{\pgfqpoint{1.465521in}{3.073809in}}{\pgfqpoint{1.457708in}{3.081622in}}%
\pgfpathcurveto{\pgfqpoint{1.449894in}{3.089436in}}{\pgfqpoint{1.439295in}{3.093826in}}{\pgfqpoint{1.428245in}{3.093826in}}%
\pgfpathcurveto{\pgfqpoint{1.417195in}{3.093826in}}{\pgfqpoint{1.406596in}{3.089436in}}{\pgfqpoint{1.398782in}{3.081622in}}%
\pgfpathcurveto{\pgfqpoint{1.390968in}{3.073809in}}{\pgfqpoint{1.386578in}{3.063210in}}{\pgfqpoint{1.386578in}{3.052160in}}%
\pgfpathcurveto{\pgfqpoint{1.386578in}{3.041109in}}{\pgfqpoint{1.390968in}{3.030510in}}{\pgfqpoint{1.398782in}{3.022697in}}%
\pgfpathcurveto{\pgfqpoint{1.406596in}{3.014883in}}{\pgfqpoint{1.417195in}{3.010493in}}{\pgfqpoint{1.428245in}{3.010493in}}%
\pgfpathclose%
\pgfusepath{stroke,fill}%
\end{pgfscope}%
\begin{pgfscope}%
\pgfpathrectangle{\pgfqpoint{0.511823in}{0.504323in}}{\pgfqpoint{3.218177in}{3.225677in}} %
\pgfusepath{clip}%
\pgfsetbuttcap%
\pgfsetroundjoin%
\definecolor{currentfill}{rgb}{0.501961,0.000000,0.000000}%
\pgfsetfillcolor{currentfill}%
\pgfsetfillopacity{0.400000}%
\pgfsetlinewidth{0.501875pt}%
\definecolor{currentstroke}{rgb}{0.501961,0.000000,0.000000}%
\pgfsetstrokecolor{currentstroke}%
\pgfsetstrokeopacity{0.400000}%
\pgfsetdash{}{0pt}%
\pgfpathmoveto{\pgfqpoint{1.442113in}{3.083449in}}%
\pgfpathcurveto{\pgfqpoint{1.453163in}{3.083449in}}{\pgfqpoint{1.463762in}{3.087839in}}{\pgfqpoint{1.471576in}{3.095653in}}%
\pgfpathcurveto{\pgfqpoint{1.479389in}{3.103466in}}{\pgfqpoint{1.483779in}{3.114065in}}{\pgfqpoint{1.483779in}{3.125115in}}%
\pgfpathcurveto{\pgfqpoint{1.483779in}{3.136165in}}{\pgfqpoint{1.479389in}{3.146764in}}{\pgfqpoint{1.471576in}{3.154578in}}%
\pgfpathcurveto{\pgfqpoint{1.463762in}{3.162392in}}{\pgfqpoint{1.453163in}{3.166782in}}{\pgfqpoint{1.442113in}{3.166782in}}%
\pgfpathcurveto{\pgfqpoint{1.431063in}{3.166782in}}{\pgfqpoint{1.420464in}{3.162392in}}{\pgfqpoint{1.412650in}{3.154578in}}%
\pgfpathcurveto{\pgfqpoint{1.404836in}{3.146764in}}{\pgfqpoint{1.400446in}{3.136165in}}{\pgfqpoint{1.400446in}{3.125115in}}%
\pgfpathcurveto{\pgfqpoint{1.400446in}{3.114065in}}{\pgfqpoint{1.404836in}{3.103466in}}{\pgfqpoint{1.412650in}{3.095653in}}%
\pgfpathcurveto{\pgfqpoint{1.420464in}{3.087839in}}{\pgfqpoint{1.431063in}{3.083449in}}{\pgfqpoint{1.442113in}{3.083449in}}%
\pgfpathclose%
\pgfusepath{stroke,fill}%
\end{pgfscope}%
\begin{pgfscope}%
\pgfpathrectangle{\pgfqpoint{0.511823in}{0.504323in}}{\pgfqpoint{3.218177in}{3.225677in}} %
\pgfusepath{clip}%
\pgfsetbuttcap%
\pgfsetroundjoin%
\definecolor{currentfill}{rgb}{0.501961,0.000000,0.000000}%
\pgfsetfillcolor{currentfill}%
\pgfsetfillopacity{0.400000}%
\pgfsetlinewidth{0.501875pt}%
\definecolor{currentstroke}{rgb}{0.501961,0.000000,0.000000}%
\pgfsetstrokecolor{currentstroke}%
\pgfsetstrokeopacity{0.400000}%
\pgfsetdash{}{0pt}%
\pgfpathmoveto{\pgfqpoint{1.377208in}{2.900315in}}%
\pgfpathcurveto{\pgfqpoint{1.388258in}{2.900315in}}{\pgfqpoint{1.398857in}{2.904705in}}{\pgfqpoint{1.406671in}{2.912519in}}%
\pgfpathcurveto{\pgfqpoint{1.414484in}{2.920332in}}{\pgfqpoint{1.418874in}{2.930931in}}{\pgfqpoint{1.418874in}{2.941981in}}%
\pgfpathcurveto{\pgfqpoint{1.418874in}{2.953032in}}{\pgfqpoint{1.414484in}{2.963631in}}{\pgfqpoint{1.406671in}{2.971444in}}%
\pgfpathcurveto{\pgfqpoint{1.398857in}{2.979258in}}{\pgfqpoint{1.388258in}{2.983648in}}{\pgfqpoint{1.377208in}{2.983648in}}%
\pgfpathcurveto{\pgfqpoint{1.366158in}{2.983648in}}{\pgfqpoint{1.355559in}{2.979258in}}{\pgfqpoint{1.347745in}{2.971444in}}%
\pgfpathcurveto{\pgfqpoint{1.339931in}{2.963631in}}{\pgfqpoint{1.335541in}{2.953032in}}{\pgfqpoint{1.335541in}{2.941981in}}%
\pgfpathcurveto{\pgfqpoint{1.335541in}{2.930931in}}{\pgfqpoint{1.339931in}{2.920332in}}{\pgfqpoint{1.347745in}{2.912519in}}%
\pgfpathcurveto{\pgfqpoint{1.355559in}{2.904705in}}{\pgfqpoint{1.366158in}{2.900315in}}{\pgfqpoint{1.377208in}{2.900315in}}%
\pgfpathclose%
\pgfusepath{stroke,fill}%
\end{pgfscope}%
\begin{pgfscope}%
\pgfpathrectangle{\pgfqpoint{0.511823in}{0.504323in}}{\pgfqpoint{3.218177in}{3.225677in}} %
\pgfusepath{clip}%
\pgfsetbuttcap%
\pgfsetroundjoin%
\definecolor{currentfill}{rgb}{0.501961,0.000000,0.000000}%
\pgfsetfillcolor{currentfill}%
\pgfsetfillopacity{0.400000}%
\pgfsetlinewidth{0.501875pt}%
\definecolor{currentstroke}{rgb}{0.501961,0.000000,0.000000}%
\pgfsetstrokecolor{currentstroke}%
\pgfsetstrokeopacity{0.400000}%
\pgfsetdash{}{0pt}%
\pgfpathmoveto{\pgfqpoint{1.372219in}{2.911144in}}%
\pgfpathcurveto{\pgfqpoint{1.383269in}{2.911144in}}{\pgfqpoint{1.393868in}{2.915534in}}{\pgfqpoint{1.401682in}{2.923348in}}%
\pgfpathcurveto{\pgfqpoint{1.409495in}{2.931162in}}{\pgfqpoint{1.413885in}{2.941761in}}{\pgfqpoint{1.413885in}{2.952811in}}%
\pgfpathcurveto{\pgfqpoint{1.413885in}{2.963861in}}{\pgfqpoint{1.409495in}{2.974460in}}{\pgfqpoint{1.401682in}{2.982274in}}%
\pgfpathcurveto{\pgfqpoint{1.393868in}{2.990087in}}{\pgfqpoint{1.383269in}{2.994477in}}{\pgfqpoint{1.372219in}{2.994477in}}%
\pgfpathcurveto{\pgfqpoint{1.361169in}{2.994477in}}{\pgfqpoint{1.350570in}{2.990087in}}{\pgfqpoint{1.342756in}{2.982274in}}%
\pgfpathcurveto{\pgfqpoint{1.334942in}{2.974460in}}{\pgfqpoint{1.330552in}{2.963861in}}{\pgfqpoint{1.330552in}{2.952811in}}%
\pgfpathcurveto{\pgfqpoint{1.330552in}{2.941761in}}{\pgfqpoint{1.334942in}{2.931162in}}{\pgfqpoint{1.342756in}{2.923348in}}%
\pgfpathcurveto{\pgfqpoint{1.350570in}{2.915534in}}{\pgfqpoint{1.361169in}{2.911144in}}{\pgfqpoint{1.372219in}{2.911144in}}%
\pgfpathclose%
\pgfusepath{stroke,fill}%
\end{pgfscope}%
\begin{pgfscope}%
\pgfpathrectangle{\pgfqpoint{0.511823in}{0.504323in}}{\pgfqpoint{3.218177in}{3.225677in}} %
\pgfusepath{clip}%
\pgfsetbuttcap%
\pgfsetroundjoin%
\definecolor{currentfill}{rgb}{0.501961,0.000000,0.000000}%
\pgfsetfillcolor{currentfill}%
\pgfsetfillopacity{0.400000}%
\pgfsetlinewidth{0.501875pt}%
\definecolor{currentstroke}{rgb}{0.501961,0.000000,0.000000}%
\pgfsetstrokecolor{currentstroke}%
\pgfsetstrokeopacity{0.400000}%
\pgfsetdash{}{0pt}%
\pgfpathmoveto{\pgfqpoint{1.356556in}{2.886406in}}%
\pgfpathcurveto{\pgfqpoint{1.367606in}{2.886406in}}{\pgfqpoint{1.378206in}{2.890796in}}{\pgfqpoint{1.386019in}{2.898610in}}%
\pgfpathcurveto{\pgfqpoint{1.393833in}{2.906423in}}{\pgfqpoint{1.398223in}{2.917022in}}{\pgfqpoint{1.398223in}{2.928073in}}%
\pgfpathcurveto{\pgfqpoint{1.398223in}{2.939123in}}{\pgfqpoint{1.393833in}{2.949722in}}{\pgfqpoint{1.386019in}{2.957535in}}%
\pgfpathcurveto{\pgfqpoint{1.378206in}{2.965349in}}{\pgfqpoint{1.367606in}{2.969739in}}{\pgfqpoint{1.356556in}{2.969739in}}%
\pgfpathcurveto{\pgfqpoint{1.345506in}{2.969739in}}{\pgfqpoint{1.334907in}{2.965349in}}{\pgfqpoint{1.327094in}{2.957535in}}%
\pgfpathcurveto{\pgfqpoint{1.319280in}{2.949722in}}{\pgfqpoint{1.314890in}{2.939123in}}{\pgfqpoint{1.314890in}{2.928073in}}%
\pgfpathcurveto{\pgfqpoint{1.314890in}{2.917022in}}{\pgfqpoint{1.319280in}{2.906423in}}{\pgfqpoint{1.327094in}{2.898610in}}%
\pgfpathcurveto{\pgfqpoint{1.334907in}{2.890796in}}{\pgfqpoint{1.345506in}{2.886406in}}{\pgfqpoint{1.356556in}{2.886406in}}%
\pgfpathclose%
\pgfusepath{stroke,fill}%
\end{pgfscope}%
\begin{pgfscope}%
\pgfpathrectangle{\pgfqpoint{0.511823in}{0.504323in}}{\pgfqpoint{3.218177in}{3.225677in}} %
\pgfusepath{clip}%
\pgfsetbuttcap%
\pgfsetroundjoin%
\definecolor{currentfill}{rgb}{0.501961,0.000000,0.000000}%
\pgfsetfillcolor{currentfill}%
\pgfsetfillopacity{0.400000}%
\pgfsetlinewidth{0.501875pt}%
\definecolor{currentstroke}{rgb}{0.501961,0.000000,0.000000}%
\pgfsetstrokecolor{currentstroke}%
\pgfsetstrokeopacity{0.400000}%
\pgfsetdash{}{0pt}%
\pgfpathmoveto{\pgfqpoint{1.391225in}{3.031833in}}%
\pgfpathcurveto{\pgfqpoint{1.402275in}{3.031833in}}{\pgfqpoint{1.412874in}{3.036224in}}{\pgfqpoint{1.420688in}{3.044037in}}%
\pgfpathcurveto{\pgfqpoint{1.428502in}{3.051851in}}{\pgfqpoint{1.432892in}{3.062450in}}{\pgfqpoint{1.432892in}{3.073500in}}%
\pgfpathcurveto{\pgfqpoint{1.432892in}{3.084550in}}{\pgfqpoint{1.428502in}{3.095149in}}{\pgfqpoint{1.420688in}{3.102963in}}%
\pgfpathcurveto{\pgfqpoint{1.412874in}{3.110777in}}{\pgfqpoint{1.402275in}{3.115167in}}{\pgfqpoint{1.391225in}{3.115167in}}%
\pgfpathcurveto{\pgfqpoint{1.380175in}{3.115167in}}{\pgfqpoint{1.369576in}{3.110777in}}{\pgfqpoint{1.361762in}{3.102963in}}%
\pgfpathcurveto{\pgfqpoint{1.353949in}{3.095149in}}{\pgfqpoint{1.349558in}{3.084550in}}{\pgfqpoint{1.349558in}{3.073500in}}%
\pgfpathcurveto{\pgfqpoint{1.349558in}{3.062450in}}{\pgfqpoint{1.353949in}{3.051851in}}{\pgfqpoint{1.361762in}{3.044037in}}%
\pgfpathcurveto{\pgfqpoint{1.369576in}{3.036224in}}{\pgfqpoint{1.380175in}{3.031833in}}{\pgfqpoint{1.391225in}{3.031833in}}%
\pgfpathclose%
\pgfusepath{stroke,fill}%
\end{pgfscope}%
\begin{pgfscope}%
\pgfpathrectangle{\pgfqpoint{0.511823in}{0.504323in}}{\pgfqpoint{3.218177in}{3.225677in}} %
\pgfusepath{clip}%
\pgfsetbuttcap%
\pgfsetroundjoin%
\definecolor{currentfill}{rgb}{0.501961,0.000000,0.000000}%
\pgfsetfillcolor{currentfill}%
\pgfsetfillopacity{0.400000}%
\pgfsetlinewidth{0.501875pt}%
\definecolor{currentstroke}{rgb}{0.501961,0.000000,0.000000}%
\pgfsetstrokecolor{currentstroke}%
\pgfsetstrokeopacity{0.400000}%
\pgfsetdash{}{0pt}%
\pgfpathmoveto{\pgfqpoint{1.405323in}{3.110153in}}%
\pgfpathcurveto{\pgfqpoint{1.416373in}{3.110153in}}{\pgfqpoint{1.426972in}{3.114543in}}{\pgfqpoint{1.434785in}{3.122357in}}%
\pgfpathcurveto{\pgfqpoint{1.442599in}{3.130170in}}{\pgfqpoint{1.446989in}{3.140769in}}{\pgfqpoint{1.446989in}{3.151820in}}%
\pgfpathcurveto{\pgfqpoint{1.446989in}{3.162870in}}{\pgfqpoint{1.442599in}{3.173469in}}{\pgfqpoint{1.434785in}{3.181282in}}%
\pgfpathcurveto{\pgfqpoint{1.426972in}{3.189096in}}{\pgfqpoint{1.416373in}{3.193486in}}{\pgfqpoint{1.405323in}{3.193486in}}%
\pgfpathcurveto{\pgfqpoint{1.394272in}{3.193486in}}{\pgfqpoint{1.383673in}{3.189096in}}{\pgfqpoint{1.375860in}{3.181282in}}%
\pgfpathcurveto{\pgfqpoint{1.368046in}{3.173469in}}{\pgfqpoint{1.363656in}{3.162870in}}{\pgfqpoint{1.363656in}{3.151820in}}%
\pgfpathcurveto{\pgfqpoint{1.363656in}{3.140769in}}{\pgfqpoint{1.368046in}{3.130170in}}{\pgfqpoint{1.375860in}{3.122357in}}%
\pgfpathcurveto{\pgfqpoint{1.383673in}{3.114543in}}{\pgfqpoint{1.394272in}{3.110153in}}{\pgfqpoint{1.405323in}{3.110153in}}%
\pgfpathclose%
\pgfusepath{stroke,fill}%
\end{pgfscope}%
\begin{pgfscope}%
\pgfpathrectangle{\pgfqpoint{0.511823in}{0.504323in}}{\pgfqpoint{3.218177in}{3.225677in}} %
\pgfusepath{clip}%
\pgfsetbuttcap%
\pgfsetroundjoin%
\definecolor{currentfill}{rgb}{0.501961,0.000000,0.000000}%
\pgfsetfillcolor{currentfill}%
\pgfsetfillopacity{0.400000}%
\pgfsetlinewidth{0.501875pt}%
\definecolor{currentstroke}{rgb}{0.501961,0.000000,0.000000}%
\pgfsetstrokecolor{currentstroke}%
\pgfsetstrokeopacity{0.400000}%
\pgfsetdash{}{0pt}%
\pgfpathmoveto{\pgfqpoint{1.411533in}{3.162954in}}%
\pgfpathcurveto{\pgfqpoint{1.422583in}{3.162954in}}{\pgfqpoint{1.433182in}{3.167344in}}{\pgfqpoint{1.440996in}{3.175158in}}%
\pgfpathcurveto{\pgfqpoint{1.448810in}{3.182971in}}{\pgfqpoint{1.453200in}{3.193571in}}{\pgfqpoint{1.453200in}{3.204621in}}%
\pgfpathcurveto{\pgfqpoint{1.453200in}{3.215671in}}{\pgfqpoint{1.448810in}{3.226270in}}{\pgfqpoint{1.440996in}{3.234083in}}%
\pgfpathcurveto{\pgfqpoint{1.433182in}{3.241897in}}{\pgfqpoint{1.422583in}{3.246287in}}{\pgfqpoint{1.411533in}{3.246287in}}%
\pgfpathcurveto{\pgfqpoint{1.400483in}{3.246287in}}{\pgfqpoint{1.389884in}{3.241897in}}{\pgfqpoint{1.382070in}{3.234083in}}%
\pgfpathcurveto{\pgfqpoint{1.374257in}{3.226270in}}{\pgfqpoint{1.369867in}{3.215671in}}{\pgfqpoint{1.369867in}{3.204621in}}%
\pgfpathcurveto{\pgfqpoint{1.369867in}{3.193571in}}{\pgfqpoint{1.374257in}{3.182971in}}{\pgfqpoint{1.382070in}{3.175158in}}%
\pgfpathcurveto{\pgfqpoint{1.389884in}{3.167344in}}{\pgfqpoint{1.400483in}{3.162954in}}{\pgfqpoint{1.411533in}{3.162954in}}%
\pgfpathclose%
\pgfusepath{stroke,fill}%
\end{pgfscope}%
\begin{pgfscope}%
\pgfpathrectangle{\pgfqpoint{0.511823in}{0.504323in}}{\pgfqpoint{3.218177in}{3.225677in}} %
\pgfusepath{clip}%
\pgfsetbuttcap%
\pgfsetroundjoin%
\definecolor{currentfill}{rgb}{0.501961,0.000000,0.000000}%
\pgfsetfillcolor{currentfill}%
\pgfsetfillopacity{0.400000}%
\pgfsetlinewidth{0.501875pt}%
\definecolor{currentstroke}{rgb}{0.501961,0.000000,0.000000}%
\pgfsetstrokecolor{currentstroke}%
\pgfsetstrokeopacity{0.400000}%
\pgfsetdash{}{0pt}%
\pgfpathmoveto{\pgfqpoint{1.369523in}{3.047107in}}%
\pgfpathcurveto{\pgfqpoint{1.380573in}{3.047107in}}{\pgfqpoint{1.391172in}{3.051497in}}{\pgfqpoint{1.398986in}{3.059311in}}%
\pgfpathcurveto{\pgfqpoint{1.406799in}{3.067125in}}{\pgfqpoint{1.411189in}{3.077724in}}{\pgfqpoint{1.411189in}{3.088774in}}%
\pgfpathcurveto{\pgfqpoint{1.411189in}{3.099824in}}{\pgfqpoint{1.406799in}{3.110423in}}{\pgfqpoint{1.398986in}{3.118237in}}%
\pgfpathcurveto{\pgfqpoint{1.391172in}{3.126050in}}{\pgfqpoint{1.380573in}{3.130441in}}{\pgfqpoint{1.369523in}{3.130441in}}%
\pgfpathcurveto{\pgfqpoint{1.358473in}{3.130441in}}{\pgfqpoint{1.347874in}{3.126050in}}{\pgfqpoint{1.340060in}{3.118237in}}%
\pgfpathcurveto{\pgfqpoint{1.332246in}{3.110423in}}{\pgfqpoint{1.327856in}{3.099824in}}{\pgfqpoint{1.327856in}{3.088774in}}%
\pgfpathcurveto{\pgfqpoint{1.327856in}{3.077724in}}{\pgfqpoint{1.332246in}{3.067125in}}{\pgfqpoint{1.340060in}{3.059311in}}%
\pgfpathcurveto{\pgfqpoint{1.347874in}{3.051497in}}{\pgfqpoint{1.358473in}{3.047107in}}{\pgfqpoint{1.369523in}{3.047107in}}%
\pgfpathclose%
\pgfusepath{stroke,fill}%
\end{pgfscope}%
\begin{pgfscope}%
\pgfpathrectangle{\pgfqpoint{0.511823in}{0.504323in}}{\pgfqpoint{3.218177in}{3.225677in}} %
\pgfusepath{clip}%
\pgfsetbuttcap%
\pgfsetroundjoin%
\definecolor{currentfill}{rgb}{0.501961,0.000000,0.000000}%
\pgfsetfillcolor{currentfill}%
\pgfsetfillopacity{0.400000}%
\pgfsetlinewidth{0.501875pt}%
\definecolor{currentstroke}{rgb}{0.501961,0.000000,0.000000}%
\pgfsetstrokecolor{currentstroke}%
\pgfsetstrokeopacity{0.400000}%
\pgfsetdash{}{0pt}%
\pgfpathmoveto{\pgfqpoint{1.351122in}{3.012496in}}%
\pgfpathcurveto{\pgfqpoint{1.362172in}{3.012496in}}{\pgfqpoint{1.372771in}{3.016887in}}{\pgfqpoint{1.380585in}{3.024700in}}%
\pgfpathcurveto{\pgfqpoint{1.388399in}{3.032514in}}{\pgfqpoint{1.392789in}{3.043113in}}{\pgfqpoint{1.392789in}{3.054163in}}%
\pgfpathcurveto{\pgfqpoint{1.392789in}{3.065213in}}{\pgfqpoint{1.388399in}{3.075812in}}{\pgfqpoint{1.380585in}{3.083626in}}%
\pgfpathcurveto{\pgfqpoint{1.372771in}{3.091439in}}{\pgfqpoint{1.362172in}{3.095830in}}{\pgfqpoint{1.351122in}{3.095830in}}%
\pgfpathcurveto{\pgfqpoint{1.340072in}{3.095830in}}{\pgfqpoint{1.329473in}{3.091439in}}{\pgfqpoint{1.321659in}{3.083626in}}%
\pgfpathcurveto{\pgfqpoint{1.313846in}{3.075812in}}{\pgfqpoint{1.309455in}{3.065213in}}{\pgfqpoint{1.309455in}{3.054163in}}%
\pgfpathcurveto{\pgfqpoint{1.309455in}{3.043113in}}{\pgfqpoint{1.313846in}{3.032514in}}{\pgfqpoint{1.321659in}{3.024700in}}%
\pgfpathcurveto{\pgfqpoint{1.329473in}{3.016887in}}{\pgfqpoint{1.340072in}{3.012496in}}{\pgfqpoint{1.351122in}{3.012496in}}%
\pgfpathclose%
\pgfusepath{stroke,fill}%
\end{pgfscope}%
\begin{pgfscope}%
\pgfpathrectangle{\pgfqpoint{0.511823in}{0.504323in}}{\pgfqpoint{3.218177in}{3.225677in}} %
\pgfusepath{clip}%
\pgfsetbuttcap%
\pgfsetroundjoin%
\definecolor{currentfill}{rgb}{0.501961,0.000000,0.000000}%
\pgfsetfillcolor{currentfill}%
\pgfsetfillopacity{0.400000}%
\pgfsetlinewidth{0.501875pt}%
\definecolor{currentstroke}{rgb}{0.501961,0.000000,0.000000}%
\pgfsetstrokecolor{currentstroke}%
\pgfsetstrokeopacity{0.400000}%
\pgfsetdash{}{0pt}%
\pgfpathmoveto{\pgfqpoint{1.372192in}{3.119668in}}%
\pgfpathcurveto{\pgfqpoint{1.383242in}{3.119668in}}{\pgfqpoint{1.393841in}{3.124059in}}{\pgfqpoint{1.401655in}{3.131872in}}%
\pgfpathcurveto{\pgfqpoint{1.409468in}{3.139686in}}{\pgfqpoint{1.413859in}{3.150285in}}{\pgfqpoint{1.413859in}{3.161335in}}%
\pgfpathcurveto{\pgfqpoint{1.413859in}{3.172385in}}{\pgfqpoint{1.409468in}{3.182984in}}{\pgfqpoint{1.401655in}{3.190798in}}%
\pgfpathcurveto{\pgfqpoint{1.393841in}{3.198611in}}{\pgfqpoint{1.383242in}{3.203002in}}{\pgfqpoint{1.372192in}{3.203002in}}%
\pgfpathcurveto{\pgfqpoint{1.361142in}{3.203002in}}{\pgfqpoint{1.350543in}{3.198611in}}{\pgfqpoint{1.342729in}{3.190798in}}%
\pgfpathcurveto{\pgfqpoint{1.334916in}{3.182984in}}{\pgfqpoint{1.330525in}{3.172385in}}{\pgfqpoint{1.330525in}{3.161335in}}%
\pgfpathcurveto{\pgfqpoint{1.330525in}{3.150285in}}{\pgfqpoint{1.334916in}{3.139686in}}{\pgfqpoint{1.342729in}{3.131872in}}%
\pgfpathcurveto{\pgfqpoint{1.350543in}{3.124059in}}{\pgfqpoint{1.361142in}{3.119668in}}{\pgfqpoint{1.372192in}{3.119668in}}%
\pgfpathclose%
\pgfusepath{stroke,fill}%
\end{pgfscope}%
\begin{pgfscope}%
\pgfpathrectangle{\pgfqpoint{0.511823in}{0.504323in}}{\pgfqpoint{3.218177in}{3.225677in}} %
\pgfusepath{clip}%
\pgfsetbuttcap%
\pgfsetroundjoin%
\definecolor{currentfill}{rgb}{0.501961,0.000000,0.000000}%
\pgfsetfillcolor{currentfill}%
\pgfsetfillopacity{0.400000}%
\pgfsetlinewidth{0.501875pt}%
\definecolor{currentstroke}{rgb}{0.501961,0.000000,0.000000}%
\pgfsetstrokecolor{currentstroke}%
\pgfsetstrokeopacity{0.400000}%
\pgfsetdash{}{0pt}%
\pgfpathmoveto{\pgfqpoint{1.337973in}{3.027036in}}%
\pgfpathcurveto{\pgfqpoint{1.349023in}{3.027036in}}{\pgfqpoint{1.359622in}{3.031426in}}{\pgfqpoint{1.367435in}{3.039240in}}%
\pgfpathcurveto{\pgfqpoint{1.375249in}{3.047053in}}{\pgfqpoint{1.379639in}{3.057652in}}{\pgfqpoint{1.379639in}{3.068703in}}%
\pgfpathcurveto{\pgfqpoint{1.379639in}{3.079753in}}{\pgfqpoint{1.375249in}{3.090352in}}{\pgfqpoint{1.367435in}{3.098165in}}%
\pgfpathcurveto{\pgfqpoint{1.359622in}{3.105979in}}{\pgfqpoint{1.349023in}{3.110369in}}{\pgfqpoint{1.337973in}{3.110369in}}%
\pgfpathcurveto{\pgfqpoint{1.326923in}{3.110369in}}{\pgfqpoint{1.316324in}{3.105979in}}{\pgfqpoint{1.308510in}{3.098165in}}%
\pgfpathcurveto{\pgfqpoint{1.300696in}{3.090352in}}{\pgfqpoint{1.296306in}{3.079753in}}{\pgfqpoint{1.296306in}{3.068703in}}%
\pgfpathcurveto{\pgfqpoint{1.296306in}{3.057652in}}{\pgfqpoint{1.300696in}{3.047053in}}{\pgfqpoint{1.308510in}{3.039240in}}%
\pgfpathcurveto{\pgfqpoint{1.316324in}{3.031426in}}{\pgfqpoint{1.326923in}{3.027036in}}{\pgfqpoint{1.337973in}{3.027036in}}%
\pgfpathclose%
\pgfusepath{stroke,fill}%
\end{pgfscope}%
\begin{pgfscope}%
\pgfpathrectangle{\pgfqpoint{0.511823in}{0.504323in}}{\pgfqpoint{3.218177in}{3.225677in}} %
\pgfusepath{clip}%
\pgfsetbuttcap%
\pgfsetroundjoin%
\definecolor{currentfill}{rgb}{0.501961,0.000000,0.000000}%
\pgfsetfillcolor{currentfill}%
\pgfsetfillopacity{0.400000}%
\pgfsetlinewidth{0.501875pt}%
\definecolor{currentstroke}{rgb}{0.501961,0.000000,0.000000}%
\pgfsetstrokecolor{currentstroke}%
\pgfsetstrokeopacity{0.400000}%
\pgfsetdash{}{0pt}%
\pgfpathmoveto{\pgfqpoint{1.299181in}{2.914896in}}%
\pgfpathcurveto{\pgfqpoint{1.310231in}{2.914896in}}{\pgfqpoint{1.320830in}{2.919286in}}{\pgfqpoint{1.328643in}{2.927100in}}%
\pgfpathcurveto{\pgfqpoint{1.336457in}{2.934914in}}{\pgfqpoint{1.340847in}{2.945513in}}{\pgfqpoint{1.340847in}{2.956563in}}%
\pgfpathcurveto{\pgfqpoint{1.340847in}{2.967613in}}{\pgfqpoint{1.336457in}{2.978212in}}{\pgfqpoint{1.328643in}{2.986026in}}%
\pgfpathcurveto{\pgfqpoint{1.320830in}{2.993839in}}{\pgfqpoint{1.310231in}{2.998229in}}{\pgfqpoint{1.299181in}{2.998229in}}%
\pgfpathcurveto{\pgfqpoint{1.288130in}{2.998229in}}{\pgfqpoint{1.277531in}{2.993839in}}{\pgfqpoint{1.269718in}{2.986026in}}%
\pgfpathcurveto{\pgfqpoint{1.261904in}{2.978212in}}{\pgfqpoint{1.257514in}{2.967613in}}{\pgfqpoint{1.257514in}{2.956563in}}%
\pgfpathcurveto{\pgfqpoint{1.257514in}{2.945513in}}{\pgfqpoint{1.261904in}{2.934914in}}{\pgfqpoint{1.269718in}{2.927100in}}%
\pgfpathcurveto{\pgfqpoint{1.277531in}{2.919286in}}{\pgfqpoint{1.288130in}{2.914896in}}{\pgfqpoint{1.299181in}{2.914896in}}%
\pgfpathclose%
\pgfusepath{stroke,fill}%
\end{pgfscope}%
\begin{pgfscope}%
\pgfpathrectangle{\pgfqpoint{0.511823in}{0.504323in}}{\pgfqpoint{3.218177in}{3.225677in}} %
\pgfusepath{clip}%
\pgfsetbuttcap%
\pgfsetroundjoin%
\definecolor{currentfill}{rgb}{0.501961,0.000000,0.000000}%
\pgfsetfillcolor{currentfill}%
\pgfsetfillopacity{0.400000}%
\pgfsetlinewidth{0.501875pt}%
\definecolor{currentstroke}{rgb}{0.501961,0.000000,0.000000}%
\pgfsetstrokecolor{currentstroke}%
\pgfsetstrokeopacity{0.400000}%
\pgfsetdash{}{0pt}%
\pgfpathmoveto{\pgfqpoint{1.347068in}{3.125870in}}%
\pgfpathcurveto{\pgfqpoint{1.358118in}{3.125870in}}{\pgfqpoint{1.368717in}{3.130261in}}{\pgfqpoint{1.376531in}{3.138074in}}%
\pgfpathcurveto{\pgfqpoint{1.384345in}{3.145888in}}{\pgfqpoint{1.388735in}{3.156487in}}{\pgfqpoint{1.388735in}{3.167537in}}%
\pgfpathcurveto{\pgfqpoint{1.388735in}{3.178587in}}{\pgfqpoint{1.384345in}{3.189186in}}{\pgfqpoint{1.376531in}{3.197000in}}%
\pgfpathcurveto{\pgfqpoint{1.368717in}{3.204813in}}{\pgfqpoint{1.358118in}{3.209204in}}{\pgfqpoint{1.347068in}{3.209204in}}%
\pgfpathcurveto{\pgfqpoint{1.336018in}{3.209204in}}{\pgfqpoint{1.325419in}{3.204813in}}{\pgfqpoint{1.317606in}{3.197000in}}%
\pgfpathcurveto{\pgfqpoint{1.309792in}{3.189186in}}{\pgfqpoint{1.305402in}{3.178587in}}{\pgfqpoint{1.305402in}{3.167537in}}%
\pgfpathcurveto{\pgfqpoint{1.305402in}{3.156487in}}{\pgfqpoint{1.309792in}{3.145888in}}{\pgfqpoint{1.317606in}{3.138074in}}%
\pgfpathcurveto{\pgfqpoint{1.325419in}{3.130261in}}{\pgfqpoint{1.336018in}{3.125870in}}{\pgfqpoint{1.347068in}{3.125870in}}%
\pgfpathclose%
\pgfusepath{stroke,fill}%
\end{pgfscope}%
\begin{pgfscope}%
\pgfpathrectangle{\pgfqpoint{0.511823in}{0.504323in}}{\pgfqpoint{3.218177in}{3.225677in}} %
\pgfusepath{clip}%
\pgfsetbuttcap%
\pgfsetroundjoin%
\definecolor{currentfill}{rgb}{0.501961,0.000000,0.000000}%
\pgfsetfillcolor{currentfill}%
\pgfsetfillopacity{0.400000}%
\pgfsetlinewidth{0.501875pt}%
\definecolor{currentstroke}{rgb}{0.501961,0.000000,0.000000}%
\pgfsetstrokecolor{currentstroke}%
\pgfsetstrokeopacity{0.400000}%
\pgfsetdash{}{0pt}%
\pgfpathmoveto{\pgfqpoint{1.302024in}{2.988058in}}%
\pgfpathcurveto{\pgfqpoint{1.313074in}{2.988058in}}{\pgfqpoint{1.323673in}{2.992448in}}{\pgfqpoint{1.331487in}{3.000262in}}%
\pgfpathcurveto{\pgfqpoint{1.339300in}{3.008076in}}{\pgfqpoint{1.343691in}{3.018675in}}{\pgfqpoint{1.343691in}{3.029725in}}%
\pgfpathcurveto{\pgfqpoint{1.343691in}{3.040775in}}{\pgfqpoint{1.339300in}{3.051374in}}{\pgfqpoint{1.331487in}{3.059188in}}%
\pgfpathcurveto{\pgfqpoint{1.323673in}{3.067001in}}{\pgfqpoint{1.313074in}{3.071391in}}{\pgfqpoint{1.302024in}{3.071391in}}%
\pgfpathcurveto{\pgfqpoint{1.290974in}{3.071391in}}{\pgfqpoint{1.280375in}{3.067001in}}{\pgfqpoint{1.272561in}{3.059188in}}%
\pgfpathcurveto{\pgfqpoint{1.264747in}{3.051374in}}{\pgfqpoint{1.260357in}{3.040775in}}{\pgfqpoint{1.260357in}{3.029725in}}%
\pgfpathcurveto{\pgfqpoint{1.260357in}{3.018675in}}{\pgfqpoint{1.264747in}{3.008076in}}{\pgfqpoint{1.272561in}{3.000262in}}%
\pgfpathcurveto{\pgfqpoint{1.280375in}{2.992448in}}{\pgfqpoint{1.290974in}{2.988058in}}{\pgfqpoint{1.302024in}{2.988058in}}%
\pgfpathclose%
\pgfusepath{stroke,fill}%
\end{pgfscope}%
\begin{pgfscope}%
\pgfpathrectangle{\pgfqpoint{0.511823in}{0.504323in}}{\pgfqpoint{3.218177in}{3.225677in}} %
\pgfusepath{clip}%
\pgfsetbuttcap%
\pgfsetroundjoin%
\definecolor{currentfill}{rgb}{0.501961,0.000000,0.000000}%
\pgfsetfillcolor{currentfill}%
\pgfsetfillopacity{0.400000}%
\pgfsetlinewidth{0.501875pt}%
\definecolor{currentstroke}{rgb}{0.501961,0.000000,0.000000}%
\pgfsetstrokecolor{currentstroke}%
\pgfsetstrokeopacity{0.400000}%
\pgfsetdash{}{0pt}%
\pgfpathmoveto{\pgfqpoint{1.261174in}{2.862597in}}%
\pgfpathcurveto{\pgfqpoint{1.272224in}{2.862597in}}{\pgfqpoint{1.282823in}{2.866988in}}{\pgfqpoint{1.290637in}{2.874801in}}%
\pgfpathcurveto{\pgfqpoint{1.298450in}{2.882615in}}{\pgfqpoint{1.302841in}{2.893214in}}{\pgfqpoint{1.302841in}{2.904264in}}%
\pgfpathcurveto{\pgfqpoint{1.302841in}{2.915314in}}{\pgfqpoint{1.298450in}{2.925913in}}{\pgfqpoint{1.290637in}{2.933727in}}%
\pgfpathcurveto{\pgfqpoint{1.282823in}{2.941541in}}{\pgfqpoint{1.272224in}{2.945931in}}{\pgfqpoint{1.261174in}{2.945931in}}%
\pgfpathcurveto{\pgfqpoint{1.250124in}{2.945931in}}{\pgfqpoint{1.239525in}{2.941541in}}{\pgfqpoint{1.231711in}{2.933727in}}%
\pgfpathcurveto{\pgfqpoint{1.223898in}{2.925913in}}{\pgfqpoint{1.219507in}{2.915314in}}{\pgfqpoint{1.219507in}{2.904264in}}%
\pgfpathcurveto{\pgfqpoint{1.219507in}{2.893214in}}{\pgfqpoint{1.223898in}{2.882615in}}{\pgfqpoint{1.231711in}{2.874801in}}%
\pgfpathcurveto{\pgfqpoint{1.239525in}{2.866988in}}{\pgfqpoint{1.250124in}{2.862597in}}{\pgfqpoint{1.261174in}{2.862597in}}%
\pgfpathclose%
\pgfusepath{stroke,fill}%
\end{pgfscope}%
\begin{pgfscope}%
\pgfpathrectangle{\pgfqpoint{0.511823in}{0.504323in}}{\pgfqpoint{3.218177in}{3.225677in}} %
\pgfusepath{clip}%
\pgfsetbuttcap%
\pgfsetroundjoin%
\definecolor{currentfill}{rgb}{0.501961,0.000000,0.000000}%
\pgfsetfillcolor{currentfill}%
\pgfsetfillopacity{0.400000}%
\pgfsetlinewidth{0.501875pt}%
\definecolor{currentstroke}{rgb}{0.501961,0.000000,0.000000}%
\pgfsetstrokecolor{currentstroke}%
\pgfsetstrokeopacity{0.400000}%
\pgfsetdash{}{0pt}%
\pgfpathmoveto{\pgfqpoint{1.293286in}{3.019710in}}%
\pgfpathcurveto{\pgfqpoint{1.304336in}{3.019710in}}{\pgfqpoint{1.314935in}{3.024101in}}{\pgfqpoint{1.322748in}{3.031914in}}%
\pgfpathcurveto{\pgfqpoint{1.330562in}{3.039728in}}{\pgfqpoint{1.334952in}{3.050327in}}{\pgfqpoint{1.334952in}{3.061377in}}%
\pgfpathcurveto{\pgfqpoint{1.334952in}{3.072427in}}{\pgfqpoint{1.330562in}{3.083026in}}{\pgfqpoint{1.322748in}{3.090840in}}%
\pgfpathcurveto{\pgfqpoint{1.314935in}{3.098654in}}{\pgfqpoint{1.304336in}{3.103044in}}{\pgfqpoint{1.293286in}{3.103044in}}%
\pgfpathcurveto{\pgfqpoint{1.282235in}{3.103044in}}{\pgfqpoint{1.271636in}{3.098654in}}{\pgfqpoint{1.263823in}{3.090840in}}%
\pgfpathcurveto{\pgfqpoint{1.256009in}{3.083026in}}{\pgfqpoint{1.251619in}{3.072427in}}{\pgfqpoint{1.251619in}{3.061377in}}%
\pgfpathcurveto{\pgfqpoint{1.251619in}{3.050327in}}{\pgfqpoint{1.256009in}{3.039728in}}{\pgfqpoint{1.263823in}{3.031914in}}%
\pgfpathcurveto{\pgfqpoint{1.271636in}{3.024101in}}{\pgfqpoint{1.282235in}{3.019710in}}{\pgfqpoint{1.293286in}{3.019710in}}%
\pgfpathclose%
\pgfusepath{stroke,fill}%
\end{pgfscope}%
\begin{pgfscope}%
\pgfpathrectangle{\pgfqpoint{0.511823in}{0.504323in}}{\pgfqpoint{3.218177in}{3.225677in}} %
\pgfusepath{clip}%
\pgfsetbuttcap%
\pgfsetroundjoin%
\definecolor{currentfill}{rgb}{0.501961,0.000000,0.000000}%
\pgfsetfillcolor{currentfill}%
\pgfsetfillopacity{0.400000}%
\pgfsetlinewidth{0.501875pt}%
\definecolor{currentstroke}{rgb}{0.501961,0.000000,0.000000}%
\pgfsetstrokecolor{currentstroke}%
\pgfsetstrokeopacity{0.400000}%
\pgfsetdash{}{0pt}%
\pgfpathmoveto{\pgfqpoint{1.265797in}{2.944178in}}%
\pgfpathcurveto{\pgfqpoint{1.276847in}{2.944178in}}{\pgfqpoint{1.287446in}{2.948568in}}{\pgfqpoint{1.295260in}{2.956382in}}%
\pgfpathcurveto{\pgfqpoint{1.303074in}{2.964195in}}{\pgfqpoint{1.307464in}{2.974794in}}{\pgfqpoint{1.307464in}{2.985844in}}%
\pgfpathcurveto{\pgfqpoint{1.307464in}{2.996895in}}{\pgfqpoint{1.303074in}{3.007494in}}{\pgfqpoint{1.295260in}{3.015307in}}%
\pgfpathcurveto{\pgfqpoint{1.287446in}{3.023121in}}{\pgfqpoint{1.276847in}{3.027511in}}{\pgfqpoint{1.265797in}{3.027511in}}%
\pgfpathcurveto{\pgfqpoint{1.254747in}{3.027511in}}{\pgfqpoint{1.244148in}{3.023121in}}{\pgfqpoint{1.236334in}{3.015307in}}%
\pgfpathcurveto{\pgfqpoint{1.228521in}{3.007494in}}{\pgfqpoint{1.224130in}{2.996895in}}{\pgfqpoint{1.224130in}{2.985844in}}%
\pgfpathcurveto{\pgfqpoint{1.224130in}{2.974794in}}{\pgfqpoint{1.228521in}{2.964195in}}{\pgfqpoint{1.236334in}{2.956382in}}%
\pgfpathcurveto{\pgfqpoint{1.244148in}{2.948568in}}{\pgfqpoint{1.254747in}{2.944178in}}{\pgfqpoint{1.265797in}{2.944178in}}%
\pgfpathclose%
\pgfusepath{stroke,fill}%
\end{pgfscope}%
\begin{pgfscope}%
\pgfpathrectangle{\pgfqpoint{0.511823in}{0.504323in}}{\pgfqpoint{3.218177in}{3.225677in}} %
\pgfusepath{clip}%
\pgfsetbuttcap%
\pgfsetroundjoin%
\definecolor{currentfill}{rgb}{0.501961,0.000000,0.000000}%
\pgfsetfillcolor{currentfill}%
\pgfsetfillopacity{0.400000}%
\pgfsetlinewidth{0.501875pt}%
\definecolor{currentstroke}{rgb}{0.501961,0.000000,0.000000}%
\pgfsetstrokecolor{currentstroke}%
\pgfsetstrokeopacity{0.400000}%
\pgfsetdash{}{0pt}%
\pgfpathmoveto{\pgfqpoint{1.281514in}{3.040745in}}%
\pgfpathcurveto{\pgfqpoint{1.292564in}{3.040745in}}{\pgfqpoint{1.303163in}{3.045135in}}{\pgfqpoint{1.310977in}{3.052948in}}%
\pgfpathcurveto{\pgfqpoint{1.318790in}{3.060762in}}{\pgfqpoint{1.323181in}{3.071361in}}{\pgfqpoint{1.323181in}{3.082411in}}%
\pgfpathcurveto{\pgfqpoint{1.323181in}{3.093461in}}{\pgfqpoint{1.318790in}{3.104060in}}{\pgfqpoint{1.310977in}{3.111874in}}%
\pgfpathcurveto{\pgfqpoint{1.303163in}{3.119688in}}{\pgfqpoint{1.292564in}{3.124078in}}{\pgfqpoint{1.281514in}{3.124078in}}%
\pgfpathcurveto{\pgfqpoint{1.270464in}{3.124078in}}{\pgfqpoint{1.259865in}{3.119688in}}{\pgfqpoint{1.252051in}{3.111874in}}%
\pgfpathcurveto{\pgfqpoint{1.244238in}{3.104060in}}{\pgfqpoint{1.239847in}{3.093461in}}{\pgfqpoint{1.239847in}{3.082411in}}%
\pgfpathcurveto{\pgfqpoint{1.239847in}{3.071361in}}{\pgfqpoint{1.244238in}{3.060762in}}{\pgfqpoint{1.252051in}{3.052948in}}%
\pgfpathcurveto{\pgfqpoint{1.259865in}{3.045135in}}{\pgfqpoint{1.270464in}{3.040745in}}{\pgfqpoint{1.281514in}{3.040745in}}%
\pgfpathclose%
\pgfusepath{stroke,fill}%
\end{pgfscope}%
\begin{pgfscope}%
\pgfpathrectangle{\pgfqpoint{0.511823in}{0.504323in}}{\pgfqpoint{3.218177in}{3.225677in}} %
\pgfusepath{clip}%
\pgfsetbuttcap%
\pgfsetroundjoin%
\definecolor{currentfill}{rgb}{0.501961,0.000000,0.000000}%
\pgfsetfillcolor{currentfill}%
\pgfsetfillopacity{0.400000}%
\pgfsetlinewidth{0.501875pt}%
\definecolor{currentstroke}{rgb}{0.501961,0.000000,0.000000}%
\pgfsetstrokecolor{currentstroke}%
\pgfsetstrokeopacity{0.400000}%
\pgfsetdash{}{0pt}%
\pgfpathmoveto{\pgfqpoint{1.253361in}{2.960493in}}%
\pgfpathcurveto{\pgfqpoint{1.264411in}{2.960493in}}{\pgfqpoint{1.275010in}{2.964883in}}{\pgfqpoint{1.282824in}{2.972697in}}%
\pgfpathcurveto{\pgfqpoint{1.290637in}{2.980511in}}{\pgfqpoint{1.295028in}{2.991110in}}{\pgfqpoint{1.295028in}{3.002160in}}%
\pgfpathcurveto{\pgfqpoint{1.295028in}{3.013210in}}{\pgfqpoint{1.290637in}{3.023809in}}{\pgfqpoint{1.282824in}{3.031623in}}%
\pgfpathcurveto{\pgfqpoint{1.275010in}{3.039436in}}{\pgfqpoint{1.264411in}{3.043826in}}{\pgfqpoint{1.253361in}{3.043826in}}%
\pgfpathcurveto{\pgfqpoint{1.242311in}{3.043826in}}{\pgfqpoint{1.231712in}{3.039436in}}{\pgfqpoint{1.223898in}{3.031623in}}%
\pgfpathcurveto{\pgfqpoint{1.216085in}{3.023809in}}{\pgfqpoint{1.211694in}{3.013210in}}{\pgfqpoint{1.211694in}{3.002160in}}%
\pgfpathcurveto{\pgfqpoint{1.211694in}{2.991110in}}{\pgfqpoint{1.216085in}{2.980511in}}{\pgfqpoint{1.223898in}{2.972697in}}%
\pgfpathcurveto{\pgfqpoint{1.231712in}{2.964883in}}{\pgfqpoint{1.242311in}{2.960493in}}{\pgfqpoint{1.253361in}{2.960493in}}%
\pgfpathclose%
\pgfusepath{stroke,fill}%
\end{pgfscope}%
\begin{pgfscope}%
\pgfpathrectangle{\pgfqpoint{0.511823in}{0.504323in}}{\pgfqpoint{3.218177in}{3.225677in}} %
\pgfusepath{clip}%
\pgfsetbuttcap%
\pgfsetroundjoin%
\definecolor{currentfill}{rgb}{0.501961,0.000000,0.000000}%
\pgfsetfillcolor{currentfill}%
\pgfsetfillopacity{0.400000}%
\pgfsetlinewidth{0.501875pt}%
\definecolor{currentstroke}{rgb}{0.501961,0.000000,0.000000}%
\pgfsetstrokecolor{currentstroke}%
\pgfsetstrokeopacity{0.400000}%
\pgfsetdash{}{0pt}%
\pgfpathmoveto{\pgfqpoint{1.256958in}{3.009711in}}%
\pgfpathcurveto{\pgfqpoint{1.268008in}{3.009711in}}{\pgfqpoint{1.278607in}{3.014102in}}{\pgfqpoint{1.286421in}{3.021915in}}%
\pgfpathcurveto{\pgfqpoint{1.294235in}{3.029729in}}{\pgfqpoint{1.298625in}{3.040328in}}{\pgfqpoint{1.298625in}{3.051378in}}%
\pgfpathcurveto{\pgfqpoint{1.298625in}{3.062428in}}{\pgfqpoint{1.294235in}{3.073027in}}{\pgfqpoint{1.286421in}{3.080841in}}%
\pgfpathcurveto{\pgfqpoint{1.278607in}{3.088654in}}{\pgfqpoint{1.268008in}{3.093045in}}{\pgfqpoint{1.256958in}{3.093045in}}%
\pgfpathcurveto{\pgfqpoint{1.245908in}{3.093045in}}{\pgfqpoint{1.235309in}{3.088654in}}{\pgfqpoint{1.227495in}{3.080841in}}%
\pgfpathcurveto{\pgfqpoint{1.219682in}{3.073027in}}{\pgfqpoint{1.215291in}{3.062428in}}{\pgfqpoint{1.215291in}{3.051378in}}%
\pgfpathcurveto{\pgfqpoint{1.215291in}{3.040328in}}{\pgfqpoint{1.219682in}{3.029729in}}{\pgfqpoint{1.227495in}{3.021915in}}%
\pgfpathcurveto{\pgfqpoint{1.235309in}{3.014102in}}{\pgfqpoint{1.245908in}{3.009711in}}{\pgfqpoint{1.256958in}{3.009711in}}%
\pgfpathclose%
\pgfusepath{stroke,fill}%
\end{pgfscope}%
\begin{pgfscope}%
\pgfpathrectangle{\pgfqpoint{0.511823in}{0.504323in}}{\pgfqpoint{3.218177in}{3.225677in}} %
\pgfusepath{clip}%
\pgfsetbuttcap%
\pgfsetroundjoin%
\definecolor{currentfill}{rgb}{0.501961,0.000000,0.000000}%
\pgfsetfillcolor{currentfill}%
\pgfsetfillopacity{0.400000}%
\pgfsetlinewidth{0.501875pt}%
\definecolor{currentstroke}{rgb}{0.501961,0.000000,0.000000}%
\pgfsetstrokecolor{currentstroke}%
\pgfsetstrokeopacity{0.400000}%
\pgfsetdash{}{0pt}%
\pgfpathmoveto{\pgfqpoint{1.240844in}{2.977245in}}%
\pgfpathcurveto{\pgfqpoint{1.251894in}{2.977245in}}{\pgfqpoint{1.262493in}{2.981635in}}{\pgfqpoint{1.270306in}{2.989449in}}%
\pgfpathcurveto{\pgfqpoint{1.278120in}{2.997263in}}{\pgfqpoint{1.282510in}{3.007862in}}{\pgfqpoint{1.282510in}{3.018912in}}%
\pgfpathcurveto{\pgfqpoint{1.282510in}{3.029962in}}{\pgfqpoint{1.278120in}{3.040561in}}{\pgfqpoint{1.270306in}{3.048375in}}%
\pgfpathcurveto{\pgfqpoint{1.262493in}{3.056188in}}{\pgfqpoint{1.251894in}{3.060578in}}{\pgfqpoint{1.240844in}{3.060578in}}%
\pgfpathcurveto{\pgfqpoint{1.229793in}{3.060578in}}{\pgfqpoint{1.219194in}{3.056188in}}{\pgfqpoint{1.211381in}{3.048375in}}%
\pgfpathcurveto{\pgfqpoint{1.203567in}{3.040561in}}{\pgfqpoint{1.199177in}{3.029962in}}{\pgfqpoint{1.199177in}{3.018912in}}%
\pgfpathcurveto{\pgfqpoint{1.199177in}{3.007862in}}{\pgfqpoint{1.203567in}{2.997263in}}{\pgfqpoint{1.211381in}{2.989449in}}%
\pgfpathcurveto{\pgfqpoint{1.219194in}{2.981635in}}{\pgfqpoint{1.229793in}{2.977245in}}{\pgfqpoint{1.240844in}{2.977245in}}%
\pgfpathclose%
\pgfusepath{stroke,fill}%
\end{pgfscope}%
\begin{pgfscope}%
\pgfpathrectangle{\pgfqpoint{0.511823in}{0.504323in}}{\pgfqpoint{3.218177in}{3.225677in}} %
\pgfusepath{clip}%
\pgfsetbuttcap%
\pgfsetroundjoin%
\definecolor{currentfill}{rgb}{0.501961,0.000000,0.000000}%
\pgfsetfillcolor{currentfill}%
\pgfsetfillopacity{0.400000}%
\pgfsetlinewidth{0.501875pt}%
\definecolor{currentstroke}{rgb}{0.501961,0.000000,0.000000}%
\pgfsetstrokecolor{currentstroke}%
\pgfsetstrokeopacity{0.400000}%
\pgfsetdash{}{0pt}%
\pgfpathmoveto{\pgfqpoint{1.246611in}{3.037404in}}%
\pgfpathcurveto{\pgfqpoint{1.257661in}{3.037404in}}{\pgfqpoint{1.268260in}{3.041794in}}{\pgfqpoint{1.276074in}{3.049608in}}%
\pgfpathcurveto{\pgfqpoint{1.283888in}{3.057421in}}{\pgfqpoint{1.288278in}{3.068020in}}{\pgfqpoint{1.288278in}{3.079070in}}%
\pgfpathcurveto{\pgfqpoint{1.288278in}{3.090120in}}{\pgfqpoint{1.283888in}{3.100719in}}{\pgfqpoint{1.276074in}{3.108533in}}%
\pgfpathcurveto{\pgfqpoint{1.268260in}{3.116347in}}{\pgfqpoint{1.257661in}{3.120737in}}{\pgfqpoint{1.246611in}{3.120737in}}%
\pgfpathcurveto{\pgfqpoint{1.235561in}{3.120737in}}{\pgfqpoint{1.224962in}{3.116347in}}{\pgfqpoint{1.217148in}{3.108533in}}%
\pgfpathcurveto{\pgfqpoint{1.209335in}{3.100719in}}{\pgfqpoint{1.204944in}{3.090120in}}{\pgfqpoint{1.204944in}{3.079070in}}%
\pgfpathcurveto{\pgfqpoint{1.204944in}{3.068020in}}{\pgfqpoint{1.209335in}{3.057421in}}{\pgfqpoint{1.217148in}{3.049608in}}%
\pgfpathcurveto{\pgfqpoint{1.224962in}{3.041794in}}{\pgfqpoint{1.235561in}{3.037404in}}{\pgfqpoint{1.246611in}{3.037404in}}%
\pgfpathclose%
\pgfusepath{stroke,fill}%
\end{pgfscope}%
\begin{pgfscope}%
\pgfpathrectangle{\pgfqpoint{0.511823in}{0.504323in}}{\pgfqpoint{3.218177in}{3.225677in}} %
\pgfusepath{clip}%
\pgfsetbuttcap%
\pgfsetroundjoin%
\definecolor{currentfill}{rgb}{0.501961,0.000000,0.000000}%
\pgfsetfillcolor{currentfill}%
\pgfsetfillopacity{0.400000}%
\pgfsetlinewidth{0.501875pt}%
\definecolor{currentstroke}{rgb}{0.501961,0.000000,0.000000}%
\pgfsetstrokecolor{currentstroke}%
\pgfsetstrokeopacity{0.400000}%
\pgfsetdash{}{0pt}%
\pgfpathmoveto{\pgfqpoint{1.257864in}{3.123205in}}%
\pgfpathcurveto{\pgfqpoint{1.268914in}{3.123205in}}{\pgfqpoint{1.279513in}{3.127595in}}{\pgfqpoint{1.287327in}{3.135409in}}%
\pgfpathcurveto{\pgfqpoint{1.295141in}{3.143222in}}{\pgfqpoint{1.299531in}{3.153821in}}{\pgfqpoint{1.299531in}{3.164872in}}%
\pgfpathcurveto{\pgfqpoint{1.299531in}{3.175922in}}{\pgfqpoint{1.295141in}{3.186521in}}{\pgfqpoint{1.287327in}{3.194334in}}%
\pgfpathcurveto{\pgfqpoint{1.279513in}{3.202148in}}{\pgfqpoint{1.268914in}{3.206538in}}{\pgfqpoint{1.257864in}{3.206538in}}%
\pgfpathcurveto{\pgfqpoint{1.246814in}{3.206538in}}{\pgfqpoint{1.236215in}{3.202148in}}{\pgfqpoint{1.228402in}{3.194334in}}%
\pgfpathcurveto{\pgfqpoint{1.220588in}{3.186521in}}{\pgfqpoint{1.216198in}{3.175922in}}{\pgfqpoint{1.216198in}{3.164872in}}%
\pgfpathcurveto{\pgfqpoint{1.216198in}{3.153821in}}{\pgfqpoint{1.220588in}{3.143222in}}{\pgfqpoint{1.228402in}{3.135409in}}%
\pgfpathcurveto{\pgfqpoint{1.236215in}{3.127595in}}{\pgfqpoint{1.246814in}{3.123205in}}{\pgfqpoint{1.257864in}{3.123205in}}%
\pgfpathclose%
\pgfusepath{stroke,fill}%
\end{pgfscope}%
\begin{pgfscope}%
\pgfpathrectangle{\pgfqpoint{0.511823in}{0.504323in}}{\pgfqpoint{3.218177in}{3.225677in}} %
\pgfusepath{clip}%
\pgfsetbuttcap%
\pgfsetroundjoin%
\definecolor{currentfill}{rgb}{0.501961,0.000000,0.000000}%
\pgfsetfillcolor{currentfill}%
\pgfsetfillopacity{0.400000}%
\pgfsetlinewidth{0.501875pt}%
\definecolor{currentstroke}{rgb}{0.501961,0.000000,0.000000}%
\pgfsetstrokecolor{currentstroke}%
\pgfsetstrokeopacity{0.400000}%
\pgfsetdash{}{0pt}%
\pgfpathmoveto{\pgfqpoint{1.240387in}{3.084822in}}%
\pgfpathcurveto{\pgfqpoint{1.251437in}{3.084822in}}{\pgfqpoint{1.262036in}{3.089212in}}{\pgfqpoint{1.269850in}{3.097025in}}%
\pgfpathcurveto{\pgfqpoint{1.277664in}{3.104839in}}{\pgfqpoint{1.282054in}{3.115438in}}{\pgfqpoint{1.282054in}{3.126488in}}%
\pgfpathcurveto{\pgfqpoint{1.282054in}{3.137538in}}{\pgfqpoint{1.277664in}{3.148137in}}{\pgfqpoint{1.269850in}{3.155951in}}%
\pgfpathcurveto{\pgfqpoint{1.262036in}{3.163765in}}{\pgfqpoint{1.251437in}{3.168155in}}{\pgfqpoint{1.240387in}{3.168155in}}%
\pgfpathcurveto{\pgfqpoint{1.229337in}{3.168155in}}{\pgfqpoint{1.218738in}{3.163765in}}{\pgfqpoint{1.210924in}{3.155951in}}%
\pgfpathcurveto{\pgfqpoint{1.203111in}{3.148137in}}{\pgfqpoint{1.198720in}{3.137538in}}{\pgfqpoint{1.198720in}{3.126488in}}%
\pgfpathcurveto{\pgfqpoint{1.198720in}{3.115438in}}{\pgfqpoint{1.203111in}{3.104839in}}{\pgfqpoint{1.210924in}{3.097025in}}%
\pgfpathcurveto{\pgfqpoint{1.218738in}{3.089212in}}{\pgfqpoint{1.229337in}{3.084822in}}{\pgfqpoint{1.240387in}{3.084822in}}%
\pgfpathclose%
\pgfusepath{stroke,fill}%
\end{pgfscope}%
\begin{pgfscope}%
\pgfpathrectangle{\pgfqpoint{0.511823in}{0.504323in}}{\pgfqpoint{3.218177in}{3.225677in}} %
\pgfusepath{clip}%
\pgfsetbuttcap%
\pgfsetroundjoin%
\definecolor{currentfill}{rgb}{0.501961,0.000000,0.000000}%
\pgfsetfillcolor{currentfill}%
\pgfsetfillopacity{0.400000}%
\pgfsetlinewidth{0.501875pt}%
\definecolor{currentstroke}{rgb}{0.501961,0.000000,0.000000}%
\pgfsetstrokecolor{currentstroke}%
\pgfsetstrokeopacity{0.400000}%
\pgfsetdash{}{0pt}%
\pgfpathmoveto{\pgfqpoint{1.225525in}{3.056896in}}%
\pgfpathcurveto{\pgfqpoint{1.236575in}{3.056896in}}{\pgfqpoint{1.247174in}{3.061286in}}{\pgfqpoint{1.254988in}{3.069100in}}%
\pgfpathcurveto{\pgfqpoint{1.262801in}{3.076914in}}{\pgfqpoint{1.267192in}{3.087513in}}{\pgfqpoint{1.267192in}{3.098563in}}%
\pgfpathcurveto{\pgfqpoint{1.267192in}{3.109613in}}{\pgfqpoint{1.262801in}{3.120212in}}{\pgfqpoint{1.254988in}{3.128025in}}%
\pgfpathcurveto{\pgfqpoint{1.247174in}{3.135839in}}{\pgfqpoint{1.236575in}{3.140229in}}{\pgfqpoint{1.225525in}{3.140229in}}%
\pgfpathcurveto{\pgfqpoint{1.214475in}{3.140229in}}{\pgfqpoint{1.203876in}{3.135839in}}{\pgfqpoint{1.196062in}{3.128025in}}%
\pgfpathcurveto{\pgfqpoint{1.188249in}{3.120212in}}{\pgfqpoint{1.183858in}{3.109613in}}{\pgfqpoint{1.183858in}{3.098563in}}%
\pgfpathcurveto{\pgfqpoint{1.183858in}{3.087513in}}{\pgfqpoint{1.188249in}{3.076914in}}{\pgfqpoint{1.196062in}{3.069100in}}%
\pgfpathcurveto{\pgfqpoint{1.203876in}{3.061286in}}{\pgfqpoint{1.214475in}{3.056896in}}{\pgfqpoint{1.225525in}{3.056896in}}%
\pgfpathclose%
\pgfusepath{stroke,fill}%
\end{pgfscope}%
\begin{pgfscope}%
\pgfpathrectangle{\pgfqpoint{0.511823in}{0.504323in}}{\pgfqpoint{3.218177in}{3.225677in}} %
\pgfusepath{clip}%
\pgfsetbuttcap%
\pgfsetroundjoin%
\definecolor{currentfill}{rgb}{0.501961,0.000000,0.000000}%
\pgfsetfillcolor{currentfill}%
\pgfsetfillopacity{0.400000}%
\pgfsetlinewidth{0.501875pt}%
\definecolor{currentstroke}{rgb}{0.501961,0.000000,0.000000}%
\pgfsetstrokecolor{currentstroke}%
\pgfsetstrokeopacity{0.400000}%
\pgfsetdash{}{0pt}%
\pgfpathmoveto{\pgfqpoint{1.244511in}{3.182228in}}%
\pgfpathcurveto{\pgfqpoint{1.255561in}{3.182228in}}{\pgfqpoint{1.266160in}{3.186618in}}{\pgfqpoint{1.273974in}{3.194432in}}%
\pgfpathcurveto{\pgfqpoint{1.281788in}{3.202246in}}{\pgfqpoint{1.286178in}{3.212845in}}{\pgfqpoint{1.286178in}{3.223895in}}%
\pgfpathcurveto{\pgfqpoint{1.286178in}{3.234945in}}{\pgfqpoint{1.281788in}{3.245544in}}{\pgfqpoint{1.273974in}{3.253358in}}%
\pgfpathcurveto{\pgfqpoint{1.266160in}{3.261171in}}{\pgfqpoint{1.255561in}{3.265561in}}{\pgfqpoint{1.244511in}{3.265561in}}%
\pgfpathcurveto{\pgfqpoint{1.233461in}{3.265561in}}{\pgfqpoint{1.222862in}{3.261171in}}{\pgfqpoint{1.215049in}{3.253358in}}%
\pgfpathcurveto{\pgfqpoint{1.207235in}{3.245544in}}{\pgfqpoint{1.202845in}{3.234945in}}{\pgfqpoint{1.202845in}{3.223895in}}%
\pgfpathcurveto{\pgfqpoint{1.202845in}{3.212845in}}{\pgfqpoint{1.207235in}{3.202246in}}{\pgfqpoint{1.215049in}{3.194432in}}%
\pgfpathcurveto{\pgfqpoint{1.222862in}{3.186618in}}{\pgfqpoint{1.233461in}{3.182228in}}{\pgfqpoint{1.244511in}{3.182228in}}%
\pgfpathclose%
\pgfusepath{stroke,fill}%
\end{pgfscope}%
\begin{pgfscope}%
\pgfpathrectangle{\pgfqpoint{0.511823in}{0.504323in}}{\pgfqpoint{3.218177in}{3.225677in}} %
\pgfusepath{clip}%
\pgfsetbuttcap%
\pgfsetroundjoin%
\definecolor{currentfill}{rgb}{0.501961,0.000000,0.000000}%
\pgfsetfillcolor{currentfill}%
\pgfsetfillopacity{0.400000}%
\pgfsetlinewidth{0.501875pt}%
\definecolor{currentstroke}{rgb}{0.501961,0.000000,0.000000}%
\pgfsetstrokecolor{currentstroke}%
\pgfsetstrokeopacity{0.400000}%
\pgfsetdash{}{0pt}%
\pgfpathmoveto{\pgfqpoint{1.173589in}{2.895313in}}%
\pgfpathcurveto{\pgfqpoint{1.184639in}{2.895313in}}{\pgfqpoint{1.195238in}{2.899703in}}{\pgfqpoint{1.203052in}{2.907517in}}%
\pgfpathcurveto{\pgfqpoint{1.210866in}{2.915331in}}{\pgfqpoint{1.215256in}{2.925930in}}{\pgfqpoint{1.215256in}{2.936980in}}%
\pgfpathcurveto{\pgfqpoint{1.215256in}{2.948030in}}{\pgfqpoint{1.210866in}{2.958629in}}{\pgfqpoint{1.203052in}{2.966443in}}%
\pgfpathcurveto{\pgfqpoint{1.195238in}{2.974256in}}{\pgfqpoint{1.184639in}{2.978646in}}{\pgfqpoint{1.173589in}{2.978646in}}%
\pgfpathcurveto{\pgfqpoint{1.162539in}{2.978646in}}{\pgfqpoint{1.151940in}{2.974256in}}{\pgfqpoint{1.144126in}{2.966443in}}%
\pgfpathcurveto{\pgfqpoint{1.136313in}{2.958629in}}{\pgfqpoint{1.131923in}{2.948030in}}{\pgfqpoint{1.131923in}{2.936980in}}%
\pgfpathcurveto{\pgfqpoint{1.131923in}{2.925930in}}{\pgfqpoint{1.136313in}{2.915331in}}{\pgfqpoint{1.144126in}{2.907517in}}%
\pgfpathcurveto{\pgfqpoint{1.151940in}{2.899703in}}{\pgfqpoint{1.162539in}{2.895313in}}{\pgfqpoint{1.173589in}{2.895313in}}%
\pgfpathclose%
\pgfusepath{stroke,fill}%
\end{pgfscope}%
\begin{pgfscope}%
\pgfpathrectangle{\pgfqpoint{0.511823in}{0.504323in}}{\pgfqpoint{3.218177in}{3.225677in}} %
\pgfusepath{clip}%
\pgfsetbuttcap%
\pgfsetroundjoin%
\definecolor{currentfill}{rgb}{0.501961,0.000000,0.000000}%
\pgfsetfillcolor{currentfill}%
\pgfsetfillopacity{0.400000}%
\pgfsetlinewidth{0.501875pt}%
\definecolor{currentstroke}{rgb}{0.501961,0.000000,0.000000}%
\pgfsetstrokecolor{currentstroke}%
\pgfsetstrokeopacity{0.400000}%
\pgfsetdash{}{0pt}%
\pgfpathmoveto{\pgfqpoint{1.205277in}{3.081802in}}%
\pgfpathcurveto{\pgfqpoint{1.216327in}{3.081802in}}{\pgfqpoint{1.226926in}{3.086192in}}{\pgfqpoint{1.234739in}{3.094005in}}%
\pgfpathcurveto{\pgfqpoint{1.242553in}{3.101819in}}{\pgfqpoint{1.246943in}{3.112418in}}{\pgfqpoint{1.246943in}{3.123468in}}%
\pgfpathcurveto{\pgfqpoint{1.246943in}{3.134518in}}{\pgfqpoint{1.242553in}{3.145117in}}{\pgfqpoint{1.234739in}{3.152931in}}%
\pgfpathcurveto{\pgfqpoint{1.226926in}{3.160745in}}{\pgfqpoint{1.216327in}{3.165135in}}{\pgfqpoint{1.205277in}{3.165135in}}%
\pgfpathcurveto{\pgfqpoint{1.194227in}{3.165135in}}{\pgfqpoint{1.183627in}{3.160745in}}{\pgfqpoint{1.175814in}{3.152931in}}%
\pgfpathcurveto{\pgfqpoint{1.168000in}{3.145117in}}{\pgfqpoint{1.163610in}{3.134518in}}{\pgfqpoint{1.163610in}{3.123468in}}%
\pgfpathcurveto{\pgfqpoint{1.163610in}{3.112418in}}{\pgfqpoint{1.168000in}{3.101819in}}{\pgfqpoint{1.175814in}{3.094005in}}%
\pgfpathcurveto{\pgfqpoint{1.183627in}{3.086192in}}{\pgfqpoint{1.194227in}{3.081802in}}{\pgfqpoint{1.205277in}{3.081802in}}%
\pgfpathclose%
\pgfusepath{stroke,fill}%
\end{pgfscope}%
\begin{pgfscope}%
\pgfpathrectangle{\pgfqpoint{0.511823in}{0.504323in}}{\pgfqpoint{3.218177in}{3.225677in}} %
\pgfusepath{clip}%
\pgfsetbuttcap%
\pgfsetroundjoin%
\definecolor{currentfill}{rgb}{0.501961,0.000000,0.000000}%
\pgfsetfillcolor{currentfill}%
\pgfsetfillopacity{0.400000}%
\pgfsetlinewidth{0.501875pt}%
\definecolor{currentstroke}{rgb}{0.501961,0.000000,0.000000}%
\pgfsetstrokecolor{currentstroke}%
\pgfsetstrokeopacity{0.400000}%
\pgfsetdash{}{0pt}%
\pgfpathmoveto{\pgfqpoint{1.222412in}{3.204868in}}%
\pgfpathcurveto{\pgfqpoint{1.233462in}{3.204868in}}{\pgfqpoint{1.244061in}{3.209259in}}{\pgfqpoint{1.251875in}{3.217072in}}%
\pgfpathcurveto{\pgfqpoint{1.259688in}{3.224886in}}{\pgfqpoint{1.264078in}{3.235485in}}{\pgfqpoint{1.264078in}{3.246535in}}%
\pgfpathcurveto{\pgfqpoint{1.264078in}{3.257585in}}{\pgfqpoint{1.259688in}{3.268184in}}{\pgfqpoint{1.251875in}{3.275998in}}%
\pgfpathcurveto{\pgfqpoint{1.244061in}{3.283811in}}{\pgfqpoint{1.233462in}{3.288202in}}{\pgfqpoint{1.222412in}{3.288202in}}%
\pgfpathcurveto{\pgfqpoint{1.211362in}{3.288202in}}{\pgfqpoint{1.200763in}{3.283811in}}{\pgfqpoint{1.192949in}{3.275998in}}%
\pgfpathcurveto{\pgfqpoint{1.185135in}{3.268184in}}{\pgfqpoint{1.180745in}{3.257585in}}{\pgfqpoint{1.180745in}{3.246535in}}%
\pgfpathcurveto{\pgfqpoint{1.180745in}{3.235485in}}{\pgfqpoint{1.185135in}{3.224886in}}{\pgfqpoint{1.192949in}{3.217072in}}%
\pgfpathcurveto{\pgfqpoint{1.200763in}{3.209259in}}{\pgfqpoint{1.211362in}{3.204868in}}{\pgfqpoint{1.222412in}{3.204868in}}%
\pgfpathclose%
\pgfusepath{stroke,fill}%
\end{pgfscope}%
\begin{pgfscope}%
\pgfpathrectangle{\pgfqpoint{0.511823in}{0.504323in}}{\pgfqpoint{3.218177in}{3.225677in}} %
\pgfusepath{clip}%
\pgfsetbuttcap%
\pgfsetroundjoin%
\definecolor{currentfill}{rgb}{0.501961,0.000000,0.000000}%
\pgfsetfillcolor{currentfill}%
\pgfsetfillopacity{0.400000}%
\pgfsetlinewidth{0.501875pt}%
\definecolor{currentstroke}{rgb}{0.501961,0.000000,0.000000}%
\pgfsetstrokecolor{currentstroke}%
\pgfsetstrokeopacity{0.400000}%
\pgfsetdash{}{0pt}%
\pgfpathmoveto{\pgfqpoint{1.198530in}{3.132435in}}%
\pgfpathcurveto{\pgfqpoint{1.209581in}{3.132435in}}{\pgfqpoint{1.220180in}{3.136825in}}{\pgfqpoint{1.227993in}{3.144639in}}%
\pgfpathcurveto{\pgfqpoint{1.235807in}{3.152452in}}{\pgfqpoint{1.240197in}{3.163051in}}{\pgfqpoint{1.240197in}{3.174102in}}%
\pgfpathcurveto{\pgfqpoint{1.240197in}{3.185152in}}{\pgfqpoint{1.235807in}{3.195751in}}{\pgfqpoint{1.227993in}{3.203564in}}%
\pgfpathcurveto{\pgfqpoint{1.220180in}{3.211378in}}{\pgfqpoint{1.209581in}{3.215768in}}{\pgfqpoint{1.198530in}{3.215768in}}%
\pgfpathcurveto{\pgfqpoint{1.187480in}{3.215768in}}{\pgfqpoint{1.176881in}{3.211378in}}{\pgfqpoint{1.169068in}{3.203564in}}%
\pgfpathcurveto{\pgfqpoint{1.161254in}{3.195751in}}{\pgfqpoint{1.156864in}{3.185152in}}{\pgfqpoint{1.156864in}{3.174102in}}%
\pgfpathcurveto{\pgfqpoint{1.156864in}{3.163051in}}{\pgfqpoint{1.161254in}{3.152452in}}{\pgfqpoint{1.169068in}{3.144639in}}%
\pgfpathcurveto{\pgfqpoint{1.176881in}{3.136825in}}{\pgfqpoint{1.187480in}{3.132435in}}{\pgfqpoint{1.198530in}{3.132435in}}%
\pgfpathclose%
\pgfusepath{stroke,fill}%
\end{pgfscope}%
\begin{pgfscope}%
\pgfpathrectangle{\pgfqpoint{0.511823in}{0.504323in}}{\pgfqpoint{3.218177in}{3.225677in}} %
\pgfusepath{clip}%
\pgfsetbuttcap%
\pgfsetroundjoin%
\definecolor{currentfill}{rgb}{0.501961,0.000000,0.000000}%
\pgfsetfillcolor{currentfill}%
\pgfsetfillopacity{0.400000}%
\pgfsetlinewidth{0.501875pt}%
\definecolor{currentstroke}{rgb}{0.501961,0.000000,0.000000}%
\pgfsetstrokecolor{currentstroke}%
\pgfsetstrokeopacity{0.400000}%
\pgfsetdash{}{0pt}%
\pgfpathmoveto{\pgfqpoint{1.203941in}{3.202491in}}%
\pgfpathcurveto{\pgfqpoint{1.214991in}{3.202491in}}{\pgfqpoint{1.225590in}{3.206882in}}{\pgfqpoint{1.233404in}{3.214695in}}%
\pgfpathcurveto{\pgfqpoint{1.241218in}{3.222509in}}{\pgfqpoint{1.245608in}{3.233108in}}{\pgfqpoint{1.245608in}{3.244158in}}%
\pgfpathcurveto{\pgfqpoint{1.245608in}{3.255208in}}{\pgfqpoint{1.241218in}{3.265807in}}{\pgfqpoint{1.233404in}{3.273621in}}%
\pgfpathcurveto{\pgfqpoint{1.225590in}{3.281434in}}{\pgfqpoint{1.214991in}{3.285825in}}{\pgfqpoint{1.203941in}{3.285825in}}%
\pgfpathcurveto{\pgfqpoint{1.192891in}{3.285825in}}{\pgfqpoint{1.182292in}{3.281434in}}{\pgfqpoint{1.174478in}{3.273621in}}%
\pgfpathcurveto{\pgfqpoint{1.166665in}{3.265807in}}{\pgfqpoint{1.162275in}{3.255208in}}{\pgfqpoint{1.162275in}{3.244158in}}%
\pgfpathcurveto{\pgfqpoint{1.162275in}{3.233108in}}{\pgfqpoint{1.166665in}{3.222509in}}{\pgfqpoint{1.174478in}{3.214695in}}%
\pgfpathcurveto{\pgfqpoint{1.182292in}{3.206882in}}{\pgfqpoint{1.192891in}{3.202491in}}{\pgfqpoint{1.203941in}{3.202491in}}%
\pgfpathclose%
\pgfusepath{stroke,fill}%
\end{pgfscope}%
\begin{pgfscope}%
\pgfpathrectangle{\pgfqpoint{0.511823in}{0.504323in}}{\pgfqpoint{3.218177in}{3.225677in}} %
\pgfusepath{clip}%
\pgfsetbuttcap%
\pgfsetroundjoin%
\definecolor{currentfill}{rgb}{0.501961,0.000000,0.000000}%
\pgfsetfillcolor{currentfill}%
\pgfsetfillopacity{0.400000}%
\pgfsetlinewidth{0.501875pt}%
\definecolor{currentstroke}{rgb}{0.501961,0.000000,0.000000}%
\pgfsetstrokecolor{currentstroke}%
\pgfsetstrokeopacity{0.400000}%
\pgfsetdash{}{0pt}%
\pgfpathmoveto{\pgfqpoint{1.145867in}{2.955037in}}%
\pgfpathcurveto{\pgfqpoint{1.156917in}{2.955037in}}{\pgfqpoint{1.167516in}{2.959428in}}{\pgfqpoint{1.175330in}{2.967241in}}%
\pgfpathcurveto{\pgfqpoint{1.183144in}{2.975055in}}{\pgfqpoint{1.187534in}{2.985654in}}{\pgfqpoint{1.187534in}{2.996704in}}%
\pgfpathcurveto{\pgfqpoint{1.187534in}{3.007754in}}{\pgfqpoint{1.183144in}{3.018353in}}{\pgfqpoint{1.175330in}{3.026167in}}%
\pgfpathcurveto{\pgfqpoint{1.167516in}{3.033980in}}{\pgfqpoint{1.156917in}{3.038371in}}{\pgfqpoint{1.145867in}{3.038371in}}%
\pgfpathcurveto{\pgfqpoint{1.134817in}{3.038371in}}{\pgfqpoint{1.124218in}{3.033980in}}{\pgfqpoint{1.116404in}{3.026167in}}%
\pgfpathcurveto{\pgfqpoint{1.108591in}{3.018353in}}{\pgfqpoint{1.104200in}{3.007754in}}{\pgfqpoint{1.104200in}{2.996704in}}%
\pgfpathcurveto{\pgfqpoint{1.104200in}{2.985654in}}{\pgfqpoint{1.108591in}{2.975055in}}{\pgfqpoint{1.116404in}{2.967241in}}%
\pgfpathcurveto{\pgfqpoint{1.124218in}{2.959428in}}{\pgfqpoint{1.134817in}{2.955037in}}{\pgfqpoint{1.145867in}{2.955037in}}%
\pgfpathclose%
\pgfusepath{stroke,fill}%
\end{pgfscope}%
\begin{pgfscope}%
\pgfpathrectangle{\pgfqpoint{0.511823in}{0.504323in}}{\pgfqpoint{3.218177in}{3.225677in}} %
\pgfusepath{clip}%
\pgfsetbuttcap%
\pgfsetroundjoin%
\definecolor{currentfill}{rgb}{0.501961,0.000000,0.000000}%
\pgfsetfillcolor{currentfill}%
\pgfsetfillopacity{0.400000}%
\pgfsetlinewidth{0.501875pt}%
\definecolor{currentstroke}{rgb}{0.501961,0.000000,0.000000}%
\pgfsetstrokecolor{currentstroke}%
\pgfsetstrokeopacity{0.400000}%
\pgfsetdash{}{0pt}%
\pgfpathmoveto{\pgfqpoint{1.187094in}{3.208056in}}%
\pgfpathcurveto{\pgfqpoint{1.198145in}{3.208056in}}{\pgfqpoint{1.208744in}{3.212446in}}{\pgfqpoint{1.216557in}{3.220260in}}%
\pgfpathcurveto{\pgfqpoint{1.224371in}{3.228074in}}{\pgfqpoint{1.228761in}{3.238673in}}{\pgfqpoint{1.228761in}{3.249723in}}%
\pgfpathcurveto{\pgfqpoint{1.228761in}{3.260773in}}{\pgfqpoint{1.224371in}{3.271372in}}{\pgfqpoint{1.216557in}{3.279185in}}%
\pgfpathcurveto{\pgfqpoint{1.208744in}{3.286999in}}{\pgfqpoint{1.198145in}{3.291389in}}{\pgfqpoint{1.187094in}{3.291389in}}%
\pgfpathcurveto{\pgfqpoint{1.176044in}{3.291389in}}{\pgfqpoint{1.165445in}{3.286999in}}{\pgfqpoint{1.157632in}{3.279185in}}%
\pgfpathcurveto{\pgfqpoint{1.149818in}{3.271372in}}{\pgfqpoint{1.145428in}{3.260773in}}{\pgfqpoint{1.145428in}{3.249723in}}%
\pgfpathcurveto{\pgfqpoint{1.145428in}{3.238673in}}{\pgfqpoint{1.149818in}{3.228074in}}{\pgfqpoint{1.157632in}{3.220260in}}%
\pgfpathcurveto{\pgfqpoint{1.165445in}{3.212446in}}{\pgfqpoint{1.176044in}{3.208056in}}{\pgfqpoint{1.187094in}{3.208056in}}%
\pgfpathclose%
\pgfusepath{stroke,fill}%
\end{pgfscope}%
\begin{pgfscope}%
\pgfpathrectangle{\pgfqpoint{0.511823in}{0.504323in}}{\pgfqpoint{3.218177in}{3.225677in}} %
\pgfusepath{clip}%
\pgfsetbuttcap%
\pgfsetroundjoin%
\definecolor{currentfill}{rgb}{0.501961,0.000000,0.000000}%
\pgfsetfillcolor{currentfill}%
\pgfsetfillopacity{0.400000}%
\pgfsetlinewidth{0.501875pt}%
\definecolor{currentstroke}{rgb}{0.501961,0.000000,0.000000}%
\pgfsetstrokecolor{currentstroke}%
\pgfsetstrokeopacity{0.400000}%
\pgfsetdash{}{0pt}%
\pgfpathmoveto{\pgfqpoint{1.135430in}{2.985043in}}%
\pgfpathcurveto{\pgfqpoint{1.146480in}{2.985043in}}{\pgfqpoint{1.157079in}{2.989433in}}{\pgfqpoint{1.164893in}{2.997247in}}%
\pgfpathcurveto{\pgfqpoint{1.172706in}{3.005060in}}{\pgfqpoint{1.177097in}{3.015659in}}{\pgfqpoint{1.177097in}{3.026709in}}%
\pgfpathcurveto{\pgfqpoint{1.177097in}{3.037759in}}{\pgfqpoint{1.172706in}{3.048359in}}{\pgfqpoint{1.164893in}{3.056172in}}%
\pgfpathcurveto{\pgfqpoint{1.157079in}{3.063986in}}{\pgfqpoint{1.146480in}{3.068376in}}{\pgfqpoint{1.135430in}{3.068376in}}%
\pgfpathcurveto{\pgfqpoint{1.124380in}{3.068376in}}{\pgfqpoint{1.113781in}{3.063986in}}{\pgfqpoint{1.105967in}{3.056172in}}%
\pgfpathcurveto{\pgfqpoint{1.098154in}{3.048359in}}{\pgfqpoint{1.093763in}{3.037759in}}{\pgfqpoint{1.093763in}{3.026709in}}%
\pgfpathcurveto{\pgfqpoint{1.093763in}{3.015659in}}{\pgfqpoint{1.098154in}{3.005060in}}{\pgfqpoint{1.105967in}{2.997247in}}%
\pgfpathcurveto{\pgfqpoint{1.113781in}{2.989433in}}{\pgfqpoint{1.124380in}{2.985043in}}{\pgfqpoint{1.135430in}{2.985043in}}%
\pgfpathclose%
\pgfusepath{stroke,fill}%
\end{pgfscope}%
\begin{pgfscope}%
\pgfpathrectangle{\pgfqpoint{0.511823in}{0.504323in}}{\pgfqpoint{3.218177in}{3.225677in}} %
\pgfusepath{clip}%
\pgfsetbuttcap%
\pgfsetroundjoin%
\definecolor{currentfill}{rgb}{0.501961,0.000000,0.000000}%
\pgfsetfillcolor{currentfill}%
\pgfsetfillopacity{0.400000}%
\pgfsetlinewidth{0.501875pt}%
\definecolor{currentstroke}{rgb}{0.501961,0.000000,0.000000}%
\pgfsetstrokecolor{currentstroke}%
\pgfsetstrokeopacity{0.400000}%
\pgfsetdash{}{0pt}%
\pgfpathmoveto{\pgfqpoint{1.169346in}{3.209017in}}%
\pgfpathcurveto{\pgfqpoint{1.180396in}{3.209017in}}{\pgfqpoint{1.190995in}{3.213407in}}{\pgfqpoint{1.198808in}{3.221221in}}%
\pgfpathcurveto{\pgfqpoint{1.206622in}{3.229034in}}{\pgfqpoint{1.211012in}{3.239633in}}{\pgfqpoint{1.211012in}{3.250683in}}%
\pgfpathcurveto{\pgfqpoint{1.211012in}{3.261733in}}{\pgfqpoint{1.206622in}{3.272333in}}{\pgfqpoint{1.198808in}{3.280146in}}%
\pgfpathcurveto{\pgfqpoint{1.190995in}{3.287960in}}{\pgfqpoint{1.180396in}{3.292350in}}{\pgfqpoint{1.169346in}{3.292350in}}%
\pgfpathcurveto{\pgfqpoint{1.158295in}{3.292350in}}{\pgfqpoint{1.147696in}{3.287960in}}{\pgfqpoint{1.139883in}{3.280146in}}%
\pgfpathcurveto{\pgfqpoint{1.132069in}{3.272333in}}{\pgfqpoint{1.127679in}{3.261733in}}{\pgfqpoint{1.127679in}{3.250683in}}%
\pgfpathcurveto{\pgfqpoint{1.127679in}{3.239633in}}{\pgfqpoint{1.132069in}{3.229034in}}{\pgfqpoint{1.139883in}{3.221221in}}%
\pgfpathcurveto{\pgfqpoint{1.147696in}{3.213407in}}{\pgfqpoint{1.158295in}{3.209017in}}{\pgfqpoint{1.169346in}{3.209017in}}%
\pgfpathclose%
\pgfusepath{stroke,fill}%
\end{pgfscope}%
\begin{pgfscope}%
\pgfpathrectangle{\pgfqpoint{0.511823in}{0.504323in}}{\pgfqpoint{3.218177in}{3.225677in}} %
\pgfusepath{clip}%
\pgfsetbuttcap%
\pgfsetroundjoin%
\definecolor{currentfill}{rgb}{0.501961,0.000000,0.000000}%
\pgfsetfillcolor{currentfill}%
\pgfsetfillopacity{0.400000}%
\pgfsetlinewidth{0.501875pt}%
\definecolor{currentstroke}{rgb}{0.501961,0.000000,0.000000}%
\pgfsetstrokecolor{currentstroke}%
\pgfsetstrokeopacity{0.400000}%
\pgfsetdash{}{0pt}%
\pgfpathmoveto{\pgfqpoint{1.140994in}{3.103857in}}%
\pgfpathcurveto{\pgfqpoint{1.152044in}{3.103857in}}{\pgfqpoint{1.162643in}{3.108247in}}{\pgfqpoint{1.170456in}{3.116061in}}%
\pgfpathcurveto{\pgfqpoint{1.178270in}{3.123875in}}{\pgfqpoint{1.182660in}{3.134474in}}{\pgfqpoint{1.182660in}{3.145524in}}%
\pgfpathcurveto{\pgfqpoint{1.182660in}{3.156574in}}{\pgfqpoint{1.178270in}{3.167173in}}{\pgfqpoint{1.170456in}{3.174987in}}%
\pgfpathcurveto{\pgfqpoint{1.162643in}{3.182800in}}{\pgfqpoint{1.152044in}{3.187190in}}{\pgfqpoint{1.140994in}{3.187190in}}%
\pgfpathcurveto{\pgfqpoint{1.129943in}{3.187190in}}{\pgfqpoint{1.119344in}{3.182800in}}{\pgfqpoint{1.111531in}{3.174987in}}%
\pgfpathcurveto{\pgfqpoint{1.103717in}{3.167173in}}{\pgfqpoint{1.099327in}{3.156574in}}{\pgfqpoint{1.099327in}{3.145524in}}%
\pgfpathcurveto{\pgfqpoint{1.099327in}{3.134474in}}{\pgfqpoint{1.103717in}{3.123875in}}{\pgfqpoint{1.111531in}{3.116061in}}%
\pgfpathcurveto{\pgfqpoint{1.119344in}{3.108247in}}{\pgfqpoint{1.129943in}{3.103857in}}{\pgfqpoint{1.140994in}{3.103857in}}%
\pgfpathclose%
\pgfusepath{stroke,fill}%
\end{pgfscope}%
\begin{pgfscope}%
\pgfpathrectangle{\pgfqpoint{0.511823in}{0.504323in}}{\pgfqpoint{3.218177in}{3.225677in}} %
\pgfusepath{clip}%
\pgfsetbuttcap%
\pgfsetroundjoin%
\definecolor{currentfill}{rgb}{0.501961,0.000000,0.000000}%
\pgfsetfillcolor{currentfill}%
\pgfsetfillopacity{0.400000}%
\pgfsetlinewidth{0.501875pt}%
\definecolor{currentstroke}{rgb}{0.501961,0.000000,0.000000}%
\pgfsetstrokecolor{currentstroke}%
\pgfsetstrokeopacity{0.400000}%
\pgfsetdash{}{0pt}%
\pgfpathmoveto{\pgfqpoint{1.145995in}{3.178880in}}%
\pgfpathcurveto{\pgfqpoint{1.157045in}{3.178880in}}{\pgfqpoint{1.167644in}{3.183271in}}{\pgfqpoint{1.175457in}{3.191084in}}%
\pgfpathcurveto{\pgfqpoint{1.183271in}{3.198898in}}{\pgfqpoint{1.187661in}{3.209497in}}{\pgfqpoint{1.187661in}{3.220547in}}%
\pgfpathcurveto{\pgfqpoint{1.187661in}{3.231597in}}{\pgfqpoint{1.183271in}{3.242196in}}{\pgfqpoint{1.175457in}{3.250010in}}%
\pgfpathcurveto{\pgfqpoint{1.167644in}{3.257823in}}{\pgfqpoint{1.157045in}{3.262214in}}{\pgfqpoint{1.145995in}{3.262214in}}%
\pgfpathcurveto{\pgfqpoint{1.134944in}{3.262214in}}{\pgfqpoint{1.124345in}{3.257823in}}{\pgfqpoint{1.116532in}{3.250010in}}%
\pgfpathcurveto{\pgfqpoint{1.108718in}{3.242196in}}{\pgfqpoint{1.104328in}{3.231597in}}{\pgfqpoint{1.104328in}{3.220547in}}%
\pgfpathcurveto{\pgfqpoint{1.104328in}{3.209497in}}{\pgfqpoint{1.108718in}{3.198898in}}{\pgfqpoint{1.116532in}{3.191084in}}%
\pgfpathcurveto{\pgfqpoint{1.124345in}{3.183271in}}{\pgfqpoint{1.134944in}{3.178880in}}{\pgfqpoint{1.145995in}{3.178880in}}%
\pgfpathclose%
\pgfusepath{stroke,fill}%
\end{pgfscope}%
\begin{pgfscope}%
\pgfpathrectangle{\pgfqpoint{0.511823in}{0.504323in}}{\pgfqpoint{3.218177in}{3.225677in}} %
\pgfusepath{clip}%
\pgfsetbuttcap%
\pgfsetroundjoin%
\definecolor{currentfill}{rgb}{0.501961,0.000000,0.000000}%
\pgfsetfillcolor{currentfill}%
\pgfsetfillopacity{0.400000}%
\pgfsetlinewidth{0.501875pt}%
\definecolor{currentstroke}{rgb}{0.501961,0.000000,0.000000}%
\pgfsetstrokecolor{currentstroke}%
\pgfsetstrokeopacity{0.400000}%
\pgfsetdash{}{0pt}%
\pgfpathmoveto{\pgfqpoint{1.101936in}{2.980419in}}%
\pgfpathcurveto{\pgfqpoint{1.112986in}{2.980419in}}{\pgfqpoint{1.123585in}{2.984810in}}{\pgfqpoint{1.131399in}{2.992623in}}%
\pgfpathcurveto{\pgfqpoint{1.139212in}{3.000437in}}{\pgfqpoint{1.143603in}{3.011036in}}{\pgfqpoint{1.143603in}{3.022086in}}%
\pgfpathcurveto{\pgfqpoint{1.143603in}{3.033136in}}{\pgfqpoint{1.139212in}{3.043735in}}{\pgfqpoint{1.131399in}{3.051549in}}%
\pgfpathcurveto{\pgfqpoint{1.123585in}{3.059362in}}{\pgfqpoint{1.112986in}{3.063753in}}{\pgfqpoint{1.101936in}{3.063753in}}%
\pgfpathcurveto{\pgfqpoint{1.090886in}{3.063753in}}{\pgfqpoint{1.080287in}{3.059362in}}{\pgfqpoint{1.072473in}{3.051549in}}%
\pgfpathcurveto{\pgfqpoint{1.064660in}{3.043735in}}{\pgfqpoint{1.060269in}{3.033136in}}{\pgfqpoint{1.060269in}{3.022086in}}%
\pgfpathcurveto{\pgfqpoint{1.060269in}{3.011036in}}{\pgfqpoint{1.064660in}{3.000437in}}{\pgfqpoint{1.072473in}{2.992623in}}%
\pgfpathcurveto{\pgfqpoint{1.080287in}{2.984810in}}{\pgfqpoint{1.090886in}{2.980419in}}{\pgfqpoint{1.101936in}{2.980419in}}%
\pgfpathclose%
\pgfusepath{stroke,fill}%
\end{pgfscope}%
\begin{pgfscope}%
\pgfpathrectangle{\pgfqpoint{0.511823in}{0.504323in}}{\pgfqpoint{3.218177in}{3.225677in}} %
\pgfusepath{clip}%
\pgfsetbuttcap%
\pgfsetroundjoin%
\definecolor{currentfill}{rgb}{0.501961,0.000000,0.000000}%
\pgfsetfillcolor{currentfill}%
\pgfsetfillopacity{0.400000}%
\pgfsetlinewidth{0.501875pt}%
\definecolor{currentstroke}{rgb}{0.501961,0.000000,0.000000}%
\pgfsetstrokecolor{currentstroke}%
\pgfsetstrokeopacity{0.400000}%
\pgfsetdash{}{0pt}%
\pgfpathmoveto{\pgfqpoint{1.106011in}{3.050551in}}%
\pgfpathcurveto{\pgfqpoint{1.117061in}{3.050551in}}{\pgfqpoint{1.127660in}{3.054941in}}{\pgfqpoint{1.135474in}{3.062755in}}%
\pgfpathcurveto{\pgfqpoint{1.143287in}{3.070569in}}{\pgfqpoint{1.147678in}{3.081168in}}{\pgfqpoint{1.147678in}{3.092218in}}%
\pgfpathcurveto{\pgfqpoint{1.147678in}{3.103268in}}{\pgfqpoint{1.143287in}{3.113867in}}{\pgfqpoint{1.135474in}{3.121680in}}%
\pgfpathcurveto{\pgfqpoint{1.127660in}{3.129494in}}{\pgfqpoint{1.117061in}{3.133884in}}{\pgfqpoint{1.106011in}{3.133884in}}%
\pgfpathcurveto{\pgfqpoint{1.094961in}{3.133884in}}{\pgfqpoint{1.084362in}{3.129494in}}{\pgfqpoint{1.076548in}{3.121680in}}%
\pgfpathcurveto{\pgfqpoint{1.068735in}{3.113867in}}{\pgfqpoint{1.064344in}{3.103268in}}{\pgfqpoint{1.064344in}{3.092218in}}%
\pgfpathcurveto{\pgfqpoint{1.064344in}{3.081168in}}{\pgfqpoint{1.068735in}{3.070569in}}{\pgfqpoint{1.076548in}{3.062755in}}%
\pgfpathcurveto{\pgfqpoint{1.084362in}{3.054941in}}{\pgfqpoint{1.094961in}{3.050551in}}{\pgfqpoint{1.106011in}{3.050551in}}%
\pgfpathclose%
\pgfusepath{stroke,fill}%
\end{pgfscope}%
\begin{pgfscope}%
\pgfpathrectangle{\pgfqpoint{0.511823in}{0.504323in}}{\pgfqpoint{3.218177in}{3.225677in}} %
\pgfusepath{clip}%
\pgfsetbuttcap%
\pgfsetroundjoin%
\definecolor{currentfill}{rgb}{0.501961,0.000000,0.000000}%
\pgfsetfillcolor{currentfill}%
\pgfsetfillopacity{0.400000}%
\pgfsetlinewidth{0.501875pt}%
\definecolor{currentstroke}{rgb}{0.501961,0.000000,0.000000}%
\pgfsetstrokecolor{currentstroke}%
\pgfsetstrokeopacity{0.400000}%
\pgfsetdash{}{0pt}%
\pgfpathmoveto{\pgfqpoint{1.064090in}{2.853824in}}%
\pgfpathcurveto{\pgfqpoint{1.075140in}{2.853824in}}{\pgfqpoint{1.085739in}{2.858214in}}{\pgfqpoint{1.093552in}{2.866028in}}%
\pgfpathcurveto{\pgfqpoint{1.101366in}{2.873841in}}{\pgfqpoint{1.105756in}{2.884440in}}{\pgfqpoint{1.105756in}{2.895491in}}%
\pgfpathcurveto{\pgfqpoint{1.105756in}{2.906541in}}{\pgfqpoint{1.101366in}{2.917140in}}{\pgfqpoint{1.093552in}{2.924953in}}%
\pgfpathcurveto{\pgfqpoint{1.085739in}{2.932767in}}{\pgfqpoint{1.075140in}{2.937157in}}{\pgfqpoint{1.064090in}{2.937157in}}%
\pgfpathcurveto{\pgfqpoint{1.053040in}{2.937157in}}{\pgfqpoint{1.042441in}{2.932767in}}{\pgfqpoint{1.034627in}{2.924953in}}%
\pgfpathcurveto{\pgfqpoint{1.026813in}{2.917140in}}{\pgfqpoint{1.022423in}{2.906541in}}{\pgfqpoint{1.022423in}{2.895491in}}%
\pgfpathcurveto{\pgfqpoint{1.022423in}{2.884440in}}{\pgfqpoint{1.026813in}{2.873841in}}{\pgfqpoint{1.034627in}{2.866028in}}%
\pgfpathcurveto{\pgfqpoint{1.042441in}{2.858214in}}{\pgfqpoint{1.053040in}{2.853824in}}{\pgfqpoint{1.064090in}{2.853824in}}%
\pgfpathclose%
\pgfusepath{stroke,fill}%
\end{pgfscope}%
\begin{pgfscope}%
\pgfpathrectangle{\pgfqpoint{0.511823in}{0.504323in}}{\pgfqpoint{3.218177in}{3.225677in}} %
\pgfusepath{clip}%
\pgfsetbuttcap%
\pgfsetroundjoin%
\definecolor{currentfill}{rgb}{0.501961,0.000000,0.000000}%
\pgfsetfillcolor{currentfill}%
\pgfsetfillopacity{0.400000}%
\pgfsetlinewidth{0.501875pt}%
\definecolor{currentstroke}{rgb}{0.501961,0.000000,0.000000}%
\pgfsetstrokecolor{currentstroke}%
\pgfsetstrokeopacity{0.400000}%
\pgfsetdash{}{0pt}%
\pgfpathmoveto{\pgfqpoint{1.101512in}{3.123704in}}%
\pgfpathcurveto{\pgfqpoint{1.112562in}{3.123704in}}{\pgfqpoint{1.123161in}{3.128094in}}{\pgfqpoint{1.130975in}{3.135908in}}%
\pgfpathcurveto{\pgfqpoint{1.138788in}{3.143722in}}{\pgfqpoint{1.143179in}{3.154321in}}{\pgfqpoint{1.143179in}{3.165371in}}%
\pgfpathcurveto{\pgfqpoint{1.143179in}{3.176421in}}{\pgfqpoint{1.138788in}{3.187020in}}{\pgfqpoint{1.130975in}{3.194833in}}%
\pgfpathcurveto{\pgfqpoint{1.123161in}{3.202647in}}{\pgfqpoint{1.112562in}{3.207037in}}{\pgfqpoint{1.101512in}{3.207037in}}%
\pgfpathcurveto{\pgfqpoint{1.090462in}{3.207037in}}{\pgfqpoint{1.079863in}{3.202647in}}{\pgfqpoint{1.072049in}{3.194833in}}%
\pgfpathcurveto{\pgfqpoint{1.064236in}{3.187020in}}{\pgfqpoint{1.059845in}{3.176421in}}{\pgfqpoint{1.059845in}{3.165371in}}%
\pgfpathcurveto{\pgfqpoint{1.059845in}{3.154321in}}{\pgfqpoint{1.064236in}{3.143722in}}{\pgfqpoint{1.072049in}{3.135908in}}%
\pgfpathcurveto{\pgfqpoint{1.079863in}{3.128094in}}{\pgfqpoint{1.090462in}{3.123704in}}{\pgfqpoint{1.101512in}{3.123704in}}%
\pgfpathclose%
\pgfusepath{stroke,fill}%
\end{pgfscope}%
\begin{pgfscope}%
\pgfpathrectangle{\pgfqpoint{0.511823in}{0.504323in}}{\pgfqpoint{3.218177in}{3.225677in}} %
\pgfusepath{clip}%
\pgfsetbuttcap%
\pgfsetroundjoin%
\definecolor{currentfill}{rgb}{0.501961,0.000000,0.000000}%
\pgfsetfillcolor{currentfill}%
\pgfsetfillopacity{0.400000}%
\pgfsetlinewidth{0.501875pt}%
\definecolor{currentstroke}{rgb}{0.501961,0.000000,0.000000}%
\pgfsetstrokecolor{currentstroke}%
\pgfsetstrokeopacity{0.400000}%
\pgfsetdash{}{0pt}%
\pgfpathmoveto{\pgfqpoint{1.100786in}{3.171836in}}%
\pgfpathcurveto{\pgfqpoint{1.111836in}{3.171836in}}{\pgfqpoint{1.122435in}{3.176226in}}{\pgfqpoint{1.130248in}{3.184039in}}%
\pgfpathcurveto{\pgfqpoint{1.138062in}{3.191853in}}{\pgfqpoint{1.142452in}{3.202452in}}{\pgfqpoint{1.142452in}{3.213502in}}%
\pgfpathcurveto{\pgfqpoint{1.142452in}{3.224552in}}{\pgfqpoint{1.138062in}{3.235151in}}{\pgfqpoint{1.130248in}{3.242965in}}%
\pgfpathcurveto{\pgfqpoint{1.122435in}{3.250779in}}{\pgfqpoint{1.111836in}{3.255169in}}{\pgfqpoint{1.100786in}{3.255169in}}%
\pgfpathcurveto{\pgfqpoint{1.089735in}{3.255169in}}{\pgfqpoint{1.079136in}{3.250779in}}{\pgfqpoint{1.071323in}{3.242965in}}%
\pgfpathcurveto{\pgfqpoint{1.063509in}{3.235151in}}{\pgfqpoint{1.059119in}{3.224552in}}{\pgfqpoint{1.059119in}{3.213502in}}%
\pgfpathcurveto{\pgfqpoint{1.059119in}{3.202452in}}{\pgfqpoint{1.063509in}{3.191853in}}{\pgfqpoint{1.071323in}{3.184039in}}%
\pgfpathcurveto{\pgfqpoint{1.079136in}{3.176226in}}{\pgfqpoint{1.089735in}{3.171836in}}{\pgfqpoint{1.100786in}{3.171836in}}%
\pgfpathclose%
\pgfusepath{stroke,fill}%
\end{pgfscope}%
\begin{pgfscope}%
\pgfpathrectangle{\pgfqpoint{0.511823in}{0.504323in}}{\pgfqpoint{3.218177in}{3.225677in}} %
\pgfusepath{clip}%
\pgfsetbuttcap%
\pgfsetroundjoin%
\definecolor{currentfill}{rgb}{0.501961,0.000000,0.000000}%
\pgfsetfillcolor{currentfill}%
\pgfsetfillopacity{0.400000}%
\pgfsetlinewidth{0.501875pt}%
\definecolor{currentstroke}{rgb}{0.501961,0.000000,0.000000}%
\pgfsetstrokecolor{currentstroke}%
\pgfsetstrokeopacity{0.400000}%
\pgfsetdash{}{0pt}%
\pgfpathmoveto{\pgfqpoint{1.085090in}{3.128542in}}%
\pgfpathcurveto{\pgfqpoint{1.096140in}{3.128542in}}{\pgfqpoint{1.106739in}{3.132932in}}{\pgfqpoint{1.114553in}{3.140746in}}%
\pgfpathcurveto{\pgfqpoint{1.122366in}{3.148559in}}{\pgfqpoint{1.126757in}{3.159158in}}{\pgfqpoint{1.126757in}{3.170208in}}%
\pgfpathcurveto{\pgfqpoint{1.126757in}{3.181259in}}{\pgfqpoint{1.122366in}{3.191858in}}{\pgfqpoint{1.114553in}{3.199671in}}%
\pgfpathcurveto{\pgfqpoint{1.106739in}{3.207485in}}{\pgfqpoint{1.096140in}{3.211875in}}{\pgfqpoint{1.085090in}{3.211875in}}%
\pgfpathcurveto{\pgfqpoint{1.074040in}{3.211875in}}{\pgfqpoint{1.063441in}{3.207485in}}{\pgfqpoint{1.055627in}{3.199671in}}%
\pgfpathcurveto{\pgfqpoint{1.047814in}{3.191858in}}{\pgfqpoint{1.043423in}{3.181259in}}{\pgfqpoint{1.043423in}{3.170208in}}%
\pgfpathcurveto{\pgfqpoint{1.043423in}{3.159158in}}{\pgfqpoint{1.047814in}{3.148559in}}{\pgfqpoint{1.055627in}{3.140746in}}%
\pgfpathcurveto{\pgfqpoint{1.063441in}{3.132932in}}{\pgfqpoint{1.074040in}{3.128542in}}{\pgfqpoint{1.085090in}{3.128542in}}%
\pgfpathclose%
\pgfusepath{stroke,fill}%
\end{pgfscope}%
\begin{pgfscope}%
\pgfpathrectangle{\pgfqpoint{0.511823in}{0.504323in}}{\pgfqpoint{3.218177in}{3.225677in}} %
\pgfusepath{clip}%
\pgfsetbuttcap%
\pgfsetroundjoin%
\definecolor{currentfill}{rgb}{0.501961,0.000000,0.000000}%
\pgfsetfillcolor{currentfill}%
\pgfsetfillopacity{0.400000}%
\pgfsetlinewidth{0.501875pt}%
\definecolor{currentstroke}{rgb}{0.501961,0.000000,0.000000}%
\pgfsetstrokecolor{currentstroke}%
\pgfsetstrokeopacity{0.400000}%
\pgfsetdash{}{0pt}%
\pgfpathmoveto{\pgfqpoint{1.091312in}{3.223136in}}%
\pgfpathcurveto{\pgfqpoint{1.102362in}{3.223136in}}{\pgfqpoint{1.112961in}{3.227526in}}{\pgfqpoint{1.120775in}{3.235340in}}%
\pgfpathcurveto{\pgfqpoint{1.128588in}{3.243153in}}{\pgfqpoint{1.132979in}{3.253752in}}{\pgfqpoint{1.132979in}{3.264803in}}%
\pgfpathcurveto{\pgfqpoint{1.132979in}{3.275853in}}{\pgfqpoint{1.128588in}{3.286452in}}{\pgfqpoint{1.120775in}{3.294265in}}%
\pgfpathcurveto{\pgfqpoint{1.112961in}{3.302079in}}{\pgfqpoint{1.102362in}{3.306469in}}{\pgfqpoint{1.091312in}{3.306469in}}%
\pgfpathcurveto{\pgfqpoint{1.080262in}{3.306469in}}{\pgfqpoint{1.069663in}{3.302079in}}{\pgfqpoint{1.061849in}{3.294265in}}%
\pgfpathcurveto{\pgfqpoint{1.054036in}{3.286452in}}{\pgfqpoint{1.049645in}{3.275853in}}{\pgfqpoint{1.049645in}{3.264803in}}%
\pgfpathcurveto{\pgfqpoint{1.049645in}{3.253752in}}{\pgfqpoint{1.054036in}{3.243153in}}{\pgfqpoint{1.061849in}{3.235340in}}%
\pgfpathcurveto{\pgfqpoint{1.069663in}{3.227526in}}{\pgfqpoint{1.080262in}{3.223136in}}{\pgfqpoint{1.091312in}{3.223136in}}%
\pgfpathclose%
\pgfusepath{stroke,fill}%
\end{pgfscope}%
\begin{pgfscope}%
\pgfpathrectangle{\pgfqpoint{0.511823in}{0.504323in}}{\pgfqpoint{3.218177in}{3.225677in}} %
\pgfusepath{clip}%
\pgfsetbuttcap%
\pgfsetroundjoin%
\definecolor{currentfill}{rgb}{0.501961,0.000000,0.000000}%
\pgfsetfillcolor{currentfill}%
\pgfsetfillopacity{0.400000}%
\pgfsetlinewidth{0.501875pt}%
\definecolor{currentstroke}{rgb}{0.501961,0.000000,0.000000}%
\pgfsetstrokecolor{currentstroke}%
\pgfsetstrokeopacity{0.400000}%
\pgfsetdash{}{0pt}%
\pgfpathmoveto{\pgfqpoint{1.072940in}{3.161478in}}%
\pgfpathcurveto{\pgfqpoint{1.083990in}{3.161478in}}{\pgfqpoint{1.094589in}{3.165868in}}{\pgfqpoint{1.102403in}{3.173682in}}%
\pgfpathcurveto{\pgfqpoint{1.110216in}{3.181495in}}{\pgfqpoint{1.114607in}{3.192095in}}{\pgfqpoint{1.114607in}{3.203145in}}%
\pgfpathcurveto{\pgfqpoint{1.114607in}{3.214195in}}{\pgfqpoint{1.110216in}{3.224794in}}{\pgfqpoint{1.102403in}{3.232607in}}%
\pgfpathcurveto{\pgfqpoint{1.094589in}{3.240421in}}{\pgfqpoint{1.083990in}{3.244811in}}{\pgfqpoint{1.072940in}{3.244811in}}%
\pgfpathcurveto{\pgfqpoint{1.061890in}{3.244811in}}{\pgfqpoint{1.051291in}{3.240421in}}{\pgfqpoint{1.043477in}{3.232607in}}%
\pgfpathcurveto{\pgfqpoint{1.035664in}{3.224794in}}{\pgfqpoint{1.031273in}{3.214195in}}{\pgfqpoint{1.031273in}{3.203145in}}%
\pgfpathcurveto{\pgfqpoint{1.031273in}{3.192095in}}{\pgfqpoint{1.035664in}{3.181495in}}{\pgfqpoint{1.043477in}{3.173682in}}%
\pgfpathcurveto{\pgfqpoint{1.051291in}{3.165868in}}{\pgfqpoint{1.061890in}{3.161478in}}{\pgfqpoint{1.072940in}{3.161478in}}%
\pgfpathclose%
\pgfusepath{stroke,fill}%
\end{pgfscope}%
\begin{pgfscope}%
\pgfpathrectangle{\pgfqpoint{0.511823in}{0.504323in}}{\pgfqpoint{3.218177in}{3.225677in}} %
\pgfusepath{clip}%
\pgfsetbuttcap%
\pgfsetroundjoin%
\definecolor{currentfill}{rgb}{0.501961,0.000000,0.000000}%
\pgfsetfillcolor{currentfill}%
\pgfsetfillopacity{0.400000}%
\pgfsetlinewidth{0.501875pt}%
\definecolor{currentstroke}{rgb}{0.501961,0.000000,0.000000}%
\pgfsetstrokecolor{currentstroke}%
\pgfsetstrokeopacity{0.400000}%
\pgfsetdash{}{0pt}%
\pgfpathmoveto{\pgfqpoint{1.055790in}{3.105085in}}%
\pgfpathcurveto{\pgfqpoint{1.066840in}{3.105085in}}{\pgfqpoint{1.077439in}{3.109475in}}{\pgfqpoint{1.085253in}{3.117289in}}%
\pgfpathcurveto{\pgfqpoint{1.093067in}{3.125102in}}{\pgfqpoint{1.097457in}{3.135701in}}{\pgfqpoint{1.097457in}{3.146752in}}%
\pgfpathcurveto{\pgfqpoint{1.097457in}{3.157802in}}{\pgfqpoint{1.093067in}{3.168401in}}{\pgfqpoint{1.085253in}{3.176214in}}%
\pgfpathcurveto{\pgfqpoint{1.077439in}{3.184028in}}{\pgfqpoint{1.066840in}{3.188418in}}{\pgfqpoint{1.055790in}{3.188418in}}%
\pgfpathcurveto{\pgfqpoint{1.044740in}{3.188418in}}{\pgfqpoint{1.034141in}{3.184028in}}{\pgfqpoint{1.026327in}{3.176214in}}%
\pgfpathcurveto{\pgfqpoint{1.018514in}{3.168401in}}{\pgfqpoint{1.014124in}{3.157802in}}{\pgfqpoint{1.014124in}{3.146752in}}%
\pgfpathcurveto{\pgfqpoint{1.014124in}{3.135701in}}{\pgfqpoint{1.018514in}{3.125102in}}{\pgfqpoint{1.026327in}{3.117289in}}%
\pgfpathcurveto{\pgfqpoint{1.034141in}{3.109475in}}{\pgfqpoint{1.044740in}{3.105085in}}{\pgfqpoint{1.055790in}{3.105085in}}%
\pgfpathclose%
\pgfusepath{stroke,fill}%
\end{pgfscope}%
\begin{pgfscope}%
\pgfpathrectangle{\pgfqpoint{0.511823in}{0.504323in}}{\pgfqpoint{3.218177in}{3.225677in}} %
\pgfusepath{clip}%
\pgfsetbuttcap%
\pgfsetroundjoin%
\definecolor{currentfill}{rgb}{0.501961,0.000000,0.000000}%
\pgfsetfillcolor{currentfill}%
\pgfsetfillopacity{0.400000}%
\pgfsetlinewidth{0.501875pt}%
\definecolor{currentstroke}{rgb}{0.501961,0.000000,0.000000}%
\pgfsetstrokecolor{currentstroke}%
\pgfsetstrokeopacity{0.400000}%
\pgfsetdash{}{0pt}%
\pgfpathmoveto{\pgfqpoint{1.055105in}{3.158469in}}%
\pgfpathcurveto{\pgfqpoint{1.066155in}{3.158469in}}{\pgfqpoint{1.076754in}{3.162860in}}{\pgfqpoint{1.084568in}{3.170673in}}%
\pgfpathcurveto{\pgfqpoint{1.092382in}{3.178487in}}{\pgfqpoint{1.096772in}{3.189086in}}{\pgfqpoint{1.096772in}{3.200136in}}%
\pgfpathcurveto{\pgfqpoint{1.096772in}{3.211186in}}{\pgfqpoint{1.092382in}{3.221785in}}{\pgfqpoint{1.084568in}{3.229599in}}%
\pgfpathcurveto{\pgfqpoint{1.076754in}{3.237412in}}{\pgfqpoint{1.066155in}{3.241803in}}{\pgfqpoint{1.055105in}{3.241803in}}%
\pgfpathcurveto{\pgfqpoint{1.044055in}{3.241803in}}{\pgfqpoint{1.033456in}{3.237412in}}{\pgfqpoint{1.025642in}{3.229599in}}%
\pgfpathcurveto{\pgfqpoint{1.017829in}{3.221785in}}{\pgfqpoint{1.013439in}{3.211186in}}{\pgfqpoint{1.013439in}{3.200136in}}%
\pgfpathcurveto{\pgfqpoint{1.013439in}{3.189086in}}{\pgfqpoint{1.017829in}{3.178487in}}{\pgfqpoint{1.025642in}{3.170673in}}%
\pgfpathcurveto{\pgfqpoint{1.033456in}{3.162860in}}{\pgfqpoint{1.044055in}{3.158469in}}{\pgfqpoint{1.055105in}{3.158469in}}%
\pgfpathclose%
\pgfusepath{stroke,fill}%
\end{pgfscope}%
\begin{pgfscope}%
\pgfpathrectangle{\pgfqpoint{0.511823in}{0.504323in}}{\pgfqpoint{3.218177in}{3.225677in}} %
\pgfusepath{clip}%
\pgfsetbuttcap%
\pgfsetroundjoin%
\definecolor{currentfill}{rgb}{0.501961,0.000000,0.000000}%
\pgfsetfillcolor{currentfill}%
\pgfsetfillopacity{0.400000}%
\pgfsetlinewidth{0.501875pt}%
\definecolor{currentstroke}{rgb}{0.501961,0.000000,0.000000}%
\pgfsetstrokecolor{currentstroke}%
\pgfsetstrokeopacity{0.400000}%
\pgfsetdash{}{0pt}%
\pgfpathmoveto{\pgfqpoint{1.027414in}{3.025427in}}%
\pgfpathcurveto{\pgfqpoint{1.038465in}{3.025427in}}{\pgfqpoint{1.049064in}{3.029817in}}{\pgfqpoint{1.056877in}{3.037631in}}%
\pgfpathcurveto{\pgfqpoint{1.064691in}{3.045445in}}{\pgfqpoint{1.069081in}{3.056044in}}{\pgfqpoint{1.069081in}{3.067094in}}%
\pgfpathcurveto{\pgfqpoint{1.069081in}{3.078144in}}{\pgfqpoint{1.064691in}{3.088743in}}{\pgfqpoint{1.056877in}{3.096557in}}%
\pgfpathcurveto{\pgfqpoint{1.049064in}{3.104370in}}{\pgfqpoint{1.038465in}{3.108760in}}{\pgfqpoint{1.027414in}{3.108760in}}%
\pgfpathcurveto{\pgfqpoint{1.016364in}{3.108760in}}{\pgfqpoint{1.005765in}{3.104370in}}{\pgfqpoint{0.997952in}{3.096557in}}%
\pgfpathcurveto{\pgfqpoint{0.990138in}{3.088743in}}{\pgfqpoint{0.985748in}{3.078144in}}{\pgfqpoint{0.985748in}{3.067094in}}%
\pgfpathcurveto{\pgfqpoint{0.985748in}{3.056044in}}{\pgfqpoint{0.990138in}{3.045445in}}{\pgfqpoint{0.997952in}{3.037631in}}%
\pgfpathcurveto{\pgfqpoint{1.005765in}{3.029817in}}{\pgfqpoint{1.016364in}{3.025427in}}{\pgfqpoint{1.027414in}{3.025427in}}%
\pgfpathclose%
\pgfusepath{stroke,fill}%
\end{pgfscope}%
\begin{pgfscope}%
\pgfpathrectangle{\pgfqpoint{0.511823in}{0.504323in}}{\pgfqpoint{3.218177in}{3.225677in}} %
\pgfusepath{clip}%
\pgfsetbuttcap%
\pgfsetroundjoin%
\definecolor{currentfill}{rgb}{0.501961,0.000000,0.000000}%
\pgfsetfillcolor{currentfill}%
\pgfsetfillopacity{0.400000}%
\pgfsetlinewidth{0.501875pt}%
\definecolor{currentstroke}{rgb}{0.501961,0.000000,0.000000}%
\pgfsetstrokecolor{currentstroke}%
\pgfsetstrokeopacity{0.400000}%
\pgfsetdash{}{0pt}%
\pgfpathmoveto{\pgfqpoint{1.044472in}{3.206576in}}%
\pgfpathcurveto{\pgfqpoint{1.055522in}{3.206576in}}{\pgfqpoint{1.066121in}{3.210966in}}{\pgfqpoint{1.073934in}{3.218780in}}%
\pgfpathcurveto{\pgfqpoint{1.081748in}{3.226594in}}{\pgfqpoint{1.086138in}{3.237193in}}{\pgfqpoint{1.086138in}{3.248243in}}%
\pgfpathcurveto{\pgfqpoint{1.086138in}{3.259293in}}{\pgfqpoint{1.081748in}{3.269892in}}{\pgfqpoint{1.073934in}{3.277706in}}%
\pgfpathcurveto{\pgfqpoint{1.066121in}{3.285519in}}{\pgfqpoint{1.055522in}{3.289910in}}{\pgfqpoint{1.044472in}{3.289910in}}%
\pgfpathcurveto{\pgfqpoint{1.033421in}{3.289910in}}{\pgfqpoint{1.022822in}{3.285519in}}{\pgfqpoint{1.015009in}{3.277706in}}%
\pgfpathcurveto{\pgfqpoint{1.007195in}{3.269892in}}{\pgfqpoint{1.002805in}{3.259293in}}{\pgfqpoint{1.002805in}{3.248243in}}%
\pgfpathcurveto{\pgfqpoint{1.002805in}{3.237193in}}{\pgfqpoint{1.007195in}{3.226594in}}{\pgfqpoint{1.015009in}{3.218780in}}%
\pgfpathcurveto{\pgfqpoint{1.022822in}{3.210966in}}{\pgfqpoint{1.033421in}{3.206576in}}{\pgfqpoint{1.044472in}{3.206576in}}%
\pgfpathclose%
\pgfusepath{stroke,fill}%
\end{pgfscope}%
\begin{pgfscope}%
\pgfpathrectangle{\pgfqpoint{0.511823in}{0.504323in}}{\pgfqpoint{3.218177in}{3.225677in}} %
\pgfusepath{clip}%
\pgfsetbuttcap%
\pgfsetroundjoin%
\definecolor{currentfill}{rgb}{0.501961,0.000000,0.000000}%
\pgfsetfillcolor{currentfill}%
\pgfsetfillopacity{0.400000}%
\pgfsetlinewidth{0.501875pt}%
\definecolor{currentstroke}{rgb}{0.501961,0.000000,0.000000}%
\pgfsetstrokecolor{currentstroke}%
\pgfsetstrokeopacity{0.400000}%
\pgfsetdash{}{0pt}%
\pgfpathmoveto{\pgfqpoint{1.025003in}{3.128354in}}%
\pgfpathcurveto{\pgfqpoint{1.036053in}{3.128354in}}{\pgfqpoint{1.046652in}{3.132744in}}{\pgfqpoint{1.054466in}{3.140558in}}%
\pgfpathcurveto{\pgfqpoint{1.062279in}{3.148371in}}{\pgfqpoint{1.066670in}{3.158970in}}{\pgfqpoint{1.066670in}{3.170020in}}%
\pgfpathcurveto{\pgfqpoint{1.066670in}{3.181071in}}{\pgfqpoint{1.062279in}{3.191670in}}{\pgfqpoint{1.054466in}{3.199483in}}%
\pgfpathcurveto{\pgfqpoint{1.046652in}{3.207297in}}{\pgfqpoint{1.036053in}{3.211687in}}{\pgfqpoint{1.025003in}{3.211687in}}%
\pgfpathcurveto{\pgfqpoint{1.013953in}{3.211687in}}{\pgfqpoint{1.003354in}{3.207297in}}{\pgfqpoint{0.995540in}{3.199483in}}%
\pgfpathcurveto{\pgfqpoint{0.987727in}{3.191670in}}{\pgfqpoint{0.983336in}{3.181071in}}{\pgfqpoint{0.983336in}{3.170020in}}%
\pgfpathcurveto{\pgfqpoint{0.983336in}{3.158970in}}{\pgfqpoint{0.987727in}{3.148371in}}{\pgfqpoint{0.995540in}{3.140558in}}%
\pgfpathcurveto{\pgfqpoint{1.003354in}{3.132744in}}{\pgfqpoint{1.013953in}{3.128354in}}{\pgfqpoint{1.025003in}{3.128354in}}%
\pgfpathclose%
\pgfusepath{stroke,fill}%
\end{pgfscope}%
\begin{pgfscope}%
\pgfpathrectangle{\pgfqpoint{0.511823in}{0.504323in}}{\pgfqpoint{3.218177in}{3.225677in}} %
\pgfusepath{clip}%
\pgfsetbuttcap%
\pgfsetroundjoin%
\definecolor{currentfill}{rgb}{0.501961,0.000000,0.000000}%
\pgfsetfillcolor{currentfill}%
\pgfsetfillopacity{0.400000}%
\pgfsetlinewidth{0.501875pt}%
\definecolor{currentstroke}{rgb}{0.501961,0.000000,0.000000}%
\pgfsetstrokecolor{currentstroke}%
\pgfsetstrokeopacity{0.400000}%
\pgfsetdash{}{0pt}%
\pgfpathmoveto{\pgfqpoint{1.017574in}{3.136812in}}%
\pgfpathcurveto{\pgfqpoint{1.028624in}{3.136812in}}{\pgfqpoint{1.039223in}{3.141203in}}{\pgfqpoint{1.047037in}{3.149016in}}%
\pgfpathcurveto{\pgfqpoint{1.054850in}{3.156830in}}{\pgfqpoint{1.059241in}{3.167429in}}{\pgfqpoint{1.059241in}{3.178479in}}%
\pgfpathcurveto{\pgfqpoint{1.059241in}{3.189529in}}{\pgfqpoint{1.054850in}{3.200128in}}{\pgfqpoint{1.047037in}{3.207942in}}%
\pgfpathcurveto{\pgfqpoint{1.039223in}{3.215755in}}{\pgfqpoint{1.028624in}{3.220146in}}{\pgfqpoint{1.017574in}{3.220146in}}%
\pgfpathcurveto{\pgfqpoint{1.006524in}{3.220146in}}{\pgfqpoint{0.995925in}{3.215755in}}{\pgfqpoint{0.988111in}{3.207942in}}%
\pgfpathcurveto{\pgfqpoint{0.980297in}{3.200128in}}{\pgfqpoint{0.975907in}{3.189529in}}{\pgfqpoint{0.975907in}{3.178479in}}%
\pgfpathcurveto{\pgfqpoint{0.975907in}{3.167429in}}{\pgfqpoint{0.980297in}{3.156830in}}{\pgfqpoint{0.988111in}{3.149016in}}%
\pgfpathcurveto{\pgfqpoint{0.995925in}{3.141203in}}{\pgfqpoint{1.006524in}{3.136812in}}{\pgfqpoint{1.017574in}{3.136812in}}%
\pgfpathclose%
\pgfusepath{stroke,fill}%
\end{pgfscope}%
\begin{pgfscope}%
\pgfpathrectangle{\pgfqpoint{0.511823in}{0.504323in}}{\pgfqpoint{3.218177in}{3.225677in}} %
\pgfusepath{clip}%
\pgfsetbuttcap%
\pgfsetroundjoin%
\definecolor{currentfill}{rgb}{0.501961,0.000000,0.000000}%
\pgfsetfillcolor{currentfill}%
\pgfsetfillopacity{0.400000}%
\pgfsetlinewidth{0.501875pt}%
\definecolor{currentstroke}{rgb}{0.501961,0.000000,0.000000}%
\pgfsetstrokecolor{currentstroke}%
\pgfsetstrokeopacity{0.400000}%
\pgfsetdash{}{0pt}%
\pgfpathmoveto{\pgfqpoint{1.001694in}{3.080251in}}%
\pgfpathcurveto{\pgfqpoint{1.012744in}{3.080251in}}{\pgfqpoint{1.023343in}{3.084641in}}{\pgfqpoint{1.031157in}{3.092455in}}%
\pgfpathcurveto{\pgfqpoint{1.038971in}{3.100268in}}{\pgfqpoint{1.043361in}{3.110867in}}{\pgfqpoint{1.043361in}{3.121917in}}%
\pgfpathcurveto{\pgfqpoint{1.043361in}{3.132967in}}{\pgfqpoint{1.038971in}{3.143566in}}{\pgfqpoint{1.031157in}{3.151380in}}%
\pgfpathcurveto{\pgfqpoint{1.023343in}{3.159194in}}{\pgfqpoint{1.012744in}{3.163584in}}{\pgfqpoint{1.001694in}{3.163584in}}%
\pgfpathcurveto{\pgfqpoint{0.990644in}{3.163584in}}{\pgfqpoint{0.980045in}{3.159194in}}{\pgfqpoint{0.972231in}{3.151380in}}%
\pgfpathcurveto{\pgfqpoint{0.964418in}{3.143566in}}{\pgfqpoint{0.960027in}{3.132967in}}{\pgfqpoint{0.960027in}{3.121917in}}%
\pgfpathcurveto{\pgfqpoint{0.960027in}{3.110867in}}{\pgfqpoint{0.964418in}{3.100268in}}{\pgfqpoint{0.972231in}{3.092455in}}%
\pgfpathcurveto{\pgfqpoint{0.980045in}{3.084641in}}{\pgfqpoint{0.990644in}{3.080251in}}{\pgfqpoint{1.001694in}{3.080251in}}%
\pgfpathclose%
\pgfusepath{stroke,fill}%
\end{pgfscope}%
\begin{pgfscope}%
\pgfpathrectangle{\pgfqpoint{0.511823in}{0.504323in}}{\pgfqpoint{3.218177in}{3.225677in}} %
\pgfusepath{clip}%
\pgfsetbuttcap%
\pgfsetroundjoin%
\definecolor{currentfill}{rgb}{0.501961,0.000000,0.000000}%
\pgfsetfillcolor{currentfill}%
\pgfsetfillopacity{0.400000}%
\pgfsetlinewidth{0.501875pt}%
\definecolor{currentstroke}{rgb}{0.501961,0.000000,0.000000}%
\pgfsetstrokecolor{currentstroke}%
\pgfsetstrokeopacity{0.400000}%
\pgfsetdash{}{0pt}%
\pgfpathmoveto{\pgfqpoint{0.986654in}{3.027229in}}%
\pgfpathcurveto{\pgfqpoint{0.997704in}{3.027229in}}{\pgfqpoint{1.008303in}{3.031619in}}{\pgfqpoint{1.016117in}{3.039433in}}%
\pgfpathcurveto{\pgfqpoint{1.023931in}{3.047246in}}{\pgfqpoint{1.028321in}{3.057846in}}{\pgfqpoint{1.028321in}{3.068896in}}%
\pgfpathcurveto{\pgfqpoint{1.028321in}{3.079946in}}{\pgfqpoint{1.023931in}{3.090545in}}{\pgfqpoint{1.016117in}{3.098358in}}%
\pgfpathcurveto{\pgfqpoint{1.008303in}{3.106172in}}{\pgfqpoint{0.997704in}{3.110562in}}{\pgfqpoint{0.986654in}{3.110562in}}%
\pgfpathcurveto{\pgfqpoint{0.975604in}{3.110562in}}{\pgfqpoint{0.965005in}{3.106172in}}{\pgfqpoint{0.957191in}{3.098358in}}%
\pgfpathcurveto{\pgfqpoint{0.949378in}{3.090545in}}{\pgfqpoint{0.944987in}{3.079946in}}{\pgfqpoint{0.944987in}{3.068896in}}%
\pgfpathcurveto{\pgfqpoint{0.944987in}{3.057846in}}{\pgfqpoint{0.949378in}{3.047246in}}{\pgfqpoint{0.957191in}{3.039433in}}%
\pgfpathcurveto{\pgfqpoint{0.965005in}{3.031619in}}{\pgfqpoint{0.975604in}{3.027229in}}{\pgfqpoint{0.986654in}{3.027229in}}%
\pgfpathclose%
\pgfusepath{stroke,fill}%
\end{pgfscope}%
\begin{pgfscope}%
\pgfpathrectangle{\pgfqpoint{0.511823in}{0.504323in}}{\pgfqpoint{3.218177in}{3.225677in}} %
\pgfusepath{clip}%
\pgfsetbuttcap%
\pgfsetroundjoin%
\definecolor{currentfill}{rgb}{0.501961,0.000000,0.000000}%
\pgfsetfillcolor{currentfill}%
\pgfsetfillopacity{0.400000}%
\pgfsetlinewidth{0.501875pt}%
\definecolor{currentstroke}{rgb}{0.501961,0.000000,0.000000}%
\pgfsetstrokecolor{currentstroke}%
\pgfsetstrokeopacity{0.400000}%
\pgfsetdash{}{0pt}%
\pgfpathmoveto{\pgfqpoint{0.973036in}{2.982809in}}%
\pgfpathcurveto{\pgfqpoint{0.984086in}{2.982809in}}{\pgfqpoint{0.994685in}{2.987199in}}{\pgfqpoint{1.002499in}{2.995013in}}%
\pgfpathcurveto{\pgfqpoint{1.010312in}{3.002826in}}{\pgfqpoint{1.014703in}{3.013425in}}{\pgfqpoint{1.014703in}{3.024476in}}%
\pgfpathcurveto{\pgfqpoint{1.014703in}{3.035526in}}{\pgfqpoint{1.010312in}{3.046125in}}{\pgfqpoint{1.002499in}{3.053938in}}%
\pgfpathcurveto{\pgfqpoint{0.994685in}{3.061752in}}{\pgfqpoint{0.984086in}{3.066142in}}{\pgfqpoint{0.973036in}{3.066142in}}%
\pgfpathcurveto{\pgfqpoint{0.961986in}{3.066142in}}{\pgfqpoint{0.951387in}{3.061752in}}{\pgfqpoint{0.943573in}{3.053938in}}%
\pgfpathcurveto{\pgfqpoint{0.935760in}{3.046125in}}{\pgfqpoint{0.931369in}{3.035526in}}{\pgfqpoint{0.931369in}{3.024476in}}%
\pgfpathcurveto{\pgfqpoint{0.931369in}{3.013425in}}{\pgfqpoint{0.935760in}{3.002826in}}{\pgfqpoint{0.943573in}{2.995013in}}%
\pgfpathcurveto{\pgfqpoint{0.951387in}{2.987199in}}{\pgfqpoint{0.961986in}{2.982809in}}{\pgfqpoint{0.973036in}{2.982809in}}%
\pgfpathclose%
\pgfusepath{stroke,fill}%
\end{pgfscope}%
\begin{pgfscope}%
\pgfpathrectangle{\pgfqpoint{0.511823in}{0.504323in}}{\pgfqpoint{3.218177in}{3.225677in}} %
\pgfusepath{clip}%
\pgfsetbuttcap%
\pgfsetroundjoin%
\definecolor{currentfill}{rgb}{0.501961,0.000000,0.000000}%
\pgfsetfillcolor{currentfill}%
\pgfsetfillopacity{0.400000}%
\pgfsetlinewidth{0.501875pt}%
\definecolor{currentstroke}{rgb}{0.501961,0.000000,0.000000}%
\pgfsetstrokecolor{currentstroke}%
\pgfsetstrokeopacity{0.400000}%
\pgfsetdash{}{0pt}%
\pgfpathmoveto{\pgfqpoint{0.974930in}{3.066255in}}%
\pgfpathcurveto{\pgfqpoint{0.985980in}{3.066255in}}{\pgfqpoint{0.996579in}{3.070645in}}{\pgfqpoint{1.004393in}{3.078459in}}%
\pgfpathcurveto{\pgfqpoint{1.012207in}{3.086272in}}{\pgfqpoint{1.016597in}{3.096872in}}{\pgfqpoint{1.016597in}{3.107922in}}%
\pgfpathcurveto{\pgfqpoint{1.016597in}{3.118972in}}{\pgfqpoint{1.012207in}{3.129571in}}{\pgfqpoint{1.004393in}{3.137384in}}%
\pgfpathcurveto{\pgfqpoint{0.996579in}{3.145198in}}{\pgfqpoint{0.985980in}{3.149588in}}{\pgfqpoint{0.974930in}{3.149588in}}%
\pgfpathcurveto{\pgfqpoint{0.963880in}{3.149588in}}{\pgfqpoint{0.953281in}{3.145198in}}{\pgfqpoint{0.945468in}{3.137384in}}%
\pgfpathcurveto{\pgfqpoint{0.937654in}{3.129571in}}{\pgfqpoint{0.933264in}{3.118972in}}{\pgfqpoint{0.933264in}{3.107922in}}%
\pgfpathcurveto{\pgfqpoint{0.933264in}{3.096872in}}{\pgfqpoint{0.937654in}{3.086272in}}{\pgfqpoint{0.945468in}{3.078459in}}%
\pgfpathcurveto{\pgfqpoint{0.953281in}{3.070645in}}{\pgfqpoint{0.963880in}{3.066255in}}{\pgfqpoint{0.974930in}{3.066255in}}%
\pgfpathclose%
\pgfusepath{stroke,fill}%
\end{pgfscope}%
\begin{pgfscope}%
\pgfpathrectangle{\pgfqpoint{0.511823in}{0.504323in}}{\pgfqpoint{3.218177in}{3.225677in}} %
\pgfusepath{clip}%
\pgfsetbuttcap%
\pgfsetroundjoin%
\definecolor{currentfill}{rgb}{0.501961,0.000000,0.000000}%
\pgfsetfillcolor{currentfill}%
\pgfsetfillopacity{0.400000}%
\pgfsetlinewidth{0.501875pt}%
\definecolor{currentstroke}{rgb}{0.501961,0.000000,0.000000}%
\pgfsetstrokecolor{currentstroke}%
\pgfsetstrokeopacity{0.400000}%
\pgfsetdash{}{0pt}%
\pgfpathmoveto{\pgfqpoint{0.980792in}{3.188959in}}%
\pgfpathcurveto{\pgfqpoint{0.991842in}{3.188959in}}{\pgfqpoint{1.002441in}{3.193349in}}{\pgfqpoint{1.010255in}{3.201163in}}%
\pgfpathcurveto{\pgfqpoint{1.018068in}{3.208976in}}{\pgfqpoint{1.022459in}{3.219575in}}{\pgfqpoint{1.022459in}{3.230625in}}%
\pgfpathcurveto{\pgfqpoint{1.022459in}{3.241676in}}{\pgfqpoint{1.018068in}{3.252275in}}{\pgfqpoint{1.010255in}{3.260088in}}%
\pgfpathcurveto{\pgfqpoint{1.002441in}{3.267902in}}{\pgfqpoint{0.991842in}{3.272292in}}{\pgfqpoint{0.980792in}{3.272292in}}%
\pgfpathcurveto{\pgfqpoint{0.969742in}{3.272292in}}{\pgfqpoint{0.959143in}{3.267902in}}{\pgfqpoint{0.951329in}{3.260088in}}%
\pgfpathcurveto{\pgfqpoint{0.943516in}{3.252275in}}{\pgfqpoint{0.939125in}{3.241676in}}{\pgfqpoint{0.939125in}{3.230625in}}%
\pgfpathcurveto{\pgfqpoint{0.939125in}{3.219575in}}{\pgfqpoint{0.943516in}{3.208976in}}{\pgfqpoint{0.951329in}{3.201163in}}%
\pgfpathcurveto{\pgfqpoint{0.959143in}{3.193349in}}{\pgfqpoint{0.969742in}{3.188959in}}{\pgfqpoint{0.980792in}{3.188959in}}%
\pgfpathclose%
\pgfusepath{stroke,fill}%
\end{pgfscope}%
\begin{pgfscope}%
\pgfpathrectangle{\pgfqpoint{0.511823in}{0.504323in}}{\pgfqpoint{3.218177in}{3.225677in}} %
\pgfusepath{clip}%
\pgfsetbuttcap%
\pgfsetroundjoin%
\definecolor{currentfill}{rgb}{0.501961,0.000000,0.000000}%
\pgfsetfillcolor{currentfill}%
\pgfsetfillopacity{0.400000}%
\pgfsetlinewidth{0.501875pt}%
\definecolor{currentstroke}{rgb}{0.501961,0.000000,0.000000}%
\pgfsetstrokecolor{currentstroke}%
\pgfsetstrokeopacity{0.400000}%
\pgfsetdash{}{0pt}%
\pgfpathmoveto{\pgfqpoint{0.938079in}{2.885592in}}%
\pgfpathcurveto{\pgfqpoint{0.949130in}{2.885592in}}{\pgfqpoint{0.959729in}{2.889982in}}{\pgfqpoint{0.967542in}{2.897796in}}%
\pgfpathcurveto{\pgfqpoint{0.975356in}{2.905610in}}{\pgfqpoint{0.979746in}{2.916209in}}{\pgfqpoint{0.979746in}{2.927259in}}%
\pgfpathcurveto{\pgfqpoint{0.979746in}{2.938309in}}{\pgfqpoint{0.975356in}{2.948908in}}{\pgfqpoint{0.967542in}{2.956722in}}%
\pgfpathcurveto{\pgfqpoint{0.959729in}{2.964535in}}{\pgfqpoint{0.949130in}{2.968925in}}{\pgfqpoint{0.938079in}{2.968925in}}%
\pgfpathcurveto{\pgfqpoint{0.927029in}{2.968925in}}{\pgfqpoint{0.916430in}{2.964535in}}{\pgfqpoint{0.908617in}{2.956722in}}%
\pgfpathcurveto{\pgfqpoint{0.900803in}{2.948908in}}{\pgfqpoint{0.896413in}{2.938309in}}{\pgfqpoint{0.896413in}{2.927259in}}%
\pgfpathcurveto{\pgfqpoint{0.896413in}{2.916209in}}{\pgfqpoint{0.900803in}{2.905610in}}{\pgfqpoint{0.908617in}{2.897796in}}%
\pgfpathcurveto{\pgfqpoint{0.916430in}{2.889982in}}{\pgfqpoint{0.927029in}{2.885592in}}{\pgfqpoint{0.938079in}{2.885592in}}%
\pgfpathclose%
\pgfusepath{stroke,fill}%
\end{pgfscope}%
\begin{pgfscope}%
\pgfpathrectangle{\pgfqpoint{0.511823in}{0.504323in}}{\pgfqpoint{3.218177in}{3.225677in}} %
\pgfusepath{clip}%
\pgfsetbuttcap%
\pgfsetroundjoin%
\definecolor{currentfill}{rgb}{0.501961,0.000000,0.000000}%
\pgfsetfillcolor{currentfill}%
\pgfsetfillopacity{0.400000}%
\pgfsetlinewidth{0.501875pt}%
\definecolor{currentstroke}{rgb}{0.501961,0.000000,0.000000}%
\pgfsetstrokecolor{currentstroke}%
\pgfsetstrokeopacity{0.400000}%
\pgfsetdash{}{0pt}%
\pgfpathmoveto{\pgfqpoint{0.955657in}{3.118087in}}%
\pgfpathcurveto{\pgfqpoint{0.966707in}{3.118087in}}{\pgfqpoint{0.977306in}{3.122478in}}{\pgfqpoint{0.985120in}{3.130291in}}%
\pgfpathcurveto{\pgfqpoint{0.992933in}{3.138105in}}{\pgfqpoint{0.997324in}{3.148704in}}{\pgfqpoint{0.997324in}{3.159754in}}%
\pgfpathcurveto{\pgfqpoint{0.997324in}{3.170804in}}{\pgfqpoint{0.992933in}{3.181403in}}{\pgfqpoint{0.985120in}{3.189217in}}%
\pgfpathcurveto{\pgfqpoint{0.977306in}{3.197031in}}{\pgfqpoint{0.966707in}{3.201421in}}{\pgfqpoint{0.955657in}{3.201421in}}%
\pgfpathcurveto{\pgfqpoint{0.944607in}{3.201421in}}{\pgfqpoint{0.934008in}{3.197031in}}{\pgfqpoint{0.926194in}{3.189217in}}%
\pgfpathcurveto{\pgfqpoint{0.918381in}{3.181403in}}{\pgfqpoint{0.913990in}{3.170804in}}{\pgfqpoint{0.913990in}{3.159754in}}%
\pgfpathcurveto{\pgfqpoint{0.913990in}{3.148704in}}{\pgfqpoint{0.918381in}{3.138105in}}{\pgfqpoint{0.926194in}{3.130291in}}%
\pgfpathcurveto{\pgfqpoint{0.934008in}{3.122478in}}{\pgfqpoint{0.944607in}{3.118087in}}{\pgfqpoint{0.955657in}{3.118087in}}%
\pgfpathclose%
\pgfusepath{stroke,fill}%
\end{pgfscope}%
\begin{pgfscope}%
\pgfpathrectangle{\pgfqpoint{0.511823in}{0.504323in}}{\pgfqpoint{3.218177in}{3.225677in}} %
\pgfusepath{clip}%
\pgfsetbuttcap%
\pgfsetroundjoin%
\definecolor{currentfill}{rgb}{0.501961,0.000000,0.000000}%
\pgfsetfillcolor{currentfill}%
\pgfsetfillopacity{0.400000}%
\pgfsetlinewidth{0.501875pt}%
\definecolor{currentstroke}{rgb}{0.501961,0.000000,0.000000}%
\pgfsetstrokecolor{currentstroke}%
\pgfsetstrokeopacity{0.400000}%
\pgfsetdash{}{0pt}%
\pgfpathmoveto{\pgfqpoint{0.991636in}{3.540533in}}%
\pgfpathcurveto{\pgfqpoint{1.002686in}{3.540533in}}{\pgfqpoint{1.013285in}{3.544924in}}{\pgfqpoint{1.021099in}{3.552737in}}%
\pgfpathcurveto{\pgfqpoint{1.028912in}{3.560551in}}{\pgfqpoint{1.033303in}{3.571150in}}{\pgfqpoint{1.033303in}{3.582200in}}%
\pgfpathcurveto{\pgfqpoint{1.033303in}{3.593250in}}{\pgfqpoint{1.028912in}{3.603849in}}{\pgfqpoint{1.021099in}{3.611663in}}%
\pgfpathcurveto{\pgfqpoint{1.013285in}{3.619476in}}{\pgfqpoint{1.002686in}{3.623867in}}{\pgfqpoint{0.991636in}{3.623867in}}%
\pgfpathcurveto{\pgfqpoint{0.980586in}{3.623867in}}{\pgfqpoint{0.969987in}{3.619476in}}{\pgfqpoint{0.962173in}{3.611663in}}%
\pgfpathcurveto{\pgfqpoint{0.954360in}{3.603849in}}{\pgfqpoint{0.949969in}{3.593250in}}{\pgfqpoint{0.949969in}{3.582200in}}%
\pgfpathcurveto{\pgfqpoint{0.949969in}{3.571150in}}{\pgfqpoint{0.954360in}{3.560551in}}{\pgfqpoint{0.962173in}{3.552737in}}%
\pgfpathcurveto{\pgfqpoint{0.969987in}{3.544924in}}{\pgfqpoint{0.980586in}{3.540533in}}{\pgfqpoint{0.991636in}{3.540533in}}%
\pgfpathclose%
\pgfusepath{stroke,fill}%
\end{pgfscope}%
\begin{pgfscope}%
\pgfpathrectangle{\pgfqpoint{0.511823in}{0.504323in}}{\pgfqpoint{3.218177in}{3.225677in}} %
\pgfusepath{clip}%
\pgfsetbuttcap%
\pgfsetroundjoin%
\definecolor{currentfill}{rgb}{0.501961,0.000000,0.000000}%
\pgfsetfillcolor{currentfill}%
\pgfsetfillopacity{0.400000}%
\pgfsetlinewidth{0.501875pt}%
\definecolor{currentstroke}{rgb}{0.501961,0.000000,0.000000}%
\pgfsetstrokecolor{currentstroke}%
\pgfsetstrokeopacity{0.400000}%
\pgfsetdash{}{0pt}%
\pgfpathmoveto{\pgfqpoint{0.940328in}{3.133814in}}%
\pgfpathcurveto{\pgfqpoint{0.951378in}{3.133814in}}{\pgfqpoint{0.961977in}{3.138204in}}{\pgfqpoint{0.969791in}{3.146018in}}%
\pgfpathcurveto{\pgfqpoint{0.977604in}{3.153831in}}{\pgfqpoint{0.981995in}{3.164430in}}{\pgfqpoint{0.981995in}{3.175480in}}%
\pgfpathcurveto{\pgfqpoint{0.981995in}{3.186531in}}{\pgfqpoint{0.977604in}{3.197130in}}{\pgfqpoint{0.969791in}{3.204943in}}%
\pgfpathcurveto{\pgfqpoint{0.961977in}{3.212757in}}{\pgfqpoint{0.951378in}{3.217147in}}{\pgfqpoint{0.940328in}{3.217147in}}%
\pgfpathcurveto{\pgfqpoint{0.929278in}{3.217147in}}{\pgfqpoint{0.918679in}{3.212757in}}{\pgfqpoint{0.910865in}{3.204943in}}%
\pgfpathcurveto{\pgfqpoint{0.903052in}{3.197130in}}{\pgfqpoint{0.898661in}{3.186531in}}{\pgfqpoint{0.898661in}{3.175480in}}%
\pgfpathcurveto{\pgfqpoint{0.898661in}{3.164430in}}{\pgfqpoint{0.903052in}{3.153831in}}{\pgfqpoint{0.910865in}{3.146018in}}%
\pgfpathcurveto{\pgfqpoint{0.918679in}{3.138204in}}{\pgfqpoint{0.929278in}{3.133814in}}{\pgfqpoint{0.940328in}{3.133814in}}%
\pgfpathclose%
\pgfusepath{stroke,fill}%
\end{pgfscope}%
\begin{pgfscope}%
\pgfpathrectangle{\pgfqpoint{0.511823in}{0.504323in}}{\pgfqpoint{3.218177in}{3.225677in}} %
\pgfusepath{clip}%
\pgfsetbuttcap%
\pgfsetroundjoin%
\definecolor{currentfill}{rgb}{0.501961,0.000000,0.000000}%
\pgfsetfillcolor{currentfill}%
\pgfsetfillopacity{0.400000}%
\pgfsetlinewidth{0.501875pt}%
\definecolor{currentstroke}{rgb}{0.501961,0.000000,0.000000}%
\pgfsetstrokecolor{currentstroke}%
\pgfsetstrokeopacity{0.400000}%
\pgfsetdash{}{0pt}%
\pgfpathmoveto{\pgfqpoint{0.927526in}{3.090203in}}%
\pgfpathcurveto{\pgfqpoint{0.938577in}{3.090203in}}{\pgfqpoint{0.949176in}{3.094593in}}{\pgfqpoint{0.956989in}{3.102407in}}%
\pgfpathcurveto{\pgfqpoint{0.964803in}{3.110220in}}{\pgfqpoint{0.969193in}{3.120819in}}{\pgfqpoint{0.969193in}{3.131870in}}%
\pgfpathcurveto{\pgfqpoint{0.969193in}{3.142920in}}{\pgfqpoint{0.964803in}{3.153519in}}{\pgfqpoint{0.956989in}{3.161332in}}%
\pgfpathcurveto{\pgfqpoint{0.949176in}{3.169146in}}{\pgfqpoint{0.938577in}{3.173536in}}{\pgfqpoint{0.927526in}{3.173536in}}%
\pgfpathcurveto{\pgfqpoint{0.916476in}{3.173536in}}{\pgfqpoint{0.905877in}{3.169146in}}{\pgfqpoint{0.898064in}{3.161332in}}%
\pgfpathcurveto{\pgfqpoint{0.890250in}{3.153519in}}{\pgfqpoint{0.885860in}{3.142920in}}{\pgfqpoint{0.885860in}{3.131870in}}%
\pgfpathcurveto{\pgfqpoint{0.885860in}{3.120819in}}{\pgfqpoint{0.890250in}{3.110220in}}{\pgfqpoint{0.898064in}{3.102407in}}%
\pgfpathcurveto{\pgfqpoint{0.905877in}{3.094593in}}{\pgfqpoint{0.916476in}{3.090203in}}{\pgfqpoint{0.927526in}{3.090203in}}%
\pgfpathclose%
\pgfusepath{stroke,fill}%
\end{pgfscope}%
\begin{pgfscope}%
\pgfpathrectangle{\pgfqpoint{0.511823in}{0.504323in}}{\pgfqpoint{3.218177in}{3.225677in}} %
\pgfusepath{clip}%
\pgfsetbuttcap%
\pgfsetroundjoin%
\definecolor{currentfill}{rgb}{0.501961,0.000000,0.000000}%
\pgfsetfillcolor{currentfill}%
\pgfsetfillopacity{0.400000}%
\pgfsetlinewidth{0.501875pt}%
\definecolor{currentstroke}{rgb}{0.501961,0.000000,0.000000}%
\pgfsetstrokecolor{currentstroke}%
\pgfsetstrokeopacity{0.400000}%
\pgfsetdash{}{0pt}%
\pgfpathmoveto{\pgfqpoint{0.925646in}{3.158396in}}%
\pgfpathcurveto{\pgfqpoint{0.936696in}{3.158396in}}{\pgfqpoint{0.947295in}{3.162786in}}{\pgfqpoint{0.955108in}{3.170600in}}%
\pgfpathcurveto{\pgfqpoint{0.962922in}{3.178414in}}{\pgfqpoint{0.967312in}{3.189013in}}{\pgfqpoint{0.967312in}{3.200063in}}%
\pgfpathcurveto{\pgfqpoint{0.967312in}{3.211113in}}{\pgfqpoint{0.962922in}{3.221712in}}{\pgfqpoint{0.955108in}{3.229526in}}%
\pgfpathcurveto{\pgfqpoint{0.947295in}{3.237339in}}{\pgfqpoint{0.936696in}{3.241729in}}{\pgfqpoint{0.925646in}{3.241729in}}%
\pgfpathcurveto{\pgfqpoint{0.914596in}{3.241729in}}{\pgfqpoint{0.903997in}{3.237339in}}{\pgfqpoint{0.896183in}{3.229526in}}%
\pgfpathcurveto{\pgfqpoint{0.888369in}{3.221712in}}{\pgfqpoint{0.883979in}{3.211113in}}{\pgfqpoint{0.883979in}{3.200063in}}%
\pgfpathcurveto{\pgfqpoint{0.883979in}{3.189013in}}{\pgfqpoint{0.888369in}{3.178414in}}{\pgfqpoint{0.896183in}{3.170600in}}%
\pgfpathcurveto{\pgfqpoint{0.903997in}{3.162786in}}{\pgfqpoint{0.914596in}{3.158396in}}{\pgfqpoint{0.925646in}{3.158396in}}%
\pgfpathclose%
\pgfusepath{stroke,fill}%
\end{pgfscope}%
\begin{pgfscope}%
\pgfpathrectangle{\pgfqpoint{0.511823in}{0.504323in}}{\pgfqpoint{3.218177in}{3.225677in}} %
\pgfusepath{clip}%
\pgfsetbuttcap%
\pgfsetroundjoin%
\definecolor{currentfill}{rgb}{0.501961,0.000000,0.000000}%
\pgfsetfillcolor{currentfill}%
\pgfsetfillopacity{0.400000}%
\pgfsetlinewidth{0.501875pt}%
\definecolor{currentstroke}{rgb}{0.501961,0.000000,0.000000}%
\pgfsetstrokecolor{currentstroke}%
\pgfsetstrokeopacity{0.400000}%
\pgfsetdash{}{0pt}%
\pgfpathmoveto{\pgfqpoint{0.911289in}{3.095361in}}%
\pgfpathcurveto{\pgfqpoint{0.922339in}{3.095361in}}{\pgfqpoint{0.932938in}{3.099752in}}{\pgfqpoint{0.940751in}{3.107565in}}%
\pgfpathcurveto{\pgfqpoint{0.948565in}{3.115379in}}{\pgfqpoint{0.952955in}{3.125978in}}{\pgfqpoint{0.952955in}{3.137028in}}%
\pgfpathcurveto{\pgfqpoint{0.952955in}{3.148078in}}{\pgfqpoint{0.948565in}{3.158677in}}{\pgfqpoint{0.940751in}{3.166491in}}%
\pgfpathcurveto{\pgfqpoint{0.932938in}{3.174304in}}{\pgfqpoint{0.922339in}{3.178695in}}{\pgfqpoint{0.911289in}{3.178695in}}%
\pgfpathcurveto{\pgfqpoint{0.900239in}{3.178695in}}{\pgfqpoint{0.889640in}{3.174304in}}{\pgfqpoint{0.881826in}{3.166491in}}%
\pgfpathcurveto{\pgfqpoint{0.874012in}{3.158677in}}{\pgfqpoint{0.869622in}{3.148078in}}{\pgfqpoint{0.869622in}{3.137028in}}%
\pgfpathcurveto{\pgfqpoint{0.869622in}{3.125978in}}{\pgfqpoint{0.874012in}{3.115379in}}{\pgfqpoint{0.881826in}{3.107565in}}%
\pgfpathcurveto{\pgfqpoint{0.889640in}{3.099752in}}{\pgfqpoint{0.900239in}{3.095361in}}{\pgfqpoint{0.911289in}{3.095361in}}%
\pgfpathclose%
\pgfusepath{stroke,fill}%
\end{pgfscope}%
\begin{pgfscope}%
\pgfpathrectangle{\pgfqpoint{0.511823in}{0.504323in}}{\pgfqpoint{3.218177in}{3.225677in}} %
\pgfusepath{clip}%
\pgfsetbuttcap%
\pgfsetroundjoin%
\definecolor{currentfill}{rgb}{0.501961,0.000000,0.000000}%
\pgfsetfillcolor{currentfill}%
\pgfsetfillopacity{0.400000}%
\pgfsetlinewidth{0.501875pt}%
\definecolor{currentstroke}{rgb}{0.501961,0.000000,0.000000}%
\pgfsetstrokecolor{currentstroke}%
\pgfsetstrokeopacity{0.400000}%
\pgfsetdash{}{0pt}%
\pgfpathmoveto{\pgfqpoint{0.905668in}{3.126505in}}%
\pgfpathcurveto{\pgfqpoint{0.916719in}{3.126505in}}{\pgfqpoint{0.927318in}{3.130895in}}{\pgfqpoint{0.935131in}{3.138709in}}%
\pgfpathcurveto{\pgfqpoint{0.942945in}{3.146522in}}{\pgfqpoint{0.947335in}{3.157121in}}{\pgfqpoint{0.947335in}{3.168172in}}%
\pgfpathcurveto{\pgfqpoint{0.947335in}{3.179222in}}{\pgfqpoint{0.942945in}{3.189821in}}{\pgfqpoint{0.935131in}{3.197634in}}%
\pgfpathcurveto{\pgfqpoint{0.927318in}{3.205448in}}{\pgfqpoint{0.916719in}{3.209838in}}{\pgfqpoint{0.905668in}{3.209838in}}%
\pgfpathcurveto{\pgfqpoint{0.894618in}{3.209838in}}{\pgfqpoint{0.884019in}{3.205448in}}{\pgfqpoint{0.876206in}{3.197634in}}%
\pgfpathcurveto{\pgfqpoint{0.868392in}{3.189821in}}{\pgfqpoint{0.864002in}{3.179222in}}{\pgfqpoint{0.864002in}{3.168172in}}%
\pgfpathcurveto{\pgfqpoint{0.864002in}{3.157121in}}{\pgfqpoint{0.868392in}{3.146522in}}{\pgfqpoint{0.876206in}{3.138709in}}%
\pgfpathcurveto{\pgfqpoint{0.884019in}{3.130895in}}{\pgfqpoint{0.894618in}{3.126505in}}{\pgfqpoint{0.905668in}{3.126505in}}%
\pgfpathclose%
\pgfusepath{stroke,fill}%
\end{pgfscope}%
\begin{pgfscope}%
\pgfpathrectangle{\pgfqpoint{0.511823in}{0.504323in}}{\pgfqpoint{3.218177in}{3.225677in}} %
\pgfusepath{clip}%
\pgfsetbuttcap%
\pgfsetroundjoin%
\definecolor{currentfill}{rgb}{0.501961,0.000000,0.000000}%
\pgfsetfillcolor{currentfill}%
\pgfsetfillopacity{0.400000}%
\pgfsetlinewidth{0.501875pt}%
\definecolor{currentstroke}{rgb}{0.501961,0.000000,0.000000}%
\pgfsetstrokecolor{currentstroke}%
\pgfsetstrokeopacity{0.400000}%
\pgfsetdash{}{0pt}%
\pgfpathmoveto{\pgfqpoint{0.897869in}{3.134389in}}%
\pgfpathcurveto{\pgfqpoint{0.908919in}{3.134389in}}{\pgfqpoint{0.919518in}{3.138780in}}{\pgfqpoint{0.927332in}{3.146593in}}%
\pgfpathcurveto{\pgfqpoint{0.935146in}{3.154407in}}{\pgfqpoint{0.939536in}{3.165006in}}{\pgfqpoint{0.939536in}{3.176056in}}%
\pgfpathcurveto{\pgfqpoint{0.939536in}{3.187106in}}{\pgfqpoint{0.935146in}{3.197705in}}{\pgfqpoint{0.927332in}{3.205519in}}%
\pgfpathcurveto{\pgfqpoint{0.919518in}{3.213332in}}{\pgfqpoint{0.908919in}{3.217723in}}{\pgfqpoint{0.897869in}{3.217723in}}%
\pgfpathcurveto{\pgfqpoint{0.886819in}{3.217723in}}{\pgfqpoint{0.876220in}{3.213332in}}{\pgfqpoint{0.868406in}{3.205519in}}%
\pgfpathcurveto{\pgfqpoint{0.860593in}{3.197705in}}{\pgfqpoint{0.856203in}{3.187106in}}{\pgfqpoint{0.856203in}{3.176056in}}%
\pgfpathcurveto{\pgfqpoint{0.856203in}{3.165006in}}{\pgfqpoint{0.860593in}{3.154407in}}{\pgfqpoint{0.868406in}{3.146593in}}%
\pgfpathcurveto{\pgfqpoint{0.876220in}{3.138780in}}{\pgfqpoint{0.886819in}{3.134389in}}{\pgfqpoint{0.897869in}{3.134389in}}%
\pgfpathclose%
\pgfusepath{stroke,fill}%
\end{pgfscope}%
\begin{pgfscope}%
\pgfpathrectangle{\pgfqpoint{0.511823in}{0.504323in}}{\pgfqpoint{3.218177in}{3.225677in}} %
\pgfusepath{clip}%
\pgfsetbuttcap%
\pgfsetroundjoin%
\definecolor{currentfill}{rgb}{0.501961,0.000000,0.000000}%
\pgfsetfillcolor{currentfill}%
\pgfsetfillopacity{0.400000}%
\pgfsetlinewidth{0.501875pt}%
\definecolor{currentstroke}{rgb}{0.501961,0.000000,0.000000}%
\pgfsetstrokecolor{currentstroke}%
\pgfsetstrokeopacity{0.400000}%
\pgfsetdash{}{0pt}%
\pgfpathmoveto{\pgfqpoint{0.878028in}{2.994857in}}%
\pgfpathcurveto{\pgfqpoint{0.889078in}{2.994857in}}{\pgfqpoint{0.899677in}{2.999247in}}{\pgfqpoint{0.907491in}{3.007061in}}%
\pgfpathcurveto{\pgfqpoint{0.915304in}{3.014874in}}{\pgfqpoint{0.919695in}{3.025473in}}{\pgfqpoint{0.919695in}{3.036523in}}%
\pgfpathcurveto{\pgfqpoint{0.919695in}{3.047573in}}{\pgfqpoint{0.915304in}{3.058173in}}{\pgfqpoint{0.907491in}{3.065986in}}%
\pgfpathcurveto{\pgfqpoint{0.899677in}{3.073800in}}{\pgfqpoint{0.889078in}{3.078190in}}{\pgfqpoint{0.878028in}{3.078190in}}%
\pgfpathcurveto{\pgfqpoint{0.866978in}{3.078190in}}{\pgfqpoint{0.856379in}{3.073800in}}{\pgfqpoint{0.848565in}{3.065986in}}%
\pgfpathcurveto{\pgfqpoint{0.840752in}{3.058173in}}{\pgfqpoint{0.836361in}{3.047573in}}{\pgfqpoint{0.836361in}{3.036523in}}%
\pgfpathcurveto{\pgfqpoint{0.836361in}{3.025473in}}{\pgfqpoint{0.840752in}{3.014874in}}{\pgfqpoint{0.848565in}{3.007061in}}%
\pgfpathcurveto{\pgfqpoint{0.856379in}{2.999247in}}{\pgfqpoint{0.866978in}{2.994857in}}{\pgfqpoint{0.878028in}{2.994857in}}%
\pgfpathclose%
\pgfusepath{stroke,fill}%
\end{pgfscope}%
\begin{pgfscope}%
\pgfpathrectangle{\pgfqpoint{0.511823in}{0.504323in}}{\pgfqpoint{3.218177in}{3.225677in}} %
\pgfusepath{clip}%
\pgfsetbuttcap%
\pgfsetroundjoin%
\definecolor{currentfill}{rgb}{0.501961,0.000000,0.000000}%
\pgfsetfillcolor{currentfill}%
\pgfsetfillopacity{0.400000}%
\pgfsetlinewidth{0.501875pt}%
\definecolor{currentstroke}{rgb}{0.501961,0.000000,0.000000}%
\pgfsetstrokecolor{currentstroke}%
\pgfsetstrokeopacity{0.400000}%
\pgfsetdash{}{0pt}%
\pgfpathmoveto{\pgfqpoint{0.891035in}{3.264506in}}%
\pgfpathcurveto{\pgfqpoint{0.902086in}{3.264506in}}{\pgfqpoint{0.912685in}{3.268897in}}{\pgfqpoint{0.920498in}{3.276710in}}%
\pgfpathcurveto{\pgfqpoint{0.928312in}{3.284524in}}{\pgfqpoint{0.932702in}{3.295123in}}{\pgfqpoint{0.932702in}{3.306173in}}%
\pgfpathcurveto{\pgfqpoint{0.932702in}{3.317223in}}{\pgfqpoint{0.928312in}{3.327822in}}{\pgfqpoint{0.920498in}{3.335636in}}%
\pgfpathcurveto{\pgfqpoint{0.912685in}{3.343449in}}{\pgfqpoint{0.902086in}{3.347840in}}{\pgfqpoint{0.891035in}{3.347840in}}%
\pgfpathcurveto{\pgfqpoint{0.879985in}{3.347840in}}{\pgfqpoint{0.869386in}{3.343449in}}{\pgfqpoint{0.861573in}{3.335636in}}%
\pgfpathcurveto{\pgfqpoint{0.853759in}{3.327822in}}{\pgfqpoint{0.849369in}{3.317223in}}{\pgfqpoint{0.849369in}{3.306173in}}%
\pgfpathcurveto{\pgfqpoint{0.849369in}{3.295123in}}{\pgfqpoint{0.853759in}{3.284524in}}{\pgfqpoint{0.861573in}{3.276710in}}%
\pgfpathcurveto{\pgfqpoint{0.869386in}{3.268897in}}{\pgfqpoint{0.879985in}{3.264506in}}{\pgfqpoint{0.891035in}{3.264506in}}%
\pgfpathclose%
\pgfusepath{stroke,fill}%
\end{pgfscope}%
\begin{pgfscope}%
\pgfpathrectangle{\pgfqpoint{0.511823in}{0.504323in}}{\pgfqpoint{3.218177in}{3.225677in}} %
\pgfusepath{clip}%
\pgfsetbuttcap%
\pgfsetroundjoin%
\definecolor{currentfill}{rgb}{0.501961,0.000000,0.000000}%
\pgfsetfillcolor{currentfill}%
\pgfsetfillopacity{0.400000}%
\pgfsetlinewidth{0.501875pt}%
\definecolor{currentstroke}{rgb}{0.501961,0.000000,0.000000}%
\pgfsetstrokecolor{currentstroke}%
\pgfsetstrokeopacity{0.400000}%
\pgfsetdash{}{0pt}%
\pgfpathmoveto{\pgfqpoint{0.873167in}{3.144535in}}%
\pgfpathcurveto{\pgfqpoint{0.884217in}{3.144535in}}{\pgfqpoint{0.894816in}{3.148925in}}{\pgfqpoint{0.902630in}{3.156739in}}%
\pgfpathcurveto{\pgfqpoint{0.910444in}{3.164553in}}{\pgfqpoint{0.914834in}{3.175152in}}{\pgfqpoint{0.914834in}{3.186202in}}%
\pgfpathcurveto{\pgfqpoint{0.914834in}{3.197252in}}{\pgfqpoint{0.910444in}{3.207851in}}{\pgfqpoint{0.902630in}{3.215665in}}%
\pgfpathcurveto{\pgfqpoint{0.894816in}{3.223478in}}{\pgfqpoint{0.884217in}{3.227868in}}{\pgfqpoint{0.873167in}{3.227868in}}%
\pgfpathcurveto{\pgfqpoint{0.862117in}{3.227868in}}{\pgfqpoint{0.851518in}{3.223478in}}{\pgfqpoint{0.843704in}{3.215665in}}%
\pgfpathcurveto{\pgfqpoint{0.835891in}{3.207851in}}{\pgfqpoint{0.831500in}{3.197252in}}{\pgfqpoint{0.831500in}{3.186202in}}%
\pgfpathcurveto{\pgfqpoint{0.831500in}{3.175152in}}{\pgfqpoint{0.835891in}{3.164553in}}{\pgfqpoint{0.843704in}{3.156739in}}%
\pgfpathcurveto{\pgfqpoint{0.851518in}{3.148925in}}{\pgfqpoint{0.862117in}{3.144535in}}{\pgfqpoint{0.873167in}{3.144535in}}%
\pgfpathclose%
\pgfusepath{stroke,fill}%
\end{pgfscope}%
\begin{pgfscope}%
\pgfpathrectangle{\pgfqpoint{0.511823in}{0.504323in}}{\pgfqpoint{3.218177in}{3.225677in}} %
\pgfusepath{clip}%
\pgfsetbuttcap%
\pgfsetroundjoin%
\definecolor{currentfill}{rgb}{0.501961,0.000000,0.000000}%
\pgfsetfillcolor{currentfill}%
\pgfsetfillopacity{0.400000}%
\pgfsetlinewidth{0.501875pt}%
\definecolor{currentstroke}{rgb}{0.501961,0.000000,0.000000}%
\pgfsetstrokecolor{currentstroke}%
\pgfsetstrokeopacity{0.400000}%
\pgfsetdash{}{0pt}%
\pgfpathmoveto{\pgfqpoint{0.872578in}{3.255901in}}%
\pgfpathcurveto{\pgfqpoint{0.883628in}{3.255901in}}{\pgfqpoint{0.894227in}{3.260291in}}{\pgfqpoint{0.902041in}{3.268105in}}%
\pgfpathcurveto{\pgfqpoint{0.909854in}{3.275919in}}{\pgfqpoint{0.914244in}{3.286518in}}{\pgfqpoint{0.914244in}{3.297568in}}%
\pgfpathcurveto{\pgfqpoint{0.914244in}{3.308618in}}{\pgfqpoint{0.909854in}{3.319217in}}{\pgfqpoint{0.902041in}{3.327031in}}%
\pgfpathcurveto{\pgfqpoint{0.894227in}{3.334844in}}{\pgfqpoint{0.883628in}{3.339235in}}{\pgfqpoint{0.872578in}{3.339235in}}%
\pgfpathcurveto{\pgfqpoint{0.861528in}{3.339235in}}{\pgfqpoint{0.850929in}{3.334844in}}{\pgfqpoint{0.843115in}{3.327031in}}%
\pgfpathcurveto{\pgfqpoint{0.835301in}{3.319217in}}{\pgfqpoint{0.830911in}{3.308618in}}{\pgfqpoint{0.830911in}{3.297568in}}%
\pgfpathcurveto{\pgfqpoint{0.830911in}{3.286518in}}{\pgfqpoint{0.835301in}{3.275919in}}{\pgfqpoint{0.843115in}{3.268105in}}%
\pgfpathcurveto{\pgfqpoint{0.850929in}{3.260291in}}{\pgfqpoint{0.861528in}{3.255901in}}{\pgfqpoint{0.872578in}{3.255901in}}%
\pgfpathclose%
\pgfusepath{stroke,fill}%
\end{pgfscope}%
\begin{pgfscope}%
\pgfpathrectangle{\pgfqpoint{0.511823in}{0.504323in}}{\pgfqpoint{3.218177in}{3.225677in}} %
\pgfusepath{clip}%
\pgfsetbuttcap%
\pgfsetroundjoin%
\definecolor{currentfill}{rgb}{0.501961,0.000000,0.000000}%
\pgfsetfillcolor{currentfill}%
\pgfsetfillopacity{0.400000}%
\pgfsetlinewidth{0.501875pt}%
\definecolor{currentstroke}{rgb}{0.501961,0.000000,0.000000}%
\pgfsetstrokecolor{currentstroke}%
\pgfsetstrokeopacity{0.400000}%
\pgfsetdash{}{0pt}%
\pgfpathmoveto{\pgfqpoint{0.855864in}{3.140394in}}%
\pgfpathcurveto{\pgfqpoint{0.866914in}{3.140394in}}{\pgfqpoint{0.877513in}{3.144784in}}{\pgfqpoint{0.885327in}{3.152598in}}%
\pgfpathcurveto{\pgfqpoint{0.893140in}{3.160411in}}{\pgfqpoint{0.897531in}{3.171010in}}{\pgfqpoint{0.897531in}{3.182060in}}%
\pgfpathcurveto{\pgfqpoint{0.897531in}{3.193111in}}{\pgfqpoint{0.893140in}{3.203710in}}{\pgfqpoint{0.885327in}{3.211523in}}%
\pgfpathcurveto{\pgfqpoint{0.877513in}{3.219337in}}{\pgfqpoint{0.866914in}{3.223727in}}{\pgfqpoint{0.855864in}{3.223727in}}%
\pgfpathcurveto{\pgfqpoint{0.844814in}{3.223727in}}{\pgfqpoint{0.834215in}{3.219337in}}{\pgfqpoint{0.826401in}{3.211523in}}%
\pgfpathcurveto{\pgfqpoint{0.818588in}{3.203710in}}{\pgfqpoint{0.814197in}{3.193111in}}{\pgfqpoint{0.814197in}{3.182060in}}%
\pgfpathcurveto{\pgfqpoint{0.814197in}{3.171010in}}{\pgfqpoint{0.818588in}{3.160411in}}{\pgfqpoint{0.826401in}{3.152598in}}%
\pgfpathcurveto{\pgfqpoint{0.834215in}{3.144784in}}{\pgfqpoint{0.844814in}{3.140394in}}{\pgfqpoint{0.855864in}{3.140394in}}%
\pgfpathclose%
\pgfusepath{stroke,fill}%
\end{pgfscope}%
\begin{pgfscope}%
\pgfpathrectangle{\pgfqpoint{0.511823in}{0.504323in}}{\pgfqpoint{3.218177in}{3.225677in}} %
\pgfusepath{clip}%
\pgfsetbuttcap%
\pgfsetroundjoin%
\definecolor{currentfill}{rgb}{0.501961,0.000000,0.000000}%
\pgfsetfillcolor{currentfill}%
\pgfsetfillopacity{0.400000}%
\pgfsetlinewidth{0.501875pt}%
\definecolor{currentstroke}{rgb}{0.501961,0.000000,0.000000}%
\pgfsetstrokecolor{currentstroke}%
\pgfsetstrokeopacity{0.400000}%
\pgfsetdash{}{0pt}%
\pgfpathmoveto{\pgfqpoint{0.854509in}{3.251286in}}%
\pgfpathcurveto{\pgfqpoint{0.865559in}{3.251286in}}{\pgfqpoint{0.876158in}{3.255676in}}{\pgfqpoint{0.883971in}{3.263490in}}%
\pgfpathcurveto{\pgfqpoint{0.891785in}{3.271303in}}{\pgfqpoint{0.896175in}{3.281902in}}{\pgfqpoint{0.896175in}{3.292952in}}%
\pgfpathcurveto{\pgfqpoint{0.896175in}{3.304002in}}{\pgfqpoint{0.891785in}{3.314601in}}{\pgfqpoint{0.883971in}{3.322415in}}%
\pgfpathcurveto{\pgfqpoint{0.876158in}{3.330229in}}{\pgfqpoint{0.865559in}{3.334619in}}{\pgfqpoint{0.854509in}{3.334619in}}%
\pgfpathcurveto{\pgfqpoint{0.843459in}{3.334619in}}{\pgfqpoint{0.832860in}{3.330229in}}{\pgfqpoint{0.825046in}{3.322415in}}%
\pgfpathcurveto{\pgfqpoint{0.817232in}{3.314601in}}{\pgfqpoint{0.812842in}{3.304002in}}{\pgfqpoint{0.812842in}{3.292952in}}%
\pgfpathcurveto{\pgfqpoint{0.812842in}{3.281902in}}{\pgfqpoint{0.817232in}{3.271303in}}{\pgfqpoint{0.825046in}{3.263490in}}%
\pgfpathcurveto{\pgfqpoint{0.832860in}{3.255676in}}{\pgfqpoint{0.843459in}{3.251286in}}{\pgfqpoint{0.854509in}{3.251286in}}%
\pgfpathclose%
\pgfusepath{stroke,fill}%
\end{pgfscope}%
\begin{pgfscope}%
\pgfpathrectangle{\pgfqpoint{0.511823in}{0.504323in}}{\pgfqpoint{3.218177in}{3.225677in}} %
\pgfusepath{clip}%
\pgfsetbuttcap%
\pgfsetroundjoin%
\definecolor{currentfill}{rgb}{0.501961,0.000000,0.000000}%
\pgfsetfillcolor{currentfill}%
\pgfsetfillopacity{0.400000}%
\pgfsetlinewidth{0.501875pt}%
\definecolor{currentstroke}{rgb}{0.501961,0.000000,0.000000}%
\pgfsetstrokecolor{currentstroke}%
\pgfsetstrokeopacity{0.400000}%
\pgfsetdash{}{0pt}%
\pgfpathmoveto{\pgfqpoint{0.838138in}{3.128172in}}%
\pgfpathcurveto{\pgfqpoint{0.849188in}{3.128172in}}{\pgfqpoint{0.859787in}{3.132562in}}{\pgfqpoint{0.867601in}{3.140376in}}%
\pgfpathcurveto{\pgfqpoint{0.875415in}{3.148190in}}{\pgfqpoint{0.879805in}{3.158789in}}{\pgfqpoint{0.879805in}{3.169839in}}%
\pgfpathcurveto{\pgfqpoint{0.879805in}{3.180889in}}{\pgfqpoint{0.875415in}{3.191488in}}{\pgfqpoint{0.867601in}{3.199302in}}%
\pgfpathcurveto{\pgfqpoint{0.859787in}{3.207115in}}{\pgfqpoint{0.849188in}{3.211505in}}{\pgfqpoint{0.838138in}{3.211505in}}%
\pgfpathcurveto{\pgfqpoint{0.827088in}{3.211505in}}{\pgfqpoint{0.816489in}{3.207115in}}{\pgfqpoint{0.808675in}{3.199302in}}%
\pgfpathcurveto{\pgfqpoint{0.800862in}{3.191488in}}{\pgfqpoint{0.796472in}{3.180889in}}{\pgfqpoint{0.796472in}{3.169839in}}%
\pgfpathcurveto{\pgfqpoint{0.796472in}{3.158789in}}{\pgfqpoint{0.800862in}{3.148190in}}{\pgfqpoint{0.808675in}{3.140376in}}%
\pgfpathcurveto{\pgfqpoint{0.816489in}{3.132562in}}{\pgfqpoint{0.827088in}{3.128172in}}{\pgfqpoint{0.838138in}{3.128172in}}%
\pgfpathclose%
\pgfusepath{stroke,fill}%
\end{pgfscope}%
\begin{pgfscope}%
\pgfpathrectangle{\pgfqpoint{0.511823in}{0.504323in}}{\pgfqpoint{3.218177in}{3.225677in}} %
\pgfusepath{clip}%
\pgfsetbuttcap%
\pgfsetroundjoin%
\definecolor{currentfill}{rgb}{0.501961,0.000000,0.000000}%
\pgfsetfillcolor{currentfill}%
\pgfsetfillopacity{0.400000}%
\pgfsetlinewidth{0.501875pt}%
\definecolor{currentstroke}{rgb}{0.501961,0.000000,0.000000}%
\pgfsetstrokecolor{currentstroke}%
\pgfsetstrokeopacity{0.400000}%
\pgfsetdash{}{0pt}%
\pgfpathmoveto{\pgfqpoint{0.850863in}{3.495891in}}%
\pgfpathcurveto{\pgfqpoint{0.861913in}{3.495891in}}{\pgfqpoint{0.872512in}{3.500281in}}{\pgfqpoint{0.880326in}{3.508095in}}%
\pgfpathcurveto{\pgfqpoint{0.888139in}{3.515908in}}{\pgfqpoint{0.892530in}{3.526507in}}{\pgfqpoint{0.892530in}{3.537558in}}%
\pgfpathcurveto{\pgfqpoint{0.892530in}{3.548608in}}{\pgfqpoint{0.888139in}{3.559207in}}{\pgfqpoint{0.880326in}{3.567020in}}%
\pgfpathcurveto{\pgfqpoint{0.872512in}{3.574834in}}{\pgfqpoint{0.861913in}{3.579224in}}{\pgfqpoint{0.850863in}{3.579224in}}%
\pgfpathcurveto{\pgfqpoint{0.839813in}{3.579224in}}{\pgfqpoint{0.829214in}{3.574834in}}{\pgfqpoint{0.821400in}{3.567020in}}%
\pgfpathcurveto{\pgfqpoint{0.813587in}{3.559207in}}{\pgfqpoint{0.809196in}{3.548608in}}{\pgfqpoint{0.809196in}{3.537558in}}%
\pgfpathcurveto{\pgfqpoint{0.809196in}{3.526507in}}{\pgfqpoint{0.813587in}{3.515908in}}{\pgfqpoint{0.821400in}{3.508095in}}%
\pgfpathcurveto{\pgfqpoint{0.829214in}{3.500281in}}{\pgfqpoint{0.839813in}{3.495891in}}{\pgfqpoint{0.850863in}{3.495891in}}%
\pgfpathclose%
\pgfusepath{stroke,fill}%
\end{pgfscope}%
\begin{pgfscope}%
\pgfpathrectangle{\pgfqpoint{0.511823in}{0.504323in}}{\pgfqpoint{3.218177in}{3.225677in}} %
\pgfusepath{clip}%
\pgfsetbuttcap%
\pgfsetroundjoin%
\definecolor{currentfill}{rgb}{0.501961,0.000000,0.000000}%
\pgfsetfillcolor{currentfill}%
\pgfsetfillopacity{0.400000}%
\pgfsetlinewidth{0.501875pt}%
\definecolor{currentstroke}{rgb}{0.501961,0.000000,0.000000}%
\pgfsetstrokecolor{currentstroke}%
\pgfsetstrokeopacity{0.400000}%
\pgfsetdash{}{0pt}%
\pgfpathmoveto{\pgfqpoint{0.828630in}{3.264351in}}%
\pgfpathcurveto{\pgfqpoint{0.839680in}{3.264351in}}{\pgfqpoint{0.850279in}{3.268742in}}{\pgfqpoint{0.858093in}{3.276555in}}%
\pgfpathcurveto{\pgfqpoint{0.865907in}{3.284369in}}{\pgfqpoint{0.870297in}{3.294968in}}{\pgfqpoint{0.870297in}{3.306018in}}%
\pgfpathcurveto{\pgfqpoint{0.870297in}{3.317068in}}{\pgfqpoint{0.865907in}{3.327667in}}{\pgfqpoint{0.858093in}{3.335481in}}%
\pgfpathcurveto{\pgfqpoint{0.850279in}{3.343294in}}{\pgfqpoint{0.839680in}{3.347685in}}{\pgfqpoint{0.828630in}{3.347685in}}%
\pgfpathcurveto{\pgfqpoint{0.817580in}{3.347685in}}{\pgfqpoint{0.806981in}{3.343294in}}{\pgfqpoint{0.799167in}{3.335481in}}%
\pgfpathcurveto{\pgfqpoint{0.791354in}{3.327667in}}{\pgfqpoint{0.786963in}{3.317068in}}{\pgfqpoint{0.786963in}{3.306018in}}%
\pgfpathcurveto{\pgfqpoint{0.786963in}{3.294968in}}{\pgfqpoint{0.791354in}{3.284369in}}{\pgfqpoint{0.799167in}{3.276555in}}%
\pgfpathcurveto{\pgfqpoint{0.806981in}{3.268742in}}{\pgfqpoint{0.817580in}{3.264351in}}{\pgfqpoint{0.828630in}{3.264351in}}%
\pgfpathclose%
\pgfusepath{stroke,fill}%
\end{pgfscope}%
\begin{pgfscope}%
\pgfpathrectangle{\pgfqpoint{0.511823in}{0.504323in}}{\pgfqpoint{3.218177in}{3.225677in}} %
\pgfusepath{clip}%
\pgfsetbuttcap%
\pgfsetroundjoin%
\definecolor{currentfill}{rgb}{0.501961,0.000000,0.000000}%
\pgfsetfillcolor{currentfill}%
\pgfsetfillopacity{0.400000}%
\pgfsetlinewidth{0.501875pt}%
\definecolor{currentstroke}{rgb}{0.501961,0.000000,0.000000}%
\pgfsetstrokecolor{currentstroke}%
\pgfsetstrokeopacity{0.400000}%
\pgfsetdash{}{0pt}%
\pgfpathmoveto{\pgfqpoint{0.820871in}{3.286831in}}%
\pgfpathcurveto{\pgfqpoint{0.831921in}{3.286831in}}{\pgfqpoint{0.842520in}{3.291222in}}{\pgfqpoint{0.850334in}{3.299035in}}%
\pgfpathcurveto{\pgfqpoint{0.858148in}{3.306849in}}{\pgfqpoint{0.862538in}{3.317448in}}{\pgfqpoint{0.862538in}{3.328498in}}%
\pgfpathcurveto{\pgfqpoint{0.862538in}{3.339548in}}{\pgfqpoint{0.858148in}{3.350147in}}{\pgfqpoint{0.850334in}{3.357961in}}%
\pgfpathcurveto{\pgfqpoint{0.842520in}{3.365774in}}{\pgfqpoint{0.831921in}{3.370165in}}{\pgfqpoint{0.820871in}{3.370165in}}%
\pgfpathcurveto{\pgfqpoint{0.809821in}{3.370165in}}{\pgfqpoint{0.799222in}{3.365774in}}{\pgfqpoint{0.791408in}{3.357961in}}%
\pgfpathcurveto{\pgfqpoint{0.783595in}{3.350147in}}{\pgfqpoint{0.779204in}{3.339548in}}{\pgfqpoint{0.779204in}{3.328498in}}%
\pgfpathcurveto{\pgfqpoint{0.779204in}{3.317448in}}{\pgfqpoint{0.783595in}{3.306849in}}{\pgfqpoint{0.791408in}{3.299035in}}%
\pgfpathcurveto{\pgfqpoint{0.799222in}{3.291222in}}{\pgfqpoint{0.809821in}{3.286831in}}{\pgfqpoint{0.820871in}{3.286831in}}%
\pgfpathclose%
\pgfusepath{stroke,fill}%
\end{pgfscope}%
\begin{pgfscope}%
\pgfpathrectangle{\pgfqpoint{0.511823in}{0.504323in}}{\pgfqpoint{3.218177in}{3.225677in}} %
\pgfusepath{clip}%
\pgfsetbuttcap%
\pgfsetroundjoin%
\definecolor{currentfill}{rgb}{0.501961,0.000000,0.000000}%
\pgfsetfillcolor{currentfill}%
\pgfsetfillopacity{0.400000}%
\pgfsetlinewidth{0.501875pt}%
\definecolor{currentstroke}{rgb}{0.501961,0.000000,0.000000}%
\pgfsetstrokecolor{currentstroke}%
\pgfsetstrokeopacity{0.400000}%
\pgfsetdash{}{0pt}%
\pgfpathmoveto{\pgfqpoint{0.803871in}{3.117491in}}%
\pgfpathcurveto{\pgfqpoint{0.814921in}{3.117491in}}{\pgfqpoint{0.825520in}{3.121881in}}{\pgfqpoint{0.833334in}{3.129694in}}%
\pgfpathcurveto{\pgfqpoint{0.841147in}{3.137508in}}{\pgfqpoint{0.845538in}{3.148107in}}{\pgfqpoint{0.845538in}{3.159157in}}%
\pgfpathcurveto{\pgfqpoint{0.845538in}{3.170207in}}{\pgfqpoint{0.841147in}{3.180806in}}{\pgfqpoint{0.833334in}{3.188620in}}%
\pgfpathcurveto{\pgfqpoint{0.825520in}{3.196434in}}{\pgfqpoint{0.814921in}{3.200824in}}{\pgfqpoint{0.803871in}{3.200824in}}%
\pgfpathcurveto{\pgfqpoint{0.792821in}{3.200824in}}{\pgfqpoint{0.782222in}{3.196434in}}{\pgfqpoint{0.774408in}{3.188620in}}%
\pgfpathcurveto{\pgfqpoint{0.766595in}{3.180806in}}{\pgfqpoint{0.762204in}{3.170207in}}{\pgfqpoint{0.762204in}{3.159157in}}%
\pgfpathcurveto{\pgfqpoint{0.762204in}{3.148107in}}{\pgfqpoint{0.766595in}{3.137508in}}{\pgfqpoint{0.774408in}{3.129694in}}%
\pgfpathcurveto{\pgfqpoint{0.782222in}{3.121881in}}{\pgfqpoint{0.792821in}{3.117491in}}{\pgfqpoint{0.803871in}{3.117491in}}%
\pgfpathclose%
\pgfusepath{stroke,fill}%
\end{pgfscope}%
\begin{pgfscope}%
\pgfpathrectangle{\pgfqpoint{0.511823in}{0.504323in}}{\pgfqpoint{3.218177in}{3.225677in}} %
\pgfusepath{clip}%
\pgfsetbuttcap%
\pgfsetroundjoin%
\definecolor{currentfill}{rgb}{0.501961,0.000000,0.000000}%
\pgfsetfillcolor{currentfill}%
\pgfsetfillopacity{0.400000}%
\pgfsetlinewidth{0.501875pt}%
\definecolor{currentstroke}{rgb}{0.501961,0.000000,0.000000}%
\pgfsetstrokecolor{currentstroke}%
\pgfsetstrokeopacity{0.400000}%
\pgfsetdash{}{0pt}%
\pgfpathmoveto{\pgfqpoint{0.787517in}{2.936687in}}%
\pgfpathcurveto{\pgfqpoint{0.798567in}{2.936687in}}{\pgfqpoint{0.809166in}{2.941077in}}{\pgfqpoint{0.816980in}{2.948891in}}%
\pgfpathcurveto{\pgfqpoint{0.824793in}{2.956705in}}{\pgfqpoint{0.829184in}{2.967304in}}{\pgfqpoint{0.829184in}{2.978354in}}%
\pgfpathcurveto{\pgfqpoint{0.829184in}{2.989404in}}{\pgfqpoint{0.824793in}{3.000003in}}{\pgfqpoint{0.816980in}{3.007816in}}%
\pgfpathcurveto{\pgfqpoint{0.809166in}{3.015630in}}{\pgfqpoint{0.798567in}{3.020020in}}{\pgfqpoint{0.787517in}{3.020020in}}%
\pgfpathcurveto{\pgfqpoint{0.776467in}{3.020020in}}{\pgfqpoint{0.765868in}{3.015630in}}{\pgfqpoint{0.758054in}{3.007816in}}%
\pgfpathcurveto{\pgfqpoint{0.750241in}{3.000003in}}{\pgfqpoint{0.745850in}{2.989404in}}{\pgfqpoint{0.745850in}{2.978354in}}%
\pgfpathcurveto{\pgfqpoint{0.745850in}{2.967304in}}{\pgfqpoint{0.750241in}{2.956705in}}{\pgfqpoint{0.758054in}{2.948891in}}%
\pgfpathcurveto{\pgfqpoint{0.765868in}{2.941077in}}{\pgfqpoint{0.776467in}{2.936687in}}{\pgfqpoint{0.787517in}{2.936687in}}%
\pgfpathclose%
\pgfusepath{stroke,fill}%
\end{pgfscope}%
\begin{pgfscope}%
\pgfpathrectangle{\pgfqpoint{0.511823in}{0.504323in}}{\pgfqpoint{3.218177in}{3.225677in}} %
\pgfusepath{clip}%
\pgfsetbuttcap%
\pgfsetroundjoin%
\definecolor{currentfill}{rgb}{0.501961,0.000000,0.000000}%
\pgfsetfillcolor{currentfill}%
\pgfsetfillopacity{0.400000}%
\pgfsetlinewidth{0.501875pt}%
\definecolor{currentstroke}{rgb}{0.501961,0.000000,0.000000}%
\pgfsetstrokecolor{currentstroke}%
\pgfsetstrokeopacity{0.400000}%
\pgfsetdash{}{0pt}%
\pgfpathmoveto{\pgfqpoint{0.785853in}{3.087327in}}%
\pgfpathcurveto{\pgfqpoint{0.796903in}{3.087327in}}{\pgfqpoint{0.807502in}{3.091717in}}{\pgfqpoint{0.815315in}{3.099531in}}%
\pgfpathcurveto{\pgfqpoint{0.823129in}{3.107344in}}{\pgfqpoint{0.827519in}{3.117943in}}{\pgfqpoint{0.827519in}{3.128993in}}%
\pgfpathcurveto{\pgfqpoint{0.827519in}{3.140043in}}{\pgfqpoint{0.823129in}{3.150642in}}{\pgfqpoint{0.815315in}{3.158456in}}%
\pgfpathcurveto{\pgfqpoint{0.807502in}{3.166270in}}{\pgfqpoint{0.796903in}{3.170660in}}{\pgfqpoint{0.785853in}{3.170660in}}%
\pgfpathcurveto{\pgfqpoint{0.774802in}{3.170660in}}{\pgfqpoint{0.764203in}{3.166270in}}{\pgfqpoint{0.756390in}{3.158456in}}%
\pgfpathcurveto{\pgfqpoint{0.748576in}{3.150642in}}{\pgfqpoint{0.744186in}{3.140043in}}{\pgfqpoint{0.744186in}{3.128993in}}%
\pgfpathcurveto{\pgfqpoint{0.744186in}{3.117943in}}{\pgfqpoint{0.748576in}{3.107344in}}{\pgfqpoint{0.756390in}{3.099531in}}%
\pgfpathcurveto{\pgfqpoint{0.764203in}{3.091717in}}{\pgfqpoint{0.774802in}{3.087327in}}{\pgfqpoint{0.785853in}{3.087327in}}%
\pgfpathclose%
\pgfusepath{stroke,fill}%
\end{pgfscope}%
\begin{pgfscope}%
\pgfpathrectangle{\pgfqpoint{0.511823in}{0.504323in}}{\pgfqpoint{3.218177in}{3.225677in}} %
\pgfusepath{clip}%
\pgfsetbuttcap%
\pgfsetroundjoin%
\definecolor{currentfill}{rgb}{0.501961,0.000000,0.000000}%
\pgfsetfillcolor{currentfill}%
\pgfsetfillopacity{0.400000}%
\pgfsetlinewidth{0.501875pt}%
\definecolor{currentstroke}{rgb}{0.501961,0.000000,0.000000}%
\pgfsetstrokecolor{currentstroke}%
\pgfsetstrokeopacity{0.400000}%
\pgfsetdash{}{0pt}%
\pgfpathmoveto{\pgfqpoint{0.781375in}{3.189777in}}%
\pgfpathcurveto{\pgfqpoint{0.792426in}{3.189777in}}{\pgfqpoint{0.803025in}{3.194167in}}{\pgfqpoint{0.810838in}{3.201981in}}%
\pgfpathcurveto{\pgfqpoint{0.818652in}{3.209794in}}{\pgfqpoint{0.823042in}{3.220393in}}{\pgfqpoint{0.823042in}{3.231444in}}%
\pgfpathcurveto{\pgfqpoint{0.823042in}{3.242494in}}{\pgfqpoint{0.818652in}{3.253093in}}{\pgfqpoint{0.810838in}{3.260906in}}%
\pgfpathcurveto{\pgfqpoint{0.803025in}{3.268720in}}{\pgfqpoint{0.792426in}{3.273110in}}{\pgfqpoint{0.781375in}{3.273110in}}%
\pgfpathcurveto{\pgfqpoint{0.770325in}{3.273110in}}{\pgfqpoint{0.759726in}{3.268720in}}{\pgfqpoint{0.751913in}{3.260906in}}%
\pgfpathcurveto{\pgfqpoint{0.744099in}{3.253093in}}{\pgfqpoint{0.739709in}{3.242494in}}{\pgfqpoint{0.739709in}{3.231444in}}%
\pgfpathcurveto{\pgfqpoint{0.739709in}{3.220393in}}{\pgfqpoint{0.744099in}{3.209794in}}{\pgfqpoint{0.751913in}{3.201981in}}%
\pgfpathcurveto{\pgfqpoint{0.759726in}{3.194167in}}{\pgfqpoint{0.770325in}{3.189777in}}{\pgfqpoint{0.781375in}{3.189777in}}%
\pgfpathclose%
\pgfusepath{stroke,fill}%
\end{pgfscope}%
\begin{pgfscope}%
\pgfpathrectangle{\pgfqpoint{0.511823in}{0.504323in}}{\pgfqpoint{3.218177in}{3.225677in}} %
\pgfusepath{clip}%
\pgfsetbuttcap%
\pgfsetroundjoin%
\definecolor{currentfill}{rgb}{0.501961,0.000000,0.000000}%
\pgfsetfillcolor{currentfill}%
\pgfsetfillopacity{0.400000}%
\pgfsetlinewidth{0.501875pt}%
\definecolor{currentstroke}{rgb}{0.501961,0.000000,0.000000}%
\pgfsetstrokecolor{currentstroke}%
\pgfsetstrokeopacity{0.400000}%
\pgfsetdash{}{0pt}%
\pgfpathmoveto{\pgfqpoint{0.771183in}{3.143803in}}%
\pgfpathcurveto{\pgfqpoint{0.782233in}{3.143803in}}{\pgfqpoint{0.792832in}{3.148194in}}{\pgfqpoint{0.800646in}{3.156007in}}%
\pgfpathcurveto{\pgfqpoint{0.808459in}{3.163821in}}{\pgfqpoint{0.812850in}{3.174420in}}{\pgfqpoint{0.812850in}{3.185470in}}%
\pgfpathcurveto{\pgfqpoint{0.812850in}{3.196520in}}{\pgfqpoint{0.808459in}{3.207119in}}{\pgfqpoint{0.800646in}{3.214933in}}%
\pgfpathcurveto{\pgfqpoint{0.792832in}{3.222746in}}{\pgfqpoint{0.782233in}{3.227137in}}{\pgfqpoint{0.771183in}{3.227137in}}%
\pgfpathcurveto{\pgfqpoint{0.760133in}{3.227137in}}{\pgfqpoint{0.749534in}{3.222746in}}{\pgfqpoint{0.741720in}{3.214933in}}%
\pgfpathcurveto{\pgfqpoint{0.733907in}{3.207119in}}{\pgfqpoint{0.729516in}{3.196520in}}{\pgfqpoint{0.729516in}{3.185470in}}%
\pgfpathcurveto{\pgfqpoint{0.729516in}{3.174420in}}{\pgfqpoint{0.733907in}{3.163821in}}{\pgfqpoint{0.741720in}{3.156007in}}%
\pgfpathcurveto{\pgfqpoint{0.749534in}{3.148194in}}{\pgfqpoint{0.760133in}{3.143803in}}{\pgfqpoint{0.771183in}{3.143803in}}%
\pgfpathclose%
\pgfusepath{stroke,fill}%
\end{pgfscope}%
\begin{pgfscope}%
\pgfpathrectangle{\pgfqpoint{0.511823in}{0.504323in}}{\pgfqpoint{3.218177in}{3.225677in}} %
\pgfusepath{clip}%
\pgfsetbuttcap%
\pgfsetroundjoin%
\definecolor{currentfill}{rgb}{0.501961,0.000000,0.000000}%
\pgfsetfillcolor{currentfill}%
\pgfsetfillopacity{0.400000}%
\pgfsetlinewidth{0.501875pt}%
\definecolor{currentstroke}{rgb}{0.501961,0.000000,0.000000}%
\pgfsetstrokecolor{currentstroke}%
\pgfsetstrokeopacity{0.400000}%
\pgfsetdash{}{0pt}%
\pgfpathmoveto{\pgfqpoint{0.760074in}{3.057462in}}%
\pgfpathcurveto{\pgfqpoint{0.771124in}{3.057462in}}{\pgfqpoint{0.781723in}{3.061853in}}{\pgfqpoint{0.789537in}{3.069666in}}%
\pgfpathcurveto{\pgfqpoint{0.797351in}{3.077480in}}{\pgfqpoint{0.801741in}{3.088079in}}{\pgfqpoint{0.801741in}{3.099129in}}%
\pgfpathcurveto{\pgfqpoint{0.801741in}{3.110179in}}{\pgfqpoint{0.797351in}{3.120778in}}{\pgfqpoint{0.789537in}{3.128592in}}%
\pgfpathcurveto{\pgfqpoint{0.781723in}{3.136405in}}{\pgfqpoint{0.771124in}{3.140796in}}{\pgfqpoint{0.760074in}{3.140796in}}%
\pgfpathcurveto{\pgfqpoint{0.749024in}{3.140796in}}{\pgfqpoint{0.738425in}{3.136405in}}{\pgfqpoint{0.730611in}{3.128592in}}%
\pgfpathcurveto{\pgfqpoint{0.722798in}{3.120778in}}{\pgfqpoint{0.718407in}{3.110179in}}{\pgfqpoint{0.718407in}{3.099129in}}%
\pgfpathcurveto{\pgfqpoint{0.718407in}{3.088079in}}{\pgfqpoint{0.722798in}{3.077480in}}{\pgfqpoint{0.730611in}{3.069666in}}%
\pgfpathcurveto{\pgfqpoint{0.738425in}{3.061853in}}{\pgfqpoint{0.749024in}{3.057462in}}{\pgfqpoint{0.760074in}{3.057462in}}%
\pgfpathclose%
\pgfusepath{stroke,fill}%
\end{pgfscope}%
\begin{pgfscope}%
\pgfpathrectangle{\pgfqpoint{0.511823in}{0.504323in}}{\pgfqpoint{3.218177in}{3.225677in}} %
\pgfusepath{clip}%
\pgfsetbuttcap%
\pgfsetroundjoin%
\definecolor{currentfill}{rgb}{0.501961,0.000000,0.000000}%
\pgfsetfillcolor{currentfill}%
\pgfsetfillopacity{0.400000}%
\pgfsetlinewidth{0.501875pt}%
\definecolor{currentstroke}{rgb}{0.501961,0.000000,0.000000}%
\pgfsetstrokecolor{currentstroke}%
\pgfsetstrokeopacity{0.400000}%
\pgfsetdash{}{0pt}%
\pgfpathmoveto{\pgfqpoint{0.751292in}{3.035404in}}%
\pgfpathcurveto{\pgfqpoint{0.762342in}{3.035404in}}{\pgfqpoint{0.772941in}{3.039794in}}{\pgfqpoint{0.780754in}{3.047607in}}%
\pgfpathcurveto{\pgfqpoint{0.788568in}{3.055421in}}{\pgfqpoint{0.792958in}{3.066020in}}{\pgfqpoint{0.792958in}{3.077070in}}%
\pgfpathcurveto{\pgfqpoint{0.792958in}{3.088120in}}{\pgfqpoint{0.788568in}{3.098719in}}{\pgfqpoint{0.780754in}{3.106533in}}%
\pgfpathcurveto{\pgfqpoint{0.772941in}{3.114347in}}{\pgfqpoint{0.762342in}{3.118737in}}{\pgfqpoint{0.751292in}{3.118737in}}%
\pgfpathcurveto{\pgfqpoint{0.740242in}{3.118737in}}{\pgfqpoint{0.729643in}{3.114347in}}{\pgfqpoint{0.721829in}{3.106533in}}%
\pgfpathcurveto{\pgfqpoint{0.714015in}{3.098719in}}{\pgfqpoint{0.709625in}{3.088120in}}{\pgfqpoint{0.709625in}{3.077070in}}%
\pgfpathcurveto{\pgfqpoint{0.709625in}{3.066020in}}{\pgfqpoint{0.714015in}{3.055421in}}{\pgfqpoint{0.721829in}{3.047607in}}%
\pgfpathcurveto{\pgfqpoint{0.729643in}{3.039794in}}{\pgfqpoint{0.740242in}{3.035404in}}{\pgfqpoint{0.751292in}{3.035404in}}%
\pgfpathclose%
\pgfusepath{stroke,fill}%
\end{pgfscope}%
\begin{pgfscope}%
\pgfpathrectangle{\pgfqpoint{0.511823in}{0.504323in}}{\pgfqpoint{3.218177in}{3.225677in}} %
\pgfusepath{clip}%
\pgfsetbuttcap%
\pgfsetroundjoin%
\definecolor{currentfill}{rgb}{0.501961,0.000000,0.000000}%
\pgfsetfillcolor{currentfill}%
\pgfsetfillopacity{0.400000}%
\pgfsetlinewidth{0.501875pt}%
\definecolor{currentstroke}{rgb}{0.501961,0.000000,0.000000}%
\pgfsetstrokecolor{currentstroke}%
\pgfsetstrokeopacity{0.400000}%
\pgfsetdash{}{0pt}%
\pgfpathmoveto{\pgfqpoint{0.741776in}{2.976047in}}%
\pgfpathcurveto{\pgfqpoint{0.752827in}{2.976047in}}{\pgfqpoint{0.763426in}{2.980437in}}{\pgfqpoint{0.771239in}{2.988251in}}%
\pgfpathcurveto{\pgfqpoint{0.779053in}{2.996065in}}{\pgfqpoint{0.783443in}{3.006664in}}{\pgfqpoint{0.783443in}{3.017714in}}%
\pgfpathcurveto{\pgfqpoint{0.783443in}{3.028764in}}{\pgfqpoint{0.779053in}{3.039363in}}{\pgfqpoint{0.771239in}{3.047177in}}%
\pgfpathcurveto{\pgfqpoint{0.763426in}{3.054990in}}{\pgfqpoint{0.752827in}{3.059381in}}{\pgfqpoint{0.741776in}{3.059381in}}%
\pgfpathcurveto{\pgfqpoint{0.730726in}{3.059381in}}{\pgfqpoint{0.720127in}{3.054990in}}{\pgfqpoint{0.712314in}{3.047177in}}%
\pgfpathcurveto{\pgfqpoint{0.704500in}{3.039363in}}{\pgfqpoint{0.700110in}{3.028764in}}{\pgfqpoint{0.700110in}{3.017714in}}%
\pgfpathcurveto{\pgfqpoint{0.700110in}{3.006664in}}{\pgfqpoint{0.704500in}{2.996065in}}{\pgfqpoint{0.712314in}{2.988251in}}%
\pgfpathcurveto{\pgfqpoint{0.720127in}{2.980437in}}{\pgfqpoint{0.730726in}{2.976047in}}{\pgfqpoint{0.741776in}{2.976047in}}%
\pgfpathclose%
\pgfusepath{stroke,fill}%
\end{pgfscope}%
\begin{pgfscope}%
\pgfpathrectangle{\pgfqpoint{0.511823in}{0.504323in}}{\pgfqpoint{3.218177in}{3.225677in}} %
\pgfusepath{clip}%
\pgfsetbuttcap%
\pgfsetroundjoin%
\definecolor{currentfill}{rgb}{0.501961,0.000000,0.000000}%
\pgfsetfillcolor{currentfill}%
\pgfsetfillopacity{0.400000}%
\pgfsetlinewidth{0.501875pt}%
\definecolor{currentstroke}{rgb}{0.501961,0.000000,0.000000}%
\pgfsetstrokecolor{currentstroke}%
\pgfsetstrokeopacity{0.400000}%
\pgfsetdash{}{0pt}%
\pgfpathmoveto{\pgfqpoint{0.735790in}{3.070661in}}%
\pgfpathcurveto{\pgfqpoint{0.746840in}{3.070661in}}{\pgfqpoint{0.757439in}{3.075051in}}{\pgfqpoint{0.765253in}{3.082864in}}%
\pgfpathcurveto{\pgfqpoint{0.773066in}{3.090678in}}{\pgfqpoint{0.777456in}{3.101277in}}{\pgfqpoint{0.777456in}{3.112327in}}%
\pgfpathcurveto{\pgfqpoint{0.777456in}{3.123377in}}{\pgfqpoint{0.773066in}{3.133976in}}{\pgfqpoint{0.765253in}{3.141790in}}%
\pgfpathcurveto{\pgfqpoint{0.757439in}{3.149604in}}{\pgfqpoint{0.746840in}{3.153994in}}{\pgfqpoint{0.735790in}{3.153994in}}%
\pgfpathcurveto{\pgfqpoint{0.724740in}{3.153994in}}{\pgfqpoint{0.714141in}{3.149604in}}{\pgfqpoint{0.706327in}{3.141790in}}%
\pgfpathcurveto{\pgfqpoint{0.698513in}{3.133976in}}{\pgfqpoint{0.694123in}{3.123377in}}{\pgfqpoint{0.694123in}{3.112327in}}%
\pgfpathcurveto{\pgfqpoint{0.694123in}{3.101277in}}{\pgfqpoint{0.698513in}{3.090678in}}{\pgfqpoint{0.706327in}{3.082864in}}%
\pgfpathcurveto{\pgfqpoint{0.714141in}{3.075051in}}{\pgfqpoint{0.724740in}{3.070661in}}{\pgfqpoint{0.735790in}{3.070661in}}%
\pgfpathclose%
\pgfusepath{stroke,fill}%
\end{pgfscope}%
\begin{pgfscope}%
\pgfpathrectangle{\pgfqpoint{0.511823in}{0.504323in}}{\pgfqpoint{3.218177in}{3.225677in}} %
\pgfusepath{clip}%
\pgfsetbuttcap%
\pgfsetroundjoin%
\definecolor{currentfill}{rgb}{0.501961,0.000000,0.000000}%
\pgfsetfillcolor{currentfill}%
\pgfsetfillopacity{0.400000}%
\pgfsetlinewidth{0.501875pt}%
\definecolor{currentstroke}{rgb}{0.501961,0.000000,0.000000}%
\pgfsetstrokecolor{currentstroke}%
\pgfsetstrokeopacity{0.400000}%
\pgfsetdash{}{0pt}%
\pgfpathmoveto{\pgfqpoint{0.729802in}{3.203028in}}%
\pgfpathcurveto{\pgfqpoint{0.740853in}{3.203028in}}{\pgfqpoint{0.751452in}{3.207418in}}{\pgfqpoint{0.759265in}{3.215231in}}%
\pgfpathcurveto{\pgfqpoint{0.767079in}{3.223045in}}{\pgfqpoint{0.771469in}{3.233644in}}{\pgfqpoint{0.771469in}{3.244694in}}%
\pgfpathcurveto{\pgfqpoint{0.771469in}{3.255744in}}{\pgfqpoint{0.767079in}{3.266343in}}{\pgfqpoint{0.759265in}{3.274157in}}%
\pgfpathcurveto{\pgfqpoint{0.751452in}{3.281971in}}{\pgfqpoint{0.740853in}{3.286361in}}{\pgfqpoint{0.729802in}{3.286361in}}%
\pgfpathcurveto{\pgfqpoint{0.718752in}{3.286361in}}{\pgfqpoint{0.708153in}{3.281971in}}{\pgfqpoint{0.700340in}{3.274157in}}%
\pgfpathcurveto{\pgfqpoint{0.692526in}{3.266343in}}{\pgfqpoint{0.688136in}{3.255744in}}{\pgfqpoint{0.688136in}{3.244694in}}%
\pgfpathcurveto{\pgfqpoint{0.688136in}{3.233644in}}{\pgfqpoint{0.692526in}{3.223045in}}{\pgfqpoint{0.700340in}{3.215231in}}%
\pgfpathcurveto{\pgfqpoint{0.708153in}{3.207418in}}{\pgfqpoint{0.718752in}{3.203028in}}{\pgfqpoint{0.729802in}{3.203028in}}%
\pgfpathclose%
\pgfusepath{stroke,fill}%
\end{pgfscope}%
\begin{pgfscope}%
\pgfpathrectangle{\pgfqpoint{0.511823in}{0.504323in}}{\pgfqpoint{3.218177in}{3.225677in}} %
\pgfusepath{clip}%
\pgfsetbuttcap%
\pgfsetroundjoin%
\definecolor{currentfill}{rgb}{0.501961,0.000000,0.000000}%
\pgfsetfillcolor{currentfill}%
\pgfsetfillopacity{0.400000}%
\pgfsetlinewidth{0.501875pt}%
\definecolor{currentstroke}{rgb}{0.501961,0.000000,0.000000}%
\pgfsetstrokecolor{currentstroke}%
\pgfsetstrokeopacity{0.400000}%
\pgfsetdash{}{0pt}%
\pgfpathmoveto{\pgfqpoint{0.721182in}{3.206976in}}%
\pgfpathcurveto{\pgfqpoint{0.732232in}{3.206976in}}{\pgfqpoint{0.742831in}{3.211367in}}{\pgfqpoint{0.750645in}{3.219180in}}%
\pgfpathcurveto{\pgfqpoint{0.758459in}{3.226994in}}{\pgfqpoint{0.762849in}{3.237593in}}{\pgfqpoint{0.762849in}{3.248643in}}%
\pgfpathcurveto{\pgfqpoint{0.762849in}{3.259693in}}{\pgfqpoint{0.758459in}{3.270292in}}{\pgfqpoint{0.750645in}{3.278106in}}%
\pgfpathcurveto{\pgfqpoint{0.742831in}{3.285919in}}{\pgfqpoint{0.732232in}{3.290310in}}{\pgfqpoint{0.721182in}{3.290310in}}%
\pgfpathcurveto{\pgfqpoint{0.710132in}{3.290310in}}{\pgfqpoint{0.699533in}{3.285919in}}{\pgfqpoint{0.691719in}{3.278106in}}%
\pgfpathcurveto{\pgfqpoint{0.683906in}{3.270292in}}{\pgfqpoint{0.679516in}{3.259693in}}{\pgfqpoint{0.679516in}{3.248643in}}%
\pgfpathcurveto{\pgfqpoint{0.679516in}{3.237593in}}{\pgfqpoint{0.683906in}{3.226994in}}{\pgfqpoint{0.691719in}{3.219180in}}%
\pgfpathcurveto{\pgfqpoint{0.699533in}{3.211367in}}{\pgfqpoint{0.710132in}{3.206976in}}{\pgfqpoint{0.721182in}{3.206976in}}%
\pgfpathclose%
\pgfusepath{stroke,fill}%
\end{pgfscope}%
\begin{pgfscope}%
\pgfpathrectangle{\pgfqpoint{0.511823in}{0.504323in}}{\pgfqpoint{3.218177in}{3.225677in}} %
\pgfusepath{clip}%
\pgfsetbuttcap%
\pgfsetroundjoin%
\definecolor{currentfill}{rgb}{0.501961,0.000000,0.000000}%
\pgfsetfillcolor{currentfill}%
\pgfsetfillopacity{0.400000}%
\pgfsetlinewidth{0.501875pt}%
\definecolor{currentstroke}{rgb}{0.501961,0.000000,0.000000}%
\pgfsetstrokecolor{currentstroke}%
\pgfsetstrokeopacity{0.400000}%
\pgfsetdash{}{0pt}%
\pgfpathmoveto{\pgfqpoint{0.712464in}{3.203811in}}%
\pgfpathcurveto{\pgfqpoint{0.723514in}{3.203811in}}{\pgfqpoint{0.734113in}{3.208201in}}{\pgfqpoint{0.741927in}{3.216015in}}%
\pgfpathcurveto{\pgfqpoint{0.749740in}{3.223828in}}{\pgfqpoint{0.754130in}{3.234427in}}{\pgfqpoint{0.754130in}{3.245477in}}%
\pgfpathcurveto{\pgfqpoint{0.754130in}{3.256528in}}{\pgfqpoint{0.749740in}{3.267127in}}{\pgfqpoint{0.741927in}{3.274940in}}%
\pgfpathcurveto{\pgfqpoint{0.734113in}{3.282754in}}{\pgfqpoint{0.723514in}{3.287144in}}{\pgfqpoint{0.712464in}{3.287144in}}%
\pgfpathcurveto{\pgfqpoint{0.701414in}{3.287144in}}{\pgfqpoint{0.690815in}{3.282754in}}{\pgfqpoint{0.683001in}{3.274940in}}%
\pgfpathcurveto{\pgfqpoint{0.675187in}{3.267127in}}{\pgfqpoint{0.670797in}{3.256528in}}{\pgfqpoint{0.670797in}{3.245477in}}%
\pgfpathcurveto{\pgfqpoint{0.670797in}{3.234427in}}{\pgfqpoint{0.675187in}{3.223828in}}{\pgfqpoint{0.683001in}{3.216015in}}%
\pgfpathcurveto{\pgfqpoint{0.690815in}{3.208201in}}{\pgfqpoint{0.701414in}{3.203811in}}{\pgfqpoint{0.712464in}{3.203811in}}%
\pgfpathclose%
\pgfusepath{stroke,fill}%
\end{pgfscope}%
\begin{pgfscope}%
\pgfpathrectangle{\pgfqpoint{0.511823in}{0.504323in}}{\pgfqpoint{3.218177in}{3.225677in}} %
\pgfusepath{clip}%
\pgfsetbuttcap%
\pgfsetroundjoin%
\definecolor{currentfill}{rgb}{0.501961,0.000000,0.000000}%
\pgfsetfillcolor{currentfill}%
\pgfsetfillopacity{0.400000}%
\pgfsetlinewidth{0.501875pt}%
\definecolor{currentstroke}{rgb}{0.501961,0.000000,0.000000}%
\pgfsetstrokecolor{currentstroke}%
\pgfsetstrokeopacity{0.400000}%
\pgfsetdash{}{0pt}%
\pgfpathmoveto{\pgfqpoint{0.703098in}{3.101685in}}%
\pgfpathcurveto{\pgfqpoint{0.714148in}{3.101685in}}{\pgfqpoint{0.724747in}{3.106076in}}{\pgfqpoint{0.732560in}{3.113889in}}%
\pgfpathcurveto{\pgfqpoint{0.740374in}{3.121703in}}{\pgfqpoint{0.744764in}{3.132302in}}{\pgfqpoint{0.744764in}{3.143352in}}%
\pgfpathcurveto{\pgfqpoint{0.744764in}{3.154402in}}{\pgfqpoint{0.740374in}{3.165001in}}{\pgfqpoint{0.732560in}{3.172815in}}%
\pgfpathcurveto{\pgfqpoint{0.724747in}{3.180629in}}{\pgfqpoint{0.714148in}{3.185019in}}{\pgfqpoint{0.703098in}{3.185019in}}%
\pgfpathcurveto{\pgfqpoint{0.692047in}{3.185019in}}{\pgfqpoint{0.681448in}{3.180629in}}{\pgfqpoint{0.673635in}{3.172815in}}%
\pgfpathcurveto{\pgfqpoint{0.665821in}{3.165001in}}{\pgfqpoint{0.661431in}{3.154402in}}{\pgfqpoint{0.661431in}{3.143352in}}%
\pgfpathcurveto{\pgfqpoint{0.661431in}{3.132302in}}{\pgfqpoint{0.665821in}{3.121703in}}{\pgfqpoint{0.673635in}{3.113889in}}%
\pgfpathcurveto{\pgfqpoint{0.681448in}{3.106076in}}{\pgfqpoint{0.692047in}{3.101685in}}{\pgfqpoint{0.703098in}{3.101685in}}%
\pgfpathclose%
\pgfusepath{stroke,fill}%
\end{pgfscope}%
\begin{pgfscope}%
\pgfpathrectangle{\pgfqpoint{0.511823in}{0.504323in}}{\pgfqpoint{3.218177in}{3.225677in}} %
\pgfusepath{clip}%
\pgfsetbuttcap%
\pgfsetroundjoin%
\definecolor{currentfill}{rgb}{0.501961,0.000000,0.000000}%
\pgfsetfillcolor{currentfill}%
\pgfsetfillopacity{0.400000}%
\pgfsetlinewidth{0.501875pt}%
\definecolor{currentstroke}{rgb}{0.501961,0.000000,0.000000}%
\pgfsetstrokecolor{currentstroke}%
\pgfsetstrokeopacity{0.400000}%
\pgfsetdash{}{0pt}%
\pgfpathmoveto{\pgfqpoint{0.694924in}{3.147611in}}%
\pgfpathcurveto{\pgfqpoint{0.705974in}{3.147611in}}{\pgfqpoint{0.716573in}{3.152001in}}{\pgfqpoint{0.724386in}{3.159815in}}%
\pgfpathcurveto{\pgfqpoint{0.732200in}{3.167629in}}{\pgfqpoint{0.736590in}{3.178228in}}{\pgfqpoint{0.736590in}{3.189278in}}%
\pgfpathcurveto{\pgfqpoint{0.736590in}{3.200328in}}{\pgfqpoint{0.732200in}{3.210927in}}{\pgfqpoint{0.724386in}{3.218741in}}%
\pgfpathcurveto{\pgfqpoint{0.716573in}{3.226554in}}{\pgfqpoint{0.705974in}{3.230944in}}{\pgfqpoint{0.694924in}{3.230944in}}%
\pgfpathcurveto{\pgfqpoint{0.683874in}{3.230944in}}{\pgfqpoint{0.673274in}{3.226554in}}{\pgfqpoint{0.665461in}{3.218741in}}%
\pgfpathcurveto{\pgfqpoint{0.657647in}{3.210927in}}{\pgfqpoint{0.653257in}{3.200328in}}{\pgfqpoint{0.653257in}{3.189278in}}%
\pgfpathcurveto{\pgfqpoint{0.653257in}{3.178228in}}{\pgfqpoint{0.657647in}{3.167629in}}{\pgfqpoint{0.665461in}{3.159815in}}%
\pgfpathcurveto{\pgfqpoint{0.673274in}{3.152001in}}{\pgfqpoint{0.683874in}{3.147611in}}{\pgfqpoint{0.694924in}{3.147611in}}%
\pgfpathclose%
\pgfusepath{stroke,fill}%
\end{pgfscope}%
\begin{pgfscope}%
\pgfpathrectangle{\pgfqpoint{0.511823in}{0.504323in}}{\pgfqpoint{3.218177in}{3.225677in}} %
\pgfusepath{clip}%
\pgfsetbuttcap%
\pgfsetroundjoin%
\definecolor{currentfill}{rgb}{0.501961,0.000000,0.000000}%
\pgfsetfillcolor{currentfill}%
\pgfsetfillopacity{0.400000}%
\pgfsetlinewidth{0.501875pt}%
\definecolor{currentstroke}{rgb}{0.501961,0.000000,0.000000}%
\pgfsetstrokecolor{currentstroke}%
\pgfsetstrokeopacity{0.400000}%
\pgfsetdash{}{0pt}%
\pgfpathmoveto{\pgfqpoint{0.686439in}{3.122288in}}%
\pgfpathcurveto{\pgfqpoint{0.697489in}{3.122288in}}{\pgfqpoint{0.708088in}{3.126678in}}{\pgfqpoint{0.715902in}{3.134492in}}%
\pgfpathcurveto{\pgfqpoint{0.723716in}{3.142306in}}{\pgfqpoint{0.728106in}{3.152905in}}{\pgfqpoint{0.728106in}{3.163955in}}%
\pgfpathcurveto{\pgfqpoint{0.728106in}{3.175005in}}{\pgfqpoint{0.723716in}{3.185604in}}{\pgfqpoint{0.715902in}{3.193418in}}%
\pgfpathcurveto{\pgfqpoint{0.708088in}{3.201231in}}{\pgfqpoint{0.697489in}{3.205622in}}{\pgfqpoint{0.686439in}{3.205622in}}%
\pgfpathcurveto{\pgfqpoint{0.675389in}{3.205622in}}{\pgfqpoint{0.664790in}{3.201231in}}{\pgfqpoint{0.656976in}{3.193418in}}%
\pgfpathcurveto{\pgfqpoint{0.649163in}{3.185604in}}{\pgfqpoint{0.644772in}{3.175005in}}{\pgfqpoint{0.644772in}{3.163955in}}%
\pgfpathcurveto{\pgfqpoint{0.644772in}{3.152905in}}{\pgfqpoint{0.649163in}{3.142306in}}{\pgfqpoint{0.656976in}{3.134492in}}%
\pgfpathcurveto{\pgfqpoint{0.664790in}{3.126678in}}{\pgfqpoint{0.675389in}{3.122288in}}{\pgfqpoint{0.686439in}{3.122288in}}%
\pgfpathclose%
\pgfusepath{stroke,fill}%
\end{pgfscope}%
\begin{pgfscope}%
\pgfpathrectangle{\pgfqpoint{0.511823in}{0.504323in}}{\pgfqpoint{3.218177in}{3.225677in}} %
\pgfusepath{clip}%
\pgfsetbuttcap%
\pgfsetroundjoin%
\definecolor{currentfill}{rgb}{0.000000,0.000000,0.545098}%
\pgfsetfillcolor{currentfill}%
\pgfsetfillopacity{0.400000}%
\pgfsetlinewidth{0.501875pt}%
\definecolor{currentstroke}{rgb}{0.000000,0.000000,0.545098}%
\pgfsetstrokecolor{currentstroke}%
\pgfsetstrokeopacity{0.400000}%
\pgfsetdash{}{0pt}%
\pgfpathmoveto{\pgfqpoint{2.734919in}{0.639761in}}%
\pgfpathcurveto{\pgfqpoint{2.745969in}{0.639761in}}{\pgfqpoint{2.756568in}{0.644152in}}{\pgfqpoint{2.764382in}{0.651965in}}%
\pgfpathcurveto{\pgfqpoint{2.772195in}{0.659779in}}{\pgfqpoint{2.776586in}{0.670378in}}{\pgfqpoint{2.776586in}{0.681428in}}%
\pgfpathcurveto{\pgfqpoint{2.776586in}{0.692478in}}{\pgfqpoint{2.772195in}{0.703077in}}{\pgfqpoint{2.764382in}{0.710891in}}%
\pgfpathcurveto{\pgfqpoint{2.756568in}{0.718704in}}{\pgfqpoint{2.745969in}{0.723095in}}{\pgfqpoint{2.734919in}{0.723095in}}%
\pgfpathcurveto{\pgfqpoint{2.723869in}{0.723095in}}{\pgfqpoint{2.713270in}{0.718704in}}{\pgfqpoint{2.705456in}{0.710891in}}%
\pgfpathcurveto{\pgfqpoint{2.697643in}{0.703077in}}{\pgfqpoint{2.693252in}{0.692478in}}{\pgfqpoint{2.693252in}{0.681428in}}%
\pgfpathcurveto{\pgfqpoint{2.693252in}{0.670378in}}{\pgfqpoint{2.697643in}{0.659779in}}{\pgfqpoint{2.705456in}{0.651965in}}%
\pgfpathcurveto{\pgfqpoint{2.713270in}{0.644152in}}{\pgfqpoint{2.723869in}{0.639761in}}{\pgfqpoint{2.734919in}{0.639761in}}%
\pgfpathclose%
\pgfusepath{stroke,fill}%
\end{pgfscope}%
\begin{pgfscope}%
\pgfpathrectangle{\pgfqpoint{0.511823in}{0.504323in}}{\pgfqpoint{3.218177in}{3.225677in}} %
\pgfusepath{clip}%
\pgfsetbuttcap%
\pgfsetroundjoin%
\definecolor{currentfill}{rgb}{0.000000,0.000000,0.545098}%
\pgfsetfillcolor{currentfill}%
\pgfsetfillopacity{0.400000}%
\pgfsetlinewidth{0.501875pt}%
\definecolor{currentstroke}{rgb}{0.000000,0.000000,0.545098}%
\pgfsetstrokecolor{currentstroke}%
\pgfsetstrokeopacity{0.400000}%
\pgfsetdash{}{0pt}%
\pgfpathmoveto{\pgfqpoint{2.632480in}{0.645461in}}%
\pgfpathcurveto{\pgfqpoint{2.643531in}{0.645461in}}{\pgfqpoint{2.654130in}{0.649852in}}{\pgfqpoint{2.661943in}{0.657665in}}%
\pgfpathcurveto{\pgfqpoint{2.669757in}{0.665479in}}{\pgfqpoint{2.674147in}{0.676078in}}{\pgfqpoint{2.674147in}{0.687128in}}%
\pgfpathcurveto{\pgfqpoint{2.674147in}{0.698178in}}{\pgfqpoint{2.669757in}{0.708777in}}{\pgfqpoint{2.661943in}{0.716591in}}%
\pgfpathcurveto{\pgfqpoint{2.654130in}{0.724404in}}{\pgfqpoint{2.643531in}{0.728795in}}{\pgfqpoint{2.632480in}{0.728795in}}%
\pgfpathcurveto{\pgfqpoint{2.621430in}{0.728795in}}{\pgfqpoint{2.610831in}{0.724404in}}{\pgfqpoint{2.603018in}{0.716591in}}%
\pgfpathcurveto{\pgfqpoint{2.595204in}{0.708777in}}{\pgfqpoint{2.590814in}{0.698178in}}{\pgfqpoint{2.590814in}{0.687128in}}%
\pgfpathcurveto{\pgfqpoint{2.590814in}{0.676078in}}{\pgfqpoint{2.595204in}{0.665479in}}{\pgfqpoint{2.603018in}{0.657665in}}%
\pgfpathcurveto{\pgfqpoint{2.610831in}{0.649852in}}{\pgfqpoint{2.621430in}{0.645461in}}{\pgfqpoint{2.632480in}{0.645461in}}%
\pgfpathclose%
\pgfusepath{stroke,fill}%
\end{pgfscope}%
\begin{pgfscope}%
\pgfpathrectangle{\pgfqpoint{0.511823in}{0.504323in}}{\pgfqpoint{3.218177in}{3.225677in}} %
\pgfusepath{clip}%
\pgfsetbuttcap%
\pgfsetroundjoin%
\definecolor{currentfill}{rgb}{0.000000,0.000000,0.545098}%
\pgfsetfillcolor{currentfill}%
\pgfsetfillopacity{0.400000}%
\pgfsetlinewidth{0.501875pt}%
\definecolor{currentstroke}{rgb}{0.000000,0.000000,0.545098}%
\pgfsetstrokecolor{currentstroke}%
\pgfsetstrokeopacity{0.400000}%
\pgfsetdash{}{0pt}%
\pgfpathmoveto{\pgfqpoint{2.998812in}{0.653307in}}%
\pgfpathcurveto{\pgfqpoint{3.009862in}{0.653307in}}{\pgfqpoint{3.020461in}{0.657698in}}{\pgfqpoint{3.028275in}{0.665511in}}%
\pgfpathcurveto{\pgfqpoint{3.036088in}{0.673325in}}{\pgfqpoint{3.040479in}{0.683924in}}{\pgfqpoint{3.040479in}{0.694974in}}%
\pgfpathcurveto{\pgfqpoint{3.040479in}{0.706024in}}{\pgfqpoint{3.036088in}{0.716623in}}{\pgfqpoint{3.028275in}{0.724437in}}%
\pgfpathcurveto{\pgfqpoint{3.020461in}{0.732250in}}{\pgfqpoint{3.009862in}{0.736641in}}{\pgfqpoint{2.998812in}{0.736641in}}%
\pgfpathcurveto{\pgfqpoint{2.987762in}{0.736641in}}{\pgfqpoint{2.977163in}{0.732250in}}{\pgfqpoint{2.969349in}{0.724437in}}%
\pgfpathcurveto{\pgfqpoint{2.961536in}{0.716623in}}{\pgfqpoint{2.957145in}{0.706024in}}{\pgfqpoint{2.957145in}{0.694974in}}%
\pgfpathcurveto{\pgfqpoint{2.957145in}{0.683924in}}{\pgfqpoint{2.961536in}{0.673325in}}{\pgfqpoint{2.969349in}{0.665511in}}%
\pgfpathcurveto{\pgfqpoint{2.977163in}{0.657698in}}{\pgfqpoint{2.987762in}{0.653307in}}{\pgfqpoint{2.998812in}{0.653307in}}%
\pgfpathclose%
\pgfusepath{stroke,fill}%
\end{pgfscope}%
\begin{pgfscope}%
\pgfpathrectangle{\pgfqpoint{0.511823in}{0.504323in}}{\pgfqpoint{3.218177in}{3.225677in}} %
\pgfusepath{clip}%
\pgfsetbuttcap%
\pgfsetroundjoin%
\definecolor{currentfill}{rgb}{0.000000,0.000000,0.545098}%
\pgfsetfillcolor{currentfill}%
\pgfsetfillopacity{0.400000}%
\pgfsetlinewidth{0.501875pt}%
\definecolor{currentstroke}{rgb}{0.000000,0.000000,0.545098}%
\pgfsetstrokecolor{currentstroke}%
\pgfsetstrokeopacity{0.400000}%
\pgfsetdash{}{0pt}%
\pgfpathmoveto{\pgfqpoint{2.642854in}{0.656953in}}%
\pgfpathcurveto{\pgfqpoint{2.653904in}{0.656953in}}{\pgfqpoint{2.664503in}{0.661343in}}{\pgfqpoint{2.672317in}{0.669157in}}%
\pgfpathcurveto{\pgfqpoint{2.680130in}{0.676970in}}{\pgfqpoint{2.684520in}{0.687569in}}{\pgfqpoint{2.684520in}{0.698619in}}%
\pgfpathcurveto{\pgfqpoint{2.684520in}{0.709670in}}{\pgfqpoint{2.680130in}{0.720269in}}{\pgfqpoint{2.672317in}{0.728082in}}%
\pgfpathcurveto{\pgfqpoint{2.664503in}{0.735896in}}{\pgfqpoint{2.653904in}{0.740286in}}{\pgfqpoint{2.642854in}{0.740286in}}%
\pgfpathcurveto{\pgfqpoint{2.631804in}{0.740286in}}{\pgfqpoint{2.621205in}{0.735896in}}{\pgfqpoint{2.613391in}{0.728082in}}%
\pgfpathcurveto{\pgfqpoint{2.605577in}{0.720269in}}{\pgfqpoint{2.601187in}{0.709670in}}{\pgfqpoint{2.601187in}{0.698619in}}%
\pgfpathcurveto{\pgfqpoint{2.601187in}{0.687569in}}{\pgfqpoint{2.605577in}{0.676970in}}{\pgfqpoint{2.613391in}{0.669157in}}%
\pgfpathcurveto{\pgfqpoint{2.621205in}{0.661343in}}{\pgfqpoint{2.631804in}{0.656953in}}{\pgfqpoint{2.642854in}{0.656953in}}%
\pgfpathclose%
\pgfusepath{stroke,fill}%
\end{pgfscope}%
\begin{pgfscope}%
\pgfpathrectangle{\pgfqpoint{0.511823in}{0.504323in}}{\pgfqpoint{3.218177in}{3.225677in}} %
\pgfusepath{clip}%
\pgfsetbuttcap%
\pgfsetroundjoin%
\definecolor{currentfill}{rgb}{0.000000,0.000000,0.545098}%
\pgfsetfillcolor{currentfill}%
\pgfsetfillopacity{0.400000}%
\pgfsetlinewidth{0.501875pt}%
\definecolor{currentstroke}{rgb}{0.000000,0.000000,0.545098}%
\pgfsetstrokecolor{currentstroke}%
\pgfsetstrokeopacity{0.400000}%
\pgfsetdash{}{0pt}%
\pgfpathmoveto{\pgfqpoint{2.690767in}{0.663245in}}%
\pgfpathcurveto{\pgfqpoint{2.701817in}{0.663245in}}{\pgfqpoint{2.712416in}{0.667635in}}{\pgfqpoint{2.720229in}{0.675449in}}%
\pgfpathcurveto{\pgfqpoint{2.728043in}{0.683263in}}{\pgfqpoint{2.732433in}{0.693862in}}{\pgfqpoint{2.732433in}{0.704912in}}%
\pgfpathcurveto{\pgfqpoint{2.732433in}{0.715962in}}{\pgfqpoint{2.728043in}{0.726561in}}{\pgfqpoint{2.720229in}{0.734375in}}%
\pgfpathcurveto{\pgfqpoint{2.712416in}{0.742188in}}{\pgfqpoint{2.701817in}{0.746578in}}{\pgfqpoint{2.690767in}{0.746578in}}%
\pgfpathcurveto{\pgfqpoint{2.679716in}{0.746578in}}{\pgfqpoint{2.669117in}{0.742188in}}{\pgfqpoint{2.661304in}{0.734375in}}%
\pgfpathcurveto{\pgfqpoint{2.653490in}{0.726561in}}{\pgfqpoint{2.649100in}{0.715962in}}{\pgfqpoint{2.649100in}{0.704912in}}%
\pgfpathcurveto{\pgfqpoint{2.649100in}{0.693862in}}{\pgfqpoint{2.653490in}{0.683263in}}{\pgfqpoint{2.661304in}{0.675449in}}%
\pgfpathcurveto{\pgfqpoint{2.669117in}{0.667635in}}{\pgfqpoint{2.679716in}{0.663245in}}{\pgfqpoint{2.690767in}{0.663245in}}%
\pgfpathclose%
\pgfusepath{stroke,fill}%
\end{pgfscope}%
\begin{pgfscope}%
\pgfpathrectangle{\pgfqpoint{0.511823in}{0.504323in}}{\pgfqpoint{3.218177in}{3.225677in}} %
\pgfusepath{clip}%
\pgfsetbuttcap%
\pgfsetroundjoin%
\definecolor{currentfill}{rgb}{0.000000,0.000000,0.545098}%
\pgfsetfillcolor{currentfill}%
\pgfsetfillopacity{0.400000}%
\pgfsetlinewidth{0.501875pt}%
\definecolor{currentstroke}{rgb}{0.000000,0.000000,0.545098}%
\pgfsetstrokecolor{currentstroke}%
\pgfsetstrokeopacity{0.400000}%
\pgfsetdash{}{0pt}%
\pgfpathmoveto{\pgfqpoint{2.886049in}{0.671977in}}%
\pgfpathcurveto{\pgfqpoint{2.897099in}{0.671977in}}{\pgfqpoint{2.907698in}{0.676367in}}{\pgfqpoint{2.915511in}{0.684181in}}%
\pgfpathcurveto{\pgfqpoint{2.923325in}{0.691995in}}{\pgfqpoint{2.927715in}{0.702594in}}{\pgfqpoint{2.927715in}{0.713644in}}%
\pgfpathcurveto{\pgfqpoint{2.927715in}{0.724694in}}{\pgfqpoint{2.923325in}{0.735293in}}{\pgfqpoint{2.915511in}{0.743106in}}%
\pgfpathcurveto{\pgfqpoint{2.907698in}{0.750920in}}{\pgfqpoint{2.897099in}{0.755310in}}{\pgfqpoint{2.886049in}{0.755310in}}%
\pgfpathcurveto{\pgfqpoint{2.874998in}{0.755310in}}{\pgfqpoint{2.864399in}{0.750920in}}{\pgfqpoint{2.856586in}{0.743106in}}%
\pgfpathcurveto{\pgfqpoint{2.848772in}{0.735293in}}{\pgfqpoint{2.844382in}{0.724694in}}{\pgfqpoint{2.844382in}{0.713644in}}%
\pgfpathcurveto{\pgfqpoint{2.844382in}{0.702594in}}{\pgfqpoint{2.848772in}{0.691995in}}{\pgfqpoint{2.856586in}{0.684181in}}%
\pgfpathcurveto{\pgfqpoint{2.864399in}{0.676367in}}{\pgfqpoint{2.874998in}{0.671977in}}{\pgfqpoint{2.886049in}{0.671977in}}%
\pgfpathclose%
\pgfusepath{stroke,fill}%
\end{pgfscope}%
\begin{pgfscope}%
\pgfpathrectangle{\pgfqpoint{0.511823in}{0.504323in}}{\pgfqpoint{3.218177in}{3.225677in}} %
\pgfusepath{clip}%
\pgfsetbuttcap%
\pgfsetroundjoin%
\definecolor{currentfill}{rgb}{0.000000,0.000000,0.545098}%
\pgfsetfillcolor{currentfill}%
\pgfsetfillopacity{0.400000}%
\pgfsetlinewidth{0.501875pt}%
\definecolor{currentstroke}{rgb}{0.000000,0.000000,0.545098}%
\pgfsetstrokecolor{currentstroke}%
\pgfsetstrokeopacity{0.400000}%
\pgfsetdash{}{0pt}%
\pgfpathmoveto{\pgfqpoint{2.555429in}{0.672611in}}%
\pgfpathcurveto{\pgfqpoint{2.566480in}{0.672611in}}{\pgfqpoint{2.577079in}{0.677001in}}{\pgfqpoint{2.584892in}{0.684814in}}%
\pgfpathcurveto{\pgfqpoint{2.592706in}{0.692628in}}{\pgfqpoint{2.597096in}{0.703227in}}{\pgfqpoint{2.597096in}{0.714277in}}%
\pgfpathcurveto{\pgfqpoint{2.597096in}{0.725327in}}{\pgfqpoint{2.592706in}{0.735926in}}{\pgfqpoint{2.584892in}{0.743740in}}%
\pgfpathcurveto{\pgfqpoint{2.577079in}{0.751554in}}{\pgfqpoint{2.566480in}{0.755944in}}{\pgfqpoint{2.555429in}{0.755944in}}%
\pgfpathcurveto{\pgfqpoint{2.544379in}{0.755944in}}{\pgfqpoint{2.533780in}{0.751554in}}{\pgfqpoint{2.525967in}{0.743740in}}%
\pgfpathcurveto{\pgfqpoint{2.518153in}{0.735926in}}{\pgfqpoint{2.513763in}{0.725327in}}{\pgfqpoint{2.513763in}{0.714277in}}%
\pgfpathcurveto{\pgfqpoint{2.513763in}{0.703227in}}{\pgfqpoint{2.518153in}{0.692628in}}{\pgfqpoint{2.525967in}{0.684814in}}%
\pgfpathcurveto{\pgfqpoint{2.533780in}{0.677001in}}{\pgfqpoint{2.544379in}{0.672611in}}{\pgfqpoint{2.555429in}{0.672611in}}%
\pgfpathclose%
\pgfusepath{stroke,fill}%
\end{pgfscope}%
\begin{pgfscope}%
\pgfpathrectangle{\pgfqpoint{0.511823in}{0.504323in}}{\pgfqpoint{3.218177in}{3.225677in}} %
\pgfusepath{clip}%
\pgfsetbuttcap%
\pgfsetroundjoin%
\definecolor{currentfill}{rgb}{0.000000,0.000000,0.545098}%
\pgfsetfillcolor{currentfill}%
\pgfsetfillopacity{0.400000}%
\pgfsetlinewidth{0.501875pt}%
\definecolor{currentstroke}{rgb}{0.000000,0.000000,0.545098}%
\pgfsetstrokecolor{currentstroke}%
\pgfsetstrokeopacity{0.400000}%
\pgfsetdash{}{0pt}%
\pgfpathmoveto{\pgfqpoint{2.597432in}{0.678948in}}%
\pgfpathcurveto{\pgfqpoint{2.608482in}{0.678948in}}{\pgfqpoint{2.619081in}{0.683339in}}{\pgfqpoint{2.626895in}{0.691152in}}%
\pgfpathcurveto{\pgfqpoint{2.634708in}{0.698966in}}{\pgfqpoint{2.639099in}{0.709565in}}{\pgfqpoint{2.639099in}{0.720615in}}%
\pgfpathcurveto{\pgfqpoint{2.639099in}{0.731665in}}{\pgfqpoint{2.634708in}{0.742264in}}{\pgfqpoint{2.626895in}{0.750078in}}%
\pgfpathcurveto{\pgfqpoint{2.619081in}{0.757891in}}{\pgfqpoint{2.608482in}{0.762282in}}{\pgfqpoint{2.597432in}{0.762282in}}%
\pgfpathcurveto{\pgfqpoint{2.586382in}{0.762282in}}{\pgfqpoint{2.575783in}{0.757891in}}{\pgfqpoint{2.567969in}{0.750078in}}%
\pgfpathcurveto{\pgfqpoint{2.560156in}{0.742264in}}{\pgfqpoint{2.555765in}{0.731665in}}{\pgfqpoint{2.555765in}{0.720615in}}%
\pgfpathcurveto{\pgfqpoint{2.555765in}{0.709565in}}{\pgfqpoint{2.560156in}{0.698966in}}{\pgfqpoint{2.567969in}{0.691152in}}%
\pgfpathcurveto{\pgfqpoint{2.575783in}{0.683339in}}{\pgfqpoint{2.586382in}{0.678948in}}{\pgfqpoint{2.597432in}{0.678948in}}%
\pgfpathclose%
\pgfusepath{stroke,fill}%
\end{pgfscope}%
\begin{pgfscope}%
\pgfpathrectangle{\pgfqpoint{0.511823in}{0.504323in}}{\pgfqpoint{3.218177in}{3.225677in}} %
\pgfusepath{clip}%
\pgfsetbuttcap%
\pgfsetroundjoin%
\definecolor{currentfill}{rgb}{0.000000,0.000000,0.545098}%
\pgfsetfillcolor{currentfill}%
\pgfsetfillopacity{0.400000}%
\pgfsetlinewidth{0.501875pt}%
\definecolor{currentstroke}{rgb}{0.000000,0.000000,0.545098}%
\pgfsetstrokecolor{currentstroke}%
\pgfsetstrokeopacity{0.400000}%
\pgfsetdash{}{0pt}%
\pgfpathmoveto{\pgfqpoint{2.700633in}{0.686968in}}%
\pgfpathcurveto{\pgfqpoint{2.711684in}{0.686968in}}{\pgfqpoint{2.722283in}{0.691358in}}{\pgfqpoint{2.730096in}{0.699171in}}%
\pgfpathcurveto{\pgfqpoint{2.737910in}{0.706985in}}{\pgfqpoint{2.742300in}{0.717584in}}{\pgfqpoint{2.742300in}{0.728634in}}%
\pgfpathcurveto{\pgfqpoint{2.742300in}{0.739684in}}{\pgfqpoint{2.737910in}{0.750283in}}{\pgfqpoint{2.730096in}{0.758097in}}%
\pgfpathcurveto{\pgfqpoint{2.722283in}{0.765911in}}{\pgfqpoint{2.711684in}{0.770301in}}{\pgfqpoint{2.700633in}{0.770301in}}%
\pgfpathcurveto{\pgfqpoint{2.689583in}{0.770301in}}{\pgfqpoint{2.678984in}{0.765911in}}{\pgfqpoint{2.671171in}{0.758097in}}%
\pgfpathcurveto{\pgfqpoint{2.663357in}{0.750283in}}{\pgfqpoint{2.658967in}{0.739684in}}{\pgfqpoint{2.658967in}{0.728634in}}%
\pgfpathcurveto{\pgfqpoint{2.658967in}{0.717584in}}{\pgfqpoint{2.663357in}{0.706985in}}{\pgfqpoint{2.671171in}{0.699171in}}%
\pgfpathcurveto{\pgfqpoint{2.678984in}{0.691358in}}{\pgfqpoint{2.689583in}{0.686968in}}{\pgfqpoint{2.700633in}{0.686968in}}%
\pgfpathclose%
\pgfusepath{stroke,fill}%
\end{pgfscope}%
\begin{pgfscope}%
\pgfpathrectangle{\pgfqpoint{0.511823in}{0.504323in}}{\pgfqpoint{3.218177in}{3.225677in}} %
\pgfusepath{clip}%
\pgfsetbuttcap%
\pgfsetroundjoin%
\definecolor{currentfill}{rgb}{0.000000,0.000000,0.545098}%
\pgfsetfillcolor{currentfill}%
\pgfsetfillopacity{0.400000}%
\pgfsetlinewidth{0.501875pt}%
\definecolor{currentstroke}{rgb}{0.000000,0.000000,0.545098}%
\pgfsetstrokecolor{currentstroke}%
\pgfsetstrokeopacity{0.400000}%
\pgfsetdash{}{0pt}%
\pgfpathmoveto{\pgfqpoint{2.631842in}{0.691057in}}%
\pgfpathcurveto{\pgfqpoint{2.642892in}{0.691057in}}{\pgfqpoint{2.653491in}{0.695448in}}{\pgfqpoint{2.661305in}{0.703261in}}%
\pgfpathcurveto{\pgfqpoint{2.669118in}{0.711075in}}{\pgfqpoint{2.673508in}{0.721674in}}{\pgfqpoint{2.673508in}{0.732724in}}%
\pgfpathcurveto{\pgfqpoint{2.673508in}{0.743774in}}{\pgfqpoint{2.669118in}{0.754373in}}{\pgfqpoint{2.661305in}{0.762187in}}%
\pgfpathcurveto{\pgfqpoint{2.653491in}{0.770000in}}{\pgfqpoint{2.642892in}{0.774391in}}{\pgfqpoint{2.631842in}{0.774391in}}%
\pgfpathcurveto{\pgfqpoint{2.620792in}{0.774391in}}{\pgfqpoint{2.610193in}{0.770000in}}{\pgfqpoint{2.602379in}{0.762187in}}%
\pgfpathcurveto{\pgfqpoint{2.594565in}{0.754373in}}{\pgfqpoint{2.590175in}{0.743774in}}{\pgfqpoint{2.590175in}{0.732724in}}%
\pgfpathcurveto{\pgfqpoint{2.590175in}{0.721674in}}{\pgfqpoint{2.594565in}{0.711075in}}{\pgfqpoint{2.602379in}{0.703261in}}%
\pgfpathcurveto{\pgfqpoint{2.610193in}{0.695448in}}{\pgfqpoint{2.620792in}{0.691057in}}{\pgfqpoint{2.631842in}{0.691057in}}%
\pgfpathclose%
\pgfusepath{stroke,fill}%
\end{pgfscope}%
\begin{pgfscope}%
\pgfpathrectangle{\pgfqpoint{0.511823in}{0.504323in}}{\pgfqpoint{3.218177in}{3.225677in}} %
\pgfusepath{clip}%
\pgfsetbuttcap%
\pgfsetroundjoin%
\definecolor{currentfill}{rgb}{0.000000,0.000000,0.545098}%
\pgfsetfillcolor{currentfill}%
\pgfsetfillopacity{0.400000}%
\pgfsetlinewidth{0.501875pt}%
\definecolor{currentstroke}{rgb}{0.000000,0.000000,0.545098}%
\pgfsetstrokecolor{currentstroke}%
\pgfsetstrokeopacity{0.400000}%
\pgfsetdash{}{0pt}%
\pgfpathmoveto{\pgfqpoint{2.517241in}{0.693403in}}%
\pgfpathcurveto{\pgfqpoint{2.528291in}{0.693403in}}{\pgfqpoint{2.538890in}{0.697793in}}{\pgfqpoint{2.546704in}{0.705607in}}%
\pgfpathcurveto{\pgfqpoint{2.554518in}{0.713420in}}{\pgfqpoint{2.558908in}{0.724019in}}{\pgfqpoint{2.558908in}{0.735069in}}%
\pgfpathcurveto{\pgfqpoint{2.558908in}{0.746120in}}{\pgfqpoint{2.554518in}{0.756719in}}{\pgfqpoint{2.546704in}{0.764532in}}%
\pgfpathcurveto{\pgfqpoint{2.538890in}{0.772346in}}{\pgfqpoint{2.528291in}{0.776736in}}{\pgfqpoint{2.517241in}{0.776736in}}%
\pgfpathcurveto{\pgfqpoint{2.506191in}{0.776736in}}{\pgfqpoint{2.495592in}{0.772346in}}{\pgfqpoint{2.487778in}{0.764532in}}%
\pgfpathcurveto{\pgfqpoint{2.479965in}{0.756719in}}{\pgfqpoint{2.475574in}{0.746120in}}{\pgfqpoint{2.475574in}{0.735069in}}%
\pgfpathcurveto{\pgfqpoint{2.475574in}{0.724019in}}{\pgfqpoint{2.479965in}{0.713420in}}{\pgfqpoint{2.487778in}{0.705607in}}%
\pgfpathcurveto{\pgfqpoint{2.495592in}{0.697793in}}{\pgfqpoint{2.506191in}{0.693403in}}{\pgfqpoint{2.517241in}{0.693403in}}%
\pgfpathclose%
\pgfusepath{stroke,fill}%
\end{pgfscope}%
\begin{pgfscope}%
\pgfpathrectangle{\pgfqpoint{0.511823in}{0.504323in}}{\pgfqpoint{3.218177in}{3.225677in}} %
\pgfusepath{clip}%
\pgfsetbuttcap%
\pgfsetroundjoin%
\definecolor{currentfill}{rgb}{0.000000,0.000000,0.545098}%
\pgfsetfillcolor{currentfill}%
\pgfsetfillopacity{0.400000}%
\pgfsetlinewidth{0.501875pt}%
\definecolor{currentstroke}{rgb}{0.000000,0.000000,0.545098}%
\pgfsetstrokecolor{currentstroke}%
\pgfsetstrokeopacity{0.400000}%
\pgfsetdash{}{0pt}%
\pgfpathmoveto{\pgfqpoint{2.648028in}{0.702986in}}%
\pgfpathcurveto{\pgfqpoint{2.659078in}{0.702986in}}{\pgfqpoint{2.669677in}{0.707377in}}{\pgfqpoint{2.677491in}{0.715190in}}%
\pgfpathcurveto{\pgfqpoint{2.685305in}{0.723004in}}{\pgfqpoint{2.689695in}{0.733603in}}{\pgfqpoint{2.689695in}{0.744653in}}%
\pgfpathcurveto{\pgfqpoint{2.689695in}{0.755703in}}{\pgfqpoint{2.685305in}{0.766302in}}{\pgfqpoint{2.677491in}{0.774116in}}%
\pgfpathcurveto{\pgfqpoint{2.669677in}{0.781930in}}{\pgfqpoint{2.659078in}{0.786320in}}{\pgfqpoint{2.648028in}{0.786320in}}%
\pgfpathcurveto{\pgfqpoint{2.636978in}{0.786320in}}{\pgfqpoint{2.626379in}{0.781930in}}{\pgfqpoint{2.618565in}{0.774116in}}%
\pgfpathcurveto{\pgfqpoint{2.610752in}{0.766302in}}{\pgfqpoint{2.606362in}{0.755703in}}{\pgfqpoint{2.606362in}{0.744653in}}%
\pgfpathcurveto{\pgfqpoint{2.606362in}{0.733603in}}{\pgfqpoint{2.610752in}{0.723004in}}{\pgfqpoint{2.618565in}{0.715190in}}%
\pgfpathcurveto{\pgfqpoint{2.626379in}{0.707377in}}{\pgfqpoint{2.636978in}{0.702986in}}{\pgfqpoint{2.648028in}{0.702986in}}%
\pgfpathclose%
\pgfusepath{stroke,fill}%
\end{pgfscope}%
\begin{pgfscope}%
\pgfpathrectangle{\pgfqpoint{0.511823in}{0.504323in}}{\pgfqpoint{3.218177in}{3.225677in}} %
\pgfusepath{clip}%
\pgfsetbuttcap%
\pgfsetroundjoin%
\definecolor{currentfill}{rgb}{0.000000,0.000000,0.545098}%
\pgfsetfillcolor{currentfill}%
\pgfsetfillopacity{0.400000}%
\pgfsetlinewidth{0.501875pt}%
\definecolor{currentstroke}{rgb}{0.000000,0.000000,0.545098}%
\pgfsetstrokecolor{currentstroke}%
\pgfsetstrokeopacity{0.400000}%
\pgfsetdash{}{0pt}%
\pgfpathmoveto{\pgfqpoint{2.627395in}{0.708014in}}%
\pgfpathcurveto{\pgfqpoint{2.638446in}{0.708014in}}{\pgfqpoint{2.649045in}{0.712404in}}{\pgfqpoint{2.656858in}{0.720218in}}%
\pgfpathcurveto{\pgfqpoint{2.664672in}{0.728031in}}{\pgfqpoint{2.669062in}{0.738630in}}{\pgfqpoint{2.669062in}{0.749681in}}%
\pgfpathcurveto{\pgfqpoint{2.669062in}{0.760731in}}{\pgfqpoint{2.664672in}{0.771330in}}{\pgfqpoint{2.656858in}{0.779143in}}%
\pgfpathcurveto{\pgfqpoint{2.649045in}{0.786957in}}{\pgfqpoint{2.638446in}{0.791347in}}{\pgfqpoint{2.627395in}{0.791347in}}%
\pgfpathcurveto{\pgfqpoint{2.616345in}{0.791347in}}{\pgfqpoint{2.605746in}{0.786957in}}{\pgfqpoint{2.597933in}{0.779143in}}%
\pgfpathcurveto{\pgfqpoint{2.590119in}{0.771330in}}{\pgfqpoint{2.585729in}{0.760731in}}{\pgfqpoint{2.585729in}{0.749681in}}%
\pgfpathcurveto{\pgfqpoint{2.585729in}{0.738630in}}{\pgfqpoint{2.590119in}{0.728031in}}{\pgfqpoint{2.597933in}{0.720218in}}%
\pgfpathcurveto{\pgfqpoint{2.605746in}{0.712404in}}{\pgfqpoint{2.616345in}{0.708014in}}{\pgfqpoint{2.627395in}{0.708014in}}%
\pgfpathclose%
\pgfusepath{stroke,fill}%
\end{pgfscope}%
\begin{pgfscope}%
\pgfpathrectangle{\pgfqpoint{0.511823in}{0.504323in}}{\pgfqpoint{3.218177in}{3.225677in}} %
\pgfusepath{clip}%
\pgfsetbuttcap%
\pgfsetroundjoin%
\definecolor{currentfill}{rgb}{0.000000,0.000000,0.545098}%
\pgfsetfillcolor{currentfill}%
\pgfsetfillopacity{0.400000}%
\pgfsetlinewidth{0.501875pt}%
\definecolor{currentstroke}{rgb}{0.000000,0.000000,0.545098}%
\pgfsetstrokecolor{currentstroke}%
\pgfsetstrokeopacity{0.400000}%
\pgfsetdash{}{0pt}%
\pgfpathmoveto{\pgfqpoint{2.754710in}{0.718558in}}%
\pgfpathcurveto{\pgfqpoint{2.765760in}{0.718558in}}{\pgfqpoint{2.776359in}{0.722948in}}{\pgfqpoint{2.784172in}{0.730762in}}%
\pgfpathcurveto{\pgfqpoint{2.791986in}{0.738576in}}{\pgfqpoint{2.796376in}{0.749175in}}{\pgfqpoint{2.796376in}{0.760225in}}%
\pgfpathcurveto{\pgfqpoint{2.796376in}{0.771275in}}{\pgfqpoint{2.791986in}{0.781874in}}{\pgfqpoint{2.784172in}{0.789688in}}%
\pgfpathcurveto{\pgfqpoint{2.776359in}{0.797501in}}{\pgfqpoint{2.765760in}{0.801892in}}{\pgfqpoint{2.754710in}{0.801892in}}%
\pgfpathcurveto{\pgfqpoint{2.743660in}{0.801892in}}{\pgfqpoint{2.733061in}{0.797501in}}{\pgfqpoint{2.725247in}{0.789688in}}%
\pgfpathcurveto{\pgfqpoint{2.717433in}{0.781874in}}{\pgfqpoint{2.713043in}{0.771275in}}{\pgfqpoint{2.713043in}{0.760225in}}%
\pgfpathcurveto{\pgfqpoint{2.713043in}{0.749175in}}{\pgfqpoint{2.717433in}{0.738576in}}{\pgfqpoint{2.725247in}{0.730762in}}%
\pgfpathcurveto{\pgfqpoint{2.733061in}{0.722948in}}{\pgfqpoint{2.743660in}{0.718558in}}{\pgfqpoint{2.754710in}{0.718558in}}%
\pgfpathclose%
\pgfusepath{stroke,fill}%
\end{pgfscope}%
\begin{pgfscope}%
\pgfpathrectangle{\pgfqpoint{0.511823in}{0.504323in}}{\pgfqpoint{3.218177in}{3.225677in}} %
\pgfusepath{clip}%
\pgfsetbuttcap%
\pgfsetroundjoin%
\definecolor{currentfill}{rgb}{0.000000,0.000000,0.545098}%
\pgfsetfillcolor{currentfill}%
\pgfsetfillopacity{0.400000}%
\pgfsetlinewidth{0.501875pt}%
\definecolor{currentstroke}{rgb}{0.000000,0.000000,0.545098}%
\pgfsetstrokecolor{currentstroke}%
\pgfsetstrokeopacity{0.400000}%
\pgfsetdash{}{0pt}%
\pgfpathmoveto{\pgfqpoint{2.453811in}{0.712281in}}%
\pgfpathcurveto{\pgfqpoint{2.464861in}{0.712281in}}{\pgfqpoint{2.475460in}{0.716671in}}{\pgfqpoint{2.483274in}{0.724484in}}%
\pgfpathcurveto{\pgfqpoint{2.491087in}{0.732298in}}{\pgfqpoint{2.495477in}{0.742897in}}{\pgfqpoint{2.495477in}{0.753947in}}%
\pgfpathcurveto{\pgfqpoint{2.495477in}{0.764997in}}{\pgfqpoint{2.491087in}{0.775596in}}{\pgfqpoint{2.483274in}{0.783410in}}%
\pgfpathcurveto{\pgfqpoint{2.475460in}{0.791224in}}{\pgfqpoint{2.464861in}{0.795614in}}{\pgfqpoint{2.453811in}{0.795614in}}%
\pgfpathcurveto{\pgfqpoint{2.442761in}{0.795614in}}{\pgfqpoint{2.432162in}{0.791224in}}{\pgfqpoint{2.424348in}{0.783410in}}%
\pgfpathcurveto{\pgfqpoint{2.416534in}{0.775596in}}{\pgfqpoint{2.412144in}{0.764997in}}{\pgfqpoint{2.412144in}{0.753947in}}%
\pgfpathcurveto{\pgfqpoint{2.412144in}{0.742897in}}{\pgfqpoint{2.416534in}{0.732298in}}{\pgfqpoint{2.424348in}{0.724484in}}%
\pgfpathcurveto{\pgfqpoint{2.432162in}{0.716671in}}{\pgfqpoint{2.442761in}{0.712281in}}{\pgfqpoint{2.453811in}{0.712281in}}%
\pgfpathclose%
\pgfusepath{stroke,fill}%
\end{pgfscope}%
\begin{pgfscope}%
\pgfpathrectangle{\pgfqpoint{0.511823in}{0.504323in}}{\pgfqpoint{3.218177in}{3.225677in}} %
\pgfusepath{clip}%
\pgfsetbuttcap%
\pgfsetroundjoin%
\definecolor{currentfill}{rgb}{0.000000,0.000000,0.545098}%
\pgfsetfillcolor{currentfill}%
\pgfsetfillopacity{0.400000}%
\pgfsetlinewidth{0.501875pt}%
\definecolor{currentstroke}{rgb}{0.000000,0.000000,0.545098}%
\pgfsetstrokecolor{currentstroke}%
\pgfsetstrokeopacity{0.400000}%
\pgfsetdash{}{0pt}%
\pgfpathmoveto{\pgfqpoint{2.581382in}{0.723077in}}%
\pgfpathcurveto{\pgfqpoint{2.592432in}{0.723077in}}{\pgfqpoint{2.603031in}{0.727467in}}{\pgfqpoint{2.610845in}{0.735281in}}%
\pgfpathcurveto{\pgfqpoint{2.618658in}{0.743094in}}{\pgfqpoint{2.623049in}{0.753693in}}{\pgfqpoint{2.623049in}{0.764744in}}%
\pgfpathcurveto{\pgfqpoint{2.623049in}{0.775794in}}{\pgfqpoint{2.618658in}{0.786393in}}{\pgfqpoint{2.610845in}{0.794206in}}%
\pgfpathcurveto{\pgfqpoint{2.603031in}{0.802020in}}{\pgfqpoint{2.592432in}{0.806410in}}{\pgfqpoint{2.581382in}{0.806410in}}%
\pgfpathcurveto{\pgfqpoint{2.570332in}{0.806410in}}{\pgfqpoint{2.559733in}{0.802020in}}{\pgfqpoint{2.551919in}{0.794206in}}%
\pgfpathcurveto{\pgfqpoint{2.544106in}{0.786393in}}{\pgfqpoint{2.539715in}{0.775794in}}{\pgfqpoint{2.539715in}{0.764744in}}%
\pgfpathcurveto{\pgfqpoint{2.539715in}{0.753693in}}{\pgfqpoint{2.544106in}{0.743094in}}{\pgfqpoint{2.551919in}{0.735281in}}%
\pgfpathcurveto{\pgfqpoint{2.559733in}{0.727467in}}{\pgfqpoint{2.570332in}{0.723077in}}{\pgfqpoint{2.581382in}{0.723077in}}%
\pgfpathclose%
\pgfusepath{stroke,fill}%
\end{pgfscope}%
\begin{pgfscope}%
\pgfpathrectangle{\pgfqpoint{0.511823in}{0.504323in}}{\pgfqpoint{3.218177in}{3.225677in}} %
\pgfusepath{clip}%
\pgfsetbuttcap%
\pgfsetroundjoin%
\definecolor{currentfill}{rgb}{0.000000,0.000000,0.545098}%
\pgfsetfillcolor{currentfill}%
\pgfsetfillopacity{0.400000}%
\pgfsetlinewidth{0.501875pt}%
\definecolor{currentstroke}{rgb}{0.000000,0.000000,0.545098}%
\pgfsetstrokecolor{currentstroke}%
\pgfsetstrokeopacity{0.400000}%
\pgfsetdash{}{0pt}%
\pgfpathmoveto{\pgfqpoint{2.826414in}{0.740133in}}%
\pgfpathcurveto{\pgfqpoint{2.837464in}{0.740133in}}{\pgfqpoint{2.848063in}{0.744523in}}{\pgfqpoint{2.855877in}{0.752337in}}%
\pgfpathcurveto{\pgfqpoint{2.863691in}{0.760151in}}{\pgfqpoint{2.868081in}{0.770750in}}{\pgfqpoint{2.868081in}{0.781800in}}%
\pgfpathcurveto{\pgfqpoint{2.868081in}{0.792850in}}{\pgfqpoint{2.863691in}{0.803449in}}{\pgfqpoint{2.855877in}{0.811263in}}%
\pgfpathcurveto{\pgfqpoint{2.848063in}{0.819076in}}{\pgfqpoint{2.837464in}{0.823466in}}{\pgfqpoint{2.826414in}{0.823466in}}%
\pgfpathcurveto{\pgfqpoint{2.815364in}{0.823466in}}{\pgfqpoint{2.804765in}{0.819076in}}{\pgfqpoint{2.796951in}{0.811263in}}%
\pgfpathcurveto{\pgfqpoint{2.789138in}{0.803449in}}{\pgfqpoint{2.784748in}{0.792850in}}{\pgfqpoint{2.784748in}{0.781800in}}%
\pgfpathcurveto{\pgfqpoint{2.784748in}{0.770750in}}{\pgfqpoint{2.789138in}{0.760151in}}{\pgfqpoint{2.796951in}{0.752337in}}%
\pgfpathcurveto{\pgfqpoint{2.804765in}{0.744523in}}{\pgfqpoint{2.815364in}{0.740133in}}{\pgfqpoint{2.826414in}{0.740133in}}%
\pgfpathclose%
\pgfusepath{stroke,fill}%
\end{pgfscope}%
\begin{pgfscope}%
\pgfpathrectangle{\pgfqpoint{0.511823in}{0.504323in}}{\pgfqpoint{3.218177in}{3.225677in}} %
\pgfusepath{clip}%
\pgfsetbuttcap%
\pgfsetroundjoin%
\definecolor{currentfill}{rgb}{0.000000,0.000000,0.545098}%
\pgfsetfillcolor{currentfill}%
\pgfsetfillopacity{0.400000}%
\pgfsetlinewidth{0.501875pt}%
\definecolor{currentstroke}{rgb}{0.000000,0.000000,0.545098}%
\pgfsetstrokecolor{currentstroke}%
\pgfsetstrokeopacity{0.400000}%
\pgfsetdash{}{0pt}%
\pgfpathmoveto{\pgfqpoint{2.919582in}{0.751062in}}%
\pgfpathcurveto{\pgfqpoint{2.930632in}{0.751062in}}{\pgfqpoint{2.941231in}{0.755452in}}{\pgfqpoint{2.949045in}{0.763265in}}%
\pgfpathcurveto{\pgfqpoint{2.956858in}{0.771079in}}{\pgfqpoint{2.961249in}{0.781678in}}{\pgfqpoint{2.961249in}{0.792728in}}%
\pgfpathcurveto{\pgfqpoint{2.961249in}{0.803778in}}{\pgfqpoint{2.956858in}{0.814377in}}{\pgfqpoint{2.949045in}{0.822191in}}%
\pgfpathcurveto{\pgfqpoint{2.941231in}{0.830005in}}{\pgfqpoint{2.930632in}{0.834395in}}{\pgfqpoint{2.919582in}{0.834395in}}%
\pgfpathcurveto{\pgfqpoint{2.908532in}{0.834395in}}{\pgfqpoint{2.897933in}{0.830005in}}{\pgfqpoint{2.890119in}{0.822191in}}%
\pgfpathcurveto{\pgfqpoint{2.882306in}{0.814377in}}{\pgfqpoint{2.877915in}{0.803778in}}{\pgfqpoint{2.877915in}{0.792728in}}%
\pgfpathcurveto{\pgfqpoint{2.877915in}{0.781678in}}{\pgfqpoint{2.882306in}{0.771079in}}{\pgfqpoint{2.890119in}{0.763265in}}%
\pgfpathcurveto{\pgfqpoint{2.897933in}{0.755452in}}{\pgfqpoint{2.908532in}{0.751062in}}{\pgfqpoint{2.919582in}{0.751062in}}%
\pgfpathclose%
\pgfusepath{stroke,fill}%
\end{pgfscope}%
\begin{pgfscope}%
\pgfpathrectangle{\pgfqpoint{0.511823in}{0.504323in}}{\pgfqpoint{3.218177in}{3.225677in}} %
\pgfusepath{clip}%
\pgfsetbuttcap%
\pgfsetroundjoin%
\definecolor{currentfill}{rgb}{0.000000,0.000000,0.545098}%
\pgfsetfillcolor{currentfill}%
\pgfsetfillopacity{0.400000}%
\pgfsetlinewidth{0.501875pt}%
\definecolor{currentstroke}{rgb}{0.000000,0.000000,0.545098}%
\pgfsetstrokecolor{currentstroke}%
\pgfsetstrokeopacity{0.400000}%
\pgfsetdash{}{0pt}%
\pgfpathmoveto{\pgfqpoint{2.473035in}{0.734054in}}%
\pgfpathcurveto{\pgfqpoint{2.484085in}{0.734054in}}{\pgfqpoint{2.494684in}{0.738445in}}{\pgfqpoint{2.502498in}{0.746258in}}%
\pgfpathcurveto{\pgfqpoint{2.510311in}{0.754072in}}{\pgfqpoint{2.514702in}{0.764671in}}{\pgfqpoint{2.514702in}{0.775721in}}%
\pgfpathcurveto{\pgfqpoint{2.514702in}{0.786771in}}{\pgfqpoint{2.510311in}{0.797370in}}{\pgfqpoint{2.502498in}{0.805184in}}%
\pgfpathcurveto{\pgfqpoint{2.494684in}{0.812997in}}{\pgfqpoint{2.484085in}{0.817388in}}{\pgfqpoint{2.473035in}{0.817388in}}%
\pgfpathcurveto{\pgfqpoint{2.461985in}{0.817388in}}{\pgfqpoint{2.451386in}{0.812997in}}{\pgfqpoint{2.443572in}{0.805184in}}%
\pgfpathcurveto{\pgfqpoint{2.435758in}{0.797370in}}{\pgfqpoint{2.431368in}{0.786771in}}{\pgfqpoint{2.431368in}{0.775721in}}%
\pgfpathcurveto{\pgfqpoint{2.431368in}{0.764671in}}{\pgfqpoint{2.435758in}{0.754072in}}{\pgfqpoint{2.443572in}{0.746258in}}%
\pgfpathcurveto{\pgfqpoint{2.451386in}{0.738445in}}{\pgfqpoint{2.461985in}{0.734054in}}{\pgfqpoint{2.473035in}{0.734054in}}%
\pgfpathclose%
\pgfusepath{stroke,fill}%
\end{pgfscope}%
\begin{pgfscope}%
\pgfpathrectangle{\pgfqpoint{0.511823in}{0.504323in}}{\pgfqpoint{3.218177in}{3.225677in}} %
\pgfusepath{clip}%
\pgfsetbuttcap%
\pgfsetroundjoin%
\definecolor{currentfill}{rgb}{0.000000,0.000000,0.545098}%
\pgfsetfillcolor{currentfill}%
\pgfsetfillopacity{0.400000}%
\pgfsetlinewidth{0.501875pt}%
\definecolor{currentstroke}{rgb}{0.000000,0.000000,0.545098}%
\pgfsetstrokecolor{currentstroke}%
\pgfsetstrokeopacity{0.400000}%
\pgfsetdash{}{0pt}%
\pgfpathmoveto{\pgfqpoint{2.758221in}{0.755195in}}%
\pgfpathcurveto{\pgfqpoint{2.769271in}{0.755195in}}{\pgfqpoint{2.779870in}{0.759585in}}{\pgfqpoint{2.787683in}{0.767399in}}%
\pgfpathcurveto{\pgfqpoint{2.795497in}{0.775212in}}{\pgfqpoint{2.799887in}{0.785811in}}{\pgfqpoint{2.799887in}{0.796861in}}%
\pgfpathcurveto{\pgfqpoint{2.799887in}{0.807912in}}{\pgfqpoint{2.795497in}{0.818511in}}{\pgfqpoint{2.787683in}{0.826324in}}%
\pgfpathcurveto{\pgfqpoint{2.779870in}{0.834138in}}{\pgfqpoint{2.769271in}{0.838528in}}{\pgfqpoint{2.758221in}{0.838528in}}%
\pgfpathcurveto{\pgfqpoint{2.747170in}{0.838528in}}{\pgfqpoint{2.736571in}{0.834138in}}{\pgfqpoint{2.728758in}{0.826324in}}%
\pgfpathcurveto{\pgfqpoint{2.720944in}{0.818511in}}{\pgfqpoint{2.716554in}{0.807912in}}{\pgfqpoint{2.716554in}{0.796861in}}%
\pgfpathcurveto{\pgfqpoint{2.716554in}{0.785811in}}{\pgfqpoint{2.720944in}{0.775212in}}{\pgfqpoint{2.728758in}{0.767399in}}%
\pgfpathcurveto{\pgfqpoint{2.736571in}{0.759585in}}{\pgfqpoint{2.747170in}{0.755195in}}{\pgfqpoint{2.758221in}{0.755195in}}%
\pgfpathclose%
\pgfusepath{stroke,fill}%
\end{pgfscope}%
\begin{pgfscope}%
\pgfpathrectangle{\pgfqpoint{0.511823in}{0.504323in}}{\pgfqpoint{3.218177in}{3.225677in}} %
\pgfusepath{clip}%
\pgfsetbuttcap%
\pgfsetroundjoin%
\definecolor{currentfill}{rgb}{0.000000,0.000000,0.545098}%
\pgfsetfillcolor{currentfill}%
\pgfsetfillopacity{0.400000}%
\pgfsetlinewidth{0.501875pt}%
\definecolor{currentstroke}{rgb}{0.000000,0.000000,0.545098}%
\pgfsetstrokecolor{currentstroke}%
\pgfsetstrokeopacity{0.400000}%
\pgfsetdash{}{0pt}%
\pgfpathmoveto{\pgfqpoint{2.542609in}{0.748639in}}%
\pgfpathcurveto{\pgfqpoint{2.553659in}{0.748639in}}{\pgfqpoint{2.564258in}{0.753029in}}{\pgfqpoint{2.572072in}{0.760843in}}%
\pgfpathcurveto{\pgfqpoint{2.579886in}{0.768656in}}{\pgfqpoint{2.584276in}{0.779255in}}{\pgfqpoint{2.584276in}{0.790305in}}%
\pgfpathcurveto{\pgfqpoint{2.584276in}{0.801356in}}{\pgfqpoint{2.579886in}{0.811955in}}{\pgfqpoint{2.572072in}{0.819768in}}%
\pgfpathcurveto{\pgfqpoint{2.564258in}{0.827582in}}{\pgfqpoint{2.553659in}{0.831972in}}{\pgfqpoint{2.542609in}{0.831972in}}%
\pgfpathcurveto{\pgfqpoint{2.531559in}{0.831972in}}{\pgfqpoint{2.520960in}{0.827582in}}{\pgfqpoint{2.513146in}{0.819768in}}%
\pgfpathcurveto{\pgfqpoint{2.505333in}{0.811955in}}{\pgfqpoint{2.500943in}{0.801356in}}{\pgfqpoint{2.500943in}{0.790305in}}%
\pgfpathcurveto{\pgfqpoint{2.500943in}{0.779255in}}{\pgfqpoint{2.505333in}{0.768656in}}{\pgfqpoint{2.513146in}{0.760843in}}%
\pgfpathcurveto{\pgfqpoint{2.520960in}{0.753029in}}{\pgfqpoint{2.531559in}{0.748639in}}{\pgfqpoint{2.542609in}{0.748639in}}%
\pgfpathclose%
\pgfusepath{stroke,fill}%
\end{pgfscope}%
\begin{pgfscope}%
\pgfpathrectangle{\pgfqpoint{0.511823in}{0.504323in}}{\pgfqpoint{3.218177in}{3.225677in}} %
\pgfusepath{clip}%
\pgfsetbuttcap%
\pgfsetroundjoin%
\definecolor{currentfill}{rgb}{0.000000,0.000000,0.545098}%
\pgfsetfillcolor{currentfill}%
\pgfsetfillopacity{0.400000}%
\pgfsetlinewidth{0.501875pt}%
\definecolor{currentstroke}{rgb}{0.000000,0.000000,0.545098}%
\pgfsetstrokecolor{currentstroke}%
\pgfsetstrokeopacity{0.400000}%
\pgfsetdash{}{0pt}%
\pgfpathmoveto{\pgfqpoint{2.558613in}{0.755084in}}%
\pgfpathcurveto{\pgfqpoint{2.569664in}{0.755084in}}{\pgfqpoint{2.580263in}{0.759474in}}{\pgfqpoint{2.588076in}{0.767288in}}%
\pgfpathcurveto{\pgfqpoint{2.595890in}{0.775101in}}{\pgfqpoint{2.600280in}{0.785700in}}{\pgfqpoint{2.600280in}{0.796751in}}%
\pgfpathcurveto{\pgfqpoint{2.600280in}{0.807801in}}{\pgfqpoint{2.595890in}{0.818400in}}{\pgfqpoint{2.588076in}{0.826213in}}%
\pgfpathcurveto{\pgfqpoint{2.580263in}{0.834027in}}{\pgfqpoint{2.569664in}{0.838417in}}{\pgfqpoint{2.558613in}{0.838417in}}%
\pgfpathcurveto{\pgfqpoint{2.547563in}{0.838417in}}{\pgfqpoint{2.536964in}{0.834027in}}{\pgfqpoint{2.529151in}{0.826213in}}%
\pgfpathcurveto{\pgfqpoint{2.521337in}{0.818400in}}{\pgfqpoint{2.516947in}{0.807801in}}{\pgfqpoint{2.516947in}{0.796751in}}%
\pgfpathcurveto{\pgfqpoint{2.516947in}{0.785700in}}{\pgfqpoint{2.521337in}{0.775101in}}{\pgfqpoint{2.529151in}{0.767288in}}%
\pgfpathcurveto{\pgfqpoint{2.536964in}{0.759474in}}{\pgfqpoint{2.547563in}{0.755084in}}{\pgfqpoint{2.558613in}{0.755084in}}%
\pgfpathclose%
\pgfusepath{stroke,fill}%
\end{pgfscope}%
\begin{pgfscope}%
\pgfpathrectangle{\pgfqpoint{0.511823in}{0.504323in}}{\pgfqpoint{3.218177in}{3.225677in}} %
\pgfusepath{clip}%
\pgfsetbuttcap%
\pgfsetroundjoin%
\definecolor{currentfill}{rgb}{0.000000,0.000000,0.545098}%
\pgfsetfillcolor{currentfill}%
\pgfsetfillopacity{0.400000}%
\pgfsetlinewidth{0.501875pt}%
\definecolor{currentstroke}{rgb}{0.000000,0.000000,0.545098}%
\pgfsetstrokecolor{currentstroke}%
\pgfsetstrokeopacity{0.400000}%
\pgfsetdash{}{0pt}%
\pgfpathmoveto{\pgfqpoint{2.556225in}{0.760439in}}%
\pgfpathcurveto{\pgfqpoint{2.567276in}{0.760439in}}{\pgfqpoint{2.577875in}{0.764829in}}{\pgfqpoint{2.585688in}{0.772642in}}%
\pgfpathcurveto{\pgfqpoint{2.593502in}{0.780456in}}{\pgfqpoint{2.597892in}{0.791055in}}{\pgfqpoint{2.597892in}{0.802105in}}%
\pgfpathcurveto{\pgfqpoint{2.597892in}{0.813155in}}{\pgfqpoint{2.593502in}{0.823754in}}{\pgfqpoint{2.585688in}{0.831568in}}%
\pgfpathcurveto{\pgfqpoint{2.577875in}{0.839382in}}{\pgfqpoint{2.567276in}{0.843772in}}{\pgfqpoint{2.556225in}{0.843772in}}%
\pgfpathcurveto{\pgfqpoint{2.545175in}{0.843772in}}{\pgfqpoint{2.534576in}{0.839382in}}{\pgfqpoint{2.526763in}{0.831568in}}%
\pgfpathcurveto{\pgfqpoint{2.518949in}{0.823754in}}{\pgfqpoint{2.514559in}{0.813155in}}{\pgfqpoint{2.514559in}{0.802105in}}%
\pgfpathcurveto{\pgfqpoint{2.514559in}{0.791055in}}{\pgfqpoint{2.518949in}{0.780456in}}{\pgfqpoint{2.526763in}{0.772642in}}%
\pgfpathcurveto{\pgfqpoint{2.534576in}{0.764829in}}{\pgfqpoint{2.545175in}{0.760439in}}{\pgfqpoint{2.556225in}{0.760439in}}%
\pgfpathclose%
\pgfusepath{stroke,fill}%
\end{pgfscope}%
\begin{pgfscope}%
\pgfpathrectangle{\pgfqpoint{0.511823in}{0.504323in}}{\pgfqpoint{3.218177in}{3.225677in}} %
\pgfusepath{clip}%
\pgfsetbuttcap%
\pgfsetroundjoin%
\definecolor{currentfill}{rgb}{0.000000,0.000000,0.545098}%
\pgfsetfillcolor{currentfill}%
\pgfsetfillopacity{0.400000}%
\pgfsetlinewidth{0.501875pt}%
\definecolor{currentstroke}{rgb}{0.000000,0.000000,0.545098}%
\pgfsetstrokecolor{currentstroke}%
\pgfsetstrokeopacity{0.400000}%
\pgfsetdash{}{0pt}%
\pgfpathmoveto{\pgfqpoint{2.615267in}{0.769927in}}%
\pgfpathcurveto{\pgfqpoint{2.626317in}{0.769927in}}{\pgfqpoint{2.636916in}{0.774317in}}{\pgfqpoint{2.644729in}{0.782131in}}%
\pgfpathcurveto{\pgfqpoint{2.652543in}{0.789945in}}{\pgfqpoint{2.656933in}{0.800544in}}{\pgfqpoint{2.656933in}{0.811594in}}%
\pgfpathcurveto{\pgfqpoint{2.656933in}{0.822644in}}{\pgfqpoint{2.652543in}{0.833243in}}{\pgfqpoint{2.644729in}{0.841056in}}%
\pgfpathcurveto{\pgfqpoint{2.636916in}{0.848870in}}{\pgfqpoint{2.626317in}{0.853260in}}{\pgfqpoint{2.615267in}{0.853260in}}%
\pgfpathcurveto{\pgfqpoint{2.604217in}{0.853260in}}{\pgfqpoint{2.593618in}{0.848870in}}{\pgfqpoint{2.585804in}{0.841056in}}%
\pgfpathcurveto{\pgfqpoint{2.577990in}{0.833243in}}{\pgfqpoint{2.573600in}{0.822644in}}{\pgfqpoint{2.573600in}{0.811594in}}%
\pgfpathcurveto{\pgfqpoint{2.573600in}{0.800544in}}{\pgfqpoint{2.577990in}{0.789945in}}{\pgfqpoint{2.585804in}{0.782131in}}%
\pgfpathcurveto{\pgfqpoint{2.593618in}{0.774317in}}{\pgfqpoint{2.604217in}{0.769927in}}{\pgfqpoint{2.615267in}{0.769927in}}%
\pgfpathclose%
\pgfusepath{stroke,fill}%
\end{pgfscope}%
\begin{pgfscope}%
\pgfpathrectangle{\pgfqpoint{0.511823in}{0.504323in}}{\pgfqpoint{3.218177in}{3.225677in}} %
\pgfusepath{clip}%
\pgfsetbuttcap%
\pgfsetroundjoin%
\definecolor{currentfill}{rgb}{0.000000,0.000000,0.545098}%
\pgfsetfillcolor{currentfill}%
\pgfsetfillopacity{0.400000}%
\pgfsetlinewidth{0.501875pt}%
\definecolor{currentstroke}{rgb}{0.000000,0.000000,0.545098}%
\pgfsetstrokecolor{currentstroke}%
\pgfsetstrokeopacity{0.400000}%
\pgfsetdash{}{0pt}%
\pgfpathmoveto{\pgfqpoint{2.802522in}{0.788796in}}%
\pgfpathcurveto{\pgfqpoint{2.813572in}{0.788796in}}{\pgfqpoint{2.824171in}{0.793186in}}{\pgfqpoint{2.831984in}{0.801000in}}%
\pgfpathcurveto{\pgfqpoint{2.839798in}{0.808813in}}{\pgfqpoint{2.844188in}{0.819412in}}{\pgfqpoint{2.844188in}{0.830462in}}%
\pgfpathcurveto{\pgfqpoint{2.844188in}{0.841513in}}{\pgfqpoint{2.839798in}{0.852112in}}{\pgfqpoint{2.831984in}{0.859925in}}%
\pgfpathcurveto{\pgfqpoint{2.824171in}{0.867739in}}{\pgfqpoint{2.813572in}{0.872129in}}{\pgfqpoint{2.802522in}{0.872129in}}%
\pgfpathcurveto{\pgfqpoint{2.791472in}{0.872129in}}{\pgfqpoint{2.780873in}{0.867739in}}{\pgfqpoint{2.773059in}{0.859925in}}%
\pgfpathcurveto{\pgfqpoint{2.765245in}{0.852112in}}{\pgfqpoint{2.760855in}{0.841513in}}{\pgfqpoint{2.760855in}{0.830462in}}%
\pgfpathcurveto{\pgfqpoint{2.760855in}{0.819412in}}{\pgfqpoint{2.765245in}{0.808813in}}{\pgfqpoint{2.773059in}{0.801000in}}%
\pgfpathcurveto{\pgfqpoint{2.780873in}{0.793186in}}{\pgfqpoint{2.791472in}{0.788796in}}{\pgfqpoint{2.802522in}{0.788796in}}%
\pgfpathclose%
\pgfusepath{stroke,fill}%
\end{pgfscope}%
\begin{pgfscope}%
\pgfpathrectangle{\pgfqpoint{0.511823in}{0.504323in}}{\pgfqpoint{3.218177in}{3.225677in}} %
\pgfusepath{clip}%
\pgfsetbuttcap%
\pgfsetroundjoin%
\definecolor{currentfill}{rgb}{0.000000,0.000000,0.545098}%
\pgfsetfillcolor{currentfill}%
\pgfsetfillopacity{0.400000}%
\pgfsetlinewidth{0.501875pt}%
\definecolor{currentstroke}{rgb}{0.000000,0.000000,0.545098}%
\pgfsetstrokecolor{currentstroke}%
\pgfsetstrokeopacity{0.400000}%
\pgfsetdash{}{0pt}%
\pgfpathmoveto{\pgfqpoint{2.764445in}{0.792237in}}%
\pgfpathcurveto{\pgfqpoint{2.775495in}{0.792237in}}{\pgfqpoint{2.786094in}{0.796627in}}{\pgfqpoint{2.793908in}{0.804441in}}%
\pgfpathcurveto{\pgfqpoint{2.801721in}{0.812254in}}{\pgfqpoint{2.806112in}{0.822853in}}{\pgfqpoint{2.806112in}{0.833904in}}%
\pgfpathcurveto{\pgfqpoint{2.806112in}{0.844954in}}{\pgfqpoint{2.801721in}{0.855553in}}{\pgfqpoint{2.793908in}{0.863366in}}%
\pgfpathcurveto{\pgfqpoint{2.786094in}{0.871180in}}{\pgfqpoint{2.775495in}{0.875570in}}{\pgfqpoint{2.764445in}{0.875570in}}%
\pgfpathcurveto{\pgfqpoint{2.753395in}{0.875570in}}{\pgfqpoint{2.742796in}{0.871180in}}{\pgfqpoint{2.734982in}{0.863366in}}%
\pgfpathcurveto{\pgfqpoint{2.727168in}{0.855553in}}{\pgfqpoint{2.722778in}{0.844954in}}{\pgfqpoint{2.722778in}{0.833904in}}%
\pgfpathcurveto{\pgfqpoint{2.722778in}{0.822853in}}{\pgfqpoint{2.727168in}{0.812254in}}{\pgfqpoint{2.734982in}{0.804441in}}%
\pgfpathcurveto{\pgfqpoint{2.742796in}{0.796627in}}{\pgfqpoint{2.753395in}{0.792237in}}{\pgfqpoint{2.764445in}{0.792237in}}%
\pgfpathclose%
\pgfusepath{stroke,fill}%
\end{pgfscope}%
\begin{pgfscope}%
\pgfpathrectangle{\pgfqpoint{0.511823in}{0.504323in}}{\pgfqpoint{3.218177in}{3.225677in}} %
\pgfusepath{clip}%
\pgfsetbuttcap%
\pgfsetroundjoin%
\definecolor{currentfill}{rgb}{0.000000,0.000000,0.545098}%
\pgfsetfillcolor{currentfill}%
\pgfsetfillopacity{0.400000}%
\pgfsetlinewidth{0.501875pt}%
\definecolor{currentstroke}{rgb}{0.000000,0.000000,0.545098}%
\pgfsetstrokecolor{currentstroke}%
\pgfsetstrokeopacity{0.400000}%
\pgfsetdash{}{0pt}%
\pgfpathmoveto{\pgfqpoint{2.677543in}{0.791730in}}%
\pgfpathcurveto{\pgfqpoint{2.688593in}{0.791730in}}{\pgfqpoint{2.699192in}{0.796120in}}{\pgfqpoint{2.707006in}{0.803934in}}%
\pgfpathcurveto{\pgfqpoint{2.714820in}{0.811748in}}{\pgfqpoint{2.719210in}{0.822347in}}{\pgfqpoint{2.719210in}{0.833397in}}%
\pgfpathcurveto{\pgfqpoint{2.719210in}{0.844447in}}{\pgfqpoint{2.714820in}{0.855046in}}{\pgfqpoint{2.707006in}{0.862859in}}%
\pgfpathcurveto{\pgfqpoint{2.699192in}{0.870673in}}{\pgfqpoint{2.688593in}{0.875063in}}{\pgfqpoint{2.677543in}{0.875063in}}%
\pgfpathcurveto{\pgfqpoint{2.666493in}{0.875063in}}{\pgfqpoint{2.655894in}{0.870673in}}{\pgfqpoint{2.648080in}{0.862859in}}%
\pgfpathcurveto{\pgfqpoint{2.640267in}{0.855046in}}{\pgfqpoint{2.635876in}{0.844447in}}{\pgfqpoint{2.635876in}{0.833397in}}%
\pgfpathcurveto{\pgfqpoint{2.635876in}{0.822347in}}{\pgfqpoint{2.640267in}{0.811748in}}{\pgfqpoint{2.648080in}{0.803934in}}%
\pgfpathcurveto{\pgfqpoint{2.655894in}{0.796120in}}{\pgfqpoint{2.666493in}{0.791730in}}{\pgfqpoint{2.677543in}{0.791730in}}%
\pgfpathclose%
\pgfusepath{stroke,fill}%
\end{pgfscope}%
\begin{pgfscope}%
\pgfpathrectangle{\pgfqpoint{0.511823in}{0.504323in}}{\pgfqpoint{3.218177in}{3.225677in}} %
\pgfusepath{clip}%
\pgfsetbuttcap%
\pgfsetroundjoin%
\definecolor{currentfill}{rgb}{0.000000,0.000000,0.545098}%
\pgfsetfillcolor{currentfill}%
\pgfsetfillopacity{0.400000}%
\pgfsetlinewidth{0.501875pt}%
\definecolor{currentstroke}{rgb}{0.000000,0.000000,0.545098}%
\pgfsetstrokecolor{currentstroke}%
\pgfsetstrokeopacity{0.400000}%
\pgfsetdash{}{0pt}%
\pgfpathmoveto{\pgfqpoint{2.621177in}{0.793134in}}%
\pgfpathcurveto{\pgfqpoint{2.632227in}{0.793134in}}{\pgfqpoint{2.642826in}{0.797525in}}{\pgfqpoint{2.650640in}{0.805338in}}%
\pgfpathcurveto{\pgfqpoint{2.658454in}{0.813152in}}{\pgfqpoint{2.662844in}{0.823751in}}{\pgfqpoint{2.662844in}{0.834801in}}%
\pgfpathcurveto{\pgfqpoint{2.662844in}{0.845851in}}{\pgfqpoint{2.658454in}{0.856450in}}{\pgfqpoint{2.650640in}{0.864264in}}%
\pgfpathcurveto{\pgfqpoint{2.642826in}{0.872077in}}{\pgfqpoint{2.632227in}{0.876468in}}{\pgfqpoint{2.621177in}{0.876468in}}%
\pgfpathcurveto{\pgfqpoint{2.610127in}{0.876468in}}{\pgfqpoint{2.599528in}{0.872077in}}{\pgfqpoint{2.591714in}{0.864264in}}%
\pgfpathcurveto{\pgfqpoint{2.583901in}{0.856450in}}{\pgfqpoint{2.579511in}{0.845851in}}{\pgfqpoint{2.579511in}{0.834801in}}%
\pgfpathcurveto{\pgfqpoint{2.579511in}{0.823751in}}{\pgfqpoint{2.583901in}{0.813152in}}{\pgfqpoint{2.591714in}{0.805338in}}%
\pgfpathcurveto{\pgfqpoint{2.599528in}{0.797525in}}{\pgfqpoint{2.610127in}{0.793134in}}{\pgfqpoint{2.621177in}{0.793134in}}%
\pgfpathclose%
\pgfusepath{stroke,fill}%
\end{pgfscope}%
\begin{pgfscope}%
\pgfpathrectangle{\pgfqpoint{0.511823in}{0.504323in}}{\pgfqpoint{3.218177in}{3.225677in}} %
\pgfusepath{clip}%
\pgfsetbuttcap%
\pgfsetroundjoin%
\definecolor{currentfill}{rgb}{0.000000,0.000000,0.545098}%
\pgfsetfillcolor{currentfill}%
\pgfsetfillopacity{0.400000}%
\pgfsetlinewidth{0.501875pt}%
\definecolor{currentstroke}{rgb}{0.000000,0.000000,0.545098}%
\pgfsetstrokecolor{currentstroke}%
\pgfsetstrokeopacity{0.400000}%
\pgfsetdash{}{0pt}%
\pgfpathmoveto{\pgfqpoint{2.669094in}{0.802784in}}%
\pgfpathcurveto{\pgfqpoint{2.680144in}{0.802784in}}{\pgfqpoint{2.690743in}{0.807174in}}{\pgfqpoint{2.698557in}{0.814988in}}%
\pgfpathcurveto{\pgfqpoint{2.706371in}{0.822801in}}{\pgfqpoint{2.710761in}{0.833400in}}{\pgfqpoint{2.710761in}{0.844450in}}%
\pgfpathcurveto{\pgfqpoint{2.710761in}{0.855501in}}{\pgfqpoint{2.706371in}{0.866100in}}{\pgfqpoint{2.698557in}{0.873913in}}%
\pgfpathcurveto{\pgfqpoint{2.690743in}{0.881727in}}{\pgfqpoint{2.680144in}{0.886117in}}{\pgfqpoint{2.669094in}{0.886117in}}%
\pgfpathcurveto{\pgfqpoint{2.658044in}{0.886117in}}{\pgfqpoint{2.647445in}{0.881727in}}{\pgfqpoint{2.639631in}{0.873913in}}%
\pgfpathcurveto{\pgfqpoint{2.631818in}{0.866100in}}{\pgfqpoint{2.627428in}{0.855501in}}{\pgfqpoint{2.627428in}{0.844450in}}%
\pgfpathcurveto{\pgfqpoint{2.627428in}{0.833400in}}{\pgfqpoint{2.631818in}{0.822801in}}{\pgfqpoint{2.639631in}{0.814988in}}%
\pgfpathcurveto{\pgfqpoint{2.647445in}{0.807174in}}{\pgfqpoint{2.658044in}{0.802784in}}{\pgfqpoint{2.669094in}{0.802784in}}%
\pgfpathclose%
\pgfusepath{stroke,fill}%
\end{pgfscope}%
\begin{pgfscope}%
\pgfpathrectangle{\pgfqpoint{0.511823in}{0.504323in}}{\pgfqpoint{3.218177in}{3.225677in}} %
\pgfusepath{clip}%
\pgfsetbuttcap%
\pgfsetroundjoin%
\definecolor{currentfill}{rgb}{0.000000,0.000000,0.545098}%
\pgfsetfillcolor{currentfill}%
\pgfsetfillopacity{0.400000}%
\pgfsetlinewidth{0.501875pt}%
\definecolor{currentstroke}{rgb}{0.000000,0.000000,0.545098}%
\pgfsetstrokecolor{currentstroke}%
\pgfsetstrokeopacity{0.400000}%
\pgfsetdash{}{0pt}%
\pgfpathmoveto{\pgfqpoint{2.654104in}{0.807361in}}%
\pgfpathcurveto{\pgfqpoint{2.665154in}{0.807361in}}{\pgfqpoint{2.675753in}{0.811751in}}{\pgfqpoint{2.683567in}{0.819565in}}%
\pgfpathcurveto{\pgfqpoint{2.691380in}{0.827379in}}{\pgfqpoint{2.695770in}{0.837978in}}{\pgfqpoint{2.695770in}{0.849028in}}%
\pgfpathcurveto{\pgfqpoint{2.695770in}{0.860078in}}{\pgfqpoint{2.691380in}{0.870677in}}{\pgfqpoint{2.683567in}{0.878491in}}%
\pgfpathcurveto{\pgfqpoint{2.675753in}{0.886304in}}{\pgfqpoint{2.665154in}{0.890694in}}{\pgfqpoint{2.654104in}{0.890694in}}%
\pgfpathcurveto{\pgfqpoint{2.643054in}{0.890694in}}{\pgfqpoint{2.632455in}{0.886304in}}{\pgfqpoint{2.624641in}{0.878491in}}%
\pgfpathcurveto{\pgfqpoint{2.616827in}{0.870677in}}{\pgfqpoint{2.612437in}{0.860078in}}{\pgfqpoint{2.612437in}{0.849028in}}%
\pgfpathcurveto{\pgfqpoint{2.612437in}{0.837978in}}{\pgfqpoint{2.616827in}{0.827379in}}{\pgfqpoint{2.624641in}{0.819565in}}%
\pgfpathcurveto{\pgfqpoint{2.632455in}{0.811751in}}{\pgfqpoint{2.643054in}{0.807361in}}{\pgfqpoint{2.654104in}{0.807361in}}%
\pgfpathclose%
\pgfusepath{stroke,fill}%
\end{pgfscope}%
\begin{pgfscope}%
\pgfpathrectangle{\pgfqpoint{0.511823in}{0.504323in}}{\pgfqpoint{3.218177in}{3.225677in}} %
\pgfusepath{clip}%
\pgfsetbuttcap%
\pgfsetroundjoin%
\definecolor{currentfill}{rgb}{0.000000,0.000000,0.545098}%
\pgfsetfillcolor{currentfill}%
\pgfsetfillopacity{0.400000}%
\pgfsetlinewidth{0.501875pt}%
\definecolor{currentstroke}{rgb}{0.000000,0.000000,0.545098}%
\pgfsetstrokecolor{currentstroke}%
\pgfsetstrokeopacity{0.400000}%
\pgfsetdash{}{0pt}%
\pgfpathmoveto{\pgfqpoint{2.772258in}{0.823587in}}%
\pgfpathcurveto{\pgfqpoint{2.783308in}{0.823587in}}{\pgfqpoint{2.793907in}{0.827978in}}{\pgfqpoint{2.801720in}{0.835791in}}%
\pgfpathcurveto{\pgfqpoint{2.809534in}{0.843605in}}{\pgfqpoint{2.813924in}{0.854204in}}{\pgfqpoint{2.813924in}{0.865254in}}%
\pgfpathcurveto{\pgfqpoint{2.813924in}{0.876304in}}{\pgfqpoint{2.809534in}{0.886903in}}{\pgfqpoint{2.801720in}{0.894717in}}%
\pgfpathcurveto{\pgfqpoint{2.793907in}{0.902530in}}{\pgfqpoint{2.783308in}{0.906921in}}{\pgfqpoint{2.772258in}{0.906921in}}%
\pgfpathcurveto{\pgfqpoint{2.761207in}{0.906921in}}{\pgfqpoint{2.750608in}{0.902530in}}{\pgfqpoint{2.742795in}{0.894717in}}%
\pgfpathcurveto{\pgfqpoint{2.734981in}{0.886903in}}{\pgfqpoint{2.730591in}{0.876304in}}{\pgfqpoint{2.730591in}{0.865254in}}%
\pgfpathcurveto{\pgfqpoint{2.730591in}{0.854204in}}{\pgfqpoint{2.734981in}{0.843605in}}{\pgfqpoint{2.742795in}{0.835791in}}%
\pgfpathcurveto{\pgfqpoint{2.750608in}{0.827978in}}{\pgfqpoint{2.761207in}{0.823587in}}{\pgfqpoint{2.772258in}{0.823587in}}%
\pgfpathclose%
\pgfusepath{stroke,fill}%
\end{pgfscope}%
\begin{pgfscope}%
\pgfpathrectangle{\pgfqpoint{0.511823in}{0.504323in}}{\pgfqpoint{3.218177in}{3.225677in}} %
\pgfusepath{clip}%
\pgfsetbuttcap%
\pgfsetroundjoin%
\definecolor{currentfill}{rgb}{0.000000,0.000000,0.545098}%
\pgfsetfillcolor{currentfill}%
\pgfsetfillopacity{0.400000}%
\pgfsetlinewidth{0.501875pt}%
\definecolor{currentstroke}{rgb}{0.000000,0.000000,0.545098}%
\pgfsetstrokecolor{currentstroke}%
\pgfsetstrokeopacity{0.400000}%
\pgfsetdash{}{0pt}%
\pgfpathmoveto{\pgfqpoint{2.583529in}{0.812563in}}%
\pgfpathcurveto{\pgfqpoint{2.594580in}{0.812563in}}{\pgfqpoint{2.605179in}{0.816953in}}{\pgfqpoint{2.612992in}{0.824766in}}%
\pgfpathcurveto{\pgfqpoint{2.620806in}{0.832580in}}{\pgfqpoint{2.625196in}{0.843179in}}{\pgfqpoint{2.625196in}{0.854229in}}%
\pgfpathcurveto{\pgfqpoint{2.625196in}{0.865279in}}{\pgfqpoint{2.620806in}{0.875878in}}{\pgfqpoint{2.612992in}{0.883692in}}%
\pgfpathcurveto{\pgfqpoint{2.605179in}{0.891506in}}{\pgfqpoint{2.594580in}{0.895896in}}{\pgfqpoint{2.583529in}{0.895896in}}%
\pgfpathcurveto{\pgfqpoint{2.572479in}{0.895896in}}{\pgfqpoint{2.561880in}{0.891506in}}{\pgfqpoint{2.554067in}{0.883692in}}%
\pgfpathcurveto{\pgfqpoint{2.546253in}{0.875878in}}{\pgfqpoint{2.541863in}{0.865279in}}{\pgfqpoint{2.541863in}{0.854229in}}%
\pgfpathcurveto{\pgfqpoint{2.541863in}{0.843179in}}{\pgfqpoint{2.546253in}{0.832580in}}{\pgfqpoint{2.554067in}{0.824766in}}%
\pgfpathcurveto{\pgfqpoint{2.561880in}{0.816953in}}{\pgfqpoint{2.572479in}{0.812563in}}{\pgfqpoint{2.583529in}{0.812563in}}%
\pgfpathclose%
\pgfusepath{stroke,fill}%
\end{pgfscope}%
\begin{pgfscope}%
\pgfpathrectangle{\pgfqpoint{0.511823in}{0.504323in}}{\pgfqpoint{3.218177in}{3.225677in}} %
\pgfusepath{clip}%
\pgfsetbuttcap%
\pgfsetroundjoin%
\definecolor{currentfill}{rgb}{0.000000,0.000000,0.545098}%
\pgfsetfillcolor{currentfill}%
\pgfsetfillopacity{0.400000}%
\pgfsetlinewidth{0.501875pt}%
\definecolor{currentstroke}{rgb}{0.000000,0.000000,0.545098}%
\pgfsetstrokecolor{currentstroke}%
\pgfsetstrokeopacity{0.400000}%
\pgfsetdash{}{0pt}%
\pgfpathmoveto{\pgfqpoint{2.665479in}{0.825881in}}%
\pgfpathcurveto{\pgfqpoint{2.676529in}{0.825881in}}{\pgfqpoint{2.687128in}{0.830271in}}{\pgfqpoint{2.694942in}{0.838085in}}%
\pgfpathcurveto{\pgfqpoint{2.702756in}{0.845899in}}{\pgfqpoint{2.707146in}{0.856498in}}{\pgfqpoint{2.707146in}{0.867548in}}%
\pgfpathcurveto{\pgfqpoint{2.707146in}{0.878598in}}{\pgfqpoint{2.702756in}{0.889197in}}{\pgfqpoint{2.694942in}{0.897011in}}%
\pgfpathcurveto{\pgfqpoint{2.687128in}{0.904824in}}{\pgfqpoint{2.676529in}{0.909214in}}{\pgfqpoint{2.665479in}{0.909214in}}%
\pgfpathcurveto{\pgfqpoint{2.654429in}{0.909214in}}{\pgfqpoint{2.643830in}{0.904824in}}{\pgfqpoint{2.636016in}{0.897011in}}%
\pgfpathcurveto{\pgfqpoint{2.628203in}{0.889197in}}{\pgfqpoint{2.623813in}{0.878598in}}{\pgfqpoint{2.623813in}{0.867548in}}%
\pgfpathcurveto{\pgfqpoint{2.623813in}{0.856498in}}{\pgfqpoint{2.628203in}{0.845899in}}{\pgfqpoint{2.636016in}{0.838085in}}%
\pgfpathcurveto{\pgfqpoint{2.643830in}{0.830271in}}{\pgfqpoint{2.654429in}{0.825881in}}{\pgfqpoint{2.665479in}{0.825881in}}%
\pgfpathclose%
\pgfusepath{stroke,fill}%
\end{pgfscope}%
\begin{pgfscope}%
\pgfpathrectangle{\pgfqpoint{0.511823in}{0.504323in}}{\pgfqpoint{3.218177in}{3.225677in}} %
\pgfusepath{clip}%
\pgfsetbuttcap%
\pgfsetroundjoin%
\definecolor{currentfill}{rgb}{0.000000,0.000000,0.545098}%
\pgfsetfillcolor{currentfill}%
\pgfsetfillopacity{0.400000}%
\pgfsetlinewidth{0.501875pt}%
\definecolor{currentstroke}{rgb}{0.000000,0.000000,0.545098}%
\pgfsetstrokecolor{currentstroke}%
\pgfsetstrokeopacity{0.400000}%
\pgfsetdash{}{0pt}%
\pgfpathmoveto{\pgfqpoint{2.652321in}{0.830462in}}%
\pgfpathcurveto{\pgfqpoint{2.663372in}{0.830462in}}{\pgfqpoint{2.673971in}{0.834853in}}{\pgfqpoint{2.681784in}{0.842666in}}%
\pgfpathcurveto{\pgfqpoint{2.689598in}{0.850480in}}{\pgfqpoint{2.693988in}{0.861079in}}{\pgfqpoint{2.693988in}{0.872129in}}%
\pgfpathcurveto{\pgfqpoint{2.693988in}{0.883179in}}{\pgfqpoint{2.689598in}{0.893778in}}{\pgfqpoint{2.681784in}{0.901592in}}%
\pgfpathcurveto{\pgfqpoint{2.673971in}{0.909405in}}{\pgfqpoint{2.663372in}{0.913796in}}{\pgfqpoint{2.652321in}{0.913796in}}%
\pgfpathcurveto{\pgfqpoint{2.641271in}{0.913796in}}{\pgfqpoint{2.630672in}{0.909405in}}{\pgfqpoint{2.622859in}{0.901592in}}%
\pgfpathcurveto{\pgfqpoint{2.615045in}{0.893778in}}{\pgfqpoint{2.610655in}{0.883179in}}{\pgfqpoint{2.610655in}{0.872129in}}%
\pgfpathcurveto{\pgfqpoint{2.610655in}{0.861079in}}{\pgfqpoint{2.615045in}{0.850480in}}{\pgfqpoint{2.622859in}{0.842666in}}%
\pgfpathcurveto{\pgfqpoint{2.630672in}{0.834853in}}{\pgfqpoint{2.641271in}{0.830462in}}{\pgfqpoint{2.652321in}{0.830462in}}%
\pgfpathclose%
\pgfusepath{stroke,fill}%
\end{pgfscope}%
\begin{pgfscope}%
\pgfpathrectangle{\pgfqpoint{0.511823in}{0.504323in}}{\pgfqpoint{3.218177in}{3.225677in}} %
\pgfusepath{clip}%
\pgfsetbuttcap%
\pgfsetroundjoin%
\definecolor{currentfill}{rgb}{0.000000,0.000000,0.545098}%
\pgfsetfillcolor{currentfill}%
\pgfsetfillopacity{0.400000}%
\pgfsetlinewidth{0.501875pt}%
\definecolor{currentstroke}{rgb}{0.000000,0.000000,0.545098}%
\pgfsetstrokecolor{currentstroke}%
\pgfsetstrokeopacity{0.400000}%
\pgfsetdash{}{0pt}%
\pgfpathmoveto{\pgfqpoint{2.658098in}{0.836862in}}%
\pgfpathcurveto{\pgfqpoint{2.669148in}{0.836862in}}{\pgfqpoint{2.679747in}{0.841253in}}{\pgfqpoint{2.687561in}{0.849066in}}%
\pgfpathcurveto{\pgfqpoint{2.695375in}{0.856880in}}{\pgfqpoint{2.699765in}{0.867479in}}{\pgfqpoint{2.699765in}{0.878529in}}%
\pgfpathcurveto{\pgfqpoint{2.699765in}{0.889579in}}{\pgfqpoint{2.695375in}{0.900178in}}{\pgfqpoint{2.687561in}{0.907992in}}%
\pgfpathcurveto{\pgfqpoint{2.679747in}{0.915805in}}{\pgfqpoint{2.669148in}{0.920196in}}{\pgfqpoint{2.658098in}{0.920196in}}%
\pgfpathcurveto{\pgfqpoint{2.647048in}{0.920196in}}{\pgfqpoint{2.636449in}{0.915805in}}{\pgfqpoint{2.628635in}{0.907992in}}%
\pgfpathcurveto{\pgfqpoint{2.620822in}{0.900178in}}{\pgfqpoint{2.616431in}{0.889579in}}{\pgfqpoint{2.616431in}{0.878529in}}%
\pgfpathcurveto{\pgfqpoint{2.616431in}{0.867479in}}{\pgfqpoint{2.620822in}{0.856880in}}{\pgfqpoint{2.628635in}{0.849066in}}%
\pgfpathcurveto{\pgfqpoint{2.636449in}{0.841253in}}{\pgfqpoint{2.647048in}{0.836862in}}{\pgfqpoint{2.658098in}{0.836862in}}%
\pgfpathclose%
\pgfusepath{stroke,fill}%
\end{pgfscope}%
\begin{pgfscope}%
\pgfpathrectangle{\pgfqpoint{0.511823in}{0.504323in}}{\pgfqpoint{3.218177in}{3.225677in}} %
\pgfusepath{clip}%
\pgfsetbuttcap%
\pgfsetroundjoin%
\definecolor{currentfill}{rgb}{0.000000,0.000000,0.545098}%
\pgfsetfillcolor{currentfill}%
\pgfsetfillopacity{0.400000}%
\pgfsetlinewidth{0.501875pt}%
\definecolor{currentstroke}{rgb}{0.000000,0.000000,0.545098}%
\pgfsetstrokecolor{currentstroke}%
\pgfsetstrokeopacity{0.400000}%
\pgfsetdash{}{0pt}%
\pgfpathmoveto{\pgfqpoint{2.697180in}{0.846729in}}%
\pgfpathcurveto{\pgfqpoint{2.708230in}{0.846729in}}{\pgfqpoint{2.718829in}{0.851119in}}{\pgfqpoint{2.726643in}{0.858933in}}%
\pgfpathcurveto{\pgfqpoint{2.734457in}{0.866746in}}{\pgfqpoint{2.738847in}{0.877345in}}{\pgfqpoint{2.738847in}{0.888395in}}%
\pgfpathcurveto{\pgfqpoint{2.738847in}{0.899446in}}{\pgfqpoint{2.734457in}{0.910045in}}{\pgfqpoint{2.726643in}{0.917858in}}%
\pgfpathcurveto{\pgfqpoint{2.718829in}{0.925672in}}{\pgfqpoint{2.708230in}{0.930062in}}{\pgfqpoint{2.697180in}{0.930062in}}%
\pgfpathcurveto{\pgfqpoint{2.686130in}{0.930062in}}{\pgfqpoint{2.675531in}{0.925672in}}{\pgfqpoint{2.667718in}{0.917858in}}%
\pgfpathcurveto{\pgfqpoint{2.659904in}{0.910045in}}{\pgfqpoint{2.655514in}{0.899446in}}{\pgfqpoint{2.655514in}{0.888395in}}%
\pgfpathcurveto{\pgfqpoint{2.655514in}{0.877345in}}{\pgfqpoint{2.659904in}{0.866746in}}{\pgfqpoint{2.667718in}{0.858933in}}%
\pgfpathcurveto{\pgfqpoint{2.675531in}{0.851119in}}{\pgfqpoint{2.686130in}{0.846729in}}{\pgfqpoint{2.697180in}{0.846729in}}%
\pgfpathclose%
\pgfusepath{stroke,fill}%
\end{pgfscope}%
\begin{pgfscope}%
\pgfpathrectangle{\pgfqpoint{0.511823in}{0.504323in}}{\pgfqpoint{3.218177in}{3.225677in}} %
\pgfusepath{clip}%
\pgfsetbuttcap%
\pgfsetroundjoin%
\definecolor{currentfill}{rgb}{0.000000,0.000000,0.545098}%
\pgfsetfillcolor{currentfill}%
\pgfsetfillopacity{0.400000}%
\pgfsetlinewidth{0.501875pt}%
\definecolor{currentstroke}{rgb}{0.000000,0.000000,0.545098}%
\pgfsetstrokecolor{currentstroke}%
\pgfsetstrokeopacity{0.400000}%
\pgfsetdash{}{0pt}%
\pgfpathmoveto{\pgfqpoint{2.689674in}{0.851897in}}%
\pgfpathcurveto{\pgfqpoint{2.700724in}{0.851897in}}{\pgfqpoint{2.711323in}{0.856288in}}{\pgfqpoint{2.719137in}{0.864101in}}%
\pgfpathcurveto{\pgfqpoint{2.726951in}{0.871915in}}{\pgfqpoint{2.731341in}{0.882514in}}{\pgfqpoint{2.731341in}{0.893564in}}%
\pgfpathcurveto{\pgfqpoint{2.731341in}{0.904614in}}{\pgfqpoint{2.726951in}{0.915213in}}{\pgfqpoint{2.719137in}{0.923027in}}%
\pgfpathcurveto{\pgfqpoint{2.711323in}{0.930841in}}{\pgfqpoint{2.700724in}{0.935231in}}{\pgfqpoint{2.689674in}{0.935231in}}%
\pgfpathcurveto{\pgfqpoint{2.678624in}{0.935231in}}{\pgfqpoint{2.668025in}{0.930841in}}{\pgfqpoint{2.660211in}{0.923027in}}%
\pgfpathcurveto{\pgfqpoint{2.652398in}{0.915213in}}{\pgfqpoint{2.648008in}{0.904614in}}{\pgfqpoint{2.648008in}{0.893564in}}%
\pgfpathcurveto{\pgfqpoint{2.648008in}{0.882514in}}{\pgfqpoint{2.652398in}{0.871915in}}{\pgfqpoint{2.660211in}{0.864101in}}%
\pgfpathcurveto{\pgfqpoint{2.668025in}{0.856288in}}{\pgfqpoint{2.678624in}{0.851897in}}{\pgfqpoint{2.689674in}{0.851897in}}%
\pgfpathclose%
\pgfusepath{stroke,fill}%
\end{pgfscope}%
\begin{pgfscope}%
\pgfpathrectangle{\pgfqpoint{0.511823in}{0.504323in}}{\pgfqpoint{3.218177in}{3.225677in}} %
\pgfusepath{clip}%
\pgfsetbuttcap%
\pgfsetroundjoin%
\definecolor{currentfill}{rgb}{0.000000,0.000000,0.545098}%
\pgfsetfillcolor{currentfill}%
\pgfsetfillopacity{0.400000}%
\pgfsetlinewidth{0.501875pt}%
\definecolor{currentstroke}{rgb}{0.000000,0.000000,0.545098}%
\pgfsetstrokecolor{currentstroke}%
\pgfsetstrokeopacity{0.400000}%
\pgfsetdash{}{0pt}%
\pgfpathmoveto{\pgfqpoint{2.838787in}{0.874076in}}%
\pgfpathcurveto{\pgfqpoint{2.849837in}{0.874076in}}{\pgfqpoint{2.860437in}{0.878466in}}{\pgfqpoint{2.868250in}{0.886280in}}%
\pgfpathcurveto{\pgfqpoint{2.876064in}{0.894094in}}{\pgfqpoint{2.880454in}{0.904693in}}{\pgfqpoint{2.880454in}{0.915743in}}%
\pgfpathcurveto{\pgfqpoint{2.880454in}{0.926793in}}{\pgfqpoint{2.876064in}{0.937392in}}{\pgfqpoint{2.868250in}{0.945206in}}%
\pgfpathcurveto{\pgfqpoint{2.860437in}{0.953019in}}{\pgfqpoint{2.849837in}{0.957410in}}{\pgfqpoint{2.838787in}{0.957410in}}%
\pgfpathcurveto{\pgfqpoint{2.827737in}{0.957410in}}{\pgfqpoint{2.817138in}{0.953019in}}{\pgfqpoint{2.809325in}{0.945206in}}%
\pgfpathcurveto{\pgfqpoint{2.801511in}{0.937392in}}{\pgfqpoint{2.797121in}{0.926793in}}{\pgfqpoint{2.797121in}{0.915743in}}%
\pgfpathcurveto{\pgfqpoint{2.797121in}{0.904693in}}{\pgfqpoint{2.801511in}{0.894094in}}{\pgfqpoint{2.809325in}{0.886280in}}%
\pgfpathcurveto{\pgfqpoint{2.817138in}{0.878466in}}{\pgfqpoint{2.827737in}{0.874076in}}{\pgfqpoint{2.838787in}{0.874076in}}%
\pgfpathclose%
\pgfusepath{stroke,fill}%
\end{pgfscope}%
\begin{pgfscope}%
\pgfpathrectangle{\pgfqpoint{0.511823in}{0.504323in}}{\pgfqpoint{3.218177in}{3.225677in}} %
\pgfusepath{clip}%
\pgfsetbuttcap%
\pgfsetroundjoin%
\definecolor{currentfill}{rgb}{0.000000,0.000000,0.545098}%
\pgfsetfillcolor{currentfill}%
\pgfsetfillopacity{0.400000}%
\pgfsetlinewidth{0.501875pt}%
\definecolor{currentstroke}{rgb}{0.000000,0.000000,0.545098}%
\pgfsetstrokecolor{currentstroke}%
\pgfsetstrokeopacity{0.400000}%
\pgfsetdash{}{0pt}%
\pgfpathmoveto{\pgfqpoint{2.650722in}{0.859437in}}%
\pgfpathcurveto{\pgfqpoint{2.661772in}{0.859437in}}{\pgfqpoint{2.672371in}{0.863827in}}{\pgfqpoint{2.680184in}{0.871641in}}%
\pgfpathcurveto{\pgfqpoint{2.687998in}{0.879454in}}{\pgfqpoint{2.692388in}{0.890053in}}{\pgfqpoint{2.692388in}{0.901103in}}%
\pgfpathcurveto{\pgfqpoint{2.692388in}{0.912154in}}{\pgfqpoint{2.687998in}{0.922753in}}{\pgfqpoint{2.680184in}{0.930566in}}%
\pgfpathcurveto{\pgfqpoint{2.672371in}{0.938380in}}{\pgfqpoint{2.661772in}{0.942770in}}{\pgfqpoint{2.650722in}{0.942770in}}%
\pgfpathcurveto{\pgfqpoint{2.639671in}{0.942770in}}{\pgfqpoint{2.629072in}{0.938380in}}{\pgfqpoint{2.621259in}{0.930566in}}%
\pgfpathcurveto{\pgfqpoint{2.613445in}{0.922753in}}{\pgfqpoint{2.609055in}{0.912154in}}{\pgfqpoint{2.609055in}{0.901103in}}%
\pgfpathcurveto{\pgfqpoint{2.609055in}{0.890053in}}{\pgfqpoint{2.613445in}{0.879454in}}{\pgfqpoint{2.621259in}{0.871641in}}%
\pgfpathcurveto{\pgfqpoint{2.629072in}{0.863827in}}{\pgfqpoint{2.639671in}{0.859437in}}{\pgfqpoint{2.650722in}{0.859437in}}%
\pgfpathclose%
\pgfusepath{stroke,fill}%
\end{pgfscope}%
\begin{pgfscope}%
\pgfpathrectangle{\pgfqpoint{0.511823in}{0.504323in}}{\pgfqpoint{3.218177in}{3.225677in}} %
\pgfusepath{clip}%
\pgfsetbuttcap%
\pgfsetroundjoin%
\definecolor{currentfill}{rgb}{0.000000,0.000000,0.545098}%
\pgfsetfillcolor{currentfill}%
\pgfsetfillopacity{0.400000}%
\pgfsetlinewidth{0.501875pt}%
\definecolor{currentstroke}{rgb}{0.000000,0.000000,0.545098}%
\pgfsetstrokecolor{currentstroke}%
\pgfsetstrokeopacity{0.400000}%
\pgfsetdash{}{0pt}%
\pgfpathmoveto{\pgfqpoint{2.867005in}{0.890106in}}%
\pgfpathcurveto{\pgfqpoint{2.878056in}{0.890106in}}{\pgfqpoint{2.888655in}{0.894497in}}{\pgfqpoint{2.896468in}{0.902310in}}%
\pgfpathcurveto{\pgfqpoint{2.904282in}{0.910124in}}{\pgfqpoint{2.908672in}{0.920723in}}{\pgfqpoint{2.908672in}{0.931773in}}%
\pgfpathcurveto{\pgfqpoint{2.908672in}{0.942823in}}{\pgfqpoint{2.904282in}{0.953422in}}{\pgfqpoint{2.896468in}{0.961236in}}%
\pgfpathcurveto{\pgfqpoint{2.888655in}{0.969049in}}{\pgfqpoint{2.878056in}{0.973440in}}{\pgfqpoint{2.867005in}{0.973440in}}%
\pgfpathcurveto{\pgfqpoint{2.855955in}{0.973440in}}{\pgfqpoint{2.845356in}{0.969049in}}{\pgfqpoint{2.837543in}{0.961236in}}%
\pgfpathcurveto{\pgfqpoint{2.829729in}{0.953422in}}{\pgfqpoint{2.825339in}{0.942823in}}{\pgfqpoint{2.825339in}{0.931773in}}%
\pgfpathcurveto{\pgfqpoint{2.825339in}{0.920723in}}{\pgfqpoint{2.829729in}{0.910124in}}{\pgfqpoint{2.837543in}{0.902310in}}%
\pgfpathcurveto{\pgfqpoint{2.845356in}{0.894497in}}{\pgfqpoint{2.855955in}{0.890106in}}{\pgfqpoint{2.867005in}{0.890106in}}%
\pgfpathclose%
\pgfusepath{stroke,fill}%
\end{pgfscope}%
\begin{pgfscope}%
\pgfpathrectangle{\pgfqpoint{0.511823in}{0.504323in}}{\pgfqpoint{3.218177in}{3.225677in}} %
\pgfusepath{clip}%
\pgfsetbuttcap%
\pgfsetroundjoin%
\definecolor{currentfill}{rgb}{0.000000,0.000000,0.545098}%
\pgfsetfillcolor{currentfill}%
\pgfsetfillopacity{0.400000}%
\pgfsetlinewidth{0.501875pt}%
\definecolor{currentstroke}{rgb}{0.000000,0.000000,0.545098}%
\pgfsetstrokecolor{currentstroke}%
\pgfsetstrokeopacity{0.400000}%
\pgfsetdash{}{0pt}%
\pgfpathmoveto{\pgfqpoint{2.620376in}{0.867544in}}%
\pgfpathcurveto{\pgfqpoint{2.631426in}{0.867544in}}{\pgfqpoint{2.642025in}{0.871935in}}{\pgfqpoint{2.649839in}{0.879748in}}%
\pgfpathcurveto{\pgfqpoint{2.657652in}{0.887562in}}{\pgfqpoint{2.662043in}{0.898161in}}{\pgfqpoint{2.662043in}{0.909211in}}%
\pgfpathcurveto{\pgfqpoint{2.662043in}{0.920261in}}{\pgfqpoint{2.657652in}{0.930860in}}{\pgfqpoint{2.649839in}{0.938674in}}%
\pgfpathcurveto{\pgfqpoint{2.642025in}{0.946488in}}{\pgfqpoint{2.631426in}{0.950878in}}{\pgfqpoint{2.620376in}{0.950878in}}%
\pgfpathcurveto{\pgfqpoint{2.609326in}{0.950878in}}{\pgfqpoint{2.598727in}{0.946488in}}{\pgfqpoint{2.590913in}{0.938674in}}%
\pgfpathcurveto{\pgfqpoint{2.583100in}{0.930860in}}{\pgfqpoint{2.578709in}{0.920261in}}{\pgfqpoint{2.578709in}{0.909211in}}%
\pgfpathcurveto{\pgfqpoint{2.578709in}{0.898161in}}{\pgfqpoint{2.583100in}{0.887562in}}{\pgfqpoint{2.590913in}{0.879748in}}%
\pgfpathcurveto{\pgfqpoint{2.598727in}{0.871935in}}{\pgfqpoint{2.609326in}{0.867544in}}{\pgfqpoint{2.620376in}{0.867544in}}%
\pgfpathclose%
\pgfusepath{stroke,fill}%
\end{pgfscope}%
\begin{pgfscope}%
\pgfpathrectangle{\pgfqpoint{0.511823in}{0.504323in}}{\pgfqpoint{3.218177in}{3.225677in}} %
\pgfusepath{clip}%
\pgfsetbuttcap%
\pgfsetroundjoin%
\definecolor{currentfill}{rgb}{0.000000,0.000000,0.545098}%
\pgfsetfillcolor{currentfill}%
\pgfsetfillopacity{0.400000}%
\pgfsetlinewidth{0.501875pt}%
\definecolor{currentstroke}{rgb}{0.000000,0.000000,0.545098}%
\pgfsetstrokecolor{currentstroke}%
\pgfsetstrokeopacity{0.400000}%
\pgfsetdash{}{0pt}%
\pgfpathmoveto{\pgfqpoint{2.596984in}{0.870477in}}%
\pgfpathcurveto{\pgfqpoint{2.608034in}{0.870477in}}{\pgfqpoint{2.618633in}{0.874867in}}{\pgfqpoint{2.626446in}{0.882681in}}%
\pgfpathcurveto{\pgfqpoint{2.634260in}{0.890495in}}{\pgfqpoint{2.638650in}{0.901094in}}{\pgfqpoint{2.638650in}{0.912144in}}%
\pgfpathcurveto{\pgfqpoint{2.638650in}{0.923194in}}{\pgfqpoint{2.634260in}{0.933793in}}{\pgfqpoint{2.626446in}{0.941607in}}%
\pgfpathcurveto{\pgfqpoint{2.618633in}{0.949420in}}{\pgfqpoint{2.608034in}{0.953810in}}{\pgfqpoint{2.596984in}{0.953810in}}%
\pgfpathcurveto{\pgfqpoint{2.585933in}{0.953810in}}{\pgfqpoint{2.575334in}{0.949420in}}{\pgfqpoint{2.567521in}{0.941607in}}%
\pgfpathcurveto{\pgfqpoint{2.559707in}{0.933793in}}{\pgfqpoint{2.555317in}{0.923194in}}{\pgfqpoint{2.555317in}{0.912144in}}%
\pgfpathcurveto{\pgfqpoint{2.555317in}{0.901094in}}{\pgfqpoint{2.559707in}{0.890495in}}{\pgfqpoint{2.567521in}{0.882681in}}%
\pgfpathcurveto{\pgfqpoint{2.575334in}{0.874867in}}{\pgfqpoint{2.585933in}{0.870477in}}{\pgfqpoint{2.596984in}{0.870477in}}%
\pgfpathclose%
\pgfusepath{stroke,fill}%
\end{pgfscope}%
\begin{pgfscope}%
\pgfpathrectangle{\pgfqpoint{0.511823in}{0.504323in}}{\pgfqpoint{3.218177in}{3.225677in}} %
\pgfusepath{clip}%
\pgfsetbuttcap%
\pgfsetroundjoin%
\definecolor{currentfill}{rgb}{0.000000,0.000000,0.545098}%
\pgfsetfillcolor{currentfill}%
\pgfsetfillopacity{0.400000}%
\pgfsetlinewidth{0.501875pt}%
\definecolor{currentstroke}{rgb}{0.000000,0.000000,0.545098}%
\pgfsetstrokecolor{currentstroke}%
\pgfsetstrokeopacity{0.400000}%
\pgfsetdash{}{0pt}%
\pgfpathmoveto{\pgfqpoint{2.672265in}{0.885485in}}%
\pgfpathcurveto{\pgfqpoint{2.683315in}{0.885485in}}{\pgfqpoint{2.693914in}{0.889875in}}{\pgfqpoint{2.701728in}{0.897689in}}%
\pgfpathcurveto{\pgfqpoint{2.709542in}{0.905502in}}{\pgfqpoint{2.713932in}{0.916102in}}{\pgfqpoint{2.713932in}{0.927152in}}%
\pgfpathcurveto{\pgfqpoint{2.713932in}{0.938202in}}{\pgfqpoint{2.709542in}{0.948801in}}{\pgfqpoint{2.701728in}{0.956614in}}%
\pgfpathcurveto{\pgfqpoint{2.693914in}{0.964428in}}{\pgfqpoint{2.683315in}{0.968818in}}{\pgfqpoint{2.672265in}{0.968818in}}%
\pgfpathcurveto{\pgfqpoint{2.661215in}{0.968818in}}{\pgfqpoint{2.650616in}{0.964428in}}{\pgfqpoint{2.642803in}{0.956614in}}%
\pgfpathcurveto{\pgfqpoint{2.634989in}{0.948801in}}{\pgfqpoint{2.630599in}{0.938202in}}{\pgfqpoint{2.630599in}{0.927152in}}%
\pgfpathcurveto{\pgfqpoint{2.630599in}{0.916102in}}{\pgfqpoint{2.634989in}{0.905502in}}{\pgfqpoint{2.642803in}{0.897689in}}%
\pgfpathcurveto{\pgfqpoint{2.650616in}{0.889875in}}{\pgfqpoint{2.661215in}{0.885485in}}{\pgfqpoint{2.672265in}{0.885485in}}%
\pgfpathclose%
\pgfusepath{stroke,fill}%
\end{pgfscope}%
\begin{pgfscope}%
\pgfpathrectangle{\pgfqpoint{0.511823in}{0.504323in}}{\pgfqpoint{3.218177in}{3.225677in}} %
\pgfusepath{clip}%
\pgfsetbuttcap%
\pgfsetroundjoin%
\definecolor{currentfill}{rgb}{0.000000,0.000000,0.545098}%
\pgfsetfillcolor{currentfill}%
\pgfsetfillopacity{0.400000}%
\pgfsetlinewidth{0.501875pt}%
\definecolor{currentstroke}{rgb}{0.000000,0.000000,0.545098}%
\pgfsetstrokecolor{currentstroke}%
\pgfsetstrokeopacity{0.400000}%
\pgfsetdash{}{0pt}%
\pgfpathmoveto{\pgfqpoint{2.704904in}{0.895543in}}%
\pgfpathcurveto{\pgfqpoint{2.715955in}{0.895543in}}{\pgfqpoint{2.726554in}{0.899933in}}{\pgfqpoint{2.734367in}{0.907747in}}%
\pgfpathcurveto{\pgfqpoint{2.742181in}{0.915560in}}{\pgfqpoint{2.746571in}{0.926159in}}{\pgfqpoint{2.746571in}{0.937209in}}%
\pgfpathcurveto{\pgfqpoint{2.746571in}{0.948260in}}{\pgfqpoint{2.742181in}{0.958859in}}{\pgfqpoint{2.734367in}{0.966672in}}%
\pgfpathcurveto{\pgfqpoint{2.726554in}{0.974486in}}{\pgfqpoint{2.715955in}{0.978876in}}{\pgfqpoint{2.704904in}{0.978876in}}%
\pgfpathcurveto{\pgfqpoint{2.693854in}{0.978876in}}{\pgfqpoint{2.683255in}{0.974486in}}{\pgfqpoint{2.675442in}{0.966672in}}%
\pgfpathcurveto{\pgfqpoint{2.667628in}{0.958859in}}{\pgfqpoint{2.663238in}{0.948260in}}{\pgfqpoint{2.663238in}{0.937209in}}%
\pgfpathcurveto{\pgfqpoint{2.663238in}{0.926159in}}{\pgfqpoint{2.667628in}{0.915560in}}{\pgfqpoint{2.675442in}{0.907747in}}%
\pgfpathcurveto{\pgfqpoint{2.683255in}{0.899933in}}{\pgfqpoint{2.693854in}{0.895543in}}{\pgfqpoint{2.704904in}{0.895543in}}%
\pgfpathclose%
\pgfusepath{stroke,fill}%
\end{pgfscope}%
\begin{pgfscope}%
\pgfpathrectangle{\pgfqpoint{0.511823in}{0.504323in}}{\pgfqpoint{3.218177in}{3.225677in}} %
\pgfusepath{clip}%
\pgfsetbuttcap%
\pgfsetroundjoin%
\definecolor{currentfill}{rgb}{0.000000,0.000000,0.545098}%
\pgfsetfillcolor{currentfill}%
\pgfsetfillopacity{0.400000}%
\pgfsetlinewidth{0.501875pt}%
\definecolor{currentstroke}{rgb}{0.000000,0.000000,0.545098}%
\pgfsetstrokecolor{currentstroke}%
\pgfsetstrokeopacity{0.400000}%
\pgfsetdash{}{0pt}%
\pgfpathmoveto{\pgfqpoint{2.692752in}{0.899991in}}%
\pgfpathcurveto{\pgfqpoint{2.703802in}{0.899991in}}{\pgfqpoint{2.714401in}{0.904381in}}{\pgfqpoint{2.722215in}{0.912195in}}%
\pgfpathcurveto{\pgfqpoint{2.730029in}{0.920008in}}{\pgfqpoint{2.734419in}{0.930607in}}{\pgfqpoint{2.734419in}{0.941658in}}%
\pgfpathcurveto{\pgfqpoint{2.734419in}{0.952708in}}{\pgfqpoint{2.730029in}{0.963307in}}{\pgfqpoint{2.722215in}{0.971120in}}%
\pgfpathcurveto{\pgfqpoint{2.714401in}{0.978934in}}{\pgfqpoint{2.703802in}{0.983324in}}{\pgfqpoint{2.692752in}{0.983324in}}%
\pgfpathcurveto{\pgfqpoint{2.681702in}{0.983324in}}{\pgfqpoint{2.671103in}{0.978934in}}{\pgfqpoint{2.663289in}{0.971120in}}%
\pgfpathcurveto{\pgfqpoint{2.655476in}{0.963307in}}{\pgfqpoint{2.651085in}{0.952708in}}{\pgfqpoint{2.651085in}{0.941658in}}%
\pgfpathcurveto{\pgfqpoint{2.651085in}{0.930607in}}{\pgfqpoint{2.655476in}{0.920008in}}{\pgfqpoint{2.663289in}{0.912195in}}%
\pgfpathcurveto{\pgfqpoint{2.671103in}{0.904381in}}{\pgfqpoint{2.681702in}{0.899991in}}{\pgfqpoint{2.692752in}{0.899991in}}%
\pgfpathclose%
\pgfusepath{stroke,fill}%
\end{pgfscope}%
\begin{pgfscope}%
\pgfpathrectangle{\pgfqpoint{0.511823in}{0.504323in}}{\pgfqpoint{3.218177in}{3.225677in}} %
\pgfusepath{clip}%
\pgfsetbuttcap%
\pgfsetroundjoin%
\definecolor{currentfill}{rgb}{0.000000,0.000000,0.545098}%
\pgfsetfillcolor{currentfill}%
\pgfsetfillopacity{0.400000}%
\pgfsetlinewidth{0.501875pt}%
\definecolor{currentstroke}{rgb}{0.000000,0.000000,0.545098}%
\pgfsetstrokecolor{currentstroke}%
\pgfsetstrokeopacity{0.400000}%
\pgfsetdash{}{0pt}%
\pgfpathmoveto{\pgfqpoint{2.826131in}{0.923683in}}%
\pgfpathcurveto{\pgfqpoint{2.837182in}{0.923683in}}{\pgfqpoint{2.847781in}{0.928073in}}{\pgfqpoint{2.855594in}{0.935887in}}%
\pgfpathcurveto{\pgfqpoint{2.863408in}{0.943700in}}{\pgfqpoint{2.867798in}{0.954299in}}{\pgfqpoint{2.867798in}{0.965349in}}%
\pgfpathcurveto{\pgfqpoint{2.867798in}{0.976399in}}{\pgfqpoint{2.863408in}{0.986998in}}{\pgfqpoint{2.855594in}{0.994812in}}%
\pgfpathcurveto{\pgfqpoint{2.847781in}{1.002626in}}{\pgfqpoint{2.837182in}{1.007016in}}{\pgfqpoint{2.826131in}{1.007016in}}%
\pgfpathcurveto{\pgfqpoint{2.815081in}{1.007016in}}{\pgfqpoint{2.804482in}{1.002626in}}{\pgfqpoint{2.796669in}{0.994812in}}%
\pgfpathcurveto{\pgfqpoint{2.788855in}{0.986998in}}{\pgfqpoint{2.784465in}{0.976399in}}{\pgfqpoint{2.784465in}{0.965349in}}%
\pgfpathcurveto{\pgfqpoint{2.784465in}{0.954299in}}{\pgfqpoint{2.788855in}{0.943700in}}{\pgfqpoint{2.796669in}{0.935887in}}%
\pgfpathcurveto{\pgfqpoint{2.804482in}{0.928073in}}{\pgfqpoint{2.815081in}{0.923683in}}{\pgfqpoint{2.826131in}{0.923683in}}%
\pgfpathclose%
\pgfusepath{stroke,fill}%
\end{pgfscope}%
\begin{pgfscope}%
\pgfpathrectangle{\pgfqpoint{0.511823in}{0.504323in}}{\pgfqpoint{3.218177in}{3.225677in}} %
\pgfusepath{clip}%
\pgfsetbuttcap%
\pgfsetroundjoin%
\definecolor{currentfill}{rgb}{0.000000,0.000000,0.545098}%
\pgfsetfillcolor{currentfill}%
\pgfsetfillopacity{0.400000}%
\pgfsetlinewidth{0.501875pt}%
\definecolor{currentstroke}{rgb}{0.000000,0.000000,0.545098}%
\pgfsetstrokecolor{currentstroke}%
\pgfsetstrokeopacity{0.400000}%
\pgfsetdash{}{0pt}%
\pgfpathmoveto{\pgfqpoint{2.805125in}{0.927230in}}%
\pgfpathcurveto{\pgfqpoint{2.816175in}{0.927230in}}{\pgfqpoint{2.826774in}{0.931620in}}{\pgfqpoint{2.834588in}{0.939434in}}%
\pgfpathcurveto{\pgfqpoint{2.842402in}{0.947247in}}{\pgfqpoint{2.846792in}{0.957846in}}{\pgfqpoint{2.846792in}{0.968897in}}%
\pgfpathcurveto{\pgfqpoint{2.846792in}{0.979947in}}{\pgfqpoint{2.842402in}{0.990546in}}{\pgfqpoint{2.834588in}{0.998359in}}%
\pgfpathcurveto{\pgfqpoint{2.826774in}{1.006173in}}{\pgfqpoint{2.816175in}{1.010563in}}{\pgfqpoint{2.805125in}{1.010563in}}%
\pgfpathcurveto{\pgfqpoint{2.794075in}{1.010563in}}{\pgfqpoint{2.783476in}{1.006173in}}{\pgfqpoint{2.775662in}{0.998359in}}%
\pgfpathcurveto{\pgfqpoint{2.767849in}{0.990546in}}{\pgfqpoint{2.763458in}{0.979947in}}{\pgfqpoint{2.763458in}{0.968897in}}%
\pgfpathcurveto{\pgfqpoint{2.763458in}{0.957846in}}{\pgfqpoint{2.767849in}{0.947247in}}{\pgfqpoint{2.775662in}{0.939434in}}%
\pgfpathcurveto{\pgfqpoint{2.783476in}{0.931620in}}{\pgfqpoint{2.794075in}{0.927230in}}{\pgfqpoint{2.805125in}{0.927230in}}%
\pgfpathclose%
\pgfusepath{stroke,fill}%
\end{pgfscope}%
\begin{pgfscope}%
\pgfpathrectangle{\pgfqpoint{0.511823in}{0.504323in}}{\pgfqpoint{3.218177in}{3.225677in}} %
\pgfusepath{clip}%
\pgfsetbuttcap%
\pgfsetroundjoin%
\definecolor{currentfill}{rgb}{0.000000,0.000000,0.545098}%
\pgfsetfillcolor{currentfill}%
\pgfsetfillopacity{0.400000}%
\pgfsetlinewidth{0.501875pt}%
\definecolor{currentstroke}{rgb}{0.000000,0.000000,0.545098}%
\pgfsetstrokecolor{currentstroke}%
\pgfsetstrokeopacity{0.400000}%
\pgfsetdash{}{0pt}%
\pgfpathmoveto{\pgfqpoint{2.719522in}{0.921699in}}%
\pgfpathcurveto{\pgfqpoint{2.730572in}{0.921699in}}{\pgfqpoint{2.741171in}{0.926090in}}{\pgfqpoint{2.748985in}{0.933903in}}%
\pgfpathcurveto{\pgfqpoint{2.756798in}{0.941717in}}{\pgfqpoint{2.761189in}{0.952316in}}{\pgfqpoint{2.761189in}{0.963366in}}%
\pgfpathcurveto{\pgfqpoint{2.761189in}{0.974416in}}{\pgfqpoint{2.756798in}{0.985015in}}{\pgfqpoint{2.748985in}{0.992829in}}%
\pgfpathcurveto{\pgfqpoint{2.741171in}{1.000642in}}{\pgfqpoint{2.730572in}{1.005033in}}{\pgfqpoint{2.719522in}{1.005033in}}%
\pgfpathcurveto{\pgfqpoint{2.708472in}{1.005033in}}{\pgfqpoint{2.697873in}{1.000642in}}{\pgfqpoint{2.690059in}{0.992829in}}%
\pgfpathcurveto{\pgfqpoint{2.682246in}{0.985015in}}{\pgfqpoint{2.677855in}{0.974416in}}{\pgfqpoint{2.677855in}{0.963366in}}%
\pgfpathcurveto{\pgfqpoint{2.677855in}{0.952316in}}{\pgfqpoint{2.682246in}{0.941717in}}{\pgfqpoint{2.690059in}{0.933903in}}%
\pgfpathcurveto{\pgfqpoint{2.697873in}{0.926090in}}{\pgfqpoint{2.708472in}{0.921699in}}{\pgfqpoint{2.719522in}{0.921699in}}%
\pgfpathclose%
\pgfusepath{stroke,fill}%
\end{pgfscope}%
\begin{pgfscope}%
\pgfpathrectangle{\pgfqpoint{0.511823in}{0.504323in}}{\pgfqpoint{3.218177in}{3.225677in}} %
\pgfusepath{clip}%
\pgfsetbuttcap%
\pgfsetroundjoin%
\definecolor{currentfill}{rgb}{0.000000,0.000000,0.545098}%
\pgfsetfillcolor{currentfill}%
\pgfsetfillopacity{0.400000}%
\pgfsetlinewidth{0.501875pt}%
\definecolor{currentstroke}{rgb}{0.000000,0.000000,0.545098}%
\pgfsetstrokecolor{currentstroke}%
\pgfsetstrokeopacity{0.400000}%
\pgfsetdash{}{0pt}%
\pgfpathmoveto{\pgfqpoint{2.609601in}{0.912217in}}%
\pgfpathcurveto{\pgfqpoint{2.620651in}{0.912217in}}{\pgfqpoint{2.631250in}{0.916607in}}{\pgfqpoint{2.639063in}{0.924421in}}%
\pgfpathcurveto{\pgfqpoint{2.646877in}{0.932234in}}{\pgfqpoint{2.651267in}{0.942833in}}{\pgfqpoint{2.651267in}{0.953883in}}%
\pgfpathcurveto{\pgfqpoint{2.651267in}{0.964933in}}{\pgfqpoint{2.646877in}{0.975533in}}{\pgfqpoint{2.639063in}{0.983346in}}%
\pgfpathcurveto{\pgfqpoint{2.631250in}{0.991160in}}{\pgfqpoint{2.620651in}{0.995550in}}{\pgfqpoint{2.609601in}{0.995550in}}%
\pgfpathcurveto{\pgfqpoint{2.598550in}{0.995550in}}{\pgfqpoint{2.587951in}{0.991160in}}{\pgfqpoint{2.580138in}{0.983346in}}%
\pgfpathcurveto{\pgfqpoint{2.572324in}{0.975533in}}{\pgfqpoint{2.567934in}{0.964933in}}{\pgfqpoint{2.567934in}{0.953883in}}%
\pgfpathcurveto{\pgfqpoint{2.567934in}{0.942833in}}{\pgfqpoint{2.572324in}{0.932234in}}{\pgfqpoint{2.580138in}{0.924421in}}%
\pgfpathcurveto{\pgfqpoint{2.587951in}{0.916607in}}{\pgfqpoint{2.598550in}{0.912217in}}{\pgfqpoint{2.609601in}{0.912217in}}%
\pgfpathclose%
\pgfusepath{stroke,fill}%
\end{pgfscope}%
\begin{pgfscope}%
\pgfpathrectangle{\pgfqpoint{0.511823in}{0.504323in}}{\pgfqpoint{3.218177in}{3.225677in}} %
\pgfusepath{clip}%
\pgfsetbuttcap%
\pgfsetroundjoin%
\definecolor{currentfill}{rgb}{0.000000,0.000000,0.545098}%
\pgfsetfillcolor{currentfill}%
\pgfsetfillopacity{0.400000}%
\pgfsetlinewidth{0.501875pt}%
\definecolor{currentstroke}{rgb}{0.000000,0.000000,0.545098}%
\pgfsetstrokecolor{currentstroke}%
\pgfsetstrokeopacity{0.400000}%
\pgfsetdash{}{0pt}%
\pgfpathmoveto{\pgfqpoint{2.639570in}{0.922319in}}%
\pgfpathcurveto{\pgfqpoint{2.650620in}{0.922319in}}{\pgfqpoint{2.661219in}{0.926709in}}{\pgfqpoint{2.669032in}{0.934522in}}%
\pgfpathcurveto{\pgfqpoint{2.676846in}{0.942336in}}{\pgfqpoint{2.681236in}{0.952935in}}{\pgfqpoint{2.681236in}{0.963985in}}%
\pgfpathcurveto{\pgfqpoint{2.681236in}{0.975035in}}{\pgfqpoint{2.676846in}{0.985634in}}{\pgfqpoint{2.669032in}{0.993448in}}%
\pgfpathcurveto{\pgfqpoint{2.661219in}{1.001262in}}{\pgfqpoint{2.650620in}{1.005652in}}{\pgfqpoint{2.639570in}{1.005652in}}%
\pgfpathcurveto{\pgfqpoint{2.628519in}{1.005652in}}{\pgfqpoint{2.617920in}{1.001262in}}{\pgfqpoint{2.610107in}{0.993448in}}%
\pgfpathcurveto{\pgfqpoint{2.602293in}{0.985634in}}{\pgfqpoint{2.597903in}{0.975035in}}{\pgfqpoint{2.597903in}{0.963985in}}%
\pgfpathcurveto{\pgfqpoint{2.597903in}{0.952935in}}{\pgfqpoint{2.602293in}{0.942336in}}{\pgfqpoint{2.610107in}{0.934522in}}%
\pgfpathcurveto{\pgfqpoint{2.617920in}{0.926709in}}{\pgfqpoint{2.628519in}{0.922319in}}{\pgfqpoint{2.639570in}{0.922319in}}%
\pgfpathclose%
\pgfusepath{stroke,fill}%
\end{pgfscope}%
\begin{pgfscope}%
\pgfpathrectangle{\pgfqpoint{0.511823in}{0.504323in}}{\pgfqpoint{3.218177in}{3.225677in}} %
\pgfusepath{clip}%
\pgfsetbuttcap%
\pgfsetroundjoin%
\definecolor{currentfill}{rgb}{0.000000,0.000000,0.545098}%
\pgfsetfillcolor{currentfill}%
\pgfsetfillopacity{0.400000}%
\pgfsetlinewidth{0.501875pt}%
\definecolor{currentstroke}{rgb}{0.000000,0.000000,0.545098}%
\pgfsetstrokecolor{currentstroke}%
\pgfsetstrokeopacity{0.400000}%
\pgfsetdash{}{0pt}%
\pgfpathmoveto{\pgfqpoint{2.836975in}{0.957331in}}%
\pgfpathcurveto{\pgfqpoint{2.848025in}{0.957331in}}{\pgfqpoint{2.858624in}{0.961721in}}{\pgfqpoint{2.866437in}{0.969535in}}%
\pgfpathcurveto{\pgfqpoint{2.874251in}{0.977349in}}{\pgfqpoint{2.878641in}{0.987948in}}{\pgfqpoint{2.878641in}{0.998998in}}%
\pgfpathcurveto{\pgfqpoint{2.878641in}{1.010048in}}{\pgfqpoint{2.874251in}{1.020647in}}{\pgfqpoint{2.866437in}{1.028461in}}%
\pgfpathcurveto{\pgfqpoint{2.858624in}{1.036274in}}{\pgfqpoint{2.848025in}{1.040665in}}{\pgfqpoint{2.836975in}{1.040665in}}%
\pgfpathcurveto{\pgfqpoint{2.825924in}{1.040665in}}{\pgfqpoint{2.815325in}{1.036274in}}{\pgfqpoint{2.807512in}{1.028461in}}%
\pgfpathcurveto{\pgfqpoint{2.799698in}{1.020647in}}{\pgfqpoint{2.795308in}{1.010048in}}{\pgfqpoint{2.795308in}{0.998998in}}%
\pgfpathcurveto{\pgfqpoint{2.795308in}{0.987948in}}{\pgfqpoint{2.799698in}{0.977349in}}{\pgfqpoint{2.807512in}{0.969535in}}%
\pgfpathcurveto{\pgfqpoint{2.815325in}{0.961721in}}{\pgfqpoint{2.825924in}{0.957331in}}{\pgfqpoint{2.836975in}{0.957331in}}%
\pgfpathclose%
\pgfusepath{stroke,fill}%
\end{pgfscope}%
\begin{pgfscope}%
\pgfpathrectangle{\pgfqpoint{0.511823in}{0.504323in}}{\pgfqpoint{3.218177in}{3.225677in}} %
\pgfusepath{clip}%
\pgfsetbuttcap%
\pgfsetroundjoin%
\definecolor{currentfill}{rgb}{0.000000,0.000000,0.545098}%
\pgfsetfillcolor{currentfill}%
\pgfsetfillopacity{0.400000}%
\pgfsetlinewidth{0.501875pt}%
\definecolor{currentstroke}{rgb}{0.000000,0.000000,0.545098}%
\pgfsetstrokecolor{currentstroke}%
\pgfsetstrokeopacity{0.400000}%
\pgfsetdash{}{0pt}%
\pgfpathmoveto{\pgfqpoint{2.679770in}{0.940105in}}%
\pgfpathcurveto{\pgfqpoint{2.690820in}{0.940105in}}{\pgfqpoint{2.701419in}{0.944496in}}{\pgfqpoint{2.709233in}{0.952309in}}%
\pgfpathcurveto{\pgfqpoint{2.717046in}{0.960123in}}{\pgfqpoint{2.721437in}{0.970722in}}{\pgfqpoint{2.721437in}{0.981772in}}%
\pgfpathcurveto{\pgfqpoint{2.721437in}{0.992822in}}{\pgfqpoint{2.717046in}{1.003421in}}{\pgfqpoint{2.709233in}{1.011235in}}%
\pgfpathcurveto{\pgfqpoint{2.701419in}{1.019048in}}{\pgfqpoint{2.690820in}{1.023439in}}{\pgfqpoint{2.679770in}{1.023439in}}%
\pgfpathcurveto{\pgfqpoint{2.668720in}{1.023439in}}{\pgfqpoint{2.658121in}{1.019048in}}{\pgfqpoint{2.650307in}{1.011235in}}%
\pgfpathcurveto{\pgfqpoint{2.642493in}{1.003421in}}{\pgfqpoint{2.638103in}{0.992822in}}{\pgfqpoint{2.638103in}{0.981772in}}%
\pgfpathcurveto{\pgfqpoint{2.638103in}{0.970722in}}{\pgfqpoint{2.642493in}{0.960123in}}{\pgfqpoint{2.650307in}{0.952309in}}%
\pgfpathcurveto{\pgfqpoint{2.658121in}{0.944496in}}{\pgfqpoint{2.668720in}{0.940105in}}{\pgfqpoint{2.679770in}{0.940105in}}%
\pgfpathclose%
\pgfusepath{stroke,fill}%
\end{pgfscope}%
\begin{pgfscope}%
\pgfpathrectangle{\pgfqpoint{0.511823in}{0.504323in}}{\pgfqpoint{3.218177in}{3.225677in}} %
\pgfusepath{clip}%
\pgfsetbuttcap%
\pgfsetroundjoin%
\definecolor{currentfill}{rgb}{0.000000,0.000000,0.545098}%
\pgfsetfillcolor{currentfill}%
\pgfsetfillopacity{0.400000}%
\pgfsetlinewidth{0.501875pt}%
\definecolor{currentstroke}{rgb}{0.000000,0.000000,0.545098}%
\pgfsetstrokecolor{currentstroke}%
\pgfsetstrokeopacity{0.400000}%
\pgfsetdash{}{0pt}%
\pgfpathmoveto{\pgfqpoint{2.478217in}{0.915125in}}%
\pgfpathcurveto{\pgfqpoint{2.489268in}{0.915125in}}{\pgfqpoint{2.499867in}{0.919515in}}{\pgfqpoint{2.507680in}{0.927329in}}%
\pgfpathcurveto{\pgfqpoint{2.515494in}{0.935142in}}{\pgfqpoint{2.519884in}{0.945741in}}{\pgfqpoint{2.519884in}{0.956792in}}%
\pgfpathcurveto{\pgfqpoint{2.519884in}{0.967842in}}{\pgfqpoint{2.515494in}{0.978441in}}{\pgfqpoint{2.507680in}{0.986254in}}%
\pgfpathcurveto{\pgfqpoint{2.499867in}{0.994068in}}{\pgfqpoint{2.489268in}{0.998458in}}{\pgfqpoint{2.478217in}{0.998458in}}%
\pgfpathcurveto{\pgfqpoint{2.467167in}{0.998458in}}{\pgfqpoint{2.456568in}{0.994068in}}{\pgfqpoint{2.448755in}{0.986254in}}%
\pgfpathcurveto{\pgfqpoint{2.440941in}{0.978441in}}{\pgfqpoint{2.436551in}{0.967842in}}{\pgfqpoint{2.436551in}{0.956792in}}%
\pgfpathcurveto{\pgfqpoint{2.436551in}{0.945741in}}{\pgfqpoint{2.440941in}{0.935142in}}{\pgfqpoint{2.448755in}{0.927329in}}%
\pgfpathcurveto{\pgfqpoint{2.456568in}{0.919515in}}{\pgfqpoint{2.467167in}{0.915125in}}{\pgfqpoint{2.478217in}{0.915125in}}%
\pgfpathclose%
\pgfusepath{stroke,fill}%
\end{pgfscope}%
\begin{pgfscope}%
\pgfpathrectangle{\pgfqpoint{0.511823in}{0.504323in}}{\pgfqpoint{3.218177in}{3.225677in}} %
\pgfusepath{clip}%
\pgfsetbuttcap%
\pgfsetroundjoin%
\definecolor{currentfill}{rgb}{0.000000,0.000000,0.545098}%
\pgfsetfillcolor{currentfill}%
\pgfsetfillopacity{0.400000}%
\pgfsetlinewidth{0.501875pt}%
\definecolor{currentstroke}{rgb}{0.000000,0.000000,0.545098}%
\pgfsetstrokecolor{currentstroke}%
\pgfsetstrokeopacity{0.400000}%
\pgfsetdash{}{0pt}%
\pgfpathmoveto{\pgfqpoint{2.558544in}{0.933106in}}%
\pgfpathcurveto{\pgfqpoint{2.569594in}{0.933106in}}{\pgfqpoint{2.580193in}{0.937496in}}{\pgfqpoint{2.588007in}{0.945309in}}%
\pgfpathcurveto{\pgfqpoint{2.595820in}{0.953123in}}{\pgfqpoint{2.600211in}{0.963722in}}{\pgfqpoint{2.600211in}{0.974772in}}%
\pgfpathcurveto{\pgfqpoint{2.600211in}{0.985822in}}{\pgfqpoint{2.595820in}{0.996421in}}{\pgfqpoint{2.588007in}{1.004235in}}%
\pgfpathcurveto{\pgfqpoint{2.580193in}{1.012049in}}{\pgfqpoint{2.569594in}{1.016439in}}{\pgfqpoint{2.558544in}{1.016439in}}%
\pgfpathcurveto{\pgfqpoint{2.547494in}{1.016439in}}{\pgfqpoint{2.536895in}{1.012049in}}{\pgfqpoint{2.529081in}{1.004235in}}%
\pgfpathcurveto{\pgfqpoint{2.521268in}{0.996421in}}{\pgfqpoint{2.516877in}{0.985822in}}{\pgfqpoint{2.516877in}{0.974772in}}%
\pgfpathcurveto{\pgfqpoint{2.516877in}{0.963722in}}{\pgfqpoint{2.521268in}{0.953123in}}{\pgfqpoint{2.529081in}{0.945309in}}%
\pgfpathcurveto{\pgfqpoint{2.536895in}{0.937496in}}{\pgfqpoint{2.547494in}{0.933106in}}{\pgfqpoint{2.558544in}{0.933106in}}%
\pgfpathclose%
\pgfusepath{stroke,fill}%
\end{pgfscope}%
\begin{pgfscope}%
\pgfpathrectangle{\pgfqpoint{0.511823in}{0.504323in}}{\pgfqpoint{3.218177in}{3.225677in}} %
\pgfusepath{clip}%
\pgfsetbuttcap%
\pgfsetroundjoin%
\definecolor{currentfill}{rgb}{0.000000,0.000000,0.545098}%
\pgfsetfillcolor{currentfill}%
\pgfsetfillopacity{0.400000}%
\pgfsetlinewidth{0.501875pt}%
\definecolor{currentstroke}{rgb}{0.000000,0.000000,0.545098}%
\pgfsetstrokecolor{currentstroke}%
\pgfsetstrokeopacity{0.400000}%
\pgfsetdash{}{0pt}%
\pgfpathmoveto{\pgfqpoint{2.572193in}{0.940927in}}%
\pgfpathcurveto{\pgfqpoint{2.583243in}{0.940927in}}{\pgfqpoint{2.593842in}{0.945318in}}{\pgfqpoint{2.601656in}{0.953131in}}%
\pgfpathcurveto{\pgfqpoint{2.609470in}{0.960945in}}{\pgfqpoint{2.613860in}{0.971544in}}{\pgfqpoint{2.613860in}{0.982594in}}%
\pgfpathcurveto{\pgfqpoint{2.613860in}{0.993644in}}{\pgfqpoint{2.609470in}{1.004243in}}{\pgfqpoint{2.601656in}{1.012057in}}%
\pgfpathcurveto{\pgfqpoint{2.593842in}{1.019870in}}{\pgfqpoint{2.583243in}{1.024261in}}{\pgfqpoint{2.572193in}{1.024261in}}%
\pgfpathcurveto{\pgfqpoint{2.561143in}{1.024261in}}{\pgfqpoint{2.550544in}{1.019870in}}{\pgfqpoint{2.542730in}{1.012057in}}%
\pgfpathcurveto{\pgfqpoint{2.534917in}{1.004243in}}{\pgfqpoint{2.530526in}{0.993644in}}{\pgfqpoint{2.530526in}{0.982594in}}%
\pgfpathcurveto{\pgfqpoint{2.530526in}{0.971544in}}{\pgfqpoint{2.534917in}{0.960945in}}{\pgfqpoint{2.542730in}{0.953131in}}%
\pgfpathcurveto{\pgfqpoint{2.550544in}{0.945318in}}{\pgfqpoint{2.561143in}{0.940927in}}{\pgfqpoint{2.572193in}{0.940927in}}%
\pgfpathclose%
\pgfusepath{stroke,fill}%
\end{pgfscope}%
\begin{pgfscope}%
\pgfpathrectangle{\pgfqpoint{0.511823in}{0.504323in}}{\pgfqpoint{3.218177in}{3.225677in}} %
\pgfusepath{clip}%
\pgfsetbuttcap%
\pgfsetroundjoin%
\definecolor{currentfill}{rgb}{0.000000,0.000000,0.545098}%
\pgfsetfillcolor{currentfill}%
\pgfsetfillopacity{0.400000}%
\pgfsetlinewidth{0.501875pt}%
\definecolor{currentstroke}{rgb}{0.000000,0.000000,0.545098}%
\pgfsetstrokecolor{currentstroke}%
\pgfsetstrokeopacity{0.400000}%
\pgfsetdash{}{0pt}%
\pgfpathmoveto{\pgfqpoint{2.797204in}{0.983231in}}%
\pgfpathcurveto{\pgfqpoint{2.808254in}{0.983231in}}{\pgfqpoint{2.818853in}{0.987621in}}{\pgfqpoint{2.826667in}{0.995435in}}%
\pgfpathcurveto{\pgfqpoint{2.834480in}{1.003248in}}{\pgfqpoint{2.838871in}{1.013847in}}{\pgfqpoint{2.838871in}{1.024897in}}%
\pgfpathcurveto{\pgfqpoint{2.838871in}{1.035948in}}{\pgfqpoint{2.834480in}{1.046547in}}{\pgfqpoint{2.826667in}{1.054360in}}%
\pgfpathcurveto{\pgfqpoint{2.818853in}{1.062174in}}{\pgfqpoint{2.808254in}{1.066564in}}{\pgfqpoint{2.797204in}{1.066564in}}%
\pgfpathcurveto{\pgfqpoint{2.786154in}{1.066564in}}{\pgfqpoint{2.775555in}{1.062174in}}{\pgfqpoint{2.767741in}{1.054360in}}%
\pgfpathcurveto{\pgfqpoint{2.759927in}{1.046547in}}{\pgfqpoint{2.755537in}{1.035948in}}{\pgfqpoint{2.755537in}{1.024897in}}%
\pgfpathcurveto{\pgfqpoint{2.755537in}{1.013847in}}{\pgfqpoint{2.759927in}{1.003248in}}{\pgfqpoint{2.767741in}{0.995435in}}%
\pgfpathcurveto{\pgfqpoint{2.775555in}{0.987621in}}{\pgfqpoint{2.786154in}{0.983231in}}{\pgfqpoint{2.797204in}{0.983231in}}%
\pgfpathclose%
\pgfusepath{stroke,fill}%
\end{pgfscope}%
\begin{pgfscope}%
\pgfpathrectangle{\pgfqpoint{0.511823in}{0.504323in}}{\pgfqpoint{3.218177in}{3.225677in}} %
\pgfusepath{clip}%
\pgfsetbuttcap%
\pgfsetroundjoin%
\definecolor{currentfill}{rgb}{0.000000,0.000000,0.545098}%
\pgfsetfillcolor{currentfill}%
\pgfsetfillopacity{0.400000}%
\pgfsetlinewidth{0.501875pt}%
\definecolor{currentstroke}{rgb}{0.000000,0.000000,0.545098}%
\pgfsetstrokecolor{currentstroke}%
\pgfsetstrokeopacity{0.400000}%
\pgfsetdash{}{0pt}%
\pgfpathmoveto{\pgfqpoint{2.631549in}{0.962150in}}%
\pgfpathcurveto{\pgfqpoint{2.642599in}{0.962150in}}{\pgfqpoint{2.653198in}{0.966540in}}{\pgfqpoint{2.661012in}{0.974353in}}%
\pgfpathcurveto{\pgfqpoint{2.668825in}{0.982167in}}{\pgfqpoint{2.673216in}{0.992766in}}{\pgfqpoint{2.673216in}{1.003816in}}%
\pgfpathcurveto{\pgfqpoint{2.673216in}{1.014866in}}{\pgfqpoint{2.668825in}{1.025465in}}{\pgfqpoint{2.661012in}{1.033279in}}%
\pgfpathcurveto{\pgfqpoint{2.653198in}{1.041093in}}{\pgfqpoint{2.642599in}{1.045483in}}{\pgfqpoint{2.631549in}{1.045483in}}%
\pgfpathcurveto{\pgfqpoint{2.620499in}{1.045483in}}{\pgfqpoint{2.609900in}{1.041093in}}{\pgfqpoint{2.602086in}{1.033279in}}%
\pgfpathcurveto{\pgfqpoint{2.594273in}{1.025465in}}{\pgfqpoint{2.589882in}{1.014866in}}{\pgfqpoint{2.589882in}{1.003816in}}%
\pgfpathcurveto{\pgfqpoint{2.589882in}{0.992766in}}{\pgfqpoint{2.594273in}{0.982167in}}{\pgfqpoint{2.602086in}{0.974353in}}%
\pgfpathcurveto{\pgfqpoint{2.609900in}{0.966540in}}{\pgfqpoint{2.620499in}{0.962150in}}{\pgfqpoint{2.631549in}{0.962150in}}%
\pgfpathclose%
\pgfusepath{stroke,fill}%
\end{pgfscope}%
\begin{pgfscope}%
\pgfpathrectangle{\pgfqpoint{0.511823in}{0.504323in}}{\pgfqpoint{3.218177in}{3.225677in}} %
\pgfusepath{clip}%
\pgfsetbuttcap%
\pgfsetroundjoin%
\definecolor{currentfill}{rgb}{0.000000,0.000000,0.545098}%
\pgfsetfillcolor{currentfill}%
\pgfsetfillopacity{0.400000}%
\pgfsetlinewidth{0.501875pt}%
\definecolor{currentstroke}{rgb}{0.000000,0.000000,0.545098}%
\pgfsetstrokecolor{currentstroke}%
\pgfsetstrokeopacity{0.400000}%
\pgfsetdash{}{0pt}%
\pgfpathmoveto{\pgfqpoint{2.939456in}{1.019995in}}%
\pgfpathcurveto{\pgfqpoint{2.950506in}{1.019995in}}{\pgfqpoint{2.961105in}{1.024386in}}{\pgfqpoint{2.968919in}{1.032199in}}%
\pgfpathcurveto{\pgfqpoint{2.976732in}{1.040013in}}{\pgfqpoint{2.981123in}{1.050612in}}{\pgfqpoint{2.981123in}{1.061662in}}%
\pgfpathcurveto{\pgfqpoint{2.981123in}{1.072712in}}{\pgfqpoint{2.976732in}{1.083311in}}{\pgfqpoint{2.968919in}{1.091125in}}%
\pgfpathcurveto{\pgfqpoint{2.961105in}{1.098938in}}{\pgfqpoint{2.950506in}{1.103329in}}{\pgfqpoint{2.939456in}{1.103329in}}%
\pgfpathcurveto{\pgfqpoint{2.928406in}{1.103329in}}{\pgfqpoint{2.917807in}{1.098938in}}{\pgfqpoint{2.909993in}{1.091125in}}%
\pgfpathcurveto{\pgfqpoint{2.902180in}{1.083311in}}{\pgfqpoint{2.897789in}{1.072712in}}{\pgfqpoint{2.897789in}{1.061662in}}%
\pgfpathcurveto{\pgfqpoint{2.897789in}{1.050612in}}{\pgfqpoint{2.902180in}{1.040013in}}{\pgfqpoint{2.909993in}{1.032199in}}%
\pgfpathcurveto{\pgfqpoint{2.917807in}{1.024386in}}{\pgfqpoint{2.928406in}{1.019995in}}{\pgfqpoint{2.939456in}{1.019995in}}%
\pgfpathclose%
\pgfusepath{stroke,fill}%
\end{pgfscope}%
\begin{pgfscope}%
\pgfpathrectangle{\pgfqpoint{0.511823in}{0.504323in}}{\pgfqpoint{3.218177in}{3.225677in}} %
\pgfusepath{clip}%
\pgfsetbuttcap%
\pgfsetroundjoin%
\definecolor{currentfill}{rgb}{0.000000,0.000000,0.545098}%
\pgfsetfillcolor{currentfill}%
\pgfsetfillopacity{0.400000}%
\pgfsetlinewidth{0.501875pt}%
\definecolor{currentstroke}{rgb}{0.000000,0.000000,0.545098}%
\pgfsetstrokecolor{currentstroke}%
\pgfsetstrokeopacity{0.400000}%
\pgfsetdash{}{0pt}%
\pgfpathmoveto{\pgfqpoint{2.728971in}{0.990655in}}%
\pgfpathcurveto{\pgfqpoint{2.740021in}{0.990655in}}{\pgfqpoint{2.750620in}{0.995046in}}{\pgfqpoint{2.758433in}{1.002859in}}%
\pgfpathcurveto{\pgfqpoint{2.766247in}{1.010673in}}{\pgfqpoint{2.770637in}{1.021272in}}{\pgfqpoint{2.770637in}{1.032322in}}%
\pgfpathcurveto{\pgfqpoint{2.770637in}{1.043372in}}{\pgfqpoint{2.766247in}{1.053971in}}{\pgfqpoint{2.758433in}{1.061785in}}%
\pgfpathcurveto{\pgfqpoint{2.750620in}{1.069598in}}{\pgfqpoint{2.740021in}{1.073989in}}{\pgfqpoint{2.728971in}{1.073989in}}%
\pgfpathcurveto{\pgfqpoint{2.717921in}{1.073989in}}{\pgfqpoint{2.707322in}{1.069598in}}{\pgfqpoint{2.699508in}{1.061785in}}%
\pgfpathcurveto{\pgfqpoint{2.691694in}{1.053971in}}{\pgfqpoint{2.687304in}{1.043372in}}{\pgfqpoint{2.687304in}{1.032322in}}%
\pgfpathcurveto{\pgfqpoint{2.687304in}{1.021272in}}{\pgfqpoint{2.691694in}{1.010673in}}{\pgfqpoint{2.699508in}{1.002859in}}%
\pgfpathcurveto{\pgfqpoint{2.707322in}{0.995046in}}{\pgfqpoint{2.717921in}{0.990655in}}{\pgfqpoint{2.728971in}{0.990655in}}%
\pgfpathclose%
\pgfusepath{stroke,fill}%
\end{pgfscope}%
\begin{pgfscope}%
\pgfpathrectangle{\pgfqpoint{0.511823in}{0.504323in}}{\pgfqpoint{3.218177in}{3.225677in}} %
\pgfusepath{clip}%
\pgfsetbuttcap%
\pgfsetroundjoin%
\definecolor{currentfill}{rgb}{0.000000,0.000000,0.545098}%
\pgfsetfillcolor{currentfill}%
\pgfsetfillopacity{0.400000}%
\pgfsetlinewidth{0.501875pt}%
\definecolor{currentstroke}{rgb}{0.000000,0.000000,0.545098}%
\pgfsetstrokecolor{currentstroke}%
\pgfsetstrokeopacity{0.400000}%
\pgfsetdash{}{0pt}%
\pgfpathmoveto{\pgfqpoint{2.658435in}{0.984514in}}%
\pgfpathcurveto{\pgfqpoint{2.669485in}{0.984514in}}{\pgfqpoint{2.680084in}{0.988904in}}{\pgfqpoint{2.687898in}{0.996718in}}%
\pgfpathcurveto{\pgfqpoint{2.695711in}{1.004532in}}{\pgfqpoint{2.700102in}{1.015131in}}{\pgfqpoint{2.700102in}{1.026181in}}%
\pgfpathcurveto{\pgfqpoint{2.700102in}{1.037231in}}{\pgfqpoint{2.695711in}{1.047830in}}{\pgfqpoint{2.687898in}{1.055643in}}%
\pgfpathcurveto{\pgfqpoint{2.680084in}{1.063457in}}{\pgfqpoint{2.669485in}{1.067847in}}{\pgfqpoint{2.658435in}{1.067847in}}%
\pgfpathcurveto{\pgfqpoint{2.647385in}{1.067847in}}{\pgfqpoint{2.636786in}{1.063457in}}{\pgfqpoint{2.628972in}{1.055643in}}%
\pgfpathcurveto{\pgfqpoint{2.621158in}{1.047830in}}{\pgfqpoint{2.616768in}{1.037231in}}{\pgfqpoint{2.616768in}{1.026181in}}%
\pgfpathcurveto{\pgfqpoint{2.616768in}{1.015131in}}{\pgfqpoint{2.621158in}{1.004532in}}{\pgfqpoint{2.628972in}{0.996718in}}%
\pgfpathcurveto{\pgfqpoint{2.636786in}{0.988904in}}{\pgfqpoint{2.647385in}{0.984514in}}{\pgfqpoint{2.658435in}{0.984514in}}%
\pgfpathclose%
\pgfusepath{stroke,fill}%
\end{pgfscope}%
\begin{pgfscope}%
\pgfpathrectangle{\pgfqpoint{0.511823in}{0.504323in}}{\pgfqpoint{3.218177in}{3.225677in}} %
\pgfusepath{clip}%
\pgfsetbuttcap%
\pgfsetroundjoin%
\definecolor{currentfill}{rgb}{0.000000,0.000000,0.545098}%
\pgfsetfillcolor{currentfill}%
\pgfsetfillopacity{0.400000}%
\pgfsetlinewidth{0.501875pt}%
\definecolor{currentstroke}{rgb}{0.000000,0.000000,0.545098}%
\pgfsetstrokecolor{currentstroke}%
\pgfsetstrokeopacity{0.400000}%
\pgfsetdash{}{0pt}%
\pgfpathmoveto{\pgfqpoint{2.584222in}{0.977298in}}%
\pgfpathcurveto{\pgfqpoint{2.595272in}{0.977298in}}{\pgfqpoint{2.605871in}{0.981688in}}{\pgfqpoint{2.613685in}{0.989502in}}%
\pgfpathcurveto{\pgfqpoint{2.621498in}{0.997316in}}{\pgfqpoint{2.625889in}{1.007915in}}{\pgfqpoint{2.625889in}{1.018965in}}%
\pgfpathcurveto{\pgfqpoint{2.625889in}{1.030015in}}{\pgfqpoint{2.621498in}{1.040614in}}{\pgfqpoint{2.613685in}{1.048427in}}%
\pgfpathcurveto{\pgfqpoint{2.605871in}{1.056241in}}{\pgfqpoint{2.595272in}{1.060631in}}{\pgfqpoint{2.584222in}{1.060631in}}%
\pgfpathcurveto{\pgfqpoint{2.573172in}{1.060631in}}{\pgfqpoint{2.562573in}{1.056241in}}{\pgfqpoint{2.554759in}{1.048427in}}%
\pgfpathcurveto{\pgfqpoint{2.546946in}{1.040614in}}{\pgfqpoint{2.542555in}{1.030015in}}{\pgfqpoint{2.542555in}{1.018965in}}%
\pgfpathcurveto{\pgfqpoint{2.542555in}{1.007915in}}{\pgfqpoint{2.546946in}{0.997316in}}{\pgfqpoint{2.554759in}{0.989502in}}%
\pgfpathcurveto{\pgfqpoint{2.562573in}{0.981688in}}{\pgfqpoint{2.573172in}{0.977298in}}{\pgfqpoint{2.584222in}{0.977298in}}%
\pgfpathclose%
\pgfusepath{stroke,fill}%
\end{pgfscope}%
\begin{pgfscope}%
\pgfpathrectangle{\pgfqpoint{0.511823in}{0.504323in}}{\pgfqpoint{3.218177in}{3.225677in}} %
\pgfusepath{clip}%
\pgfsetbuttcap%
\pgfsetroundjoin%
\definecolor{currentfill}{rgb}{0.000000,0.000000,0.545098}%
\pgfsetfillcolor{currentfill}%
\pgfsetfillopacity{0.400000}%
\pgfsetlinewidth{0.501875pt}%
\definecolor{currentstroke}{rgb}{0.000000,0.000000,0.545098}%
\pgfsetstrokecolor{currentstroke}%
\pgfsetstrokeopacity{0.400000}%
\pgfsetdash{}{0pt}%
\pgfpathmoveto{\pgfqpoint{2.766412in}{1.016021in}}%
\pgfpathcurveto{\pgfqpoint{2.777462in}{1.016021in}}{\pgfqpoint{2.788061in}{1.020411in}}{\pgfqpoint{2.795875in}{1.028225in}}%
\pgfpathcurveto{\pgfqpoint{2.803688in}{1.036038in}}{\pgfqpoint{2.808079in}{1.046637in}}{\pgfqpoint{2.808079in}{1.057687in}}%
\pgfpathcurveto{\pgfqpoint{2.808079in}{1.068737in}}{\pgfqpoint{2.803688in}{1.079336in}}{\pgfqpoint{2.795875in}{1.087150in}}%
\pgfpathcurveto{\pgfqpoint{2.788061in}{1.094964in}}{\pgfqpoint{2.777462in}{1.099354in}}{\pgfqpoint{2.766412in}{1.099354in}}%
\pgfpathcurveto{\pgfqpoint{2.755362in}{1.099354in}}{\pgfqpoint{2.744763in}{1.094964in}}{\pgfqpoint{2.736949in}{1.087150in}}%
\pgfpathcurveto{\pgfqpoint{2.729136in}{1.079336in}}{\pgfqpoint{2.724745in}{1.068737in}}{\pgfqpoint{2.724745in}{1.057687in}}%
\pgfpathcurveto{\pgfqpoint{2.724745in}{1.046637in}}{\pgfqpoint{2.729136in}{1.036038in}}{\pgfqpoint{2.736949in}{1.028225in}}%
\pgfpathcurveto{\pgfqpoint{2.744763in}{1.020411in}}{\pgfqpoint{2.755362in}{1.016021in}}{\pgfqpoint{2.766412in}{1.016021in}}%
\pgfpathclose%
\pgfusepath{stroke,fill}%
\end{pgfscope}%
\begin{pgfscope}%
\pgfpathrectangle{\pgfqpoint{0.511823in}{0.504323in}}{\pgfqpoint{3.218177in}{3.225677in}} %
\pgfusepath{clip}%
\pgfsetbuttcap%
\pgfsetroundjoin%
\definecolor{currentfill}{rgb}{0.000000,0.000000,0.545098}%
\pgfsetfillcolor{currentfill}%
\pgfsetfillopacity{0.400000}%
\pgfsetlinewidth{0.501875pt}%
\definecolor{currentstroke}{rgb}{0.000000,0.000000,0.545098}%
\pgfsetstrokecolor{currentstroke}%
\pgfsetstrokeopacity{0.400000}%
\pgfsetdash{}{0pt}%
\pgfpathmoveto{\pgfqpoint{2.607278in}{0.993076in}}%
\pgfpathcurveto{\pgfqpoint{2.618328in}{0.993076in}}{\pgfqpoint{2.628927in}{0.997466in}}{\pgfqpoint{2.636741in}{1.005280in}}%
\pgfpathcurveto{\pgfqpoint{2.644554in}{1.013094in}}{\pgfqpoint{2.648944in}{1.023693in}}{\pgfqpoint{2.648944in}{1.034743in}}%
\pgfpathcurveto{\pgfqpoint{2.648944in}{1.045793in}}{\pgfqpoint{2.644554in}{1.056392in}}{\pgfqpoint{2.636741in}{1.064206in}}%
\pgfpathcurveto{\pgfqpoint{2.628927in}{1.072019in}}{\pgfqpoint{2.618328in}{1.076410in}}{\pgfqpoint{2.607278in}{1.076410in}}%
\pgfpathcurveto{\pgfqpoint{2.596228in}{1.076410in}}{\pgfqpoint{2.585629in}{1.072019in}}{\pgfqpoint{2.577815in}{1.064206in}}%
\pgfpathcurveto{\pgfqpoint{2.570001in}{1.056392in}}{\pgfqpoint{2.565611in}{1.045793in}}{\pgfqpoint{2.565611in}{1.034743in}}%
\pgfpathcurveto{\pgfqpoint{2.565611in}{1.023693in}}{\pgfqpoint{2.570001in}{1.013094in}}{\pgfqpoint{2.577815in}{1.005280in}}%
\pgfpathcurveto{\pgfqpoint{2.585629in}{0.997466in}}{\pgfqpoint{2.596228in}{0.993076in}}{\pgfqpoint{2.607278in}{0.993076in}}%
\pgfpathclose%
\pgfusepath{stroke,fill}%
\end{pgfscope}%
\begin{pgfscope}%
\pgfpathrectangle{\pgfqpoint{0.511823in}{0.504323in}}{\pgfqpoint{3.218177in}{3.225677in}} %
\pgfusepath{clip}%
\pgfsetbuttcap%
\pgfsetroundjoin%
\definecolor{currentfill}{rgb}{0.000000,0.000000,0.545098}%
\pgfsetfillcolor{currentfill}%
\pgfsetfillopacity{0.400000}%
\pgfsetlinewidth{0.501875pt}%
\definecolor{currentstroke}{rgb}{0.000000,0.000000,0.545098}%
\pgfsetstrokecolor{currentstroke}%
\pgfsetstrokeopacity{0.400000}%
\pgfsetdash{}{0pt}%
\pgfpathmoveto{\pgfqpoint{2.530193in}{0.984512in}}%
\pgfpathcurveto{\pgfqpoint{2.541243in}{0.984512in}}{\pgfqpoint{2.551842in}{0.988902in}}{\pgfqpoint{2.559656in}{0.996716in}}%
\pgfpathcurveto{\pgfqpoint{2.567469in}{1.004530in}}{\pgfqpoint{2.571859in}{1.015129in}}{\pgfqpoint{2.571859in}{1.026179in}}%
\pgfpathcurveto{\pgfqpoint{2.571859in}{1.037229in}}{\pgfqpoint{2.567469in}{1.047828in}}{\pgfqpoint{2.559656in}{1.055642in}}%
\pgfpathcurveto{\pgfqpoint{2.551842in}{1.063455in}}{\pgfqpoint{2.541243in}{1.067846in}}{\pgfqpoint{2.530193in}{1.067846in}}%
\pgfpathcurveto{\pgfqpoint{2.519143in}{1.067846in}}{\pgfqpoint{2.508544in}{1.063455in}}{\pgfqpoint{2.500730in}{1.055642in}}%
\pgfpathcurveto{\pgfqpoint{2.492916in}{1.047828in}}{\pgfqpoint{2.488526in}{1.037229in}}{\pgfqpoint{2.488526in}{1.026179in}}%
\pgfpathcurveto{\pgfqpoint{2.488526in}{1.015129in}}{\pgfqpoint{2.492916in}{1.004530in}}{\pgfqpoint{2.500730in}{0.996716in}}%
\pgfpathcurveto{\pgfqpoint{2.508544in}{0.988902in}}{\pgfqpoint{2.519143in}{0.984512in}}{\pgfqpoint{2.530193in}{0.984512in}}%
\pgfpathclose%
\pgfusepath{stroke,fill}%
\end{pgfscope}%
\begin{pgfscope}%
\pgfpathrectangle{\pgfqpoint{0.511823in}{0.504323in}}{\pgfqpoint{3.218177in}{3.225677in}} %
\pgfusepath{clip}%
\pgfsetbuttcap%
\pgfsetroundjoin%
\definecolor{currentfill}{rgb}{0.000000,0.000000,0.545098}%
\pgfsetfillcolor{currentfill}%
\pgfsetfillopacity{0.400000}%
\pgfsetlinewidth{0.501875pt}%
\definecolor{currentstroke}{rgb}{0.000000,0.000000,0.545098}%
\pgfsetstrokecolor{currentstroke}%
\pgfsetstrokeopacity{0.400000}%
\pgfsetdash{}{0pt}%
\pgfpathmoveto{\pgfqpoint{2.709983in}{1.024300in}}%
\pgfpathcurveto{\pgfqpoint{2.721033in}{1.024300in}}{\pgfqpoint{2.731632in}{1.028690in}}{\pgfqpoint{2.739445in}{1.036504in}}%
\pgfpathcurveto{\pgfqpoint{2.747259in}{1.044318in}}{\pgfqpoint{2.751649in}{1.054917in}}{\pgfqpoint{2.751649in}{1.065967in}}%
\pgfpathcurveto{\pgfqpoint{2.751649in}{1.077017in}}{\pgfqpoint{2.747259in}{1.087616in}}{\pgfqpoint{2.739445in}{1.095430in}}%
\pgfpathcurveto{\pgfqpoint{2.731632in}{1.103243in}}{\pgfqpoint{2.721033in}{1.107633in}}{\pgfqpoint{2.709983in}{1.107633in}}%
\pgfpathcurveto{\pgfqpoint{2.698932in}{1.107633in}}{\pgfqpoint{2.688333in}{1.103243in}}{\pgfqpoint{2.680520in}{1.095430in}}%
\pgfpathcurveto{\pgfqpoint{2.672706in}{1.087616in}}{\pgfqpoint{2.668316in}{1.077017in}}{\pgfqpoint{2.668316in}{1.065967in}}%
\pgfpathcurveto{\pgfqpoint{2.668316in}{1.054917in}}{\pgfqpoint{2.672706in}{1.044318in}}{\pgfqpoint{2.680520in}{1.036504in}}%
\pgfpathcurveto{\pgfqpoint{2.688333in}{1.028690in}}{\pgfqpoint{2.698932in}{1.024300in}}{\pgfqpoint{2.709983in}{1.024300in}}%
\pgfpathclose%
\pgfusepath{stroke,fill}%
\end{pgfscope}%
\begin{pgfscope}%
\pgfpathrectangle{\pgfqpoint{0.511823in}{0.504323in}}{\pgfqpoint{3.218177in}{3.225677in}} %
\pgfusepath{clip}%
\pgfsetbuttcap%
\pgfsetroundjoin%
\definecolor{currentfill}{rgb}{0.000000,0.000000,0.545098}%
\pgfsetfillcolor{currentfill}%
\pgfsetfillopacity{0.400000}%
\pgfsetlinewidth{0.501875pt}%
\definecolor{currentstroke}{rgb}{0.000000,0.000000,0.545098}%
\pgfsetstrokecolor{currentstroke}%
\pgfsetstrokeopacity{0.400000}%
\pgfsetdash{}{0pt}%
\pgfpathmoveto{\pgfqpoint{2.636531in}{1.016296in}}%
\pgfpathcurveto{\pgfqpoint{2.647581in}{1.016296in}}{\pgfqpoint{2.658180in}{1.020686in}}{\pgfqpoint{2.665994in}{1.028500in}}%
\pgfpathcurveto{\pgfqpoint{2.673807in}{1.036313in}}{\pgfqpoint{2.678197in}{1.046912in}}{\pgfqpoint{2.678197in}{1.057962in}}%
\pgfpathcurveto{\pgfqpoint{2.678197in}{1.069013in}}{\pgfqpoint{2.673807in}{1.079612in}}{\pgfqpoint{2.665994in}{1.087425in}}%
\pgfpathcurveto{\pgfqpoint{2.658180in}{1.095239in}}{\pgfqpoint{2.647581in}{1.099629in}}{\pgfqpoint{2.636531in}{1.099629in}}%
\pgfpathcurveto{\pgfqpoint{2.625481in}{1.099629in}}{\pgfqpoint{2.614882in}{1.095239in}}{\pgfqpoint{2.607068in}{1.087425in}}%
\pgfpathcurveto{\pgfqpoint{2.599254in}{1.079612in}}{\pgfqpoint{2.594864in}{1.069013in}}{\pgfqpoint{2.594864in}{1.057962in}}%
\pgfpathcurveto{\pgfqpoint{2.594864in}{1.046912in}}{\pgfqpoint{2.599254in}{1.036313in}}{\pgfqpoint{2.607068in}{1.028500in}}%
\pgfpathcurveto{\pgfqpoint{2.614882in}{1.020686in}}{\pgfqpoint{2.625481in}{1.016296in}}{\pgfqpoint{2.636531in}{1.016296in}}%
\pgfpathclose%
\pgfusepath{stroke,fill}%
\end{pgfscope}%
\begin{pgfscope}%
\pgfpathrectangle{\pgfqpoint{0.511823in}{0.504323in}}{\pgfqpoint{3.218177in}{3.225677in}} %
\pgfusepath{clip}%
\pgfsetbuttcap%
\pgfsetroundjoin%
\definecolor{currentfill}{rgb}{0.000000,0.000000,0.545098}%
\pgfsetfillcolor{currentfill}%
\pgfsetfillopacity{0.400000}%
\pgfsetlinewidth{0.501875pt}%
\definecolor{currentstroke}{rgb}{0.000000,0.000000,0.545098}%
\pgfsetstrokecolor{currentstroke}%
\pgfsetstrokeopacity{0.400000}%
\pgfsetdash{}{0pt}%
\pgfpathmoveto{\pgfqpoint{2.787943in}{1.051956in}}%
\pgfpathcurveto{\pgfqpoint{2.798993in}{1.051956in}}{\pgfqpoint{2.809592in}{1.056346in}}{\pgfqpoint{2.817406in}{1.064160in}}%
\pgfpathcurveto{\pgfqpoint{2.825220in}{1.071973in}}{\pgfqpoint{2.829610in}{1.082572in}}{\pgfqpoint{2.829610in}{1.093622in}}%
\pgfpathcurveto{\pgfqpoint{2.829610in}{1.104673in}}{\pgfqpoint{2.825220in}{1.115272in}}{\pgfqpoint{2.817406in}{1.123085in}}%
\pgfpathcurveto{\pgfqpoint{2.809592in}{1.130899in}}{\pgfqpoint{2.798993in}{1.135289in}}{\pgfqpoint{2.787943in}{1.135289in}}%
\pgfpathcurveto{\pgfqpoint{2.776893in}{1.135289in}}{\pgfqpoint{2.766294in}{1.130899in}}{\pgfqpoint{2.758480in}{1.123085in}}%
\pgfpathcurveto{\pgfqpoint{2.750667in}{1.115272in}}{\pgfqpoint{2.746277in}{1.104673in}}{\pgfqpoint{2.746277in}{1.093622in}}%
\pgfpathcurveto{\pgfqpoint{2.746277in}{1.082572in}}{\pgfqpoint{2.750667in}{1.071973in}}{\pgfqpoint{2.758480in}{1.064160in}}%
\pgfpathcurveto{\pgfqpoint{2.766294in}{1.056346in}}{\pgfqpoint{2.776893in}{1.051956in}}{\pgfqpoint{2.787943in}{1.051956in}}%
\pgfpathclose%
\pgfusepath{stroke,fill}%
\end{pgfscope}%
\begin{pgfscope}%
\pgfpathrectangle{\pgfqpoint{0.511823in}{0.504323in}}{\pgfqpoint{3.218177in}{3.225677in}} %
\pgfusepath{clip}%
\pgfsetbuttcap%
\pgfsetroundjoin%
\definecolor{currentfill}{rgb}{0.000000,0.000000,0.545098}%
\pgfsetfillcolor{currentfill}%
\pgfsetfillopacity{0.400000}%
\pgfsetlinewidth{0.501875pt}%
\definecolor{currentstroke}{rgb}{0.000000,0.000000,0.545098}%
\pgfsetstrokecolor{currentstroke}%
\pgfsetstrokeopacity{0.400000}%
\pgfsetdash{}{0pt}%
\pgfpathmoveto{\pgfqpoint{2.622743in}{1.025480in}}%
\pgfpathcurveto{\pgfqpoint{2.633793in}{1.025480in}}{\pgfqpoint{2.644392in}{1.029871in}}{\pgfqpoint{2.652205in}{1.037684in}}%
\pgfpathcurveto{\pgfqpoint{2.660019in}{1.045498in}}{\pgfqpoint{2.664409in}{1.056097in}}{\pgfqpoint{2.664409in}{1.067147in}}%
\pgfpathcurveto{\pgfqpoint{2.664409in}{1.078197in}}{\pgfqpoint{2.660019in}{1.088796in}}{\pgfqpoint{2.652205in}{1.096610in}}%
\pgfpathcurveto{\pgfqpoint{2.644392in}{1.104423in}}{\pgfqpoint{2.633793in}{1.108814in}}{\pgfqpoint{2.622743in}{1.108814in}}%
\pgfpathcurveto{\pgfqpoint{2.611692in}{1.108814in}}{\pgfqpoint{2.601093in}{1.104423in}}{\pgfqpoint{2.593280in}{1.096610in}}%
\pgfpathcurveto{\pgfqpoint{2.585466in}{1.088796in}}{\pgfqpoint{2.581076in}{1.078197in}}{\pgfqpoint{2.581076in}{1.067147in}}%
\pgfpathcurveto{\pgfqpoint{2.581076in}{1.056097in}}{\pgfqpoint{2.585466in}{1.045498in}}{\pgfqpoint{2.593280in}{1.037684in}}%
\pgfpathcurveto{\pgfqpoint{2.601093in}{1.029871in}}{\pgfqpoint{2.611692in}{1.025480in}}{\pgfqpoint{2.622743in}{1.025480in}}%
\pgfpathclose%
\pgfusepath{stroke,fill}%
\end{pgfscope}%
\begin{pgfscope}%
\pgfpathrectangle{\pgfqpoint{0.511823in}{0.504323in}}{\pgfqpoint{3.218177in}{3.225677in}} %
\pgfusepath{clip}%
\pgfsetbuttcap%
\pgfsetroundjoin%
\definecolor{currentfill}{rgb}{0.000000,0.000000,0.545098}%
\pgfsetfillcolor{currentfill}%
\pgfsetfillopacity{0.400000}%
\pgfsetlinewidth{0.501875pt}%
\definecolor{currentstroke}{rgb}{0.000000,0.000000,0.545098}%
\pgfsetstrokecolor{currentstroke}%
\pgfsetstrokeopacity{0.400000}%
\pgfsetdash{}{0pt}%
\pgfpathmoveto{\pgfqpoint{2.260252in}{0.958095in}}%
\pgfpathcurveto{\pgfqpoint{2.271302in}{0.958095in}}{\pgfqpoint{2.281901in}{0.962485in}}{\pgfqpoint{2.289715in}{0.970299in}}%
\pgfpathcurveto{\pgfqpoint{2.297528in}{0.978112in}}{\pgfqpoint{2.301919in}{0.988712in}}{\pgfqpoint{2.301919in}{0.999762in}}%
\pgfpathcurveto{\pgfqpoint{2.301919in}{1.010812in}}{\pgfqpoint{2.297528in}{1.021411in}}{\pgfqpoint{2.289715in}{1.029224in}}%
\pgfpathcurveto{\pgfqpoint{2.281901in}{1.037038in}}{\pgfqpoint{2.271302in}{1.041428in}}{\pgfqpoint{2.260252in}{1.041428in}}%
\pgfpathcurveto{\pgfqpoint{2.249202in}{1.041428in}}{\pgfqpoint{2.238603in}{1.037038in}}{\pgfqpoint{2.230789in}{1.029224in}}%
\pgfpathcurveto{\pgfqpoint{2.222976in}{1.021411in}}{\pgfqpoint{2.218585in}{1.010812in}}{\pgfqpoint{2.218585in}{0.999762in}}%
\pgfpathcurveto{\pgfqpoint{2.218585in}{0.988712in}}{\pgfqpoint{2.222976in}{0.978112in}}{\pgfqpoint{2.230789in}{0.970299in}}%
\pgfpathcurveto{\pgfqpoint{2.238603in}{0.962485in}}{\pgfqpoint{2.249202in}{0.958095in}}{\pgfqpoint{2.260252in}{0.958095in}}%
\pgfpathclose%
\pgfusepath{stroke,fill}%
\end{pgfscope}%
\begin{pgfscope}%
\pgfpathrectangle{\pgfqpoint{0.511823in}{0.504323in}}{\pgfqpoint{3.218177in}{3.225677in}} %
\pgfusepath{clip}%
\pgfsetbuttcap%
\pgfsetroundjoin%
\definecolor{currentfill}{rgb}{0.000000,0.000000,0.545098}%
\pgfsetfillcolor{currentfill}%
\pgfsetfillopacity{0.400000}%
\pgfsetlinewidth{0.501875pt}%
\definecolor{currentstroke}{rgb}{0.000000,0.000000,0.545098}%
\pgfsetstrokecolor{currentstroke}%
\pgfsetstrokeopacity{0.400000}%
\pgfsetdash{}{0pt}%
\pgfpathmoveto{\pgfqpoint{2.731887in}{1.059771in}}%
\pgfpathcurveto{\pgfqpoint{2.742937in}{1.059771in}}{\pgfqpoint{2.753536in}{1.064161in}}{\pgfqpoint{2.761350in}{1.071975in}}%
\pgfpathcurveto{\pgfqpoint{2.769164in}{1.079788in}}{\pgfqpoint{2.773554in}{1.090387in}}{\pgfqpoint{2.773554in}{1.101437in}}%
\pgfpathcurveto{\pgfqpoint{2.773554in}{1.112488in}}{\pgfqpoint{2.769164in}{1.123087in}}{\pgfqpoint{2.761350in}{1.130900in}}%
\pgfpathcurveto{\pgfqpoint{2.753536in}{1.138714in}}{\pgfqpoint{2.742937in}{1.143104in}}{\pgfqpoint{2.731887in}{1.143104in}}%
\pgfpathcurveto{\pgfqpoint{2.720837in}{1.143104in}}{\pgfqpoint{2.710238in}{1.138714in}}{\pgfqpoint{2.702424in}{1.130900in}}%
\pgfpathcurveto{\pgfqpoint{2.694611in}{1.123087in}}{\pgfqpoint{2.690221in}{1.112488in}}{\pgfqpoint{2.690221in}{1.101437in}}%
\pgfpathcurveto{\pgfqpoint{2.690221in}{1.090387in}}{\pgfqpoint{2.694611in}{1.079788in}}{\pgfqpoint{2.702424in}{1.071975in}}%
\pgfpathcurveto{\pgfqpoint{2.710238in}{1.064161in}}{\pgfqpoint{2.720837in}{1.059771in}}{\pgfqpoint{2.731887in}{1.059771in}}%
\pgfpathclose%
\pgfusepath{stroke,fill}%
\end{pgfscope}%
\begin{pgfscope}%
\pgfpathrectangle{\pgfqpoint{0.511823in}{0.504323in}}{\pgfqpoint{3.218177in}{3.225677in}} %
\pgfusepath{clip}%
\pgfsetbuttcap%
\pgfsetroundjoin%
\definecolor{currentfill}{rgb}{0.000000,0.000000,0.545098}%
\pgfsetfillcolor{currentfill}%
\pgfsetfillopacity{0.400000}%
\pgfsetlinewidth{0.501875pt}%
\definecolor{currentstroke}{rgb}{0.000000,0.000000,0.545098}%
\pgfsetstrokecolor{currentstroke}%
\pgfsetstrokeopacity{0.400000}%
\pgfsetdash{}{0pt}%
\pgfpathmoveto{\pgfqpoint{2.385185in}{0.993801in}}%
\pgfpathcurveto{\pgfqpoint{2.396235in}{0.993801in}}{\pgfqpoint{2.406834in}{0.998191in}}{\pgfqpoint{2.414648in}{1.006005in}}%
\pgfpathcurveto{\pgfqpoint{2.422461in}{1.013818in}}{\pgfqpoint{2.426852in}{1.024417in}}{\pgfqpoint{2.426852in}{1.035468in}}%
\pgfpathcurveto{\pgfqpoint{2.426852in}{1.046518in}}{\pgfqpoint{2.422461in}{1.057117in}}{\pgfqpoint{2.414648in}{1.064930in}}%
\pgfpathcurveto{\pgfqpoint{2.406834in}{1.072744in}}{\pgfqpoint{2.396235in}{1.077134in}}{\pgfqpoint{2.385185in}{1.077134in}}%
\pgfpathcurveto{\pgfqpoint{2.374135in}{1.077134in}}{\pgfqpoint{2.363536in}{1.072744in}}{\pgfqpoint{2.355722in}{1.064930in}}%
\pgfpathcurveto{\pgfqpoint{2.347909in}{1.057117in}}{\pgfqpoint{2.343518in}{1.046518in}}{\pgfqpoint{2.343518in}{1.035468in}}%
\pgfpathcurveto{\pgfqpoint{2.343518in}{1.024417in}}{\pgfqpoint{2.347909in}{1.013818in}}{\pgfqpoint{2.355722in}{1.006005in}}%
\pgfpathcurveto{\pgfqpoint{2.363536in}{0.998191in}}{\pgfqpoint{2.374135in}{0.993801in}}{\pgfqpoint{2.385185in}{0.993801in}}%
\pgfpathclose%
\pgfusepath{stroke,fill}%
\end{pgfscope}%
\begin{pgfscope}%
\pgfpathrectangle{\pgfqpoint{0.511823in}{0.504323in}}{\pgfqpoint{3.218177in}{3.225677in}} %
\pgfusepath{clip}%
\pgfsetbuttcap%
\pgfsetroundjoin%
\definecolor{currentfill}{rgb}{0.000000,0.000000,0.545098}%
\pgfsetfillcolor{currentfill}%
\pgfsetfillopacity{0.400000}%
\pgfsetlinewidth{0.501875pt}%
\definecolor{currentstroke}{rgb}{0.000000,0.000000,0.545098}%
\pgfsetstrokecolor{currentstroke}%
\pgfsetstrokeopacity{0.400000}%
\pgfsetdash{}{0pt}%
\pgfpathmoveto{\pgfqpoint{2.825948in}{1.092247in}}%
\pgfpathcurveto{\pgfqpoint{2.836999in}{1.092247in}}{\pgfqpoint{2.847598in}{1.096637in}}{\pgfqpoint{2.855411in}{1.104451in}}%
\pgfpathcurveto{\pgfqpoint{2.863225in}{1.112265in}}{\pgfqpoint{2.867615in}{1.122864in}}{\pgfqpoint{2.867615in}{1.133914in}}%
\pgfpathcurveto{\pgfqpoint{2.867615in}{1.144964in}}{\pgfqpoint{2.863225in}{1.155563in}}{\pgfqpoint{2.855411in}{1.163377in}}%
\pgfpathcurveto{\pgfqpoint{2.847598in}{1.171190in}}{\pgfqpoint{2.836999in}{1.175580in}}{\pgfqpoint{2.825948in}{1.175580in}}%
\pgfpathcurveto{\pgfqpoint{2.814898in}{1.175580in}}{\pgfqpoint{2.804299in}{1.171190in}}{\pgfqpoint{2.796486in}{1.163377in}}%
\pgfpathcurveto{\pgfqpoint{2.788672in}{1.155563in}}{\pgfqpoint{2.784282in}{1.144964in}}{\pgfqpoint{2.784282in}{1.133914in}}%
\pgfpathcurveto{\pgfqpoint{2.784282in}{1.122864in}}{\pgfqpoint{2.788672in}{1.112265in}}{\pgfqpoint{2.796486in}{1.104451in}}%
\pgfpathcurveto{\pgfqpoint{2.804299in}{1.096637in}}{\pgfqpoint{2.814898in}{1.092247in}}{\pgfqpoint{2.825948in}{1.092247in}}%
\pgfpathclose%
\pgfusepath{stroke,fill}%
\end{pgfscope}%
\begin{pgfscope}%
\pgfpathrectangle{\pgfqpoint{0.511823in}{0.504323in}}{\pgfqpoint{3.218177in}{3.225677in}} %
\pgfusepath{clip}%
\pgfsetbuttcap%
\pgfsetroundjoin%
\definecolor{currentfill}{rgb}{0.000000,0.000000,0.545098}%
\pgfsetfillcolor{currentfill}%
\pgfsetfillopacity{0.400000}%
\pgfsetlinewidth{0.501875pt}%
\definecolor{currentstroke}{rgb}{0.000000,0.000000,0.545098}%
\pgfsetstrokecolor{currentstroke}%
\pgfsetstrokeopacity{0.400000}%
\pgfsetdash{}{0pt}%
\pgfpathmoveto{\pgfqpoint{2.860940in}{1.106350in}}%
\pgfpathcurveto{\pgfqpoint{2.871990in}{1.106350in}}{\pgfqpoint{2.882589in}{1.110741in}}{\pgfqpoint{2.890403in}{1.118554in}}%
\pgfpathcurveto{\pgfqpoint{2.898216in}{1.126368in}}{\pgfqpoint{2.902607in}{1.136967in}}{\pgfqpoint{2.902607in}{1.148017in}}%
\pgfpathcurveto{\pgfqpoint{2.902607in}{1.159067in}}{\pgfqpoint{2.898216in}{1.169666in}}{\pgfqpoint{2.890403in}{1.177480in}}%
\pgfpathcurveto{\pgfqpoint{2.882589in}{1.185293in}}{\pgfqpoint{2.871990in}{1.189684in}}{\pgfqpoint{2.860940in}{1.189684in}}%
\pgfpathcurveto{\pgfqpoint{2.849890in}{1.189684in}}{\pgfqpoint{2.839291in}{1.185293in}}{\pgfqpoint{2.831477in}{1.177480in}}%
\pgfpathcurveto{\pgfqpoint{2.823663in}{1.169666in}}{\pgfqpoint{2.819273in}{1.159067in}}{\pgfqpoint{2.819273in}{1.148017in}}%
\pgfpathcurveto{\pgfqpoint{2.819273in}{1.136967in}}{\pgfqpoint{2.823663in}{1.126368in}}{\pgfqpoint{2.831477in}{1.118554in}}%
\pgfpathcurveto{\pgfqpoint{2.839291in}{1.110741in}}{\pgfqpoint{2.849890in}{1.106350in}}{\pgfqpoint{2.860940in}{1.106350in}}%
\pgfpathclose%
\pgfusepath{stroke,fill}%
\end{pgfscope}%
\begin{pgfscope}%
\pgfpathrectangle{\pgfqpoint{0.511823in}{0.504323in}}{\pgfqpoint{3.218177in}{3.225677in}} %
\pgfusepath{clip}%
\pgfsetbuttcap%
\pgfsetroundjoin%
\definecolor{currentfill}{rgb}{0.000000,0.000000,0.545098}%
\pgfsetfillcolor{currentfill}%
\pgfsetfillopacity{0.400000}%
\pgfsetlinewidth{0.501875pt}%
\definecolor{currentstroke}{rgb}{0.000000,0.000000,0.545098}%
\pgfsetstrokecolor{currentstroke}%
\pgfsetstrokeopacity{0.400000}%
\pgfsetdash{}{0pt}%
\pgfpathmoveto{\pgfqpoint{2.568912in}{1.049500in}}%
\pgfpathcurveto{\pgfqpoint{2.579963in}{1.049500in}}{\pgfqpoint{2.590562in}{1.053891in}}{\pgfqpoint{2.598375in}{1.061704in}}%
\pgfpathcurveto{\pgfqpoint{2.606189in}{1.069518in}}{\pgfqpoint{2.610579in}{1.080117in}}{\pgfqpoint{2.610579in}{1.091167in}}%
\pgfpathcurveto{\pgfqpoint{2.610579in}{1.102217in}}{\pgfqpoint{2.606189in}{1.112816in}}{\pgfqpoint{2.598375in}{1.120630in}}%
\pgfpathcurveto{\pgfqpoint{2.590562in}{1.128443in}}{\pgfqpoint{2.579963in}{1.132834in}}{\pgfqpoint{2.568912in}{1.132834in}}%
\pgfpathcurveto{\pgfqpoint{2.557862in}{1.132834in}}{\pgfqpoint{2.547263in}{1.128443in}}{\pgfqpoint{2.539450in}{1.120630in}}%
\pgfpathcurveto{\pgfqpoint{2.531636in}{1.112816in}}{\pgfqpoint{2.527246in}{1.102217in}}{\pgfqpoint{2.527246in}{1.091167in}}%
\pgfpathcurveto{\pgfqpoint{2.527246in}{1.080117in}}{\pgfqpoint{2.531636in}{1.069518in}}{\pgfqpoint{2.539450in}{1.061704in}}%
\pgfpathcurveto{\pgfqpoint{2.547263in}{1.053891in}}{\pgfqpoint{2.557862in}{1.049500in}}{\pgfqpoint{2.568912in}{1.049500in}}%
\pgfpathclose%
\pgfusepath{stroke,fill}%
\end{pgfscope}%
\begin{pgfscope}%
\pgfpathrectangle{\pgfqpoint{0.511823in}{0.504323in}}{\pgfqpoint{3.218177in}{3.225677in}} %
\pgfusepath{clip}%
\pgfsetbuttcap%
\pgfsetroundjoin%
\definecolor{currentfill}{rgb}{0.000000,0.000000,0.545098}%
\pgfsetfillcolor{currentfill}%
\pgfsetfillopacity{0.400000}%
\pgfsetlinewidth{0.501875pt}%
\definecolor{currentstroke}{rgb}{0.000000,0.000000,0.545098}%
\pgfsetstrokecolor{currentstroke}%
\pgfsetstrokeopacity{0.400000}%
\pgfsetdash{}{0pt}%
\pgfpathmoveto{\pgfqpoint{2.472298in}{1.033992in}}%
\pgfpathcurveto{\pgfqpoint{2.483349in}{1.033992in}}{\pgfqpoint{2.493948in}{1.038383in}}{\pgfqpoint{2.501761in}{1.046196in}}%
\pgfpathcurveto{\pgfqpoint{2.509575in}{1.054010in}}{\pgfqpoint{2.513965in}{1.064609in}}{\pgfqpoint{2.513965in}{1.075659in}}%
\pgfpathcurveto{\pgfqpoint{2.513965in}{1.086709in}}{\pgfqpoint{2.509575in}{1.097308in}}{\pgfqpoint{2.501761in}{1.105122in}}%
\pgfpathcurveto{\pgfqpoint{2.493948in}{1.112935in}}{\pgfqpoint{2.483349in}{1.117326in}}{\pgfqpoint{2.472298in}{1.117326in}}%
\pgfpathcurveto{\pgfqpoint{2.461248in}{1.117326in}}{\pgfqpoint{2.450649in}{1.112935in}}{\pgfqpoint{2.442836in}{1.105122in}}%
\pgfpathcurveto{\pgfqpoint{2.435022in}{1.097308in}}{\pgfqpoint{2.430632in}{1.086709in}}{\pgfqpoint{2.430632in}{1.075659in}}%
\pgfpathcurveto{\pgfqpoint{2.430632in}{1.064609in}}{\pgfqpoint{2.435022in}{1.054010in}}{\pgfqpoint{2.442836in}{1.046196in}}%
\pgfpathcurveto{\pgfqpoint{2.450649in}{1.038383in}}{\pgfqpoint{2.461248in}{1.033992in}}{\pgfqpoint{2.472298in}{1.033992in}}%
\pgfpathclose%
\pgfusepath{stroke,fill}%
\end{pgfscope}%
\begin{pgfscope}%
\pgfpathrectangle{\pgfqpoint{0.511823in}{0.504323in}}{\pgfqpoint{3.218177in}{3.225677in}} %
\pgfusepath{clip}%
\pgfsetbuttcap%
\pgfsetroundjoin%
\definecolor{currentfill}{rgb}{0.000000,0.000000,0.545098}%
\pgfsetfillcolor{currentfill}%
\pgfsetfillopacity{0.400000}%
\pgfsetlinewidth{0.501875pt}%
\definecolor{currentstroke}{rgb}{0.000000,0.000000,0.545098}%
\pgfsetstrokecolor{currentstroke}%
\pgfsetstrokeopacity{0.400000}%
\pgfsetdash{}{0pt}%
\pgfpathmoveto{\pgfqpoint{2.440984in}{1.032512in}}%
\pgfpathcurveto{\pgfqpoint{2.452034in}{1.032512in}}{\pgfqpoint{2.462633in}{1.036902in}}{\pgfqpoint{2.470447in}{1.044716in}}%
\pgfpathcurveto{\pgfqpoint{2.478260in}{1.052530in}}{\pgfqpoint{2.482650in}{1.063129in}}{\pgfqpoint{2.482650in}{1.074179in}}%
\pgfpathcurveto{\pgfqpoint{2.482650in}{1.085229in}}{\pgfqpoint{2.478260in}{1.095828in}}{\pgfqpoint{2.470447in}{1.103641in}}%
\pgfpathcurveto{\pgfqpoint{2.462633in}{1.111455in}}{\pgfqpoint{2.452034in}{1.115845in}}{\pgfqpoint{2.440984in}{1.115845in}}%
\pgfpathcurveto{\pgfqpoint{2.429934in}{1.115845in}}{\pgfqpoint{2.419335in}{1.111455in}}{\pgfqpoint{2.411521in}{1.103641in}}%
\pgfpathcurveto{\pgfqpoint{2.403707in}{1.095828in}}{\pgfqpoint{2.399317in}{1.085229in}}{\pgfqpoint{2.399317in}{1.074179in}}%
\pgfpathcurveto{\pgfqpoint{2.399317in}{1.063129in}}{\pgfqpoint{2.403707in}{1.052530in}}{\pgfqpoint{2.411521in}{1.044716in}}%
\pgfpathcurveto{\pgfqpoint{2.419335in}{1.036902in}}{\pgfqpoint{2.429934in}{1.032512in}}{\pgfqpoint{2.440984in}{1.032512in}}%
\pgfpathclose%
\pgfusepath{stroke,fill}%
\end{pgfscope}%
\begin{pgfscope}%
\pgfpathrectangle{\pgfqpoint{0.511823in}{0.504323in}}{\pgfqpoint{3.218177in}{3.225677in}} %
\pgfusepath{clip}%
\pgfsetbuttcap%
\pgfsetroundjoin%
\definecolor{currentfill}{rgb}{0.000000,0.000000,0.545098}%
\pgfsetfillcolor{currentfill}%
\pgfsetfillopacity{0.400000}%
\pgfsetlinewidth{0.501875pt}%
\definecolor{currentstroke}{rgb}{0.000000,0.000000,0.545098}%
\pgfsetstrokecolor{currentstroke}%
\pgfsetstrokeopacity{0.400000}%
\pgfsetdash{}{0pt}%
\pgfpathmoveto{\pgfqpoint{2.732179in}{1.104039in}}%
\pgfpathcurveto{\pgfqpoint{2.743229in}{1.104039in}}{\pgfqpoint{2.753829in}{1.108429in}}{\pgfqpoint{2.761642in}{1.116243in}}%
\pgfpathcurveto{\pgfqpoint{2.769456in}{1.124057in}}{\pgfqpoint{2.773846in}{1.134656in}}{\pgfqpoint{2.773846in}{1.145706in}}%
\pgfpathcurveto{\pgfqpoint{2.773846in}{1.156756in}}{\pgfqpoint{2.769456in}{1.167355in}}{\pgfqpoint{2.761642in}{1.175168in}}%
\pgfpathcurveto{\pgfqpoint{2.753829in}{1.182982in}}{\pgfqpoint{2.743229in}{1.187372in}}{\pgfqpoint{2.732179in}{1.187372in}}%
\pgfpathcurveto{\pgfqpoint{2.721129in}{1.187372in}}{\pgfqpoint{2.710530in}{1.182982in}}{\pgfqpoint{2.702717in}{1.175168in}}%
\pgfpathcurveto{\pgfqpoint{2.694903in}{1.167355in}}{\pgfqpoint{2.690513in}{1.156756in}}{\pgfqpoint{2.690513in}{1.145706in}}%
\pgfpathcurveto{\pgfqpoint{2.690513in}{1.134656in}}{\pgfqpoint{2.694903in}{1.124057in}}{\pgfqpoint{2.702717in}{1.116243in}}%
\pgfpathcurveto{\pgfqpoint{2.710530in}{1.108429in}}{\pgfqpoint{2.721129in}{1.104039in}}{\pgfqpoint{2.732179in}{1.104039in}}%
\pgfpathclose%
\pgfusepath{stroke,fill}%
\end{pgfscope}%
\begin{pgfscope}%
\pgfpathrectangle{\pgfqpoint{0.511823in}{0.504323in}}{\pgfqpoint{3.218177in}{3.225677in}} %
\pgfusepath{clip}%
\pgfsetbuttcap%
\pgfsetroundjoin%
\definecolor{currentfill}{rgb}{0.000000,0.000000,0.545098}%
\pgfsetfillcolor{currentfill}%
\pgfsetfillopacity{0.400000}%
\pgfsetlinewidth{0.501875pt}%
\definecolor{currentstroke}{rgb}{0.000000,0.000000,0.545098}%
\pgfsetstrokecolor{currentstroke}%
\pgfsetstrokeopacity{0.400000}%
\pgfsetdash{}{0pt}%
\pgfpathmoveto{\pgfqpoint{2.592184in}{1.078186in}}%
\pgfpathcurveto{\pgfqpoint{2.603234in}{1.078186in}}{\pgfqpoint{2.613833in}{1.082576in}}{\pgfqpoint{2.621647in}{1.090390in}}%
\pgfpathcurveto{\pgfqpoint{2.629460in}{1.098203in}}{\pgfqpoint{2.633851in}{1.108802in}}{\pgfqpoint{2.633851in}{1.119852in}}%
\pgfpathcurveto{\pgfqpoint{2.633851in}{1.130903in}}{\pgfqpoint{2.629460in}{1.141502in}}{\pgfqpoint{2.621647in}{1.149315in}}%
\pgfpathcurveto{\pgfqpoint{2.613833in}{1.157129in}}{\pgfqpoint{2.603234in}{1.161519in}}{\pgfqpoint{2.592184in}{1.161519in}}%
\pgfpathcurveto{\pgfqpoint{2.581134in}{1.161519in}}{\pgfqpoint{2.570535in}{1.157129in}}{\pgfqpoint{2.562721in}{1.149315in}}%
\pgfpathcurveto{\pgfqpoint{2.554908in}{1.141502in}}{\pgfqpoint{2.550517in}{1.130903in}}{\pgfqpoint{2.550517in}{1.119852in}}%
\pgfpathcurveto{\pgfqpoint{2.550517in}{1.108802in}}{\pgfqpoint{2.554908in}{1.098203in}}{\pgfqpoint{2.562721in}{1.090390in}}%
\pgfpathcurveto{\pgfqpoint{2.570535in}{1.082576in}}{\pgfqpoint{2.581134in}{1.078186in}}{\pgfqpoint{2.592184in}{1.078186in}}%
\pgfpathclose%
\pgfusepath{stroke,fill}%
\end{pgfscope}%
\begin{pgfscope}%
\pgfpathrectangle{\pgfqpoint{0.511823in}{0.504323in}}{\pgfqpoint{3.218177in}{3.225677in}} %
\pgfusepath{clip}%
\pgfsetbuttcap%
\pgfsetroundjoin%
\definecolor{currentfill}{rgb}{0.000000,0.000000,0.545098}%
\pgfsetfillcolor{currentfill}%
\pgfsetfillopacity{0.400000}%
\pgfsetlinewidth{0.501875pt}%
\definecolor{currentstroke}{rgb}{0.000000,0.000000,0.545098}%
\pgfsetstrokecolor{currentstroke}%
\pgfsetstrokeopacity{0.400000}%
\pgfsetdash{}{0pt}%
\pgfpathmoveto{\pgfqpoint{2.539701in}{1.071876in}}%
\pgfpathcurveto{\pgfqpoint{2.550751in}{1.071876in}}{\pgfqpoint{2.561350in}{1.076267in}}{\pgfqpoint{2.569164in}{1.084080in}}%
\pgfpathcurveto{\pgfqpoint{2.576978in}{1.091894in}}{\pgfqpoint{2.581368in}{1.102493in}}{\pgfqpoint{2.581368in}{1.113543in}}%
\pgfpathcurveto{\pgfqpoint{2.581368in}{1.124593in}}{\pgfqpoint{2.576978in}{1.135192in}}{\pgfqpoint{2.569164in}{1.143006in}}%
\pgfpathcurveto{\pgfqpoint{2.561350in}{1.150819in}}{\pgfqpoint{2.550751in}{1.155210in}}{\pgfqpoint{2.539701in}{1.155210in}}%
\pgfpathcurveto{\pgfqpoint{2.528651in}{1.155210in}}{\pgfqpoint{2.518052in}{1.150819in}}{\pgfqpoint{2.510238in}{1.143006in}}%
\pgfpathcurveto{\pgfqpoint{2.502425in}{1.135192in}}{\pgfqpoint{2.498035in}{1.124593in}}{\pgfqpoint{2.498035in}{1.113543in}}%
\pgfpathcurveto{\pgfqpoint{2.498035in}{1.102493in}}{\pgfqpoint{2.502425in}{1.091894in}}{\pgfqpoint{2.510238in}{1.084080in}}%
\pgfpathcurveto{\pgfqpoint{2.518052in}{1.076267in}}{\pgfqpoint{2.528651in}{1.071876in}}{\pgfqpoint{2.539701in}{1.071876in}}%
\pgfpathclose%
\pgfusepath{stroke,fill}%
\end{pgfscope}%
\begin{pgfscope}%
\pgfpathrectangle{\pgfqpoint{0.511823in}{0.504323in}}{\pgfqpoint{3.218177in}{3.225677in}} %
\pgfusepath{clip}%
\pgfsetbuttcap%
\pgfsetroundjoin%
\definecolor{currentfill}{rgb}{0.000000,0.000000,0.545098}%
\pgfsetfillcolor{currentfill}%
\pgfsetfillopacity{0.400000}%
\pgfsetlinewidth{0.501875pt}%
\definecolor{currentstroke}{rgb}{0.000000,0.000000,0.545098}%
\pgfsetstrokecolor{currentstroke}%
\pgfsetstrokeopacity{0.400000}%
\pgfsetdash{}{0pt}%
\pgfpathmoveto{\pgfqpoint{2.583869in}{1.088086in}}%
\pgfpathcurveto{\pgfqpoint{2.594919in}{1.088086in}}{\pgfqpoint{2.605518in}{1.092476in}}{\pgfqpoint{2.613332in}{1.100290in}}%
\pgfpathcurveto{\pgfqpoint{2.621145in}{1.108103in}}{\pgfqpoint{2.625536in}{1.118703in}}{\pgfqpoint{2.625536in}{1.129753in}}%
\pgfpathcurveto{\pgfqpoint{2.625536in}{1.140803in}}{\pgfqpoint{2.621145in}{1.151402in}}{\pgfqpoint{2.613332in}{1.159215in}}%
\pgfpathcurveto{\pgfqpoint{2.605518in}{1.167029in}}{\pgfqpoint{2.594919in}{1.171419in}}{\pgfqpoint{2.583869in}{1.171419in}}%
\pgfpathcurveto{\pgfqpoint{2.572819in}{1.171419in}}{\pgfqpoint{2.562220in}{1.167029in}}{\pgfqpoint{2.554406in}{1.159215in}}%
\pgfpathcurveto{\pgfqpoint{2.546593in}{1.151402in}}{\pgfqpoint{2.542202in}{1.140803in}}{\pgfqpoint{2.542202in}{1.129753in}}%
\pgfpathcurveto{\pgfqpoint{2.542202in}{1.118703in}}{\pgfqpoint{2.546593in}{1.108103in}}{\pgfqpoint{2.554406in}{1.100290in}}%
\pgfpathcurveto{\pgfqpoint{2.562220in}{1.092476in}}{\pgfqpoint{2.572819in}{1.088086in}}{\pgfqpoint{2.583869in}{1.088086in}}%
\pgfpathclose%
\pgfusepath{stroke,fill}%
\end{pgfscope}%
\begin{pgfscope}%
\pgfpathrectangle{\pgfqpoint{0.511823in}{0.504323in}}{\pgfqpoint{3.218177in}{3.225677in}} %
\pgfusepath{clip}%
\pgfsetbuttcap%
\pgfsetroundjoin%
\definecolor{currentfill}{rgb}{0.000000,0.000000,0.545098}%
\pgfsetfillcolor{currentfill}%
\pgfsetfillopacity{0.400000}%
\pgfsetlinewidth{0.501875pt}%
\definecolor{currentstroke}{rgb}{0.000000,0.000000,0.545098}%
\pgfsetstrokecolor{currentstroke}%
\pgfsetstrokeopacity{0.400000}%
\pgfsetdash{}{0pt}%
\pgfpathmoveto{\pgfqpoint{2.881946in}{1.165366in}}%
\pgfpathcurveto{\pgfqpoint{2.892996in}{1.165366in}}{\pgfqpoint{2.903596in}{1.169757in}}{\pgfqpoint{2.911409in}{1.177570in}}%
\pgfpathcurveto{\pgfqpoint{2.919223in}{1.185384in}}{\pgfqpoint{2.923613in}{1.195983in}}{\pgfqpoint{2.923613in}{1.207033in}}%
\pgfpathcurveto{\pgfqpoint{2.923613in}{1.218083in}}{\pgfqpoint{2.919223in}{1.228682in}}{\pgfqpoint{2.911409in}{1.236496in}}%
\pgfpathcurveto{\pgfqpoint{2.903596in}{1.244310in}}{\pgfqpoint{2.892996in}{1.248700in}}{\pgfqpoint{2.881946in}{1.248700in}}%
\pgfpathcurveto{\pgfqpoint{2.870896in}{1.248700in}}{\pgfqpoint{2.860297in}{1.244310in}}{\pgfqpoint{2.852484in}{1.236496in}}%
\pgfpathcurveto{\pgfqpoint{2.844670in}{1.228682in}}{\pgfqpoint{2.840280in}{1.218083in}}{\pgfqpoint{2.840280in}{1.207033in}}%
\pgfpathcurveto{\pgfqpoint{2.840280in}{1.195983in}}{\pgfqpoint{2.844670in}{1.185384in}}{\pgfqpoint{2.852484in}{1.177570in}}%
\pgfpathcurveto{\pgfqpoint{2.860297in}{1.169757in}}{\pgfqpoint{2.870896in}{1.165366in}}{\pgfqpoint{2.881946in}{1.165366in}}%
\pgfpathclose%
\pgfusepath{stroke,fill}%
\end{pgfscope}%
\begin{pgfscope}%
\pgfpathrectangle{\pgfqpoint{0.511823in}{0.504323in}}{\pgfqpoint{3.218177in}{3.225677in}} %
\pgfusepath{clip}%
\pgfsetbuttcap%
\pgfsetroundjoin%
\definecolor{currentfill}{rgb}{0.000000,0.000000,0.545098}%
\pgfsetfillcolor{currentfill}%
\pgfsetfillopacity{0.400000}%
\pgfsetlinewidth{0.501875pt}%
\definecolor{currentstroke}{rgb}{0.000000,0.000000,0.545098}%
\pgfsetstrokecolor{currentstroke}%
\pgfsetstrokeopacity{0.400000}%
\pgfsetdash{}{0pt}%
\pgfpathmoveto{\pgfqpoint{2.656582in}{1.117572in}}%
\pgfpathcurveto{\pgfqpoint{2.667632in}{1.117572in}}{\pgfqpoint{2.678231in}{1.121962in}}{\pgfqpoint{2.686045in}{1.129776in}}%
\pgfpathcurveto{\pgfqpoint{2.693858in}{1.137589in}}{\pgfqpoint{2.698249in}{1.148188in}}{\pgfqpoint{2.698249in}{1.159238in}}%
\pgfpathcurveto{\pgfqpoint{2.698249in}{1.170288in}}{\pgfqpoint{2.693858in}{1.180887in}}{\pgfqpoint{2.686045in}{1.188701in}}%
\pgfpathcurveto{\pgfqpoint{2.678231in}{1.196515in}}{\pgfqpoint{2.667632in}{1.200905in}}{\pgfqpoint{2.656582in}{1.200905in}}%
\pgfpathcurveto{\pgfqpoint{2.645532in}{1.200905in}}{\pgfqpoint{2.634933in}{1.196515in}}{\pgfqpoint{2.627119in}{1.188701in}}%
\pgfpathcurveto{\pgfqpoint{2.619306in}{1.180887in}}{\pgfqpoint{2.614915in}{1.170288in}}{\pgfqpoint{2.614915in}{1.159238in}}%
\pgfpathcurveto{\pgfqpoint{2.614915in}{1.148188in}}{\pgfqpoint{2.619306in}{1.137589in}}{\pgfqpoint{2.627119in}{1.129776in}}%
\pgfpathcurveto{\pgfqpoint{2.634933in}{1.121962in}}{\pgfqpoint{2.645532in}{1.117572in}}{\pgfqpoint{2.656582in}{1.117572in}}%
\pgfpathclose%
\pgfusepath{stroke,fill}%
\end{pgfscope}%
\begin{pgfscope}%
\pgfpathrectangle{\pgfqpoint{0.511823in}{0.504323in}}{\pgfqpoint{3.218177in}{3.225677in}} %
\pgfusepath{clip}%
\pgfsetbuttcap%
\pgfsetroundjoin%
\definecolor{currentfill}{rgb}{0.000000,0.000000,0.545098}%
\pgfsetfillcolor{currentfill}%
\pgfsetfillopacity{0.400000}%
\pgfsetlinewidth{0.501875pt}%
\definecolor{currentstroke}{rgb}{0.000000,0.000000,0.545098}%
\pgfsetstrokecolor{currentstroke}%
\pgfsetstrokeopacity{0.400000}%
\pgfsetdash{}{0pt}%
\pgfpathmoveto{\pgfqpoint{2.706110in}{1.135906in}}%
\pgfpathcurveto{\pgfqpoint{2.717160in}{1.135906in}}{\pgfqpoint{2.727759in}{1.140296in}}{\pgfqpoint{2.735573in}{1.148110in}}%
\pgfpathcurveto{\pgfqpoint{2.743386in}{1.155924in}}{\pgfqpoint{2.747777in}{1.166523in}}{\pgfqpoint{2.747777in}{1.177573in}}%
\pgfpathcurveto{\pgfqpoint{2.747777in}{1.188623in}}{\pgfqpoint{2.743386in}{1.199222in}}{\pgfqpoint{2.735573in}{1.207036in}}%
\pgfpathcurveto{\pgfqpoint{2.727759in}{1.214849in}}{\pgfqpoint{2.717160in}{1.219239in}}{\pgfqpoint{2.706110in}{1.219239in}}%
\pgfpathcurveto{\pgfqpoint{2.695060in}{1.219239in}}{\pgfqpoint{2.684461in}{1.214849in}}{\pgfqpoint{2.676647in}{1.207036in}}%
\pgfpathcurveto{\pgfqpoint{2.668834in}{1.199222in}}{\pgfqpoint{2.664443in}{1.188623in}}{\pgfqpoint{2.664443in}{1.177573in}}%
\pgfpathcurveto{\pgfqpoint{2.664443in}{1.166523in}}{\pgfqpoint{2.668834in}{1.155924in}}{\pgfqpoint{2.676647in}{1.148110in}}%
\pgfpathcurveto{\pgfqpoint{2.684461in}{1.140296in}}{\pgfqpoint{2.695060in}{1.135906in}}{\pgfqpoint{2.706110in}{1.135906in}}%
\pgfpathclose%
\pgfusepath{stroke,fill}%
\end{pgfscope}%
\begin{pgfscope}%
\pgfpathrectangle{\pgfqpoint{0.511823in}{0.504323in}}{\pgfqpoint{3.218177in}{3.225677in}} %
\pgfusepath{clip}%
\pgfsetbuttcap%
\pgfsetroundjoin%
\definecolor{currentfill}{rgb}{0.000000,0.000000,0.545098}%
\pgfsetfillcolor{currentfill}%
\pgfsetfillopacity{0.400000}%
\pgfsetlinewidth{0.501875pt}%
\definecolor{currentstroke}{rgb}{0.000000,0.000000,0.545098}%
\pgfsetstrokecolor{currentstroke}%
\pgfsetstrokeopacity{0.400000}%
\pgfsetdash{}{0pt}%
\pgfpathmoveto{\pgfqpoint{2.708418in}{1.142813in}}%
\pgfpathcurveto{\pgfqpoint{2.719468in}{1.142813in}}{\pgfqpoint{2.730067in}{1.147204in}}{\pgfqpoint{2.737880in}{1.155017in}}%
\pgfpathcurveto{\pgfqpoint{2.745694in}{1.162831in}}{\pgfqpoint{2.750084in}{1.173430in}}{\pgfqpoint{2.750084in}{1.184480in}}%
\pgfpathcurveto{\pgfqpoint{2.750084in}{1.195530in}}{\pgfqpoint{2.745694in}{1.206129in}}{\pgfqpoint{2.737880in}{1.213943in}}%
\pgfpathcurveto{\pgfqpoint{2.730067in}{1.221757in}}{\pgfqpoint{2.719468in}{1.226147in}}{\pgfqpoint{2.708418in}{1.226147in}}%
\pgfpathcurveto{\pgfqpoint{2.697367in}{1.226147in}}{\pgfqpoint{2.686768in}{1.221757in}}{\pgfqpoint{2.678955in}{1.213943in}}%
\pgfpathcurveto{\pgfqpoint{2.671141in}{1.206129in}}{\pgfqpoint{2.666751in}{1.195530in}}{\pgfqpoint{2.666751in}{1.184480in}}%
\pgfpathcurveto{\pgfqpoint{2.666751in}{1.173430in}}{\pgfqpoint{2.671141in}{1.162831in}}{\pgfqpoint{2.678955in}{1.155017in}}%
\pgfpathcurveto{\pgfqpoint{2.686768in}{1.147204in}}{\pgfqpoint{2.697367in}{1.142813in}}{\pgfqpoint{2.708418in}{1.142813in}}%
\pgfpathclose%
\pgfusepath{stroke,fill}%
\end{pgfscope}%
\begin{pgfscope}%
\pgfpathrectangle{\pgfqpoint{0.511823in}{0.504323in}}{\pgfqpoint{3.218177in}{3.225677in}} %
\pgfusepath{clip}%
\pgfsetbuttcap%
\pgfsetroundjoin%
\definecolor{currentfill}{rgb}{0.000000,0.000000,0.545098}%
\pgfsetfillcolor{currentfill}%
\pgfsetfillopacity{0.400000}%
\pgfsetlinewidth{0.501875pt}%
\definecolor{currentstroke}{rgb}{0.000000,0.000000,0.545098}%
\pgfsetstrokecolor{currentstroke}%
\pgfsetstrokeopacity{0.400000}%
\pgfsetdash{}{0pt}%
\pgfpathmoveto{\pgfqpoint{2.737632in}{1.156525in}}%
\pgfpathcurveto{\pgfqpoint{2.748682in}{1.156525in}}{\pgfqpoint{2.759281in}{1.160915in}}{\pgfqpoint{2.767095in}{1.168728in}}%
\pgfpathcurveto{\pgfqpoint{2.774908in}{1.176542in}}{\pgfqpoint{2.779299in}{1.187141in}}{\pgfqpoint{2.779299in}{1.198191in}}%
\pgfpathcurveto{\pgfqpoint{2.779299in}{1.209241in}}{\pgfqpoint{2.774908in}{1.219840in}}{\pgfqpoint{2.767095in}{1.227654in}}%
\pgfpathcurveto{\pgfqpoint{2.759281in}{1.235468in}}{\pgfqpoint{2.748682in}{1.239858in}}{\pgfqpoint{2.737632in}{1.239858in}}%
\pgfpathcurveto{\pgfqpoint{2.726582in}{1.239858in}}{\pgfqpoint{2.715983in}{1.235468in}}{\pgfqpoint{2.708169in}{1.227654in}}%
\pgfpathcurveto{\pgfqpoint{2.700356in}{1.219840in}}{\pgfqpoint{2.695965in}{1.209241in}}{\pgfqpoint{2.695965in}{1.198191in}}%
\pgfpathcurveto{\pgfqpoint{2.695965in}{1.187141in}}{\pgfqpoint{2.700356in}{1.176542in}}{\pgfqpoint{2.708169in}{1.168728in}}%
\pgfpathcurveto{\pgfqpoint{2.715983in}{1.160915in}}{\pgfqpoint{2.726582in}{1.156525in}}{\pgfqpoint{2.737632in}{1.156525in}}%
\pgfpathclose%
\pgfusepath{stroke,fill}%
\end{pgfscope}%
\begin{pgfscope}%
\pgfpathrectangle{\pgfqpoint{0.511823in}{0.504323in}}{\pgfqpoint{3.218177in}{3.225677in}} %
\pgfusepath{clip}%
\pgfsetbuttcap%
\pgfsetroundjoin%
\definecolor{currentfill}{rgb}{0.000000,0.000000,0.545098}%
\pgfsetfillcolor{currentfill}%
\pgfsetfillopacity{0.400000}%
\pgfsetlinewidth{0.501875pt}%
\definecolor{currentstroke}{rgb}{0.000000,0.000000,0.545098}%
\pgfsetstrokecolor{currentstroke}%
\pgfsetstrokeopacity{0.400000}%
\pgfsetdash{}{0pt}%
\pgfpathmoveto{\pgfqpoint{2.741366in}{1.163931in}}%
\pgfpathcurveto{\pgfqpoint{2.752416in}{1.163931in}}{\pgfqpoint{2.763015in}{1.168321in}}{\pgfqpoint{2.770829in}{1.176135in}}%
\pgfpathcurveto{\pgfqpoint{2.778642in}{1.183948in}}{\pgfqpoint{2.783033in}{1.194548in}}{\pgfqpoint{2.783033in}{1.205598in}}%
\pgfpathcurveto{\pgfqpoint{2.783033in}{1.216648in}}{\pgfqpoint{2.778642in}{1.227247in}}{\pgfqpoint{2.770829in}{1.235060in}}%
\pgfpathcurveto{\pgfqpoint{2.763015in}{1.242874in}}{\pgfqpoint{2.752416in}{1.247264in}}{\pgfqpoint{2.741366in}{1.247264in}}%
\pgfpathcurveto{\pgfqpoint{2.730316in}{1.247264in}}{\pgfqpoint{2.719717in}{1.242874in}}{\pgfqpoint{2.711903in}{1.235060in}}%
\pgfpathcurveto{\pgfqpoint{2.704090in}{1.227247in}}{\pgfqpoint{2.699699in}{1.216648in}}{\pgfqpoint{2.699699in}{1.205598in}}%
\pgfpathcurveto{\pgfqpoint{2.699699in}{1.194548in}}{\pgfqpoint{2.704090in}{1.183948in}}{\pgfqpoint{2.711903in}{1.176135in}}%
\pgfpathcurveto{\pgfqpoint{2.719717in}{1.168321in}}{\pgfqpoint{2.730316in}{1.163931in}}{\pgfqpoint{2.741366in}{1.163931in}}%
\pgfpathclose%
\pgfusepath{stroke,fill}%
\end{pgfscope}%
\begin{pgfscope}%
\pgfpathrectangle{\pgfqpoint{0.511823in}{0.504323in}}{\pgfqpoint{3.218177in}{3.225677in}} %
\pgfusepath{clip}%
\pgfsetbuttcap%
\pgfsetroundjoin%
\definecolor{currentfill}{rgb}{0.000000,0.000000,0.545098}%
\pgfsetfillcolor{currentfill}%
\pgfsetfillopacity{0.400000}%
\pgfsetlinewidth{0.501875pt}%
\definecolor{currentstroke}{rgb}{0.000000,0.000000,0.545098}%
\pgfsetstrokecolor{currentstroke}%
\pgfsetstrokeopacity{0.400000}%
\pgfsetdash{}{0pt}%
\pgfpathmoveto{\pgfqpoint{2.513089in}{1.111459in}}%
\pgfpathcurveto{\pgfqpoint{2.524139in}{1.111459in}}{\pgfqpoint{2.534738in}{1.115850in}}{\pgfqpoint{2.542552in}{1.123663in}}%
\pgfpathcurveto{\pgfqpoint{2.550365in}{1.131477in}}{\pgfqpoint{2.554755in}{1.142076in}}{\pgfqpoint{2.554755in}{1.153126in}}%
\pgfpathcurveto{\pgfqpoint{2.554755in}{1.164176in}}{\pgfqpoint{2.550365in}{1.174775in}}{\pgfqpoint{2.542552in}{1.182589in}}%
\pgfpathcurveto{\pgfqpoint{2.534738in}{1.190402in}}{\pgfqpoint{2.524139in}{1.194793in}}{\pgfqpoint{2.513089in}{1.194793in}}%
\pgfpathcurveto{\pgfqpoint{2.502039in}{1.194793in}}{\pgfqpoint{2.491440in}{1.190402in}}{\pgfqpoint{2.483626in}{1.182589in}}%
\pgfpathcurveto{\pgfqpoint{2.475812in}{1.174775in}}{\pgfqpoint{2.471422in}{1.164176in}}{\pgfqpoint{2.471422in}{1.153126in}}%
\pgfpathcurveto{\pgfqpoint{2.471422in}{1.142076in}}{\pgfqpoint{2.475812in}{1.131477in}}{\pgfqpoint{2.483626in}{1.123663in}}%
\pgfpathcurveto{\pgfqpoint{2.491440in}{1.115850in}}{\pgfqpoint{2.502039in}{1.111459in}}{\pgfqpoint{2.513089in}{1.111459in}}%
\pgfpathclose%
\pgfusepath{stroke,fill}%
\end{pgfscope}%
\begin{pgfscope}%
\pgfpathrectangle{\pgfqpoint{0.511823in}{0.504323in}}{\pgfqpoint{3.218177in}{3.225677in}} %
\pgfusepath{clip}%
\pgfsetbuttcap%
\pgfsetroundjoin%
\definecolor{currentfill}{rgb}{0.000000,0.000000,0.545098}%
\pgfsetfillcolor{currentfill}%
\pgfsetfillopacity{0.400000}%
\pgfsetlinewidth{0.501875pt}%
\definecolor{currentstroke}{rgb}{0.000000,0.000000,0.545098}%
\pgfsetstrokecolor{currentstroke}%
\pgfsetstrokeopacity{0.400000}%
\pgfsetdash{}{0pt}%
\pgfpathmoveto{\pgfqpoint{2.574740in}{1.133342in}}%
\pgfpathcurveto{\pgfqpoint{2.585790in}{1.133342in}}{\pgfqpoint{2.596390in}{1.137732in}}{\pgfqpoint{2.604203in}{1.145546in}}%
\pgfpathcurveto{\pgfqpoint{2.612017in}{1.153359in}}{\pgfqpoint{2.616407in}{1.163958in}}{\pgfqpoint{2.616407in}{1.175008in}}%
\pgfpathcurveto{\pgfqpoint{2.616407in}{1.186059in}}{\pgfqpoint{2.612017in}{1.196658in}}{\pgfqpoint{2.604203in}{1.204471in}}%
\pgfpathcurveto{\pgfqpoint{2.596390in}{1.212285in}}{\pgfqpoint{2.585790in}{1.216675in}}{\pgfqpoint{2.574740in}{1.216675in}}%
\pgfpathcurveto{\pgfqpoint{2.563690in}{1.216675in}}{\pgfqpoint{2.553091in}{1.212285in}}{\pgfqpoint{2.545278in}{1.204471in}}%
\pgfpathcurveto{\pgfqpoint{2.537464in}{1.196658in}}{\pgfqpoint{2.533074in}{1.186059in}}{\pgfqpoint{2.533074in}{1.175008in}}%
\pgfpathcurveto{\pgfqpoint{2.533074in}{1.163958in}}{\pgfqpoint{2.537464in}{1.153359in}}{\pgfqpoint{2.545278in}{1.145546in}}%
\pgfpathcurveto{\pgfqpoint{2.553091in}{1.137732in}}{\pgfqpoint{2.563690in}{1.133342in}}{\pgfqpoint{2.574740in}{1.133342in}}%
\pgfpathclose%
\pgfusepath{stroke,fill}%
\end{pgfscope}%
\begin{pgfscope}%
\pgfpathrectangle{\pgfqpoint{0.511823in}{0.504323in}}{\pgfqpoint{3.218177in}{3.225677in}} %
\pgfusepath{clip}%
\pgfsetbuttcap%
\pgfsetroundjoin%
\definecolor{currentfill}{rgb}{0.000000,0.000000,0.545098}%
\pgfsetfillcolor{currentfill}%
\pgfsetfillopacity{0.400000}%
\pgfsetlinewidth{0.501875pt}%
\definecolor{currentstroke}{rgb}{0.000000,0.000000,0.545098}%
\pgfsetstrokecolor{currentstroke}%
\pgfsetstrokeopacity{0.400000}%
\pgfsetdash{}{0pt}%
\pgfpathmoveto{\pgfqpoint{2.503443in}{1.120453in}}%
\pgfpathcurveto{\pgfqpoint{2.514493in}{1.120453in}}{\pgfqpoint{2.525092in}{1.124843in}}{\pgfqpoint{2.532906in}{1.132656in}}%
\pgfpathcurveto{\pgfqpoint{2.540720in}{1.140470in}}{\pgfqpoint{2.545110in}{1.151069in}}{\pgfqpoint{2.545110in}{1.162119in}}%
\pgfpathcurveto{\pgfqpoint{2.545110in}{1.173169in}}{\pgfqpoint{2.540720in}{1.183768in}}{\pgfqpoint{2.532906in}{1.191582in}}%
\pgfpathcurveto{\pgfqpoint{2.525092in}{1.199396in}}{\pgfqpoint{2.514493in}{1.203786in}}{\pgfqpoint{2.503443in}{1.203786in}}%
\pgfpathcurveto{\pgfqpoint{2.492393in}{1.203786in}}{\pgfqpoint{2.481794in}{1.199396in}}{\pgfqpoint{2.473980in}{1.191582in}}%
\pgfpathcurveto{\pgfqpoint{2.466167in}{1.183768in}}{\pgfqpoint{2.461776in}{1.173169in}}{\pgfqpoint{2.461776in}{1.162119in}}%
\pgfpathcurveto{\pgfqpoint{2.461776in}{1.151069in}}{\pgfqpoint{2.466167in}{1.140470in}}{\pgfqpoint{2.473980in}{1.132656in}}%
\pgfpathcurveto{\pgfqpoint{2.481794in}{1.124843in}}{\pgfqpoint{2.492393in}{1.120453in}}{\pgfqpoint{2.503443in}{1.120453in}}%
\pgfpathclose%
\pgfusepath{stroke,fill}%
\end{pgfscope}%
\begin{pgfscope}%
\pgfpathrectangle{\pgfqpoint{0.511823in}{0.504323in}}{\pgfqpoint{3.218177in}{3.225677in}} %
\pgfusepath{clip}%
\pgfsetbuttcap%
\pgfsetroundjoin%
\definecolor{currentfill}{rgb}{0.000000,0.000000,0.545098}%
\pgfsetfillcolor{currentfill}%
\pgfsetfillopacity{0.400000}%
\pgfsetlinewidth{0.501875pt}%
\definecolor{currentstroke}{rgb}{0.000000,0.000000,0.545098}%
\pgfsetstrokecolor{currentstroke}%
\pgfsetstrokeopacity{0.400000}%
\pgfsetdash{}{0pt}%
\pgfpathmoveto{\pgfqpoint{2.508131in}{1.127465in}}%
\pgfpathcurveto{\pgfqpoint{2.519181in}{1.127465in}}{\pgfqpoint{2.529780in}{1.131855in}}{\pgfqpoint{2.537594in}{1.139669in}}%
\pgfpathcurveto{\pgfqpoint{2.545407in}{1.147483in}}{\pgfqpoint{2.549798in}{1.158082in}}{\pgfqpoint{2.549798in}{1.169132in}}%
\pgfpathcurveto{\pgfqpoint{2.549798in}{1.180182in}}{\pgfqpoint{2.545407in}{1.190781in}}{\pgfqpoint{2.537594in}{1.198594in}}%
\pgfpathcurveto{\pgfqpoint{2.529780in}{1.206408in}}{\pgfqpoint{2.519181in}{1.210798in}}{\pgfqpoint{2.508131in}{1.210798in}}%
\pgfpathcurveto{\pgfqpoint{2.497081in}{1.210798in}}{\pgfqpoint{2.486482in}{1.206408in}}{\pgfqpoint{2.478668in}{1.198594in}}%
\pgfpathcurveto{\pgfqpoint{2.470855in}{1.190781in}}{\pgfqpoint{2.466464in}{1.180182in}}{\pgfqpoint{2.466464in}{1.169132in}}%
\pgfpathcurveto{\pgfqpoint{2.466464in}{1.158082in}}{\pgfqpoint{2.470855in}{1.147483in}}{\pgfqpoint{2.478668in}{1.139669in}}%
\pgfpathcurveto{\pgfqpoint{2.486482in}{1.131855in}}{\pgfqpoint{2.497081in}{1.127465in}}{\pgfqpoint{2.508131in}{1.127465in}}%
\pgfpathclose%
\pgfusepath{stroke,fill}%
\end{pgfscope}%
\begin{pgfscope}%
\pgfpathrectangle{\pgfqpoint{0.511823in}{0.504323in}}{\pgfqpoint{3.218177in}{3.225677in}} %
\pgfusepath{clip}%
\pgfsetbuttcap%
\pgfsetroundjoin%
\definecolor{currentfill}{rgb}{0.000000,0.000000,0.545098}%
\pgfsetfillcolor{currentfill}%
\pgfsetfillopacity{0.400000}%
\pgfsetlinewidth{0.501875pt}%
\definecolor{currentstroke}{rgb}{0.000000,0.000000,0.545098}%
\pgfsetstrokecolor{currentstroke}%
\pgfsetstrokeopacity{0.400000}%
\pgfsetdash{}{0pt}%
\pgfpathmoveto{\pgfqpoint{2.742455in}{1.196725in}}%
\pgfpathcurveto{\pgfqpoint{2.753505in}{1.196725in}}{\pgfqpoint{2.764104in}{1.201115in}}{\pgfqpoint{2.771918in}{1.208929in}}%
\pgfpathcurveto{\pgfqpoint{2.779731in}{1.216742in}}{\pgfqpoint{2.784122in}{1.227341in}}{\pgfqpoint{2.784122in}{1.238391in}}%
\pgfpathcurveto{\pgfqpoint{2.784122in}{1.249442in}}{\pgfqpoint{2.779731in}{1.260041in}}{\pgfqpoint{2.771918in}{1.267854in}}%
\pgfpathcurveto{\pgfqpoint{2.764104in}{1.275668in}}{\pgfqpoint{2.753505in}{1.280058in}}{\pgfqpoint{2.742455in}{1.280058in}}%
\pgfpathcurveto{\pgfqpoint{2.731405in}{1.280058in}}{\pgfqpoint{2.720806in}{1.275668in}}{\pgfqpoint{2.712992in}{1.267854in}}%
\pgfpathcurveto{\pgfqpoint{2.705179in}{1.260041in}}{\pgfqpoint{2.700788in}{1.249442in}}{\pgfqpoint{2.700788in}{1.238391in}}%
\pgfpathcurveto{\pgfqpoint{2.700788in}{1.227341in}}{\pgfqpoint{2.705179in}{1.216742in}}{\pgfqpoint{2.712992in}{1.208929in}}%
\pgfpathcurveto{\pgfqpoint{2.720806in}{1.201115in}}{\pgfqpoint{2.731405in}{1.196725in}}{\pgfqpoint{2.742455in}{1.196725in}}%
\pgfpathclose%
\pgfusepath{stroke,fill}%
\end{pgfscope}%
\begin{pgfscope}%
\pgfpathrectangle{\pgfqpoint{0.511823in}{0.504323in}}{\pgfqpoint{3.218177in}{3.225677in}} %
\pgfusepath{clip}%
\pgfsetbuttcap%
\pgfsetroundjoin%
\definecolor{currentfill}{rgb}{0.000000,0.000000,0.545098}%
\pgfsetfillcolor{currentfill}%
\pgfsetfillopacity{0.400000}%
\pgfsetlinewidth{0.501875pt}%
\definecolor{currentstroke}{rgb}{0.000000,0.000000,0.545098}%
\pgfsetstrokecolor{currentstroke}%
\pgfsetstrokeopacity{0.400000}%
\pgfsetdash{}{0pt}%
\pgfpathmoveto{\pgfqpoint{2.531444in}{1.145430in}}%
\pgfpathcurveto{\pgfqpoint{2.542494in}{1.145430in}}{\pgfqpoint{2.553093in}{1.149821in}}{\pgfqpoint{2.560906in}{1.157634in}}%
\pgfpathcurveto{\pgfqpoint{2.568720in}{1.165448in}}{\pgfqpoint{2.573110in}{1.176047in}}{\pgfqpoint{2.573110in}{1.187097in}}%
\pgfpathcurveto{\pgfqpoint{2.573110in}{1.198147in}}{\pgfqpoint{2.568720in}{1.208746in}}{\pgfqpoint{2.560906in}{1.216560in}}%
\pgfpathcurveto{\pgfqpoint{2.553093in}{1.224373in}}{\pgfqpoint{2.542494in}{1.228764in}}{\pgfqpoint{2.531444in}{1.228764in}}%
\pgfpathcurveto{\pgfqpoint{2.520393in}{1.228764in}}{\pgfqpoint{2.509794in}{1.224373in}}{\pgfqpoint{2.501981in}{1.216560in}}%
\pgfpathcurveto{\pgfqpoint{2.494167in}{1.208746in}}{\pgfqpoint{2.489777in}{1.198147in}}{\pgfqpoint{2.489777in}{1.187097in}}%
\pgfpathcurveto{\pgfqpoint{2.489777in}{1.176047in}}{\pgfqpoint{2.494167in}{1.165448in}}{\pgfqpoint{2.501981in}{1.157634in}}%
\pgfpathcurveto{\pgfqpoint{2.509794in}{1.149821in}}{\pgfqpoint{2.520393in}{1.145430in}}{\pgfqpoint{2.531444in}{1.145430in}}%
\pgfpathclose%
\pgfusepath{stroke,fill}%
\end{pgfscope}%
\begin{pgfscope}%
\pgfpathrectangle{\pgfqpoint{0.511823in}{0.504323in}}{\pgfqpoint{3.218177in}{3.225677in}} %
\pgfusepath{clip}%
\pgfsetbuttcap%
\pgfsetroundjoin%
\definecolor{currentfill}{rgb}{0.000000,0.000000,0.545098}%
\pgfsetfillcolor{currentfill}%
\pgfsetfillopacity{0.400000}%
\pgfsetlinewidth{0.501875pt}%
\definecolor{currentstroke}{rgb}{0.000000,0.000000,0.545098}%
\pgfsetstrokecolor{currentstroke}%
\pgfsetstrokeopacity{0.400000}%
\pgfsetdash{}{0pt}%
\pgfpathmoveto{\pgfqpoint{2.583453in}{1.165729in}}%
\pgfpathcurveto{\pgfqpoint{2.594503in}{1.165729in}}{\pgfqpoint{2.605102in}{1.170119in}}{\pgfqpoint{2.612916in}{1.177933in}}%
\pgfpathcurveto{\pgfqpoint{2.620729in}{1.185746in}}{\pgfqpoint{2.625120in}{1.196345in}}{\pgfqpoint{2.625120in}{1.207395in}}%
\pgfpathcurveto{\pgfqpoint{2.625120in}{1.218446in}}{\pgfqpoint{2.620729in}{1.229045in}}{\pgfqpoint{2.612916in}{1.236858in}}%
\pgfpathcurveto{\pgfqpoint{2.605102in}{1.244672in}}{\pgfqpoint{2.594503in}{1.249062in}}{\pgfqpoint{2.583453in}{1.249062in}}%
\pgfpathcurveto{\pgfqpoint{2.572403in}{1.249062in}}{\pgfqpoint{2.561804in}{1.244672in}}{\pgfqpoint{2.553990in}{1.236858in}}%
\pgfpathcurveto{\pgfqpoint{2.546177in}{1.229045in}}{\pgfqpoint{2.541786in}{1.218446in}}{\pgfqpoint{2.541786in}{1.207395in}}%
\pgfpathcurveto{\pgfqpoint{2.541786in}{1.196345in}}{\pgfqpoint{2.546177in}{1.185746in}}{\pgfqpoint{2.553990in}{1.177933in}}%
\pgfpathcurveto{\pgfqpoint{2.561804in}{1.170119in}}{\pgfqpoint{2.572403in}{1.165729in}}{\pgfqpoint{2.583453in}{1.165729in}}%
\pgfpathclose%
\pgfusepath{stroke,fill}%
\end{pgfscope}%
\begin{pgfscope}%
\pgfpathrectangle{\pgfqpoint{0.511823in}{0.504323in}}{\pgfqpoint{3.218177in}{3.225677in}} %
\pgfusepath{clip}%
\pgfsetbuttcap%
\pgfsetroundjoin%
\definecolor{currentfill}{rgb}{0.000000,0.000000,0.545098}%
\pgfsetfillcolor{currentfill}%
\pgfsetfillopacity{0.400000}%
\pgfsetlinewidth{0.501875pt}%
\definecolor{currentstroke}{rgb}{0.000000,0.000000,0.545098}%
\pgfsetstrokecolor{currentstroke}%
\pgfsetstrokeopacity{0.400000}%
\pgfsetdash{}{0pt}%
\pgfpathmoveto{\pgfqpoint{2.503716in}{1.149422in}}%
\pgfpathcurveto{\pgfqpoint{2.514766in}{1.149422in}}{\pgfqpoint{2.525365in}{1.153812in}}{\pgfqpoint{2.533179in}{1.161626in}}%
\pgfpathcurveto{\pgfqpoint{2.540992in}{1.169439in}}{\pgfqpoint{2.545383in}{1.180038in}}{\pgfqpoint{2.545383in}{1.191088in}}%
\pgfpathcurveto{\pgfqpoint{2.545383in}{1.202139in}}{\pgfqpoint{2.540992in}{1.212738in}}{\pgfqpoint{2.533179in}{1.220551in}}%
\pgfpathcurveto{\pgfqpoint{2.525365in}{1.228365in}}{\pgfqpoint{2.514766in}{1.232755in}}{\pgfqpoint{2.503716in}{1.232755in}}%
\pgfpathcurveto{\pgfqpoint{2.492666in}{1.232755in}}{\pgfqpoint{2.482067in}{1.228365in}}{\pgfqpoint{2.474253in}{1.220551in}}%
\pgfpathcurveto{\pgfqpoint{2.466440in}{1.212738in}}{\pgfqpoint{2.462049in}{1.202139in}}{\pgfqpoint{2.462049in}{1.191088in}}%
\pgfpathcurveto{\pgfqpoint{2.462049in}{1.180038in}}{\pgfqpoint{2.466440in}{1.169439in}}{\pgfqpoint{2.474253in}{1.161626in}}%
\pgfpathcurveto{\pgfqpoint{2.482067in}{1.153812in}}{\pgfqpoint{2.492666in}{1.149422in}}{\pgfqpoint{2.503716in}{1.149422in}}%
\pgfpathclose%
\pgfusepath{stroke,fill}%
\end{pgfscope}%
\begin{pgfscope}%
\pgfpathrectangle{\pgfqpoint{0.511823in}{0.504323in}}{\pgfqpoint{3.218177in}{3.225677in}} %
\pgfusepath{clip}%
\pgfsetbuttcap%
\pgfsetroundjoin%
\definecolor{currentfill}{rgb}{0.000000,0.000000,0.545098}%
\pgfsetfillcolor{currentfill}%
\pgfsetfillopacity{0.400000}%
\pgfsetlinewidth{0.501875pt}%
\definecolor{currentstroke}{rgb}{0.000000,0.000000,0.545098}%
\pgfsetstrokecolor{currentstroke}%
\pgfsetstrokeopacity{0.400000}%
\pgfsetdash{}{0pt}%
\pgfpathmoveto{\pgfqpoint{2.514599in}{1.158321in}}%
\pgfpathcurveto{\pgfqpoint{2.525649in}{1.158321in}}{\pgfqpoint{2.536248in}{1.162711in}}{\pgfqpoint{2.544062in}{1.170525in}}%
\pgfpathcurveto{\pgfqpoint{2.551876in}{1.178338in}}{\pgfqpoint{2.556266in}{1.188937in}}{\pgfqpoint{2.556266in}{1.199987in}}%
\pgfpathcurveto{\pgfqpoint{2.556266in}{1.211037in}}{\pgfqpoint{2.551876in}{1.221636in}}{\pgfqpoint{2.544062in}{1.229450in}}%
\pgfpathcurveto{\pgfqpoint{2.536248in}{1.237264in}}{\pgfqpoint{2.525649in}{1.241654in}}{\pgfqpoint{2.514599in}{1.241654in}}%
\pgfpathcurveto{\pgfqpoint{2.503549in}{1.241654in}}{\pgfqpoint{2.492950in}{1.237264in}}{\pgfqpoint{2.485137in}{1.229450in}}%
\pgfpathcurveto{\pgfqpoint{2.477323in}{1.221636in}}{\pgfqpoint{2.472933in}{1.211037in}}{\pgfqpoint{2.472933in}{1.199987in}}%
\pgfpathcurveto{\pgfqpoint{2.472933in}{1.188937in}}{\pgfqpoint{2.477323in}{1.178338in}}{\pgfqpoint{2.485137in}{1.170525in}}%
\pgfpathcurveto{\pgfqpoint{2.492950in}{1.162711in}}{\pgfqpoint{2.503549in}{1.158321in}}{\pgfqpoint{2.514599in}{1.158321in}}%
\pgfpathclose%
\pgfusepath{stroke,fill}%
\end{pgfscope}%
\begin{pgfscope}%
\pgfpathrectangle{\pgfqpoint{0.511823in}{0.504323in}}{\pgfqpoint{3.218177in}{3.225677in}} %
\pgfusepath{clip}%
\pgfsetbuttcap%
\pgfsetroundjoin%
\definecolor{currentfill}{rgb}{0.000000,0.000000,0.545098}%
\pgfsetfillcolor{currentfill}%
\pgfsetfillopacity{0.400000}%
\pgfsetlinewidth{0.501875pt}%
\definecolor{currentstroke}{rgb}{0.000000,0.000000,0.545098}%
\pgfsetstrokecolor{currentstroke}%
\pgfsetstrokeopacity{0.400000}%
\pgfsetdash{}{0pt}%
\pgfpathmoveto{\pgfqpoint{2.687993in}{1.213918in}}%
\pgfpathcurveto{\pgfqpoint{2.699044in}{1.213918in}}{\pgfqpoint{2.709643in}{1.218308in}}{\pgfqpoint{2.717456in}{1.226121in}}%
\pgfpathcurveto{\pgfqpoint{2.725270in}{1.233935in}}{\pgfqpoint{2.729660in}{1.244534in}}{\pgfqpoint{2.729660in}{1.255584in}}%
\pgfpathcurveto{\pgfqpoint{2.729660in}{1.266634in}}{\pgfqpoint{2.725270in}{1.277233in}}{\pgfqpoint{2.717456in}{1.285047in}}%
\pgfpathcurveto{\pgfqpoint{2.709643in}{1.292861in}}{\pgfqpoint{2.699044in}{1.297251in}}{\pgfqpoint{2.687993in}{1.297251in}}%
\pgfpathcurveto{\pgfqpoint{2.676943in}{1.297251in}}{\pgfqpoint{2.666344in}{1.292861in}}{\pgfqpoint{2.658531in}{1.285047in}}%
\pgfpathcurveto{\pgfqpoint{2.650717in}{1.277233in}}{\pgfqpoint{2.646327in}{1.266634in}}{\pgfqpoint{2.646327in}{1.255584in}}%
\pgfpathcurveto{\pgfqpoint{2.646327in}{1.244534in}}{\pgfqpoint{2.650717in}{1.233935in}}{\pgfqpoint{2.658531in}{1.226121in}}%
\pgfpathcurveto{\pgfqpoint{2.666344in}{1.218308in}}{\pgfqpoint{2.676943in}{1.213918in}}{\pgfqpoint{2.687993in}{1.213918in}}%
\pgfpathclose%
\pgfusepath{stroke,fill}%
\end{pgfscope}%
\begin{pgfscope}%
\pgfpathrectangle{\pgfqpoint{0.511823in}{0.504323in}}{\pgfqpoint{3.218177in}{3.225677in}} %
\pgfusepath{clip}%
\pgfsetbuttcap%
\pgfsetroundjoin%
\definecolor{currentfill}{rgb}{0.000000,0.000000,0.545098}%
\pgfsetfillcolor{currentfill}%
\pgfsetfillopacity{0.400000}%
\pgfsetlinewidth{0.501875pt}%
\definecolor{currentstroke}{rgb}{0.000000,0.000000,0.545098}%
\pgfsetstrokecolor{currentstroke}%
\pgfsetstrokeopacity{0.400000}%
\pgfsetdash{}{0pt}%
\pgfpathmoveto{\pgfqpoint{2.529795in}{1.174456in}}%
\pgfpathcurveto{\pgfqpoint{2.540845in}{1.174456in}}{\pgfqpoint{2.551444in}{1.178846in}}{\pgfqpoint{2.559258in}{1.186660in}}%
\pgfpathcurveto{\pgfqpoint{2.567072in}{1.194473in}}{\pgfqpoint{2.571462in}{1.205072in}}{\pgfqpoint{2.571462in}{1.216122in}}%
\pgfpathcurveto{\pgfqpoint{2.571462in}{1.227172in}}{\pgfqpoint{2.567072in}{1.237772in}}{\pgfqpoint{2.559258in}{1.245585in}}%
\pgfpathcurveto{\pgfqpoint{2.551444in}{1.253399in}}{\pgfqpoint{2.540845in}{1.257789in}}{\pgfqpoint{2.529795in}{1.257789in}}%
\pgfpathcurveto{\pgfqpoint{2.518745in}{1.257789in}}{\pgfqpoint{2.508146in}{1.253399in}}{\pgfqpoint{2.500333in}{1.245585in}}%
\pgfpathcurveto{\pgfqpoint{2.492519in}{1.237772in}}{\pgfqpoint{2.488129in}{1.227172in}}{\pgfqpoint{2.488129in}{1.216122in}}%
\pgfpathcurveto{\pgfqpoint{2.488129in}{1.205072in}}{\pgfqpoint{2.492519in}{1.194473in}}{\pgfqpoint{2.500333in}{1.186660in}}%
\pgfpathcurveto{\pgfqpoint{2.508146in}{1.178846in}}{\pgfqpoint{2.518745in}{1.174456in}}{\pgfqpoint{2.529795in}{1.174456in}}%
\pgfpathclose%
\pgfusepath{stroke,fill}%
\end{pgfscope}%
\begin{pgfscope}%
\pgfpathrectangle{\pgfqpoint{0.511823in}{0.504323in}}{\pgfqpoint{3.218177in}{3.225677in}} %
\pgfusepath{clip}%
\pgfsetbuttcap%
\pgfsetroundjoin%
\definecolor{currentfill}{rgb}{0.000000,0.000000,0.545098}%
\pgfsetfillcolor{currentfill}%
\pgfsetfillopacity{0.400000}%
\pgfsetlinewidth{0.501875pt}%
\definecolor{currentstroke}{rgb}{0.000000,0.000000,0.545098}%
\pgfsetstrokecolor{currentstroke}%
\pgfsetstrokeopacity{0.400000}%
\pgfsetdash{}{0pt}%
\pgfpathmoveto{\pgfqpoint{2.523335in}{1.178491in}}%
\pgfpathcurveto{\pgfqpoint{2.534385in}{1.178491in}}{\pgfqpoint{2.544984in}{1.182881in}}{\pgfqpoint{2.552798in}{1.190695in}}%
\pgfpathcurveto{\pgfqpoint{2.560611in}{1.198508in}}{\pgfqpoint{2.565002in}{1.209107in}}{\pgfqpoint{2.565002in}{1.220157in}}%
\pgfpathcurveto{\pgfqpoint{2.565002in}{1.231207in}}{\pgfqpoint{2.560611in}{1.241807in}}{\pgfqpoint{2.552798in}{1.249620in}}%
\pgfpathcurveto{\pgfqpoint{2.544984in}{1.257434in}}{\pgfqpoint{2.534385in}{1.261824in}}{\pgfqpoint{2.523335in}{1.261824in}}%
\pgfpathcurveto{\pgfqpoint{2.512285in}{1.261824in}}{\pgfqpoint{2.501686in}{1.257434in}}{\pgfqpoint{2.493872in}{1.249620in}}%
\pgfpathcurveto{\pgfqpoint{2.486059in}{1.241807in}}{\pgfqpoint{2.481668in}{1.231207in}}{\pgfqpoint{2.481668in}{1.220157in}}%
\pgfpathcurveto{\pgfqpoint{2.481668in}{1.209107in}}{\pgfqpoint{2.486059in}{1.198508in}}{\pgfqpoint{2.493872in}{1.190695in}}%
\pgfpathcurveto{\pgfqpoint{2.501686in}{1.182881in}}{\pgfqpoint{2.512285in}{1.178491in}}{\pgfqpoint{2.523335in}{1.178491in}}%
\pgfpathclose%
\pgfusepath{stroke,fill}%
\end{pgfscope}%
\begin{pgfscope}%
\pgfpathrectangle{\pgfqpoint{0.511823in}{0.504323in}}{\pgfqpoint{3.218177in}{3.225677in}} %
\pgfusepath{clip}%
\pgfsetbuttcap%
\pgfsetroundjoin%
\definecolor{currentfill}{rgb}{0.000000,0.000000,0.545098}%
\pgfsetfillcolor{currentfill}%
\pgfsetfillopacity{0.400000}%
\pgfsetlinewidth{0.501875pt}%
\definecolor{currentstroke}{rgb}{0.000000,0.000000,0.545098}%
\pgfsetstrokecolor{currentstroke}%
\pgfsetstrokeopacity{0.400000}%
\pgfsetdash{}{0pt}%
\pgfpathmoveto{\pgfqpoint{2.644829in}{1.220435in}}%
\pgfpathcurveto{\pgfqpoint{2.655879in}{1.220435in}}{\pgfqpoint{2.666478in}{1.224825in}}{\pgfqpoint{2.674292in}{1.232639in}}%
\pgfpathcurveto{\pgfqpoint{2.682106in}{1.240453in}}{\pgfqpoint{2.686496in}{1.251052in}}{\pgfqpoint{2.686496in}{1.262102in}}%
\pgfpathcurveto{\pgfqpoint{2.686496in}{1.273152in}}{\pgfqpoint{2.682106in}{1.283751in}}{\pgfqpoint{2.674292in}{1.291565in}}%
\pgfpathcurveto{\pgfqpoint{2.666478in}{1.299378in}}{\pgfqpoint{2.655879in}{1.303768in}}{\pgfqpoint{2.644829in}{1.303768in}}%
\pgfpathcurveto{\pgfqpoint{2.633779in}{1.303768in}}{\pgfqpoint{2.623180in}{1.299378in}}{\pgfqpoint{2.615366in}{1.291565in}}%
\pgfpathcurveto{\pgfqpoint{2.607553in}{1.283751in}}{\pgfqpoint{2.603163in}{1.273152in}}{\pgfqpoint{2.603163in}{1.262102in}}%
\pgfpathcurveto{\pgfqpoint{2.603163in}{1.251052in}}{\pgfqpoint{2.607553in}{1.240453in}}{\pgfqpoint{2.615366in}{1.232639in}}%
\pgfpathcurveto{\pgfqpoint{2.623180in}{1.224825in}}{\pgfqpoint{2.633779in}{1.220435in}}{\pgfqpoint{2.644829in}{1.220435in}}%
\pgfpathclose%
\pgfusepath{stroke,fill}%
\end{pgfscope}%
\begin{pgfscope}%
\pgfpathrectangle{\pgfqpoint{0.511823in}{0.504323in}}{\pgfqpoint{3.218177in}{3.225677in}} %
\pgfusepath{clip}%
\pgfsetbuttcap%
\pgfsetroundjoin%
\definecolor{currentfill}{rgb}{0.000000,0.000000,0.545098}%
\pgfsetfillcolor{currentfill}%
\pgfsetfillopacity{0.400000}%
\pgfsetlinewidth{0.501875pt}%
\definecolor{currentstroke}{rgb}{0.000000,0.000000,0.545098}%
\pgfsetstrokecolor{currentstroke}%
\pgfsetstrokeopacity{0.400000}%
\pgfsetdash{}{0pt}%
\pgfpathmoveto{\pgfqpoint{2.643730in}{1.226431in}}%
\pgfpathcurveto{\pgfqpoint{2.654780in}{1.226431in}}{\pgfqpoint{2.665379in}{1.230821in}}{\pgfqpoint{2.673193in}{1.238634in}}%
\pgfpathcurveto{\pgfqpoint{2.681006in}{1.246448in}}{\pgfqpoint{2.685397in}{1.257047in}}{\pgfqpoint{2.685397in}{1.268097in}}%
\pgfpathcurveto{\pgfqpoint{2.685397in}{1.279147in}}{\pgfqpoint{2.681006in}{1.289746in}}{\pgfqpoint{2.673193in}{1.297560in}}%
\pgfpathcurveto{\pgfqpoint{2.665379in}{1.305374in}}{\pgfqpoint{2.654780in}{1.309764in}}{\pgfqpoint{2.643730in}{1.309764in}}%
\pgfpathcurveto{\pgfqpoint{2.632680in}{1.309764in}}{\pgfqpoint{2.622081in}{1.305374in}}{\pgfqpoint{2.614267in}{1.297560in}}%
\pgfpathcurveto{\pgfqpoint{2.606454in}{1.289746in}}{\pgfqpoint{2.602063in}{1.279147in}}{\pgfqpoint{2.602063in}{1.268097in}}%
\pgfpathcurveto{\pgfqpoint{2.602063in}{1.257047in}}{\pgfqpoint{2.606454in}{1.246448in}}{\pgfqpoint{2.614267in}{1.238634in}}%
\pgfpathcurveto{\pgfqpoint{2.622081in}{1.230821in}}{\pgfqpoint{2.632680in}{1.226431in}}{\pgfqpoint{2.643730in}{1.226431in}}%
\pgfpathclose%
\pgfusepath{stroke,fill}%
\end{pgfscope}%
\begin{pgfscope}%
\pgfpathrectangle{\pgfqpoint{0.511823in}{0.504323in}}{\pgfqpoint{3.218177in}{3.225677in}} %
\pgfusepath{clip}%
\pgfsetbuttcap%
\pgfsetroundjoin%
\definecolor{currentfill}{rgb}{0.000000,0.000000,0.545098}%
\pgfsetfillcolor{currentfill}%
\pgfsetfillopacity{0.400000}%
\pgfsetlinewidth{0.501875pt}%
\definecolor{currentstroke}{rgb}{0.000000,0.000000,0.545098}%
\pgfsetstrokecolor{currentstroke}%
\pgfsetstrokeopacity{0.400000}%
\pgfsetdash{}{0pt}%
\pgfpathmoveto{\pgfqpoint{2.678561in}{1.243318in}}%
\pgfpathcurveto{\pgfqpoint{2.689612in}{1.243318in}}{\pgfqpoint{2.700211in}{1.247708in}}{\pgfqpoint{2.708024in}{1.255522in}}%
\pgfpathcurveto{\pgfqpoint{2.715838in}{1.263335in}}{\pgfqpoint{2.720228in}{1.273934in}}{\pgfqpoint{2.720228in}{1.284984in}}%
\pgfpathcurveto{\pgfqpoint{2.720228in}{1.296034in}}{\pgfqpoint{2.715838in}{1.306634in}}{\pgfqpoint{2.708024in}{1.314447in}}%
\pgfpathcurveto{\pgfqpoint{2.700211in}{1.322261in}}{\pgfqpoint{2.689612in}{1.326651in}}{\pgfqpoint{2.678561in}{1.326651in}}%
\pgfpathcurveto{\pgfqpoint{2.667511in}{1.326651in}}{\pgfqpoint{2.656912in}{1.322261in}}{\pgfqpoint{2.649099in}{1.314447in}}%
\pgfpathcurveto{\pgfqpoint{2.641285in}{1.306634in}}{\pgfqpoint{2.636895in}{1.296034in}}{\pgfqpoint{2.636895in}{1.284984in}}%
\pgfpathcurveto{\pgfqpoint{2.636895in}{1.273934in}}{\pgfqpoint{2.641285in}{1.263335in}}{\pgfqpoint{2.649099in}{1.255522in}}%
\pgfpathcurveto{\pgfqpoint{2.656912in}{1.247708in}}{\pgfqpoint{2.667511in}{1.243318in}}{\pgfqpoint{2.678561in}{1.243318in}}%
\pgfpathclose%
\pgfusepath{stroke,fill}%
\end{pgfscope}%
\begin{pgfscope}%
\pgfpathrectangle{\pgfqpoint{0.511823in}{0.504323in}}{\pgfqpoint{3.218177in}{3.225677in}} %
\pgfusepath{clip}%
\pgfsetbuttcap%
\pgfsetroundjoin%
\definecolor{currentfill}{rgb}{0.000000,0.000000,0.545098}%
\pgfsetfillcolor{currentfill}%
\pgfsetfillopacity{0.400000}%
\pgfsetlinewidth{0.501875pt}%
\definecolor{currentstroke}{rgb}{0.000000,0.000000,0.545098}%
\pgfsetstrokecolor{currentstroke}%
\pgfsetstrokeopacity{0.400000}%
\pgfsetdash{}{0pt}%
\pgfpathmoveto{\pgfqpoint{2.598172in}{1.225161in}}%
\pgfpathcurveto{\pgfqpoint{2.609222in}{1.225161in}}{\pgfqpoint{2.619821in}{1.229552in}}{\pgfqpoint{2.627635in}{1.237365in}}%
\pgfpathcurveto{\pgfqpoint{2.635448in}{1.245179in}}{\pgfqpoint{2.639839in}{1.255778in}}{\pgfqpoint{2.639839in}{1.266828in}}%
\pgfpathcurveto{\pgfqpoint{2.639839in}{1.277878in}}{\pgfqpoint{2.635448in}{1.288477in}}{\pgfqpoint{2.627635in}{1.296291in}}%
\pgfpathcurveto{\pgfqpoint{2.619821in}{1.304104in}}{\pgfqpoint{2.609222in}{1.308495in}}{\pgfqpoint{2.598172in}{1.308495in}}%
\pgfpathcurveto{\pgfqpoint{2.587122in}{1.308495in}}{\pgfqpoint{2.576523in}{1.304104in}}{\pgfqpoint{2.568709in}{1.296291in}}%
\pgfpathcurveto{\pgfqpoint{2.560895in}{1.288477in}}{\pgfqpoint{2.556505in}{1.277878in}}{\pgfqpoint{2.556505in}{1.266828in}}%
\pgfpathcurveto{\pgfqpoint{2.556505in}{1.255778in}}{\pgfqpoint{2.560895in}{1.245179in}}{\pgfqpoint{2.568709in}{1.237365in}}%
\pgfpathcurveto{\pgfqpoint{2.576523in}{1.229552in}}{\pgfqpoint{2.587122in}{1.225161in}}{\pgfqpoint{2.598172in}{1.225161in}}%
\pgfpathclose%
\pgfusepath{stroke,fill}%
\end{pgfscope}%
\begin{pgfscope}%
\pgfpathrectangle{\pgfqpoint{0.511823in}{0.504323in}}{\pgfqpoint{3.218177in}{3.225677in}} %
\pgfusepath{clip}%
\pgfsetbuttcap%
\pgfsetroundjoin%
\definecolor{currentfill}{rgb}{0.000000,0.000000,0.545098}%
\pgfsetfillcolor{currentfill}%
\pgfsetfillopacity{0.400000}%
\pgfsetlinewidth{0.501875pt}%
\definecolor{currentstroke}{rgb}{0.000000,0.000000,0.545098}%
\pgfsetstrokecolor{currentstroke}%
\pgfsetstrokeopacity{0.400000}%
\pgfsetdash{}{0pt}%
\pgfpathmoveto{\pgfqpoint{2.353919in}{1.155786in}}%
\pgfpathcurveto{\pgfqpoint{2.364969in}{1.155786in}}{\pgfqpoint{2.375568in}{1.160177in}}{\pgfqpoint{2.383382in}{1.167990in}}%
\pgfpathcurveto{\pgfqpoint{2.391196in}{1.175804in}}{\pgfqpoint{2.395586in}{1.186403in}}{\pgfqpoint{2.395586in}{1.197453in}}%
\pgfpathcurveto{\pgfqpoint{2.395586in}{1.208503in}}{\pgfqpoint{2.391196in}{1.219102in}}{\pgfqpoint{2.383382in}{1.226916in}}%
\pgfpathcurveto{\pgfqpoint{2.375568in}{1.234729in}}{\pgfqpoint{2.364969in}{1.239120in}}{\pgfqpoint{2.353919in}{1.239120in}}%
\pgfpathcurveto{\pgfqpoint{2.342869in}{1.239120in}}{\pgfqpoint{2.332270in}{1.234729in}}{\pgfqpoint{2.324456in}{1.226916in}}%
\pgfpathcurveto{\pgfqpoint{2.316643in}{1.219102in}}{\pgfqpoint{2.312253in}{1.208503in}}{\pgfqpoint{2.312253in}{1.197453in}}%
\pgfpathcurveto{\pgfqpoint{2.312253in}{1.186403in}}{\pgfqpoint{2.316643in}{1.175804in}}{\pgfqpoint{2.324456in}{1.167990in}}%
\pgfpathcurveto{\pgfqpoint{2.332270in}{1.160177in}}{\pgfqpoint{2.342869in}{1.155786in}}{\pgfqpoint{2.353919in}{1.155786in}}%
\pgfpathclose%
\pgfusepath{stroke,fill}%
\end{pgfscope}%
\begin{pgfscope}%
\pgfpathrectangle{\pgfqpoint{0.511823in}{0.504323in}}{\pgfqpoint{3.218177in}{3.225677in}} %
\pgfusepath{clip}%
\pgfsetbuttcap%
\pgfsetroundjoin%
\definecolor{currentfill}{rgb}{0.000000,0.000000,0.545098}%
\pgfsetfillcolor{currentfill}%
\pgfsetfillopacity{0.400000}%
\pgfsetlinewidth{0.501875pt}%
\definecolor{currentstroke}{rgb}{0.000000,0.000000,0.545098}%
\pgfsetstrokecolor{currentstroke}%
\pgfsetstrokeopacity{0.400000}%
\pgfsetdash{}{0pt}%
\pgfpathmoveto{\pgfqpoint{2.651959in}{1.254420in}}%
\pgfpathcurveto{\pgfqpoint{2.663010in}{1.254420in}}{\pgfqpoint{2.673609in}{1.258810in}}{\pgfqpoint{2.681422in}{1.266623in}}%
\pgfpathcurveto{\pgfqpoint{2.689236in}{1.274437in}}{\pgfqpoint{2.693626in}{1.285036in}}{\pgfqpoint{2.693626in}{1.296086in}}%
\pgfpathcurveto{\pgfqpoint{2.693626in}{1.307136in}}{\pgfqpoint{2.689236in}{1.317735in}}{\pgfqpoint{2.681422in}{1.325549in}}%
\pgfpathcurveto{\pgfqpoint{2.673609in}{1.333363in}}{\pgfqpoint{2.663010in}{1.337753in}}{\pgfqpoint{2.651959in}{1.337753in}}%
\pgfpathcurveto{\pgfqpoint{2.640909in}{1.337753in}}{\pgfqpoint{2.630310in}{1.333363in}}{\pgfqpoint{2.622497in}{1.325549in}}%
\pgfpathcurveto{\pgfqpoint{2.614683in}{1.317735in}}{\pgfqpoint{2.610293in}{1.307136in}}{\pgfqpoint{2.610293in}{1.296086in}}%
\pgfpathcurveto{\pgfqpoint{2.610293in}{1.285036in}}{\pgfqpoint{2.614683in}{1.274437in}}{\pgfqpoint{2.622497in}{1.266623in}}%
\pgfpathcurveto{\pgfqpoint{2.630310in}{1.258810in}}{\pgfqpoint{2.640909in}{1.254420in}}{\pgfqpoint{2.651959in}{1.254420in}}%
\pgfpathclose%
\pgfusepath{stroke,fill}%
\end{pgfscope}%
\begin{pgfscope}%
\pgfpathrectangle{\pgfqpoint{0.511823in}{0.504323in}}{\pgfqpoint{3.218177in}{3.225677in}} %
\pgfusepath{clip}%
\pgfsetbuttcap%
\pgfsetroundjoin%
\definecolor{currentfill}{rgb}{0.000000,0.000000,0.545098}%
\pgfsetfillcolor{currentfill}%
\pgfsetfillopacity{0.400000}%
\pgfsetlinewidth{0.501875pt}%
\definecolor{currentstroke}{rgb}{0.000000,0.000000,0.545098}%
\pgfsetstrokecolor{currentstroke}%
\pgfsetstrokeopacity{0.400000}%
\pgfsetdash{}{0pt}%
\pgfpathmoveto{\pgfqpoint{2.633293in}{1.254935in}}%
\pgfpathcurveto{\pgfqpoint{2.644343in}{1.254935in}}{\pgfqpoint{2.654942in}{1.259326in}}{\pgfqpoint{2.662756in}{1.267139in}}%
\pgfpathcurveto{\pgfqpoint{2.670569in}{1.274953in}}{\pgfqpoint{2.674960in}{1.285552in}}{\pgfqpoint{2.674960in}{1.296602in}}%
\pgfpathcurveto{\pgfqpoint{2.674960in}{1.307652in}}{\pgfqpoint{2.670569in}{1.318251in}}{\pgfqpoint{2.662756in}{1.326065in}}%
\pgfpathcurveto{\pgfqpoint{2.654942in}{1.333878in}}{\pgfqpoint{2.644343in}{1.338269in}}{\pgfqpoint{2.633293in}{1.338269in}}%
\pgfpathcurveto{\pgfqpoint{2.622243in}{1.338269in}}{\pgfqpoint{2.611644in}{1.333878in}}{\pgfqpoint{2.603830in}{1.326065in}}%
\pgfpathcurveto{\pgfqpoint{2.596017in}{1.318251in}}{\pgfqpoint{2.591626in}{1.307652in}}{\pgfqpoint{2.591626in}{1.296602in}}%
\pgfpathcurveto{\pgfqpoint{2.591626in}{1.285552in}}{\pgfqpoint{2.596017in}{1.274953in}}{\pgfqpoint{2.603830in}{1.267139in}}%
\pgfpathcurveto{\pgfqpoint{2.611644in}{1.259326in}}{\pgfqpoint{2.622243in}{1.254935in}}{\pgfqpoint{2.633293in}{1.254935in}}%
\pgfpathclose%
\pgfusepath{stroke,fill}%
\end{pgfscope}%
\begin{pgfscope}%
\pgfpathrectangle{\pgfqpoint{0.511823in}{0.504323in}}{\pgfqpoint{3.218177in}{3.225677in}} %
\pgfusepath{clip}%
\pgfsetbuttcap%
\pgfsetroundjoin%
\definecolor{currentfill}{rgb}{0.000000,0.000000,0.545098}%
\pgfsetfillcolor{currentfill}%
\pgfsetfillopacity{0.400000}%
\pgfsetlinewidth{0.501875pt}%
\definecolor{currentstroke}{rgb}{0.000000,0.000000,0.545098}%
\pgfsetstrokecolor{currentstroke}%
\pgfsetstrokeopacity{0.400000}%
\pgfsetdash{}{0pt}%
\pgfpathmoveto{\pgfqpoint{2.696429in}{1.281458in}}%
\pgfpathcurveto{\pgfqpoint{2.707479in}{1.281458in}}{\pgfqpoint{2.718078in}{1.285849in}}{\pgfqpoint{2.725891in}{1.293662in}}%
\pgfpathcurveto{\pgfqpoint{2.733705in}{1.301476in}}{\pgfqpoint{2.738095in}{1.312075in}}{\pgfqpoint{2.738095in}{1.323125in}}%
\pgfpathcurveto{\pgfqpoint{2.738095in}{1.334175in}}{\pgfqpoint{2.733705in}{1.344774in}}{\pgfqpoint{2.725891in}{1.352588in}}%
\pgfpathcurveto{\pgfqpoint{2.718078in}{1.360401in}}{\pgfqpoint{2.707479in}{1.364792in}}{\pgfqpoint{2.696429in}{1.364792in}}%
\pgfpathcurveto{\pgfqpoint{2.685379in}{1.364792in}}{\pgfqpoint{2.674780in}{1.360401in}}{\pgfqpoint{2.666966in}{1.352588in}}%
\pgfpathcurveto{\pgfqpoint{2.659152in}{1.344774in}}{\pgfqpoint{2.654762in}{1.334175in}}{\pgfqpoint{2.654762in}{1.323125in}}%
\pgfpathcurveto{\pgfqpoint{2.654762in}{1.312075in}}{\pgfqpoint{2.659152in}{1.301476in}}{\pgfqpoint{2.666966in}{1.293662in}}%
\pgfpathcurveto{\pgfqpoint{2.674780in}{1.285849in}}{\pgfqpoint{2.685379in}{1.281458in}}{\pgfqpoint{2.696429in}{1.281458in}}%
\pgfpathclose%
\pgfusepath{stroke,fill}%
\end{pgfscope}%
\begin{pgfscope}%
\pgfpathrectangle{\pgfqpoint{0.511823in}{0.504323in}}{\pgfqpoint{3.218177in}{3.225677in}} %
\pgfusepath{clip}%
\pgfsetbuttcap%
\pgfsetroundjoin%
\definecolor{currentfill}{rgb}{0.000000,0.000000,0.545098}%
\pgfsetfillcolor{currentfill}%
\pgfsetfillopacity{0.400000}%
\pgfsetlinewidth{0.501875pt}%
\definecolor{currentstroke}{rgb}{0.000000,0.000000,0.545098}%
\pgfsetstrokecolor{currentstroke}%
\pgfsetstrokeopacity{0.400000}%
\pgfsetdash{}{0pt}%
\pgfpathmoveto{\pgfqpoint{2.460527in}{1.211960in}}%
\pgfpathcurveto{\pgfqpoint{2.471577in}{1.211960in}}{\pgfqpoint{2.482176in}{1.216350in}}{\pgfqpoint{2.489990in}{1.224164in}}%
\pgfpathcurveto{\pgfqpoint{2.497803in}{1.231978in}}{\pgfqpoint{2.502194in}{1.242577in}}{\pgfqpoint{2.502194in}{1.253627in}}%
\pgfpathcurveto{\pgfqpoint{2.502194in}{1.264677in}}{\pgfqpoint{2.497803in}{1.275276in}}{\pgfqpoint{2.489990in}{1.283090in}}%
\pgfpathcurveto{\pgfqpoint{2.482176in}{1.290903in}}{\pgfqpoint{2.471577in}{1.295293in}}{\pgfqpoint{2.460527in}{1.295293in}}%
\pgfpathcurveto{\pgfqpoint{2.449477in}{1.295293in}}{\pgfqpoint{2.438878in}{1.290903in}}{\pgfqpoint{2.431064in}{1.283090in}}%
\pgfpathcurveto{\pgfqpoint{2.423251in}{1.275276in}}{\pgfqpoint{2.418860in}{1.264677in}}{\pgfqpoint{2.418860in}{1.253627in}}%
\pgfpathcurveto{\pgfqpoint{2.418860in}{1.242577in}}{\pgfqpoint{2.423251in}{1.231978in}}{\pgfqpoint{2.431064in}{1.224164in}}%
\pgfpathcurveto{\pgfqpoint{2.438878in}{1.216350in}}{\pgfqpoint{2.449477in}{1.211960in}}{\pgfqpoint{2.460527in}{1.211960in}}%
\pgfpathclose%
\pgfusepath{stroke,fill}%
\end{pgfscope}%
\begin{pgfscope}%
\pgfpathrectangle{\pgfqpoint{0.511823in}{0.504323in}}{\pgfqpoint{3.218177in}{3.225677in}} %
\pgfusepath{clip}%
\pgfsetbuttcap%
\pgfsetroundjoin%
\definecolor{currentfill}{rgb}{0.000000,0.000000,0.545098}%
\pgfsetfillcolor{currentfill}%
\pgfsetfillopacity{0.400000}%
\pgfsetlinewidth{0.501875pt}%
\definecolor{currentstroke}{rgb}{0.000000,0.000000,0.545098}%
\pgfsetstrokecolor{currentstroke}%
\pgfsetstrokeopacity{0.400000}%
\pgfsetdash{}{0pt}%
\pgfpathmoveto{\pgfqpoint{2.880194in}{1.354521in}}%
\pgfpathcurveto{\pgfqpoint{2.891244in}{1.354521in}}{\pgfqpoint{2.901843in}{1.358912in}}{\pgfqpoint{2.909657in}{1.366725in}}%
\pgfpathcurveto{\pgfqpoint{2.917470in}{1.374539in}}{\pgfqpoint{2.921861in}{1.385138in}}{\pgfqpoint{2.921861in}{1.396188in}}%
\pgfpathcurveto{\pgfqpoint{2.921861in}{1.407238in}}{\pgfqpoint{2.917470in}{1.417837in}}{\pgfqpoint{2.909657in}{1.425651in}}%
\pgfpathcurveto{\pgfqpoint{2.901843in}{1.433464in}}{\pgfqpoint{2.891244in}{1.437855in}}{\pgfqpoint{2.880194in}{1.437855in}}%
\pgfpathcurveto{\pgfqpoint{2.869144in}{1.437855in}}{\pgfqpoint{2.858545in}{1.433464in}}{\pgfqpoint{2.850731in}{1.425651in}}%
\pgfpathcurveto{\pgfqpoint{2.842918in}{1.417837in}}{\pgfqpoint{2.838527in}{1.407238in}}{\pgfqpoint{2.838527in}{1.396188in}}%
\pgfpathcurveto{\pgfqpoint{2.838527in}{1.385138in}}{\pgfqpoint{2.842918in}{1.374539in}}{\pgfqpoint{2.850731in}{1.366725in}}%
\pgfpathcurveto{\pgfqpoint{2.858545in}{1.358912in}}{\pgfqpoint{2.869144in}{1.354521in}}{\pgfqpoint{2.880194in}{1.354521in}}%
\pgfpathclose%
\pgfusepath{stroke,fill}%
\end{pgfscope}%
\begin{pgfscope}%
\pgfpathrectangle{\pgfqpoint{0.511823in}{0.504323in}}{\pgfqpoint{3.218177in}{3.225677in}} %
\pgfusepath{clip}%
\pgfsetbuttcap%
\pgfsetroundjoin%
\definecolor{currentfill}{rgb}{0.000000,0.000000,0.545098}%
\pgfsetfillcolor{currentfill}%
\pgfsetfillopacity{0.400000}%
\pgfsetlinewidth{0.501875pt}%
\definecolor{currentstroke}{rgb}{0.000000,0.000000,0.545098}%
\pgfsetstrokecolor{currentstroke}%
\pgfsetstrokeopacity{0.400000}%
\pgfsetdash{}{0pt}%
\pgfpathmoveto{\pgfqpoint{2.408353in}{1.206456in}}%
\pgfpathcurveto{\pgfqpoint{2.419404in}{1.206456in}}{\pgfqpoint{2.430003in}{1.210847in}}{\pgfqpoint{2.437816in}{1.218660in}}%
\pgfpathcurveto{\pgfqpoint{2.445630in}{1.226474in}}{\pgfqpoint{2.450020in}{1.237073in}}{\pgfqpoint{2.450020in}{1.248123in}}%
\pgfpathcurveto{\pgfqpoint{2.450020in}{1.259173in}}{\pgfqpoint{2.445630in}{1.269772in}}{\pgfqpoint{2.437816in}{1.277586in}}%
\pgfpathcurveto{\pgfqpoint{2.430003in}{1.285399in}}{\pgfqpoint{2.419404in}{1.289790in}}{\pgfqpoint{2.408353in}{1.289790in}}%
\pgfpathcurveto{\pgfqpoint{2.397303in}{1.289790in}}{\pgfqpoint{2.386704in}{1.285399in}}{\pgfqpoint{2.378891in}{1.277586in}}%
\pgfpathcurveto{\pgfqpoint{2.371077in}{1.269772in}}{\pgfqpoint{2.366687in}{1.259173in}}{\pgfqpoint{2.366687in}{1.248123in}}%
\pgfpathcurveto{\pgfqpoint{2.366687in}{1.237073in}}{\pgfqpoint{2.371077in}{1.226474in}}{\pgfqpoint{2.378891in}{1.218660in}}%
\pgfpathcurveto{\pgfqpoint{2.386704in}{1.210847in}}{\pgfqpoint{2.397303in}{1.206456in}}{\pgfqpoint{2.408353in}{1.206456in}}%
\pgfpathclose%
\pgfusepath{stroke,fill}%
\end{pgfscope}%
\begin{pgfscope}%
\pgfpathrectangle{\pgfqpoint{0.511823in}{0.504323in}}{\pgfqpoint{3.218177in}{3.225677in}} %
\pgfusepath{clip}%
\pgfsetbuttcap%
\pgfsetroundjoin%
\definecolor{currentfill}{rgb}{0.000000,0.000000,0.545098}%
\pgfsetfillcolor{currentfill}%
\pgfsetfillopacity{0.400000}%
\pgfsetlinewidth{0.501875pt}%
\definecolor{currentstroke}{rgb}{0.000000,0.000000,0.545098}%
\pgfsetstrokecolor{currentstroke}%
\pgfsetstrokeopacity{0.400000}%
\pgfsetdash{}{0pt}%
\pgfpathmoveto{\pgfqpoint{2.651208in}{1.292864in}}%
\pgfpathcurveto{\pgfqpoint{2.662258in}{1.292864in}}{\pgfqpoint{2.672857in}{1.297254in}}{\pgfqpoint{2.680671in}{1.305067in}}%
\pgfpathcurveto{\pgfqpoint{2.688484in}{1.312881in}}{\pgfqpoint{2.692875in}{1.323480in}}{\pgfqpoint{2.692875in}{1.334530in}}%
\pgfpathcurveto{\pgfqpoint{2.692875in}{1.345580in}}{\pgfqpoint{2.688484in}{1.356179in}}{\pgfqpoint{2.680671in}{1.363993in}}%
\pgfpathcurveto{\pgfqpoint{2.672857in}{1.371807in}}{\pgfqpoint{2.662258in}{1.376197in}}{\pgfqpoint{2.651208in}{1.376197in}}%
\pgfpathcurveto{\pgfqpoint{2.640158in}{1.376197in}}{\pgfqpoint{2.629559in}{1.371807in}}{\pgfqpoint{2.621745in}{1.363993in}}%
\pgfpathcurveto{\pgfqpoint{2.613932in}{1.356179in}}{\pgfqpoint{2.609541in}{1.345580in}}{\pgfqpoint{2.609541in}{1.334530in}}%
\pgfpathcurveto{\pgfqpoint{2.609541in}{1.323480in}}{\pgfqpoint{2.613932in}{1.312881in}}{\pgfqpoint{2.621745in}{1.305067in}}%
\pgfpathcurveto{\pgfqpoint{2.629559in}{1.297254in}}{\pgfqpoint{2.640158in}{1.292864in}}{\pgfqpoint{2.651208in}{1.292864in}}%
\pgfpathclose%
\pgfusepath{stroke,fill}%
\end{pgfscope}%
\begin{pgfscope}%
\pgfpathrectangle{\pgfqpoint{0.511823in}{0.504323in}}{\pgfqpoint{3.218177in}{3.225677in}} %
\pgfusepath{clip}%
\pgfsetbuttcap%
\pgfsetroundjoin%
\definecolor{currentfill}{rgb}{0.000000,0.000000,0.545098}%
\pgfsetfillcolor{currentfill}%
\pgfsetfillopacity{0.400000}%
\pgfsetlinewidth{0.501875pt}%
\definecolor{currentstroke}{rgb}{0.000000,0.000000,0.545098}%
\pgfsetstrokecolor{currentstroke}%
\pgfsetstrokeopacity{0.400000}%
\pgfsetdash{}{0pt}%
\pgfpathmoveto{\pgfqpoint{2.564008in}{1.270086in}}%
\pgfpathcurveto{\pgfqpoint{2.575058in}{1.270086in}}{\pgfqpoint{2.585657in}{1.274476in}}{\pgfqpoint{2.593471in}{1.282290in}}%
\pgfpathcurveto{\pgfqpoint{2.601284in}{1.290103in}}{\pgfqpoint{2.605674in}{1.300702in}}{\pgfqpoint{2.605674in}{1.311752in}}%
\pgfpathcurveto{\pgfqpoint{2.605674in}{1.322803in}}{\pgfqpoint{2.601284in}{1.333402in}}{\pgfqpoint{2.593471in}{1.341215in}}%
\pgfpathcurveto{\pgfqpoint{2.585657in}{1.349029in}}{\pgfqpoint{2.575058in}{1.353419in}}{\pgfqpoint{2.564008in}{1.353419in}}%
\pgfpathcurveto{\pgfqpoint{2.552958in}{1.353419in}}{\pgfqpoint{2.542359in}{1.349029in}}{\pgfqpoint{2.534545in}{1.341215in}}%
\pgfpathcurveto{\pgfqpoint{2.526731in}{1.333402in}}{\pgfqpoint{2.522341in}{1.322803in}}{\pgfqpoint{2.522341in}{1.311752in}}%
\pgfpathcurveto{\pgfqpoint{2.522341in}{1.300702in}}{\pgfqpoint{2.526731in}{1.290103in}}{\pgfqpoint{2.534545in}{1.282290in}}%
\pgfpathcurveto{\pgfqpoint{2.542359in}{1.274476in}}{\pgfqpoint{2.552958in}{1.270086in}}{\pgfqpoint{2.564008in}{1.270086in}}%
\pgfpathclose%
\pgfusepath{stroke,fill}%
\end{pgfscope}%
\begin{pgfscope}%
\pgfpathrectangle{\pgfqpoint{0.511823in}{0.504323in}}{\pgfqpoint{3.218177in}{3.225677in}} %
\pgfusepath{clip}%
\pgfsetbuttcap%
\pgfsetroundjoin%
\definecolor{currentfill}{rgb}{0.000000,0.000000,0.545098}%
\pgfsetfillcolor{currentfill}%
\pgfsetfillopacity{0.400000}%
\pgfsetlinewidth{0.501875pt}%
\definecolor{currentstroke}{rgb}{0.000000,0.000000,0.545098}%
\pgfsetstrokecolor{currentstroke}%
\pgfsetstrokeopacity{0.400000}%
\pgfsetdash{}{0pt}%
\pgfpathmoveto{\pgfqpoint{2.495418in}{1.253055in}}%
\pgfpathcurveto{\pgfqpoint{2.506469in}{1.253055in}}{\pgfqpoint{2.517068in}{1.257445in}}{\pgfqpoint{2.524881in}{1.265258in}}%
\pgfpathcurveto{\pgfqpoint{2.532695in}{1.273072in}}{\pgfqpoint{2.537085in}{1.283671in}}{\pgfqpoint{2.537085in}{1.294721in}}%
\pgfpathcurveto{\pgfqpoint{2.537085in}{1.305771in}}{\pgfqpoint{2.532695in}{1.316370in}}{\pgfqpoint{2.524881in}{1.324184in}}%
\pgfpathcurveto{\pgfqpoint{2.517068in}{1.331998in}}{\pgfqpoint{2.506469in}{1.336388in}}{\pgfqpoint{2.495418in}{1.336388in}}%
\pgfpathcurveto{\pgfqpoint{2.484368in}{1.336388in}}{\pgfqpoint{2.473769in}{1.331998in}}{\pgfqpoint{2.465956in}{1.324184in}}%
\pgfpathcurveto{\pgfqpoint{2.458142in}{1.316370in}}{\pgfqpoint{2.453752in}{1.305771in}}{\pgfqpoint{2.453752in}{1.294721in}}%
\pgfpathcurveto{\pgfqpoint{2.453752in}{1.283671in}}{\pgfqpoint{2.458142in}{1.273072in}}{\pgfqpoint{2.465956in}{1.265258in}}%
\pgfpathcurveto{\pgfqpoint{2.473769in}{1.257445in}}{\pgfqpoint{2.484368in}{1.253055in}}{\pgfqpoint{2.495418in}{1.253055in}}%
\pgfpathclose%
\pgfusepath{stroke,fill}%
\end{pgfscope}%
\begin{pgfscope}%
\pgfpathrectangle{\pgfqpoint{0.511823in}{0.504323in}}{\pgfqpoint{3.218177in}{3.225677in}} %
\pgfusepath{clip}%
\pgfsetbuttcap%
\pgfsetroundjoin%
\definecolor{currentfill}{rgb}{0.000000,0.000000,0.545098}%
\pgfsetfillcolor{currentfill}%
\pgfsetfillopacity{0.400000}%
\pgfsetlinewidth{0.501875pt}%
\definecolor{currentstroke}{rgb}{0.000000,0.000000,0.545098}%
\pgfsetstrokecolor{currentstroke}%
\pgfsetstrokeopacity{0.400000}%
\pgfsetdash{}{0pt}%
\pgfpathmoveto{\pgfqpoint{2.655595in}{1.313900in}}%
\pgfpathcurveto{\pgfqpoint{2.666645in}{1.313900in}}{\pgfqpoint{2.677244in}{1.318290in}}{\pgfqpoint{2.685058in}{1.326103in}}%
\pgfpathcurveto{\pgfqpoint{2.692871in}{1.333917in}}{\pgfqpoint{2.697262in}{1.344516in}}{\pgfqpoint{2.697262in}{1.355566in}}%
\pgfpathcurveto{\pgfqpoint{2.697262in}{1.366616in}}{\pgfqpoint{2.692871in}{1.377215in}}{\pgfqpoint{2.685058in}{1.385029in}}%
\pgfpathcurveto{\pgfqpoint{2.677244in}{1.392843in}}{\pgfqpoint{2.666645in}{1.397233in}}{\pgfqpoint{2.655595in}{1.397233in}}%
\pgfpathcurveto{\pgfqpoint{2.644545in}{1.397233in}}{\pgfqpoint{2.633946in}{1.392843in}}{\pgfqpoint{2.626132in}{1.385029in}}%
\pgfpathcurveto{\pgfqpoint{2.618319in}{1.377215in}}{\pgfqpoint{2.613928in}{1.366616in}}{\pgfqpoint{2.613928in}{1.355566in}}%
\pgfpathcurveto{\pgfqpoint{2.613928in}{1.344516in}}{\pgfqpoint{2.618319in}{1.333917in}}{\pgfqpoint{2.626132in}{1.326103in}}%
\pgfpathcurveto{\pgfqpoint{2.633946in}{1.318290in}}{\pgfqpoint{2.644545in}{1.313900in}}{\pgfqpoint{2.655595in}{1.313900in}}%
\pgfpathclose%
\pgfusepath{stroke,fill}%
\end{pgfscope}%
\begin{pgfscope}%
\pgfpathrectangle{\pgfqpoint{0.511823in}{0.504323in}}{\pgfqpoint{3.218177in}{3.225677in}} %
\pgfusepath{clip}%
\pgfsetbuttcap%
\pgfsetroundjoin%
\definecolor{currentfill}{rgb}{0.000000,0.000000,0.545098}%
\pgfsetfillcolor{currentfill}%
\pgfsetfillopacity{0.400000}%
\pgfsetlinewidth{0.501875pt}%
\definecolor{currentstroke}{rgb}{0.000000,0.000000,0.545098}%
\pgfsetstrokecolor{currentstroke}%
\pgfsetstrokeopacity{0.400000}%
\pgfsetdash{}{0pt}%
\pgfpathmoveto{\pgfqpoint{2.516814in}{1.272482in}}%
\pgfpathcurveto{\pgfqpoint{2.527864in}{1.272482in}}{\pgfqpoint{2.538463in}{1.276872in}}{\pgfqpoint{2.546276in}{1.284686in}}%
\pgfpathcurveto{\pgfqpoint{2.554090in}{1.292499in}}{\pgfqpoint{2.558480in}{1.303098in}}{\pgfqpoint{2.558480in}{1.314149in}}%
\pgfpathcurveto{\pgfqpoint{2.558480in}{1.325199in}}{\pgfqpoint{2.554090in}{1.335798in}}{\pgfqpoint{2.546276in}{1.343611in}}%
\pgfpathcurveto{\pgfqpoint{2.538463in}{1.351425in}}{\pgfqpoint{2.527864in}{1.355815in}}{\pgfqpoint{2.516814in}{1.355815in}}%
\pgfpathcurveto{\pgfqpoint{2.505763in}{1.355815in}}{\pgfqpoint{2.495164in}{1.351425in}}{\pgfqpoint{2.487351in}{1.343611in}}%
\pgfpathcurveto{\pgfqpoint{2.479537in}{1.335798in}}{\pgfqpoint{2.475147in}{1.325199in}}{\pgfqpoint{2.475147in}{1.314149in}}%
\pgfpathcurveto{\pgfqpoint{2.475147in}{1.303098in}}{\pgfqpoint{2.479537in}{1.292499in}}{\pgfqpoint{2.487351in}{1.284686in}}%
\pgfpathcurveto{\pgfqpoint{2.495164in}{1.276872in}}{\pgfqpoint{2.505763in}{1.272482in}}{\pgfqpoint{2.516814in}{1.272482in}}%
\pgfpathclose%
\pgfusepath{stroke,fill}%
\end{pgfscope}%
\begin{pgfscope}%
\pgfpathrectangle{\pgfqpoint{0.511823in}{0.504323in}}{\pgfqpoint{3.218177in}{3.225677in}} %
\pgfusepath{clip}%
\pgfsetbuttcap%
\pgfsetroundjoin%
\definecolor{currentfill}{rgb}{0.000000,0.000000,0.545098}%
\pgfsetfillcolor{currentfill}%
\pgfsetfillopacity{0.400000}%
\pgfsetlinewidth{0.501875pt}%
\definecolor{currentstroke}{rgb}{0.000000,0.000000,0.545098}%
\pgfsetstrokecolor{currentstroke}%
\pgfsetstrokeopacity{0.400000}%
\pgfsetdash{}{0pt}%
\pgfpathmoveto{\pgfqpoint{2.259508in}{1.188787in}}%
\pgfpathcurveto{\pgfqpoint{2.270559in}{1.188787in}}{\pgfqpoint{2.281158in}{1.193177in}}{\pgfqpoint{2.288971in}{1.200991in}}%
\pgfpathcurveto{\pgfqpoint{2.296785in}{1.208804in}}{\pgfqpoint{2.301175in}{1.219403in}}{\pgfqpoint{2.301175in}{1.230453in}}%
\pgfpathcurveto{\pgfqpoint{2.301175in}{1.241503in}}{\pgfqpoint{2.296785in}{1.252103in}}{\pgfqpoint{2.288971in}{1.259916in}}%
\pgfpathcurveto{\pgfqpoint{2.281158in}{1.267730in}}{\pgfqpoint{2.270559in}{1.272120in}}{\pgfqpoint{2.259508in}{1.272120in}}%
\pgfpathcurveto{\pgfqpoint{2.248458in}{1.272120in}}{\pgfqpoint{2.237859in}{1.267730in}}{\pgfqpoint{2.230046in}{1.259916in}}%
\pgfpathcurveto{\pgfqpoint{2.222232in}{1.252103in}}{\pgfqpoint{2.217842in}{1.241503in}}{\pgfqpoint{2.217842in}{1.230453in}}%
\pgfpathcurveto{\pgfqpoint{2.217842in}{1.219403in}}{\pgfqpoint{2.222232in}{1.208804in}}{\pgfqpoint{2.230046in}{1.200991in}}%
\pgfpathcurveto{\pgfqpoint{2.237859in}{1.193177in}}{\pgfqpoint{2.248458in}{1.188787in}}{\pgfqpoint{2.259508in}{1.188787in}}%
\pgfpathclose%
\pgfusepath{stroke,fill}%
\end{pgfscope}%
\begin{pgfscope}%
\pgfpathrectangle{\pgfqpoint{0.511823in}{0.504323in}}{\pgfqpoint{3.218177in}{3.225677in}} %
\pgfusepath{clip}%
\pgfsetbuttcap%
\pgfsetroundjoin%
\definecolor{currentfill}{rgb}{0.000000,0.000000,0.545098}%
\pgfsetfillcolor{currentfill}%
\pgfsetfillopacity{0.400000}%
\pgfsetlinewidth{0.501875pt}%
\definecolor{currentstroke}{rgb}{0.000000,0.000000,0.545098}%
\pgfsetstrokecolor{currentstroke}%
\pgfsetstrokeopacity{0.400000}%
\pgfsetdash{}{0pt}%
\pgfpathmoveto{\pgfqpoint{2.402594in}{1.244466in}}%
\pgfpathcurveto{\pgfqpoint{2.413644in}{1.244466in}}{\pgfqpoint{2.424243in}{1.248856in}}{\pgfqpoint{2.432056in}{1.256670in}}%
\pgfpathcurveto{\pgfqpoint{2.439870in}{1.264484in}}{\pgfqpoint{2.444260in}{1.275083in}}{\pgfqpoint{2.444260in}{1.286133in}}%
\pgfpathcurveto{\pgfqpoint{2.444260in}{1.297183in}}{\pgfqpoint{2.439870in}{1.307782in}}{\pgfqpoint{2.432056in}{1.315596in}}%
\pgfpathcurveto{\pgfqpoint{2.424243in}{1.323409in}}{\pgfqpoint{2.413644in}{1.327800in}}{\pgfqpoint{2.402594in}{1.327800in}}%
\pgfpathcurveto{\pgfqpoint{2.391543in}{1.327800in}}{\pgfqpoint{2.380944in}{1.323409in}}{\pgfqpoint{2.373131in}{1.315596in}}%
\pgfpathcurveto{\pgfqpoint{2.365317in}{1.307782in}}{\pgfqpoint{2.360927in}{1.297183in}}{\pgfqpoint{2.360927in}{1.286133in}}%
\pgfpathcurveto{\pgfqpoint{2.360927in}{1.275083in}}{\pgfqpoint{2.365317in}{1.264484in}}{\pgfqpoint{2.373131in}{1.256670in}}%
\pgfpathcurveto{\pgfqpoint{2.380944in}{1.248856in}}{\pgfqpoint{2.391543in}{1.244466in}}{\pgfqpoint{2.402594in}{1.244466in}}%
\pgfpathclose%
\pgfusepath{stroke,fill}%
\end{pgfscope}%
\begin{pgfscope}%
\pgfpathrectangle{\pgfqpoint{0.511823in}{0.504323in}}{\pgfqpoint{3.218177in}{3.225677in}} %
\pgfusepath{clip}%
\pgfsetbuttcap%
\pgfsetroundjoin%
\definecolor{currentfill}{rgb}{0.000000,0.000000,0.545098}%
\pgfsetfillcolor{currentfill}%
\pgfsetfillopacity{0.400000}%
\pgfsetlinewidth{0.501875pt}%
\definecolor{currentstroke}{rgb}{0.000000,0.000000,0.545098}%
\pgfsetstrokecolor{currentstroke}%
\pgfsetstrokeopacity{0.400000}%
\pgfsetdash{}{0pt}%
\pgfpathmoveto{\pgfqpoint{2.601895in}{1.321115in}}%
\pgfpathcurveto{\pgfqpoint{2.612946in}{1.321115in}}{\pgfqpoint{2.623545in}{1.325505in}}{\pgfqpoint{2.631358in}{1.333319in}}%
\pgfpathcurveto{\pgfqpoint{2.639172in}{1.341132in}}{\pgfqpoint{2.643562in}{1.351731in}}{\pgfqpoint{2.643562in}{1.362782in}}%
\pgfpathcurveto{\pgfqpoint{2.643562in}{1.373832in}}{\pgfqpoint{2.639172in}{1.384431in}}{\pgfqpoint{2.631358in}{1.392244in}}%
\pgfpathcurveto{\pgfqpoint{2.623545in}{1.400058in}}{\pgfqpoint{2.612946in}{1.404448in}}{\pgfqpoint{2.601895in}{1.404448in}}%
\pgfpathcurveto{\pgfqpoint{2.590845in}{1.404448in}}{\pgfqpoint{2.580246in}{1.400058in}}{\pgfqpoint{2.572433in}{1.392244in}}%
\pgfpathcurveto{\pgfqpoint{2.564619in}{1.384431in}}{\pgfqpoint{2.560229in}{1.373832in}}{\pgfqpoint{2.560229in}{1.362782in}}%
\pgfpathcurveto{\pgfqpoint{2.560229in}{1.351731in}}{\pgfqpoint{2.564619in}{1.341132in}}{\pgfqpoint{2.572433in}{1.333319in}}%
\pgfpathcurveto{\pgfqpoint{2.580246in}{1.325505in}}{\pgfqpoint{2.590845in}{1.321115in}}{\pgfqpoint{2.601895in}{1.321115in}}%
\pgfpathclose%
\pgfusepath{stroke,fill}%
\end{pgfscope}%
\begin{pgfscope}%
\pgfpathrectangle{\pgfqpoint{0.511823in}{0.504323in}}{\pgfqpoint{3.218177in}{3.225677in}} %
\pgfusepath{clip}%
\pgfsetbuttcap%
\pgfsetroundjoin%
\definecolor{currentfill}{rgb}{0.000000,0.000000,0.545098}%
\pgfsetfillcolor{currentfill}%
\pgfsetfillopacity{0.400000}%
\pgfsetlinewidth{0.501875pt}%
\definecolor{currentstroke}{rgb}{0.000000,0.000000,0.545098}%
\pgfsetstrokecolor{currentstroke}%
\pgfsetstrokeopacity{0.400000}%
\pgfsetdash{}{0pt}%
\pgfpathmoveto{\pgfqpoint{2.819923in}{1.405841in}}%
\pgfpathcurveto{\pgfqpoint{2.830973in}{1.405841in}}{\pgfqpoint{2.841573in}{1.410231in}}{\pgfqpoint{2.849386in}{1.418045in}}%
\pgfpathcurveto{\pgfqpoint{2.857200in}{1.425859in}}{\pgfqpoint{2.861590in}{1.436458in}}{\pgfqpoint{2.861590in}{1.447508in}}%
\pgfpathcurveto{\pgfqpoint{2.861590in}{1.458558in}}{\pgfqpoint{2.857200in}{1.469157in}}{\pgfqpoint{2.849386in}{1.476971in}}%
\pgfpathcurveto{\pgfqpoint{2.841573in}{1.484784in}}{\pgfqpoint{2.830973in}{1.489175in}}{\pgfqpoint{2.819923in}{1.489175in}}%
\pgfpathcurveto{\pgfqpoint{2.808873in}{1.489175in}}{\pgfqpoint{2.798274in}{1.484784in}}{\pgfqpoint{2.790461in}{1.476971in}}%
\pgfpathcurveto{\pgfqpoint{2.782647in}{1.469157in}}{\pgfqpoint{2.778257in}{1.458558in}}{\pgfqpoint{2.778257in}{1.447508in}}%
\pgfpathcurveto{\pgfqpoint{2.778257in}{1.436458in}}{\pgfqpoint{2.782647in}{1.425859in}}{\pgfqpoint{2.790461in}{1.418045in}}%
\pgfpathcurveto{\pgfqpoint{2.798274in}{1.410231in}}{\pgfqpoint{2.808873in}{1.405841in}}{\pgfqpoint{2.819923in}{1.405841in}}%
\pgfpathclose%
\pgfusepath{stroke,fill}%
\end{pgfscope}%
\begin{pgfscope}%
\pgfpathrectangle{\pgfqpoint{0.511823in}{0.504323in}}{\pgfqpoint{3.218177in}{3.225677in}} %
\pgfusepath{clip}%
\pgfsetbuttcap%
\pgfsetroundjoin%
\definecolor{currentfill}{rgb}{0.000000,0.000000,0.545098}%
\pgfsetfillcolor{currentfill}%
\pgfsetfillopacity{0.400000}%
\pgfsetlinewidth{0.501875pt}%
\definecolor{currentstroke}{rgb}{0.000000,0.000000,0.545098}%
\pgfsetstrokecolor{currentstroke}%
\pgfsetstrokeopacity{0.400000}%
\pgfsetdash{}{0pt}%
\pgfpathmoveto{\pgfqpoint{2.357567in}{1.245451in}}%
\pgfpathcurveto{\pgfqpoint{2.368617in}{1.245451in}}{\pgfqpoint{2.379216in}{1.249841in}}{\pgfqpoint{2.387030in}{1.257655in}}%
\pgfpathcurveto{\pgfqpoint{2.394843in}{1.265469in}}{\pgfqpoint{2.399233in}{1.276068in}}{\pgfqpoint{2.399233in}{1.287118in}}%
\pgfpathcurveto{\pgfqpoint{2.399233in}{1.298168in}}{\pgfqpoint{2.394843in}{1.308767in}}{\pgfqpoint{2.387030in}{1.316581in}}%
\pgfpathcurveto{\pgfqpoint{2.379216in}{1.324394in}}{\pgfqpoint{2.368617in}{1.328785in}}{\pgfqpoint{2.357567in}{1.328785in}}%
\pgfpathcurveto{\pgfqpoint{2.346517in}{1.328785in}}{\pgfqpoint{2.335918in}{1.324394in}}{\pgfqpoint{2.328104in}{1.316581in}}%
\pgfpathcurveto{\pgfqpoint{2.320290in}{1.308767in}}{\pgfqpoint{2.315900in}{1.298168in}}{\pgfqpoint{2.315900in}{1.287118in}}%
\pgfpathcurveto{\pgfqpoint{2.315900in}{1.276068in}}{\pgfqpoint{2.320290in}{1.265469in}}{\pgfqpoint{2.328104in}{1.257655in}}%
\pgfpathcurveto{\pgfqpoint{2.335918in}{1.249841in}}{\pgfqpoint{2.346517in}{1.245451in}}{\pgfqpoint{2.357567in}{1.245451in}}%
\pgfpathclose%
\pgfusepath{stroke,fill}%
\end{pgfscope}%
\begin{pgfscope}%
\pgfpathrectangle{\pgfqpoint{0.511823in}{0.504323in}}{\pgfqpoint{3.218177in}{3.225677in}} %
\pgfusepath{clip}%
\pgfsetbuttcap%
\pgfsetroundjoin%
\definecolor{currentfill}{rgb}{0.000000,0.000000,0.545098}%
\pgfsetfillcolor{currentfill}%
\pgfsetfillopacity{0.400000}%
\pgfsetlinewidth{0.501875pt}%
\definecolor{currentstroke}{rgb}{0.000000,0.000000,0.545098}%
\pgfsetstrokecolor{currentstroke}%
\pgfsetstrokeopacity{0.400000}%
\pgfsetdash{}{0pt}%
\pgfpathmoveto{\pgfqpoint{2.536299in}{1.316479in}}%
\pgfpathcurveto{\pgfqpoint{2.547349in}{1.316479in}}{\pgfqpoint{2.557948in}{1.320870in}}{\pgfqpoint{2.565761in}{1.328683in}}%
\pgfpathcurveto{\pgfqpoint{2.573575in}{1.336497in}}{\pgfqpoint{2.577965in}{1.347096in}}{\pgfqpoint{2.577965in}{1.358146in}}%
\pgfpathcurveto{\pgfqpoint{2.577965in}{1.369196in}}{\pgfqpoint{2.573575in}{1.379795in}}{\pgfqpoint{2.565761in}{1.387609in}}%
\pgfpathcurveto{\pgfqpoint{2.557948in}{1.395423in}}{\pgfqpoint{2.547349in}{1.399813in}}{\pgfqpoint{2.536299in}{1.399813in}}%
\pgfpathcurveto{\pgfqpoint{2.525248in}{1.399813in}}{\pgfqpoint{2.514649in}{1.395423in}}{\pgfqpoint{2.506836in}{1.387609in}}%
\pgfpathcurveto{\pgfqpoint{2.499022in}{1.379795in}}{\pgfqpoint{2.494632in}{1.369196in}}{\pgfqpoint{2.494632in}{1.358146in}}%
\pgfpathcurveto{\pgfqpoint{2.494632in}{1.347096in}}{\pgfqpoint{2.499022in}{1.336497in}}{\pgfqpoint{2.506836in}{1.328683in}}%
\pgfpathcurveto{\pgfqpoint{2.514649in}{1.320870in}}{\pgfqpoint{2.525248in}{1.316479in}}{\pgfqpoint{2.536299in}{1.316479in}}%
\pgfpathclose%
\pgfusepath{stroke,fill}%
\end{pgfscope}%
\begin{pgfscope}%
\pgfpathrectangle{\pgfqpoint{0.511823in}{0.504323in}}{\pgfqpoint{3.218177in}{3.225677in}} %
\pgfusepath{clip}%
\pgfsetbuttcap%
\pgfsetroundjoin%
\definecolor{currentfill}{rgb}{0.000000,0.000000,0.545098}%
\pgfsetfillcolor{currentfill}%
\pgfsetfillopacity{0.400000}%
\pgfsetlinewidth{0.501875pt}%
\definecolor{currentstroke}{rgb}{0.000000,0.000000,0.545098}%
\pgfsetstrokecolor{currentstroke}%
\pgfsetstrokeopacity{0.400000}%
\pgfsetdash{}{0pt}%
\pgfpathmoveto{\pgfqpoint{2.714979in}{1.388713in}}%
\pgfpathcurveto{\pgfqpoint{2.726030in}{1.388713in}}{\pgfqpoint{2.736629in}{1.393104in}}{\pgfqpoint{2.744442in}{1.400917in}}%
\pgfpathcurveto{\pgfqpoint{2.752256in}{1.408731in}}{\pgfqpoint{2.756646in}{1.419330in}}{\pgfqpoint{2.756646in}{1.430380in}}%
\pgfpathcurveto{\pgfqpoint{2.756646in}{1.441430in}}{\pgfqpoint{2.752256in}{1.452029in}}{\pgfqpoint{2.744442in}{1.459843in}}%
\pgfpathcurveto{\pgfqpoint{2.736629in}{1.467656in}}{\pgfqpoint{2.726030in}{1.472047in}}{\pgfqpoint{2.714979in}{1.472047in}}%
\pgfpathcurveto{\pgfqpoint{2.703929in}{1.472047in}}{\pgfqpoint{2.693330in}{1.467656in}}{\pgfqpoint{2.685517in}{1.459843in}}%
\pgfpathcurveto{\pgfqpoint{2.677703in}{1.452029in}}{\pgfqpoint{2.673313in}{1.441430in}}{\pgfqpoint{2.673313in}{1.430380in}}%
\pgfpathcurveto{\pgfqpoint{2.673313in}{1.419330in}}{\pgfqpoint{2.677703in}{1.408731in}}{\pgfqpoint{2.685517in}{1.400917in}}%
\pgfpathcurveto{\pgfqpoint{2.693330in}{1.393104in}}{\pgfqpoint{2.703929in}{1.388713in}}{\pgfqpoint{2.714979in}{1.388713in}}%
\pgfpathclose%
\pgfusepath{stroke,fill}%
\end{pgfscope}%
\begin{pgfscope}%
\pgfpathrectangle{\pgfqpoint{0.511823in}{0.504323in}}{\pgfqpoint{3.218177in}{3.225677in}} %
\pgfusepath{clip}%
\pgfsetbuttcap%
\pgfsetroundjoin%
\definecolor{currentfill}{rgb}{0.000000,0.000000,0.545098}%
\pgfsetfillcolor{currentfill}%
\pgfsetfillopacity{0.400000}%
\pgfsetlinewidth{0.501875pt}%
\definecolor{currentstroke}{rgb}{0.000000,0.000000,0.545098}%
\pgfsetstrokecolor{currentstroke}%
\pgfsetstrokeopacity{0.400000}%
\pgfsetdash{}{0pt}%
\pgfpathmoveto{\pgfqpoint{2.608509in}{1.355928in}}%
\pgfpathcurveto{\pgfqpoint{2.619559in}{1.355928in}}{\pgfqpoint{2.630158in}{1.360318in}}{\pgfqpoint{2.637972in}{1.368132in}}%
\pgfpathcurveto{\pgfqpoint{2.645786in}{1.375945in}}{\pgfqpoint{2.650176in}{1.386544in}}{\pgfqpoint{2.650176in}{1.397595in}}%
\pgfpathcurveto{\pgfqpoint{2.650176in}{1.408645in}}{\pgfqpoint{2.645786in}{1.419244in}}{\pgfqpoint{2.637972in}{1.427057in}}%
\pgfpathcurveto{\pgfqpoint{2.630158in}{1.434871in}}{\pgfqpoint{2.619559in}{1.439261in}}{\pgfqpoint{2.608509in}{1.439261in}}%
\pgfpathcurveto{\pgfqpoint{2.597459in}{1.439261in}}{\pgfqpoint{2.586860in}{1.434871in}}{\pgfqpoint{2.579046in}{1.427057in}}%
\pgfpathcurveto{\pgfqpoint{2.571233in}{1.419244in}}{\pgfqpoint{2.566843in}{1.408645in}}{\pgfqpoint{2.566843in}{1.397595in}}%
\pgfpathcurveto{\pgfqpoint{2.566843in}{1.386544in}}{\pgfqpoint{2.571233in}{1.375945in}}{\pgfqpoint{2.579046in}{1.368132in}}%
\pgfpathcurveto{\pgfqpoint{2.586860in}{1.360318in}}{\pgfqpoint{2.597459in}{1.355928in}}{\pgfqpoint{2.608509in}{1.355928in}}%
\pgfpathclose%
\pgfusepath{stroke,fill}%
\end{pgfscope}%
\begin{pgfscope}%
\pgfpathrectangle{\pgfqpoint{0.511823in}{0.504323in}}{\pgfqpoint{3.218177in}{3.225677in}} %
\pgfusepath{clip}%
\pgfsetbuttcap%
\pgfsetroundjoin%
\definecolor{currentfill}{rgb}{0.000000,0.000000,0.545098}%
\pgfsetfillcolor{currentfill}%
\pgfsetfillopacity{0.400000}%
\pgfsetlinewidth{0.501875pt}%
\definecolor{currentstroke}{rgb}{0.000000,0.000000,0.545098}%
\pgfsetstrokecolor{currentstroke}%
\pgfsetstrokeopacity{0.400000}%
\pgfsetdash{}{0pt}%
\pgfpathmoveto{\pgfqpoint{2.485231in}{1.316115in}}%
\pgfpathcurveto{\pgfqpoint{2.496281in}{1.316115in}}{\pgfqpoint{2.506880in}{1.320506in}}{\pgfqpoint{2.514694in}{1.328319in}}%
\pgfpathcurveto{\pgfqpoint{2.522507in}{1.336133in}}{\pgfqpoint{2.526898in}{1.346732in}}{\pgfqpoint{2.526898in}{1.357782in}}%
\pgfpathcurveto{\pgfqpoint{2.526898in}{1.368832in}}{\pgfqpoint{2.522507in}{1.379431in}}{\pgfqpoint{2.514694in}{1.387245in}}%
\pgfpathcurveto{\pgfqpoint{2.506880in}{1.395059in}}{\pgfqpoint{2.496281in}{1.399449in}}{\pgfqpoint{2.485231in}{1.399449in}}%
\pgfpathcurveto{\pgfqpoint{2.474181in}{1.399449in}}{\pgfqpoint{2.463582in}{1.395059in}}{\pgfqpoint{2.455768in}{1.387245in}}%
\pgfpathcurveto{\pgfqpoint{2.447954in}{1.379431in}}{\pgfqpoint{2.443564in}{1.368832in}}{\pgfqpoint{2.443564in}{1.357782in}}%
\pgfpathcurveto{\pgfqpoint{2.443564in}{1.346732in}}{\pgfqpoint{2.447954in}{1.336133in}}{\pgfqpoint{2.455768in}{1.328319in}}%
\pgfpathcurveto{\pgfqpoint{2.463582in}{1.320506in}}{\pgfqpoint{2.474181in}{1.316115in}}{\pgfqpoint{2.485231in}{1.316115in}}%
\pgfpathclose%
\pgfusepath{stroke,fill}%
\end{pgfscope}%
\begin{pgfscope}%
\pgfpathrectangle{\pgfqpoint{0.511823in}{0.504323in}}{\pgfqpoint{3.218177in}{3.225677in}} %
\pgfusepath{clip}%
\pgfsetbuttcap%
\pgfsetroundjoin%
\definecolor{currentfill}{rgb}{0.000000,0.000000,0.545098}%
\pgfsetfillcolor{currentfill}%
\pgfsetfillopacity{0.400000}%
\pgfsetlinewidth{0.501875pt}%
\definecolor{currentstroke}{rgb}{0.000000,0.000000,0.545098}%
\pgfsetstrokecolor{currentstroke}%
\pgfsetstrokeopacity{0.400000}%
\pgfsetdash{}{0pt}%
\pgfpathmoveto{\pgfqpoint{2.354403in}{1.272614in}}%
\pgfpathcurveto{\pgfqpoint{2.365453in}{1.272614in}}{\pgfqpoint{2.376052in}{1.277004in}}{\pgfqpoint{2.383866in}{1.284818in}}%
\pgfpathcurveto{\pgfqpoint{2.391680in}{1.292632in}}{\pgfqpoint{2.396070in}{1.303231in}}{\pgfqpoint{2.396070in}{1.314281in}}%
\pgfpathcurveto{\pgfqpoint{2.396070in}{1.325331in}}{\pgfqpoint{2.391680in}{1.335930in}}{\pgfqpoint{2.383866in}{1.343744in}}%
\pgfpathcurveto{\pgfqpoint{2.376052in}{1.351557in}}{\pgfqpoint{2.365453in}{1.355947in}}{\pgfqpoint{2.354403in}{1.355947in}}%
\pgfpathcurveto{\pgfqpoint{2.343353in}{1.355947in}}{\pgfqpoint{2.332754in}{1.351557in}}{\pgfqpoint{2.324940in}{1.343744in}}%
\pgfpathcurveto{\pgfqpoint{2.317127in}{1.335930in}}{\pgfqpoint{2.312737in}{1.325331in}}{\pgfqpoint{2.312737in}{1.314281in}}%
\pgfpathcurveto{\pgfqpoint{2.312737in}{1.303231in}}{\pgfqpoint{2.317127in}{1.292632in}}{\pgfqpoint{2.324940in}{1.284818in}}%
\pgfpathcurveto{\pgfqpoint{2.332754in}{1.277004in}}{\pgfqpoint{2.343353in}{1.272614in}}{\pgfqpoint{2.354403in}{1.272614in}}%
\pgfpathclose%
\pgfusepath{stroke,fill}%
\end{pgfscope}%
\begin{pgfscope}%
\pgfpathrectangle{\pgfqpoint{0.511823in}{0.504323in}}{\pgfqpoint{3.218177in}{3.225677in}} %
\pgfusepath{clip}%
\pgfsetbuttcap%
\pgfsetroundjoin%
\definecolor{currentfill}{rgb}{0.000000,0.000000,0.545098}%
\pgfsetfillcolor{currentfill}%
\pgfsetfillopacity{0.400000}%
\pgfsetlinewidth{0.501875pt}%
\definecolor{currentstroke}{rgb}{0.000000,0.000000,0.545098}%
\pgfsetstrokecolor{currentstroke}%
\pgfsetstrokeopacity{0.400000}%
\pgfsetdash{}{0pt}%
\pgfpathmoveto{\pgfqpoint{2.308182in}{1.260624in}}%
\pgfpathcurveto{\pgfqpoint{2.319233in}{1.260624in}}{\pgfqpoint{2.329832in}{1.265015in}}{\pgfqpoint{2.337645in}{1.272828in}}%
\pgfpathcurveto{\pgfqpoint{2.345459in}{1.280642in}}{\pgfqpoint{2.349849in}{1.291241in}}{\pgfqpoint{2.349849in}{1.302291in}}%
\pgfpathcurveto{\pgfqpoint{2.349849in}{1.313341in}}{\pgfqpoint{2.345459in}{1.323940in}}{\pgfqpoint{2.337645in}{1.331754in}}%
\pgfpathcurveto{\pgfqpoint{2.329832in}{1.339567in}}{\pgfqpoint{2.319233in}{1.343958in}}{\pgfqpoint{2.308182in}{1.343958in}}%
\pgfpathcurveto{\pgfqpoint{2.297132in}{1.343958in}}{\pgfqpoint{2.286533in}{1.339567in}}{\pgfqpoint{2.278720in}{1.331754in}}%
\pgfpathcurveto{\pgfqpoint{2.270906in}{1.323940in}}{\pgfqpoint{2.266516in}{1.313341in}}{\pgfqpoint{2.266516in}{1.302291in}}%
\pgfpathcurveto{\pgfqpoint{2.266516in}{1.291241in}}{\pgfqpoint{2.270906in}{1.280642in}}{\pgfqpoint{2.278720in}{1.272828in}}%
\pgfpathcurveto{\pgfqpoint{2.286533in}{1.265015in}}{\pgfqpoint{2.297132in}{1.260624in}}{\pgfqpoint{2.308182in}{1.260624in}}%
\pgfpathclose%
\pgfusepath{stroke,fill}%
\end{pgfscope}%
\begin{pgfscope}%
\pgfpathrectangle{\pgfqpoint{0.511823in}{0.504323in}}{\pgfqpoint{3.218177in}{3.225677in}} %
\pgfusepath{clip}%
\pgfsetbuttcap%
\pgfsetroundjoin%
\definecolor{currentfill}{rgb}{0.000000,0.000000,0.545098}%
\pgfsetfillcolor{currentfill}%
\pgfsetfillopacity{0.400000}%
\pgfsetlinewidth{0.501875pt}%
\definecolor{currentstroke}{rgb}{0.000000,0.000000,0.545098}%
\pgfsetstrokecolor{currentstroke}%
\pgfsetstrokeopacity{0.400000}%
\pgfsetdash{}{0pt}%
\pgfpathmoveto{\pgfqpoint{2.432672in}{1.314272in}}%
\pgfpathcurveto{\pgfqpoint{2.443722in}{1.314272in}}{\pgfqpoint{2.454321in}{1.318662in}}{\pgfqpoint{2.462135in}{1.326476in}}%
\pgfpathcurveto{\pgfqpoint{2.469948in}{1.334289in}}{\pgfqpoint{2.474339in}{1.344888in}}{\pgfqpoint{2.474339in}{1.355938in}}%
\pgfpathcurveto{\pgfqpoint{2.474339in}{1.366989in}}{\pgfqpoint{2.469948in}{1.377588in}}{\pgfqpoint{2.462135in}{1.385401in}}%
\pgfpathcurveto{\pgfqpoint{2.454321in}{1.393215in}}{\pgfqpoint{2.443722in}{1.397605in}}{\pgfqpoint{2.432672in}{1.397605in}}%
\pgfpathcurveto{\pgfqpoint{2.421622in}{1.397605in}}{\pgfqpoint{2.411023in}{1.393215in}}{\pgfqpoint{2.403209in}{1.385401in}}%
\pgfpathcurveto{\pgfqpoint{2.395396in}{1.377588in}}{\pgfqpoint{2.391005in}{1.366989in}}{\pgfqpoint{2.391005in}{1.355938in}}%
\pgfpathcurveto{\pgfqpoint{2.391005in}{1.344888in}}{\pgfqpoint{2.395396in}{1.334289in}}{\pgfqpoint{2.403209in}{1.326476in}}%
\pgfpathcurveto{\pgfqpoint{2.411023in}{1.318662in}}{\pgfqpoint{2.421622in}{1.314272in}}{\pgfqpoint{2.432672in}{1.314272in}}%
\pgfpathclose%
\pgfusepath{stroke,fill}%
\end{pgfscope}%
\begin{pgfscope}%
\pgfpathrectangle{\pgfqpoint{0.511823in}{0.504323in}}{\pgfqpoint{3.218177in}{3.225677in}} %
\pgfusepath{clip}%
\pgfsetbuttcap%
\pgfsetroundjoin%
\definecolor{currentfill}{rgb}{0.000000,0.000000,0.545098}%
\pgfsetfillcolor{currentfill}%
\pgfsetfillopacity{0.400000}%
\pgfsetlinewidth{0.501875pt}%
\definecolor{currentstroke}{rgb}{0.000000,0.000000,0.545098}%
\pgfsetstrokecolor{currentstroke}%
\pgfsetstrokeopacity{0.400000}%
\pgfsetdash{}{0pt}%
\pgfpathmoveto{\pgfqpoint{2.476818in}{1.337480in}}%
\pgfpathcurveto{\pgfqpoint{2.487868in}{1.337480in}}{\pgfqpoint{2.498467in}{1.341870in}}{\pgfqpoint{2.506281in}{1.349683in}}%
\pgfpathcurveto{\pgfqpoint{2.514094in}{1.357497in}}{\pgfqpoint{2.518485in}{1.368096in}}{\pgfqpoint{2.518485in}{1.379146in}}%
\pgfpathcurveto{\pgfqpoint{2.518485in}{1.390196in}}{\pgfqpoint{2.514094in}{1.400795in}}{\pgfqpoint{2.506281in}{1.408609in}}%
\pgfpathcurveto{\pgfqpoint{2.498467in}{1.416423in}}{\pgfqpoint{2.487868in}{1.420813in}}{\pgfqpoint{2.476818in}{1.420813in}}%
\pgfpathcurveto{\pgfqpoint{2.465768in}{1.420813in}}{\pgfqpoint{2.455169in}{1.416423in}}{\pgfqpoint{2.447355in}{1.408609in}}%
\pgfpathcurveto{\pgfqpoint{2.439542in}{1.400795in}}{\pgfqpoint{2.435151in}{1.390196in}}{\pgfqpoint{2.435151in}{1.379146in}}%
\pgfpathcurveto{\pgfqpoint{2.435151in}{1.368096in}}{\pgfqpoint{2.439542in}{1.357497in}}{\pgfqpoint{2.447355in}{1.349683in}}%
\pgfpathcurveto{\pgfqpoint{2.455169in}{1.341870in}}{\pgfqpoint{2.465768in}{1.337480in}}{\pgfqpoint{2.476818in}{1.337480in}}%
\pgfpathclose%
\pgfusepath{stroke,fill}%
\end{pgfscope}%
\begin{pgfscope}%
\pgfpathrectangle{\pgfqpoint{0.511823in}{0.504323in}}{\pgfqpoint{3.218177in}{3.225677in}} %
\pgfusepath{clip}%
\pgfsetbuttcap%
\pgfsetroundjoin%
\definecolor{currentfill}{rgb}{0.000000,0.000000,0.545098}%
\pgfsetfillcolor{currentfill}%
\pgfsetfillopacity{0.400000}%
\pgfsetlinewidth{0.501875pt}%
\definecolor{currentstroke}{rgb}{0.000000,0.000000,0.545098}%
\pgfsetstrokecolor{currentstroke}%
\pgfsetstrokeopacity{0.400000}%
\pgfsetdash{}{0pt}%
\pgfpathmoveto{\pgfqpoint{2.244622in}{1.252364in}}%
\pgfpathcurveto{\pgfqpoint{2.255672in}{1.252364in}}{\pgfqpoint{2.266272in}{1.256754in}}{\pgfqpoint{2.274085in}{1.264567in}}%
\pgfpathcurveto{\pgfqpoint{2.281899in}{1.272381in}}{\pgfqpoint{2.286289in}{1.282980in}}{\pgfqpoint{2.286289in}{1.294030in}}%
\pgfpathcurveto{\pgfqpoint{2.286289in}{1.305080in}}{\pgfqpoint{2.281899in}{1.315679in}}{\pgfqpoint{2.274085in}{1.323493in}}%
\pgfpathcurveto{\pgfqpoint{2.266272in}{1.331307in}}{\pgfqpoint{2.255672in}{1.335697in}}{\pgfqpoint{2.244622in}{1.335697in}}%
\pgfpathcurveto{\pgfqpoint{2.233572in}{1.335697in}}{\pgfqpoint{2.222973in}{1.331307in}}{\pgfqpoint{2.215160in}{1.323493in}}%
\pgfpathcurveto{\pgfqpoint{2.207346in}{1.315679in}}{\pgfqpoint{2.202956in}{1.305080in}}{\pgfqpoint{2.202956in}{1.294030in}}%
\pgfpathcurveto{\pgfqpoint{2.202956in}{1.282980in}}{\pgfqpoint{2.207346in}{1.272381in}}{\pgfqpoint{2.215160in}{1.264567in}}%
\pgfpathcurveto{\pgfqpoint{2.222973in}{1.256754in}}{\pgfqpoint{2.233572in}{1.252364in}}{\pgfqpoint{2.244622in}{1.252364in}}%
\pgfpathclose%
\pgfusepath{stroke,fill}%
\end{pgfscope}%
\begin{pgfscope}%
\pgfpathrectangle{\pgfqpoint{0.511823in}{0.504323in}}{\pgfqpoint{3.218177in}{3.225677in}} %
\pgfusepath{clip}%
\pgfsetbuttcap%
\pgfsetroundjoin%
\definecolor{currentfill}{rgb}{0.000000,0.000000,0.545098}%
\pgfsetfillcolor{currentfill}%
\pgfsetfillopacity{0.400000}%
\pgfsetlinewidth{0.501875pt}%
\definecolor{currentstroke}{rgb}{0.000000,0.000000,0.545098}%
\pgfsetstrokecolor{currentstroke}%
\pgfsetstrokeopacity{0.400000}%
\pgfsetdash{}{0pt}%
\pgfpathmoveto{\pgfqpoint{2.310328in}{1.283809in}}%
\pgfpathcurveto{\pgfqpoint{2.321378in}{1.283809in}}{\pgfqpoint{2.331977in}{1.288199in}}{\pgfqpoint{2.339791in}{1.296013in}}%
\pgfpathcurveto{\pgfqpoint{2.347604in}{1.303826in}}{\pgfqpoint{2.351994in}{1.314425in}}{\pgfqpoint{2.351994in}{1.325475in}}%
\pgfpathcurveto{\pgfqpoint{2.351994in}{1.336525in}}{\pgfqpoint{2.347604in}{1.347125in}}{\pgfqpoint{2.339791in}{1.354938in}}%
\pgfpathcurveto{\pgfqpoint{2.331977in}{1.362752in}}{\pgfqpoint{2.321378in}{1.367142in}}{\pgfqpoint{2.310328in}{1.367142in}}%
\pgfpathcurveto{\pgfqpoint{2.299278in}{1.367142in}}{\pgfqpoint{2.288679in}{1.362752in}}{\pgfqpoint{2.280865in}{1.354938in}}%
\pgfpathcurveto{\pgfqpoint{2.273051in}{1.347125in}}{\pgfqpoint{2.268661in}{1.336525in}}{\pgfqpoint{2.268661in}{1.325475in}}%
\pgfpathcurveto{\pgfqpoint{2.268661in}{1.314425in}}{\pgfqpoint{2.273051in}{1.303826in}}{\pgfqpoint{2.280865in}{1.296013in}}%
\pgfpathcurveto{\pgfqpoint{2.288679in}{1.288199in}}{\pgfqpoint{2.299278in}{1.283809in}}{\pgfqpoint{2.310328in}{1.283809in}}%
\pgfpathclose%
\pgfusepath{stroke,fill}%
\end{pgfscope}%
\begin{pgfscope}%
\pgfpathrectangle{\pgfqpoint{0.511823in}{0.504323in}}{\pgfqpoint{3.218177in}{3.225677in}} %
\pgfusepath{clip}%
\pgfsetbuttcap%
\pgfsetroundjoin%
\definecolor{currentfill}{rgb}{0.000000,0.000000,0.545098}%
\pgfsetfillcolor{currentfill}%
\pgfsetfillopacity{0.400000}%
\pgfsetlinewidth{0.501875pt}%
\definecolor{currentstroke}{rgb}{0.000000,0.000000,0.545098}%
\pgfsetstrokecolor{currentstroke}%
\pgfsetstrokeopacity{0.400000}%
\pgfsetdash{}{0pt}%
\pgfpathmoveto{\pgfqpoint{2.551564in}{1.385949in}}%
\pgfpathcurveto{\pgfqpoint{2.562614in}{1.385949in}}{\pgfqpoint{2.573213in}{1.390339in}}{\pgfqpoint{2.581026in}{1.398153in}}%
\pgfpathcurveto{\pgfqpoint{2.588840in}{1.405966in}}{\pgfqpoint{2.593230in}{1.416565in}}{\pgfqpoint{2.593230in}{1.427616in}}%
\pgfpathcurveto{\pgfqpoint{2.593230in}{1.438666in}}{\pgfqpoint{2.588840in}{1.449265in}}{\pgfqpoint{2.581026in}{1.457078in}}%
\pgfpathcurveto{\pgfqpoint{2.573213in}{1.464892in}}{\pgfqpoint{2.562614in}{1.469282in}}{\pgfqpoint{2.551564in}{1.469282in}}%
\pgfpathcurveto{\pgfqpoint{2.540513in}{1.469282in}}{\pgfqpoint{2.529914in}{1.464892in}}{\pgfqpoint{2.522101in}{1.457078in}}%
\pgfpathcurveto{\pgfqpoint{2.514287in}{1.449265in}}{\pgfqpoint{2.509897in}{1.438666in}}{\pgfqpoint{2.509897in}{1.427616in}}%
\pgfpathcurveto{\pgfqpoint{2.509897in}{1.416565in}}{\pgfqpoint{2.514287in}{1.405966in}}{\pgfqpoint{2.522101in}{1.398153in}}%
\pgfpathcurveto{\pgfqpoint{2.529914in}{1.390339in}}{\pgfqpoint{2.540513in}{1.385949in}}{\pgfqpoint{2.551564in}{1.385949in}}%
\pgfpathclose%
\pgfusepath{stroke,fill}%
\end{pgfscope}%
\begin{pgfscope}%
\pgfpathrectangle{\pgfqpoint{0.511823in}{0.504323in}}{\pgfqpoint{3.218177in}{3.225677in}} %
\pgfusepath{clip}%
\pgfsetbuttcap%
\pgfsetroundjoin%
\definecolor{currentfill}{rgb}{0.000000,0.000000,0.545098}%
\pgfsetfillcolor{currentfill}%
\pgfsetfillopacity{0.400000}%
\pgfsetlinewidth{0.501875pt}%
\definecolor{currentstroke}{rgb}{0.000000,0.000000,0.545098}%
\pgfsetstrokecolor{currentstroke}%
\pgfsetstrokeopacity{0.400000}%
\pgfsetdash{}{0pt}%
\pgfpathmoveto{\pgfqpoint{2.462295in}{1.356406in}}%
\pgfpathcurveto{\pgfqpoint{2.473345in}{1.356406in}}{\pgfqpoint{2.483944in}{1.360797in}}{\pgfqpoint{2.491758in}{1.368610in}}%
\pgfpathcurveto{\pgfqpoint{2.499571in}{1.376424in}}{\pgfqpoint{2.503962in}{1.387023in}}{\pgfqpoint{2.503962in}{1.398073in}}%
\pgfpathcurveto{\pgfqpoint{2.503962in}{1.409123in}}{\pgfqpoint{2.499571in}{1.419722in}}{\pgfqpoint{2.491758in}{1.427536in}}%
\pgfpathcurveto{\pgfqpoint{2.483944in}{1.435349in}}{\pgfqpoint{2.473345in}{1.439740in}}{\pgfqpoint{2.462295in}{1.439740in}}%
\pgfpathcurveto{\pgfqpoint{2.451245in}{1.439740in}}{\pgfqpoint{2.440646in}{1.435349in}}{\pgfqpoint{2.432832in}{1.427536in}}%
\pgfpathcurveto{\pgfqpoint{2.425019in}{1.419722in}}{\pgfqpoint{2.420628in}{1.409123in}}{\pgfqpoint{2.420628in}{1.398073in}}%
\pgfpathcurveto{\pgfqpoint{2.420628in}{1.387023in}}{\pgfqpoint{2.425019in}{1.376424in}}{\pgfqpoint{2.432832in}{1.368610in}}%
\pgfpathcurveto{\pgfqpoint{2.440646in}{1.360797in}}{\pgfqpoint{2.451245in}{1.356406in}}{\pgfqpoint{2.462295in}{1.356406in}}%
\pgfpathclose%
\pgfusepath{stroke,fill}%
\end{pgfscope}%
\begin{pgfscope}%
\pgfpathrectangle{\pgfqpoint{0.511823in}{0.504323in}}{\pgfqpoint{3.218177in}{3.225677in}} %
\pgfusepath{clip}%
\pgfsetbuttcap%
\pgfsetroundjoin%
\definecolor{currentfill}{rgb}{0.000000,0.000000,0.545098}%
\pgfsetfillcolor{currentfill}%
\pgfsetfillopacity{0.400000}%
\pgfsetlinewidth{0.501875pt}%
\definecolor{currentstroke}{rgb}{0.000000,0.000000,0.545098}%
\pgfsetstrokecolor{currentstroke}%
\pgfsetstrokeopacity{0.400000}%
\pgfsetdash{}{0pt}%
\pgfpathmoveto{\pgfqpoint{2.417828in}{1.344495in}}%
\pgfpathcurveto{\pgfqpoint{2.428878in}{1.344495in}}{\pgfqpoint{2.439477in}{1.348885in}}{\pgfqpoint{2.447291in}{1.356699in}}%
\pgfpathcurveto{\pgfqpoint{2.455104in}{1.364513in}}{\pgfqpoint{2.459495in}{1.375112in}}{\pgfqpoint{2.459495in}{1.386162in}}%
\pgfpathcurveto{\pgfqpoint{2.459495in}{1.397212in}}{\pgfqpoint{2.455104in}{1.407811in}}{\pgfqpoint{2.447291in}{1.415624in}}%
\pgfpathcurveto{\pgfqpoint{2.439477in}{1.423438in}}{\pgfqpoint{2.428878in}{1.427828in}}{\pgfqpoint{2.417828in}{1.427828in}}%
\pgfpathcurveto{\pgfqpoint{2.406778in}{1.427828in}}{\pgfqpoint{2.396179in}{1.423438in}}{\pgfqpoint{2.388365in}{1.415624in}}%
\pgfpathcurveto{\pgfqpoint{2.380551in}{1.407811in}}{\pgfqpoint{2.376161in}{1.397212in}}{\pgfqpoint{2.376161in}{1.386162in}}%
\pgfpathcurveto{\pgfqpoint{2.376161in}{1.375112in}}{\pgfqpoint{2.380551in}{1.364513in}}{\pgfqpoint{2.388365in}{1.356699in}}%
\pgfpathcurveto{\pgfqpoint{2.396179in}{1.348885in}}{\pgfqpoint{2.406778in}{1.344495in}}{\pgfqpoint{2.417828in}{1.344495in}}%
\pgfpathclose%
\pgfusepath{stroke,fill}%
\end{pgfscope}%
\begin{pgfscope}%
\pgfpathrectangle{\pgfqpoint{0.511823in}{0.504323in}}{\pgfqpoint{3.218177in}{3.225677in}} %
\pgfusepath{clip}%
\pgfsetbuttcap%
\pgfsetroundjoin%
\definecolor{currentfill}{rgb}{0.000000,0.000000,0.545098}%
\pgfsetfillcolor{currentfill}%
\pgfsetfillopacity{0.400000}%
\pgfsetlinewidth{0.501875pt}%
\definecolor{currentstroke}{rgb}{0.000000,0.000000,0.545098}%
\pgfsetstrokecolor{currentstroke}%
\pgfsetstrokeopacity{0.400000}%
\pgfsetdash{}{0pt}%
\pgfpathmoveto{\pgfqpoint{2.457590in}{1.366868in}}%
\pgfpathcurveto{\pgfqpoint{2.468640in}{1.366868in}}{\pgfqpoint{2.479239in}{1.371259in}}{\pgfqpoint{2.487052in}{1.379072in}}%
\pgfpathcurveto{\pgfqpoint{2.494866in}{1.386886in}}{\pgfqpoint{2.499256in}{1.397485in}}{\pgfqpoint{2.499256in}{1.408535in}}%
\pgfpathcurveto{\pgfqpoint{2.499256in}{1.419585in}}{\pgfqpoint{2.494866in}{1.430184in}}{\pgfqpoint{2.487052in}{1.437998in}}%
\pgfpathcurveto{\pgfqpoint{2.479239in}{1.445811in}}{\pgfqpoint{2.468640in}{1.450202in}}{\pgfqpoint{2.457590in}{1.450202in}}%
\pgfpathcurveto{\pgfqpoint{2.446540in}{1.450202in}}{\pgfqpoint{2.435941in}{1.445811in}}{\pgfqpoint{2.428127in}{1.437998in}}%
\pgfpathcurveto{\pgfqpoint{2.420313in}{1.430184in}}{\pgfqpoint{2.415923in}{1.419585in}}{\pgfqpoint{2.415923in}{1.408535in}}%
\pgfpathcurveto{\pgfqpoint{2.415923in}{1.397485in}}{\pgfqpoint{2.420313in}{1.386886in}}{\pgfqpoint{2.428127in}{1.379072in}}%
\pgfpathcurveto{\pgfqpoint{2.435941in}{1.371259in}}{\pgfqpoint{2.446540in}{1.366868in}}{\pgfqpoint{2.457590in}{1.366868in}}%
\pgfpathclose%
\pgfusepath{stroke,fill}%
\end{pgfscope}%
\begin{pgfscope}%
\pgfpathrectangle{\pgfqpoint{0.511823in}{0.504323in}}{\pgfqpoint{3.218177in}{3.225677in}} %
\pgfusepath{clip}%
\pgfsetbuttcap%
\pgfsetroundjoin%
\definecolor{currentfill}{rgb}{0.000000,0.000000,0.545098}%
\pgfsetfillcolor{currentfill}%
\pgfsetfillopacity{0.400000}%
\pgfsetlinewidth{0.501875pt}%
\definecolor{currentstroke}{rgb}{0.000000,0.000000,0.545098}%
\pgfsetstrokecolor{currentstroke}%
\pgfsetstrokeopacity{0.400000}%
\pgfsetdash{}{0pt}%
\pgfpathmoveto{\pgfqpoint{2.406020in}{1.351723in}}%
\pgfpathcurveto{\pgfqpoint{2.417070in}{1.351723in}}{\pgfqpoint{2.427669in}{1.356113in}}{\pgfqpoint{2.435483in}{1.363927in}}%
\pgfpathcurveto{\pgfqpoint{2.443297in}{1.371741in}}{\pgfqpoint{2.447687in}{1.382340in}}{\pgfqpoint{2.447687in}{1.393390in}}%
\pgfpathcurveto{\pgfqpoint{2.447687in}{1.404440in}}{\pgfqpoint{2.443297in}{1.415039in}}{\pgfqpoint{2.435483in}{1.422853in}}%
\pgfpathcurveto{\pgfqpoint{2.427669in}{1.430666in}}{\pgfqpoint{2.417070in}{1.435056in}}{\pgfqpoint{2.406020in}{1.435056in}}%
\pgfpathcurveto{\pgfqpoint{2.394970in}{1.435056in}}{\pgfqpoint{2.384371in}{1.430666in}}{\pgfqpoint{2.376557in}{1.422853in}}%
\pgfpathcurveto{\pgfqpoint{2.368744in}{1.415039in}}{\pgfqpoint{2.364353in}{1.404440in}}{\pgfqpoint{2.364353in}{1.393390in}}%
\pgfpathcurveto{\pgfqpoint{2.364353in}{1.382340in}}{\pgfqpoint{2.368744in}{1.371741in}}{\pgfqpoint{2.376557in}{1.363927in}}%
\pgfpathcurveto{\pgfqpoint{2.384371in}{1.356113in}}{\pgfqpoint{2.394970in}{1.351723in}}{\pgfqpoint{2.406020in}{1.351723in}}%
\pgfpathclose%
\pgfusepath{stroke,fill}%
\end{pgfscope}%
\begin{pgfscope}%
\pgfpathrectangle{\pgfqpoint{0.511823in}{0.504323in}}{\pgfqpoint{3.218177in}{3.225677in}} %
\pgfusepath{clip}%
\pgfsetbuttcap%
\pgfsetroundjoin%
\definecolor{currentfill}{rgb}{0.000000,0.000000,0.545098}%
\pgfsetfillcolor{currentfill}%
\pgfsetfillopacity{0.400000}%
\pgfsetlinewidth{0.501875pt}%
\definecolor{currentstroke}{rgb}{0.000000,0.000000,0.545098}%
\pgfsetstrokecolor{currentstroke}%
\pgfsetstrokeopacity{0.400000}%
\pgfsetdash{}{0pt}%
\pgfpathmoveto{\pgfqpoint{2.614220in}{1.444699in}}%
\pgfpathcurveto{\pgfqpoint{2.625270in}{1.444699in}}{\pgfqpoint{2.635869in}{1.449089in}}{\pgfqpoint{2.643683in}{1.456903in}}%
\pgfpathcurveto{\pgfqpoint{2.651497in}{1.464716in}}{\pgfqpoint{2.655887in}{1.475315in}}{\pgfqpoint{2.655887in}{1.486365in}}%
\pgfpathcurveto{\pgfqpoint{2.655887in}{1.497415in}}{\pgfqpoint{2.651497in}{1.508014in}}{\pgfqpoint{2.643683in}{1.515828in}}%
\pgfpathcurveto{\pgfqpoint{2.635869in}{1.523642in}}{\pgfqpoint{2.625270in}{1.528032in}}{\pgfqpoint{2.614220in}{1.528032in}}%
\pgfpathcurveto{\pgfqpoint{2.603170in}{1.528032in}}{\pgfqpoint{2.592571in}{1.523642in}}{\pgfqpoint{2.584758in}{1.515828in}}%
\pgfpathcurveto{\pgfqpoint{2.576944in}{1.508014in}}{\pgfqpoint{2.572554in}{1.497415in}}{\pgfqpoint{2.572554in}{1.486365in}}%
\pgfpathcurveto{\pgfqpoint{2.572554in}{1.475315in}}{\pgfqpoint{2.576944in}{1.464716in}}{\pgfqpoint{2.584758in}{1.456903in}}%
\pgfpathcurveto{\pgfqpoint{2.592571in}{1.449089in}}{\pgfqpoint{2.603170in}{1.444699in}}{\pgfqpoint{2.614220in}{1.444699in}}%
\pgfpathclose%
\pgfusepath{stroke,fill}%
\end{pgfscope}%
\begin{pgfscope}%
\pgfpathrectangle{\pgfqpoint{0.511823in}{0.504323in}}{\pgfqpoint{3.218177in}{3.225677in}} %
\pgfusepath{clip}%
\pgfsetbuttcap%
\pgfsetroundjoin%
\definecolor{currentfill}{rgb}{0.000000,0.000000,0.545098}%
\pgfsetfillcolor{currentfill}%
\pgfsetfillopacity{0.400000}%
\pgfsetlinewidth{0.501875pt}%
\definecolor{currentstroke}{rgb}{0.000000,0.000000,0.545098}%
\pgfsetstrokecolor{currentstroke}%
\pgfsetstrokeopacity{0.400000}%
\pgfsetdash{}{0pt}%
\pgfpathmoveto{\pgfqpoint{2.697595in}{1.486598in}}%
\pgfpathcurveto{\pgfqpoint{2.708645in}{1.486598in}}{\pgfqpoint{2.719244in}{1.490989in}}{\pgfqpoint{2.727058in}{1.498802in}}%
\pgfpathcurveto{\pgfqpoint{2.734871in}{1.506616in}}{\pgfqpoint{2.739262in}{1.517215in}}{\pgfqpoint{2.739262in}{1.528265in}}%
\pgfpathcurveto{\pgfqpoint{2.739262in}{1.539315in}}{\pgfqpoint{2.734871in}{1.549914in}}{\pgfqpoint{2.727058in}{1.557728in}}%
\pgfpathcurveto{\pgfqpoint{2.719244in}{1.565541in}}{\pgfqpoint{2.708645in}{1.569932in}}{\pgfqpoint{2.697595in}{1.569932in}}%
\pgfpathcurveto{\pgfqpoint{2.686545in}{1.569932in}}{\pgfqpoint{2.675946in}{1.565541in}}{\pgfqpoint{2.668132in}{1.557728in}}%
\pgfpathcurveto{\pgfqpoint{2.660319in}{1.549914in}}{\pgfqpoint{2.655928in}{1.539315in}}{\pgfqpoint{2.655928in}{1.528265in}}%
\pgfpathcurveto{\pgfqpoint{2.655928in}{1.517215in}}{\pgfqpoint{2.660319in}{1.506616in}}{\pgfqpoint{2.668132in}{1.498802in}}%
\pgfpathcurveto{\pgfqpoint{2.675946in}{1.490989in}}{\pgfqpoint{2.686545in}{1.486598in}}{\pgfqpoint{2.697595in}{1.486598in}}%
\pgfpathclose%
\pgfusepath{stroke,fill}%
\end{pgfscope}%
\begin{pgfscope}%
\pgfpathrectangle{\pgfqpoint{0.511823in}{0.504323in}}{\pgfqpoint{3.218177in}{3.225677in}} %
\pgfusepath{clip}%
\pgfsetbuttcap%
\pgfsetroundjoin%
\definecolor{currentfill}{rgb}{0.000000,0.000000,0.545098}%
\pgfsetfillcolor{currentfill}%
\pgfsetfillopacity{0.400000}%
\pgfsetlinewidth{0.501875pt}%
\definecolor{currentstroke}{rgb}{0.000000,0.000000,0.545098}%
\pgfsetstrokecolor{currentstroke}%
\pgfsetstrokeopacity{0.400000}%
\pgfsetdash{}{0pt}%
\pgfpathmoveto{\pgfqpoint{2.747059in}{1.514708in}}%
\pgfpathcurveto{\pgfqpoint{2.758109in}{1.514708in}}{\pgfqpoint{2.768708in}{1.519098in}}{\pgfqpoint{2.776521in}{1.526912in}}%
\pgfpathcurveto{\pgfqpoint{2.784335in}{1.534725in}}{\pgfqpoint{2.788725in}{1.545324in}}{\pgfqpoint{2.788725in}{1.556375in}}%
\pgfpathcurveto{\pgfqpoint{2.788725in}{1.567425in}}{\pgfqpoint{2.784335in}{1.578024in}}{\pgfqpoint{2.776521in}{1.585837in}}%
\pgfpathcurveto{\pgfqpoint{2.768708in}{1.593651in}}{\pgfqpoint{2.758109in}{1.598041in}}{\pgfqpoint{2.747059in}{1.598041in}}%
\pgfpathcurveto{\pgfqpoint{2.736009in}{1.598041in}}{\pgfqpoint{2.725410in}{1.593651in}}{\pgfqpoint{2.717596in}{1.585837in}}%
\pgfpathcurveto{\pgfqpoint{2.709782in}{1.578024in}}{\pgfqpoint{2.705392in}{1.567425in}}{\pgfqpoint{2.705392in}{1.556375in}}%
\pgfpathcurveto{\pgfqpoint{2.705392in}{1.545324in}}{\pgfqpoint{2.709782in}{1.534725in}}{\pgfqpoint{2.717596in}{1.526912in}}%
\pgfpathcurveto{\pgfqpoint{2.725410in}{1.519098in}}{\pgfqpoint{2.736009in}{1.514708in}}{\pgfqpoint{2.747059in}{1.514708in}}%
\pgfpathclose%
\pgfusepath{stroke,fill}%
\end{pgfscope}%
\begin{pgfscope}%
\pgfpathrectangle{\pgfqpoint{0.511823in}{0.504323in}}{\pgfqpoint{3.218177in}{3.225677in}} %
\pgfusepath{clip}%
\pgfsetbuttcap%
\pgfsetroundjoin%
\definecolor{currentfill}{rgb}{0.000000,0.000000,0.545098}%
\pgfsetfillcolor{currentfill}%
\pgfsetfillopacity{0.400000}%
\pgfsetlinewidth{0.501875pt}%
\definecolor{currentstroke}{rgb}{0.000000,0.000000,0.545098}%
\pgfsetstrokecolor{currentstroke}%
\pgfsetstrokeopacity{0.400000}%
\pgfsetdash{}{0pt}%
\pgfpathmoveto{\pgfqpoint{2.579726in}{1.450368in}}%
\pgfpathcurveto{\pgfqpoint{2.590776in}{1.450368in}}{\pgfqpoint{2.601375in}{1.454758in}}{\pgfqpoint{2.609189in}{1.462572in}}%
\pgfpathcurveto{\pgfqpoint{2.617003in}{1.470385in}}{\pgfqpoint{2.621393in}{1.480984in}}{\pgfqpoint{2.621393in}{1.492034in}}%
\pgfpathcurveto{\pgfqpoint{2.621393in}{1.503085in}}{\pgfqpoint{2.617003in}{1.513684in}}{\pgfqpoint{2.609189in}{1.521497in}}%
\pgfpathcurveto{\pgfqpoint{2.601375in}{1.529311in}}{\pgfqpoint{2.590776in}{1.533701in}}{\pgfqpoint{2.579726in}{1.533701in}}%
\pgfpathcurveto{\pgfqpoint{2.568676in}{1.533701in}}{\pgfqpoint{2.558077in}{1.529311in}}{\pgfqpoint{2.550263in}{1.521497in}}%
\pgfpathcurveto{\pgfqpoint{2.542450in}{1.513684in}}{\pgfqpoint{2.538059in}{1.503085in}}{\pgfqpoint{2.538059in}{1.492034in}}%
\pgfpathcurveto{\pgfqpoint{2.538059in}{1.480984in}}{\pgfqpoint{2.542450in}{1.470385in}}{\pgfqpoint{2.550263in}{1.462572in}}%
\pgfpathcurveto{\pgfqpoint{2.558077in}{1.454758in}}{\pgfqpoint{2.568676in}{1.450368in}}{\pgfqpoint{2.579726in}{1.450368in}}%
\pgfpathclose%
\pgfusepath{stroke,fill}%
\end{pgfscope}%
\begin{pgfscope}%
\pgfpathrectangle{\pgfqpoint{0.511823in}{0.504323in}}{\pgfqpoint{3.218177in}{3.225677in}} %
\pgfusepath{clip}%
\pgfsetbuttcap%
\pgfsetroundjoin%
\definecolor{currentfill}{rgb}{0.000000,0.000000,0.545098}%
\pgfsetfillcolor{currentfill}%
\pgfsetfillopacity{0.400000}%
\pgfsetlinewidth{0.501875pt}%
\definecolor{currentstroke}{rgb}{0.000000,0.000000,0.545098}%
\pgfsetstrokecolor{currentstroke}%
\pgfsetstrokeopacity{0.400000}%
\pgfsetdash{}{0pt}%
\pgfpathmoveto{\pgfqpoint{2.581470in}{1.457850in}}%
\pgfpathcurveto{\pgfqpoint{2.592520in}{1.457850in}}{\pgfqpoint{2.603119in}{1.462240in}}{\pgfqpoint{2.610932in}{1.470054in}}%
\pgfpathcurveto{\pgfqpoint{2.618746in}{1.477867in}}{\pgfqpoint{2.623136in}{1.488466in}}{\pgfqpoint{2.623136in}{1.499516in}}%
\pgfpathcurveto{\pgfqpoint{2.623136in}{1.510567in}}{\pgfqpoint{2.618746in}{1.521166in}}{\pgfqpoint{2.610932in}{1.528979in}}%
\pgfpathcurveto{\pgfqpoint{2.603119in}{1.536793in}}{\pgfqpoint{2.592520in}{1.541183in}}{\pgfqpoint{2.581470in}{1.541183in}}%
\pgfpathcurveto{\pgfqpoint{2.570420in}{1.541183in}}{\pgfqpoint{2.559820in}{1.536793in}}{\pgfqpoint{2.552007in}{1.528979in}}%
\pgfpathcurveto{\pgfqpoint{2.544193in}{1.521166in}}{\pgfqpoint{2.539803in}{1.510567in}}{\pgfqpoint{2.539803in}{1.499516in}}%
\pgfpathcurveto{\pgfqpoint{2.539803in}{1.488466in}}{\pgfqpoint{2.544193in}{1.477867in}}{\pgfqpoint{2.552007in}{1.470054in}}%
\pgfpathcurveto{\pgfqpoint{2.559820in}{1.462240in}}{\pgfqpoint{2.570420in}{1.457850in}}{\pgfqpoint{2.581470in}{1.457850in}}%
\pgfpathclose%
\pgfusepath{stroke,fill}%
\end{pgfscope}%
\begin{pgfscope}%
\pgfpathrectangle{\pgfqpoint{0.511823in}{0.504323in}}{\pgfqpoint{3.218177in}{3.225677in}} %
\pgfusepath{clip}%
\pgfsetbuttcap%
\pgfsetroundjoin%
\definecolor{currentfill}{rgb}{0.000000,0.000000,0.545098}%
\pgfsetfillcolor{currentfill}%
\pgfsetfillopacity{0.400000}%
\pgfsetlinewidth{0.501875pt}%
\definecolor{currentstroke}{rgb}{0.000000,0.000000,0.545098}%
\pgfsetstrokecolor{currentstroke}%
\pgfsetstrokeopacity{0.400000}%
\pgfsetdash{}{0pt}%
\pgfpathmoveto{\pgfqpoint{2.664724in}{1.500843in}}%
\pgfpathcurveto{\pgfqpoint{2.675774in}{1.500843in}}{\pgfqpoint{2.686373in}{1.505233in}}{\pgfqpoint{2.694186in}{1.513047in}}%
\pgfpathcurveto{\pgfqpoint{2.702000in}{1.520860in}}{\pgfqpoint{2.706390in}{1.531459in}}{\pgfqpoint{2.706390in}{1.542509in}}%
\pgfpathcurveto{\pgfqpoint{2.706390in}{1.553559in}}{\pgfqpoint{2.702000in}{1.564158in}}{\pgfqpoint{2.694186in}{1.571972in}}%
\pgfpathcurveto{\pgfqpoint{2.686373in}{1.579786in}}{\pgfqpoint{2.675774in}{1.584176in}}{\pgfqpoint{2.664724in}{1.584176in}}%
\pgfpathcurveto{\pgfqpoint{2.653674in}{1.584176in}}{\pgfqpoint{2.643075in}{1.579786in}}{\pgfqpoint{2.635261in}{1.571972in}}%
\pgfpathcurveto{\pgfqpoint{2.627447in}{1.564158in}}{\pgfqpoint{2.623057in}{1.553559in}}{\pgfqpoint{2.623057in}{1.542509in}}%
\pgfpathcurveto{\pgfqpoint{2.623057in}{1.531459in}}{\pgfqpoint{2.627447in}{1.520860in}}{\pgfqpoint{2.635261in}{1.513047in}}%
\pgfpathcurveto{\pgfqpoint{2.643075in}{1.505233in}}{\pgfqpoint{2.653674in}{1.500843in}}{\pgfqpoint{2.664724in}{1.500843in}}%
\pgfpathclose%
\pgfusepath{stroke,fill}%
\end{pgfscope}%
\begin{pgfscope}%
\pgfpathrectangle{\pgfqpoint{0.511823in}{0.504323in}}{\pgfqpoint{3.218177in}{3.225677in}} %
\pgfusepath{clip}%
\pgfsetbuttcap%
\pgfsetroundjoin%
\definecolor{currentfill}{rgb}{0.000000,0.000000,0.545098}%
\pgfsetfillcolor{currentfill}%
\pgfsetfillopacity{0.400000}%
\pgfsetlinewidth{0.501875pt}%
\definecolor{currentstroke}{rgb}{0.000000,0.000000,0.545098}%
\pgfsetstrokecolor{currentstroke}%
\pgfsetstrokeopacity{0.400000}%
\pgfsetdash{}{0pt}%
\pgfpathmoveto{\pgfqpoint{2.694840in}{1.521132in}}%
\pgfpathcurveto{\pgfqpoint{2.705890in}{1.521132in}}{\pgfqpoint{2.716489in}{1.525522in}}{\pgfqpoint{2.724302in}{1.533335in}}%
\pgfpathcurveto{\pgfqpoint{2.732116in}{1.541149in}}{\pgfqpoint{2.736506in}{1.551748in}}{\pgfqpoint{2.736506in}{1.562798in}}%
\pgfpathcurveto{\pgfqpoint{2.736506in}{1.573848in}}{\pgfqpoint{2.732116in}{1.584447in}}{\pgfqpoint{2.724302in}{1.592261in}}%
\pgfpathcurveto{\pgfqpoint{2.716489in}{1.600075in}}{\pgfqpoint{2.705890in}{1.604465in}}{\pgfqpoint{2.694840in}{1.604465in}}%
\pgfpathcurveto{\pgfqpoint{2.683789in}{1.604465in}}{\pgfqpoint{2.673190in}{1.600075in}}{\pgfqpoint{2.665377in}{1.592261in}}%
\pgfpathcurveto{\pgfqpoint{2.657563in}{1.584447in}}{\pgfqpoint{2.653173in}{1.573848in}}{\pgfqpoint{2.653173in}{1.562798in}}%
\pgfpathcurveto{\pgfqpoint{2.653173in}{1.551748in}}{\pgfqpoint{2.657563in}{1.541149in}}{\pgfqpoint{2.665377in}{1.533335in}}%
\pgfpathcurveto{\pgfqpoint{2.673190in}{1.525522in}}{\pgfqpoint{2.683789in}{1.521132in}}{\pgfqpoint{2.694840in}{1.521132in}}%
\pgfpathclose%
\pgfusepath{stroke,fill}%
\end{pgfscope}%
\begin{pgfscope}%
\pgfpathrectangle{\pgfqpoint{0.511823in}{0.504323in}}{\pgfqpoint{3.218177in}{3.225677in}} %
\pgfusepath{clip}%
\pgfsetbuttcap%
\pgfsetroundjoin%
\definecolor{currentfill}{rgb}{0.000000,0.000000,0.545098}%
\pgfsetfillcolor{currentfill}%
\pgfsetfillopacity{0.400000}%
\pgfsetlinewidth{0.501875pt}%
\definecolor{currentstroke}{rgb}{0.000000,0.000000,0.545098}%
\pgfsetstrokecolor{currentstroke}%
\pgfsetstrokeopacity{0.400000}%
\pgfsetdash{}{0pt}%
\pgfpathmoveto{\pgfqpoint{2.659111in}{1.512526in}}%
\pgfpathcurveto{\pgfqpoint{2.670161in}{1.512526in}}{\pgfqpoint{2.680760in}{1.516916in}}{\pgfqpoint{2.688574in}{1.524730in}}%
\pgfpathcurveto{\pgfqpoint{2.696387in}{1.532544in}}{\pgfqpoint{2.700778in}{1.543143in}}{\pgfqpoint{2.700778in}{1.554193in}}%
\pgfpathcurveto{\pgfqpoint{2.700778in}{1.565243in}}{\pgfqpoint{2.696387in}{1.575842in}}{\pgfqpoint{2.688574in}{1.583655in}}%
\pgfpathcurveto{\pgfqpoint{2.680760in}{1.591469in}}{\pgfqpoint{2.670161in}{1.595859in}}{\pgfqpoint{2.659111in}{1.595859in}}%
\pgfpathcurveto{\pgfqpoint{2.648061in}{1.595859in}}{\pgfqpoint{2.637462in}{1.591469in}}{\pgfqpoint{2.629648in}{1.583655in}}%
\pgfpathcurveto{\pgfqpoint{2.621834in}{1.575842in}}{\pgfqpoint{2.617444in}{1.565243in}}{\pgfqpoint{2.617444in}{1.554193in}}%
\pgfpathcurveto{\pgfqpoint{2.617444in}{1.543143in}}{\pgfqpoint{2.621834in}{1.532544in}}{\pgfqpoint{2.629648in}{1.524730in}}%
\pgfpathcurveto{\pgfqpoint{2.637462in}{1.516916in}}{\pgfqpoint{2.648061in}{1.512526in}}{\pgfqpoint{2.659111in}{1.512526in}}%
\pgfpathclose%
\pgfusepath{stroke,fill}%
\end{pgfscope}%
\begin{pgfscope}%
\pgfpathrectangle{\pgfqpoint{0.511823in}{0.504323in}}{\pgfqpoint{3.218177in}{3.225677in}} %
\pgfusepath{clip}%
\pgfsetbuttcap%
\pgfsetroundjoin%
\definecolor{currentfill}{rgb}{0.000000,0.000000,0.545098}%
\pgfsetfillcolor{currentfill}%
\pgfsetfillopacity{0.400000}%
\pgfsetlinewidth{0.501875pt}%
\definecolor{currentstroke}{rgb}{0.000000,0.000000,0.545098}%
\pgfsetstrokecolor{currentstroke}%
\pgfsetstrokeopacity{0.400000}%
\pgfsetdash{}{0pt}%
\pgfpathmoveto{\pgfqpoint{2.625892in}{1.504804in}}%
\pgfpathcurveto{\pgfqpoint{2.636942in}{1.504804in}}{\pgfqpoint{2.647541in}{1.509195in}}{\pgfqpoint{2.655355in}{1.517008in}}%
\pgfpathcurveto{\pgfqpoint{2.663168in}{1.524822in}}{\pgfqpoint{2.667558in}{1.535421in}}{\pgfqpoint{2.667558in}{1.546471in}}%
\pgfpathcurveto{\pgfqpoint{2.667558in}{1.557521in}}{\pgfqpoint{2.663168in}{1.568120in}}{\pgfqpoint{2.655355in}{1.575934in}}%
\pgfpathcurveto{\pgfqpoint{2.647541in}{1.583747in}}{\pgfqpoint{2.636942in}{1.588138in}}{\pgfqpoint{2.625892in}{1.588138in}}%
\pgfpathcurveto{\pgfqpoint{2.614842in}{1.588138in}}{\pgfqpoint{2.604243in}{1.583747in}}{\pgfqpoint{2.596429in}{1.575934in}}%
\pgfpathcurveto{\pgfqpoint{2.588615in}{1.568120in}}{\pgfqpoint{2.584225in}{1.557521in}}{\pgfqpoint{2.584225in}{1.546471in}}%
\pgfpathcurveto{\pgfqpoint{2.584225in}{1.535421in}}{\pgfqpoint{2.588615in}{1.524822in}}{\pgfqpoint{2.596429in}{1.517008in}}%
\pgfpathcurveto{\pgfqpoint{2.604243in}{1.509195in}}{\pgfqpoint{2.614842in}{1.504804in}}{\pgfqpoint{2.625892in}{1.504804in}}%
\pgfpathclose%
\pgfusepath{stroke,fill}%
\end{pgfscope}%
\begin{pgfscope}%
\pgfpathrectangle{\pgfqpoint{0.511823in}{0.504323in}}{\pgfqpoint{3.218177in}{3.225677in}} %
\pgfusepath{clip}%
\pgfsetbuttcap%
\pgfsetroundjoin%
\definecolor{currentfill}{rgb}{0.000000,0.000000,0.545098}%
\pgfsetfillcolor{currentfill}%
\pgfsetfillopacity{0.400000}%
\pgfsetlinewidth{0.501875pt}%
\definecolor{currentstroke}{rgb}{0.000000,0.000000,0.545098}%
\pgfsetstrokecolor{currentstroke}%
\pgfsetstrokeopacity{0.400000}%
\pgfsetdash{}{0pt}%
\pgfpathmoveto{\pgfqpoint{2.402344in}{1.411287in}}%
\pgfpathcurveto{\pgfqpoint{2.413394in}{1.411287in}}{\pgfqpoint{2.423993in}{1.415677in}}{\pgfqpoint{2.431807in}{1.423490in}}%
\pgfpathcurveto{\pgfqpoint{2.439620in}{1.431304in}}{\pgfqpoint{2.444010in}{1.441903in}}{\pgfqpoint{2.444010in}{1.452953in}}%
\pgfpathcurveto{\pgfqpoint{2.444010in}{1.464003in}}{\pgfqpoint{2.439620in}{1.474602in}}{\pgfqpoint{2.431807in}{1.482416in}}%
\pgfpathcurveto{\pgfqpoint{2.423993in}{1.490230in}}{\pgfqpoint{2.413394in}{1.494620in}}{\pgfqpoint{2.402344in}{1.494620in}}%
\pgfpathcurveto{\pgfqpoint{2.391294in}{1.494620in}}{\pgfqpoint{2.380695in}{1.490230in}}{\pgfqpoint{2.372881in}{1.482416in}}%
\pgfpathcurveto{\pgfqpoint{2.365067in}{1.474602in}}{\pgfqpoint{2.360677in}{1.464003in}}{\pgfqpoint{2.360677in}{1.452953in}}%
\pgfpathcurveto{\pgfqpoint{2.360677in}{1.441903in}}{\pgfqpoint{2.365067in}{1.431304in}}{\pgfqpoint{2.372881in}{1.423490in}}%
\pgfpathcurveto{\pgfqpoint{2.380695in}{1.415677in}}{\pgfqpoint{2.391294in}{1.411287in}}{\pgfqpoint{2.402344in}{1.411287in}}%
\pgfpathclose%
\pgfusepath{stroke,fill}%
\end{pgfscope}%
\begin{pgfscope}%
\pgfpathrectangle{\pgfqpoint{0.511823in}{0.504323in}}{\pgfqpoint{3.218177in}{3.225677in}} %
\pgfusepath{clip}%
\pgfsetbuttcap%
\pgfsetroundjoin%
\definecolor{currentfill}{rgb}{0.000000,0.000000,0.545098}%
\pgfsetfillcolor{currentfill}%
\pgfsetfillopacity{0.400000}%
\pgfsetlinewidth{0.501875pt}%
\definecolor{currentstroke}{rgb}{0.000000,0.000000,0.545098}%
\pgfsetstrokecolor{currentstroke}%
\pgfsetstrokeopacity{0.400000}%
\pgfsetdash{}{0pt}%
\pgfpathmoveto{\pgfqpoint{2.284629in}{1.364141in}}%
\pgfpathcurveto{\pgfqpoint{2.295679in}{1.364141in}}{\pgfqpoint{2.306278in}{1.368531in}}{\pgfqpoint{2.314091in}{1.376345in}}%
\pgfpathcurveto{\pgfqpoint{2.321905in}{1.384159in}}{\pgfqpoint{2.326295in}{1.394758in}}{\pgfqpoint{2.326295in}{1.405808in}}%
\pgfpathcurveto{\pgfqpoint{2.326295in}{1.416858in}}{\pgfqpoint{2.321905in}{1.427457in}}{\pgfqpoint{2.314091in}{1.435271in}}%
\pgfpathcurveto{\pgfqpoint{2.306278in}{1.443084in}}{\pgfqpoint{2.295679in}{1.447474in}}{\pgfqpoint{2.284629in}{1.447474in}}%
\pgfpathcurveto{\pgfqpoint{2.273579in}{1.447474in}}{\pgfqpoint{2.262979in}{1.443084in}}{\pgfqpoint{2.255166in}{1.435271in}}%
\pgfpathcurveto{\pgfqpoint{2.247352in}{1.427457in}}{\pgfqpoint{2.242962in}{1.416858in}}{\pgfqpoint{2.242962in}{1.405808in}}%
\pgfpathcurveto{\pgfqpoint{2.242962in}{1.394758in}}{\pgfqpoint{2.247352in}{1.384159in}}{\pgfqpoint{2.255166in}{1.376345in}}%
\pgfpathcurveto{\pgfqpoint{2.262979in}{1.368531in}}{\pgfqpoint{2.273579in}{1.364141in}}{\pgfqpoint{2.284629in}{1.364141in}}%
\pgfpathclose%
\pgfusepath{stroke,fill}%
\end{pgfscope}%
\begin{pgfscope}%
\pgfpathrectangle{\pgfqpoint{0.511823in}{0.504323in}}{\pgfqpoint{3.218177in}{3.225677in}} %
\pgfusepath{clip}%
\pgfsetbuttcap%
\pgfsetroundjoin%
\definecolor{currentfill}{rgb}{0.000000,0.000000,0.545098}%
\pgfsetfillcolor{currentfill}%
\pgfsetfillopacity{0.400000}%
\pgfsetlinewidth{0.501875pt}%
\definecolor{currentstroke}{rgb}{0.000000,0.000000,0.545098}%
\pgfsetstrokecolor{currentstroke}%
\pgfsetstrokeopacity{0.400000}%
\pgfsetdash{}{0pt}%
\pgfpathmoveto{\pgfqpoint{2.577118in}{1.503574in}}%
\pgfpathcurveto{\pgfqpoint{2.588168in}{1.503574in}}{\pgfqpoint{2.598767in}{1.507965in}}{\pgfqpoint{2.606581in}{1.515778in}}%
\pgfpathcurveto{\pgfqpoint{2.614394in}{1.523592in}}{\pgfqpoint{2.618784in}{1.534191in}}{\pgfqpoint{2.618784in}{1.545241in}}%
\pgfpathcurveto{\pgfqpoint{2.618784in}{1.556291in}}{\pgfqpoint{2.614394in}{1.566890in}}{\pgfqpoint{2.606581in}{1.574704in}}%
\pgfpathcurveto{\pgfqpoint{2.598767in}{1.582517in}}{\pgfqpoint{2.588168in}{1.586908in}}{\pgfqpoint{2.577118in}{1.586908in}}%
\pgfpathcurveto{\pgfqpoint{2.566068in}{1.586908in}}{\pgfqpoint{2.555469in}{1.582517in}}{\pgfqpoint{2.547655in}{1.574704in}}%
\pgfpathcurveto{\pgfqpoint{2.539841in}{1.566890in}}{\pgfqpoint{2.535451in}{1.556291in}}{\pgfqpoint{2.535451in}{1.545241in}}%
\pgfpathcurveto{\pgfqpoint{2.535451in}{1.534191in}}{\pgfqpoint{2.539841in}{1.523592in}}{\pgfqpoint{2.547655in}{1.515778in}}%
\pgfpathcurveto{\pgfqpoint{2.555469in}{1.507965in}}{\pgfqpoint{2.566068in}{1.503574in}}{\pgfqpoint{2.577118in}{1.503574in}}%
\pgfpathclose%
\pgfusepath{stroke,fill}%
\end{pgfscope}%
\begin{pgfscope}%
\pgfpathrectangle{\pgfqpoint{0.511823in}{0.504323in}}{\pgfqpoint{3.218177in}{3.225677in}} %
\pgfusepath{clip}%
\pgfsetbuttcap%
\pgfsetroundjoin%
\definecolor{currentfill}{rgb}{0.000000,0.000000,0.545098}%
\pgfsetfillcolor{currentfill}%
\pgfsetfillopacity{0.400000}%
\pgfsetlinewidth{0.501875pt}%
\definecolor{currentstroke}{rgb}{0.000000,0.000000,0.545098}%
\pgfsetstrokecolor{currentstroke}%
\pgfsetstrokeopacity{0.400000}%
\pgfsetdash{}{0pt}%
\pgfpathmoveto{\pgfqpoint{2.458739in}{1.455942in}}%
\pgfpathcurveto{\pgfqpoint{2.469789in}{1.455942in}}{\pgfqpoint{2.480388in}{1.460332in}}{\pgfqpoint{2.488201in}{1.468146in}}%
\pgfpathcurveto{\pgfqpoint{2.496015in}{1.475959in}}{\pgfqpoint{2.500405in}{1.486558in}}{\pgfqpoint{2.500405in}{1.497609in}}%
\pgfpathcurveto{\pgfqpoint{2.500405in}{1.508659in}}{\pgfqpoint{2.496015in}{1.519258in}}{\pgfqpoint{2.488201in}{1.527071in}}%
\pgfpathcurveto{\pgfqpoint{2.480388in}{1.534885in}}{\pgfqpoint{2.469789in}{1.539275in}}{\pgfqpoint{2.458739in}{1.539275in}}%
\pgfpathcurveto{\pgfqpoint{2.447689in}{1.539275in}}{\pgfqpoint{2.437089in}{1.534885in}}{\pgfqpoint{2.429276in}{1.527071in}}%
\pgfpathcurveto{\pgfqpoint{2.421462in}{1.519258in}}{\pgfqpoint{2.417072in}{1.508659in}}{\pgfqpoint{2.417072in}{1.497609in}}%
\pgfpathcurveto{\pgfqpoint{2.417072in}{1.486558in}}{\pgfqpoint{2.421462in}{1.475959in}}{\pgfqpoint{2.429276in}{1.468146in}}%
\pgfpathcurveto{\pgfqpoint{2.437089in}{1.460332in}}{\pgfqpoint{2.447689in}{1.455942in}}{\pgfqpoint{2.458739in}{1.455942in}}%
\pgfpathclose%
\pgfusepath{stroke,fill}%
\end{pgfscope}%
\begin{pgfscope}%
\pgfpathrectangle{\pgfqpoint{0.511823in}{0.504323in}}{\pgfqpoint{3.218177in}{3.225677in}} %
\pgfusepath{clip}%
\pgfsetbuttcap%
\pgfsetroundjoin%
\definecolor{currentfill}{rgb}{0.000000,0.000000,0.545098}%
\pgfsetfillcolor{currentfill}%
\pgfsetfillopacity{0.400000}%
\pgfsetlinewidth{0.501875pt}%
\definecolor{currentstroke}{rgb}{0.000000,0.000000,0.545098}%
\pgfsetstrokecolor{currentstroke}%
\pgfsetstrokeopacity{0.400000}%
\pgfsetdash{}{0pt}%
\pgfpathmoveto{\pgfqpoint{2.460499in}{1.463232in}}%
\pgfpathcurveto{\pgfqpoint{2.471549in}{1.463232in}}{\pgfqpoint{2.482148in}{1.467622in}}{\pgfqpoint{2.489962in}{1.475435in}}%
\pgfpathcurveto{\pgfqpoint{2.497775in}{1.483249in}}{\pgfqpoint{2.502165in}{1.493848in}}{\pgfqpoint{2.502165in}{1.504898in}}%
\pgfpathcurveto{\pgfqpoint{2.502165in}{1.515948in}}{\pgfqpoint{2.497775in}{1.526547in}}{\pgfqpoint{2.489962in}{1.534361in}}%
\pgfpathcurveto{\pgfqpoint{2.482148in}{1.542175in}}{\pgfqpoint{2.471549in}{1.546565in}}{\pgfqpoint{2.460499in}{1.546565in}}%
\pgfpathcurveto{\pgfqpoint{2.449449in}{1.546565in}}{\pgfqpoint{2.438850in}{1.542175in}}{\pgfqpoint{2.431036in}{1.534361in}}%
\pgfpathcurveto{\pgfqpoint{2.423222in}{1.526547in}}{\pgfqpoint{2.418832in}{1.515948in}}{\pgfqpoint{2.418832in}{1.504898in}}%
\pgfpathcurveto{\pgfqpoint{2.418832in}{1.493848in}}{\pgfqpoint{2.423222in}{1.483249in}}{\pgfqpoint{2.431036in}{1.475435in}}%
\pgfpathcurveto{\pgfqpoint{2.438850in}{1.467622in}}{\pgfqpoint{2.449449in}{1.463232in}}{\pgfqpoint{2.460499in}{1.463232in}}%
\pgfpathclose%
\pgfusepath{stroke,fill}%
\end{pgfscope}%
\begin{pgfscope}%
\pgfpathrectangle{\pgfqpoint{0.511823in}{0.504323in}}{\pgfqpoint{3.218177in}{3.225677in}} %
\pgfusepath{clip}%
\pgfsetbuttcap%
\pgfsetroundjoin%
\definecolor{currentfill}{rgb}{0.000000,0.000000,0.545098}%
\pgfsetfillcolor{currentfill}%
\pgfsetfillopacity{0.400000}%
\pgfsetlinewidth{0.501875pt}%
\definecolor{currentstroke}{rgb}{0.000000,0.000000,0.545098}%
\pgfsetstrokecolor{currentstroke}%
\pgfsetstrokeopacity{0.400000}%
\pgfsetdash{}{0pt}%
\pgfpathmoveto{\pgfqpoint{2.358885in}{1.422192in}}%
\pgfpathcurveto{\pgfqpoint{2.369935in}{1.422192in}}{\pgfqpoint{2.380534in}{1.426583in}}{\pgfqpoint{2.388347in}{1.434396in}}%
\pgfpathcurveto{\pgfqpoint{2.396161in}{1.442210in}}{\pgfqpoint{2.400551in}{1.452809in}}{\pgfqpoint{2.400551in}{1.463859in}}%
\pgfpathcurveto{\pgfqpoint{2.400551in}{1.474909in}}{\pgfqpoint{2.396161in}{1.485508in}}{\pgfqpoint{2.388347in}{1.493322in}}%
\pgfpathcurveto{\pgfqpoint{2.380534in}{1.501135in}}{\pgfqpoint{2.369935in}{1.505526in}}{\pgfqpoint{2.358885in}{1.505526in}}%
\pgfpathcurveto{\pgfqpoint{2.347834in}{1.505526in}}{\pgfqpoint{2.337235in}{1.501135in}}{\pgfqpoint{2.329422in}{1.493322in}}%
\pgfpathcurveto{\pgfqpoint{2.321608in}{1.485508in}}{\pgfqpoint{2.317218in}{1.474909in}}{\pgfqpoint{2.317218in}{1.463859in}}%
\pgfpathcurveto{\pgfqpoint{2.317218in}{1.452809in}}{\pgfqpoint{2.321608in}{1.442210in}}{\pgfqpoint{2.329422in}{1.434396in}}%
\pgfpathcurveto{\pgfqpoint{2.337235in}{1.426583in}}{\pgfqpoint{2.347834in}{1.422192in}}{\pgfqpoint{2.358885in}{1.422192in}}%
\pgfpathclose%
\pgfusepath{stroke,fill}%
\end{pgfscope}%
\begin{pgfscope}%
\pgfpathrectangle{\pgfqpoint{0.511823in}{0.504323in}}{\pgfqpoint{3.218177in}{3.225677in}} %
\pgfusepath{clip}%
\pgfsetbuttcap%
\pgfsetroundjoin%
\definecolor{currentfill}{rgb}{0.000000,0.000000,0.545098}%
\pgfsetfillcolor{currentfill}%
\pgfsetfillopacity{0.400000}%
\pgfsetlinewidth{0.501875pt}%
\definecolor{currentstroke}{rgb}{0.000000,0.000000,0.545098}%
\pgfsetstrokecolor{currentstroke}%
\pgfsetstrokeopacity{0.400000}%
\pgfsetdash{}{0pt}%
\pgfpathmoveto{\pgfqpoint{2.307530in}{1.404125in}}%
\pgfpathcurveto{\pgfqpoint{2.318580in}{1.404125in}}{\pgfqpoint{2.329179in}{1.408515in}}{\pgfqpoint{2.336993in}{1.416329in}}%
\pgfpathcurveto{\pgfqpoint{2.344807in}{1.424142in}}{\pgfqpoint{2.349197in}{1.434741in}}{\pgfqpoint{2.349197in}{1.445792in}}%
\pgfpathcurveto{\pgfqpoint{2.349197in}{1.456842in}}{\pgfqpoint{2.344807in}{1.467441in}}{\pgfqpoint{2.336993in}{1.475254in}}%
\pgfpathcurveto{\pgfqpoint{2.329179in}{1.483068in}}{\pgfqpoint{2.318580in}{1.487458in}}{\pgfqpoint{2.307530in}{1.487458in}}%
\pgfpathcurveto{\pgfqpoint{2.296480in}{1.487458in}}{\pgfqpoint{2.285881in}{1.483068in}}{\pgfqpoint{2.278067in}{1.475254in}}%
\pgfpathcurveto{\pgfqpoint{2.270254in}{1.467441in}}{\pgfqpoint{2.265864in}{1.456842in}}{\pgfqpoint{2.265864in}{1.445792in}}%
\pgfpathcurveto{\pgfqpoint{2.265864in}{1.434741in}}{\pgfqpoint{2.270254in}{1.424142in}}{\pgfqpoint{2.278067in}{1.416329in}}%
\pgfpathcurveto{\pgfqpoint{2.285881in}{1.408515in}}{\pgfqpoint{2.296480in}{1.404125in}}{\pgfqpoint{2.307530in}{1.404125in}}%
\pgfpathclose%
\pgfusepath{stroke,fill}%
\end{pgfscope}%
\begin{pgfscope}%
\pgfpathrectangle{\pgfqpoint{0.511823in}{0.504323in}}{\pgfqpoint{3.218177in}{3.225677in}} %
\pgfusepath{clip}%
\pgfsetbuttcap%
\pgfsetroundjoin%
\definecolor{currentfill}{rgb}{0.000000,0.000000,0.545098}%
\pgfsetfillcolor{currentfill}%
\pgfsetfillopacity{0.400000}%
\pgfsetlinewidth{0.501875pt}%
\definecolor{currentstroke}{rgb}{0.000000,0.000000,0.545098}%
\pgfsetstrokecolor{currentstroke}%
\pgfsetstrokeopacity{0.400000}%
\pgfsetdash{}{0pt}%
\pgfpathmoveto{\pgfqpoint{2.624301in}{1.560631in}}%
\pgfpathcurveto{\pgfqpoint{2.635351in}{1.560631in}}{\pgfqpoint{2.645950in}{1.565021in}}{\pgfqpoint{2.653763in}{1.572835in}}%
\pgfpathcurveto{\pgfqpoint{2.661577in}{1.580648in}}{\pgfqpoint{2.665967in}{1.591247in}}{\pgfqpoint{2.665967in}{1.602297in}}%
\pgfpathcurveto{\pgfqpoint{2.665967in}{1.613348in}}{\pgfqpoint{2.661577in}{1.623947in}}{\pgfqpoint{2.653763in}{1.631760in}}%
\pgfpathcurveto{\pgfqpoint{2.645950in}{1.639574in}}{\pgfqpoint{2.635351in}{1.643964in}}{\pgfqpoint{2.624301in}{1.643964in}}%
\pgfpathcurveto{\pgfqpoint{2.613251in}{1.643964in}}{\pgfqpoint{2.602652in}{1.639574in}}{\pgfqpoint{2.594838in}{1.631760in}}%
\pgfpathcurveto{\pgfqpoint{2.587024in}{1.623947in}}{\pgfqpoint{2.582634in}{1.613348in}}{\pgfqpoint{2.582634in}{1.602297in}}%
\pgfpathcurveto{\pgfqpoint{2.582634in}{1.591247in}}{\pgfqpoint{2.587024in}{1.580648in}}{\pgfqpoint{2.594838in}{1.572835in}}%
\pgfpathcurveto{\pgfqpoint{2.602652in}{1.565021in}}{\pgfqpoint{2.613251in}{1.560631in}}{\pgfqpoint{2.624301in}{1.560631in}}%
\pgfpathclose%
\pgfusepath{stroke,fill}%
\end{pgfscope}%
\begin{pgfscope}%
\pgfpathrectangle{\pgfqpoint{0.511823in}{0.504323in}}{\pgfqpoint{3.218177in}{3.225677in}} %
\pgfusepath{clip}%
\pgfsetbuttcap%
\pgfsetroundjoin%
\definecolor{currentfill}{rgb}{0.000000,0.000000,0.545098}%
\pgfsetfillcolor{currentfill}%
\pgfsetfillopacity{0.400000}%
\pgfsetlinewidth{0.501875pt}%
\definecolor{currentstroke}{rgb}{0.000000,0.000000,0.545098}%
\pgfsetstrokecolor{currentstroke}%
\pgfsetstrokeopacity{0.400000}%
\pgfsetdash{}{0pt}%
\pgfpathmoveto{\pgfqpoint{2.510643in}{1.513368in}}%
\pgfpathcurveto{\pgfqpoint{2.521693in}{1.513368in}}{\pgfqpoint{2.532292in}{1.517759in}}{\pgfqpoint{2.540106in}{1.525572in}}%
\pgfpathcurveto{\pgfqpoint{2.547919in}{1.533386in}}{\pgfqpoint{2.552310in}{1.543985in}}{\pgfqpoint{2.552310in}{1.555035in}}%
\pgfpathcurveto{\pgfqpoint{2.552310in}{1.566085in}}{\pgfqpoint{2.547919in}{1.576684in}}{\pgfqpoint{2.540106in}{1.584498in}}%
\pgfpathcurveto{\pgfqpoint{2.532292in}{1.592311in}}{\pgfqpoint{2.521693in}{1.596702in}}{\pgfqpoint{2.510643in}{1.596702in}}%
\pgfpathcurveto{\pgfqpoint{2.499593in}{1.596702in}}{\pgfqpoint{2.488994in}{1.592311in}}{\pgfqpoint{2.481180in}{1.584498in}}%
\pgfpathcurveto{\pgfqpoint{2.473367in}{1.576684in}}{\pgfqpoint{2.468976in}{1.566085in}}{\pgfqpoint{2.468976in}{1.555035in}}%
\pgfpathcurveto{\pgfqpoint{2.468976in}{1.543985in}}{\pgfqpoint{2.473367in}{1.533386in}}{\pgfqpoint{2.481180in}{1.525572in}}%
\pgfpathcurveto{\pgfqpoint{2.488994in}{1.517759in}}{\pgfqpoint{2.499593in}{1.513368in}}{\pgfqpoint{2.510643in}{1.513368in}}%
\pgfpathclose%
\pgfusepath{stroke,fill}%
\end{pgfscope}%
\begin{pgfscope}%
\pgfpathrectangle{\pgfqpoint{0.511823in}{0.504323in}}{\pgfqpoint{3.218177in}{3.225677in}} %
\pgfusepath{clip}%
\pgfsetbuttcap%
\pgfsetroundjoin%
\definecolor{currentfill}{rgb}{0.000000,0.000000,0.545098}%
\pgfsetfillcolor{currentfill}%
\pgfsetfillopacity{0.400000}%
\pgfsetlinewidth{0.501875pt}%
\definecolor{currentstroke}{rgb}{0.000000,0.000000,0.545098}%
\pgfsetstrokecolor{currentstroke}%
\pgfsetstrokeopacity{0.400000}%
\pgfsetdash{}{0pt}%
\pgfpathmoveto{\pgfqpoint{2.327483in}{1.431743in}}%
\pgfpathcurveto{\pgfqpoint{2.338533in}{1.431743in}}{\pgfqpoint{2.349132in}{1.436133in}}{\pgfqpoint{2.356946in}{1.443947in}}%
\pgfpathcurveto{\pgfqpoint{2.364759in}{1.451761in}}{\pgfqpoint{2.369150in}{1.462360in}}{\pgfqpoint{2.369150in}{1.473410in}}%
\pgfpathcurveto{\pgfqpoint{2.369150in}{1.484460in}}{\pgfqpoint{2.364759in}{1.495059in}}{\pgfqpoint{2.356946in}{1.502873in}}%
\pgfpathcurveto{\pgfqpoint{2.349132in}{1.510686in}}{\pgfqpoint{2.338533in}{1.515077in}}{\pgfqpoint{2.327483in}{1.515077in}}%
\pgfpathcurveto{\pgfqpoint{2.316433in}{1.515077in}}{\pgfqpoint{2.305834in}{1.510686in}}{\pgfqpoint{2.298020in}{1.502873in}}%
\pgfpathcurveto{\pgfqpoint{2.290207in}{1.495059in}}{\pgfqpoint{2.285816in}{1.484460in}}{\pgfqpoint{2.285816in}{1.473410in}}%
\pgfpathcurveto{\pgfqpoint{2.285816in}{1.462360in}}{\pgfqpoint{2.290207in}{1.451761in}}{\pgfqpoint{2.298020in}{1.443947in}}%
\pgfpathcurveto{\pgfqpoint{2.305834in}{1.436133in}}{\pgfqpoint{2.316433in}{1.431743in}}{\pgfqpoint{2.327483in}{1.431743in}}%
\pgfpathclose%
\pgfusepath{stroke,fill}%
\end{pgfscope}%
\begin{pgfscope}%
\pgfpathrectangle{\pgfqpoint{0.511823in}{0.504323in}}{\pgfqpoint{3.218177in}{3.225677in}} %
\pgfusepath{clip}%
\pgfsetbuttcap%
\pgfsetroundjoin%
\definecolor{currentfill}{rgb}{0.000000,0.000000,0.545098}%
\pgfsetfillcolor{currentfill}%
\pgfsetfillopacity{0.400000}%
\pgfsetlinewidth{0.501875pt}%
\definecolor{currentstroke}{rgb}{0.000000,0.000000,0.545098}%
\pgfsetstrokecolor{currentstroke}%
\pgfsetstrokeopacity{0.400000}%
\pgfsetdash{}{0pt}%
\pgfpathmoveto{\pgfqpoint{2.495915in}{1.519767in}}%
\pgfpathcurveto{\pgfqpoint{2.506966in}{1.519767in}}{\pgfqpoint{2.517565in}{1.524157in}}{\pgfqpoint{2.525378in}{1.531971in}}%
\pgfpathcurveto{\pgfqpoint{2.533192in}{1.539784in}}{\pgfqpoint{2.537582in}{1.550383in}}{\pgfqpoint{2.537582in}{1.561434in}}%
\pgfpathcurveto{\pgfqpoint{2.537582in}{1.572484in}}{\pgfqpoint{2.533192in}{1.583083in}}{\pgfqpoint{2.525378in}{1.590896in}}%
\pgfpathcurveto{\pgfqpoint{2.517565in}{1.598710in}}{\pgfqpoint{2.506966in}{1.603100in}}{\pgfqpoint{2.495915in}{1.603100in}}%
\pgfpathcurveto{\pgfqpoint{2.484865in}{1.603100in}}{\pgfqpoint{2.474266in}{1.598710in}}{\pgfqpoint{2.466453in}{1.590896in}}%
\pgfpathcurveto{\pgfqpoint{2.458639in}{1.583083in}}{\pgfqpoint{2.454249in}{1.572484in}}{\pgfqpoint{2.454249in}{1.561434in}}%
\pgfpathcurveto{\pgfqpoint{2.454249in}{1.550383in}}{\pgfqpoint{2.458639in}{1.539784in}}{\pgfqpoint{2.466453in}{1.531971in}}%
\pgfpathcurveto{\pgfqpoint{2.474266in}{1.524157in}}{\pgfqpoint{2.484865in}{1.519767in}}{\pgfqpoint{2.495915in}{1.519767in}}%
\pgfpathclose%
\pgfusepath{stroke,fill}%
\end{pgfscope}%
\begin{pgfscope}%
\pgfpathrectangle{\pgfqpoint{0.511823in}{0.504323in}}{\pgfqpoint{3.218177in}{3.225677in}} %
\pgfusepath{clip}%
\pgfsetbuttcap%
\pgfsetroundjoin%
\definecolor{currentfill}{rgb}{0.000000,0.000000,0.545098}%
\pgfsetfillcolor{currentfill}%
\pgfsetfillopacity{0.400000}%
\pgfsetlinewidth{0.501875pt}%
\definecolor{currentstroke}{rgb}{0.000000,0.000000,0.545098}%
\pgfsetstrokecolor{currentstroke}%
\pgfsetstrokeopacity{0.400000}%
\pgfsetdash{}{0pt}%
\pgfpathmoveto{\pgfqpoint{2.482973in}{1.520183in}}%
\pgfpathcurveto{\pgfqpoint{2.494023in}{1.520183in}}{\pgfqpoint{2.504622in}{1.524574in}}{\pgfqpoint{2.512436in}{1.532387in}}%
\pgfpathcurveto{\pgfqpoint{2.520250in}{1.540201in}}{\pgfqpoint{2.524640in}{1.550800in}}{\pgfqpoint{2.524640in}{1.561850in}}%
\pgfpathcurveto{\pgfqpoint{2.524640in}{1.572900in}}{\pgfqpoint{2.520250in}{1.583499in}}{\pgfqpoint{2.512436in}{1.591313in}}%
\pgfpathcurveto{\pgfqpoint{2.504622in}{1.599126in}}{\pgfqpoint{2.494023in}{1.603517in}}{\pgfqpoint{2.482973in}{1.603517in}}%
\pgfpathcurveto{\pgfqpoint{2.471923in}{1.603517in}}{\pgfqpoint{2.461324in}{1.599126in}}{\pgfqpoint{2.453510in}{1.591313in}}%
\pgfpathcurveto{\pgfqpoint{2.445697in}{1.583499in}}{\pgfqpoint{2.441307in}{1.572900in}}{\pgfqpoint{2.441307in}{1.561850in}}%
\pgfpathcurveto{\pgfqpoint{2.441307in}{1.550800in}}{\pgfqpoint{2.445697in}{1.540201in}}{\pgfqpoint{2.453510in}{1.532387in}}%
\pgfpathcurveto{\pgfqpoint{2.461324in}{1.524574in}}{\pgfqpoint{2.471923in}{1.520183in}}{\pgfqpoint{2.482973in}{1.520183in}}%
\pgfpathclose%
\pgfusepath{stroke,fill}%
\end{pgfscope}%
\begin{pgfscope}%
\pgfpathrectangle{\pgfqpoint{0.511823in}{0.504323in}}{\pgfqpoint{3.218177in}{3.225677in}} %
\pgfusepath{clip}%
\pgfsetbuttcap%
\pgfsetroundjoin%
\definecolor{currentfill}{rgb}{0.000000,0.000000,0.545098}%
\pgfsetfillcolor{currentfill}%
\pgfsetfillopacity{0.400000}%
\pgfsetlinewidth{0.501875pt}%
\definecolor{currentstroke}{rgb}{0.000000,0.000000,0.545098}%
\pgfsetstrokecolor{currentstroke}%
\pgfsetstrokeopacity{0.400000}%
\pgfsetdash{}{0pt}%
\pgfpathmoveto{\pgfqpoint{2.505326in}{1.537955in}}%
\pgfpathcurveto{\pgfqpoint{2.516377in}{1.537955in}}{\pgfqpoint{2.526976in}{1.542345in}}{\pgfqpoint{2.534789in}{1.550158in}}%
\pgfpathcurveto{\pgfqpoint{2.542603in}{1.557972in}}{\pgfqpoint{2.546993in}{1.568571in}}{\pgfqpoint{2.546993in}{1.579621in}}%
\pgfpathcurveto{\pgfqpoint{2.546993in}{1.590671in}}{\pgfqpoint{2.542603in}{1.601270in}}{\pgfqpoint{2.534789in}{1.609084in}}%
\pgfpathcurveto{\pgfqpoint{2.526976in}{1.616898in}}{\pgfqpoint{2.516377in}{1.621288in}}{\pgfqpoint{2.505326in}{1.621288in}}%
\pgfpathcurveto{\pgfqpoint{2.494276in}{1.621288in}}{\pgfqpoint{2.483677in}{1.616898in}}{\pgfqpoint{2.475864in}{1.609084in}}%
\pgfpathcurveto{\pgfqpoint{2.468050in}{1.601270in}}{\pgfqpoint{2.463660in}{1.590671in}}{\pgfqpoint{2.463660in}{1.579621in}}%
\pgfpathcurveto{\pgfqpoint{2.463660in}{1.568571in}}{\pgfqpoint{2.468050in}{1.557972in}}{\pgfqpoint{2.475864in}{1.550158in}}%
\pgfpathcurveto{\pgfqpoint{2.483677in}{1.542345in}}{\pgfqpoint{2.494276in}{1.537955in}}{\pgfqpoint{2.505326in}{1.537955in}}%
\pgfpathclose%
\pgfusepath{stroke,fill}%
\end{pgfscope}%
\begin{pgfscope}%
\pgfpathrectangle{\pgfqpoint{0.511823in}{0.504323in}}{\pgfqpoint{3.218177in}{3.225677in}} %
\pgfusepath{clip}%
\pgfsetbuttcap%
\pgfsetroundjoin%
\definecolor{currentfill}{rgb}{0.000000,0.000000,0.545098}%
\pgfsetfillcolor{currentfill}%
\pgfsetfillopacity{0.400000}%
\pgfsetlinewidth{0.501875pt}%
\definecolor{currentstroke}{rgb}{0.000000,0.000000,0.545098}%
\pgfsetstrokecolor{currentstroke}%
\pgfsetstrokeopacity{0.400000}%
\pgfsetdash{}{0pt}%
\pgfpathmoveto{\pgfqpoint{2.702491in}{1.642898in}}%
\pgfpathcurveto{\pgfqpoint{2.713541in}{1.642898in}}{\pgfqpoint{2.724140in}{1.647288in}}{\pgfqpoint{2.731954in}{1.655102in}}%
\pgfpathcurveto{\pgfqpoint{2.739767in}{1.662916in}}{\pgfqpoint{2.744158in}{1.673515in}}{\pgfqpoint{2.744158in}{1.684565in}}%
\pgfpathcurveto{\pgfqpoint{2.744158in}{1.695615in}}{\pgfqpoint{2.739767in}{1.706214in}}{\pgfqpoint{2.731954in}{1.714028in}}%
\pgfpathcurveto{\pgfqpoint{2.724140in}{1.721841in}}{\pgfqpoint{2.713541in}{1.726232in}}{\pgfqpoint{2.702491in}{1.726232in}}%
\pgfpathcurveto{\pgfqpoint{2.691441in}{1.726232in}}{\pgfqpoint{2.680842in}{1.721841in}}{\pgfqpoint{2.673028in}{1.714028in}}%
\pgfpathcurveto{\pgfqpoint{2.665215in}{1.706214in}}{\pgfqpoint{2.660824in}{1.695615in}}{\pgfqpoint{2.660824in}{1.684565in}}%
\pgfpathcurveto{\pgfqpoint{2.660824in}{1.673515in}}{\pgfqpoint{2.665215in}{1.662916in}}{\pgfqpoint{2.673028in}{1.655102in}}%
\pgfpathcurveto{\pgfqpoint{2.680842in}{1.647288in}}{\pgfqpoint{2.691441in}{1.642898in}}{\pgfqpoint{2.702491in}{1.642898in}}%
\pgfpathclose%
\pgfusepath{stroke,fill}%
\end{pgfscope}%
\begin{pgfscope}%
\pgfpathrectangle{\pgfqpoint{0.511823in}{0.504323in}}{\pgfqpoint{3.218177in}{3.225677in}} %
\pgfusepath{clip}%
\pgfsetbuttcap%
\pgfsetroundjoin%
\definecolor{currentfill}{rgb}{0.000000,0.000000,0.545098}%
\pgfsetfillcolor{currentfill}%
\pgfsetfillopacity{0.400000}%
\pgfsetlinewidth{0.501875pt}%
\definecolor{currentstroke}{rgb}{0.000000,0.000000,0.545098}%
\pgfsetstrokecolor{currentstroke}%
\pgfsetstrokeopacity{0.400000}%
\pgfsetdash{}{0pt}%
\pgfpathmoveto{\pgfqpoint{2.434214in}{1.516005in}}%
\pgfpathcurveto{\pgfqpoint{2.445264in}{1.516005in}}{\pgfqpoint{2.455863in}{1.520395in}}{\pgfqpoint{2.463677in}{1.528208in}}%
\pgfpathcurveto{\pgfqpoint{2.471490in}{1.536022in}}{\pgfqpoint{2.475881in}{1.546621in}}{\pgfqpoint{2.475881in}{1.557671in}}%
\pgfpathcurveto{\pgfqpoint{2.475881in}{1.568721in}}{\pgfqpoint{2.471490in}{1.579320in}}{\pgfqpoint{2.463677in}{1.587134in}}%
\pgfpathcurveto{\pgfqpoint{2.455863in}{1.594948in}}{\pgfqpoint{2.445264in}{1.599338in}}{\pgfqpoint{2.434214in}{1.599338in}}%
\pgfpathcurveto{\pgfqpoint{2.423164in}{1.599338in}}{\pgfqpoint{2.412565in}{1.594948in}}{\pgfqpoint{2.404751in}{1.587134in}}%
\pgfpathcurveto{\pgfqpoint{2.396938in}{1.579320in}}{\pgfqpoint{2.392547in}{1.568721in}}{\pgfqpoint{2.392547in}{1.557671in}}%
\pgfpathcurveto{\pgfqpoint{2.392547in}{1.546621in}}{\pgfqpoint{2.396938in}{1.536022in}}{\pgfqpoint{2.404751in}{1.528208in}}%
\pgfpathcurveto{\pgfqpoint{2.412565in}{1.520395in}}{\pgfqpoint{2.423164in}{1.516005in}}{\pgfqpoint{2.434214in}{1.516005in}}%
\pgfpathclose%
\pgfusepath{stroke,fill}%
\end{pgfscope}%
\begin{pgfscope}%
\pgfpathrectangle{\pgfqpoint{0.511823in}{0.504323in}}{\pgfqpoint{3.218177in}{3.225677in}} %
\pgfusepath{clip}%
\pgfsetbuttcap%
\pgfsetroundjoin%
\definecolor{currentfill}{rgb}{0.000000,0.000000,0.545098}%
\pgfsetfillcolor{currentfill}%
\pgfsetfillopacity{0.400000}%
\pgfsetlinewidth{0.501875pt}%
\definecolor{currentstroke}{rgb}{0.000000,0.000000,0.545098}%
\pgfsetstrokecolor{currentstroke}%
\pgfsetstrokeopacity{0.400000}%
\pgfsetdash{}{0pt}%
\pgfpathmoveto{\pgfqpoint{2.286096in}{1.447802in}}%
\pgfpathcurveto{\pgfqpoint{2.297146in}{1.447802in}}{\pgfqpoint{2.307745in}{1.452192in}}{\pgfqpoint{2.315559in}{1.460006in}}%
\pgfpathcurveto{\pgfqpoint{2.323372in}{1.467819in}}{\pgfqpoint{2.327762in}{1.478418in}}{\pgfqpoint{2.327762in}{1.489468in}}%
\pgfpathcurveto{\pgfqpoint{2.327762in}{1.500519in}}{\pgfqpoint{2.323372in}{1.511118in}}{\pgfqpoint{2.315559in}{1.518931in}}%
\pgfpathcurveto{\pgfqpoint{2.307745in}{1.526745in}}{\pgfqpoint{2.297146in}{1.531135in}}{\pgfqpoint{2.286096in}{1.531135in}}%
\pgfpathcurveto{\pgfqpoint{2.275046in}{1.531135in}}{\pgfqpoint{2.264447in}{1.526745in}}{\pgfqpoint{2.256633in}{1.518931in}}%
\pgfpathcurveto{\pgfqpoint{2.248819in}{1.511118in}}{\pgfqpoint{2.244429in}{1.500519in}}{\pgfqpoint{2.244429in}{1.489468in}}%
\pgfpathcurveto{\pgfqpoint{2.244429in}{1.478418in}}{\pgfqpoint{2.248819in}{1.467819in}}{\pgfqpoint{2.256633in}{1.460006in}}%
\pgfpathcurveto{\pgfqpoint{2.264447in}{1.452192in}}{\pgfqpoint{2.275046in}{1.447802in}}{\pgfqpoint{2.286096in}{1.447802in}}%
\pgfpathclose%
\pgfusepath{stroke,fill}%
\end{pgfscope}%
\begin{pgfscope}%
\pgfpathrectangle{\pgfqpoint{0.511823in}{0.504323in}}{\pgfqpoint{3.218177in}{3.225677in}} %
\pgfusepath{clip}%
\pgfsetbuttcap%
\pgfsetroundjoin%
\definecolor{currentfill}{rgb}{0.000000,0.000000,0.545098}%
\pgfsetfillcolor{currentfill}%
\pgfsetfillopacity{0.400000}%
\pgfsetlinewidth{0.501875pt}%
\definecolor{currentstroke}{rgb}{0.000000,0.000000,0.545098}%
\pgfsetstrokecolor{currentstroke}%
\pgfsetstrokeopacity{0.400000}%
\pgfsetdash{}{0pt}%
\pgfpathmoveto{\pgfqpoint{2.436543in}{1.530446in}}%
\pgfpathcurveto{\pgfqpoint{2.447593in}{1.530446in}}{\pgfqpoint{2.458192in}{1.534836in}}{\pgfqpoint{2.466005in}{1.542650in}}%
\pgfpathcurveto{\pgfqpoint{2.473819in}{1.550463in}}{\pgfqpoint{2.478209in}{1.561062in}}{\pgfqpoint{2.478209in}{1.572112in}}%
\pgfpathcurveto{\pgfqpoint{2.478209in}{1.583162in}}{\pgfqpoint{2.473819in}{1.593762in}}{\pgfqpoint{2.466005in}{1.601575in}}%
\pgfpathcurveto{\pgfqpoint{2.458192in}{1.609389in}}{\pgfqpoint{2.447593in}{1.613779in}}{\pgfqpoint{2.436543in}{1.613779in}}%
\pgfpathcurveto{\pgfqpoint{2.425493in}{1.613779in}}{\pgfqpoint{2.414894in}{1.609389in}}{\pgfqpoint{2.407080in}{1.601575in}}%
\pgfpathcurveto{\pgfqpoint{2.399266in}{1.593762in}}{\pgfqpoint{2.394876in}{1.583162in}}{\pgfqpoint{2.394876in}{1.572112in}}%
\pgfpathcurveto{\pgfqpoint{2.394876in}{1.561062in}}{\pgfqpoint{2.399266in}{1.550463in}}{\pgfqpoint{2.407080in}{1.542650in}}%
\pgfpathcurveto{\pgfqpoint{2.414894in}{1.534836in}}{\pgfqpoint{2.425493in}{1.530446in}}{\pgfqpoint{2.436543in}{1.530446in}}%
\pgfpathclose%
\pgfusepath{stroke,fill}%
\end{pgfscope}%
\begin{pgfscope}%
\pgfpathrectangle{\pgfqpoint{0.511823in}{0.504323in}}{\pgfqpoint{3.218177in}{3.225677in}} %
\pgfusepath{clip}%
\pgfsetbuttcap%
\pgfsetroundjoin%
\definecolor{currentfill}{rgb}{0.000000,0.000000,0.545098}%
\pgfsetfillcolor{currentfill}%
\pgfsetfillopacity{0.400000}%
\pgfsetlinewidth{0.501875pt}%
\definecolor{currentstroke}{rgb}{0.000000,0.000000,0.545098}%
\pgfsetstrokecolor{currentstroke}%
\pgfsetstrokeopacity{0.400000}%
\pgfsetdash{}{0pt}%
\pgfpathmoveto{\pgfqpoint{2.354175in}{1.494883in}}%
\pgfpathcurveto{\pgfqpoint{2.365225in}{1.494883in}}{\pgfqpoint{2.375824in}{1.499274in}}{\pgfqpoint{2.383638in}{1.507087in}}%
\pgfpathcurveto{\pgfqpoint{2.391452in}{1.514901in}}{\pgfqpoint{2.395842in}{1.525500in}}{\pgfqpoint{2.395842in}{1.536550in}}%
\pgfpathcurveto{\pgfqpoint{2.395842in}{1.547600in}}{\pgfqpoint{2.391452in}{1.558199in}}{\pgfqpoint{2.383638in}{1.566013in}}%
\pgfpathcurveto{\pgfqpoint{2.375824in}{1.573826in}}{\pgfqpoint{2.365225in}{1.578217in}}{\pgfqpoint{2.354175in}{1.578217in}}%
\pgfpathcurveto{\pgfqpoint{2.343125in}{1.578217in}}{\pgfqpoint{2.332526in}{1.573826in}}{\pgfqpoint{2.324713in}{1.566013in}}%
\pgfpathcurveto{\pgfqpoint{2.316899in}{1.558199in}}{\pgfqpoint{2.312509in}{1.547600in}}{\pgfqpoint{2.312509in}{1.536550in}}%
\pgfpathcurveto{\pgfqpoint{2.312509in}{1.525500in}}{\pgfqpoint{2.316899in}{1.514901in}}{\pgfqpoint{2.324713in}{1.507087in}}%
\pgfpathcurveto{\pgfqpoint{2.332526in}{1.499274in}}{\pgfqpoint{2.343125in}{1.494883in}}{\pgfqpoint{2.354175in}{1.494883in}}%
\pgfpathclose%
\pgfusepath{stroke,fill}%
\end{pgfscope}%
\begin{pgfscope}%
\pgfpathrectangle{\pgfqpoint{0.511823in}{0.504323in}}{\pgfqpoint{3.218177in}{3.225677in}} %
\pgfusepath{clip}%
\pgfsetbuttcap%
\pgfsetroundjoin%
\definecolor{currentfill}{rgb}{0.000000,0.000000,0.545098}%
\pgfsetfillcolor{currentfill}%
\pgfsetfillopacity{0.400000}%
\pgfsetlinewidth{0.501875pt}%
\definecolor{currentstroke}{rgb}{0.000000,0.000000,0.545098}%
\pgfsetstrokecolor{currentstroke}%
\pgfsetstrokeopacity{0.400000}%
\pgfsetdash{}{0pt}%
\pgfpathmoveto{\pgfqpoint{2.489870in}{1.571358in}}%
\pgfpathcurveto{\pgfqpoint{2.500920in}{1.571358in}}{\pgfqpoint{2.511519in}{1.575748in}}{\pgfqpoint{2.519332in}{1.583562in}}%
\pgfpathcurveto{\pgfqpoint{2.527146in}{1.591376in}}{\pgfqpoint{2.531536in}{1.601975in}}{\pgfqpoint{2.531536in}{1.613025in}}%
\pgfpathcurveto{\pgfqpoint{2.531536in}{1.624075in}}{\pgfqpoint{2.527146in}{1.634674in}}{\pgfqpoint{2.519332in}{1.642488in}}%
\pgfpathcurveto{\pgfqpoint{2.511519in}{1.650301in}}{\pgfqpoint{2.500920in}{1.654691in}}{\pgfqpoint{2.489870in}{1.654691in}}%
\pgfpathcurveto{\pgfqpoint{2.478819in}{1.654691in}}{\pgfqpoint{2.468220in}{1.650301in}}{\pgfqpoint{2.460407in}{1.642488in}}%
\pgfpathcurveto{\pgfqpoint{2.452593in}{1.634674in}}{\pgfqpoint{2.448203in}{1.624075in}}{\pgfqpoint{2.448203in}{1.613025in}}%
\pgfpathcurveto{\pgfqpoint{2.448203in}{1.601975in}}{\pgfqpoint{2.452593in}{1.591376in}}{\pgfqpoint{2.460407in}{1.583562in}}%
\pgfpathcurveto{\pgfqpoint{2.468220in}{1.575748in}}{\pgfqpoint{2.478819in}{1.571358in}}{\pgfqpoint{2.489870in}{1.571358in}}%
\pgfpathclose%
\pgfusepath{stroke,fill}%
\end{pgfscope}%
\begin{pgfscope}%
\pgfpathrectangle{\pgfqpoint{0.511823in}{0.504323in}}{\pgfqpoint{3.218177in}{3.225677in}} %
\pgfusepath{clip}%
\pgfsetbuttcap%
\pgfsetroundjoin%
\definecolor{currentfill}{rgb}{0.000000,0.000000,0.545098}%
\pgfsetfillcolor{currentfill}%
\pgfsetfillopacity{0.400000}%
\pgfsetlinewidth{0.501875pt}%
\definecolor{currentstroke}{rgb}{0.000000,0.000000,0.545098}%
\pgfsetstrokecolor{currentstroke}%
\pgfsetstrokeopacity{0.400000}%
\pgfsetdash{}{0pt}%
\pgfpathmoveto{\pgfqpoint{2.447571in}{1.556268in}}%
\pgfpathcurveto{\pgfqpoint{2.458621in}{1.556268in}}{\pgfqpoint{2.469220in}{1.560658in}}{\pgfqpoint{2.477034in}{1.568472in}}%
\pgfpathcurveto{\pgfqpoint{2.484848in}{1.576286in}}{\pgfqpoint{2.489238in}{1.586885in}}{\pgfqpoint{2.489238in}{1.597935in}}%
\pgfpathcurveto{\pgfqpoint{2.489238in}{1.608985in}}{\pgfqpoint{2.484848in}{1.619584in}}{\pgfqpoint{2.477034in}{1.627398in}}%
\pgfpathcurveto{\pgfqpoint{2.469220in}{1.635211in}}{\pgfqpoint{2.458621in}{1.639601in}}{\pgfqpoint{2.447571in}{1.639601in}}%
\pgfpathcurveto{\pgfqpoint{2.436521in}{1.639601in}}{\pgfqpoint{2.425922in}{1.635211in}}{\pgfqpoint{2.418108in}{1.627398in}}%
\pgfpathcurveto{\pgfqpoint{2.410295in}{1.619584in}}{\pgfqpoint{2.405904in}{1.608985in}}{\pgfqpoint{2.405904in}{1.597935in}}%
\pgfpathcurveto{\pgfqpoint{2.405904in}{1.586885in}}{\pgfqpoint{2.410295in}{1.576286in}}{\pgfqpoint{2.418108in}{1.568472in}}%
\pgfpathcurveto{\pgfqpoint{2.425922in}{1.560658in}}{\pgfqpoint{2.436521in}{1.556268in}}{\pgfqpoint{2.447571in}{1.556268in}}%
\pgfpathclose%
\pgfusepath{stroke,fill}%
\end{pgfscope}%
\begin{pgfscope}%
\pgfpathrectangle{\pgfqpoint{0.511823in}{0.504323in}}{\pgfqpoint{3.218177in}{3.225677in}} %
\pgfusepath{clip}%
\pgfsetbuttcap%
\pgfsetroundjoin%
\definecolor{currentfill}{rgb}{0.000000,0.000000,0.545098}%
\pgfsetfillcolor{currentfill}%
\pgfsetfillopacity{0.400000}%
\pgfsetlinewidth{0.501875pt}%
\definecolor{currentstroke}{rgb}{0.000000,0.000000,0.545098}%
\pgfsetstrokecolor{currentstroke}%
\pgfsetstrokeopacity{0.400000}%
\pgfsetdash{}{0pt}%
\pgfpathmoveto{\pgfqpoint{2.460176in}{1.569660in}}%
\pgfpathcurveto{\pgfqpoint{2.471226in}{1.569660in}}{\pgfqpoint{2.481825in}{1.574050in}}{\pgfqpoint{2.489639in}{1.581864in}}%
\pgfpathcurveto{\pgfqpoint{2.497452in}{1.589678in}}{\pgfqpoint{2.501842in}{1.600277in}}{\pgfqpoint{2.501842in}{1.611327in}}%
\pgfpathcurveto{\pgfqpoint{2.501842in}{1.622377in}}{\pgfqpoint{2.497452in}{1.632976in}}{\pgfqpoint{2.489639in}{1.640790in}}%
\pgfpathcurveto{\pgfqpoint{2.481825in}{1.648603in}}{\pgfqpoint{2.471226in}{1.652993in}}{\pgfqpoint{2.460176in}{1.652993in}}%
\pgfpathcurveto{\pgfqpoint{2.449126in}{1.652993in}}{\pgfqpoint{2.438527in}{1.648603in}}{\pgfqpoint{2.430713in}{1.640790in}}%
\pgfpathcurveto{\pgfqpoint{2.422899in}{1.632976in}}{\pgfqpoint{2.418509in}{1.622377in}}{\pgfqpoint{2.418509in}{1.611327in}}%
\pgfpathcurveto{\pgfqpoint{2.418509in}{1.600277in}}{\pgfqpoint{2.422899in}{1.589678in}}{\pgfqpoint{2.430713in}{1.581864in}}%
\pgfpathcurveto{\pgfqpoint{2.438527in}{1.574050in}}{\pgfqpoint{2.449126in}{1.569660in}}{\pgfqpoint{2.460176in}{1.569660in}}%
\pgfpathclose%
\pgfusepath{stroke,fill}%
\end{pgfscope}%
\begin{pgfscope}%
\pgfpathrectangle{\pgfqpoint{0.511823in}{0.504323in}}{\pgfqpoint{3.218177in}{3.225677in}} %
\pgfusepath{clip}%
\pgfsetbuttcap%
\pgfsetroundjoin%
\definecolor{currentfill}{rgb}{0.000000,0.000000,0.545098}%
\pgfsetfillcolor{currentfill}%
\pgfsetfillopacity{0.400000}%
\pgfsetlinewidth{0.501875pt}%
\definecolor{currentstroke}{rgb}{0.000000,0.000000,0.545098}%
\pgfsetstrokecolor{currentstroke}%
\pgfsetstrokeopacity{0.400000}%
\pgfsetdash{}{0pt}%
\pgfpathmoveto{\pgfqpoint{2.351621in}{1.519186in}}%
\pgfpathcurveto{\pgfqpoint{2.362671in}{1.519186in}}{\pgfqpoint{2.373270in}{1.523577in}}{\pgfqpoint{2.381084in}{1.531390in}}%
\pgfpathcurveto{\pgfqpoint{2.388897in}{1.539204in}}{\pgfqpoint{2.393288in}{1.549803in}}{\pgfqpoint{2.393288in}{1.560853in}}%
\pgfpathcurveto{\pgfqpoint{2.393288in}{1.571903in}}{\pgfqpoint{2.388897in}{1.582502in}}{\pgfqpoint{2.381084in}{1.590316in}}%
\pgfpathcurveto{\pgfqpoint{2.373270in}{1.598129in}}{\pgfqpoint{2.362671in}{1.602520in}}{\pgfqpoint{2.351621in}{1.602520in}}%
\pgfpathcurveto{\pgfqpoint{2.340571in}{1.602520in}}{\pgfqpoint{2.329972in}{1.598129in}}{\pgfqpoint{2.322158in}{1.590316in}}%
\pgfpathcurveto{\pgfqpoint{2.314345in}{1.582502in}}{\pgfqpoint{2.309954in}{1.571903in}}{\pgfqpoint{2.309954in}{1.560853in}}%
\pgfpathcurveto{\pgfqpoint{2.309954in}{1.549803in}}{\pgfqpoint{2.314345in}{1.539204in}}{\pgfqpoint{2.322158in}{1.531390in}}%
\pgfpathcurveto{\pgfqpoint{2.329972in}{1.523577in}}{\pgfqpoint{2.340571in}{1.519186in}}{\pgfqpoint{2.351621in}{1.519186in}}%
\pgfpathclose%
\pgfusepath{stroke,fill}%
\end{pgfscope}%
\begin{pgfscope}%
\pgfpathrectangle{\pgfqpoint{0.511823in}{0.504323in}}{\pgfqpoint{3.218177in}{3.225677in}} %
\pgfusepath{clip}%
\pgfsetbuttcap%
\pgfsetroundjoin%
\definecolor{currentfill}{rgb}{0.000000,0.000000,0.545098}%
\pgfsetfillcolor{currentfill}%
\pgfsetfillopacity{0.400000}%
\pgfsetlinewidth{0.501875pt}%
\definecolor{currentstroke}{rgb}{0.000000,0.000000,0.545098}%
\pgfsetstrokecolor{currentstroke}%
\pgfsetstrokeopacity{0.400000}%
\pgfsetdash{}{0pt}%
\pgfpathmoveto{\pgfqpoint{2.461585in}{1.584148in}}%
\pgfpathcurveto{\pgfqpoint{2.472635in}{1.584148in}}{\pgfqpoint{2.483234in}{1.588538in}}{\pgfqpoint{2.491048in}{1.596352in}}%
\pgfpathcurveto{\pgfqpoint{2.498862in}{1.604165in}}{\pgfqpoint{2.503252in}{1.614764in}}{\pgfqpoint{2.503252in}{1.625815in}}%
\pgfpathcurveto{\pgfqpoint{2.503252in}{1.636865in}}{\pgfqpoint{2.498862in}{1.647464in}}{\pgfqpoint{2.491048in}{1.655277in}}%
\pgfpathcurveto{\pgfqpoint{2.483234in}{1.663091in}}{\pgfqpoint{2.472635in}{1.667481in}}{\pgfqpoint{2.461585in}{1.667481in}}%
\pgfpathcurveto{\pgfqpoint{2.450535in}{1.667481in}}{\pgfqpoint{2.439936in}{1.663091in}}{\pgfqpoint{2.432122in}{1.655277in}}%
\pgfpathcurveto{\pgfqpoint{2.424309in}{1.647464in}}{\pgfqpoint{2.419919in}{1.636865in}}{\pgfqpoint{2.419919in}{1.625815in}}%
\pgfpathcurveto{\pgfqpoint{2.419919in}{1.614764in}}{\pgfqpoint{2.424309in}{1.604165in}}{\pgfqpoint{2.432122in}{1.596352in}}%
\pgfpathcurveto{\pgfqpoint{2.439936in}{1.588538in}}{\pgfqpoint{2.450535in}{1.584148in}}{\pgfqpoint{2.461585in}{1.584148in}}%
\pgfpathclose%
\pgfusepath{stroke,fill}%
\end{pgfscope}%
\begin{pgfscope}%
\pgfpathrectangle{\pgfqpoint{0.511823in}{0.504323in}}{\pgfqpoint{3.218177in}{3.225677in}} %
\pgfusepath{clip}%
\pgfsetbuttcap%
\pgfsetroundjoin%
\definecolor{currentfill}{rgb}{0.000000,0.000000,0.545098}%
\pgfsetfillcolor{currentfill}%
\pgfsetfillopacity{0.400000}%
\pgfsetlinewidth{0.501875pt}%
\definecolor{currentstroke}{rgb}{0.000000,0.000000,0.545098}%
\pgfsetstrokecolor{currentstroke}%
\pgfsetstrokeopacity{0.400000}%
\pgfsetdash{}{0pt}%
\pgfpathmoveto{\pgfqpoint{2.352167in}{1.532422in}}%
\pgfpathcurveto{\pgfqpoint{2.363217in}{1.532422in}}{\pgfqpoint{2.373816in}{1.536813in}}{\pgfqpoint{2.381630in}{1.544626in}}%
\pgfpathcurveto{\pgfqpoint{2.389443in}{1.552440in}}{\pgfqpoint{2.393833in}{1.563039in}}{\pgfqpoint{2.393833in}{1.574089in}}%
\pgfpathcurveto{\pgfqpoint{2.393833in}{1.585139in}}{\pgfqpoint{2.389443in}{1.595738in}}{\pgfqpoint{2.381630in}{1.603552in}}%
\pgfpathcurveto{\pgfqpoint{2.373816in}{1.611365in}}{\pgfqpoint{2.363217in}{1.615756in}}{\pgfqpoint{2.352167in}{1.615756in}}%
\pgfpathcurveto{\pgfqpoint{2.341117in}{1.615756in}}{\pgfqpoint{2.330518in}{1.611365in}}{\pgfqpoint{2.322704in}{1.603552in}}%
\pgfpathcurveto{\pgfqpoint{2.314890in}{1.595738in}}{\pgfqpoint{2.310500in}{1.585139in}}{\pgfqpoint{2.310500in}{1.574089in}}%
\pgfpathcurveto{\pgfqpoint{2.310500in}{1.563039in}}{\pgfqpoint{2.314890in}{1.552440in}}{\pgfqpoint{2.322704in}{1.544626in}}%
\pgfpathcurveto{\pgfqpoint{2.330518in}{1.536813in}}{\pgfqpoint{2.341117in}{1.532422in}}{\pgfqpoint{2.352167in}{1.532422in}}%
\pgfpathclose%
\pgfusepath{stroke,fill}%
\end{pgfscope}%
\begin{pgfscope}%
\pgfpathrectangle{\pgfqpoint{0.511823in}{0.504323in}}{\pgfqpoint{3.218177in}{3.225677in}} %
\pgfusepath{clip}%
\pgfsetbuttcap%
\pgfsetroundjoin%
\definecolor{currentfill}{rgb}{0.000000,0.000000,0.545098}%
\pgfsetfillcolor{currentfill}%
\pgfsetfillopacity{0.400000}%
\pgfsetlinewidth{0.501875pt}%
\definecolor{currentstroke}{rgb}{0.000000,0.000000,0.545098}%
\pgfsetstrokecolor{currentstroke}%
\pgfsetstrokeopacity{0.400000}%
\pgfsetdash{}{0pt}%
\pgfpathmoveto{\pgfqpoint{2.467046in}{1.600944in}}%
\pgfpathcurveto{\pgfqpoint{2.478096in}{1.600944in}}{\pgfqpoint{2.488695in}{1.605334in}}{\pgfqpoint{2.496509in}{1.613148in}}%
\pgfpathcurveto{\pgfqpoint{2.504322in}{1.620961in}}{\pgfqpoint{2.508713in}{1.631560in}}{\pgfqpoint{2.508713in}{1.642611in}}%
\pgfpathcurveto{\pgfqpoint{2.508713in}{1.653661in}}{\pgfqpoint{2.504322in}{1.664260in}}{\pgfqpoint{2.496509in}{1.672073in}}%
\pgfpathcurveto{\pgfqpoint{2.488695in}{1.679887in}}{\pgfqpoint{2.478096in}{1.684277in}}{\pgfqpoint{2.467046in}{1.684277in}}%
\pgfpathcurveto{\pgfqpoint{2.455996in}{1.684277in}}{\pgfqpoint{2.445397in}{1.679887in}}{\pgfqpoint{2.437583in}{1.672073in}}%
\pgfpathcurveto{\pgfqpoint{2.429770in}{1.664260in}}{\pgfqpoint{2.425379in}{1.653661in}}{\pgfqpoint{2.425379in}{1.642611in}}%
\pgfpathcurveto{\pgfqpoint{2.425379in}{1.631560in}}{\pgfqpoint{2.429770in}{1.620961in}}{\pgfqpoint{2.437583in}{1.613148in}}%
\pgfpathcurveto{\pgfqpoint{2.445397in}{1.605334in}}{\pgfqpoint{2.455996in}{1.600944in}}{\pgfqpoint{2.467046in}{1.600944in}}%
\pgfpathclose%
\pgfusepath{stroke,fill}%
\end{pgfscope}%
\begin{pgfscope}%
\pgfpathrectangle{\pgfqpoint{0.511823in}{0.504323in}}{\pgfqpoint{3.218177in}{3.225677in}} %
\pgfusepath{clip}%
\pgfsetbuttcap%
\pgfsetroundjoin%
\definecolor{currentfill}{rgb}{0.000000,0.000000,0.545098}%
\pgfsetfillcolor{currentfill}%
\pgfsetfillopacity{0.400000}%
\pgfsetlinewidth{0.501875pt}%
\definecolor{currentstroke}{rgb}{0.000000,0.000000,0.545098}%
\pgfsetstrokecolor{currentstroke}%
\pgfsetstrokeopacity{0.400000}%
\pgfsetdash{}{0pt}%
\pgfpathmoveto{\pgfqpoint{2.370061in}{1.555194in}}%
\pgfpathcurveto{\pgfqpoint{2.381111in}{1.555194in}}{\pgfqpoint{2.391710in}{1.559584in}}{\pgfqpoint{2.399524in}{1.567398in}}%
\pgfpathcurveto{\pgfqpoint{2.407338in}{1.575212in}}{\pgfqpoint{2.411728in}{1.585811in}}{\pgfqpoint{2.411728in}{1.596861in}}%
\pgfpathcurveto{\pgfqpoint{2.411728in}{1.607911in}}{\pgfqpoint{2.407338in}{1.618510in}}{\pgfqpoint{2.399524in}{1.626324in}}%
\pgfpathcurveto{\pgfqpoint{2.391710in}{1.634137in}}{\pgfqpoint{2.381111in}{1.638527in}}{\pgfqpoint{2.370061in}{1.638527in}}%
\pgfpathcurveto{\pgfqpoint{2.359011in}{1.638527in}}{\pgfqpoint{2.348412in}{1.634137in}}{\pgfqpoint{2.340599in}{1.626324in}}%
\pgfpathcurveto{\pgfqpoint{2.332785in}{1.618510in}}{\pgfqpoint{2.328395in}{1.607911in}}{\pgfqpoint{2.328395in}{1.596861in}}%
\pgfpathcurveto{\pgfqpoint{2.328395in}{1.585811in}}{\pgfqpoint{2.332785in}{1.575212in}}{\pgfqpoint{2.340599in}{1.567398in}}%
\pgfpathcurveto{\pgfqpoint{2.348412in}{1.559584in}}{\pgfqpoint{2.359011in}{1.555194in}}{\pgfqpoint{2.370061in}{1.555194in}}%
\pgfpathclose%
\pgfusepath{stroke,fill}%
\end{pgfscope}%
\begin{pgfscope}%
\pgfpathrectangle{\pgfqpoint{0.511823in}{0.504323in}}{\pgfqpoint{3.218177in}{3.225677in}} %
\pgfusepath{clip}%
\pgfsetbuttcap%
\pgfsetroundjoin%
\definecolor{currentfill}{rgb}{0.000000,0.000000,0.545098}%
\pgfsetfillcolor{currentfill}%
\pgfsetfillopacity{0.400000}%
\pgfsetlinewidth{0.501875pt}%
\definecolor{currentstroke}{rgb}{0.000000,0.000000,0.545098}%
\pgfsetstrokecolor{currentstroke}%
\pgfsetstrokeopacity{0.400000}%
\pgfsetdash{}{0pt}%
\pgfpathmoveto{\pgfqpoint{2.273504in}{1.508940in}}%
\pgfpathcurveto{\pgfqpoint{2.284554in}{1.508940in}}{\pgfqpoint{2.295153in}{1.513331in}}{\pgfqpoint{2.302966in}{1.521144in}}%
\pgfpathcurveto{\pgfqpoint{2.310780in}{1.528958in}}{\pgfqpoint{2.315170in}{1.539557in}}{\pgfqpoint{2.315170in}{1.550607in}}%
\pgfpathcurveto{\pgfqpoint{2.315170in}{1.561657in}}{\pgfqpoint{2.310780in}{1.572256in}}{\pgfqpoint{2.302966in}{1.580070in}}%
\pgfpathcurveto{\pgfqpoint{2.295153in}{1.587883in}}{\pgfqpoint{2.284554in}{1.592274in}}{\pgfqpoint{2.273504in}{1.592274in}}%
\pgfpathcurveto{\pgfqpoint{2.262453in}{1.592274in}}{\pgfqpoint{2.251854in}{1.587883in}}{\pgfqpoint{2.244041in}{1.580070in}}%
\pgfpathcurveto{\pgfqpoint{2.236227in}{1.572256in}}{\pgfqpoint{2.231837in}{1.561657in}}{\pgfqpoint{2.231837in}{1.550607in}}%
\pgfpathcurveto{\pgfqpoint{2.231837in}{1.539557in}}{\pgfqpoint{2.236227in}{1.528958in}}{\pgfqpoint{2.244041in}{1.521144in}}%
\pgfpathcurveto{\pgfqpoint{2.251854in}{1.513331in}}{\pgfqpoint{2.262453in}{1.508940in}}{\pgfqpoint{2.273504in}{1.508940in}}%
\pgfpathclose%
\pgfusepath{stroke,fill}%
\end{pgfscope}%
\begin{pgfscope}%
\pgfpathrectangle{\pgfqpoint{0.511823in}{0.504323in}}{\pgfqpoint{3.218177in}{3.225677in}} %
\pgfusepath{clip}%
\pgfsetbuttcap%
\pgfsetroundjoin%
\definecolor{currentfill}{rgb}{0.000000,0.000000,0.545098}%
\pgfsetfillcolor{currentfill}%
\pgfsetfillopacity{0.400000}%
\pgfsetlinewidth{0.501875pt}%
\definecolor{currentstroke}{rgb}{0.000000,0.000000,0.545098}%
\pgfsetstrokecolor{currentstroke}%
\pgfsetstrokeopacity{0.400000}%
\pgfsetdash{}{0pt}%
\pgfpathmoveto{\pgfqpoint{2.455073in}{1.615368in}}%
\pgfpathcurveto{\pgfqpoint{2.466123in}{1.615368in}}{\pgfqpoint{2.476722in}{1.619758in}}{\pgfqpoint{2.484536in}{1.627571in}}%
\pgfpathcurveto{\pgfqpoint{2.492350in}{1.635385in}}{\pgfqpoint{2.496740in}{1.645984in}}{\pgfqpoint{2.496740in}{1.657034in}}%
\pgfpathcurveto{\pgfqpoint{2.496740in}{1.668084in}}{\pgfqpoint{2.492350in}{1.678683in}}{\pgfqpoint{2.484536in}{1.686497in}}%
\pgfpathcurveto{\pgfqpoint{2.476722in}{1.694311in}}{\pgfqpoint{2.466123in}{1.698701in}}{\pgfqpoint{2.455073in}{1.698701in}}%
\pgfpathcurveto{\pgfqpoint{2.444023in}{1.698701in}}{\pgfqpoint{2.433424in}{1.694311in}}{\pgfqpoint{2.425610in}{1.686497in}}%
\pgfpathcurveto{\pgfqpoint{2.417797in}{1.678683in}}{\pgfqpoint{2.413406in}{1.668084in}}{\pgfqpoint{2.413406in}{1.657034in}}%
\pgfpathcurveto{\pgfqpoint{2.413406in}{1.645984in}}{\pgfqpoint{2.417797in}{1.635385in}}{\pgfqpoint{2.425610in}{1.627571in}}%
\pgfpathcurveto{\pgfqpoint{2.433424in}{1.619758in}}{\pgfqpoint{2.444023in}{1.615368in}}{\pgfqpoint{2.455073in}{1.615368in}}%
\pgfpathclose%
\pgfusepath{stroke,fill}%
\end{pgfscope}%
\begin{pgfscope}%
\pgfpathrectangle{\pgfqpoint{0.511823in}{0.504323in}}{\pgfqpoint{3.218177in}{3.225677in}} %
\pgfusepath{clip}%
\pgfsetbuttcap%
\pgfsetroundjoin%
\definecolor{currentfill}{rgb}{0.000000,0.000000,0.545098}%
\pgfsetfillcolor{currentfill}%
\pgfsetfillopacity{0.400000}%
\pgfsetlinewidth{0.501875pt}%
\definecolor{currentstroke}{rgb}{0.000000,0.000000,0.545098}%
\pgfsetstrokecolor{currentstroke}%
\pgfsetstrokeopacity{0.400000}%
\pgfsetdash{}{0pt}%
\pgfpathmoveto{\pgfqpoint{2.314554in}{1.544312in}}%
\pgfpathcurveto{\pgfqpoint{2.325605in}{1.544312in}}{\pgfqpoint{2.336204in}{1.548702in}}{\pgfqpoint{2.344017in}{1.556516in}}%
\pgfpathcurveto{\pgfqpoint{2.351831in}{1.564330in}}{\pgfqpoint{2.356221in}{1.574929in}}{\pgfqpoint{2.356221in}{1.585979in}}%
\pgfpathcurveto{\pgfqpoint{2.356221in}{1.597029in}}{\pgfqpoint{2.351831in}{1.607628in}}{\pgfqpoint{2.344017in}{1.615441in}}%
\pgfpathcurveto{\pgfqpoint{2.336204in}{1.623255in}}{\pgfqpoint{2.325605in}{1.627645in}}{\pgfqpoint{2.314554in}{1.627645in}}%
\pgfpathcurveto{\pgfqpoint{2.303504in}{1.627645in}}{\pgfqpoint{2.292905in}{1.623255in}}{\pgfqpoint{2.285092in}{1.615441in}}%
\pgfpathcurveto{\pgfqpoint{2.277278in}{1.607628in}}{\pgfqpoint{2.272888in}{1.597029in}}{\pgfqpoint{2.272888in}{1.585979in}}%
\pgfpathcurveto{\pgfqpoint{2.272888in}{1.574929in}}{\pgfqpoint{2.277278in}{1.564330in}}{\pgfqpoint{2.285092in}{1.556516in}}%
\pgfpathcurveto{\pgfqpoint{2.292905in}{1.548702in}}{\pgfqpoint{2.303504in}{1.544312in}}{\pgfqpoint{2.314554in}{1.544312in}}%
\pgfpathclose%
\pgfusepath{stroke,fill}%
\end{pgfscope}%
\begin{pgfscope}%
\pgfpathrectangle{\pgfqpoint{0.511823in}{0.504323in}}{\pgfqpoint{3.218177in}{3.225677in}} %
\pgfusepath{clip}%
\pgfsetbuttcap%
\pgfsetroundjoin%
\definecolor{currentfill}{rgb}{0.000000,0.000000,0.545098}%
\pgfsetfillcolor{currentfill}%
\pgfsetfillopacity{0.400000}%
\pgfsetlinewidth{0.501875pt}%
\definecolor{currentstroke}{rgb}{0.000000,0.000000,0.545098}%
\pgfsetstrokecolor{currentstroke}%
\pgfsetstrokeopacity{0.400000}%
\pgfsetdash{}{0pt}%
\pgfpathmoveto{\pgfqpoint{2.555913in}{1.685848in}}%
\pgfpathcurveto{\pgfqpoint{2.566963in}{1.685848in}}{\pgfqpoint{2.577562in}{1.690238in}}{\pgfqpoint{2.585376in}{1.698052in}}%
\pgfpathcurveto{\pgfqpoint{2.593189in}{1.705866in}}{\pgfqpoint{2.597579in}{1.716465in}}{\pgfqpoint{2.597579in}{1.727515in}}%
\pgfpathcurveto{\pgfqpoint{2.597579in}{1.738565in}}{\pgfqpoint{2.593189in}{1.749164in}}{\pgfqpoint{2.585376in}{1.756978in}}%
\pgfpathcurveto{\pgfqpoint{2.577562in}{1.764791in}}{\pgfqpoint{2.566963in}{1.769181in}}{\pgfqpoint{2.555913in}{1.769181in}}%
\pgfpathcurveto{\pgfqpoint{2.544863in}{1.769181in}}{\pgfqpoint{2.534264in}{1.764791in}}{\pgfqpoint{2.526450in}{1.756978in}}%
\pgfpathcurveto{\pgfqpoint{2.518636in}{1.749164in}}{\pgfqpoint{2.514246in}{1.738565in}}{\pgfqpoint{2.514246in}{1.727515in}}%
\pgfpathcurveto{\pgfqpoint{2.514246in}{1.716465in}}{\pgfqpoint{2.518636in}{1.705866in}}{\pgfqpoint{2.526450in}{1.698052in}}%
\pgfpathcurveto{\pgfqpoint{2.534264in}{1.690238in}}{\pgfqpoint{2.544863in}{1.685848in}}{\pgfqpoint{2.555913in}{1.685848in}}%
\pgfpathclose%
\pgfusepath{stroke,fill}%
\end{pgfscope}%
\begin{pgfscope}%
\pgfpathrectangle{\pgfqpoint{0.511823in}{0.504323in}}{\pgfqpoint{3.218177in}{3.225677in}} %
\pgfusepath{clip}%
\pgfsetbuttcap%
\pgfsetroundjoin%
\definecolor{currentfill}{rgb}{0.000000,0.000000,0.545098}%
\pgfsetfillcolor{currentfill}%
\pgfsetfillopacity{0.400000}%
\pgfsetlinewidth{0.501875pt}%
\definecolor{currentstroke}{rgb}{0.000000,0.000000,0.545098}%
\pgfsetstrokecolor{currentstroke}%
\pgfsetstrokeopacity{0.400000}%
\pgfsetdash{}{0pt}%
\pgfpathmoveto{\pgfqpoint{2.438787in}{1.627311in}}%
\pgfpathcurveto{\pgfqpoint{2.449837in}{1.627311in}}{\pgfqpoint{2.460436in}{1.631701in}}{\pgfqpoint{2.468250in}{1.639515in}}%
\pgfpathcurveto{\pgfqpoint{2.476064in}{1.647329in}}{\pgfqpoint{2.480454in}{1.657928in}}{\pgfqpoint{2.480454in}{1.668978in}}%
\pgfpathcurveto{\pgfqpoint{2.480454in}{1.680028in}}{\pgfqpoint{2.476064in}{1.690627in}}{\pgfqpoint{2.468250in}{1.698441in}}%
\pgfpathcurveto{\pgfqpoint{2.460436in}{1.706254in}}{\pgfqpoint{2.449837in}{1.710645in}}{\pgfqpoint{2.438787in}{1.710645in}}%
\pgfpathcurveto{\pgfqpoint{2.427737in}{1.710645in}}{\pgfqpoint{2.417138in}{1.706254in}}{\pgfqpoint{2.409324in}{1.698441in}}%
\pgfpathcurveto{\pgfqpoint{2.401511in}{1.690627in}}{\pgfqpoint{2.397120in}{1.680028in}}{\pgfqpoint{2.397120in}{1.668978in}}%
\pgfpathcurveto{\pgfqpoint{2.397120in}{1.657928in}}{\pgfqpoint{2.401511in}{1.647329in}}{\pgfqpoint{2.409324in}{1.639515in}}%
\pgfpathcurveto{\pgfqpoint{2.417138in}{1.631701in}}{\pgfqpoint{2.427737in}{1.627311in}}{\pgfqpoint{2.438787in}{1.627311in}}%
\pgfpathclose%
\pgfusepath{stroke,fill}%
\end{pgfscope}%
\begin{pgfscope}%
\pgfpathrectangle{\pgfqpoint{0.511823in}{0.504323in}}{\pgfqpoint{3.218177in}{3.225677in}} %
\pgfusepath{clip}%
\pgfsetbuttcap%
\pgfsetroundjoin%
\definecolor{currentfill}{rgb}{0.000000,0.000000,0.545098}%
\pgfsetfillcolor{currentfill}%
\pgfsetfillopacity{0.400000}%
\pgfsetlinewidth{0.501875pt}%
\definecolor{currentstroke}{rgb}{0.000000,0.000000,0.545098}%
\pgfsetstrokecolor{currentstroke}%
\pgfsetstrokeopacity{0.400000}%
\pgfsetdash{}{0pt}%
\pgfpathmoveto{\pgfqpoint{2.438240in}{1.634030in}}%
\pgfpathcurveto{\pgfqpoint{2.449291in}{1.634030in}}{\pgfqpoint{2.459890in}{1.638420in}}{\pgfqpoint{2.467703in}{1.646234in}}%
\pgfpathcurveto{\pgfqpoint{2.475517in}{1.654047in}}{\pgfqpoint{2.479907in}{1.664646in}}{\pgfqpoint{2.479907in}{1.675696in}}%
\pgfpathcurveto{\pgfqpoint{2.479907in}{1.686747in}}{\pgfqpoint{2.475517in}{1.697346in}}{\pgfqpoint{2.467703in}{1.705159in}}%
\pgfpathcurveto{\pgfqpoint{2.459890in}{1.712973in}}{\pgfqpoint{2.449291in}{1.717363in}}{\pgfqpoint{2.438240in}{1.717363in}}%
\pgfpathcurveto{\pgfqpoint{2.427190in}{1.717363in}}{\pgfqpoint{2.416591in}{1.712973in}}{\pgfqpoint{2.408778in}{1.705159in}}%
\pgfpathcurveto{\pgfqpoint{2.400964in}{1.697346in}}{\pgfqpoint{2.396574in}{1.686747in}}{\pgfqpoint{2.396574in}{1.675696in}}%
\pgfpathcurveto{\pgfqpoint{2.396574in}{1.664646in}}{\pgfqpoint{2.400964in}{1.654047in}}{\pgfqpoint{2.408778in}{1.646234in}}%
\pgfpathcurveto{\pgfqpoint{2.416591in}{1.638420in}}{\pgfqpoint{2.427190in}{1.634030in}}{\pgfqpoint{2.438240in}{1.634030in}}%
\pgfpathclose%
\pgfusepath{stroke,fill}%
\end{pgfscope}%
\begin{pgfscope}%
\pgfpathrectangle{\pgfqpoint{0.511823in}{0.504323in}}{\pgfqpoint{3.218177in}{3.225677in}} %
\pgfusepath{clip}%
\pgfsetbuttcap%
\pgfsetroundjoin%
\definecolor{currentfill}{rgb}{0.000000,0.000000,0.545098}%
\pgfsetfillcolor{currentfill}%
\pgfsetfillopacity{0.400000}%
\pgfsetlinewidth{0.501875pt}%
\definecolor{currentstroke}{rgb}{0.000000,0.000000,0.545098}%
\pgfsetstrokecolor{currentstroke}%
\pgfsetstrokeopacity{0.400000}%
\pgfsetdash{}{0pt}%
\pgfpathmoveto{\pgfqpoint{2.393456in}{1.615485in}}%
\pgfpathcurveto{\pgfqpoint{2.404506in}{1.615485in}}{\pgfqpoint{2.415105in}{1.619875in}}{\pgfqpoint{2.422918in}{1.627689in}}%
\pgfpathcurveto{\pgfqpoint{2.430732in}{1.635502in}}{\pgfqpoint{2.435122in}{1.646101in}}{\pgfqpoint{2.435122in}{1.657151in}}%
\pgfpathcurveto{\pgfqpoint{2.435122in}{1.668201in}}{\pgfqpoint{2.430732in}{1.678800in}}{\pgfqpoint{2.422918in}{1.686614in}}%
\pgfpathcurveto{\pgfqpoint{2.415105in}{1.694428in}}{\pgfqpoint{2.404506in}{1.698818in}}{\pgfqpoint{2.393456in}{1.698818in}}%
\pgfpathcurveto{\pgfqpoint{2.382405in}{1.698818in}}{\pgfqpoint{2.371806in}{1.694428in}}{\pgfqpoint{2.363993in}{1.686614in}}%
\pgfpathcurveto{\pgfqpoint{2.356179in}{1.678800in}}{\pgfqpoint{2.351789in}{1.668201in}}{\pgfqpoint{2.351789in}{1.657151in}}%
\pgfpathcurveto{\pgfqpoint{2.351789in}{1.646101in}}{\pgfqpoint{2.356179in}{1.635502in}}{\pgfqpoint{2.363993in}{1.627689in}}%
\pgfpathcurveto{\pgfqpoint{2.371806in}{1.619875in}}{\pgfqpoint{2.382405in}{1.615485in}}{\pgfqpoint{2.393456in}{1.615485in}}%
\pgfpathclose%
\pgfusepath{stroke,fill}%
\end{pgfscope}%
\begin{pgfscope}%
\pgfpathrectangle{\pgfqpoint{0.511823in}{0.504323in}}{\pgfqpoint{3.218177in}{3.225677in}} %
\pgfusepath{clip}%
\pgfsetbuttcap%
\pgfsetroundjoin%
\definecolor{currentfill}{rgb}{0.000000,0.000000,0.545098}%
\pgfsetfillcolor{currentfill}%
\pgfsetfillopacity{0.400000}%
\pgfsetlinewidth{0.501875pt}%
\definecolor{currentstroke}{rgb}{0.000000,0.000000,0.545098}%
\pgfsetstrokecolor{currentstroke}%
\pgfsetstrokeopacity{0.400000}%
\pgfsetdash{}{0pt}%
\pgfpathmoveto{\pgfqpoint{2.372017in}{1.610044in}}%
\pgfpathcurveto{\pgfqpoint{2.383067in}{1.610044in}}{\pgfqpoint{2.393666in}{1.614434in}}{\pgfqpoint{2.401479in}{1.622247in}}%
\pgfpathcurveto{\pgfqpoint{2.409293in}{1.630061in}}{\pgfqpoint{2.413683in}{1.640660in}}{\pgfqpoint{2.413683in}{1.651710in}}%
\pgfpathcurveto{\pgfqpoint{2.413683in}{1.662760in}}{\pgfqpoint{2.409293in}{1.673359in}}{\pgfqpoint{2.401479in}{1.681173in}}%
\pgfpathcurveto{\pgfqpoint{2.393666in}{1.688987in}}{\pgfqpoint{2.383067in}{1.693377in}}{\pgfqpoint{2.372017in}{1.693377in}}%
\pgfpathcurveto{\pgfqpoint{2.360966in}{1.693377in}}{\pgfqpoint{2.350367in}{1.688987in}}{\pgfqpoint{2.342554in}{1.681173in}}%
\pgfpathcurveto{\pgfqpoint{2.334740in}{1.673359in}}{\pgfqpoint{2.330350in}{1.662760in}}{\pgfqpoint{2.330350in}{1.651710in}}%
\pgfpathcurveto{\pgfqpoint{2.330350in}{1.640660in}}{\pgfqpoint{2.334740in}{1.630061in}}{\pgfqpoint{2.342554in}{1.622247in}}%
\pgfpathcurveto{\pgfqpoint{2.350367in}{1.614434in}}{\pgfqpoint{2.360966in}{1.610044in}}{\pgfqpoint{2.372017in}{1.610044in}}%
\pgfpathclose%
\pgfusepath{stroke,fill}%
\end{pgfscope}%
\begin{pgfscope}%
\pgfpathrectangle{\pgfqpoint{0.511823in}{0.504323in}}{\pgfqpoint{3.218177in}{3.225677in}} %
\pgfusepath{clip}%
\pgfsetbuttcap%
\pgfsetroundjoin%
\definecolor{currentfill}{rgb}{0.000000,0.000000,0.545098}%
\pgfsetfillcolor{currentfill}%
\pgfsetfillopacity{0.400000}%
\pgfsetlinewidth{0.501875pt}%
\definecolor{currentstroke}{rgb}{0.000000,0.000000,0.545098}%
\pgfsetstrokecolor{currentstroke}%
\pgfsetstrokeopacity{0.400000}%
\pgfsetdash{}{0pt}%
\pgfpathmoveto{\pgfqpoint{2.106671in}{1.463064in}}%
\pgfpathcurveto{\pgfqpoint{2.117721in}{1.463064in}}{\pgfqpoint{2.128320in}{1.467454in}}{\pgfqpoint{2.136134in}{1.475268in}}%
\pgfpathcurveto{\pgfqpoint{2.143948in}{1.483081in}}{\pgfqpoint{2.148338in}{1.493680in}}{\pgfqpoint{2.148338in}{1.504731in}}%
\pgfpathcurveto{\pgfqpoint{2.148338in}{1.515781in}}{\pgfqpoint{2.143948in}{1.526380in}}{\pgfqpoint{2.136134in}{1.534193in}}%
\pgfpathcurveto{\pgfqpoint{2.128320in}{1.542007in}}{\pgfqpoint{2.117721in}{1.546397in}}{\pgfqpoint{2.106671in}{1.546397in}}%
\pgfpathcurveto{\pgfqpoint{2.095621in}{1.546397in}}{\pgfqpoint{2.085022in}{1.542007in}}{\pgfqpoint{2.077208in}{1.534193in}}%
\pgfpathcurveto{\pgfqpoint{2.069395in}{1.526380in}}{\pgfqpoint{2.065004in}{1.515781in}}{\pgfqpoint{2.065004in}{1.504731in}}%
\pgfpathcurveto{\pgfqpoint{2.065004in}{1.493680in}}{\pgfqpoint{2.069395in}{1.483081in}}{\pgfqpoint{2.077208in}{1.475268in}}%
\pgfpathcurveto{\pgfqpoint{2.085022in}{1.467454in}}{\pgfqpoint{2.095621in}{1.463064in}}{\pgfqpoint{2.106671in}{1.463064in}}%
\pgfpathclose%
\pgfusepath{stroke,fill}%
\end{pgfscope}%
\begin{pgfscope}%
\pgfpathrectangle{\pgfqpoint{0.511823in}{0.504323in}}{\pgfqpoint{3.218177in}{3.225677in}} %
\pgfusepath{clip}%
\pgfsetbuttcap%
\pgfsetroundjoin%
\definecolor{currentfill}{rgb}{0.000000,0.000000,0.545098}%
\pgfsetfillcolor{currentfill}%
\pgfsetfillopacity{0.400000}%
\pgfsetlinewidth{0.501875pt}%
\definecolor{currentstroke}{rgb}{0.000000,0.000000,0.545098}%
\pgfsetstrokecolor{currentstroke}%
\pgfsetstrokeopacity{0.400000}%
\pgfsetdash{}{0pt}%
\pgfpathmoveto{\pgfqpoint{2.509431in}{1.703969in}}%
\pgfpathcurveto{\pgfqpoint{2.520481in}{1.703969in}}{\pgfqpoint{2.531080in}{1.708359in}}{\pgfqpoint{2.538894in}{1.716173in}}%
\pgfpathcurveto{\pgfqpoint{2.546707in}{1.723986in}}{\pgfqpoint{2.551098in}{1.734585in}}{\pgfqpoint{2.551098in}{1.745635in}}%
\pgfpathcurveto{\pgfqpoint{2.551098in}{1.756686in}}{\pgfqpoint{2.546707in}{1.767285in}}{\pgfqpoint{2.538894in}{1.775098in}}%
\pgfpathcurveto{\pgfqpoint{2.531080in}{1.782912in}}{\pgfqpoint{2.520481in}{1.787302in}}{\pgfqpoint{2.509431in}{1.787302in}}%
\pgfpathcurveto{\pgfqpoint{2.498381in}{1.787302in}}{\pgfqpoint{2.487782in}{1.782912in}}{\pgfqpoint{2.479968in}{1.775098in}}%
\pgfpathcurveto{\pgfqpoint{2.472155in}{1.767285in}}{\pgfqpoint{2.467764in}{1.756686in}}{\pgfqpoint{2.467764in}{1.745635in}}%
\pgfpathcurveto{\pgfqpoint{2.467764in}{1.734585in}}{\pgfqpoint{2.472155in}{1.723986in}}{\pgfqpoint{2.479968in}{1.716173in}}%
\pgfpathcurveto{\pgfqpoint{2.487782in}{1.708359in}}{\pgfqpoint{2.498381in}{1.703969in}}{\pgfqpoint{2.509431in}{1.703969in}}%
\pgfpathclose%
\pgfusepath{stroke,fill}%
\end{pgfscope}%
\begin{pgfscope}%
\pgfpathrectangle{\pgfqpoint{0.511823in}{0.504323in}}{\pgfqpoint{3.218177in}{3.225677in}} %
\pgfusepath{clip}%
\pgfsetbuttcap%
\pgfsetroundjoin%
\definecolor{currentfill}{rgb}{0.000000,0.000000,0.545098}%
\pgfsetfillcolor{currentfill}%
\pgfsetfillopacity{0.400000}%
\pgfsetlinewidth{0.501875pt}%
\definecolor{currentstroke}{rgb}{0.000000,0.000000,0.545098}%
\pgfsetstrokecolor{currentstroke}%
\pgfsetstrokeopacity{0.400000}%
\pgfsetdash{}{0pt}%
\pgfpathmoveto{\pgfqpoint{2.277467in}{1.575062in}}%
\pgfpathcurveto{\pgfqpoint{2.288517in}{1.575062in}}{\pgfqpoint{2.299116in}{1.579453in}}{\pgfqpoint{2.306929in}{1.587266in}}%
\pgfpathcurveto{\pgfqpoint{2.314743in}{1.595080in}}{\pgfqpoint{2.319133in}{1.605679in}}{\pgfqpoint{2.319133in}{1.616729in}}%
\pgfpathcurveto{\pgfqpoint{2.319133in}{1.627779in}}{\pgfqpoint{2.314743in}{1.638378in}}{\pgfqpoint{2.306929in}{1.646192in}}%
\pgfpathcurveto{\pgfqpoint{2.299116in}{1.654005in}}{\pgfqpoint{2.288517in}{1.658396in}}{\pgfqpoint{2.277467in}{1.658396in}}%
\pgfpathcurveto{\pgfqpoint{2.266417in}{1.658396in}}{\pgfqpoint{2.255817in}{1.654005in}}{\pgfqpoint{2.248004in}{1.646192in}}%
\pgfpathcurveto{\pgfqpoint{2.240190in}{1.638378in}}{\pgfqpoint{2.235800in}{1.627779in}}{\pgfqpoint{2.235800in}{1.616729in}}%
\pgfpathcurveto{\pgfqpoint{2.235800in}{1.605679in}}{\pgfqpoint{2.240190in}{1.595080in}}{\pgfqpoint{2.248004in}{1.587266in}}%
\pgfpathcurveto{\pgfqpoint{2.255817in}{1.579453in}}{\pgfqpoint{2.266417in}{1.575062in}}{\pgfqpoint{2.277467in}{1.575062in}}%
\pgfpathclose%
\pgfusepath{stroke,fill}%
\end{pgfscope}%
\begin{pgfscope}%
\pgfpathrectangle{\pgfqpoint{0.511823in}{0.504323in}}{\pgfqpoint{3.218177in}{3.225677in}} %
\pgfusepath{clip}%
\pgfsetbuttcap%
\pgfsetroundjoin%
\definecolor{currentfill}{rgb}{0.000000,0.000000,0.545098}%
\pgfsetfillcolor{currentfill}%
\pgfsetfillopacity{0.400000}%
\pgfsetlinewidth{0.501875pt}%
\definecolor{currentstroke}{rgb}{0.000000,0.000000,0.545098}%
\pgfsetstrokecolor{currentstroke}%
\pgfsetstrokeopacity{0.400000}%
\pgfsetdash{}{0pt}%
\pgfpathmoveto{\pgfqpoint{2.472431in}{1.697008in}}%
\pgfpathcurveto{\pgfqpoint{2.483481in}{1.697008in}}{\pgfqpoint{2.494080in}{1.701398in}}{\pgfqpoint{2.501894in}{1.709212in}}%
\pgfpathcurveto{\pgfqpoint{2.509707in}{1.717025in}}{\pgfqpoint{2.514098in}{1.727624in}}{\pgfqpoint{2.514098in}{1.738674in}}%
\pgfpathcurveto{\pgfqpoint{2.514098in}{1.749725in}}{\pgfqpoint{2.509707in}{1.760324in}}{\pgfqpoint{2.501894in}{1.768137in}}%
\pgfpathcurveto{\pgfqpoint{2.494080in}{1.775951in}}{\pgfqpoint{2.483481in}{1.780341in}}{\pgfqpoint{2.472431in}{1.780341in}}%
\pgfpathcurveto{\pgfqpoint{2.461381in}{1.780341in}}{\pgfqpoint{2.450782in}{1.775951in}}{\pgfqpoint{2.442968in}{1.768137in}}%
\pgfpathcurveto{\pgfqpoint{2.435154in}{1.760324in}}{\pgfqpoint{2.430764in}{1.749725in}}{\pgfqpoint{2.430764in}{1.738674in}}%
\pgfpathcurveto{\pgfqpoint{2.430764in}{1.727624in}}{\pgfqpoint{2.435154in}{1.717025in}}{\pgfqpoint{2.442968in}{1.709212in}}%
\pgfpathcurveto{\pgfqpoint{2.450782in}{1.701398in}}{\pgfqpoint{2.461381in}{1.697008in}}{\pgfqpoint{2.472431in}{1.697008in}}%
\pgfpathclose%
\pgfusepath{stroke,fill}%
\end{pgfscope}%
\begin{pgfscope}%
\pgfpathrectangle{\pgfqpoint{0.511823in}{0.504323in}}{\pgfqpoint{3.218177in}{3.225677in}} %
\pgfusepath{clip}%
\pgfsetbuttcap%
\pgfsetroundjoin%
\definecolor{currentfill}{rgb}{0.000000,0.000000,0.545098}%
\pgfsetfillcolor{currentfill}%
\pgfsetfillopacity{0.400000}%
\pgfsetlinewidth{0.501875pt}%
\definecolor{currentstroke}{rgb}{0.000000,0.000000,0.545098}%
\pgfsetstrokecolor{currentstroke}%
\pgfsetstrokeopacity{0.400000}%
\pgfsetdash{}{0pt}%
\pgfpathmoveto{\pgfqpoint{2.290700in}{1.596043in}}%
\pgfpathcurveto{\pgfqpoint{2.301750in}{1.596043in}}{\pgfqpoint{2.312349in}{1.600434in}}{\pgfqpoint{2.320163in}{1.608247in}}%
\pgfpathcurveto{\pgfqpoint{2.327976in}{1.616061in}}{\pgfqpoint{2.332367in}{1.626660in}}{\pgfqpoint{2.332367in}{1.637710in}}%
\pgfpathcurveto{\pgfqpoint{2.332367in}{1.648760in}}{\pgfqpoint{2.327976in}{1.659359in}}{\pgfqpoint{2.320163in}{1.667173in}}%
\pgfpathcurveto{\pgfqpoint{2.312349in}{1.674986in}}{\pgfqpoint{2.301750in}{1.679377in}}{\pgfqpoint{2.290700in}{1.679377in}}%
\pgfpathcurveto{\pgfqpoint{2.279650in}{1.679377in}}{\pgfqpoint{2.269051in}{1.674986in}}{\pgfqpoint{2.261237in}{1.667173in}}%
\pgfpathcurveto{\pgfqpoint{2.253424in}{1.659359in}}{\pgfqpoint{2.249033in}{1.648760in}}{\pgfqpoint{2.249033in}{1.637710in}}%
\pgfpathcurveto{\pgfqpoint{2.249033in}{1.626660in}}{\pgfqpoint{2.253424in}{1.616061in}}{\pgfqpoint{2.261237in}{1.608247in}}%
\pgfpathcurveto{\pgfqpoint{2.269051in}{1.600434in}}{\pgfqpoint{2.279650in}{1.596043in}}{\pgfqpoint{2.290700in}{1.596043in}}%
\pgfpathclose%
\pgfusepath{stroke,fill}%
\end{pgfscope}%
\begin{pgfscope}%
\pgfpathrectangle{\pgfqpoint{0.511823in}{0.504323in}}{\pgfqpoint{3.218177in}{3.225677in}} %
\pgfusepath{clip}%
\pgfsetbuttcap%
\pgfsetroundjoin%
\definecolor{currentfill}{rgb}{0.000000,0.000000,0.545098}%
\pgfsetfillcolor{currentfill}%
\pgfsetfillopacity{0.400000}%
\pgfsetlinewidth{0.501875pt}%
\definecolor{currentstroke}{rgb}{0.000000,0.000000,0.545098}%
\pgfsetstrokecolor{currentstroke}%
\pgfsetstrokeopacity{0.400000}%
\pgfsetdash{}{0pt}%
\pgfpathmoveto{\pgfqpoint{2.342682in}{1.633885in}}%
\pgfpathcurveto{\pgfqpoint{2.353732in}{1.633885in}}{\pgfqpoint{2.364331in}{1.638275in}}{\pgfqpoint{2.372145in}{1.646089in}}%
\pgfpathcurveto{\pgfqpoint{2.379959in}{1.653903in}}{\pgfqpoint{2.384349in}{1.664502in}}{\pgfqpoint{2.384349in}{1.675552in}}%
\pgfpathcurveto{\pgfqpoint{2.384349in}{1.686602in}}{\pgfqpoint{2.379959in}{1.697201in}}{\pgfqpoint{2.372145in}{1.705015in}}%
\pgfpathcurveto{\pgfqpoint{2.364331in}{1.712828in}}{\pgfqpoint{2.353732in}{1.717219in}}{\pgfqpoint{2.342682in}{1.717219in}}%
\pgfpathcurveto{\pgfqpoint{2.331632in}{1.717219in}}{\pgfqpoint{2.321033in}{1.712828in}}{\pgfqpoint{2.313219in}{1.705015in}}%
\pgfpathcurveto{\pgfqpoint{2.305406in}{1.697201in}}{\pgfqpoint{2.301015in}{1.686602in}}{\pgfqpoint{2.301015in}{1.675552in}}%
\pgfpathcurveto{\pgfqpoint{2.301015in}{1.664502in}}{\pgfqpoint{2.305406in}{1.653903in}}{\pgfqpoint{2.313219in}{1.646089in}}%
\pgfpathcurveto{\pgfqpoint{2.321033in}{1.638275in}}{\pgfqpoint{2.331632in}{1.633885in}}{\pgfqpoint{2.342682in}{1.633885in}}%
\pgfpathclose%
\pgfusepath{stroke,fill}%
\end{pgfscope}%
\begin{pgfscope}%
\pgfpathrectangle{\pgfqpoint{0.511823in}{0.504323in}}{\pgfqpoint{3.218177in}{3.225677in}} %
\pgfusepath{clip}%
\pgfsetbuttcap%
\pgfsetroundjoin%
\definecolor{currentfill}{rgb}{0.000000,0.000000,0.545098}%
\pgfsetfillcolor{currentfill}%
\pgfsetfillopacity{0.400000}%
\pgfsetlinewidth{0.501875pt}%
\definecolor{currentstroke}{rgb}{0.000000,0.000000,0.545098}%
\pgfsetstrokecolor{currentstroke}%
\pgfsetstrokeopacity{0.400000}%
\pgfsetdash{}{0pt}%
\pgfpathmoveto{\pgfqpoint{2.330292in}{1.633281in}}%
\pgfpathcurveto{\pgfqpoint{2.341342in}{1.633281in}}{\pgfqpoint{2.351941in}{1.637671in}}{\pgfqpoint{2.359755in}{1.645485in}}%
\pgfpathcurveto{\pgfqpoint{2.367568in}{1.653298in}}{\pgfqpoint{2.371959in}{1.663897in}}{\pgfqpoint{2.371959in}{1.674947in}}%
\pgfpathcurveto{\pgfqpoint{2.371959in}{1.685997in}}{\pgfqpoint{2.367568in}{1.696597in}}{\pgfqpoint{2.359755in}{1.704410in}}%
\pgfpathcurveto{\pgfqpoint{2.351941in}{1.712224in}}{\pgfqpoint{2.341342in}{1.716614in}}{\pgfqpoint{2.330292in}{1.716614in}}%
\pgfpathcurveto{\pgfqpoint{2.319242in}{1.716614in}}{\pgfqpoint{2.308643in}{1.712224in}}{\pgfqpoint{2.300829in}{1.704410in}}%
\pgfpathcurveto{\pgfqpoint{2.293016in}{1.696597in}}{\pgfqpoint{2.288625in}{1.685997in}}{\pgfqpoint{2.288625in}{1.674947in}}%
\pgfpathcurveto{\pgfqpoint{2.288625in}{1.663897in}}{\pgfqpoint{2.293016in}{1.653298in}}{\pgfqpoint{2.300829in}{1.645485in}}%
\pgfpathcurveto{\pgfqpoint{2.308643in}{1.637671in}}{\pgfqpoint{2.319242in}{1.633281in}}{\pgfqpoint{2.330292in}{1.633281in}}%
\pgfpathclose%
\pgfusepath{stroke,fill}%
\end{pgfscope}%
\begin{pgfscope}%
\pgfpathrectangle{\pgfqpoint{0.511823in}{0.504323in}}{\pgfqpoint{3.218177in}{3.225677in}} %
\pgfusepath{clip}%
\pgfsetbuttcap%
\pgfsetroundjoin%
\definecolor{currentfill}{rgb}{0.000000,0.000000,0.545098}%
\pgfsetfillcolor{currentfill}%
\pgfsetfillopacity{0.400000}%
\pgfsetlinewidth{0.501875pt}%
\definecolor{currentstroke}{rgb}{0.000000,0.000000,0.545098}%
\pgfsetstrokecolor{currentstroke}%
\pgfsetstrokeopacity{0.400000}%
\pgfsetdash{}{0pt}%
\pgfpathmoveto{\pgfqpoint{2.236514in}{1.583072in}}%
\pgfpathcurveto{\pgfqpoint{2.247565in}{1.583072in}}{\pgfqpoint{2.258164in}{1.587462in}}{\pgfqpoint{2.265977in}{1.595276in}}%
\pgfpathcurveto{\pgfqpoint{2.273791in}{1.603089in}}{\pgfqpoint{2.278181in}{1.613688in}}{\pgfqpoint{2.278181in}{1.624739in}}%
\pgfpathcurveto{\pgfqpoint{2.278181in}{1.635789in}}{\pgfqpoint{2.273791in}{1.646388in}}{\pgfqpoint{2.265977in}{1.654201in}}%
\pgfpathcurveto{\pgfqpoint{2.258164in}{1.662015in}}{\pgfqpoint{2.247565in}{1.666405in}}{\pgfqpoint{2.236514in}{1.666405in}}%
\pgfpathcurveto{\pgfqpoint{2.225464in}{1.666405in}}{\pgfqpoint{2.214865in}{1.662015in}}{\pgfqpoint{2.207052in}{1.654201in}}%
\pgfpathcurveto{\pgfqpoint{2.199238in}{1.646388in}}{\pgfqpoint{2.194848in}{1.635789in}}{\pgfqpoint{2.194848in}{1.624739in}}%
\pgfpathcurveto{\pgfqpoint{2.194848in}{1.613688in}}{\pgfqpoint{2.199238in}{1.603089in}}{\pgfqpoint{2.207052in}{1.595276in}}%
\pgfpathcurveto{\pgfqpoint{2.214865in}{1.587462in}}{\pgfqpoint{2.225464in}{1.583072in}}{\pgfqpoint{2.236514in}{1.583072in}}%
\pgfpathclose%
\pgfusepath{stroke,fill}%
\end{pgfscope}%
\begin{pgfscope}%
\pgfpathrectangle{\pgfqpoint{0.511823in}{0.504323in}}{\pgfqpoint{3.218177in}{3.225677in}} %
\pgfusepath{clip}%
\pgfsetbuttcap%
\pgfsetroundjoin%
\definecolor{currentfill}{rgb}{0.000000,0.000000,0.545098}%
\pgfsetfillcolor{currentfill}%
\pgfsetfillopacity{0.400000}%
\pgfsetlinewidth{0.501875pt}%
\definecolor{currentstroke}{rgb}{0.000000,0.000000,0.545098}%
\pgfsetstrokecolor{currentstroke}%
\pgfsetstrokeopacity{0.400000}%
\pgfsetdash{}{0pt}%
\pgfpathmoveto{\pgfqpoint{2.329352in}{1.646453in}}%
\pgfpathcurveto{\pgfqpoint{2.340402in}{1.646453in}}{\pgfqpoint{2.351001in}{1.650844in}}{\pgfqpoint{2.358815in}{1.658657in}}%
\pgfpathcurveto{\pgfqpoint{2.366628in}{1.666471in}}{\pgfqpoint{2.371018in}{1.677070in}}{\pgfqpoint{2.371018in}{1.688120in}}%
\pgfpathcurveto{\pgfqpoint{2.371018in}{1.699170in}}{\pgfqpoint{2.366628in}{1.709769in}}{\pgfqpoint{2.358815in}{1.717583in}}%
\pgfpathcurveto{\pgfqpoint{2.351001in}{1.725397in}}{\pgfqpoint{2.340402in}{1.729787in}}{\pgfqpoint{2.329352in}{1.729787in}}%
\pgfpathcurveto{\pgfqpoint{2.318302in}{1.729787in}}{\pgfqpoint{2.307703in}{1.725397in}}{\pgfqpoint{2.299889in}{1.717583in}}%
\pgfpathcurveto{\pgfqpoint{2.292075in}{1.709769in}}{\pgfqpoint{2.287685in}{1.699170in}}{\pgfqpoint{2.287685in}{1.688120in}}%
\pgfpathcurveto{\pgfqpoint{2.287685in}{1.677070in}}{\pgfqpoint{2.292075in}{1.666471in}}{\pgfqpoint{2.299889in}{1.658657in}}%
\pgfpathcurveto{\pgfqpoint{2.307703in}{1.650844in}}{\pgfqpoint{2.318302in}{1.646453in}}{\pgfqpoint{2.329352in}{1.646453in}}%
\pgfpathclose%
\pgfusepath{stroke,fill}%
\end{pgfscope}%
\begin{pgfscope}%
\pgfpathrectangle{\pgfqpoint{0.511823in}{0.504323in}}{\pgfqpoint{3.218177in}{3.225677in}} %
\pgfusepath{clip}%
\pgfsetbuttcap%
\pgfsetroundjoin%
\definecolor{currentfill}{rgb}{0.000000,0.000000,0.545098}%
\pgfsetfillcolor{currentfill}%
\pgfsetfillopacity{0.400000}%
\pgfsetlinewidth{0.501875pt}%
\definecolor{currentstroke}{rgb}{0.000000,0.000000,0.545098}%
\pgfsetstrokecolor{currentstroke}%
\pgfsetstrokeopacity{0.400000}%
\pgfsetdash{}{0pt}%
\pgfpathmoveto{\pgfqpoint{2.397590in}{1.695467in}}%
\pgfpathcurveto{\pgfqpoint{2.408641in}{1.695467in}}{\pgfqpoint{2.419240in}{1.699857in}}{\pgfqpoint{2.427053in}{1.707671in}}%
\pgfpathcurveto{\pgfqpoint{2.434867in}{1.715485in}}{\pgfqpoint{2.439257in}{1.726084in}}{\pgfqpoint{2.439257in}{1.737134in}}%
\pgfpathcurveto{\pgfqpoint{2.439257in}{1.748184in}}{\pgfqpoint{2.434867in}{1.758783in}}{\pgfqpoint{2.427053in}{1.766596in}}%
\pgfpathcurveto{\pgfqpoint{2.419240in}{1.774410in}}{\pgfqpoint{2.408641in}{1.778800in}}{\pgfqpoint{2.397590in}{1.778800in}}%
\pgfpathcurveto{\pgfqpoint{2.386540in}{1.778800in}}{\pgfqpoint{2.375941in}{1.774410in}}{\pgfqpoint{2.368128in}{1.766596in}}%
\pgfpathcurveto{\pgfqpoint{2.360314in}{1.758783in}}{\pgfqpoint{2.355924in}{1.748184in}}{\pgfqpoint{2.355924in}{1.737134in}}%
\pgfpathcurveto{\pgfqpoint{2.355924in}{1.726084in}}{\pgfqpoint{2.360314in}{1.715485in}}{\pgfqpoint{2.368128in}{1.707671in}}%
\pgfpathcurveto{\pgfqpoint{2.375941in}{1.699857in}}{\pgfqpoint{2.386540in}{1.695467in}}{\pgfqpoint{2.397590in}{1.695467in}}%
\pgfpathclose%
\pgfusepath{stroke,fill}%
\end{pgfscope}%
\begin{pgfscope}%
\pgfpathrectangle{\pgfqpoint{0.511823in}{0.504323in}}{\pgfqpoint{3.218177in}{3.225677in}} %
\pgfusepath{clip}%
\pgfsetbuttcap%
\pgfsetroundjoin%
\definecolor{currentfill}{rgb}{0.000000,0.000000,0.545098}%
\pgfsetfillcolor{currentfill}%
\pgfsetfillopacity{0.400000}%
\pgfsetlinewidth{0.501875pt}%
\definecolor{currentstroke}{rgb}{0.000000,0.000000,0.545098}%
\pgfsetstrokecolor{currentstroke}%
\pgfsetstrokeopacity{0.400000}%
\pgfsetdash{}{0pt}%
\pgfpathmoveto{\pgfqpoint{2.295755in}{1.639439in}}%
\pgfpathcurveto{\pgfqpoint{2.306805in}{1.639439in}}{\pgfqpoint{2.317404in}{1.643830in}}{\pgfqpoint{2.325218in}{1.651643in}}%
\pgfpathcurveto{\pgfqpoint{2.333031in}{1.659457in}}{\pgfqpoint{2.337422in}{1.670056in}}{\pgfqpoint{2.337422in}{1.681106in}}%
\pgfpathcurveto{\pgfqpoint{2.337422in}{1.692156in}}{\pgfqpoint{2.333031in}{1.702755in}}{\pgfqpoint{2.325218in}{1.710569in}}%
\pgfpathcurveto{\pgfqpoint{2.317404in}{1.718382in}}{\pgfqpoint{2.306805in}{1.722773in}}{\pgfqpoint{2.295755in}{1.722773in}}%
\pgfpathcurveto{\pgfqpoint{2.284705in}{1.722773in}}{\pgfqpoint{2.274106in}{1.718382in}}{\pgfqpoint{2.266292in}{1.710569in}}%
\pgfpathcurveto{\pgfqpoint{2.258479in}{1.702755in}}{\pgfqpoint{2.254088in}{1.692156in}}{\pgfqpoint{2.254088in}{1.681106in}}%
\pgfpathcurveto{\pgfqpoint{2.254088in}{1.670056in}}{\pgfqpoint{2.258479in}{1.659457in}}{\pgfqpoint{2.266292in}{1.651643in}}%
\pgfpathcurveto{\pgfqpoint{2.274106in}{1.643830in}}{\pgfqpoint{2.284705in}{1.639439in}}{\pgfqpoint{2.295755in}{1.639439in}}%
\pgfpathclose%
\pgfusepath{stroke,fill}%
\end{pgfscope}%
\begin{pgfscope}%
\pgfpathrectangle{\pgfqpoint{0.511823in}{0.504323in}}{\pgfqpoint{3.218177in}{3.225677in}} %
\pgfusepath{clip}%
\pgfsetbuttcap%
\pgfsetroundjoin%
\definecolor{currentfill}{rgb}{0.000000,0.000000,0.545098}%
\pgfsetfillcolor{currentfill}%
\pgfsetfillopacity{0.400000}%
\pgfsetlinewidth{0.501875pt}%
\definecolor{currentstroke}{rgb}{0.000000,0.000000,0.545098}%
\pgfsetstrokecolor{currentstroke}%
\pgfsetstrokeopacity{0.400000}%
\pgfsetdash{}{0pt}%
\pgfpathmoveto{\pgfqpoint{2.198789in}{1.585623in}}%
\pgfpathcurveto{\pgfqpoint{2.209839in}{1.585623in}}{\pgfqpoint{2.220438in}{1.590013in}}{\pgfqpoint{2.228252in}{1.597827in}}%
\pgfpathcurveto{\pgfqpoint{2.236065in}{1.605641in}}{\pgfqpoint{2.240455in}{1.616240in}}{\pgfqpoint{2.240455in}{1.627290in}}%
\pgfpathcurveto{\pgfqpoint{2.240455in}{1.638340in}}{\pgfqpoint{2.236065in}{1.648939in}}{\pgfqpoint{2.228252in}{1.656753in}}%
\pgfpathcurveto{\pgfqpoint{2.220438in}{1.664566in}}{\pgfqpoint{2.209839in}{1.668956in}}{\pgfqpoint{2.198789in}{1.668956in}}%
\pgfpathcurveto{\pgfqpoint{2.187739in}{1.668956in}}{\pgfqpoint{2.177140in}{1.664566in}}{\pgfqpoint{2.169326in}{1.656753in}}%
\pgfpathcurveto{\pgfqpoint{2.161512in}{1.648939in}}{\pgfqpoint{2.157122in}{1.638340in}}{\pgfqpoint{2.157122in}{1.627290in}}%
\pgfpathcurveto{\pgfqpoint{2.157122in}{1.616240in}}{\pgfqpoint{2.161512in}{1.605641in}}{\pgfqpoint{2.169326in}{1.597827in}}%
\pgfpathcurveto{\pgfqpoint{2.177140in}{1.590013in}}{\pgfqpoint{2.187739in}{1.585623in}}{\pgfqpoint{2.198789in}{1.585623in}}%
\pgfpathclose%
\pgfusepath{stroke,fill}%
\end{pgfscope}%
\begin{pgfscope}%
\pgfpathrectangle{\pgfqpoint{0.511823in}{0.504323in}}{\pgfqpoint{3.218177in}{3.225677in}} %
\pgfusepath{clip}%
\pgfsetbuttcap%
\pgfsetroundjoin%
\definecolor{currentfill}{rgb}{0.000000,0.000000,0.545098}%
\pgfsetfillcolor{currentfill}%
\pgfsetfillopacity{0.400000}%
\pgfsetlinewidth{0.501875pt}%
\definecolor{currentstroke}{rgb}{0.000000,0.000000,0.545098}%
\pgfsetstrokecolor{currentstroke}%
\pgfsetstrokeopacity{0.400000}%
\pgfsetdash{}{0pt}%
\pgfpathmoveto{\pgfqpoint{2.271486in}{1.637844in}}%
\pgfpathcurveto{\pgfqpoint{2.282536in}{1.637844in}}{\pgfqpoint{2.293135in}{1.642234in}}{\pgfqpoint{2.300949in}{1.650048in}}%
\pgfpathcurveto{\pgfqpoint{2.308762in}{1.657862in}}{\pgfqpoint{2.313152in}{1.668461in}}{\pgfqpoint{2.313152in}{1.679511in}}%
\pgfpathcurveto{\pgfqpoint{2.313152in}{1.690561in}}{\pgfqpoint{2.308762in}{1.701160in}}{\pgfqpoint{2.300949in}{1.708974in}}%
\pgfpathcurveto{\pgfqpoint{2.293135in}{1.716787in}}{\pgfqpoint{2.282536in}{1.721177in}}{\pgfqpoint{2.271486in}{1.721177in}}%
\pgfpathcurveto{\pgfqpoint{2.260436in}{1.721177in}}{\pgfqpoint{2.249837in}{1.716787in}}{\pgfqpoint{2.242023in}{1.708974in}}%
\pgfpathcurveto{\pgfqpoint{2.234209in}{1.701160in}}{\pgfqpoint{2.229819in}{1.690561in}}{\pgfqpoint{2.229819in}{1.679511in}}%
\pgfpathcurveto{\pgfqpoint{2.229819in}{1.668461in}}{\pgfqpoint{2.234209in}{1.657862in}}{\pgfqpoint{2.242023in}{1.650048in}}%
\pgfpathcurveto{\pgfqpoint{2.249837in}{1.642234in}}{\pgfqpoint{2.260436in}{1.637844in}}{\pgfqpoint{2.271486in}{1.637844in}}%
\pgfpathclose%
\pgfusepath{stroke,fill}%
\end{pgfscope}%
\begin{pgfscope}%
\pgfpathrectangle{\pgfqpoint{0.511823in}{0.504323in}}{\pgfqpoint{3.218177in}{3.225677in}} %
\pgfusepath{clip}%
\pgfsetbuttcap%
\pgfsetroundjoin%
\definecolor{currentfill}{rgb}{0.000000,0.000000,0.545098}%
\pgfsetfillcolor{currentfill}%
\pgfsetfillopacity{0.400000}%
\pgfsetlinewidth{0.501875pt}%
\definecolor{currentstroke}{rgb}{0.000000,0.000000,0.545098}%
\pgfsetstrokecolor{currentstroke}%
\pgfsetstrokeopacity{0.400000}%
\pgfsetdash{}{0pt}%
\pgfpathmoveto{\pgfqpoint{2.358333in}{1.699686in}}%
\pgfpathcurveto{\pgfqpoint{2.369384in}{1.699686in}}{\pgfqpoint{2.379983in}{1.704076in}}{\pgfqpoint{2.387796in}{1.711890in}}%
\pgfpathcurveto{\pgfqpoint{2.395610in}{1.719703in}}{\pgfqpoint{2.400000in}{1.730302in}}{\pgfqpoint{2.400000in}{1.741352in}}%
\pgfpathcurveto{\pgfqpoint{2.400000in}{1.752403in}}{\pgfqpoint{2.395610in}{1.763002in}}{\pgfqpoint{2.387796in}{1.770815in}}%
\pgfpathcurveto{\pgfqpoint{2.379983in}{1.778629in}}{\pgfqpoint{2.369384in}{1.783019in}}{\pgfqpoint{2.358333in}{1.783019in}}%
\pgfpathcurveto{\pgfqpoint{2.347283in}{1.783019in}}{\pgfqpoint{2.336684in}{1.778629in}}{\pgfqpoint{2.328871in}{1.770815in}}%
\pgfpathcurveto{\pgfqpoint{2.321057in}{1.763002in}}{\pgfqpoint{2.316667in}{1.752403in}}{\pgfqpoint{2.316667in}{1.741352in}}%
\pgfpathcurveto{\pgfqpoint{2.316667in}{1.730302in}}{\pgfqpoint{2.321057in}{1.719703in}}{\pgfqpoint{2.328871in}{1.711890in}}%
\pgfpathcurveto{\pgfqpoint{2.336684in}{1.704076in}}{\pgfqpoint{2.347283in}{1.699686in}}{\pgfqpoint{2.358333in}{1.699686in}}%
\pgfpathclose%
\pgfusepath{stroke,fill}%
\end{pgfscope}%
\begin{pgfscope}%
\pgfpathrectangle{\pgfqpoint{0.511823in}{0.504323in}}{\pgfqpoint{3.218177in}{3.225677in}} %
\pgfusepath{clip}%
\pgfsetbuttcap%
\pgfsetroundjoin%
\definecolor{currentfill}{rgb}{0.000000,0.000000,0.545098}%
\pgfsetfillcolor{currentfill}%
\pgfsetfillopacity{0.400000}%
\pgfsetlinewidth{0.501875pt}%
\definecolor{currentstroke}{rgb}{0.000000,0.000000,0.545098}%
\pgfsetstrokecolor{currentstroke}%
\pgfsetstrokeopacity{0.400000}%
\pgfsetdash{}{0pt}%
\pgfpathmoveto{\pgfqpoint{2.452911in}{1.767237in}}%
\pgfpathcurveto{\pgfqpoint{2.463962in}{1.767237in}}{\pgfqpoint{2.474561in}{1.771627in}}{\pgfqpoint{2.482374in}{1.779441in}}%
\pgfpathcurveto{\pgfqpoint{2.490188in}{1.787255in}}{\pgfqpoint{2.494578in}{1.797854in}}{\pgfqpoint{2.494578in}{1.808904in}}%
\pgfpathcurveto{\pgfqpoint{2.494578in}{1.819954in}}{\pgfqpoint{2.490188in}{1.830553in}}{\pgfqpoint{2.482374in}{1.838367in}}%
\pgfpathcurveto{\pgfqpoint{2.474561in}{1.846180in}}{\pgfqpoint{2.463962in}{1.850571in}}{\pgfqpoint{2.452911in}{1.850571in}}%
\pgfpathcurveto{\pgfqpoint{2.441861in}{1.850571in}}{\pgfqpoint{2.431262in}{1.846180in}}{\pgfqpoint{2.423449in}{1.838367in}}%
\pgfpathcurveto{\pgfqpoint{2.415635in}{1.830553in}}{\pgfqpoint{2.411245in}{1.819954in}}{\pgfqpoint{2.411245in}{1.808904in}}%
\pgfpathcurveto{\pgfqpoint{2.411245in}{1.797854in}}{\pgfqpoint{2.415635in}{1.787255in}}{\pgfqpoint{2.423449in}{1.779441in}}%
\pgfpathcurveto{\pgfqpoint{2.431262in}{1.771627in}}{\pgfqpoint{2.441861in}{1.767237in}}{\pgfqpoint{2.452911in}{1.767237in}}%
\pgfpathclose%
\pgfusepath{stroke,fill}%
\end{pgfscope}%
\begin{pgfscope}%
\pgfpathrectangle{\pgfqpoint{0.511823in}{0.504323in}}{\pgfqpoint{3.218177in}{3.225677in}} %
\pgfusepath{clip}%
\pgfsetbuttcap%
\pgfsetroundjoin%
\definecolor{currentfill}{rgb}{0.000000,0.000000,0.545098}%
\pgfsetfillcolor{currentfill}%
\pgfsetfillopacity{0.400000}%
\pgfsetlinewidth{0.501875pt}%
\definecolor{currentstroke}{rgb}{0.000000,0.000000,0.545098}%
\pgfsetstrokecolor{currentstroke}%
\pgfsetstrokeopacity{0.400000}%
\pgfsetdash{}{0pt}%
\pgfpathmoveto{\pgfqpoint{2.415111in}{1.750572in}}%
\pgfpathcurveto{\pgfqpoint{2.426161in}{1.750572in}}{\pgfqpoint{2.436760in}{1.754963in}}{\pgfqpoint{2.444574in}{1.762776in}}%
\pgfpathcurveto{\pgfqpoint{2.452387in}{1.770590in}}{\pgfqpoint{2.456778in}{1.781189in}}{\pgfqpoint{2.456778in}{1.792239in}}%
\pgfpathcurveto{\pgfqpoint{2.456778in}{1.803289in}}{\pgfqpoint{2.452387in}{1.813888in}}{\pgfqpoint{2.444574in}{1.821702in}}%
\pgfpathcurveto{\pgfqpoint{2.436760in}{1.829515in}}{\pgfqpoint{2.426161in}{1.833906in}}{\pgfqpoint{2.415111in}{1.833906in}}%
\pgfpathcurveto{\pgfqpoint{2.404061in}{1.833906in}}{\pgfqpoint{2.393462in}{1.829515in}}{\pgfqpoint{2.385648in}{1.821702in}}%
\pgfpathcurveto{\pgfqpoint{2.377834in}{1.813888in}}{\pgfqpoint{2.373444in}{1.803289in}}{\pgfqpoint{2.373444in}{1.792239in}}%
\pgfpathcurveto{\pgfqpoint{2.373444in}{1.781189in}}{\pgfqpoint{2.377834in}{1.770590in}}{\pgfqpoint{2.385648in}{1.762776in}}%
\pgfpathcurveto{\pgfqpoint{2.393462in}{1.754963in}}{\pgfqpoint{2.404061in}{1.750572in}}{\pgfqpoint{2.415111in}{1.750572in}}%
\pgfpathclose%
\pgfusepath{stroke,fill}%
\end{pgfscope}%
\begin{pgfscope}%
\pgfpathrectangle{\pgfqpoint{0.511823in}{0.504323in}}{\pgfqpoint{3.218177in}{3.225677in}} %
\pgfusepath{clip}%
\pgfsetbuttcap%
\pgfsetroundjoin%
\definecolor{currentfill}{rgb}{0.000000,0.000000,0.545098}%
\pgfsetfillcolor{currentfill}%
\pgfsetfillopacity{0.400000}%
\pgfsetlinewidth{0.501875pt}%
\definecolor{currentstroke}{rgb}{0.000000,0.000000,0.545098}%
\pgfsetstrokecolor{currentstroke}%
\pgfsetstrokeopacity{0.400000}%
\pgfsetdash{}{0pt}%
\pgfpathmoveto{\pgfqpoint{2.351595in}{1.716978in}}%
\pgfpathcurveto{\pgfqpoint{2.362645in}{1.716978in}}{\pgfqpoint{2.373245in}{1.721368in}}{\pgfqpoint{2.381058in}{1.729182in}}%
\pgfpathcurveto{\pgfqpoint{2.388872in}{1.736995in}}{\pgfqpoint{2.393262in}{1.747594in}}{\pgfqpoint{2.393262in}{1.758644in}}%
\pgfpathcurveto{\pgfqpoint{2.393262in}{1.769694in}}{\pgfqpoint{2.388872in}{1.780293in}}{\pgfqpoint{2.381058in}{1.788107in}}%
\pgfpathcurveto{\pgfqpoint{2.373245in}{1.795921in}}{\pgfqpoint{2.362645in}{1.800311in}}{\pgfqpoint{2.351595in}{1.800311in}}%
\pgfpathcurveto{\pgfqpoint{2.340545in}{1.800311in}}{\pgfqpoint{2.329946in}{1.795921in}}{\pgfqpoint{2.322133in}{1.788107in}}%
\pgfpathcurveto{\pgfqpoint{2.314319in}{1.780293in}}{\pgfqpoint{2.309929in}{1.769694in}}{\pgfqpoint{2.309929in}{1.758644in}}%
\pgfpathcurveto{\pgfqpoint{2.309929in}{1.747594in}}{\pgfqpoint{2.314319in}{1.736995in}}{\pgfqpoint{2.322133in}{1.729182in}}%
\pgfpathcurveto{\pgfqpoint{2.329946in}{1.721368in}}{\pgfqpoint{2.340545in}{1.716978in}}{\pgfqpoint{2.351595in}{1.716978in}}%
\pgfpathclose%
\pgfusepath{stroke,fill}%
\end{pgfscope}%
\begin{pgfscope}%
\pgfpathrectangle{\pgfqpoint{0.511823in}{0.504323in}}{\pgfqpoint{3.218177in}{3.225677in}} %
\pgfusepath{clip}%
\pgfsetbuttcap%
\pgfsetroundjoin%
\definecolor{currentfill}{rgb}{0.000000,0.000000,0.545098}%
\pgfsetfillcolor{currentfill}%
\pgfsetfillopacity{0.400000}%
\pgfsetlinewidth{0.501875pt}%
\definecolor{currentstroke}{rgb}{0.000000,0.000000,0.545098}%
\pgfsetstrokecolor{currentstroke}%
\pgfsetstrokeopacity{0.400000}%
\pgfsetdash{}{0pt}%
\pgfpathmoveto{\pgfqpoint{2.208501in}{1.631035in}}%
\pgfpathcurveto{\pgfqpoint{2.219551in}{1.631035in}}{\pgfqpoint{2.230150in}{1.635425in}}{\pgfqpoint{2.237963in}{1.643239in}}%
\pgfpathcurveto{\pgfqpoint{2.245777in}{1.651052in}}{\pgfqpoint{2.250167in}{1.661651in}}{\pgfqpoint{2.250167in}{1.672702in}}%
\pgfpathcurveto{\pgfqpoint{2.250167in}{1.683752in}}{\pgfqpoint{2.245777in}{1.694351in}}{\pgfqpoint{2.237963in}{1.702164in}}%
\pgfpathcurveto{\pgfqpoint{2.230150in}{1.709978in}}{\pgfqpoint{2.219551in}{1.714368in}}{\pgfqpoint{2.208501in}{1.714368in}}%
\pgfpathcurveto{\pgfqpoint{2.197450in}{1.714368in}}{\pgfqpoint{2.186851in}{1.709978in}}{\pgfqpoint{2.179038in}{1.702164in}}%
\pgfpathcurveto{\pgfqpoint{2.171224in}{1.694351in}}{\pgfqpoint{2.166834in}{1.683752in}}{\pgfqpoint{2.166834in}{1.672702in}}%
\pgfpathcurveto{\pgfqpoint{2.166834in}{1.661651in}}{\pgfqpoint{2.171224in}{1.651052in}}{\pgfqpoint{2.179038in}{1.643239in}}%
\pgfpathcurveto{\pgfqpoint{2.186851in}{1.635425in}}{\pgfqpoint{2.197450in}{1.631035in}}{\pgfqpoint{2.208501in}{1.631035in}}%
\pgfpathclose%
\pgfusepath{stroke,fill}%
\end{pgfscope}%
\begin{pgfscope}%
\pgfpathrectangle{\pgfqpoint{0.511823in}{0.504323in}}{\pgfqpoint{3.218177in}{3.225677in}} %
\pgfusepath{clip}%
\pgfsetbuttcap%
\pgfsetroundjoin%
\definecolor{currentfill}{rgb}{0.000000,0.000000,0.545098}%
\pgfsetfillcolor{currentfill}%
\pgfsetfillopacity{0.400000}%
\pgfsetlinewidth{0.501875pt}%
\definecolor{currentstroke}{rgb}{0.000000,0.000000,0.545098}%
\pgfsetstrokecolor{currentstroke}%
\pgfsetstrokeopacity{0.400000}%
\pgfsetdash{}{0pt}%
\pgfpathmoveto{\pgfqpoint{2.411453in}{1.770757in}}%
\pgfpathcurveto{\pgfqpoint{2.422504in}{1.770757in}}{\pgfqpoint{2.433103in}{1.775147in}}{\pgfqpoint{2.440916in}{1.782961in}}%
\pgfpathcurveto{\pgfqpoint{2.448730in}{1.790774in}}{\pgfqpoint{2.453120in}{1.801373in}}{\pgfqpoint{2.453120in}{1.812423in}}%
\pgfpathcurveto{\pgfqpoint{2.453120in}{1.823473in}}{\pgfqpoint{2.448730in}{1.834072in}}{\pgfqpoint{2.440916in}{1.841886in}}%
\pgfpathcurveto{\pgfqpoint{2.433103in}{1.849700in}}{\pgfqpoint{2.422504in}{1.854090in}}{\pgfqpoint{2.411453in}{1.854090in}}%
\pgfpathcurveto{\pgfqpoint{2.400403in}{1.854090in}}{\pgfqpoint{2.389804in}{1.849700in}}{\pgfqpoint{2.381991in}{1.841886in}}%
\pgfpathcurveto{\pgfqpoint{2.374177in}{1.834072in}}{\pgfqpoint{2.369787in}{1.823473in}}{\pgfqpoint{2.369787in}{1.812423in}}%
\pgfpathcurveto{\pgfqpoint{2.369787in}{1.801373in}}{\pgfqpoint{2.374177in}{1.790774in}}{\pgfqpoint{2.381991in}{1.782961in}}%
\pgfpathcurveto{\pgfqpoint{2.389804in}{1.775147in}}{\pgfqpoint{2.400403in}{1.770757in}}{\pgfqpoint{2.411453in}{1.770757in}}%
\pgfpathclose%
\pgfusepath{stroke,fill}%
\end{pgfscope}%
\begin{pgfscope}%
\pgfpathrectangle{\pgfqpoint{0.511823in}{0.504323in}}{\pgfqpoint{3.218177in}{3.225677in}} %
\pgfusepath{clip}%
\pgfsetbuttcap%
\pgfsetroundjoin%
\definecolor{currentfill}{rgb}{0.000000,0.000000,0.545098}%
\pgfsetfillcolor{currentfill}%
\pgfsetfillopacity{0.400000}%
\pgfsetlinewidth{0.501875pt}%
\definecolor{currentstroke}{rgb}{0.000000,0.000000,0.545098}%
\pgfsetstrokecolor{currentstroke}%
\pgfsetstrokeopacity{0.400000}%
\pgfsetdash{}{0pt}%
\pgfpathmoveto{\pgfqpoint{2.312110in}{1.712764in}}%
\pgfpathcurveto{\pgfqpoint{2.323160in}{1.712764in}}{\pgfqpoint{2.333759in}{1.717154in}}{\pgfqpoint{2.341572in}{1.724968in}}%
\pgfpathcurveto{\pgfqpoint{2.349386in}{1.732782in}}{\pgfqpoint{2.353776in}{1.743381in}}{\pgfqpoint{2.353776in}{1.754431in}}%
\pgfpathcurveto{\pgfqpoint{2.353776in}{1.765481in}}{\pgfqpoint{2.349386in}{1.776080in}}{\pgfqpoint{2.341572in}{1.783894in}}%
\pgfpathcurveto{\pgfqpoint{2.333759in}{1.791707in}}{\pgfqpoint{2.323160in}{1.796098in}}{\pgfqpoint{2.312110in}{1.796098in}}%
\pgfpathcurveto{\pgfqpoint{2.301060in}{1.796098in}}{\pgfqpoint{2.290461in}{1.791707in}}{\pgfqpoint{2.282647in}{1.783894in}}%
\pgfpathcurveto{\pgfqpoint{2.274833in}{1.776080in}}{\pgfqpoint{2.270443in}{1.765481in}}{\pgfqpoint{2.270443in}{1.754431in}}%
\pgfpathcurveto{\pgfqpoint{2.270443in}{1.743381in}}{\pgfqpoint{2.274833in}{1.732782in}}{\pgfqpoint{2.282647in}{1.724968in}}%
\pgfpathcurveto{\pgfqpoint{2.290461in}{1.717154in}}{\pgfqpoint{2.301060in}{1.712764in}}{\pgfqpoint{2.312110in}{1.712764in}}%
\pgfpathclose%
\pgfusepath{stroke,fill}%
\end{pgfscope}%
\begin{pgfscope}%
\pgfpathrectangle{\pgfqpoint{0.511823in}{0.504323in}}{\pgfqpoint{3.218177in}{3.225677in}} %
\pgfusepath{clip}%
\pgfsetbuttcap%
\pgfsetroundjoin%
\definecolor{currentfill}{rgb}{0.000000,0.000000,0.545098}%
\pgfsetfillcolor{currentfill}%
\pgfsetfillopacity{0.400000}%
\pgfsetlinewidth{0.501875pt}%
\definecolor{currentstroke}{rgb}{0.000000,0.000000,0.545098}%
\pgfsetstrokecolor{currentstroke}%
\pgfsetstrokeopacity{0.400000}%
\pgfsetdash{}{0pt}%
\pgfpathmoveto{\pgfqpoint{2.279949in}{1.698569in}}%
\pgfpathcurveto{\pgfqpoint{2.290999in}{1.698569in}}{\pgfqpoint{2.301598in}{1.702959in}}{\pgfqpoint{2.309412in}{1.710773in}}%
\pgfpathcurveto{\pgfqpoint{2.317225in}{1.718586in}}{\pgfqpoint{2.321616in}{1.729185in}}{\pgfqpoint{2.321616in}{1.740235in}}%
\pgfpathcurveto{\pgfqpoint{2.321616in}{1.751285in}}{\pgfqpoint{2.317225in}{1.761884in}}{\pgfqpoint{2.309412in}{1.769698in}}%
\pgfpathcurveto{\pgfqpoint{2.301598in}{1.777512in}}{\pgfqpoint{2.290999in}{1.781902in}}{\pgfqpoint{2.279949in}{1.781902in}}%
\pgfpathcurveto{\pgfqpoint{2.268899in}{1.781902in}}{\pgfqpoint{2.258300in}{1.777512in}}{\pgfqpoint{2.250486in}{1.769698in}}%
\pgfpathcurveto{\pgfqpoint{2.242673in}{1.761884in}}{\pgfqpoint{2.238282in}{1.751285in}}{\pgfqpoint{2.238282in}{1.740235in}}%
\pgfpathcurveto{\pgfqpoint{2.238282in}{1.729185in}}{\pgfqpoint{2.242673in}{1.718586in}}{\pgfqpoint{2.250486in}{1.710773in}}%
\pgfpathcurveto{\pgfqpoint{2.258300in}{1.702959in}}{\pgfqpoint{2.268899in}{1.698569in}}{\pgfqpoint{2.279949in}{1.698569in}}%
\pgfpathclose%
\pgfusepath{stroke,fill}%
\end{pgfscope}%
\begin{pgfscope}%
\pgfpathrectangle{\pgfqpoint{0.511823in}{0.504323in}}{\pgfqpoint{3.218177in}{3.225677in}} %
\pgfusepath{clip}%
\pgfsetbuttcap%
\pgfsetroundjoin%
\definecolor{currentfill}{rgb}{0.000000,0.000000,0.545098}%
\pgfsetfillcolor{currentfill}%
\pgfsetfillopacity{0.400000}%
\pgfsetlinewidth{0.501875pt}%
\definecolor{currentstroke}{rgb}{0.000000,0.000000,0.545098}%
\pgfsetstrokecolor{currentstroke}%
\pgfsetstrokeopacity{0.400000}%
\pgfsetdash{}{0pt}%
\pgfpathmoveto{\pgfqpoint{2.400245in}{1.786096in}}%
\pgfpathcurveto{\pgfqpoint{2.411295in}{1.786096in}}{\pgfqpoint{2.421894in}{1.790486in}}{\pgfqpoint{2.429707in}{1.798300in}}%
\pgfpathcurveto{\pgfqpoint{2.437521in}{1.806113in}}{\pgfqpoint{2.441911in}{1.816712in}}{\pgfqpoint{2.441911in}{1.827762in}}%
\pgfpathcurveto{\pgfqpoint{2.441911in}{1.838812in}}{\pgfqpoint{2.437521in}{1.849411in}}{\pgfqpoint{2.429707in}{1.857225in}}%
\pgfpathcurveto{\pgfqpoint{2.421894in}{1.865039in}}{\pgfqpoint{2.411295in}{1.869429in}}{\pgfqpoint{2.400245in}{1.869429in}}%
\pgfpathcurveto{\pgfqpoint{2.389194in}{1.869429in}}{\pgfqpoint{2.378595in}{1.865039in}}{\pgfqpoint{2.370782in}{1.857225in}}%
\pgfpathcurveto{\pgfqpoint{2.362968in}{1.849411in}}{\pgfqpoint{2.358578in}{1.838812in}}{\pgfqpoint{2.358578in}{1.827762in}}%
\pgfpathcurveto{\pgfqpoint{2.358578in}{1.816712in}}{\pgfqpoint{2.362968in}{1.806113in}}{\pgfqpoint{2.370782in}{1.798300in}}%
\pgfpathcurveto{\pgfqpoint{2.378595in}{1.790486in}}{\pgfqpoint{2.389194in}{1.786096in}}{\pgfqpoint{2.400245in}{1.786096in}}%
\pgfpathclose%
\pgfusepath{stroke,fill}%
\end{pgfscope}%
\begin{pgfscope}%
\pgfpathrectangle{\pgfqpoint{0.511823in}{0.504323in}}{\pgfqpoint{3.218177in}{3.225677in}} %
\pgfusepath{clip}%
\pgfsetbuttcap%
\pgfsetroundjoin%
\definecolor{currentfill}{rgb}{0.000000,0.000000,0.545098}%
\pgfsetfillcolor{currentfill}%
\pgfsetfillopacity{0.400000}%
\pgfsetlinewidth{0.501875pt}%
\definecolor{currentstroke}{rgb}{0.000000,0.000000,0.545098}%
\pgfsetstrokecolor{currentstroke}%
\pgfsetstrokeopacity{0.400000}%
\pgfsetdash{}{0pt}%
\pgfpathmoveto{\pgfqpoint{2.364947in}{1.769960in}}%
\pgfpathcurveto{\pgfqpoint{2.375997in}{1.769960in}}{\pgfqpoint{2.386596in}{1.774350in}}{\pgfqpoint{2.394410in}{1.782163in}}%
\pgfpathcurveto{\pgfqpoint{2.402223in}{1.789977in}}{\pgfqpoint{2.406614in}{1.800576in}}{\pgfqpoint{2.406614in}{1.811626in}}%
\pgfpathcurveto{\pgfqpoint{2.406614in}{1.822676in}}{\pgfqpoint{2.402223in}{1.833275in}}{\pgfqpoint{2.394410in}{1.841089in}}%
\pgfpathcurveto{\pgfqpoint{2.386596in}{1.848903in}}{\pgfqpoint{2.375997in}{1.853293in}}{\pgfqpoint{2.364947in}{1.853293in}}%
\pgfpathcurveto{\pgfqpoint{2.353897in}{1.853293in}}{\pgfqpoint{2.343298in}{1.848903in}}{\pgfqpoint{2.335484in}{1.841089in}}%
\pgfpathcurveto{\pgfqpoint{2.327671in}{1.833275in}}{\pgfqpoint{2.323280in}{1.822676in}}{\pgfqpoint{2.323280in}{1.811626in}}%
\pgfpathcurveto{\pgfqpoint{2.323280in}{1.800576in}}{\pgfqpoint{2.327671in}{1.789977in}}{\pgfqpoint{2.335484in}{1.782163in}}%
\pgfpathcurveto{\pgfqpoint{2.343298in}{1.774350in}}{\pgfqpoint{2.353897in}{1.769960in}}{\pgfqpoint{2.364947in}{1.769960in}}%
\pgfpathclose%
\pgfusepath{stroke,fill}%
\end{pgfscope}%
\begin{pgfscope}%
\pgfpathrectangle{\pgfqpoint{0.511823in}{0.504323in}}{\pgfqpoint{3.218177in}{3.225677in}} %
\pgfusepath{clip}%
\pgfsetbuttcap%
\pgfsetroundjoin%
\definecolor{currentfill}{rgb}{0.000000,0.000000,0.545098}%
\pgfsetfillcolor{currentfill}%
\pgfsetfillopacity{0.400000}%
\pgfsetlinewidth{0.501875pt}%
\definecolor{currentstroke}{rgb}{0.000000,0.000000,0.545098}%
\pgfsetstrokecolor{currentstroke}%
\pgfsetstrokeopacity{0.400000}%
\pgfsetdash{}{0pt}%
\pgfpathmoveto{\pgfqpoint{2.348438in}{1.766278in}}%
\pgfpathcurveto{\pgfqpoint{2.359488in}{1.766278in}}{\pgfqpoint{2.370087in}{1.770668in}}{\pgfqpoint{2.377901in}{1.778481in}}%
\pgfpathcurveto{\pgfqpoint{2.385714in}{1.786295in}}{\pgfqpoint{2.390104in}{1.796894in}}{\pgfqpoint{2.390104in}{1.807944in}}%
\pgfpathcurveto{\pgfqpoint{2.390104in}{1.818994in}}{\pgfqpoint{2.385714in}{1.829593in}}{\pgfqpoint{2.377901in}{1.837407in}}%
\pgfpathcurveto{\pgfqpoint{2.370087in}{1.845221in}}{\pgfqpoint{2.359488in}{1.849611in}}{\pgfqpoint{2.348438in}{1.849611in}}%
\pgfpathcurveto{\pgfqpoint{2.337388in}{1.849611in}}{\pgfqpoint{2.326789in}{1.845221in}}{\pgfqpoint{2.318975in}{1.837407in}}%
\pgfpathcurveto{\pgfqpoint{2.311161in}{1.829593in}}{\pgfqpoint{2.306771in}{1.818994in}}{\pgfqpoint{2.306771in}{1.807944in}}%
\pgfpathcurveto{\pgfqpoint{2.306771in}{1.796894in}}{\pgfqpoint{2.311161in}{1.786295in}}{\pgfqpoint{2.318975in}{1.778481in}}%
\pgfpathcurveto{\pgfqpoint{2.326789in}{1.770668in}}{\pgfqpoint{2.337388in}{1.766278in}}{\pgfqpoint{2.348438in}{1.766278in}}%
\pgfpathclose%
\pgfusepath{stroke,fill}%
\end{pgfscope}%
\begin{pgfscope}%
\pgfpathrectangle{\pgfqpoint{0.511823in}{0.504323in}}{\pgfqpoint{3.218177in}{3.225677in}} %
\pgfusepath{clip}%
\pgfsetbuttcap%
\pgfsetroundjoin%
\definecolor{currentfill}{rgb}{0.000000,0.000000,0.545098}%
\pgfsetfillcolor{currentfill}%
\pgfsetfillopacity{0.400000}%
\pgfsetlinewidth{0.501875pt}%
\definecolor{currentstroke}{rgb}{0.000000,0.000000,0.545098}%
\pgfsetstrokecolor{currentstroke}%
\pgfsetstrokeopacity{0.400000}%
\pgfsetdash{}{0pt}%
\pgfpathmoveto{\pgfqpoint{2.167974in}{1.650615in}}%
\pgfpathcurveto{\pgfqpoint{2.179024in}{1.650615in}}{\pgfqpoint{2.189623in}{1.655005in}}{\pgfqpoint{2.197437in}{1.662819in}}%
\pgfpathcurveto{\pgfqpoint{2.205250in}{1.670633in}}{\pgfqpoint{2.209640in}{1.681232in}}{\pgfqpoint{2.209640in}{1.692282in}}%
\pgfpathcurveto{\pgfqpoint{2.209640in}{1.703332in}}{\pgfqpoint{2.205250in}{1.713931in}}{\pgfqpoint{2.197437in}{1.721744in}}%
\pgfpathcurveto{\pgfqpoint{2.189623in}{1.729558in}}{\pgfqpoint{2.179024in}{1.733948in}}{\pgfqpoint{2.167974in}{1.733948in}}%
\pgfpathcurveto{\pgfqpoint{2.156924in}{1.733948in}}{\pgfqpoint{2.146325in}{1.729558in}}{\pgfqpoint{2.138511in}{1.721744in}}%
\pgfpathcurveto{\pgfqpoint{2.130697in}{1.713931in}}{\pgfqpoint{2.126307in}{1.703332in}}{\pgfqpoint{2.126307in}{1.692282in}}%
\pgfpathcurveto{\pgfqpoint{2.126307in}{1.681232in}}{\pgfqpoint{2.130697in}{1.670633in}}{\pgfqpoint{2.138511in}{1.662819in}}%
\pgfpathcurveto{\pgfqpoint{2.146325in}{1.655005in}}{\pgfqpoint{2.156924in}{1.650615in}}{\pgfqpoint{2.167974in}{1.650615in}}%
\pgfpathclose%
\pgfusepath{stroke,fill}%
\end{pgfscope}%
\begin{pgfscope}%
\pgfpathrectangle{\pgfqpoint{0.511823in}{0.504323in}}{\pgfqpoint{3.218177in}{3.225677in}} %
\pgfusepath{clip}%
\pgfsetbuttcap%
\pgfsetroundjoin%
\definecolor{currentfill}{rgb}{0.000000,0.000000,0.545098}%
\pgfsetfillcolor{currentfill}%
\pgfsetfillopacity{0.400000}%
\pgfsetlinewidth{0.501875pt}%
\definecolor{currentstroke}{rgb}{0.000000,0.000000,0.545098}%
\pgfsetstrokecolor{currentstroke}%
\pgfsetstrokeopacity{0.400000}%
\pgfsetdash{}{0pt}%
\pgfpathmoveto{\pgfqpoint{2.222293in}{1.694610in}}%
\pgfpathcurveto{\pgfqpoint{2.233343in}{1.694610in}}{\pgfqpoint{2.243942in}{1.699001in}}{\pgfqpoint{2.251756in}{1.706814in}}%
\pgfpathcurveto{\pgfqpoint{2.259569in}{1.714628in}}{\pgfqpoint{2.263960in}{1.725227in}}{\pgfqpoint{2.263960in}{1.736277in}}%
\pgfpathcurveto{\pgfqpoint{2.263960in}{1.747327in}}{\pgfqpoint{2.259569in}{1.757926in}}{\pgfqpoint{2.251756in}{1.765740in}}%
\pgfpathcurveto{\pgfqpoint{2.243942in}{1.773554in}}{\pgfqpoint{2.233343in}{1.777944in}}{\pgfqpoint{2.222293in}{1.777944in}}%
\pgfpathcurveto{\pgfqpoint{2.211243in}{1.777944in}}{\pgfqpoint{2.200644in}{1.773554in}}{\pgfqpoint{2.192830in}{1.765740in}}%
\pgfpathcurveto{\pgfqpoint{2.185017in}{1.757926in}}{\pgfqpoint{2.180626in}{1.747327in}}{\pgfqpoint{2.180626in}{1.736277in}}%
\pgfpathcurveto{\pgfqpoint{2.180626in}{1.725227in}}{\pgfqpoint{2.185017in}{1.714628in}}{\pgfqpoint{2.192830in}{1.706814in}}%
\pgfpathcurveto{\pgfqpoint{2.200644in}{1.699001in}}{\pgfqpoint{2.211243in}{1.694610in}}{\pgfqpoint{2.222293in}{1.694610in}}%
\pgfpathclose%
\pgfusepath{stroke,fill}%
\end{pgfscope}%
\begin{pgfscope}%
\pgfpathrectangle{\pgfqpoint{0.511823in}{0.504323in}}{\pgfqpoint{3.218177in}{3.225677in}} %
\pgfusepath{clip}%
\pgfsetbuttcap%
\pgfsetroundjoin%
\definecolor{currentfill}{rgb}{0.000000,0.000000,0.545098}%
\pgfsetfillcolor{currentfill}%
\pgfsetfillopacity{0.400000}%
\pgfsetlinewidth{0.501875pt}%
\definecolor{currentstroke}{rgb}{0.000000,0.000000,0.545098}%
\pgfsetstrokecolor{currentstroke}%
\pgfsetstrokeopacity{0.400000}%
\pgfsetdash{}{0pt}%
\pgfpathmoveto{\pgfqpoint{2.102288in}{1.618611in}}%
\pgfpathcurveto{\pgfqpoint{2.113338in}{1.618611in}}{\pgfqpoint{2.123937in}{1.623001in}}{\pgfqpoint{2.131751in}{1.630815in}}%
\pgfpathcurveto{\pgfqpoint{2.139565in}{1.638628in}}{\pgfqpoint{2.143955in}{1.649227in}}{\pgfqpoint{2.143955in}{1.660277in}}%
\pgfpathcurveto{\pgfqpoint{2.143955in}{1.671328in}}{\pgfqpoint{2.139565in}{1.681927in}}{\pgfqpoint{2.131751in}{1.689740in}}%
\pgfpathcurveto{\pgfqpoint{2.123937in}{1.697554in}}{\pgfqpoint{2.113338in}{1.701944in}}{\pgfqpoint{2.102288in}{1.701944in}}%
\pgfpathcurveto{\pgfqpoint{2.091238in}{1.701944in}}{\pgfqpoint{2.080639in}{1.697554in}}{\pgfqpoint{2.072825in}{1.689740in}}%
\pgfpathcurveto{\pgfqpoint{2.065012in}{1.681927in}}{\pgfqpoint{2.060622in}{1.671328in}}{\pgfqpoint{2.060622in}{1.660277in}}%
\pgfpathcurveto{\pgfqpoint{2.060622in}{1.649227in}}{\pgfqpoint{2.065012in}{1.638628in}}{\pgfqpoint{2.072825in}{1.630815in}}%
\pgfpathcurveto{\pgfqpoint{2.080639in}{1.623001in}}{\pgfqpoint{2.091238in}{1.618611in}}{\pgfqpoint{2.102288in}{1.618611in}}%
\pgfpathclose%
\pgfusepath{stroke,fill}%
\end{pgfscope}%
\begin{pgfscope}%
\pgfpathrectangle{\pgfqpoint{0.511823in}{0.504323in}}{\pgfqpoint{3.218177in}{3.225677in}} %
\pgfusepath{clip}%
\pgfsetbuttcap%
\pgfsetroundjoin%
\definecolor{currentfill}{rgb}{0.000000,0.000000,0.545098}%
\pgfsetfillcolor{currentfill}%
\pgfsetfillopacity{0.400000}%
\pgfsetlinewidth{0.501875pt}%
\definecolor{currentstroke}{rgb}{0.000000,0.000000,0.545098}%
\pgfsetstrokecolor{currentstroke}%
\pgfsetstrokeopacity{0.400000}%
\pgfsetdash{}{0pt}%
\pgfpathmoveto{\pgfqpoint{2.300640in}{1.763097in}}%
\pgfpathcurveto{\pgfqpoint{2.311691in}{1.763097in}}{\pgfqpoint{2.322290in}{1.767488in}}{\pgfqpoint{2.330103in}{1.775301in}}%
\pgfpathcurveto{\pgfqpoint{2.337917in}{1.783115in}}{\pgfqpoint{2.342307in}{1.793714in}}{\pgfqpoint{2.342307in}{1.804764in}}%
\pgfpathcurveto{\pgfqpoint{2.342307in}{1.815814in}}{\pgfqpoint{2.337917in}{1.826413in}}{\pgfqpoint{2.330103in}{1.834227in}}%
\pgfpathcurveto{\pgfqpoint{2.322290in}{1.842040in}}{\pgfqpoint{2.311691in}{1.846431in}}{\pgfqpoint{2.300640in}{1.846431in}}%
\pgfpathcurveto{\pgfqpoint{2.289590in}{1.846431in}}{\pgfqpoint{2.278991in}{1.842040in}}{\pgfqpoint{2.271178in}{1.834227in}}%
\pgfpathcurveto{\pgfqpoint{2.263364in}{1.826413in}}{\pgfqpoint{2.258974in}{1.815814in}}{\pgfqpoint{2.258974in}{1.804764in}}%
\pgfpathcurveto{\pgfqpoint{2.258974in}{1.793714in}}{\pgfqpoint{2.263364in}{1.783115in}}{\pgfqpoint{2.271178in}{1.775301in}}%
\pgfpathcurveto{\pgfqpoint{2.278991in}{1.767488in}}{\pgfqpoint{2.289590in}{1.763097in}}{\pgfqpoint{2.300640in}{1.763097in}}%
\pgfpathclose%
\pgfusepath{stroke,fill}%
\end{pgfscope}%
\begin{pgfscope}%
\pgfpathrectangle{\pgfqpoint{0.511823in}{0.504323in}}{\pgfqpoint{3.218177in}{3.225677in}} %
\pgfusepath{clip}%
\pgfsetbuttcap%
\pgfsetroundjoin%
\definecolor{currentfill}{rgb}{0.000000,0.000000,0.545098}%
\pgfsetfillcolor{currentfill}%
\pgfsetfillopacity{0.400000}%
\pgfsetlinewidth{0.501875pt}%
\definecolor{currentstroke}{rgb}{0.000000,0.000000,0.545098}%
\pgfsetstrokecolor{currentstroke}%
\pgfsetstrokeopacity{0.400000}%
\pgfsetdash{}{0pt}%
\pgfpathmoveto{\pgfqpoint{2.279218in}{1.755481in}}%
\pgfpathcurveto{\pgfqpoint{2.290268in}{1.755481in}}{\pgfqpoint{2.300867in}{1.759871in}}{\pgfqpoint{2.308680in}{1.767685in}}%
\pgfpathcurveto{\pgfqpoint{2.316494in}{1.775499in}}{\pgfqpoint{2.320884in}{1.786098in}}{\pgfqpoint{2.320884in}{1.797148in}}%
\pgfpathcurveto{\pgfqpoint{2.320884in}{1.808198in}}{\pgfqpoint{2.316494in}{1.818797in}}{\pgfqpoint{2.308680in}{1.826610in}}%
\pgfpathcurveto{\pgfqpoint{2.300867in}{1.834424in}}{\pgfqpoint{2.290268in}{1.838814in}}{\pgfqpoint{2.279218in}{1.838814in}}%
\pgfpathcurveto{\pgfqpoint{2.268167in}{1.838814in}}{\pgfqpoint{2.257568in}{1.834424in}}{\pgfqpoint{2.249755in}{1.826610in}}%
\pgfpathcurveto{\pgfqpoint{2.241941in}{1.818797in}}{\pgfqpoint{2.237551in}{1.808198in}}{\pgfqpoint{2.237551in}{1.797148in}}%
\pgfpathcurveto{\pgfqpoint{2.237551in}{1.786098in}}{\pgfqpoint{2.241941in}{1.775499in}}{\pgfqpoint{2.249755in}{1.767685in}}%
\pgfpathcurveto{\pgfqpoint{2.257568in}{1.759871in}}{\pgfqpoint{2.268167in}{1.755481in}}{\pgfqpoint{2.279218in}{1.755481in}}%
\pgfpathclose%
\pgfusepath{stroke,fill}%
\end{pgfscope}%
\begin{pgfscope}%
\pgfpathrectangle{\pgfqpoint{0.511823in}{0.504323in}}{\pgfqpoint{3.218177in}{3.225677in}} %
\pgfusepath{clip}%
\pgfsetbuttcap%
\pgfsetroundjoin%
\definecolor{currentfill}{rgb}{0.000000,0.000000,0.545098}%
\pgfsetfillcolor{currentfill}%
\pgfsetfillopacity{0.400000}%
\pgfsetlinewidth{0.501875pt}%
\definecolor{currentstroke}{rgb}{0.000000,0.000000,0.545098}%
\pgfsetstrokecolor{currentstroke}%
\pgfsetstrokeopacity{0.400000}%
\pgfsetdash{}{0pt}%
\pgfpathmoveto{\pgfqpoint{2.371052in}{1.827559in}}%
\pgfpathcurveto{\pgfqpoint{2.382102in}{1.827559in}}{\pgfqpoint{2.392701in}{1.831949in}}{\pgfqpoint{2.400515in}{1.839763in}}%
\pgfpathcurveto{\pgfqpoint{2.408329in}{1.847577in}}{\pgfqpoint{2.412719in}{1.858176in}}{\pgfqpoint{2.412719in}{1.869226in}}%
\pgfpathcurveto{\pgfqpoint{2.412719in}{1.880276in}}{\pgfqpoint{2.408329in}{1.890875in}}{\pgfqpoint{2.400515in}{1.898689in}}%
\pgfpathcurveto{\pgfqpoint{2.392701in}{1.906502in}}{\pgfqpoint{2.382102in}{1.910892in}}{\pgfqpoint{2.371052in}{1.910892in}}%
\pgfpathcurveto{\pgfqpoint{2.360002in}{1.910892in}}{\pgfqpoint{2.349403in}{1.906502in}}{\pgfqpoint{2.341590in}{1.898689in}}%
\pgfpathcurveto{\pgfqpoint{2.333776in}{1.890875in}}{\pgfqpoint{2.329386in}{1.880276in}}{\pgfqpoint{2.329386in}{1.869226in}}%
\pgfpathcurveto{\pgfqpoint{2.329386in}{1.858176in}}{\pgfqpoint{2.333776in}{1.847577in}}{\pgfqpoint{2.341590in}{1.839763in}}%
\pgfpathcurveto{\pgfqpoint{2.349403in}{1.831949in}}{\pgfqpoint{2.360002in}{1.827559in}}{\pgfqpoint{2.371052in}{1.827559in}}%
\pgfpathclose%
\pgfusepath{stroke,fill}%
\end{pgfscope}%
\begin{pgfscope}%
\pgfpathrectangle{\pgfqpoint{0.511823in}{0.504323in}}{\pgfqpoint{3.218177in}{3.225677in}} %
\pgfusepath{clip}%
\pgfsetbuttcap%
\pgfsetroundjoin%
\definecolor{currentfill}{rgb}{0.000000,0.000000,0.545098}%
\pgfsetfillcolor{currentfill}%
\pgfsetfillopacity{0.400000}%
\pgfsetlinewidth{0.501875pt}%
\definecolor{currentstroke}{rgb}{0.000000,0.000000,0.545098}%
\pgfsetstrokecolor{currentstroke}%
\pgfsetstrokeopacity{0.400000}%
\pgfsetdash{}{0pt}%
\pgfpathmoveto{\pgfqpoint{2.291024in}{1.778549in}}%
\pgfpathcurveto{\pgfqpoint{2.302075in}{1.778549in}}{\pgfqpoint{2.312674in}{1.782939in}}{\pgfqpoint{2.320487in}{1.790753in}}%
\pgfpathcurveto{\pgfqpoint{2.328301in}{1.798566in}}{\pgfqpoint{2.332691in}{1.809165in}}{\pgfqpoint{2.332691in}{1.820216in}}%
\pgfpathcurveto{\pgfqpoint{2.332691in}{1.831266in}}{\pgfqpoint{2.328301in}{1.841865in}}{\pgfqpoint{2.320487in}{1.849678in}}%
\pgfpathcurveto{\pgfqpoint{2.312674in}{1.857492in}}{\pgfqpoint{2.302075in}{1.861882in}}{\pgfqpoint{2.291024in}{1.861882in}}%
\pgfpathcurveto{\pgfqpoint{2.279974in}{1.861882in}}{\pgfqpoint{2.269375in}{1.857492in}}{\pgfqpoint{2.261562in}{1.849678in}}%
\pgfpathcurveto{\pgfqpoint{2.253748in}{1.841865in}}{\pgfqpoint{2.249358in}{1.831266in}}{\pgfqpoint{2.249358in}{1.820216in}}%
\pgfpathcurveto{\pgfqpoint{2.249358in}{1.809165in}}{\pgfqpoint{2.253748in}{1.798566in}}{\pgfqpoint{2.261562in}{1.790753in}}%
\pgfpathcurveto{\pgfqpoint{2.269375in}{1.782939in}}{\pgfqpoint{2.279974in}{1.778549in}}{\pgfqpoint{2.291024in}{1.778549in}}%
\pgfpathclose%
\pgfusepath{stroke,fill}%
\end{pgfscope}%
\begin{pgfscope}%
\pgfpathrectangle{\pgfqpoint{0.511823in}{0.504323in}}{\pgfqpoint{3.218177in}{3.225677in}} %
\pgfusepath{clip}%
\pgfsetbuttcap%
\pgfsetroundjoin%
\definecolor{currentfill}{rgb}{0.000000,0.000000,0.545098}%
\pgfsetfillcolor{currentfill}%
\pgfsetfillopacity{0.400000}%
\pgfsetlinewidth{0.501875pt}%
\definecolor{currentstroke}{rgb}{0.000000,0.000000,0.545098}%
\pgfsetstrokecolor{currentstroke}%
\pgfsetstrokeopacity{0.400000}%
\pgfsetdash{}{0pt}%
\pgfpathmoveto{\pgfqpoint{2.226133in}{1.739646in}}%
\pgfpathcurveto{\pgfqpoint{2.237183in}{1.739646in}}{\pgfqpoint{2.247782in}{1.744036in}}{\pgfqpoint{2.255596in}{1.751849in}}%
\pgfpathcurveto{\pgfqpoint{2.263410in}{1.759663in}}{\pgfqpoint{2.267800in}{1.770262in}}{\pgfqpoint{2.267800in}{1.781312in}}%
\pgfpathcurveto{\pgfqpoint{2.267800in}{1.792362in}}{\pgfqpoint{2.263410in}{1.802961in}}{\pgfqpoint{2.255596in}{1.810775in}}%
\pgfpathcurveto{\pgfqpoint{2.247782in}{1.818589in}}{\pgfqpoint{2.237183in}{1.822979in}}{\pgfqpoint{2.226133in}{1.822979in}}%
\pgfpathcurveto{\pgfqpoint{2.215083in}{1.822979in}}{\pgfqpoint{2.204484in}{1.818589in}}{\pgfqpoint{2.196670in}{1.810775in}}%
\pgfpathcurveto{\pgfqpoint{2.188857in}{1.802961in}}{\pgfqpoint{2.184466in}{1.792362in}}{\pgfqpoint{2.184466in}{1.781312in}}%
\pgfpathcurveto{\pgfqpoint{2.184466in}{1.770262in}}{\pgfqpoint{2.188857in}{1.759663in}}{\pgfqpoint{2.196670in}{1.751849in}}%
\pgfpathcurveto{\pgfqpoint{2.204484in}{1.744036in}}{\pgfqpoint{2.215083in}{1.739646in}}{\pgfqpoint{2.226133in}{1.739646in}}%
\pgfpathclose%
\pgfusepath{stroke,fill}%
\end{pgfscope}%
\begin{pgfscope}%
\pgfpathrectangle{\pgfqpoint{0.511823in}{0.504323in}}{\pgfqpoint{3.218177in}{3.225677in}} %
\pgfusepath{clip}%
\pgfsetbuttcap%
\pgfsetroundjoin%
\definecolor{currentfill}{rgb}{0.000000,0.000000,0.545098}%
\pgfsetfillcolor{currentfill}%
\pgfsetfillopacity{0.400000}%
\pgfsetlinewidth{0.501875pt}%
\definecolor{currentstroke}{rgb}{0.000000,0.000000,0.545098}%
\pgfsetstrokecolor{currentstroke}%
\pgfsetstrokeopacity{0.400000}%
\pgfsetdash{}{0pt}%
\pgfpathmoveto{\pgfqpoint{2.482677in}{1.931291in}}%
\pgfpathcurveto{\pgfqpoint{2.493727in}{1.931291in}}{\pgfqpoint{2.504326in}{1.935681in}}{\pgfqpoint{2.512140in}{1.943495in}}%
\pgfpathcurveto{\pgfqpoint{2.519953in}{1.951308in}}{\pgfqpoint{2.524344in}{1.961907in}}{\pgfqpoint{2.524344in}{1.972958in}}%
\pgfpathcurveto{\pgfqpoint{2.524344in}{1.984008in}}{\pgfqpoint{2.519953in}{1.994607in}}{\pgfqpoint{2.512140in}{2.002420in}}%
\pgfpathcurveto{\pgfqpoint{2.504326in}{2.010234in}}{\pgfqpoint{2.493727in}{2.014624in}}{\pgfqpoint{2.482677in}{2.014624in}}%
\pgfpathcurveto{\pgfqpoint{2.471627in}{2.014624in}}{\pgfqpoint{2.461028in}{2.010234in}}{\pgfqpoint{2.453214in}{2.002420in}}%
\pgfpathcurveto{\pgfqpoint{2.445401in}{1.994607in}}{\pgfqpoint{2.441010in}{1.984008in}}{\pgfqpoint{2.441010in}{1.972958in}}%
\pgfpathcurveto{\pgfqpoint{2.441010in}{1.961907in}}{\pgfqpoint{2.445401in}{1.951308in}}{\pgfqpoint{2.453214in}{1.943495in}}%
\pgfpathcurveto{\pgfqpoint{2.461028in}{1.935681in}}{\pgfqpoint{2.471627in}{1.931291in}}{\pgfqpoint{2.482677in}{1.931291in}}%
\pgfpathclose%
\pgfusepath{stroke,fill}%
\end{pgfscope}%
\begin{pgfscope}%
\pgfpathrectangle{\pgfqpoint{0.511823in}{0.504323in}}{\pgfqpoint{3.218177in}{3.225677in}} %
\pgfusepath{clip}%
\pgfsetbuttcap%
\pgfsetroundjoin%
\definecolor{currentfill}{rgb}{0.000000,0.000000,0.545098}%
\pgfsetfillcolor{currentfill}%
\pgfsetfillopacity{0.400000}%
\pgfsetlinewidth{0.501875pt}%
\definecolor{currentstroke}{rgb}{0.000000,0.000000,0.545098}%
\pgfsetstrokecolor{currentstroke}%
\pgfsetstrokeopacity{0.400000}%
\pgfsetdash{}{0pt}%
\pgfpathmoveto{\pgfqpoint{2.149377in}{1.698502in}}%
\pgfpathcurveto{\pgfqpoint{2.160427in}{1.698502in}}{\pgfqpoint{2.171026in}{1.702892in}}{\pgfqpoint{2.178840in}{1.710706in}}%
\pgfpathcurveto{\pgfqpoint{2.186653in}{1.718520in}}{\pgfqpoint{2.191043in}{1.729119in}}{\pgfqpoint{2.191043in}{1.740169in}}%
\pgfpathcurveto{\pgfqpoint{2.191043in}{1.751219in}}{\pgfqpoint{2.186653in}{1.761818in}}{\pgfqpoint{2.178840in}{1.769631in}}%
\pgfpathcurveto{\pgfqpoint{2.171026in}{1.777445in}}{\pgfqpoint{2.160427in}{1.781835in}}{\pgfqpoint{2.149377in}{1.781835in}}%
\pgfpathcurveto{\pgfqpoint{2.138327in}{1.781835in}}{\pgfqpoint{2.127728in}{1.777445in}}{\pgfqpoint{2.119914in}{1.769631in}}%
\pgfpathcurveto{\pgfqpoint{2.112100in}{1.761818in}}{\pgfqpoint{2.107710in}{1.751219in}}{\pgfqpoint{2.107710in}{1.740169in}}%
\pgfpathcurveto{\pgfqpoint{2.107710in}{1.729119in}}{\pgfqpoint{2.112100in}{1.718520in}}{\pgfqpoint{2.119914in}{1.710706in}}%
\pgfpathcurveto{\pgfqpoint{2.127728in}{1.702892in}}{\pgfqpoint{2.138327in}{1.698502in}}{\pgfqpoint{2.149377in}{1.698502in}}%
\pgfpathclose%
\pgfusepath{stroke,fill}%
\end{pgfscope}%
\begin{pgfscope}%
\pgfpathrectangle{\pgfqpoint{0.511823in}{0.504323in}}{\pgfqpoint{3.218177in}{3.225677in}} %
\pgfusepath{clip}%
\pgfsetbuttcap%
\pgfsetroundjoin%
\definecolor{currentfill}{rgb}{0.000000,0.000000,0.545098}%
\pgfsetfillcolor{currentfill}%
\pgfsetfillopacity{0.400000}%
\pgfsetlinewidth{0.501875pt}%
\definecolor{currentstroke}{rgb}{0.000000,0.000000,0.545098}%
\pgfsetstrokecolor{currentstroke}%
\pgfsetstrokeopacity{0.400000}%
\pgfsetdash{}{0pt}%
\pgfpathmoveto{\pgfqpoint{2.113749in}{1.679444in}}%
\pgfpathcurveto{\pgfqpoint{2.124799in}{1.679444in}}{\pgfqpoint{2.135398in}{1.683834in}}{\pgfqpoint{2.143212in}{1.691648in}}%
\pgfpathcurveto{\pgfqpoint{2.151026in}{1.699462in}}{\pgfqpoint{2.155416in}{1.710061in}}{\pgfqpoint{2.155416in}{1.721111in}}%
\pgfpathcurveto{\pgfqpoint{2.155416in}{1.732161in}}{\pgfqpoint{2.151026in}{1.742760in}}{\pgfqpoint{2.143212in}{1.750574in}}%
\pgfpathcurveto{\pgfqpoint{2.135398in}{1.758387in}}{\pgfqpoint{2.124799in}{1.762777in}}{\pgfqpoint{2.113749in}{1.762777in}}%
\pgfpathcurveto{\pgfqpoint{2.102699in}{1.762777in}}{\pgfqpoint{2.092100in}{1.758387in}}{\pgfqpoint{2.084286in}{1.750574in}}%
\pgfpathcurveto{\pgfqpoint{2.076473in}{1.742760in}}{\pgfqpoint{2.072082in}{1.732161in}}{\pgfqpoint{2.072082in}{1.721111in}}%
\pgfpathcurveto{\pgfqpoint{2.072082in}{1.710061in}}{\pgfqpoint{2.076473in}{1.699462in}}{\pgfqpoint{2.084286in}{1.691648in}}%
\pgfpathcurveto{\pgfqpoint{2.092100in}{1.683834in}}{\pgfqpoint{2.102699in}{1.679444in}}{\pgfqpoint{2.113749in}{1.679444in}}%
\pgfpathclose%
\pgfusepath{stroke,fill}%
\end{pgfscope}%
\begin{pgfscope}%
\pgfpathrectangle{\pgfqpoint{0.511823in}{0.504323in}}{\pgfqpoint{3.218177in}{3.225677in}} %
\pgfusepath{clip}%
\pgfsetbuttcap%
\pgfsetroundjoin%
\definecolor{currentfill}{rgb}{0.000000,0.000000,0.545098}%
\pgfsetfillcolor{currentfill}%
\pgfsetfillopacity{0.400000}%
\pgfsetlinewidth{0.501875pt}%
\definecolor{currentstroke}{rgb}{0.000000,0.000000,0.545098}%
\pgfsetstrokecolor{currentstroke}%
\pgfsetstrokeopacity{0.400000}%
\pgfsetdash{}{0pt}%
\pgfpathmoveto{\pgfqpoint{2.185845in}{1.739061in}}%
\pgfpathcurveto{\pgfqpoint{2.196895in}{1.739061in}}{\pgfqpoint{2.207494in}{1.743452in}}{\pgfqpoint{2.215308in}{1.751265in}}%
\pgfpathcurveto{\pgfqpoint{2.223122in}{1.759079in}}{\pgfqpoint{2.227512in}{1.769678in}}{\pgfqpoint{2.227512in}{1.780728in}}%
\pgfpathcurveto{\pgfqpoint{2.227512in}{1.791778in}}{\pgfqpoint{2.223122in}{1.802377in}}{\pgfqpoint{2.215308in}{1.810191in}}%
\pgfpathcurveto{\pgfqpoint{2.207494in}{1.818004in}}{\pgfqpoint{2.196895in}{1.822395in}}{\pgfqpoint{2.185845in}{1.822395in}}%
\pgfpathcurveto{\pgfqpoint{2.174795in}{1.822395in}}{\pgfqpoint{2.164196in}{1.818004in}}{\pgfqpoint{2.156383in}{1.810191in}}%
\pgfpathcurveto{\pgfqpoint{2.148569in}{1.802377in}}{\pgfqpoint{2.144179in}{1.791778in}}{\pgfqpoint{2.144179in}{1.780728in}}%
\pgfpathcurveto{\pgfqpoint{2.144179in}{1.769678in}}{\pgfqpoint{2.148569in}{1.759079in}}{\pgfqpoint{2.156383in}{1.751265in}}%
\pgfpathcurveto{\pgfqpoint{2.164196in}{1.743452in}}{\pgfqpoint{2.174795in}{1.739061in}}{\pgfqpoint{2.185845in}{1.739061in}}%
\pgfpathclose%
\pgfusepath{stroke,fill}%
\end{pgfscope}%
\begin{pgfscope}%
\pgfpathrectangle{\pgfqpoint{0.511823in}{0.504323in}}{\pgfqpoint{3.218177in}{3.225677in}} %
\pgfusepath{clip}%
\pgfsetbuttcap%
\pgfsetroundjoin%
\definecolor{currentfill}{rgb}{0.000000,0.000000,0.545098}%
\pgfsetfillcolor{currentfill}%
\pgfsetfillopacity{0.400000}%
\pgfsetlinewidth{0.501875pt}%
\definecolor{currentstroke}{rgb}{0.000000,0.000000,0.545098}%
\pgfsetstrokecolor{currentstroke}%
\pgfsetstrokeopacity{0.400000}%
\pgfsetdash{}{0pt}%
\pgfpathmoveto{\pgfqpoint{2.300121in}{1.830523in}}%
\pgfpathcurveto{\pgfqpoint{2.311171in}{1.830523in}}{\pgfqpoint{2.321770in}{1.834914in}}{\pgfqpoint{2.329584in}{1.842727in}}%
\pgfpathcurveto{\pgfqpoint{2.337398in}{1.850541in}}{\pgfqpoint{2.341788in}{1.861140in}}{\pgfqpoint{2.341788in}{1.872190in}}%
\pgfpathcurveto{\pgfqpoint{2.341788in}{1.883240in}}{\pgfqpoint{2.337398in}{1.893839in}}{\pgfqpoint{2.329584in}{1.901653in}}%
\pgfpathcurveto{\pgfqpoint{2.321770in}{1.909467in}}{\pgfqpoint{2.311171in}{1.913857in}}{\pgfqpoint{2.300121in}{1.913857in}}%
\pgfpathcurveto{\pgfqpoint{2.289071in}{1.913857in}}{\pgfqpoint{2.278472in}{1.909467in}}{\pgfqpoint{2.270658in}{1.901653in}}%
\pgfpathcurveto{\pgfqpoint{2.262845in}{1.893839in}}{\pgfqpoint{2.258455in}{1.883240in}}{\pgfqpoint{2.258455in}{1.872190in}}%
\pgfpathcurveto{\pgfqpoint{2.258455in}{1.861140in}}{\pgfqpoint{2.262845in}{1.850541in}}{\pgfqpoint{2.270658in}{1.842727in}}%
\pgfpathcurveto{\pgfqpoint{2.278472in}{1.834914in}}{\pgfqpoint{2.289071in}{1.830523in}}{\pgfqpoint{2.300121in}{1.830523in}}%
\pgfpathclose%
\pgfusepath{stroke,fill}%
\end{pgfscope}%
\begin{pgfscope}%
\pgfpathrectangle{\pgfqpoint{0.511823in}{0.504323in}}{\pgfqpoint{3.218177in}{3.225677in}} %
\pgfusepath{clip}%
\pgfsetbuttcap%
\pgfsetroundjoin%
\definecolor{currentfill}{rgb}{0.000000,0.000000,0.545098}%
\pgfsetfillcolor{currentfill}%
\pgfsetfillopacity{0.400000}%
\pgfsetlinewidth{0.501875pt}%
\definecolor{currentstroke}{rgb}{0.000000,0.000000,0.545098}%
\pgfsetstrokecolor{currentstroke}%
\pgfsetstrokeopacity{0.400000}%
\pgfsetdash{}{0pt}%
\pgfpathmoveto{\pgfqpoint{2.205466in}{1.767941in}}%
\pgfpathcurveto{\pgfqpoint{2.216516in}{1.767941in}}{\pgfqpoint{2.227115in}{1.772332in}}{\pgfqpoint{2.234928in}{1.780145in}}%
\pgfpathcurveto{\pgfqpoint{2.242742in}{1.787959in}}{\pgfqpoint{2.247132in}{1.798558in}}{\pgfqpoint{2.247132in}{1.809608in}}%
\pgfpathcurveto{\pgfqpoint{2.247132in}{1.820658in}}{\pgfqpoint{2.242742in}{1.831257in}}{\pgfqpoint{2.234928in}{1.839071in}}%
\pgfpathcurveto{\pgfqpoint{2.227115in}{1.846884in}}{\pgfqpoint{2.216516in}{1.851275in}}{\pgfqpoint{2.205466in}{1.851275in}}%
\pgfpathcurveto{\pgfqpoint{2.194415in}{1.851275in}}{\pgfqpoint{2.183816in}{1.846884in}}{\pgfqpoint{2.176003in}{1.839071in}}%
\pgfpathcurveto{\pgfqpoint{2.168189in}{1.831257in}}{\pgfqpoint{2.163799in}{1.820658in}}{\pgfqpoint{2.163799in}{1.809608in}}%
\pgfpathcurveto{\pgfqpoint{2.163799in}{1.798558in}}{\pgfqpoint{2.168189in}{1.787959in}}{\pgfqpoint{2.176003in}{1.780145in}}%
\pgfpathcurveto{\pgfqpoint{2.183816in}{1.772332in}}{\pgfqpoint{2.194415in}{1.767941in}}{\pgfqpoint{2.205466in}{1.767941in}}%
\pgfpathclose%
\pgfusepath{stroke,fill}%
\end{pgfscope}%
\begin{pgfscope}%
\pgfpathrectangle{\pgfqpoint{0.511823in}{0.504323in}}{\pgfqpoint{3.218177in}{3.225677in}} %
\pgfusepath{clip}%
\pgfsetbuttcap%
\pgfsetroundjoin%
\definecolor{currentfill}{rgb}{0.000000,0.000000,0.545098}%
\pgfsetfillcolor{currentfill}%
\pgfsetfillopacity{0.400000}%
\pgfsetlinewidth{0.501875pt}%
\definecolor{currentstroke}{rgb}{0.000000,0.000000,0.545098}%
\pgfsetstrokecolor{currentstroke}%
\pgfsetstrokeopacity{0.400000}%
\pgfsetdash{}{0pt}%
\pgfpathmoveto{\pgfqpoint{2.192185in}{1.765317in}}%
\pgfpathcurveto{\pgfqpoint{2.203235in}{1.765317in}}{\pgfqpoint{2.213834in}{1.769707in}}{\pgfqpoint{2.221648in}{1.777520in}}%
\pgfpathcurveto{\pgfqpoint{2.229462in}{1.785334in}}{\pgfqpoint{2.233852in}{1.795933in}}{\pgfqpoint{2.233852in}{1.806983in}}%
\pgfpathcurveto{\pgfqpoint{2.233852in}{1.818033in}}{\pgfqpoint{2.229462in}{1.828632in}}{\pgfqpoint{2.221648in}{1.836446in}}%
\pgfpathcurveto{\pgfqpoint{2.213834in}{1.844260in}}{\pgfqpoint{2.203235in}{1.848650in}}{\pgfqpoint{2.192185in}{1.848650in}}%
\pgfpathcurveto{\pgfqpoint{2.181135in}{1.848650in}}{\pgfqpoint{2.170536in}{1.844260in}}{\pgfqpoint{2.162723in}{1.836446in}}%
\pgfpathcurveto{\pgfqpoint{2.154909in}{1.828632in}}{\pgfqpoint{2.150519in}{1.818033in}}{\pgfqpoint{2.150519in}{1.806983in}}%
\pgfpathcurveto{\pgfqpoint{2.150519in}{1.795933in}}{\pgfqpoint{2.154909in}{1.785334in}}{\pgfqpoint{2.162723in}{1.777520in}}%
\pgfpathcurveto{\pgfqpoint{2.170536in}{1.769707in}}{\pgfqpoint{2.181135in}{1.765317in}}{\pgfqpoint{2.192185in}{1.765317in}}%
\pgfpathclose%
\pgfusepath{stroke,fill}%
\end{pgfscope}%
\begin{pgfscope}%
\pgfpathrectangle{\pgfqpoint{0.511823in}{0.504323in}}{\pgfqpoint{3.218177in}{3.225677in}} %
\pgfusepath{clip}%
\pgfsetbuttcap%
\pgfsetroundjoin%
\definecolor{currentfill}{rgb}{0.000000,0.000000,0.545098}%
\pgfsetfillcolor{currentfill}%
\pgfsetfillopacity{0.400000}%
\pgfsetlinewidth{0.501875pt}%
\definecolor{currentstroke}{rgb}{0.000000,0.000000,0.545098}%
\pgfsetstrokecolor{currentstroke}%
\pgfsetstrokeopacity{0.400000}%
\pgfsetdash{}{0pt}%
\pgfpathmoveto{\pgfqpoint{2.118165in}{1.716903in}}%
\pgfpathcurveto{\pgfqpoint{2.129215in}{1.716903in}}{\pgfqpoint{2.139814in}{1.721293in}}{\pgfqpoint{2.147627in}{1.729107in}}%
\pgfpathcurveto{\pgfqpoint{2.155441in}{1.736921in}}{\pgfqpoint{2.159831in}{1.747520in}}{\pgfqpoint{2.159831in}{1.758570in}}%
\pgfpathcurveto{\pgfqpoint{2.159831in}{1.769620in}}{\pgfqpoint{2.155441in}{1.780219in}}{\pgfqpoint{2.147627in}{1.788033in}}%
\pgfpathcurveto{\pgfqpoint{2.139814in}{1.795846in}}{\pgfqpoint{2.129215in}{1.800236in}}{\pgfqpoint{2.118165in}{1.800236in}}%
\pgfpathcurveto{\pgfqpoint{2.107114in}{1.800236in}}{\pgfqpoint{2.096515in}{1.795846in}}{\pgfqpoint{2.088702in}{1.788033in}}%
\pgfpathcurveto{\pgfqpoint{2.080888in}{1.780219in}}{\pgfqpoint{2.076498in}{1.769620in}}{\pgfqpoint{2.076498in}{1.758570in}}%
\pgfpathcurveto{\pgfqpoint{2.076498in}{1.747520in}}{\pgfqpoint{2.080888in}{1.736921in}}{\pgfqpoint{2.088702in}{1.729107in}}%
\pgfpathcurveto{\pgfqpoint{2.096515in}{1.721293in}}{\pgfqpoint{2.107114in}{1.716903in}}{\pgfqpoint{2.118165in}{1.716903in}}%
\pgfpathclose%
\pgfusepath{stroke,fill}%
\end{pgfscope}%
\begin{pgfscope}%
\pgfpathrectangle{\pgfqpoint{0.511823in}{0.504323in}}{\pgfqpoint{3.218177in}{3.225677in}} %
\pgfusepath{clip}%
\pgfsetbuttcap%
\pgfsetroundjoin%
\definecolor{currentfill}{rgb}{0.000000,0.000000,0.545098}%
\pgfsetfillcolor{currentfill}%
\pgfsetfillopacity{0.400000}%
\pgfsetlinewidth{0.501875pt}%
\definecolor{currentstroke}{rgb}{0.000000,0.000000,0.545098}%
\pgfsetstrokecolor{currentstroke}%
\pgfsetstrokeopacity{0.400000}%
\pgfsetdash{}{0pt}%
\pgfpathmoveto{\pgfqpoint{2.381441in}{1.923207in}}%
\pgfpathcurveto{\pgfqpoint{2.392491in}{1.923207in}}{\pgfqpoint{2.403090in}{1.927598in}}{\pgfqpoint{2.410904in}{1.935411in}}%
\pgfpathcurveto{\pgfqpoint{2.418718in}{1.943225in}}{\pgfqpoint{2.423108in}{1.953824in}}{\pgfqpoint{2.423108in}{1.964874in}}%
\pgfpathcurveto{\pgfqpoint{2.423108in}{1.975924in}}{\pgfqpoint{2.418718in}{1.986523in}}{\pgfqpoint{2.410904in}{1.994337in}}%
\pgfpathcurveto{\pgfqpoint{2.403090in}{2.002151in}}{\pgfqpoint{2.392491in}{2.006541in}}{\pgfqpoint{2.381441in}{2.006541in}}%
\pgfpathcurveto{\pgfqpoint{2.370391in}{2.006541in}}{\pgfqpoint{2.359792in}{2.002151in}}{\pgfqpoint{2.351978in}{1.994337in}}%
\pgfpathcurveto{\pgfqpoint{2.344165in}{1.986523in}}{\pgfqpoint{2.339774in}{1.975924in}}{\pgfqpoint{2.339774in}{1.964874in}}%
\pgfpathcurveto{\pgfqpoint{2.339774in}{1.953824in}}{\pgfqpoint{2.344165in}{1.943225in}}{\pgfqpoint{2.351978in}{1.935411in}}%
\pgfpathcurveto{\pgfqpoint{2.359792in}{1.927598in}}{\pgfqpoint{2.370391in}{1.923207in}}{\pgfqpoint{2.381441in}{1.923207in}}%
\pgfpathclose%
\pgfusepath{stroke,fill}%
\end{pgfscope}%
\begin{pgfscope}%
\pgfpathrectangle{\pgfqpoint{0.511823in}{0.504323in}}{\pgfqpoint{3.218177in}{3.225677in}} %
\pgfusepath{clip}%
\pgfsetbuttcap%
\pgfsetroundjoin%
\definecolor{currentfill}{rgb}{0.000000,0.000000,0.545098}%
\pgfsetfillcolor{currentfill}%
\pgfsetfillopacity{0.400000}%
\pgfsetlinewidth{0.501875pt}%
\definecolor{currentstroke}{rgb}{0.000000,0.000000,0.545098}%
\pgfsetstrokecolor{currentstroke}%
\pgfsetstrokeopacity{0.400000}%
\pgfsetdash{}{0pt}%
\pgfpathmoveto{\pgfqpoint{2.161950in}{1.764213in}}%
\pgfpathcurveto{\pgfqpoint{2.173000in}{1.764213in}}{\pgfqpoint{2.183600in}{1.768603in}}{\pgfqpoint{2.191413in}{1.776416in}}%
\pgfpathcurveto{\pgfqpoint{2.199227in}{1.784230in}}{\pgfqpoint{2.203617in}{1.794829in}}{\pgfqpoint{2.203617in}{1.805879in}}%
\pgfpathcurveto{\pgfqpoint{2.203617in}{1.816929in}}{\pgfqpoint{2.199227in}{1.827528in}}{\pgfqpoint{2.191413in}{1.835342in}}%
\pgfpathcurveto{\pgfqpoint{2.183600in}{1.843156in}}{\pgfqpoint{2.173000in}{1.847546in}}{\pgfqpoint{2.161950in}{1.847546in}}%
\pgfpathcurveto{\pgfqpoint{2.150900in}{1.847546in}}{\pgfqpoint{2.140301in}{1.843156in}}{\pgfqpoint{2.132488in}{1.835342in}}%
\pgfpathcurveto{\pgfqpoint{2.124674in}{1.827528in}}{\pgfqpoint{2.120284in}{1.816929in}}{\pgfqpoint{2.120284in}{1.805879in}}%
\pgfpathcurveto{\pgfqpoint{2.120284in}{1.794829in}}{\pgfqpoint{2.124674in}{1.784230in}}{\pgfqpoint{2.132488in}{1.776416in}}%
\pgfpathcurveto{\pgfqpoint{2.140301in}{1.768603in}}{\pgfqpoint{2.150900in}{1.764213in}}{\pgfqpoint{2.161950in}{1.764213in}}%
\pgfpathclose%
\pgfusepath{stroke,fill}%
\end{pgfscope}%
\begin{pgfscope}%
\pgfpathrectangle{\pgfqpoint{0.511823in}{0.504323in}}{\pgfqpoint{3.218177in}{3.225677in}} %
\pgfusepath{clip}%
\pgfsetbuttcap%
\pgfsetroundjoin%
\definecolor{currentfill}{rgb}{0.000000,0.000000,0.545098}%
\pgfsetfillcolor{currentfill}%
\pgfsetfillopacity{0.400000}%
\pgfsetlinewidth{0.501875pt}%
\definecolor{currentstroke}{rgb}{0.000000,0.000000,0.545098}%
\pgfsetstrokecolor{currentstroke}%
\pgfsetstrokeopacity{0.400000}%
\pgfsetdash{}{0pt}%
\pgfpathmoveto{\pgfqpoint{2.111323in}{1.732622in}}%
\pgfpathcurveto{\pgfqpoint{2.122373in}{1.732622in}}{\pgfqpoint{2.132972in}{1.737012in}}{\pgfqpoint{2.140786in}{1.744826in}}%
\pgfpathcurveto{\pgfqpoint{2.148600in}{1.752639in}}{\pgfqpoint{2.152990in}{1.763238in}}{\pgfqpoint{2.152990in}{1.774289in}}%
\pgfpathcurveto{\pgfqpoint{2.152990in}{1.785339in}}{\pgfqpoint{2.148600in}{1.795938in}}{\pgfqpoint{2.140786in}{1.803751in}}%
\pgfpathcurveto{\pgfqpoint{2.132972in}{1.811565in}}{\pgfqpoint{2.122373in}{1.815955in}}{\pgfqpoint{2.111323in}{1.815955in}}%
\pgfpathcurveto{\pgfqpoint{2.100273in}{1.815955in}}{\pgfqpoint{2.089674in}{1.811565in}}{\pgfqpoint{2.081860in}{1.803751in}}%
\pgfpathcurveto{\pgfqpoint{2.074047in}{1.795938in}}{\pgfqpoint{2.069656in}{1.785339in}}{\pgfqpoint{2.069656in}{1.774289in}}%
\pgfpathcurveto{\pgfqpoint{2.069656in}{1.763238in}}{\pgfqpoint{2.074047in}{1.752639in}}{\pgfqpoint{2.081860in}{1.744826in}}%
\pgfpathcurveto{\pgfqpoint{2.089674in}{1.737012in}}{\pgfqpoint{2.100273in}{1.732622in}}{\pgfqpoint{2.111323in}{1.732622in}}%
\pgfpathclose%
\pgfusepath{stroke,fill}%
\end{pgfscope}%
\begin{pgfscope}%
\pgfpathrectangle{\pgfqpoint{0.511823in}{0.504323in}}{\pgfqpoint{3.218177in}{3.225677in}} %
\pgfusepath{clip}%
\pgfsetbuttcap%
\pgfsetroundjoin%
\definecolor{currentfill}{rgb}{0.000000,0.000000,0.545098}%
\pgfsetfillcolor{currentfill}%
\pgfsetfillopacity{0.400000}%
\pgfsetlinewidth{0.501875pt}%
\definecolor{currentstroke}{rgb}{0.000000,0.000000,0.545098}%
\pgfsetstrokecolor{currentstroke}%
\pgfsetstrokeopacity{0.400000}%
\pgfsetdash{}{0pt}%
\pgfpathmoveto{\pgfqpoint{2.205449in}{1.812307in}}%
\pgfpathcurveto{\pgfqpoint{2.216500in}{1.812307in}}{\pgfqpoint{2.227099in}{1.816697in}}{\pgfqpoint{2.234912in}{1.824511in}}%
\pgfpathcurveto{\pgfqpoint{2.242726in}{1.832324in}}{\pgfqpoint{2.247116in}{1.842923in}}{\pgfqpoint{2.247116in}{1.853973in}}%
\pgfpathcurveto{\pgfqpoint{2.247116in}{1.865024in}}{\pgfqpoint{2.242726in}{1.875623in}}{\pgfqpoint{2.234912in}{1.883436in}}%
\pgfpathcurveto{\pgfqpoint{2.227099in}{1.891250in}}{\pgfqpoint{2.216500in}{1.895640in}}{\pgfqpoint{2.205449in}{1.895640in}}%
\pgfpathcurveto{\pgfqpoint{2.194399in}{1.895640in}}{\pgfqpoint{2.183800in}{1.891250in}}{\pgfqpoint{2.175987in}{1.883436in}}%
\pgfpathcurveto{\pgfqpoint{2.168173in}{1.875623in}}{\pgfqpoint{2.163783in}{1.865024in}}{\pgfqpoint{2.163783in}{1.853973in}}%
\pgfpathcurveto{\pgfqpoint{2.163783in}{1.842923in}}{\pgfqpoint{2.168173in}{1.832324in}}{\pgfqpoint{2.175987in}{1.824511in}}%
\pgfpathcurveto{\pgfqpoint{2.183800in}{1.816697in}}{\pgfqpoint{2.194399in}{1.812307in}}{\pgfqpoint{2.205449in}{1.812307in}}%
\pgfpathclose%
\pgfusepath{stroke,fill}%
\end{pgfscope}%
\begin{pgfscope}%
\pgfpathrectangle{\pgfqpoint{0.511823in}{0.504323in}}{\pgfqpoint{3.218177in}{3.225677in}} %
\pgfusepath{clip}%
\pgfsetbuttcap%
\pgfsetroundjoin%
\definecolor{currentfill}{rgb}{0.000000,0.000000,0.545098}%
\pgfsetfillcolor{currentfill}%
\pgfsetfillopacity{0.400000}%
\pgfsetlinewidth{0.501875pt}%
\definecolor{currentstroke}{rgb}{0.000000,0.000000,0.545098}%
\pgfsetstrokecolor{currentstroke}%
\pgfsetstrokeopacity{0.400000}%
\pgfsetdash{}{0pt}%
\pgfpathmoveto{\pgfqpoint{2.131697in}{1.762541in}}%
\pgfpathcurveto{\pgfqpoint{2.142747in}{1.762541in}}{\pgfqpoint{2.153346in}{1.766932in}}{\pgfqpoint{2.161160in}{1.774745in}}%
\pgfpathcurveto{\pgfqpoint{2.168973in}{1.782559in}}{\pgfqpoint{2.173364in}{1.793158in}}{\pgfqpoint{2.173364in}{1.804208in}}%
\pgfpathcurveto{\pgfqpoint{2.173364in}{1.815258in}}{\pgfqpoint{2.168973in}{1.825857in}}{\pgfqpoint{2.161160in}{1.833671in}}%
\pgfpathcurveto{\pgfqpoint{2.153346in}{1.841484in}}{\pgfqpoint{2.142747in}{1.845875in}}{\pgfqpoint{2.131697in}{1.845875in}}%
\pgfpathcurveto{\pgfqpoint{2.120647in}{1.845875in}}{\pgfqpoint{2.110048in}{1.841484in}}{\pgfqpoint{2.102234in}{1.833671in}}%
\pgfpathcurveto{\pgfqpoint{2.094420in}{1.825857in}}{\pgfqpoint{2.090030in}{1.815258in}}{\pgfqpoint{2.090030in}{1.804208in}}%
\pgfpathcurveto{\pgfqpoint{2.090030in}{1.793158in}}{\pgfqpoint{2.094420in}{1.782559in}}{\pgfqpoint{2.102234in}{1.774745in}}%
\pgfpathcurveto{\pgfqpoint{2.110048in}{1.766932in}}{\pgfqpoint{2.120647in}{1.762541in}}{\pgfqpoint{2.131697in}{1.762541in}}%
\pgfpathclose%
\pgfusepath{stroke,fill}%
\end{pgfscope}%
\begin{pgfscope}%
\pgfpathrectangle{\pgfqpoint{0.511823in}{0.504323in}}{\pgfqpoint{3.218177in}{3.225677in}} %
\pgfusepath{clip}%
\pgfsetbuttcap%
\pgfsetroundjoin%
\definecolor{currentfill}{rgb}{0.000000,0.000000,0.545098}%
\pgfsetfillcolor{currentfill}%
\pgfsetfillopacity{0.400000}%
\pgfsetlinewidth{0.501875pt}%
\definecolor{currentstroke}{rgb}{0.000000,0.000000,0.545098}%
\pgfsetstrokecolor{currentstroke}%
\pgfsetstrokeopacity{0.400000}%
\pgfsetdash{}{0pt}%
\pgfpathmoveto{\pgfqpoint{2.186311in}{1.812445in}}%
\pgfpathcurveto{\pgfqpoint{2.197361in}{1.812445in}}{\pgfqpoint{2.207960in}{1.816835in}}{\pgfqpoint{2.215773in}{1.824648in}}%
\pgfpathcurveto{\pgfqpoint{2.223587in}{1.832462in}}{\pgfqpoint{2.227977in}{1.843061in}}{\pgfqpoint{2.227977in}{1.854111in}}%
\pgfpathcurveto{\pgfqpoint{2.227977in}{1.865161in}}{\pgfqpoint{2.223587in}{1.875760in}}{\pgfqpoint{2.215773in}{1.883574in}}%
\pgfpathcurveto{\pgfqpoint{2.207960in}{1.891388in}}{\pgfqpoint{2.197361in}{1.895778in}}{\pgfqpoint{2.186311in}{1.895778in}}%
\pgfpathcurveto{\pgfqpoint{2.175260in}{1.895778in}}{\pgfqpoint{2.164661in}{1.891388in}}{\pgfqpoint{2.156848in}{1.883574in}}%
\pgfpathcurveto{\pgfqpoint{2.149034in}{1.875760in}}{\pgfqpoint{2.144644in}{1.865161in}}{\pgfqpoint{2.144644in}{1.854111in}}%
\pgfpathcurveto{\pgfqpoint{2.144644in}{1.843061in}}{\pgfqpoint{2.149034in}{1.832462in}}{\pgfqpoint{2.156848in}{1.824648in}}%
\pgfpathcurveto{\pgfqpoint{2.164661in}{1.816835in}}{\pgfqpoint{2.175260in}{1.812445in}}{\pgfqpoint{2.186311in}{1.812445in}}%
\pgfpathclose%
\pgfusepath{stroke,fill}%
\end{pgfscope}%
\begin{pgfscope}%
\pgfpathrectangle{\pgfqpoint{0.511823in}{0.504323in}}{\pgfqpoint{3.218177in}{3.225677in}} %
\pgfusepath{clip}%
\pgfsetbuttcap%
\pgfsetroundjoin%
\definecolor{currentfill}{rgb}{0.000000,0.000000,0.545098}%
\pgfsetfillcolor{currentfill}%
\pgfsetfillopacity{0.400000}%
\pgfsetlinewidth{0.501875pt}%
\definecolor{currentstroke}{rgb}{0.000000,0.000000,0.545098}%
\pgfsetstrokecolor{currentstroke}%
\pgfsetstrokeopacity{0.400000}%
\pgfsetdash{}{0pt}%
\pgfpathmoveto{\pgfqpoint{2.341140in}{1.941789in}}%
\pgfpathcurveto{\pgfqpoint{2.352190in}{1.941789in}}{\pgfqpoint{2.362789in}{1.946179in}}{\pgfqpoint{2.370602in}{1.953993in}}%
\pgfpathcurveto{\pgfqpoint{2.378416in}{1.961807in}}{\pgfqpoint{2.382806in}{1.972406in}}{\pgfqpoint{2.382806in}{1.983456in}}%
\pgfpathcurveto{\pgfqpoint{2.382806in}{1.994506in}}{\pgfqpoint{2.378416in}{2.005105in}}{\pgfqpoint{2.370602in}{2.012919in}}%
\pgfpathcurveto{\pgfqpoint{2.362789in}{2.020732in}}{\pgfqpoint{2.352190in}{2.025123in}}{\pgfqpoint{2.341140in}{2.025123in}}%
\pgfpathcurveto{\pgfqpoint{2.330090in}{2.025123in}}{\pgfqpoint{2.319491in}{2.020732in}}{\pgfqpoint{2.311677in}{2.012919in}}%
\pgfpathcurveto{\pgfqpoint{2.303863in}{2.005105in}}{\pgfqpoint{2.299473in}{1.994506in}}{\pgfqpoint{2.299473in}{1.983456in}}%
\pgfpathcurveto{\pgfqpoint{2.299473in}{1.972406in}}{\pgfqpoint{2.303863in}{1.961807in}}{\pgfqpoint{2.311677in}{1.953993in}}%
\pgfpathcurveto{\pgfqpoint{2.319491in}{1.946179in}}{\pgfqpoint{2.330090in}{1.941789in}}{\pgfqpoint{2.341140in}{1.941789in}}%
\pgfpathclose%
\pgfusepath{stroke,fill}%
\end{pgfscope}%
\begin{pgfscope}%
\pgfpathrectangle{\pgfqpoint{0.511823in}{0.504323in}}{\pgfqpoint{3.218177in}{3.225677in}} %
\pgfusepath{clip}%
\pgfsetbuttcap%
\pgfsetroundjoin%
\definecolor{currentfill}{rgb}{0.000000,0.000000,0.545098}%
\pgfsetfillcolor{currentfill}%
\pgfsetfillopacity{0.400000}%
\pgfsetlinewidth{0.501875pt}%
\definecolor{currentstroke}{rgb}{0.000000,0.000000,0.545098}%
\pgfsetstrokecolor{currentstroke}%
\pgfsetstrokeopacity{0.400000}%
\pgfsetdash{}{0pt}%
\pgfpathmoveto{\pgfqpoint{2.156737in}{1.804095in}}%
\pgfpathcurveto{\pgfqpoint{2.167787in}{1.804095in}}{\pgfqpoint{2.178386in}{1.808485in}}{\pgfqpoint{2.186199in}{1.816299in}}%
\pgfpathcurveto{\pgfqpoint{2.194013in}{1.824112in}}{\pgfqpoint{2.198403in}{1.834711in}}{\pgfqpoint{2.198403in}{1.845761in}}%
\pgfpathcurveto{\pgfqpoint{2.198403in}{1.856811in}}{\pgfqpoint{2.194013in}{1.867411in}}{\pgfqpoint{2.186199in}{1.875224in}}%
\pgfpathcurveto{\pgfqpoint{2.178386in}{1.883038in}}{\pgfqpoint{2.167787in}{1.887428in}}{\pgfqpoint{2.156737in}{1.887428in}}%
\pgfpathcurveto{\pgfqpoint{2.145686in}{1.887428in}}{\pgfqpoint{2.135087in}{1.883038in}}{\pgfqpoint{2.127274in}{1.875224in}}%
\pgfpathcurveto{\pgfqpoint{2.119460in}{1.867411in}}{\pgfqpoint{2.115070in}{1.856811in}}{\pgfqpoint{2.115070in}{1.845761in}}%
\pgfpathcurveto{\pgfqpoint{2.115070in}{1.834711in}}{\pgfqpoint{2.119460in}{1.824112in}}{\pgfqpoint{2.127274in}{1.816299in}}%
\pgfpathcurveto{\pgfqpoint{2.135087in}{1.808485in}}{\pgfqpoint{2.145686in}{1.804095in}}{\pgfqpoint{2.156737in}{1.804095in}}%
\pgfpathclose%
\pgfusepath{stroke,fill}%
\end{pgfscope}%
\begin{pgfscope}%
\pgfpathrectangle{\pgfqpoint{0.511823in}{0.504323in}}{\pgfqpoint{3.218177in}{3.225677in}} %
\pgfusepath{clip}%
\pgfsetbuttcap%
\pgfsetroundjoin%
\definecolor{currentfill}{rgb}{0.000000,0.000000,0.545098}%
\pgfsetfillcolor{currentfill}%
\pgfsetfillopacity{0.400000}%
\pgfsetlinewidth{0.501875pt}%
\definecolor{currentstroke}{rgb}{0.000000,0.000000,0.545098}%
\pgfsetstrokecolor{currentstroke}%
\pgfsetstrokeopacity{0.400000}%
\pgfsetdash{}{0pt}%
\pgfpathmoveto{\pgfqpoint{2.192757in}{1.840247in}}%
\pgfpathcurveto{\pgfqpoint{2.203807in}{1.840247in}}{\pgfqpoint{2.214406in}{1.844638in}}{\pgfqpoint{2.222219in}{1.852451in}}%
\pgfpathcurveto{\pgfqpoint{2.230033in}{1.860265in}}{\pgfqpoint{2.234423in}{1.870864in}}{\pgfqpoint{2.234423in}{1.881914in}}%
\pgfpathcurveto{\pgfqpoint{2.234423in}{1.892964in}}{\pgfqpoint{2.230033in}{1.903563in}}{\pgfqpoint{2.222219in}{1.911377in}}%
\pgfpathcurveto{\pgfqpoint{2.214406in}{1.919190in}}{\pgfqpoint{2.203807in}{1.923581in}}{\pgfqpoint{2.192757in}{1.923581in}}%
\pgfpathcurveto{\pgfqpoint{2.181707in}{1.923581in}}{\pgfqpoint{2.171107in}{1.919190in}}{\pgfqpoint{2.163294in}{1.911377in}}%
\pgfpathcurveto{\pgfqpoint{2.155480in}{1.903563in}}{\pgfqpoint{2.151090in}{1.892964in}}{\pgfqpoint{2.151090in}{1.881914in}}%
\pgfpathcurveto{\pgfqpoint{2.151090in}{1.870864in}}{\pgfqpoint{2.155480in}{1.860265in}}{\pgfqpoint{2.163294in}{1.852451in}}%
\pgfpathcurveto{\pgfqpoint{2.171107in}{1.844638in}}{\pgfqpoint{2.181707in}{1.840247in}}{\pgfqpoint{2.192757in}{1.840247in}}%
\pgfpathclose%
\pgfusepath{stroke,fill}%
\end{pgfscope}%
\begin{pgfscope}%
\pgfpathrectangle{\pgfqpoint{0.511823in}{0.504323in}}{\pgfqpoint{3.218177in}{3.225677in}} %
\pgfusepath{clip}%
\pgfsetbuttcap%
\pgfsetroundjoin%
\definecolor{currentfill}{rgb}{0.000000,0.000000,0.545098}%
\pgfsetfillcolor{currentfill}%
\pgfsetfillopacity{0.400000}%
\pgfsetlinewidth{0.501875pt}%
\definecolor{currentstroke}{rgb}{0.000000,0.000000,0.545098}%
\pgfsetstrokecolor{currentstroke}%
\pgfsetstrokeopacity{0.400000}%
\pgfsetdash{}{0pt}%
\pgfpathmoveto{\pgfqpoint{2.156658in}{1.818964in}}%
\pgfpathcurveto{\pgfqpoint{2.167708in}{1.818964in}}{\pgfqpoint{2.178307in}{1.823354in}}{\pgfqpoint{2.186121in}{1.831168in}}%
\pgfpathcurveto{\pgfqpoint{2.193934in}{1.838981in}}{\pgfqpoint{2.198325in}{1.849580in}}{\pgfqpoint{2.198325in}{1.860630in}}%
\pgfpathcurveto{\pgfqpoint{2.198325in}{1.871680in}}{\pgfqpoint{2.193934in}{1.882280in}}{\pgfqpoint{2.186121in}{1.890093in}}%
\pgfpathcurveto{\pgfqpoint{2.178307in}{1.897907in}}{\pgfqpoint{2.167708in}{1.902297in}}{\pgfqpoint{2.156658in}{1.902297in}}%
\pgfpathcurveto{\pgfqpoint{2.145608in}{1.902297in}}{\pgfqpoint{2.135009in}{1.897907in}}{\pgfqpoint{2.127195in}{1.890093in}}%
\pgfpathcurveto{\pgfqpoint{2.119382in}{1.882280in}}{\pgfqpoint{2.114991in}{1.871680in}}{\pgfqpoint{2.114991in}{1.860630in}}%
\pgfpathcurveto{\pgfqpoint{2.114991in}{1.849580in}}{\pgfqpoint{2.119382in}{1.838981in}}{\pgfqpoint{2.127195in}{1.831168in}}%
\pgfpathcurveto{\pgfqpoint{2.135009in}{1.823354in}}{\pgfqpoint{2.145608in}{1.818964in}}{\pgfqpoint{2.156658in}{1.818964in}}%
\pgfpathclose%
\pgfusepath{stroke,fill}%
\end{pgfscope}%
\begin{pgfscope}%
\pgfpathrectangle{\pgfqpoint{0.511823in}{0.504323in}}{\pgfqpoint{3.218177in}{3.225677in}} %
\pgfusepath{clip}%
\pgfsetbuttcap%
\pgfsetroundjoin%
\definecolor{currentfill}{rgb}{0.000000,0.000000,0.545098}%
\pgfsetfillcolor{currentfill}%
\pgfsetfillopacity{0.400000}%
\pgfsetlinewidth{0.501875pt}%
\definecolor{currentstroke}{rgb}{0.000000,0.000000,0.545098}%
\pgfsetstrokecolor{currentstroke}%
\pgfsetstrokeopacity{0.400000}%
\pgfsetdash{}{0pt}%
\pgfpathmoveto{\pgfqpoint{2.170379in}{1.837565in}}%
\pgfpathcurveto{\pgfqpoint{2.181429in}{1.837565in}}{\pgfqpoint{2.192028in}{1.841956in}}{\pgfqpoint{2.199842in}{1.849769in}}%
\pgfpathcurveto{\pgfqpoint{2.207655in}{1.857583in}}{\pgfqpoint{2.212046in}{1.868182in}}{\pgfqpoint{2.212046in}{1.879232in}}%
\pgfpathcurveto{\pgfqpoint{2.212046in}{1.890282in}}{\pgfqpoint{2.207655in}{1.900881in}}{\pgfqpoint{2.199842in}{1.908695in}}%
\pgfpathcurveto{\pgfqpoint{2.192028in}{1.916508in}}{\pgfqpoint{2.181429in}{1.920899in}}{\pgfqpoint{2.170379in}{1.920899in}}%
\pgfpathcurveto{\pgfqpoint{2.159329in}{1.920899in}}{\pgfqpoint{2.148730in}{1.916508in}}{\pgfqpoint{2.140916in}{1.908695in}}%
\pgfpathcurveto{\pgfqpoint{2.133103in}{1.900881in}}{\pgfqpoint{2.128712in}{1.890282in}}{\pgfqpoint{2.128712in}{1.879232in}}%
\pgfpathcurveto{\pgfqpoint{2.128712in}{1.868182in}}{\pgfqpoint{2.133103in}{1.857583in}}{\pgfqpoint{2.140916in}{1.849769in}}%
\pgfpathcurveto{\pgfqpoint{2.148730in}{1.841956in}}{\pgfqpoint{2.159329in}{1.837565in}}{\pgfqpoint{2.170379in}{1.837565in}}%
\pgfpathclose%
\pgfusepath{stroke,fill}%
\end{pgfscope}%
\begin{pgfscope}%
\pgfpathrectangle{\pgfqpoint{0.511823in}{0.504323in}}{\pgfqpoint{3.218177in}{3.225677in}} %
\pgfusepath{clip}%
\pgfsetbuttcap%
\pgfsetroundjoin%
\definecolor{currentfill}{rgb}{0.000000,0.000000,0.545098}%
\pgfsetfillcolor{currentfill}%
\pgfsetfillopacity{0.400000}%
\pgfsetlinewidth{0.501875pt}%
\definecolor{currentstroke}{rgb}{0.000000,0.000000,0.545098}%
\pgfsetstrokecolor{currentstroke}%
\pgfsetstrokeopacity{0.400000}%
\pgfsetdash{}{0pt}%
\pgfpathmoveto{\pgfqpoint{2.267397in}{1.924014in}}%
\pgfpathcurveto{\pgfqpoint{2.278447in}{1.924014in}}{\pgfqpoint{2.289046in}{1.928404in}}{\pgfqpoint{2.296859in}{1.936217in}}%
\pgfpathcurveto{\pgfqpoint{2.304673in}{1.944031in}}{\pgfqpoint{2.309063in}{1.954630in}}{\pgfqpoint{2.309063in}{1.965680in}}%
\pgfpathcurveto{\pgfqpoint{2.309063in}{1.976730in}}{\pgfqpoint{2.304673in}{1.987329in}}{\pgfqpoint{2.296859in}{1.995143in}}%
\pgfpathcurveto{\pgfqpoint{2.289046in}{2.002957in}}{\pgfqpoint{2.278447in}{2.007347in}}{\pgfqpoint{2.267397in}{2.007347in}}%
\pgfpathcurveto{\pgfqpoint{2.256347in}{2.007347in}}{\pgfqpoint{2.245747in}{2.002957in}}{\pgfqpoint{2.237934in}{1.995143in}}%
\pgfpathcurveto{\pgfqpoint{2.230120in}{1.987329in}}{\pgfqpoint{2.225730in}{1.976730in}}{\pgfqpoint{2.225730in}{1.965680in}}%
\pgfpathcurveto{\pgfqpoint{2.225730in}{1.954630in}}{\pgfqpoint{2.230120in}{1.944031in}}{\pgfqpoint{2.237934in}{1.936217in}}%
\pgfpathcurveto{\pgfqpoint{2.245747in}{1.928404in}}{\pgfqpoint{2.256347in}{1.924014in}}{\pgfqpoint{2.267397in}{1.924014in}}%
\pgfpathclose%
\pgfusepath{stroke,fill}%
\end{pgfscope}%
\begin{pgfscope}%
\pgfpathrectangle{\pgfqpoint{0.511823in}{0.504323in}}{\pgfqpoint{3.218177in}{3.225677in}} %
\pgfusepath{clip}%
\pgfsetbuttcap%
\pgfsetroundjoin%
\definecolor{currentfill}{rgb}{0.000000,0.000000,0.545098}%
\pgfsetfillcolor{currentfill}%
\pgfsetfillopacity{0.400000}%
\pgfsetlinewidth{0.501875pt}%
\definecolor{currentstroke}{rgb}{0.000000,0.000000,0.545098}%
\pgfsetstrokecolor{currentstroke}%
\pgfsetstrokeopacity{0.400000}%
\pgfsetdash{}{0pt}%
\pgfpathmoveto{\pgfqpoint{2.234337in}{1.905170in}}%
\pgfpathcurveto{\pgfqpoint{2.245387in}{1.905170in}}{\pgfqpoint{2.255986in}{1.909560in}}{\pgfqpoint{2.263800in}{1.917374in}}%
\pgfpathcurveto{\pgfqpoint{2.271613in}{1.925187in}}{\pgfqpoint{2.276004in}{1.935786in}}{\pgfqpoint{2.276004in}{1.946836in}}%
\pgfpathcurveto{\pgfqpoint{2.276004in}{1.957887in}}{\pgfqpoint{2.271613in}{1.968486in}}{\pgfqpoint{2.263800in}{1.976299in}}%
\pgfpathcurveto{\pgfqpoint{2.255986in}{1.984113in}}{\pgfqpoint{2.245387in}{1.988503in}}{\pgfqpoint{2.234337in}{1.988503in}}%
\pgfpathcurveto{\pgfqpoint{2.223287in}{1.988503in}}{\pgfqpoint{2.212688in}{1.984113in}}{\pgfqpoint{2.204874in}{1.976299in}}%
\pgfpathcurveto{\pgfqpoint{2.197061in}{1.968486in}}{\pgfqpoint{2.192670in}{1.957887in}}{\pgfqpoint{2.192670in}{1.946836in}}%
\pgfpathcurveto{\pgfqpoint{2.192670in}{1.935786in}}{\pgfqpoint{2.197061in}{1.925187in}}{\pgfqpoint{2.204874in}{1.917374in}}%
\pgfpathcurveto{\pgfqpoint{2.212688in}{1.909560in}}{\pgfqpoint{2.223287in}{1.905170in}}{\pgfqpoint{2.234337in}{1.905170in}}%
\pgfpathclose%
\pgfusepath{stroke,fill}%
\end{pgfscope}%
\begin{pgfscope}%
\pgfpathrectangle{\pgfqpoint{0.511823in}{0.504323in}}{\pgfqpoint{3.218177in}{3.225677in}} %
\pgfusepath{clip}%
\pgfsetbuttcap%
\pgfsetroundjoin%
\definecolor{currentfill}{rgb}{0.000000,0.000000,0.545098}%
\pgfsetfillcolor{currentfill}%
\pgfsetfillopacity{0.400000}%
\pgfsetlinewidth{0.501875pt}%
\definecolor{currentstroke}{rgb}{0.000000,0.000000,0.545098}%
\pgfsetstrokecolor{currentstroke}%
\pgfsetstrokeopacity{0.400000}%
\pgfsetdash{}{0pt}%
\pgfpathmoveto{\pgfqpoint{2.143261in}{1.838297in}}%
\pgfpathcurveto{\pgfqpoint{2.154311in}{1.838297in}}{\pgfqpoint{2.164910in}{1.842687in}}{\pgfqpoint{2.172723in}{1.850500in}}%
\pgfpathcurveto{\pgfqpoint{2.180537in}{1.858314in}}{\pgfqpoint{2.184927in}{1.868913in}}{\pgfqpoint{2.184927in}{1.879963in}}%
\pgfpathcurveto{\pgfqpoint{2.184927in}{1.891013in}}{\pgfqpoint{2.180537in}{1.901612in}}{\pgfqpoint{2.172723in}{1.909426in}}%
\pgfpathcurveto{\pgfqpoint{2.164910in}{1.917240in}}{\pgfqpoint{2.154311in}{1.921630in}}{\pgfqpoint{2.143261in}{1.921630in}}%
\pgfpathcurveto{\pgfqpoint{2.132210in}{1.921630in}}{\pgfqpoint{2.121611in}{1.917240in}}{\pgfqpoint{2.113798in}{1.909426in}}%
\pgfpathcurveto{\pgfqpoint{2.105984in}{1.901612in}}{\pgfqpoint{2.101594in}{1.891013in}}{\pgfqpoint{2.101594in}{1.879963in}}%
\pgfpathcurveto{\pgfqpoint{2.101594in}{1.868913in}}{\pgfqpoint{2.105984in}{1.858314in}}{\pgfqpoint{2.113798in}{1.850500in}}%
\pgfpathcurveto{\pgfqpoint{2.121611in}{1.842687in}}{\pgfqpoint{2.132210in}{1.838297in}}{\pgfqpoint{2.143261in}{1.838297in}}%
\pgfpathclose%
\pgfusepath{stroke,fill}%
\end{pgfscope}%
\begin{pgfscope}%
\pgfpathrectangle{\pgfqpoint{0.511823in}{0.504323in}}{\pgfqpoint{3.218177in}{3.225677in}} %
\pgfusepath{clip}%
\pgfsetbuttcap%
\pgfsetroundjoin%
\definecolor{currentfill}{rgb}{0.000000,0.000000,0.545098}%
\pgfsetfillcolor{currentfill}%
\pgfsetfillopacity{0.400000}%
\pgfsetlinewidth{0.501875pt}%
\definecolor{currentstroke}{rgb}{0.000000,0.000000,0.545098}%
\pgfsetstrokecolor{currentstroke}%
\pgfsetstrokeopacity{0.400000}%
\pgfsetdash{}{0pt}%
\pgfpathmoveto{\pgfqpoint{2.080796in}{1.794204in}}%
\pgfpathcurveto{\pgfqpoint{2.091846in}{1.794204in}}{\pgfqpoint{2.102445in}{1.798594in}}{\pgfqpoint{2.110259in}{1.806408in}}%
\pgfpathcurveto{\pgfqpoint{2.118072in}{1.814221in}}{\pgfqpoint{2.122462in}{1.824820in}}{\pgfqpoint{2.122462in}{1.835870in}}%
\pgfpathcurveto{\pgfqpoint{2.122462in}{1.846920in}}{\pgfqpoint{2.118072in}{1.857520in}}{\pgfqpoint{2.110259in}{1.865333in}}%
\pgfpathcurveto{\pgfqpoint{2.102445in}{1.873147in}}{\pgfqpoint{2.091846in}{1.877537in}}{\pgfqpoint{2.080796in}{1.877537in}}%
\pgfpathcurveto{\pgfqpoint{2.069746in}{1.877537in}}{\pgfqpoint{2.059147in}{1.873147in}}{\pgfqpoint{2.051333in}{1.865333in}}%
\pgfpathcurveto{\pgfqpoint{2.043519in}{1.857520in}}{\pgfqpoint{2.039129in}{1.846920in}}{\pgfqpoint{2.039129in}{1.835870in}}%
\pgfpathcurveto{\pgfqpoint{2.039129in}{1.824820in}}{\pgfqpoint{2.043519in}{1.814221in}}{\pgfqpoint{2.051333in}{1.806408in}}%
\pgfpathcurveto{\pgfqpoint{2.059147in}{1.798594in}}{\pgfqpoint{2.069746in}{1.794204in}}{\pgfqpoint{2.080796in}{1.794204in}}%
\pgfpathclose%
\pgfusepath{stroke,fill}%
\end{pgfscope}%
\begin{pgfscope}%
\pgfpathrectangle{\pgfqpoint{0.511823in}{0.504323in}}{\pgfqpoint{3.218177in}{3.225677in}} %
\pgfusepath{clip}%
\pgfsetbuttcap%
\pgfsetroundjoin%
\definecolor{currentfill}{rgb}{0.000000,0.000000,0.545098}%
\pgfsetfillcolor{currentfill}%
\pgfsetfillopacity{0.400000}%
\pgfsetlinewidth{0.501875pt}%
\definecolor{currentstroke}{rgb}{0.000000,0.000000,0.545098}%
\pgfsetstrokecolor{currentstroke}%
\pgfsetstrokeopacity{0.400000}%
\pgfsetdash{}{0pt}%
\pgfpathmoveto{\pgfqpoint{2.155619in}{1.863885in}}%
\pgfpathcurveto{\pgfqpoint{2.166669in}{1.863885in}}{\pgfqpoint{2.177268in}{1.868275in}}{\pgfqpoint{2.185082in}{1.876089in}}%
\pgfpathcurveto{\pgfqpoint{2.192896in}{1.883902in}}{\pgfqpoint{2.197286in}{1.894501in}}{\pgfqpoint{2.197286in}{1.905552in}}%
\pgfpathcurveto{\pgfqpoint{2.197286in}{1.916602in}}{\pgfqpoint{2.192896in}{1.927201in}}{\pgfqpoint{2.185082in}{1.935014in}}%
\pgfpathcurveto{\pgfqpoint{2.177268in}{1.942828in}}{\pgfqpoint{2.166669in}{1.947218in}}{\pgfqpoint{2.155619in}{1.947218in}}%
\pgfpathcurveto{\pgfqpoint{2.144569in}{1.947218in}}{\pgfqpoint{2.133970in}{1.942828in}}{\pgfqpoint{2.126156in}{1.935014in}}%
\pgfpathcurveto{\pgfqpoint{2.118343in}{1.927201in}}{\pgfqpoint{2.113952in}{1.916602in}}{\pgfqpoint{2.113952in}{1.905552in}}%
\pgfpathcurveto{\pgfqpoint{2.113952in}{1.894501in}}{\pgfqpoint{2.118343in}{1.883902in}}{\pgfqpoint{2.126156in}{1.876089in}}%
\pgfpathcurveto{\pgfqpoint{2.133970in}{1.868275in}}{\pgfqpoint{2.144569in}{1.863885in}}{\pgfqpoint{2.155619in}{1.863885in}}%
\pgfpathclose%
\pgfusepath{stroke,fill}%
\end{pgfscope}%
\begin{pgfscope}%
\pgfpathrectangle{\pgfqpoint{0.511823in}{0.504323in}}{\pgfqpoint{3.218177in}{3.225677in}} %
\pgfusepath{clip}%
\pgfsetbuttcap%
\pgfsetroundjoin%
\definecolor{currentfill}{rgb}{0.000000,0.000000,0.545098}%
\pgfsetfillcolor{currentfill}%
\pgfsetfillopacity{0.400000}%
\pgfsetlinewidth{0.501875pt}%
\definecolor{currentstroke}{rgb}{0.000000,0.000000,0.545098}%
\pgfsetstrokecolor{currentstroke}%
\pgfsetstrokeopacity{0.400000}%
\pgfsetdash{}{0pt}%
\pgfpathmoveto{\pgfqpoint{2.246750in}{1.948074in}}%
\pgfpathcurveto{\pgfqpoint{2.257800in}{1.948074in}}{\pgfqpoint{2.268399in}{1.952464in}}{\pgfqpoint{2.276213in}{1.960277in}}%
\pgfpathcurveto{\pgfqpoint{2.284027in}{1.968091in}}{\pgfqpoint{2.288417in}{1.978690in}}{\pgfqpoint{2.288417in}{1.989740in}}%
\pgfpathcurveto{\pgfqpoint{2.288417in}{2.000790in}}{\pgfqpoint{2.284027in}{2.011389in}}{\pgfqpoint{2.276213in}{2.019203in}}%
\pgfpathcurveto{\pgfqpoint{2.268399in}{2.027017in}}{\pgfqpoint{2.257800in}{2.031407in}}{\pgfqpoint{2.246750in}{2.031407in}}%
\pgfpathcurveto{\pgfqpoint{2.235700in}{2.031407in}}{\pgfqpoint{2.225101in}{2.027017in}}{\pgfqpoint{2.217287in}{2.019203in}}%
\pgfpathcurveto{\pgfqpoint{2.209474in}{2.011389in}}{\pgfqpoint{2.205084in}{2.000790in}}{\pgfqpoint{2.205084in}{1.989740in}}%
\pgfpathcurveto{\pgfqpoint{2.205084in}{1.978690in}}{\pgfqpoint{2.209474in}{1.968091in}}{\pgfqpoint{2.217287in}{1.960277in}}%
\pgfpathcurveto{\pgfqpoint{2.225101in}{1.952464in}}{\pgfqpoint{2.235700in}{1.948074in}}{\pgfqpoint{2.246750in}{1.948074in}}%
\pgfpathclose%
\pgfusepath{stroke,fill}%
\end{pgfscope}%
\begin{pgfscope}%
\pgfpathrectangle{\pgfqpoint{0.511823in}{0.504323in}}{\pgfqpoint{3.218177in}{3.225677in}} %
\pgfusepath{clip}%
\pgfsetbuttcap%
\pgfsetroundjoin%
\definecolor{currentfill}{rgb}{0.000000,0.000000,0.545098}%
\pgfsetfillcolor{currentfill}%
\pgfsetfillopacity{0.400000}%
\pgfsetlinewidth{0.501875pt}%
\definecolor{currentstroke}{rgb}{0.000000,0.000000,0.545098}%
\pgfsetstrokecolor{currentstroke}%
\pgfsetstrokeopacity{0.400000}%
\pgfsetdash{}{0pt}%
\pgfpathmoveto{\pgfqpoint{2.158792in}{1.882158in}}%
\pgfpathcurveto{\pgfqpoint{2.169842in}{1.882158in}}{\pgfqpoint{2.180441in}{1.886548in}}{\pgfqpoint{2.188254in}{1.894362in}}%
\pgfpathcurveto{\pgfqpoint{2.196068in}{1.902175in}}{\pgfqpoint{2.200458in}{1.912774in}}{\pgfqpoint{2.200458in}{1.923824in}}%
\pgfpathcurveto{\pgfqpoint{2.200458in}{1.934874in}}{\pgfqpoint{2.196068in}{1.945473in}}{\pgfqpoint{2.188254in}{1.953287in}}%
\pgfpathcurveto{\pgfqpoint{2.180441in}{1.961101in}}{\pgfqpoint{2.169842in}{1.965491in}}{\pgfqpoint{2.158792in}{1.965491in}}%
\pgfpathcurveto{\pgfqpoint{2.147741in}{1.965491in}}{\pgfqpoint{2.137142in}{1.961101in}}{\pgfqpoint{2.129329in}{1.953287in}}%
\pgfpathcurveto{\pgfqpoint{2.121515in}{1.945473in}}{\pgfqpoint{2.117125in}{1.934874in}}{\pgfqpoint{2.117125in}{1.923824in}}%
\pgfpathcurveto{\pgfqpoint{2.117125in}{1.912774in}}{\pgfqpoint{2.121515in}{1.902175in}}{\pgfqpoint{2.129329in}{1.894362in}}%
\pgfpathcurveto{\pgfqpoint{2.137142in}{1.886548in}}{\pgfqpoint{2.147741in}{1.882158in}}{\pgfqpoint{2.158792in}{1.882158in}}%
\pgfpathclose%
\pgfusepath{stroke,fill}%
\end{pgfscope}%
\begin{pgfscope}%
\pgfpathrectangle{\pgfqpoint{0.511823in}{0.504323in}}{\pgfqpoint{3.218177in}{3.225677in}} %
\pgfusepath{clip}%
\pgfsetbuttcap%
\pgfsetroundjoin%
\definecolor{currentfill}{rgb}{0.000000,0.000000,0.545098}%
\pgfsetfillcolor{currentfill}%
\pgfsetfillopacity{0.400000}%
\pgfsetlinewidth{0.501875pt}%
\definecolor{currentstroke}{rgb}{0.000000,0.000000,0.545098}%
\pgfsetstrokecolor{currentstroke}%
\pgfsetstrokeopacity{0.400000}%
\pgfsetdash{}{0pt}%
\pgfpathmoveto{\pgfqpoint{2.143772in}{1.877285in}}%
\pgfpathcurveto{\pgfqpoint{2.154822in}{1.877285in}}{\pgfqpoint{2.165421in}{1.881676in}}{\pgfqpoint{2.173235in}{1.889489in}}%
\pgfpathcurveto{\pgfqpoint{2.181049in}{1.897303in}}{\pgfqpoint{2.185439in}{1.907902in}}{\pgfqpoint{2.185439in}{1.918952in}}%
\pgfpathcurveto{\pgfqpoint{2.185439in}{1.930002in}}{\pgfqpoint{2.181049in}{1.940601in}}{\pgfqpoint{2.173235in}{1.948415in}}%
\pgfpathcurveto{\pgfqpoint{2.165421in}{1.956229in}}{\pgfqpoint{2.154822in}{1.960619in}}{\pgfqpoint{2.143772in}{1.960619in}}%
\pgfpathcurveto{\pgfqpoint{2.132722in}{1.960619in}}{\pgfqpoint{2.122123in}{1.956229in}}{\pgfqpoint{2.114309in}{1.948415in}}%
\pgfpathcurveto{\pgfqpoint{2.106496in}{1.940601in}}{\pgfqpoint{2.102106in}{1.930002in}}{\pgfqpoint{2.102106in}{1.918952in}}%
\pgfpathcurveto{\pgfqpoint{2.102106in}{1.907902in}}{\pgfqpoint{2.106496in}{1.897303in}}{\pgfqpoint{2.114309in}{1.889489in}}%
\pgfpathcurveto{\pgfqpoint{2.122123in}{1.881676in}}{\pgfqpoint{2.132722in}{1.877285in}}{\pgfqpoint{2.143772in}{1.877285in}}%
\pgfpathclose%
\pgfusepath{stroke,fill}%
\end{pgfscope}%
\begin{pgfscope}%
\pgfpathrectangle{\pgfqpoint{0.511823in}{0.504323in}}{\pgfqpoint{3.218177in}{3.225677in}} %
\pgfusepath{clip}%
\pgfsetbuttcap%
\pgfsetroundjoin%
\definecolor{currentfill}{rgb}{0.000000,0.000000,0.545098}%
\pgfsetfillcolor{currentfill}%
\pgfsetfillopacity{0.400000}%
\pgfsetlinewidth{0.501875pt}%
\definecolor{currentstroke}{rgb}{0.000000,0.000000,0.545098}%
\pgfsetstrokecolor{currentstroke}%
\pgfsetstrokeopacity{0.400000}%
\pgfsetdash{}{0pt}%
\pgfpathmoveto{\pgfqpoint{1.975404in}{1.741252in}}%
\pgfpathcurveto{\pgfqpoint{1.986455in}{1.741252in}}{\pgfqpoint{1.997054in}{1.745643in}}{\pgfqpoint{2.004867in}{1.753456in}}%
\pgfpathcurveto{\pgfqpoint{2.012681in}{1.761270in}}{\pgfqpoint{2.017071in}{1.771869in}}{\pgfqpoint{2.017071in}{1.782919in}}%
\pgfpathcurveto{\pgfqpoint{2.017071in}{1.793969in}}{\pgfqpoint{2.012681in}{1.804568in}}{\pgfqpoint{2.004867in}{1.812382in}}%
\pgfpathcurveto{\pgfqpoint{1.997054in}{1.820195in}}{\pgfqpoint{1.986455in}{1.824586in}}{\pgfqpoint{1.975404in}{1.824586in}}%
\pgfpathcurveto{\pgfqpoint{1.964354in}{1.824586in}}{\pgfqpoint{1.953755in}{1.820195in}}{\pgfqpoint{1.945942in}{1.812382in}}%
\pgfpathcurveto{\pgfqpoint{1.938128in}{1.804568in}}{\pgfqpoint{1.933738in}{1.793969in}}{\pgfqpoint{1.933738in}{1.782919in}}%
\pgfpathcurveto{\pgfqpoint{1.933738in}{1.771869in}}{\pgfqpoint{1.938128in}{1.761270in}}{\pgfqpoint{1.945942in}{1.753456in}}%
\pgfpathcurveto{\pgfqpoint{1.953755in}{1.745643in}}{\pgfqpoint{1.964354in}{1.741252in}}{\pgfqpoint{1.975404in}{1.741252in}}%
\pgfpathclose%
\pgfusepath{stroke,fill}%
\end{pgfscope}%
\begin{pgfscope}%
\pgfpathrectangle{\pgfqpoint{0.511823in}{0.504323in}}{\pgfqpoint{3.218177in}{3.225677in}} %
\pgfusepath{clip}%
\pgfsetbuttcap%
\pgfsetroundjoin%
\definecolor{currentfill}{rgb}{0.000000,0.000000,0.545098}%
\pgfsetfillcolor{currentfill}%
\pgfsetfillopacity{0.400000}%
\pgfsetlinewidth{0.501875pt}%
\definecolor{currentstroke}{rgb}{0.000000,0.000000,0.545098}%
\pgfsetstrokecolor{currentstroke}%
\pgfsetstrokeopacity{0.400000}%
\pgfsetdash{}{0pt}%
\pgfpathmoveto{\pgfqpoint{2.244644in}{1.979770in}}%
\pgfpathcurveto{\pgfqpoint{2.255694in}{1.979770in}}{\pgfqpoint{2.266293in}{1.984160in}}{\pgfqpoint{2.274107in}{1.991974in}}%
\pgfpathcurveto{\pgfqpoint{2.281920in}{1.999788in}}{\pgfqpoint{2.286311in}{2.010387in}}{\pgfqpoint{2.286311in}{2.021437in}}%
\pgfpathcurveto{\pgfqpoint{2.286311in}{2.032487in}}{\pgfqpoint{2.281920in}{2.043086in}}{\pgfqpoint{2.274107in}{2.050900in}}%
\pgfpathcurveto{\pgfqpoint{2.266293in}{2.058713in}}{\pgfqpoint{2.255694in}{2.063104in}}{\pgfqpoint{2.244644in}{2.063104in}}%
\pgfpathcurveto{\pgfqpoint{2.233594in}{2.063104in}}{\pgfqpoint{2.222995in}{2.058713in}}{\pgfqpoint{2.215181in}{2.050900in}}%
\pgfpathcurveto{\pgfqpoint{2.207368in}{2.043086in}}{\pgfqpoint{2.202977in}{2.032487in}}{\pgfqpoint{2.202977in}{2.021437in}}%
\pgfpathcurveto{\pgfqpoint{2.202977in}{2.010387in}}{\pgfqpoint{2.207368in}{1.999788in}}{\pgfqpoint{2.215181in}{1.991974in}}%
\pgfpathcurveto{\pgfqpoint{2.222995in}{1.984160in}}{\pgfqpoint{2.233594in}{1.979770in}}{\pgfqpoint{2.244644in}{1.979770in}}%
\pgfpathclose%
\pgfusepath{stroke,fill}%
\end{pgfscope}%
\begin{pgfscope}%
\pgfpathrectangle{\pgfqpoint{0.511823in}{0.504323in}}{\pgfqpoint{3.218177in}{3.225677in}} %
\pgfusepath{clip}%
\pgfsetbuttcap%
\pgfsetroundjoin%
\definecolor{currentfill}{rgb}{0.000000,0.000000,0.545098}%
\pgfsetfillcolor{currentfill}%
\pgfsetfillopacity{0.400000}%
\pgfsetlinewidth{0.501875pt}%
\definecolor{currentstroke}{rgb}{0.000000,0.000000,0.545098}%
\pgfsetstrokecolor{currentstroke}%
\pgfsetstrokeopacity{0.400000}%
\pgfsetdash{}{0pt}%
\pgfpathmoveto{\pgfqpoint{1.998699in}{1.775413in}}%
\pgfpathcurveto{\pgfqpoint{2.009749in}{1.775413in}}{\pgfqpoint{2.020348in}{1.779803in}}{\pgfqpoint{2.028162in}{1.787616in}}%
\pgfpathcurveto{\pgfqpoint{2.035975in}{1.795430in}}{\pgfqpoint{2.040366in}{1.806029in}}{\pgfqpoint{2.040366in}{1.817079in}}%
\pgfpathcurveto{\pgfqpoint{2.040366in}{1.828129in}}{\pgfqpoint{2.035975in}{1.838728in}}{\pgfqpoint{2.028162in}{1.846542in}}%
\pgfpathcurveto{\pgfqpoint{2.020348in}{1.854356in}}{\pgfqpoint{2.009749in}{1.858746in}}{\pgfqpoint{1.998699in}{1.858746in}}%
\pgfpathcurveto{\pgfqpoint{1.987649in}{1.858746in}}{\pgfqpoint{1.977050in}{1.854356in}}{\pgfqpoint{1.969236in}{1.846542in}}%
\pgfpathcurveto{\pgfqpoint{1.961422in}{1.838728in}}{\pgfqpoint{1.957032in}{1.828129in}}{\pgfqpoint{1.957032in}{1.817079in}}%
\pgfpathcurveto{\pgfqpoint{1.957032in}{1.806029in}}{\pgfqpoint{1.961422in}{1.795430in}}{\pgfqpoint{1.969236in}{1.787616in}}%
\pgfpathcurveto{\pgfqpoint{1.977050in}{1.779803in}}{\pgfqpoint{1.987649in}{1.775413in}}{\pgfqpoint{1.998699in}{1.775413in}}%
\pgfpathclose%
\pgfusepath{stroke,fill}%
\end{pgfscope}%
\begin{pgfscope}%
\pgfpathrectangle{\pgfqpoint{0.511823in}{0.504323in}}{\pgfqpoint{3.218177in}{3.225677in}} %
\pgfusepath{clip}%
\pgfsetbuttcap%
\pgfsetroundjoin%
\definecolor{currentfill}{rgb}{0.000000,0.000000,0.545098}%
\pgfsetfillcolor{currentfill}%
\pgfsetfillopacity{0.400000}%
\pgfsetlinewidth{0.501875pt}%
\definecolor{currentstroke}{rgb}{0.000000,0.000000,0.545098}%
\pgfsetstrokecolor{currentstroke}%
\pgfsetstrokeopacity{0.400000}%
\pgfsetdash{}{0pt}%
\pgfpathmoveto{\pgfqpoint{2.015095in}{1.796882in}}%
\pgfpathcurveto{\pgfqpoint{2.026145in}{1.796882in}}{\pgfqpoint{2.036744in}{1.801272in}}{\pgfqpoint{2.044558in}{1.809086in}}%
\pgfpathcurveto{\pgfqpoint{2.052372in}{1.816899in}}{\pgfqpoint{2.056762in}{1.827498in}}{\pgfqpoint{2.056762in}{1.838548in}}%
\pgfpathcurveto{\pgfqpoint{2.056762in}{1.849599in}}{\pgfqpoint{2.052372in}{1.860198in}}{\pgfqpoint{2.044558in}{1.868011in}}%
\pgfpathcurveto{\pgfqpoint{2.036744in}{1.875825in}}{\pgfqpoint{2.026145in}{1.880215in}}{\pgfqpoint{2.015095in}{1.880215in}}%
\pgfpathcurveto{\pgfqpoint{2.004045in}{1.880215in}}{\pgfqpoint{1.993446in}{1.875825in}}{\pgfqpoint{1.985632in}{1.868011in}}%
\pgfpathcurveto{\pgfqpoint{1.977819in}{1.860198in}}{\pgfqpoint{1.973429in}{1.849599in}}{\pgfqpoint{1.973429in}{1.838548in}}%
\pgfpathcurveto{\pgfqpoint{1.973429in}{1.827498in}}{\pgfqpoint{1.977819in}{1.816899in}}{\pgfqpoint{1.985632in}{1.809086in}}%
\pgfpathcurveto{\pgfqpoint{1.993446in}{1.801272in}}{\pgfqpoint{2.004045in}{1.796882in}}{\pgfqpoint{2.015095in}{1.796882in}}%
\pgfpathclose%
\pgfusepath{stroke,fill}%
\end{pgfscope}%
\begin{pgfscope}%
\pgfpathrectangle{\pgfqpoint{0.511823in}{0.504323in}}{\pgfqpoint{3.218177in}{3.225677in}} %
\pgfusepath{clip}%
\pgfsetbuttcap%
\pgfsetroundjoin%
\definecolor{currentfill}{rgb}{0.000000,0.000000,0.545098}%
\pgfsetfillcolor{currentfill}%
\pgfsetfillopacity{0.400000}%
\pgfsetlinewidth{0.501875pt}%
\definecolor{currentstroke}{rgb}{0.000000,0.000000,0.545098}%
\pgfsetstrokecolor{currentstroke}%
\pgfsetstrokeopacity{0.400000}%
\pgfsetdash{}{0pt}%
\pgfpathmoveto{\pgfqpoint{2.098705in}{1.877480in}}%
\pgfpathcurveto{\pgfqpoint{2.109755in}{1.877480in}}{\pgfqpoint{2.120354in}{1.881871in}}{\pgfqpoint{2.128168in}{1.889684in}}%
\pgfpathcurveto{\pgfqpoint{2.135981in}{1.897498in}}{\pgfqpoint{2.140371in}{1.908097in}}{\pgfqpoint{2.140371in}{1.919147in}}%
\pgfpathcurveto{\pgfqpoint{2.140371in}{1.930197in}}{\pgfqpoint{2.135981in}{1.940796in}}{\pgfqpoint{2.128168in}{1.948610in}}%
\pgfpathcurveto{\pgfqpoint{2.120354in}{1.956423in}}{\pgfqpoint{2.109755in}{1.960814in}}{\pgfqpoint{2.098705in}{1.960814in}}%
\pgfpathcurveto{\pgfqpoint{2.087655in}{1.960814in}}{\pgfqpoint{2.077056in}{1.956423in}}{\pgfqpoint{2.069242in}{1.948610in}}%
\pgfpathcurveto{\pgfqpoint{2.061428in}{1.940796in}}{\pgfqpoint{2.057038in}{1.930197in}}{\pgfqpoint{2.057038in}{1.919147in}}%
\pgfpathcurveto{\pgfqpoint{2.057038in}{1.908097in}}{\pgfqpoint{2.061428in}{1.897498in}}{\pgfqpoint{2.069242in}{1.889684in}}%
\pgfpathcurveto{\pgfqpoint{2.077056in}{1.881871in}}{\pgfqpoint{2.087655in}{1.877480in}}{\pgfqpoint{2.098705in}{1.877480in}}%
\pgfpathclose%
\pgfusepath{stroke,fill}%
\end{pgfscope}%
\begin{pgfscope}%
\pgfpathrectangle{\pgfqpoint{0.511823in}{0.504323in}}{\pgfqpoint{3.218177in}{3.225677in}} %
\pgfusepath{clip}%
\pgfsetbuttcap%
\pgfsetroundjoin%
\definecolor{currentfill}{rgb}{0.000000,0.000000,0.545098}%
\pgfsetfillcolor{currentfill}%
\pgfsetfillopacity{0.400000}%
\pgfsetlinewidth{0.501875pt}%
\definecolor{currentstroke}{rgb}{0.000000,0.000000,0.545098}%
\pgfsetstrokecolor{currentstroke}%
\pgfsetstrokeopacity{0.400000}%
\pgfsetdash{}{0pt}%
\pgfpathmoveto{\pgfqpoint{2.134020in}{1.916456in}}%
\pgfpathcurveto{\pgfqpoint{2.145070in}{1.916456in}}{\pgfqpoint{2.155669in}{1.920847in}}{\pgfqpoint{2.163483in}{1.928660in}}%
\pgfpathcurveto{\pgfqpoint{2.171297in}{1.936474in}}{\pgfqpoint{2.175687in}{1.947073in}}{\pgfqpoint{2.175687in}{1.958123in}}%
\pgfpathcurveto{\pgfqpoint{2.175687in}{1.969173in}}{\pgfqpoint{2.171297in}{1.979772in}}{\pgfqpoint{2.163483in}{1.987586in}}%
\pgfpathcurveto{\pgfqpoint{2.155669in}{1.995399in}}{\pgfqpoint{2.145070in}{1.999790in}}{\pgfqpoint{2.134020in}{1.999790in}}%
\pgfpathcurveto{\pgfqpoint{2.122970in}{1.999790in}}{\pgfqpoint{2.112371in}{1.995399in}}{\pgfqpoint{2.104558in}{1.987586in}}%
\pgfpathcurveto{\pgfqpoint{2.096744in}{1.979772in}}{\pgfqpoint{2.092354in}{1.969173in}}{\pgfqpoint{2.092354in}{1.958123in}}%
\pgfpathcurveto{\pgfqpoint{2.092354in}{1.947073in}}{\pgfqpoint{2.096744in}{1.936474in}}{\pgfqpoint{2.104558in}{1.928660in}}%
\pgfpathcurveto{\pgfqpoint{2.112371in}{1.920847in}}{\pgfqpoint{2.122970in}{1.916456in}}{\pgfqpoint{2.134020in}{1.916456in}}%
\pgfpathclose%
\pgfusepath{stroke,fill}%
\end{pgfscope}%
\begin{pgfscope}%
\pgfpathrectangle{\pgfqpoint{0.511823in}{0.504323in}}{\pgfqpoint{3.218177in}{3.225677in}} %
\pgfusepath{clip}%
\pgfsetbuttcap%
\pgfsetroundjoin%
\definecolor{currentfill}{rgb}{0.000000,0.000000,0.545098}%
\pgfsetfillcolor{currentfill}%
\pgfsetfillopacity{0.400000}%
\pgfsetlinewidth{0.501875pt}%
\definecolor{currentstroke}{rgb}{0.000000,0.000000,0.545098}%
\pgfsetstrokecolor{currentstroke}%
\pgfsetstrokeopacity{0.400000}%
\pgfsetdash{}{0pt}%
\pgfpathmoveto{\pgfqpoint{2.130695in}{1.921579in}}%
\pgfpathcurveto{\pgfqpoint{2.141746in}{1.921579in}}{\pgfqpoint{2.152345in}{1.925969in}}{\pgfqpoint{2.160158in}{1.933783in}}%
\pgfpathcurveto{\pgfqpoint{2.167972in}{1.941596in}}{\pgfqpoint{2.172362in}{1.952196in}}{\pgfqpoint{2.172362in}{1.963246in}}%
\pgfpathcurveto{\pgfqpoint{2.172362in}{1.974296in}}{\pgfqpoint{2.167972in}{1.984895in}}{\pgfqpoint{2.160158in}{1.992708in}}%
\pgfpathcurveto{\pgfqpoint{2.152345in}{2.000522in}}{\pgfqpoint{2.141746in}{2.004912in}}{\pgfqpoint{2.130695in}{2.004912in}}%
\pgfpathcurveto{\pgfqpoint{2.119645in}{2.004912in}}{\pgfqpoint{2.109046in}{2.000522in}}{\pgfqpoint{2.101233in}{1.992708in}}%
\pgfpathcurveto{\pgfqpoint{2.093419in}{1.984895in}}{\pgfqpoint{2.089029in}{1.974296in}}{\pgfqpoint{2.089029in}{1.963246in}}%
\pgfpathcurveto{\pgfqpoint{2.089029in}{1.952196in}}{\pgfqpoint{2.093419in}{1.941596in}}{\pgfqpoint{2.101233in}{1.933783in}}%
\pgfpathcurveto{\pgfqpoint{2.109046in}{1.925969in}}{\pgfqpoint{2.119645in}{1.921579in}}{\pgfqpoint{2.130695in}{1.921579in}}%
\pgfpathclose%
\pgfusepath{stroke,fill}%
\end{pgfscope}%
\begin{pgfscope}%
\pgfpathrectangle{\pgfqpoint{0.511823in}{0.504323in}}{\pgfqpoint{3.218177in}{3.225677in}} %
\pgfusepath{clip}%
\pgfsetbuttcap%
\pgfsetroundjoin%
\definecolor{currentfill}{rgb}{0.000000,0.000000,0.545098}%
\pgfsetfillcolor{currentfill}%
\pgfsetfillopacity{0.400000}%
\pgfsetlinewidth{0.501875pt}%
\definecolor{currentstroke}{rgb}{0.000000,0.000000,0.545098}%
\pgfsetstrokecolor{currentstroke}%
\pgfsetstrokeopacity{0.400000}%
\pgfsetdash{}{0pt}%
\pgfpathmoveto{\pgfqpoint{2.048942in}{1.856665in}}%
\pgfpathcurveto{\pgfqpoint{2.059992in}{1.856665in}}{\pgfqpoint{2.070591in}{1.861055in}}{\pgfqpoint{2.078405in}{1.868869in}}%
\pgfpathcurveto{\pgfqpoint{2.086218in}{1.876683in}}{\pgfqpoint{2.090609in}{1.887282in}}{\pgfqpoint{2.090609in}{1.898332in}}%
\pgfpathcurveto{\pgfqpoint{2.090609in}{1.909382in}}{\pgfqpoint{2.086218in}{1.919981in}}{\pgfqpoint{2.078405in}{1.927795in}}%
\pgfpathcurveto{\pgfqpoint{2.070591in}{1.935608in}}{\pgfqpoint{2.059992in}{1.939998in}}{\pgfqpoint{2.048942in}{1.939998in}}%
\pgfpathcurveto{\pgfqpoint{2.037892in}{1.939998in}}{\pgfqpoint{2.027293in}{1.935608in}}{\pgfqpoint{2.019479in}{1.927795in}}%
\pgfpathcurveto{\pgfqpoint{2.011665in}{1.919981in}}{\pgfqpoint{2.007275in}{1.909382in}}{\pgfqpoint{2.007275in}{1.898332in}}%
\pgfpathcurveto{\pgfqpoint{2.007275in}{1.887282in}}{\pgfqpoint{2.011665in}{1.876683in}}{\pgfqpoint{2.019479in}{1.868869in}}%
\pgfpathcurveto{\pgfqpoint{2.027293in}{1.861055in}}{\pgfqpoint{2.037892in}{1.856665in}}{\pgfqpoint{2.048942in}{1.856665in}}%
\pgfpathclose%
\pgfusepath{stroke,fill}%
\end{pgfscope}%
\begin{pgfscope}%
\pgfpathrectangle{\pgfqpoint{0.511823in}{0.504323in}}{\pgfqpoint{3.218177in}{3.225677in}} %
\pgfusepath{clip}%
\pgfsetbuttcap%
\pgfsetroundjoin%
\definecolor{currentfill}{rgb}{0.000000,0.000000,0.545098}%
\pgfsetfillcolor{currentfill}%
\pgfsetfillopacity{0.400000}%
\pgfsetlinewidth{0.501875pt}%
\definecolor{currentstroke}{rgb}{0.000000,0.000000,0.545098}%
\pgfsetstrokecolor{currentstroke}%
\pgfsetstrokeopacity{0.400000}%
\pgfsetdash{}{0pt}%
\pgfpathmoveto{\pgfqpoint{2.087814in}{1.899294in}}%
\pgfpathcurveto{\pgfqpoint{2.098864in}{1.899294in}}{\pgfqpoint{2.109463in}{1.903684in}}{\pgfqpoint{2.117277in}{1.911497in}}%
\pgfpathcurveto{\pgfqpoint{2.125090in}{1.919311in}}{\pgfqpoint{2.129480in}{1.929910in}}{\pgfqpoint{2.129480in}{1.940960in}}%
\pgfpathcurveto{\pgfqpoint{2.129480in}{1.952010in}}{\pgfqpoint{2.125090in}{1.962609in}}{\pgfqpoint{2.117277in}{1.970423in}}%
\pgfpathcurveto{\pgfqpoint{2.109463in}{1.978237in}}{\pgfqpoint{2.098864in}{1.982627in}}{\pgfqpoint{2.087814in}{1.982627in}}%
\pgfpathcurveto{\pgfqpoint{2.076764in}{1.982627in}}{\pgfqpoint{2.066165in}{1.978237in}}{\pgfqpoint{2.058351in}{1.970423in}}%
\pgfpathcurveto{\pgfqpoint{2.050537in}{1.962609in}}{\pgfqpoint{2.046147in}{1.952010in}}{\pgfqpoint{2.046147in}{1.940960in}}%
\pgfpathcurveto{\pgfqpoint{2.046147in}{1.929910in}}{\pgfqpoint{2.050537in}{1.919311in}}{\pgfqpoint{2.058351in}{1.911497in}}%
\pgfpathcurveto{\pgfqpoint{2.066165in}{1.903684in}}{\pgfqpoint{2.076764in}{1.899294in}}{\pgfqpoint{2.087814in}{1.899294in}}%
\pgfpathclose%
\pgfusepath{stroke,fill}%
\end{pgfscope}%
\begin{pgfscope}%
\pgfpathrectangle{\pgfqpoint{0.511823in}{0.504323in}}{\pgfqpoint{3.218177in}{3.225677in}} %
\pgfusepath{clip}%
\pgfsetbuttcap%
\pgfsetroundjoin%
\definecolor{currentfill}{rgb}{0.000000,0.000000,0.545098}%
\pgfsetfillcolor{currentfill}%
\pgfsetfillopacity{0.400000}%
\pgfsetlinewidth{0.501875pt}%
\definecolor{currentstroke}{rgb}{0.000000,0.000000,0.545098}%
\pgfsetstrokecolor{currentstroke}%
\pgfsetstrokeopacity{0.400000}%
\pgfsetdash{}{0pt}%
\pgfpathmoveto{\pgfqpoint{2.055593in}{1.878109in}}%
\pgfpathcurveto{\pgfqpoint{2.066643in}{1.878109in}}{\pgfqpoint{2.077242in}{1.882500in}}{\pgfqpoint{2.085055in}{1.890313in}}%
\pgfpathcurveto{\pgfqpoint{2.092869in}{1.898127in}}{\pgfqpoint{2.097259in}{1.908726in}}{\pgfqpoint{2.097259in}{1.919776in}}%
\pgfpathcurveto{\pgfqpoint{2.097259in}{1.930826in}}{\pgfqpoint{2.092869in}{1.941425in}}{\pgfqpoint{2.085055in}{1.949239in}}%
\pgfpathcurveto{\pgfqpoint{2.077242in}{1.957053in}}{\pgfqpoint{2.066643in}{1.961443in}}{\pgfqpoint{2.055593in}{1.961443in}}%
\pgfpathcurveto{\pgfqpoint{2.044543in}{1.961443in}}{\pgfqpoint{2.033944in}{1.957053in}}{\pgfqpoint{2.026130in}{1.949239in}}%
\pgfpathcurveto{\pgfqpoint{2.018316in}{1.941425in}}{\pgfqpoint{2.013926in}{1.930826in}}{\pgfqpoint{2.013926in}{1.919776in}}%
\pgfpathcurveto{\pgfqpoint{2.013926in}{1.908726in}}{\pgfqpoint{2.018316in}{1.898127in}}{\pgfqpoint{2.026130in}{1.890313in}}%
\pgfpathcurveto{\pgfqpoint{2.033944in}{1.882500in}}{\pgfqpoint{2.044543in}{1.878109in}}{\pgfqpoint{2.055593in}{1.878109in}}%
\pgfpathclose%
\pgfusepath{stroke,fill}%
\end{pgfscope}%
\begin{pgfscope}%
\pgfpathrectangle{\pgfqpoint{0.511823in}{0.504323in}}{\pgfqpoint{3.218177in}{3.225677in}} %
\pgfusepath{clip}%
\pgfsetbuttcap%
\pgfsetroundjoin%
\definecolor{currentfill}{rgb}{0.000000,0.000000,0.545098}%
\pgfsetfillcolor{currentfill}%
\pgfsetfillopacity{0.400000}%
\pgfsetlinewidth{0.501875pt}%
\definecolor{currentstroke}{rgb}{0.000000,0.000000,0.545098}%
\pgfsetstrokecolor{currentstroke}%
\pgfsetstrokeopacity{0.400000}%
\pgfsetdash{}{0pt}%
\pgfpathmoveto{\pgfqpoint{1.937352in}{1.778313in}}%
\pgfpathcurveto{\pgfqpoint{1.948402in}{1.778313in}}{\pgfqpoint{1.959001in}{1.782703in}}{\pgfqpoint{1.966815in}{1.790517in}}%
\pgfpathcurveto{\pgfqpoint{1.974628in}{1.798330in}}{\pgfqpoint{1.979018in}{1.808929in}}{\pgfqpoint{1.979018in}{1.819979in}}%
\pgfpathcurveto{\pgfqpoint{1.979018in}{1.831030in}}{\pgfqpoint{1.974628in}{1.841629in}}{\pgfqpoint{1.966815in}{1.849442in}}%
\pgfpathcurveto{\pgfqpoint{1.959001in}{1.857256in}}{\pgfqpoint{1.948402in}{1.861646in}}{\pgfqpoint{1.937352in}{1.861646in}}%
\pgfpathcurveto{\pgfqpoint{1.926302in}{1.861646in}}{\pgfqpoint{1.915703in}{1.857256in}}{\pgfqpoint{1.907889in}{1.849442in}}%
\pgfpathcurveto{\pgfqpoint{1.900075in}{1.841629in}}{\pgfqpoint{1.895685in}{1.831030in}}{\pgfqpoint{1.895685in}{1.819979in}}%
\pgfpathcurveto{\pgfqpoint{1.895685in}{1.808929in}}{\pgfqpoint{1.900075in}{1.798330in}}{\pgfqpoint{1.907889in}{1.790517in}}%
\pgfpathcurveto{\pgfqpoint{1.915703in}{1.782703in}}{\pgfqpoint{1.926302in}{1.778313in}}{\pgfqpoint{1.937352in}{1.778313in}}%
\pgfpathclose%
\pgfusepath{stroke,fill}%
\end{pgfscope}%
\begin{pgfscope}%
\pgfpathrectangle{\pgfqpoint{0.511823in}{0.504323in}}{\pgfqpoint{3.218177in}{3.225677in}} %
\pgfusepath{clip}%
\pgfsetbuttcap%
\pgfsetroundjoin%
\definecolor{currentfill}{rgb}{0.000000,0.000000,0.545098}%
\pgfsetfillcolor{currentfill}%
\pgfsetfillopacity{0.400000}%
\pgfsetlinewidth{0.501875pt}%
\definecolor{currentstroke}{rgb}{0.000000,0.000000,0.545098}%
\pgfsetstrokecolor{currentstroke}%
\pgfsetstrokeopacity{0.400000}%
\pgfsetdash{}{0pt}%
\pgfpathmoveto{\pgfqpoint{2.080415in}{1.916541in}}%
\pgfpathcurveto{\pgfqpoint{2.091466in}{1.916541in}}{\pgfqpoint{2.102065in}{1.920931in}}{\pgfqpoint{2.109878in}{1.928744in}}%
\pgfpathcurveto{\pgfqpoint{2.117692in}{1.936558in}}{\pgfqpoint{2.122082in}{1.947157in}}{\pgfqpoint{2.122082in}{1.958207in}}%
\pgfpathcurveto{\pgfqpoint{2.122082in}{1.969257in}}{\pgfqpoint{2.117692in}{1.979856in}}{\pgfqpoint{2.109878in}{1.987670in}}%
\pgfpathcurveto{\pgfqpoint{2.102065in}{1.995484in}}{\pgfqpoint{2.091466in}{1.999874in}}{\pgfqpoint{2.080415in}{1.999874in}}%
\pgfpathcurveto{\pgfqpoint{2.069365in}{1.999874in}}{\pgfqpoint{2.058766in}{1.995484in}}{\pgfqpoint{2.050953in}{1.987670in}}%
\pgfpathcurveto{\pgfqpoint{2.043139in}{1.979856in}}{\pgfqpoint{2.038749in}{1.969257in}}{\pgfqpoint{2.038749in}{1.958207in}}%
\pgfpathcurveto{\pgfqpoint{2.038749in}{1.947157in}}{\pgfqpoint{2.043139in}{1.936558in}}{\pgfqpoint{2.050953in}{1.928744in}}%
\pgfpathcurveto{\pgfqpoint{2.058766in}{1.920931in}}{\pgfqpoint{2.069365in}{1.916541in}}{\pgfqpoint{2.080415in}{1.916541in}}%
\pgfpathclose%
\pgfusepath{stroke,fill}%
\end{pgfscope}%
\begin{pgfscope}%
\pgfpathrectangle{\pgfqpoint{0.511823in}{0.504323in}}{\pgfqpoint{3.218177in}{3.225677in}} %
\pgfusepath{clip}%
\pgfsetbuttcap%
\pgfsetroundjoin%
\definecolor{currentfill}{rgb}{0.000000,0.000000,0.545098}%
\pgfsetfillcolor{currentfill}%
\pgfsetfillopacity{0.400000}%
\pgfsetlinewidth{0.501875pt}%
\definecolor{currentstroke}{rgb}{0.000000,0.000000,0.545098}%
\pgfsetstrokecolor{currentstroke}%
\pgfsetstrokeopacity{0.400000}%
\pgfsetdash{}{0pt}%
\pgfpathmoveto{\pgfqpoint{2.034782in}{1.882544in}}%
\pgfpathcurveto{\pgfqpoint{2.045832in}{1.882544in}}{\pgfqpoint{2.056431in}{1.886934in}}{\pgfqpoint{2.064244in}{1.894747in}}%
\pgfpathcurveto{\pgfqpoint{2.072058in}{1.902561in}}{\pgfqpoint{2.076448in}{1.913160in}}{\pgfqpoint{2.076448in}{1.924210in}}%
\pgfpathcurveto{\pgfqpoint{2.076448in}{1.935260in}}{\pgfqpoint{2.072058in}{1.945859in}}{\pgfqpoint{2.064244in}{1.953673in}}%
\pgfpathcurveto{\pgfqpoint{2.056431in}{1.961487in}}{\pgfqpoint{2.045832in}{1.965877in}}{\pgfqpoint{2.034782in}{1.965877in}}%
\pgfpathcurveto{\pgfqpoint{2.023731in}{1.965877in}}{\pgfqpoint{2.013132in}{1.961487in}}{\pgfqpoint{2.005319in}{1.953673in}}%
\pgfpathcurveto{\pgfqpoint{1.997505in}{1.945859in}}{\pgfqpoint{1.993115in}{1.935260in}}{\pgfqpoint{1.993115in}{1.924210in}}%
\pgfpathcurveto{\pgfqpoint{1.993115in}{1.913160in}}{\pgfqpoint{1.997505in}{1.902561in}}{\pgfqpoint{2.005319in}{1.894747in}}%
\pgfpathcurveto{\pgfqpoint{2.013132in}{1.886934in}}{\pgfqpoint{2.023731in}{1.882544in}}{\pgfqpoint{2.034782in}{1.882544in}}%
\pgfpathclose%
\pgfusepath{stroke,fill}%
\end{pgfscope}%
\begin{pgfscope}%
\pgfpathrectangle{\pgfqpoint{0.511823in}{0.504323in}}{\pgfqpoint{3.218177in}{3.225677in}} %
\pgfusepath{clip}%
\pgfsetbuttcap%
\pgfsetroundjoin%
\definecolor{currentfill}{rgb}{0.000000,0.000000,0.545098}%
\pgfsetfillcolor{currentfill}%
\pgfsetfillopacity{0.400000}%
\pgfsetlinewidth{0.501875pt}%
\definecolor{currentstroke}{rgb}{0.000000,0.000000,0.545098}%
\pgfsetstrokecolor{currentstroke}%
\pgfsetstrokeopacity{0.400000}%
\pgfsetdash{}{0pt}%
\pgfpathmoveto{\pgfqpoint{2.135141in}{1.983480in}}%
\pgfpathcurveto{\pgfqpoint{2.146191in}{1.983480in}}{\pgfqpoint{2.156790in}{1.987870in}}{\pgfqpoint{2.164604in}{1.995684in}}%
\pgfpathcurveto{\pgfqpoint{2.172418in}{2.003497in}}{\pgfqpoint{2.176808in}{2.014096in}}{\pgfqpoint{2.176808in}{2.025146in}}%
\pgfpathcurveto{\pgfqpoint{2.176808in}{2.036196in}}{\pgfqpoint{2.172418in}{2.046795in}}{\pgfqpoint{2.164604in}{2.054609in}}%
\pgfpathcurveto{\pgfqpoint{2.156790in}{2.062423in}}{\pgfqpoint{2.146191in}{2.066813in}}{\pgfqpoint{2.135141in}{2.066813in}}%
\pgfpathcurveto{\pgfqpoint{2.124091in}{2.066813in}}{\pgfqpoint{2.113492in}{2.062423in}}{\pgfqpoint{2.105679in}{2.054609in}}%
\pgfpathcurveto{\pgfqpoint{2.097865in}{2.046795in}}{\pgfqpoint{2.093475in}{2.036196in}}{\pgfqpoint{2.093475in}{2.025146in}}%
\pgfpathcurveto{\pgfqpoint{2.093475in}{2.014096in}}{\pgfqpoint{2.097865in}{2.003497in}}{\pgfqpoint{2.105679in}{1.995684in}}%
\pgfpathcurveto{\pgfqpoint{2.113492in}{1.987870in}}{\pgfqpoint{2.124091in}{1.983480in}}{\pgfqpoint{2.135141in}{1.983480in}}%
\pgfpathclose%
\pgfusepath{stroke,fill}%
\end{pgfscope}%
\begin{pgfscope}%
\pgfpathrectangle{\pgfqpoint{0.511823in}{0.504323in}}{\pgfqpoint{3.218177in}{3.225677in}} %
\pgfusepath{clip}%
\pgfsetbuttcap%
\pgfsetroundjoin%
\definecolor{currentfill}{rgb}{0.000000,0.000000,0.545098}%
\pgfsetfillcolor{currentfill}%
\pgfsetfillopacity{0.400000}%
\pgfsetlinewidth{0.501875pt}%
\definecolor{currentstroke}{rgb}{0.000000,0.000000,0.545098}%
\pgfsetstrokecolor{currentstroke}%
\pgfsetstrokeopacity{0.400000}%
\pgfsetdash{}{0pt}%
\pgfpathmoveto{\pgfqpoint{2.077727in}{1.938376in}}%
\pgfpathcurveto{\pgfqpoint{2.088777in}{1.938376in}}{\pgfqpoint{2.099376in}{1.942766in}}{\pgfqpoint{2.107190in}{1.950580in}}%
\pgfpathcurveto{\pgfqpoint{2.115004in}{1.958394in}}{\pgfqpoint{2.119394in}{1.968993in}}{\pgfqpoint{2.119394in}{1.980043in}}%
\pgfpathcurveto{\pgfqpoint{2.119394in}{1.991093in}}{\pgfqpoint{2.115004in}{2.001692in}}{\pgfqpoint{2.107190in}{2.009506in}}%
\pgfpathcurveto{\pgfqpoint{2.099376in}{2.017319in}}{\pgfqpoint{2.088777in}{2.021710in}}{\pgfqpoint{2.077727in}{2.021710in}}%
\pgfpathcurveto{\pgfqpoint{2.066677in}{2.021710in}}{\pgfqpoint{2.056078in}{2.017319in}}{\pgfqpoint{2.048264in}{2.009506in}}%
\pgfpathcurveto{\pgfqpoint{2.040451in}{2.001692in}}{\pgfqpoint{2.036060in}{1.991093in}}{\pgfqpoint{2.036060in}{1.980043in}}%
\pgfpathcurveto{\pgfqpoint{2.036060in}{1.968993in}}{\pgfqpoint{2.040451in}{1.958394in}}{\pgfqpoint{2.048264in}{1.950580in}}%
\pgfpathcurveto{\pgfqpoint{2.056078in}{1.942766in}}{\pgfqpoint{2.066677in}{1.938376in}}{\pgfqpoint{2.077727in}{1.938376in}}%
\pgfpathclose%
\pgfusepath{stroke,fill}%
\end{pgfscope}%
\begin{pgfscope}%
\pgfpathrectangle{\pgfqpoint{0.511823in}{0.504323in}}{\pgfqpoint{3.218177in}{3.225677in}} %
\pgfusepath{clip}%
\pgfsetbuttcap%
\pgfsetroundjoin%
\definecolor{currentfill}{rgb}{0.000000,0.000000,0.545098}%
\pgfsetfillcolor{currentfill}%
\pgfsetfillopacity{0.400000}%
\pgfsetlinewidth{0.501875pt}%
\definecolor{currentstroke}{rgb}{0.000000,0.000000,0.545098}%
\pgfsetstrokecolor{currentstroke}%
\pgfsetstrokeopacity{0.400000}%
\pgfsetdash{}{0pt}%
\pgfpathmoveto{\pgfqpoint{2.176866in}{2.039698in}}%
\pgfpathcurveto{\pgfqpoint{2.187917in}{2.039698in}}{\pgfqpoint{2.198516in}{2.044088in}}{\pgfqpoint{2.206329in}{2.051902in}}%
\pgfpathcurveto{\pgfqpoint{2.214143in}{2.059716in}}{\pgfqpoint{2.218533in}{2.070315in}}{\pgfqpoint{2.218533in}{2.081365in}}%
\pgfpathcurveto{\pgfqpoint{2.218533in}{2.092415in}}{\pgfqpoint{2.214143in}{2.103014in}}{\pgfqpoint{2.206329in}{2.110828in}}%
\pgfpathcurveto{\pgfqpoint{2.198516in}{2.118641in}}{\pgfqpoint{2.187917in}{2.123031in}}{\pgfqpoint{2.176866in}{2.123031in}}%
\pgfpathcurveto{\pgfqpoint{2.165816in}{2.123031in}}{\pgfqpoint{2.155217in}{2.118641in}}{\pgfqpoint{2.147404in}{2.110828in}}%
\pgfpathcurveto{\pgfqpoint{2.139590in}{2.103014in}}{\pgfqpoint{2.135200in}{2.092415in}}{\pgfqpoint{2.135200in}{2.081365in}}%
\pgfpathcurveto{\pgfqpoint{2.135200in}{2.070315in}}{\pgfqpoint{2.139590in}{2.059716in}}{\pgfqpoint{2.147404in}{2.051902in}}%
\pgfpathcurveto{\pgfqpoint{2.155217in}{2.044088in}}{\pgfqpoint{2.165816in}{2.039698in}}{\pgfqpoint{2.176866in}{2.039698in}}%
\pgfpathclose%
\pgfusepath{stroke,fill}%
\end{pgfscope}%
\begin{pgfscope}%
\pgfpathrectangle{\pgfqpoint{0.511823in}{0.504323in}}{\pgfqpoint{3.218177in}{3.225677in}} %
\pgfusepath{clip}%
\pgfsetbuttcap%
\pgfsetroundjoin%
\definecolor{currentfill}{rgb}{0.000000,0.000000,0.545098}%
\pgfsetfillcolor{currentfill}%
\pgfsetfillopacity{0.400000}%
\pgfsetlinewidth{0.501875pt}%
\definecolor{currentstroke}{rgb}{0.000000,0.000000,0.545098}%
\pgfsetstrokecolor{currentstroke}%
\pgfsetstrokeopacity{0.400000}%
\pgfsetdash{}{0pt}%
\pgfpathmoveto{\pgfqpoint{2.147220in}{2.020518in}}%
\pgfpathcurveto{\pgfqpoint{2.158270in}{2.020518in}}{\pgfqpoint{2.168870in}{2.024909in}}{\pgfqpoint{2.176683in}{2.032722in}}%
\pgfpathcurveto{\pgfqpoint{2.184497in}{2.040536in}}{\pgfqpoint{2.188887in}{2.051135in}}{\pgfqpoint{2.188887in}{2.062185in}}%
\pgfpathcurveto{\pgfqpoint{2.188887in}{2.073235in}}{\pgfqpoint{2.184497in}{2.083834in}}{\pgfqpoint{2.176683in}{2.091648in}}%
\pgfpathcurveto{\pgfqpoint{2.168870in}{2.099461in}}{\pgfqpoint{2.158270in}{2.103852in}}{\pgfqpoint{2.147220in}{2.103852in}}%
\pgfpathcurveto{\pgfqpoint{2.136170in}{2.103852in}}{\pgfqpoint{2.125571in}{2.099461in}}{\pgfqpoint{2.117758in}{2.091648in}}%
\pgfpathcurveto{\pgfqpoint{2.109944in}{2.083834in}}{\pgfqpoint{2.105554in}{2.073235in}}{\pgfqpoint{2.105554in}{2.062185in}}%
\pgfpathcurveto{\pgfqpoint{2.105554in}{2.051135in}}{\pgfqpoint{2.109944in}{2.040536in}}{\pgfqpoint{2.117758in}{2.032722in}}%
\pgfpathcurveto{\pgfqpoint{2.125571in}{2.024909in}}{\pgfqpoint{2.136170in}{2.020518in}}{\pgfqpoint{2.147220in}{2.020518in}}%
\pgfpathclose%
\pgfusepath{stroke,fill}%
\end{pgfscope}%
\begin{pgfscope}%
\pgfpathrectangle{\pgfqpoint{0.511823in}{0.504323in}}{\pgfqpoint{3.218177in}{3.225677in}} %
\pgfusepath{clip}%
\pgfsetbuttcap%
\pgfsetroundjoin%
\definecolor{currentfill}{rgb}{0.000000,0.000000,0.545098}%
\pgfsetfillcolor{currentfill}%
\pgfsetfillopacity{0.400000}%
\pgfsetlinewidth{0.501875pt}%
\definecolor{currentstroke}{rgb}{0.000000,0.000000,0.545098}%
\pgfsetstrokecolor{currentstroke}%
\pgfsetstrokeopacity{0.400000}%
\pgfsetdash{}{0pt}%
\pgfpathmoveto{\pgfqpoint{2.161391in}{2.042719in}}%
\pgfpathcurveto{\pgfqpoint{2.172441in}{2.042719in}}{\pgfqpoint{2.183040in}{2.047109in}}{\pgfqpoint{2.190853in}{2.054923in}}%
\pgfpathcurveto{\pgfqpoint{2.198667in}{2.062736in}}{\pgfqpoint{2.203057in}{2.073335in}}{\pgfqpoint{2.203057in}{2.084386in}}%
\pgfpathcurveto{\pgfqpoint{2.203057in}{2.095436in}}{\pgfqpoint{2.198667in}{2.106035in}}{\pgfqpoint{2.190853in}{2.113848in}}%
\pgfpathcurveto{\pgfqpoint{2.183040in}{2.121662in}}{\pgfqpoint{2.172441in}{2.126052in}}{\pgfqpoint{2.161391in}{2.126052in}}%
\pgfpathcurveto{\pgfqpoint{2.150341in}{2.126052in}}{\pgfqpoint{2.139742in}{2.121662in}}{\pgfqpoint{2.131928in}{2.113848in}}%
\pgfpathcurveto{\pgfqpoint{2.124114in}{2.106035in}}{\pgfqpoint{2.119724in}{2.095436in}}{\pgfqpoint{2.119724in}{2.084386in}}%
\pgfpathcurveto{\pgfqpoint{2.119724in}{2.073335in}}{\pgfqpoint{2.124114in}{2.062736in}}{\pgfqpoint{2.131928in}{2.054923in}}%
\pgfpathcurveto{\pgfqpoint{2.139742in}{2.047109in}}{\pgfqpoint{2.150341in}{2.042719in}}{\pgfqpoint{2.161391in}{2.042719in}}%
\pgfpathclose%
\pgfusepath{stroke,fill}%
\end{pgfscope}%
\begin{pgfscope}%
\pgfpathrectangle{\pgfqpoint{0.511823in}{0.504323in}}{\pgfqpoint{3.218177in}{3.225677in}} %
\pgfusepath{clip}%
\pgfsetbuttcap%
\pgfsetroundjoin%
\definecolor{currentfill}{rgb}{0.000000,0.000000,0.545098}%
\pgfsetfillcolor{currentfill}%
\pgfsetfillopacity{0.400000}%
\pgfsetlinewidth{0.501875pt}%
\definecolor{currentstroke}{rgb}{0.000000,0.000000,0.545098}%
\pgfsetstrokecolor{currentstroke}%
\pgfsetstrokeopacity{0.400000}%
\pgfsetdash{}{0pt}%
\pgfpathmoveto{\pgfqpoint{2.185207in}{2.074379in}}%
\pgfpathcurveto{\pgfqpoint{2.196257in}{2.074379in}}{\pgfqpoint{2.206856in}{2.078770in}}{\pgfqpoint{2.214670in}{2.086583in}}%
\pgfpathcurveto{\pgfqpoint{2.222484in}{2.094397in}}{\pgfqpoint{2.226874in}{2.104996in}}{\pgfqpoint{2.226874in}{2.116046in}}%
\pgfpathcurveto{\pgfqpoint{2.226874in}{2.127096in}}{\pgfqpoint{2.222484in}{2.137695in}}{\pgfqpoint{2.214670in}{2.145509in}}%
\pgfpathcurveto{\pgfqpoint{2.206856in}{2.153323in}}{\pgfqpoint{2.196257in}{2.157713in}}{\pgfqpoint{2.185207in}{2.157713in}}%
\pgfpathcurveto{\pgfqpoint{2.174157in}{2.157713in}}{\pgfqpoint{2.163558in}{2.153323in}}{\pgfqpoint{2.155744in}{2.145509in}}%
\pgfpathcurveto{\pgfqpoint{2.147931in}{2.137695in}}{\pgfqpoint{2.143541in}{2.127096in}}{\pgfqpoint{2.143541in}{2.116046in}}%
\pgfpathcurveto{\pgfqpoint{2.143541in}{2.104996in}}{\pgfqpoint{2.147931in}{2.094397in}}{\pgfqpoint{2.155744in}{2.086583in}}%
\pgfpathcurveto{\pgfqpoint{2.163558in}{2.078770in}}{\pgfqpoint{2.174157in}{2.074379in}}{\pgfqpoint{2.185207in}{2.074379in}}%
\pgfpathclose%
\pgfusepath{stroke,fill}%
\end{pgfscope}%
\begin{pgfscope}%
\pgfpathrectangle{\pgfqpoint{0.511823in}{0.504323in}}{\pgfqpoint{3.218177in}{3.225677in}} %
\pgfusepath{clip}%
\pgfsetbuttcap%
\pgfsetroundjoin%
\definecolor{currentfill}{rgb}{0.000000,0.000000,0.545098}%
\pgfsetfillcolor{currentfill}%
\pgfsetfillopacity{0.400000}%
\pgfsetlinewidth{0.501875pt}%
\definecolor{currentstroke}{rgb}{0.000000,0.000000,0.545098}%
\pgfsetstrokecolor{currentstroke}%
\pgfsetstrokeopacity{0.400000}%
\pgfsetdash{}{0pt}%
\pgfpathmoveto{\pgfqpoint{2.132868in}{2.033029in}}%
\pgfpathcurveto{\pgfqpoint{2.143918in}{2.033029in}}{\pgfqpoint{2.154517in}{2.037419in}}{\pgfqpoint{2.162331in}{2.045233in}}%
\pgfpathcurveto{\pgfqpoint{2.170144in}{2.053046in}}{\pgfqpoint{2.174535in}{2.063645in}}{\pgfqpoint{2.174535in}{2.074695in}}%
\pgfpathcurveto{\pgfqpoint{2.174535in}{2.085746in}}{\pgfqpoint{2.170144in}{2.096345in}}{\pgfqpoint{2.162331in}{2.104158in}}%
\pgfpathcurveto{\pgfqpoint{2.154517in}{2.111972in}}{\pgfqpoint{2.143918in}{2.116362in}}{\pgfqpoint{2.132868in}{2.116362in}}%
\pgfpathcurveto{\pgfqpoint{2.121818in}{2.116362in}}{\pgfqpoint{2.111219in}{2.111972in}}{\pgfqpoint{2.103405in}{2.104158in}}%
\pgfpathcurveto{\pgfqpoint{2.095591in}{2.096345in}}{\pgfqpoint{2.091201in}{2.085746in}}{\pgfqpoint{2.091201in}{2.074695in}}%
\pgfpathcurveto{\pgfqpoint{2.091201in}{2.063645in}}{\pgfqpoint{2.095591in}{2.053046in}}{\pgfqpoint{2.103405in}{2.045233in}}%
\pgfpathcurveto{\pgfqpoint{2.111219in}{2.037419in}}{\pgfqpoint{2.121818in}{2.033029in}}{\pgfqpoint{2.132868in}{2.033029in}}%
\pgfpathclose%
\pgfusepath{stroke,fill}%
\end{pgfscope}%
\begin{pgfscope}%
\pgfpathrectangle{\pgfqpoint{0.511823in}{0.504323in}}{\pgfqpoint{3.218177in}{3.225677in}} %
\pgfusepath{clip}%
\pgfsetbuttcap%
\pgfsetroundjoin%
\definecolor{currentfill}{rgb}{0.000000,0.000000,0.545098}%
\pgfsetfillcolor{currentfill}%
\pgfsetfillopacity{0.400000}%
\pgfsetlinewidth{0.501875pt}%
\definecolor{currentstroke}{rgb}{0.000000,0.000000,0.545098}%
\pgfsetstrokecolor{currentstroke}%
\pgfsetstrokeopacity{0.400000}%
\pgfsetdash{}{0pt}%
\pgfpathmoveto{\pgfqpoint{2.049176in}{1.960709in}}%
\pgfpathcurveto{\pgfqpoint{2.060226in}{1.960709in}}{\pgfqpoint{2.070825in}{1.965099in}}{\pgfqpoint{2.078639in}{1.972913in}}%
\pgfpathcurveto{\pgfqpoint{2.086452in}{1.980727in}}{\pgfqpoint{2.090843in}{1.991326in}}{\pgfqpoint{2.090843in}{2.002376in}}%
\pgfpathcurveto{\pgfqpoint{2.090843in}{2.013426in}}{\pgfqpoint{2.086452in}{2.024025in}}{\pgfqpoint{2.078639in}{2.031839in}}%
\pgfpathcurveto{\pgfqpoint{2.070825in}{2.039652in}}{\pgfqpoint{2.060226in}{2.044042in}}{\pgfqpoint{2.049176in}{2.044042in}}%
\pgfpathcurveto{\pgfqpoint{2.038126in}{2.044042in}}{\pgfqpoint{2.027527in}{2.039652in}}{\pgfqpoint{2.019713in}{2.031839in}}%
\pgfpathcurveto{\pgfqpoint{2.011900in}{2.024025in}}{\pgfqpoint{2.007509in}{2.013426in}}{\pgfqpoint{2.007509in}{2.002376in}}%
\pgfpathcurveto{\pgfqpoint{2.007509in}{1.991326in}}{\pgfqpoint{2.011900in}{1.980727in}}{\pgfqpoint{2.019713in}{1.972913in}}%
\pgfpathcurveto{\pgfqpoint{2.027527in}{1.965099in}}{\pgfqpoint{2.038126in}{1.960709in}}{\pgfqpoint{2.049176in}{1.960709in}}%
\pgfpathclose%
\pgfusepath{stroke,fill}%
\end{pgfscope}%
\begin{pgfscope}%
\pgfpathrectangle{\pgfqpoint{0.511823in}{0.504323in}}{\pgfqpoint{3.218177in}{3.225677in}} %
\pgfusepath{clip}%
\pgfsetbuttcap%
\pgfsetroundjoin%
\definecolor{currentfill}{rgb}{0.000000,0.000000,0.545098}%
\pgfsetfillcolor{currentfill}%
\pgfsetfillopacity{0.400000}%
\pgfsetlinewidth{0.501875pt}%
\definecolor{currentstroke}{rgb}{0.000000,0.000000,0.545098}%
\pgfsetstrokecolor{currentstroke}%
\pgfsetstrokeopacity{0.400000}%
\pgfsetdash{}{0pt}%
\pgfpathmoveto{\pgfqpoint{2.183833in}{2.100413in}}%
\pgfpathcurveto{\pgfqpoint{2.194883in}{2.100413in}}{\pgfqpoint{2.205482in}{2.104803in}}{\pgfqpoint{2.213296in}{2.112616in}}%
\pgfpathcurveto{\pgfqpoint{2.221109in}{2.120430in}}{\pgfqpoint{2.225499in}{2.131029in}}{\pgfqpoint{2.225499in}{2.142079in}}%
\pgfpathcurveto{\pgfqpoint{2.225499in}{2.153129in}}{\pgfqpoint{2.221109in}{2.163728in}}{\pgfqpoint{2.213296in}{2.171542in}}%
\pgfpathcurveto{\pgfqpoint{2.205482in}{2.179356in}}{\pgfqpoint{2.194883in}{2.183746in}}{\pgfqpoint{2.183833in}{2.183746in}}%
\pgfpathcurveto{\pgfqpoint{2.172783in}{2.183746in}}{\pgfqpoint{2.162184in}{2.179356in}}{\pgfqpoint{2.154370in}{2.171542in}}%
\pgfpathcurveto{\pgfqpoint{2.146556in}{2.163728in}}{\pgfqpoint{2.142166in}{2.153129in}}{\pgfqpoint{2.142166in}{2.142079in}}%
\pgfpathcurveto{\pgfqpoint{2.142166in}{2.131029in}}{\pgfqpoint{2.146556in}{2.120430in}}{\pgfqpoint{2.154370in}{2.112616in}}%
\pgfpathcurveto{\pgfqpoint{2.162184in}{2.104803in}}{\pgfqpoint{2.172783in}{2.100413in}}{\pgfqpoint{2.183833in}{2.100413in}}%
\pgfpathclose%
\pgfusepath{stroke,fill}%
\end{pgfscope}%
\begin{pgfscope}%
\pgfpathrectangle{\pgfqpoint{0.511823in}{0.504323in}}{\pgfqpoint{3.218177in}{3.225677in}} %
\pgfusepath{clip}%
\pgfsetbuttcap%
\pgfsetroundjoin%
\definecolor{currentfill}{rgb}{0.000000,0.000000,0.545098}%
\pgfsetfillcolor{currentfill}%
\pgfsetfillopacity{0.400000}%
\pgfsetlinewidth{0.501875pt}%
\definecolor{currentstroke}{rgb}{0.000000,0.000000,0.545098}%
\pgfsetstrokecolor{currentstroke}%
\pgfsetstrokeopacity{0.400000}%
\pgfsetdash{}{0pt}%
\pgfpathmoveto{\pgfqpoint{1.970644in}{1.900376in}}%
\pgfpathcurveto{\pgfqpoint{1.981694in}{1.900376in}}{\pgfqpoint{1.992293in}{1.904767in}}{\pgfqpoint{2.000107in}{1.912580in}}%
\pgfpathcurveto{\pgfqpoint{2.007920in}{1.920394in}}{\pgfqpoint{2.012311in}{1.930993in}}{\pgfqpoint{2.012311in}{1.942043in}}%
\pgfpathcurveto{\pgfqpoint{2.012311in}{1.953093in}}{\pgfqpoint{2.007920in}{1.963692in}}{\pgfqpoint{2.000107in}{1.971506in}}%
\pgfpathcurveto{\pgfqpoint{1.992293in}{1.979319in}}{\pgfqpoint{1.981694in}{1.983710in}}{\pgfqpoint{1.970644in}{1.983710in}}%
\pgfpathcurveto{\pgfqpoint{1.959594in}{1.983710in}}{\pgfqpoint{1.948995in}{1.979319in}}{\pgfqpoint{1.941181in}{1.971506in}}%
\pgfpathcurveto{\pgfqpoint{1.933368in}{1.963692in}}{\pgfqpoint{1.928977in}{1.953093in}}{\pgfqpoint{1.928977in}{1.942043in}}%
\pgfpathcurveto{\pgfqpoint{1.928977in}{1.930993in}}{\pgfqpoint{1.933368in}{1.920394in}}{\pgfqpoint{1.941181in}{1.912580in}}%
\pgfpathcurveto{\pgfqpoint{1.948995in}{1.904767in}}{\pgfqpoint{1.959594in}{1.900376in}}{\pgfqpoint{1.970644in}{1.900376in}}%
\pgfpathclose%
\pgfusepath{stroke,fill}%
\end{pgfscope}%
\begin{pgfscope}%
\pgfpathrectangle{\pgfqpoint{0.511823in}{0.504323in}}{\pgfqpoint{3.218177in}{3.225677in}} %
\pgfusepath{clip}%
\pgfsetbuttcap%
\pgfsetroundjoin%
\definecolor{currentfill}{rgb}{0.000000,0.000000,0.545098}%
\pgfsetfillcolor{currentfill}%
\pgfsetfillopacity{0.400000}%
\pgfsetlinewidth{0.501875pt}%
\definecolor{currentstroke}{rgb}{0.000000,0.000000,0.545098}%
\pgfsetstrokecolor{currentstroke}%
\pgfsetstrokeopacity{0.400000}%
\pgfsetdash{}{0pt}%
\pgfpathmoveto{\pgfqpoint{2.047614in}{1.984385in}}%
\pgfpathcurveto{\pgfqpoint{2.058664in}{1.984385in}}{\pgfqpoint{2.069263in}{1.988775in}}{\pgfqpoint{2.077077in}{1.996589in}}%
\pgfpathcurveto{\pgfqpoint{2.084890in}{2.004403in}}{\pgfqpoint{2.089281in}{2.015002in}}{\pgfqpoint{2.089281in}{2.026052in}}%
\pgfpathcurveto{\pgfqpoint{2.089281in}{2.037102in}}{\pgfqpoint{2.084890in}{2.047701in}}{\pgfqpoint{2.077077in}{2.055515in}}%
\pgfpathcurveto{\pgfqpoint{2.069263in}{2.063328in}}{\pgfqpoint{2.058664in}{2.067718in}}{\pgfqpoint{2.047614in}{2.067718in}}%
\pgfpathcurveto{\pgfqpoint{2.036564in}{2.067718in}}{\pgfqpoint{2.025965in}{2.063328in}}{\pgfqpoint{2.018151in}{2.055515in}}%
\pgfpathcurveto{\pgfqpoint{2.010338in}{2.047701in}}{\pgfqpoint{2.005947in}{2.037102in}}{\pgfqpoint{2.005947in}{2.026052in}}%
\pgfpathcurveto{\pgfqpoint{2.005947in}{2.015002in}}{\pgfqpoint{2.010338in}{2.004403in}}{\pgfqpoint{2.018151in}{1.996589in}}%
\pgfpathcurveto{\pgfqpoint{2.025965in}{1.988775in}}{\pgfqpoint{2.036564in}{1.984385in}}{\pgfqpoint{2.047614in}{1.984385in}}%
\pgfpathclose%
\pgfusepath{stroke,fill}%
\end{pgfscope}%
\begin{pgfscope}%
\pgfpathrectangle{\pgfqpoint{0.511823in}{0.504323in}}{\pgfqpoint{3.218177in}{3.225677in}} %
\pgfusepath{clip}%
\pgfsetbuttcap%
\pgfsetroundjoin%
\definecolor{currentfill}{rgb}{0.000000,0.000000,0.545098}%
\pgfsetfillcolor{currentfill}%
\pgfsetfillopacity{0.400000}%
\pgfsetlinewidth{0.501875pt}%
\definecolor{currentstroke}{rgb}{0.000000,0.000000,0.545098}%
\pgfsetstrokecolor{currentstroke}%
\pgfsetstrokeopacity{0.400000}%
\pgfsetdash{}{0pt}%
\pgfpathmoveto{\pgfqpoint{2.124206in}{2.069033in}}%
\pgfpathcurveto{\pgfqpoint{2.135256in}{2.069033in}}{\pgfqpoint{2.145855in}{2.073424in}}{\pgfqpoint{2.153669in}{2.081237in}}%
\pgfpathcurveto{\pgfqpoint{2.161482in}{2.089051in}}{\pgfqpoint{2.165872in}{2.099650in}}{\pgfqpoint{2.165872in}{2.110700in}}%
\pgfpathcurveto{\pgfqpoint{2.165872in}{2.121750in}}{\pgfqpoint{2.161482in}{2.132349in}}{\pgfqpoint{2.153669in}{2.140163in}}%
\pgfpathcurveto{\pgfqpoint{2.145855in}{2.147976in}}{\pgfqpoint{2.135256in}{2.152367in}}{\pgfqpoint{2.124206in}{2.152367in}}%
\pgfpathcurveto{\pgfqpoint{2.113156in}{2.152367in}}{\pgfqpoint{2.102557in}{2.147976in}}{\pgfqpoint{2.094743in}{2.140163in}}%
\pgfpathcurveto{\pgfqpoint{2.086929in}{2.132349in}}{\pgfqpoint{2.082539in}{2.121750in}}{\pgfqpoint{2.082539in}{2.110700in}}%
\pgfpathcurveto{\pgfqpoint{2.082539in}{2.099650in}}{\pgfqpoint{2.086929in}{2.089051in}}{\pgfqpoint{2.094743in}{2.081237in}}%
\pgfpathcurveto{\pgfqpoint{2.102557in}{2.073424in}}{\pgfqpoint{2.113156in}{2.069033in}}{\pgfqpoint{2.124206in}{2.069033in}}%
\pgfpathclose%
\pgfusepath{stroke,fill}%
\end{pgfscope}%
\begin{pgfscope}%
\pgfpathrectangle{\pgfqpoint{0.511823in}{0.504323in}}{\pgfqpoint{3.218177in}{3.225677in}} %
\pgfusepath{clip}%
\pgfsetbuttcap%
\pgfsetroundjoin%
\definecolor{currentfill}{rgb}{0.000000,0.000000,0.545098}%
\pgfsetfillcolor{currentfill}%
\pgfsetfillopacity{0.400000}%
\pgfsetlinewidth{0.501875pt}%
\definecolor{currentstroke}{rgb}{0.000000,0.000000,0.545098}%
\pgfsetstrokecolor{currentstroke}%
\pgfsetstrokeopacity{0.400000}%
\pgfsetdash{}{0pt}%
\pgfpathmoveto{\pgfqpoint{2.015719in}{1.969553in}}%
\pgfpathcurveto{\pgfqpoint{2.026769in}{1.969553in}}{\pgfqpoint{2.037368in}{1.973943in}}{\pgfqpoint{2.045181in}{1.981757in}}%
\pgfpathcurveto{\pgfqpoint{2.052995in}{1.989570in}}{\pgfqpoint{2.057385in}{2.000169in}}{\pgfqpoint{2.057385in}{2.011220in}}%
\pgfpathcurveto{\pgfqpoint{2.057385in}{2.022270in}}{\pgfqpoint{2.052995in}{2.032869in}}{\pgfqpoint{2.045181in}{2.040682in}}%
\pgfpathcurveto{\pgfqpoint{2.037368in}{2.048496in}}{\pgfqpoint{2.026769in}{2.052886in}}{\pgfqpoint{2.015719in}{2.052886in}}%
\pgfpathcurveto{\pgfqpoint{2.004668in}{2.052886in}}{\pgfqpoint{1.994069in}{2.048496in}}{\pgfqpoint{1.986256in}{2.040682in}}%
\pgfpathcurveto{\pgfqpoint{1.978442in}{2.032869in}}{\pgfqpoint{1.974052in}{2.022270in}}{\pgfqpoint{1.974052in}{2.011220in}}%
\pgfpathcurveto{\pgfqpoint{1.974052in}{2.000169in}}{\pgfqpoint{1.978442in}{1.989570in}}{\pgfqpoint{1.986256in}{1.981757in}}%
\pgfpathcurveto{\pgfqpoint{1.994069in}{1.973943in}}{\pgfqpoint{2.004668in}{1.969553in}}{\pgfqpoint{2.015719in}{1.969553in}}%
\pgfpathclose%
\pgfusepath{stroke,fill}%
\end{pgfscope}%
\begin{pgfscope}%
\pgfpathrectangle{\pgfqpoint{0.511823in}{0.504323in}}{\pgfqpoint{3.218177in}{3.225677in}} %
\pgfusepath{clip}%
\pgfsetbuttcap%
\pgfsetroundjoin%
\definecolor{currentfill}{rgb}{0.000000,0.000000,0.545098}%
\pgfsetfillcolor{currentfill}%
\pgfsetfillopacity{0.400000}%
\pgfsetlinewidth{0.501875pt}%
\definecolor{currentstroke}{rgb}{0.000000,0.000000,0.545098}%
\pgfsetstrokecolor{currentstroke}%
\pgfsetstrokeopacity{0.400000}%
\pgfsetdash{}{0pt}%
\pgfpathmoveto{\pgfqpoint{2.101477in}{2.064311in}}%
\pgfpathcurveto{\pgfqpoint{2.112528in}{2.064311in}}{\pgfqpoint{2.123127in}{2.068701in}}{\pgfqpoint{2.130940in}{2.076515in}}%
\pgfpathcurveto{\pgfqpoint{2.138754in}{2.084328in}}{\pgfqpoint{2.143144in}{2.094927in}}{\pgfqpoint{2.143144in}{2.105977in}}%
\pgfpathcurveto{\pgfqpoint{2.143144in}{2.117028in}}{\pgfqpoint{2.138754in}{2.127627in}}{\pgfqpoint{2.130940in}{2.135440in}}%
\pgfpathcurveto{\pgfqpoint{2.123127in}{2.143254in}}{\pgfqpoint{2.112528in}{2.147644in}}{\pgfqpoint{2.101477in}{2.147644in}}%
\pgfpathcurveto{\pgfqpoint{2.090427in}{2.147644in}}{\pgfqpoint{2.079828in}{2.143254in}}{\pgfqpoint{2.072015in}{2.135440in}}%
\pgfpathcurveto{\pgfqpoint{2.064201in}{2.127627in}}{\pgfqpoint{2.059811in}{2.117028in}}{\pgfqpoint{2.059811in}{2.105977in}}%
\pgfpathcurveto{\pgfqpoint{2.059811in}{2.094927in}}{\pgfqpoint{2.064201in}{2.084328in}}{\pgfqpoint{2.072015in}{2.076515in}}%
\pgfpathcurveto{\pgfqpoint{2.079828in}{2.068701in}}{\pgfqpoint{2.090427in}{2.064311in}}{\pgfqpoint{2.101477in}{2.064311in}}%
\pgfpathclose%
\pgfusepath{stroke,fill}%
\end{pgfscope}%
\begin{pgfscope}%
\pgfpathrectangle{\pgfqpoint{0.511823in}{0.504323in}}{\pgfqpoint{3.218177in}{3.225677in}} %
\pgfusepath{clip}%
\pgfsetbuttcap%
\pgfsetroundjoin%
\definecolor{currentfill}{rgb}{0.000000,0.000000,0.545098}%
\pgfsetfillcolor{currentfill}%
\pgfsetfillopacity{0.400000}%
\pgfsetlinewidth{0.501875pt}%
\definecolor{currentstroke}{rgb}{0.000000,0.000000,0.545098}%
\pgfsetstrokecolor{currentstroke}%
\pgfsetstrokeopacity{0.400000}%
\pgfsetdash{}{0pt}%
\pgfpathmoveto{\pgfqpoint{1.960394in}{1.930406in}}%
\pgfpathcurveto{\pgfqpoint{1.971444in}{1.930406in}}{\pgfqpoint{1.982043in}{1.934797in}}{\pgfqpoint{1.989857in}{1.942610in}}%
\pgfpathcurveto{\pgfqpoint{1.997671in}{1.950424in}}{\pgfqpoint{2.002061in}{1.961023in}}{\pgfqpoint{2.002061in}{1.972073in}}%
\pgfpathcurveto{\pgfqpoint{2.002061in}{1.983123in}}{\pgfqpoint{1.997671in}{1.993722in}}{\pgfqpoint{1.989857in}{2.001536in}}%
\pgfpathcurveto{\pgfqpoint{1.982043in}{2.009349in}}{\pgfqpoint{1.971444in}{2.013740in}}{\pgfqpoint{1.960394in}{2.013740in}}%
\pgfpathcurveto{\pgfqpoint{1.949344in}{2.013740in}}{\pgfqpoint{1.938745in}{2.009349in}}{\pgfqpoint{1.930931in}{2.001536in}}%
\pgfpathcurveto{\pgfqpoint{1.923118in}{1.993722in}}{\pgfqpoint{1.918727in}{1.983123in}}{\pgfqpoint{1.918727in}{1.972073in}}%
\pgfpathcurveto{\pgfqpoint{1.918727in}{1.961023in}}{\pgfqpoint{1.923118in}{1.950424in}}{\pgfqpoint{1.930931in}{1.942610in}}%
\pgfpathcurveto{\pgfqpoint{1.938745in}{1.934797in}}{\pgfqpoint{1.949344in}{1.930406in}}{\pgfqpoint{1.960394in}{1.930406in}}%
\pgfpathclose%
\pgfusepath{stroke,fill}%
\end{pgfscope}%
\begin{pgfscope}%
\pgfpathrectangle{\pgfqpoint{0.511823in}{0.504323in}}{\pgfqpoint{3.218177in}{3.225677in}} %
\pgfusepath{clip}%
\pgfsetbuttcap%
\pgfsetroundjoin%
\definecolor{currentfill}{rgb}{0.000000,0.000000,0.545098}%
\pgfsetfillcolor{currentfill}%
\pgfsetfillopacity{0.400000}%
\pgfsetlinewidth{0.501875pt}%
\definecolor{currentstroke}{rgb}{0.000000,0.000000,0.545098}%
\pgfsetstrokecolor{currentstroke}%
\pgfsetstrokeopacity{0.400000}%
\pgfsetdash{}{0pt}%
\pgfpathmoveto{\pgfqpoint{1.979539in}{1.958108in}}%
\pgfpathcurveto{\pgfqpoint{1.990589in}{1.958108in}}{\pgfqpoint{2.001188in}{1.962498in}}{\pgfqpoint{2.009001in}{1.970312in}}%
\pgfpathcurveto{\pgfqpoint{2.016815in}{1.978125in}}{\pgfqpoint{2.021205in}{1.988724in}}{\pgfqpoint{2.021205in}{1.999774in}}%
\pgfpathcurveto{\pgfqpoint{2.021205in}{2.010825in}}{\pgfqpoint{2.016815in}{2.021424in}}{\pgfqpoint{2.009001in}{2.029237in}}%
\pgfpathcurveto{\pgfqpoint{2.001188in}{2.037051in}}{\pgfqpoint{1.990589in}{2.041441in}}{\pgfqpoint{1.979539in}{2.041441in}}%
\pgfpathcurveto{\pgfqpoint{1.968488in}{2.041441in}}{\pgfqpoint{1.957889in}{2.037051in}}{\pgfqpoint{1.950076in}{2.029237in}}%
\pgfpathcurveto{\pgfqpoint{1.942262in}{2.021424in}}{\pgfqpoint{1.937872in}{2.010825in}}{\pgfqpoint{1.937872in}{1.999774in}}%
\pgfpathcurveto{\pgfqpoint{1.937872in}{1.988724in}}{\pgfqpoint{1.942262in}{1.978125in}}{\pgfqpoint{1.950076in}{1.970312in}}%
\pgfpathcurveto{\pgfqpoint{1.957889in}{1.962498in}}{\pgfqpoint{1.968488in}{1.958108in}}{\pgfqpoint{1.979539in}{1.958108in}}%
\pgfpathclose%
\pgfusepath{stroke,fill}%
\end{pgfscope}%
\begin{pgfscope}%
\pgfpathrectangle{\pgfqpoint{0.511823in}{0.504323in}}{\pgfqpoint{3.218177in}{3.225677in}} %
\pgfusepath{clip}%
\pgfsetbuttcap%
\pgfsetroundjoin%
\definecolor{currentfill}{rgb}{0.000000,0.000000,0.545098}%
\pgfsetfillcolor{currentfill}%
\pgfsetfillopacity{0.400000}%
\pgfsetlinewidth{0.501875pt}%
\definecolor{currentstroke}{rgb}{0.000000,0.000000,0.545098}%
\pgfsetstrokecolor{currentstroke}%
\pgfsetstrokeopacity{0.400000}%
\pgfsetdash{}{0pt}%
\pgfpathmoveto{\pgfqpoint{2.071915in}{2.061249in}}%
\pgfpathcurveto{\pgfqpoint{2.082965in}{2.061249in}}{\pgfqpoint{2.093564in}{2.065639in}}{\pgfqpoint{2.101378in}{2.073453in}}%
\pgfpathcurveto{\pgfqpoint{2.109191in}{2.081266in}}{\pgfqpoint{2.113582in}{2.091865in}}{\pgfqpoint{2.113582in}{2.102916in}}%
\pgfpathcurveto{\pgfqpoint{2.113582in}{2.113966in}}{\pgfqpoint{2.109191in}{2.124565in}}{\pgfqpoint{2.101378in}{2.132378in}}%
\pgfpathcurveto{\pgfqpoint{2.093564in}{2.140192in}}{\pgfqpoint{2.082965in}{2.144582in}}{\pgfqpoint{2.071915in}{2.144582in}}%
\pgfpathcurveto{\pgfqpoint{2.060865in}{2.144582in}}{\pgfqpoint{2.050266in}{2.140192in}}{\pgfqpoint{2.042452in}{2.132378in}}%
\pgfpathcurveto{\pgfqpoint{2.034639in}{2.124565in}}{\pgfqpoint{2.030248in}{2.113966in}}{\pgfqpoint{2.030248in}{2.102916in}}%
\pgfpathcurveto{\pgfqpoint{2.030248in}{2.091865in}}{\pgfqpoint{2.034639in}{2.081266in}}{\pgfqpoint{2.042452in}{2.073453in}}%
\pgfpathcurveto{\pgfqpoint{2.050266in}{2.065639in}}{\pgfqpoint{2.060865in}{2.061249in}}{\pgfqpoint{2.071915in}{2.061249in}}%
\pgfpathclose%
\pgfusepath{stroke,fill}%
\end{pgfscope}%
\begin{pgfscope}%
\pgfpathrectangle{\pgfqpoint{0.511823in}{0.504323in}}{\pgfqpoint{3.218177in}{3.225677in}} %
\pgfusepath{clip}%
\pgfsetbuttcap%
\pgfsetroundjoin%
\definecolor{currentfill}{rgb}{0.000000,0.000000,0.545098}%
\pgfsetfillcolor{currentfill}%
\pgfsetfillopacity{0.400000}%
\pgfsetlinewidth{0.501875pt}%
\definecolor{currentstroke}{rgb}{0.000000,0.000000,0.545098}%
\pgfsetstrokecolor{currentstroke}%
\pgfsetstrokeopacity{0.400000}%
\pgfsetdash{}{0pt}%
\pgfpathmoveto{\pgfqpoint{2.081816in}{2.080495in}}%
\pgfpathcurveto{\pgfqpoint{2.092866in}{2.080495in}}{\pgfqpoint{2.103465in}{2.084885in}}{\pgfqpoint{2.111279in}{2.092699in}}%
\pgfpathcurveto{\pgfqpoint{2.119092in}{2.100513in}}{\pgfqpoint{2.123483in}{2.111112in}}{\pgfqpoint{2.123483in}{2.122162in}}%
\pgfpathcurveto{\pgfqpoint{2.123483in}{2.133212in}}{\pgfqpoint{2.119092in}{2.143811in}}{\pgfqpoint{2.111279in}{2.151625in}}%
\pgfpathcurveto{\pgfqpoint{2.103465in}{2.159438in}}{\pgfqpoint{2.092866in}{2.163829in}}{\pgfqpoint{2.081816in}{2.163829in}}%
\pgfpathcurveto{\pgfqpoint{2.070766in}{2.163829in}}{\pgfqpoint{2.060167in}{2.159438in}}{\pgfqpoint{2.052353in}{2.151625in}}%
\pgfpathcurveto{\pgfqpoint{2.044540in}{2.143811in}}{\pgfqpoint{2.040149in}{2.133212in}}{\pgfqpoint{2.040149in}{2.122162in}}%
\pgfpathcurveto{\pgfqpoint{2.040149in}{2.111112in}}{\pgfqpoint{2.044540in}{2.100513in}}{\pgfqpoint{2.052353in}{2.092699in}}%
\pgfpathcurveto{\pgfqpoint{2.060167in}{2.084885in}}{\pgfqpoint{2.070766in}{2.080495in}}{\pgfqpoint{2.081816in}{2.080495in}}%
\pgfpathclose%
\pgfusepath{stroke,fill}%
\end{pgfscope}%
\begin{pgfscope}%
\pgfpathrectangle{\pgfqpoint{0.511823in}{0.504323in}}{\pgfqpoint{3.218177in}{3.225677in}} %
\pgfusepath{clip}%
\pgfsetbuttcap%
\pgfsetroundjoin%
\definecolor{currentfill}{rgb}{0.000000,0.000000,0.545098}%
\pgfsetfillcolor{currentfill}%
\pgfsetfillopacity{0.400000}%
\pgfsetlinewidth{0.501875pt}%
\definecolor{currentstroke}{rgb}{0.000000,0.000000,0.545098}%
\pgfsetstrokecolor{currentstroke}%
\pgfsetstrokeopacity{0.400000}%
\pgfsetdash{}{0pt}%
\pgfpathmoveto{\pgfqpoint{2.119881in}{2.129198in}}%
\pgfpathcurveto{\pgfqpoint{2.130931in}{2.129198in}}{\pgfqpoint{2.141530in}{2.133589in}}{\pgfqpoint{2.149343in}{2.141402in}}%
\pgfpathcurveto{\pgfqpoint{2.157157in}{2.149216in}}{\pgfqpoint{2.161547in}{2.159815in}}{\pgfqpoint{2.161547in}{2.170865in}}%
\pgfpathcurveto{\pgfqpoint{2.161547in}{2.181915in}}{\pgfqpoint{2.157157in}{2.192514in}}{\pgfqpoint{2.149343in}{2.200328in}}%
\pgfpathcurveto{\pgfqpoint{2.141530in}{2.208142in}}{\pgfqpoint{2.130931in}{2.212532in}}{\pgfqpoint{2.119881in}{2.212532in}}%
\pgfpathcurveto{\pgfqpoint{2.108831in}{2.212532in}}{\pgfqpoint{2.098231in}{2.208142in}}{\pgfqpoint{2.090418in}{2.200328in}}%
\pgfpathcurveto{\pgfqpoint{2.082604in}{2.192514in}}{\pgfqpoint{2.078214in}{2.181915in}}{\pgfqpoint{2.078214in}{2.170865in}}%
\pgfpathcurveto{\pgfqpoint{2.078214in}{2.159815in}}{\pgfqpoint{2.082604in}{2.149216in}}{\pgfqpoint{2.090418in}{2.141402in}}%
\pgfpathcurveto{\pgfqpoint{2.098231in}{2.133589in}}{\pgfqpoint{2.108831in}{2.129198in}}{\pgfqpoint{2.119881in}{2.129198in}}%
\pgfpathclose%
\pgfusepath{stroke,fill}%
\end{pgfscope}%
\begin{pgfscope}%
\pgfpathrectangle{\pgfqpoint{0.511823in}{0.504323in}}{\pgfqpoint{3.218177in}{3.225677in}} %
\pgfusepath{clip}%
\pgfsetbuttcap%
\pgfsetroundjoin%
\definecolor{currentfill}{rgb}{0.000000,0.000000,0.545098}%
\pgfsetfillcolor{currentfill}%
\pgfsetfillopacity{0.400000}%
\pgfsetlinewidth{0.501875pt}%
\definecolor{currentstroke}{rgb}{0.000000,0.000000,0.545098}%
\pgfsetstrokecolor{currentstroke}%
\pgfsetstrokeopacity{0.400000}%
\pgfsetdash{}{0pt}%
\pgfpathmoveto{\pgfqpoint{2.003081in}{2.016532in}}%
\pgfpathcurveto{\pgfqpoint{2.014131in}{2.016532in}}{\pgfqpoint{2.024730in}{2.020922in}}{\pgfqpoint{2.032544in}{2.028736in}}%
\pgfpathcurveto{\pgfqpoint{2.040357in}{2.036550in}}{\pgfqpoint{2.044747in}{2.047149in}}{\pgfqpoint{2.044747in}{2.058199in}}%
\pgfpathcurveto{\pgfqpoint{2.044747in}{2.069249in}}{\pgfqpoint{2.040357in}{2.079848in}}{\pgfqpoint{2.032544in}{2.087662in}}%
\pgfpathcurveto{\pgfqpoint{2.024730in}{2.095475in}}{\pgfqpoint{2.014131in}{2.099866in}}{\pgfqpoint{2.003081in}{2.099866in}}%
\pgfpathcurveto{\pgfqpoint{1.992031in}{2.099866in}}{\pgfqpoint{1.981432in}{2.095475in}}{\pgfqpoint{1.973618in}{2.087662in}}%
\pgfpathcurveto{\pgfqpoint{1.965804in}{2.079848in}}{\pgfqpoint{1.961414in}{2.069249in}}{\pgfqpoint{1.961414in}{2.058199in}}%
\pgfpathcurveto{\pgfqpoint{1.961414in}{2.047149in}}{\pgfqpoint{1.965804in}{2.036550in}}{\pgfqpoint{1.973618in}{2.028736in}}%
\pgfpathcurveto{\pgfqpoint{1.981432in}{2.020922in}}{\pgfqpoint{1.992031in}{2.016532in}}{\pgfqpoint{2.003081in}{2.016532in}}%
\pgfpathclose%
\pgfusepath{stroke,fill}%
\end{pgfscope}%
\begin{pgfscope}%
\pgfpathrectangle{\pgfqpoint{0.511823in}{0.504323in}}{\pgfqpoint{3.218177in}{3.225677in}} %
\pgfusepath{clip}%
\pgfsetbuttcap%
\pgfsetroundjoin%
\definecolor{currentfill}{rgb}{0.000000,0.000000,0.545098}%
\pgfsetfillcolor{currentfill}%
\pgfsetfillopacity{0.400000}%
\pgfsetlinewidth{0.501875pt}%
\definecolor{currentstroke}{rgb}{0.000000,0.000000,0.545098}%
\pgfsetstrokecolor{currentstroke}%
\pgfsetstrokeopacity{0.400000}%
\pgfsetdash{}{0pt}%
\pgfpathmoveto{\pgfqpoint{2.030780in}{2.054439in}}%
\pgfpathcurveto{\pgfqpoint{2.041831in}{2.054439in}}{\pgfqpoint{2.052430in}{2.058829in}}{\pgfqpoint{2.060243in}{2.066643in}}%
\pgfpathcurveto{\pgfqpoint{2.068057in}{2.074457in}}{\pgfqpoint{2.072447in}{2.085056in}}{\pgfqpoint{2.072447in}{2.096106in}}%
\pgfpathcurveto{\pgfqpoint{2.072447in}{2.107156in}}{\pgfqpoint{2.068057in}{2.117755in}}{\pgfqpoint{2.060243in}{2.125569in}}%
\pgfpathcurveto{\pgfqpoint{2.052430in}{2.133382in}}{\pgfqpoint{2.041831in}{2.137772in}}{\pgfqpoint{2.030780in}{2.137772in}}%
\pgfpathcurveto{\pgfqpoint{2.019730in}{2.137772in}}{\pgfqpoint{2.009131in}{2.133382in}}{\pgfqpoint{2.001318in}{2.125569in}}%
\pgfpathcurveto{\pgfqpoint{1.993504in}{2.117755in}}{\pgfqpoint{1.989114in}{2.107156in}}{\pgfqpoint{1.989114in}{2.096106in}}%
\pgfpathcurveto{\pgfqpoint{1.989114in}{2.085056in}}{\pgfqpoint{1.993504in}{2.074457in}}{\pgfqpoint{2.001318in}{2.066643in}}%
\pgfpathcurveto{\pgfqpoint{2.009131in}{2.058829in}}{\pgfqpoint{2.019730in}{2.054439in}}{\pgfqpoint{2.030780in}{2.054439in}}%
\pgfpathclose%
\pgfusepath{stroke,fill}%
\end{pgfscope}%
\begin{pgfscope}%
\pgfpathrectangle{\pgfqpoint{0.511823in}{0.504323in}}{\pgfqpoint{3.218177in}{3.225677in}} %
\pgfusepath{clip}%
\pgfsetbuttcap%
\pgfsetroundjoin%
\definecolor{currentfill}{rgb}{0.000000,0.000000,0.545098}%
\pgfsetfillcolor{currentfill}%
\pgfsetfillopacity{0.400000}%
\pgfsetlinewidth{0.501875pt}%
\definecolor{currentstroke}{rgb}{0.000000,0.000000,0.545098}%
\pgfsetstrokecolor{currentstroke}%
\pgfsetstrokeopacity{0.400000}%
\pgfsetdash{}{0pt}%
\pgfpathmoveto{\pgfqpoint{2.054705in}{2.088782in}}%
\pgfpathcurveto{\pgfqpoint{2.065755in}{2.088782in}}{\pgfqpoint{2.076354in}{2.093172in}}{\pgfqpoint{2.084168in}{2.100986in}}%
\pgfpathcurveto{\pgfqpoint{2.091982in}{2.108799in}}{\pgfqpoint{2.096372in}{2.119399in}}{\pgfqpoint{2.096372in}{2.130449in}}%
\pgfpathcurveto{\pgfqpoint{2.096372in}{2.141499in}}{\pgfqpoint{2.091982in}{2.152098in}}{\pgfqpoint{2.084168in}{2.159911in}}%
\pgfpathcurveto{\pgfqpoint{2.076354in}{2.167725in}}{\pgfqpoint{2.065755in}{2.172115in}}{\pgfqpoint{2.054705in}{2.172115in}}%
\pgfpathcurveto{\pgfqpoint{2.043655in}{2.172115in}}{\pgfqpoint{2.033056in}{2.167725in}}{\pgfqpoint{2.025242in}{2.159911in}}%
\pgfpathcurveto{\pgfqpoint{2.017429in}{2.152098in}}{\pgfqpoint{2.013038in}{2.141499in}}{\pgfqpoint{2.013038in}{2.130449in}}%
\pgfpathcurveto{\pgfqpoint{2.013038in}{2.119399in}}{\pgfqpoint{2.017429in}{2.108799in}}{\pgfqpoint{2.025242in}{2.100986in}}%
\pgfpathcurveto{\pgfqpoint{2.033056in}{2.093172in}}{\pgfqpoint{2.043655in}{2.088782in}}{\pgfqpoint{2.054705in}{2.088782in}}%
\pgfpathclose%
\pgfusepath{stroke,fill}%
\end{pgfscope}%
\begin{pgfscope}%
\pgfpathrectangle{\pgfqpoint{0.511823in}{0.504323in}}{\pgfqpoint{3.218177in}{3.225677in}} %
\pgfusepath{clip}%
\pgfsetbuttcap%
\pgfsetroundjoin%
\definecolor{currentfill}{rgb}{0.000000,0.000000,0.545098}%
\pgfsetfillcolor{currentfill}%
\pgfsetfillopacity{0.400000}%
\pgfsetlinewidth{0.501875pt}%
\definecolor{currentstroke}{rgb}{0.000000,0.000000,0.545098}%
\pgfsetstrokecolor{currentstroke}%
\pgfsetstrokeopacity{0.400000}%
\pgfsetdash{}{0pt}%
\pgfpathmoveto{\pgfqpoint{2.035945in}{2.078020in}}%
\pgfpathcurveto{\pgfqpoint{2.046995in}{2.078020in}}{\pgfqpoint{2.057594in}{2.082410in}}{\pgfqpoint{2.065407in}{2.090224in}}%
\pgfpathcurveto{\pgfqpoint{2.073221in}{2.098037in}}{\pgfqpoint{2.077611in}{2.108636in}}{\pgfqpoint{2.077611in}{2.119687in}}%
\pgfpathcurveto{\pgfqpoint{2.077611in}{2.130737in}}{\pgfqpoint{2.073221in}{2.141336in}}{\pgfqpoint{2.065407in}{2.149149in}}%
\pgfpathcurveto{\pgfqpoint{2.057594in}{2.156963in}}{\pgfqpoint{2.046995in}{2.161353in}}{\pgfqpoint{2.035945in}{2.161353in}}%
\pgfpathcurveto{\pgfqpoint{2.024894in}{2.161353in}}{\pgfqpoint{2.014295in}{2.156963in}}{\pgfqpoint{2.006482in}{2.149149in}}%
\pgfpathcurveto{\pgfqpoint{1.998668in}{2.141336in}}{\pgfqpoint{1.994278in}{2.130737in}}{\pgfqpoint{1.994278in}{2.119687in}}%
\pgfpathcurveto{\pgfqpoint{1.994278in}{2.108636in}}{\pgfqpoint{1.998668in}{2.098037in}}{\pgfqpoint{2.006482in}{2.090224in}}%
\pgfpathcurveto{\pgfqpoint{2.014295in}{2.082410in}}{\pgfqpoint{2.024894in}{2.078020in}}{\pgfqpoint{2.035945in}{2.078020in}}%
\pgfpathclose%
\pgfusepath{stroke,fill}%
\end{pgfscope}%
\begin{pgfscope}%
\pgfpathrectangle{\pgfqpoint{0.511823in}{0.504323in}}{\pgfqpoint{3.218177in}{3.225677in}} %
\pgfusepath{clip}%
\pgfsetbuttcap%
\pgfsetroundjoin%
\definecolor{currentfill}{rgb}{0.000000,0.000000,0.545098}%
\pgfsetfillcolor{currentfill}%
\pgfsetfillopacity{0.400000}%
\pgfsetlinewidth{0.501875pt}%
\definecolor{currentstroke}{rgb}{0.000000,0.000000,0.545098}%
\pgfsetstrokecolor{currentstroke}%
\pgfsetstrokeopacity{0.400000}%
\pgfsetdash{}{0pt}%
\pgfpathmoveto{\pgfqpoint{1.877196in}{1.916923in}}%
\pgfpathcurveto{\pgfqpoint{1.888246in}{1.916923in}}{\pgfqpoint{1.898845in}{1.921314in}}{\pgfqpoint{1.906658in}{1.929127in}}%
\pgfpathcurveto{\pgfqpoint{1.914472in}{1.936941in}}{\pgfqpoint{1.918862in}{1.947540in}}{\pgfqpoint{1.918862in}{1.958590in}}%
\pgfpathcurveto{\pgfqpoint{1.918862in}{1.969640in}}{\pgfqpoint{1.914472in}{1.980239in}}{\pgfqpoint{1.906658in}{1.988053in}}%
\pgfpathcurveto{\pgfqpoint{1.898845in}{1.995866in}}{\pgfqpoint{1.888246in}{2.000257in}}{\pgfqpoint{1.877196in}{2.000257in}}%
\pgfpathcurveto{\pgfqpoint{1.866146in}{2.000257in}}{\pgfqpoint{1.855546in}{1.995866in}}{\pgfqpoint{1.847733in}{1.988053in}}%
\pgfpathcurveto{\pgfqpoint{1.839919in}{1.980239in}}{\pgfqpoint{1.835529in}{1.969640in}}{\pgfqpoint{1.835529in}{1.958590in}}%
\pgfpathcurveto{\pgfqpoint{1.835529in}{1.947540in}}{\pgfqpoint{1.839919in}{1.936941in}}{\pgfqpoint{1.847733in}{1.929127in}}%
\pgfpathcurveto{\pgfqpoint{1.855546in}{1.921314in}}{\pgfqpoint{1.866146in}{1.916923in}}{\pgfqpoint{1.877196in}{1.916923in}}%
\pgfpathclose%
\pgfusepath{stroke,fill}%
\end{pgfscope}%
\begin{pgfscope}%
\pgfpathrectangle{\pgfqpoint{0.511823in}{0.504323in}}{\pgfqpoint{3.218177in}{3.225677in}} %
\pgfusepath{clip}%
\pgfsetbuttcap%
\pgfsetroundjoin%
\definecolor{currentfill}{rgb}{0.000000,0.000000,0.545098}%
\pgfsetfillcolor{currentfill}%
\pgfsetfillopacity{0.400000}%
\pgfsetlinewidth{0.501875pt}%
\definecolor{currentstroke}{rgb}{0.000000,0.000000,0.545098}%
\pgfsetstrokecolor{currentstroke}%
\pgfsetstrokeopacity{0.400000}%
\pgfsetdash{}{0pt}%
\pgfpathmoveto{\pgfqpoint{1.936319in}{1.988893in}}%
\pgfpathcurveto{\pgfqpoint{1.947369in}{1.988893in}}{\pgfqpoint{1.957968in}{1.993283in}}{\pgfqpoint{1.965782in}{2.001097in}}%
\pgfpathcurveto{\pgfqpoint{1.973595in}{2.008911in}}{\pgfqpoint{1.977985in}{2.019510in}}{\pgfqpoint{1.977985in}{2.030560in}}%
\pgfpathcurveto{\pgfqpoint{1.977985in}{2.041610in}}{\pgfqpoint{1.973595in}{2.052209in}}{\pgfqpoint{1.965782in}{2.060022in}}%
\pgfpathcurveto{\pgfqpoint{1.957968in}{2.067836in}}{\pgfqpoint{1.947369in}{2.072226in}}{\pgfqpoint{1.936319in}{2.072226in}}%
\pgfpathcurveto{\pgfqpoint{1.925269in}{2.072226in}}{\pgfqpoint{1.914670in}{2.067836in}}{\pgfqpoint{1.906856in}{2.060022in}}%
\pgfpathcurveto{\pgfqpoint{1.899042in}{2.052209in}}{\pgfqpoint{1.894652in}{2.041610in}}{\pgfqpoint{1.894652in}{2.030560in}}%
\pgfpathcurveto{\pgfqpoint{1.894652in}{2.019510in}}{\pgfqpoint{1.899042in}{2.008911in}}{\pgfqpoint{1.906856in}{2.001097in}}%
\pgfpathcurveto{\pgfqpoint{1.914670in}{1.993283in}}{\pgfqpoint{1.925269in}{1.988893in}}{\pgfqpoint{1.936319in}{1.988893in}}%
\pgfpathclose%
\pgfusepath{stroke,fill}%
\end{pgfscope}%
\begin{pgfscope}%
\pgfpathrectangle{\pgfqpoint{0.511823in}{0.504323in}}{\pgfqpoint{3.218177in}{3.225677in}} %
\pgfusepath{clip}%
\pgfsetbuttcap%
\pgfsetroundjoin%
\definecolor{currentfill}{rgb}{0.000000,0.000000,0.545098}%
\pgfsetfillcolor{currentfill}%
\pgfsetfillopacity{0.400000}%
\pgfsetlinewidth{0.501875pt}%
\definecolor{currentstroke}{rgb}{0.000000,0.000000,0.545098}%
\pgfsetstrokecolor{currentstroke}%
\pgfsetstrokeopacity{0.400000}%
\pgfsetdash{}{0pt}%
\pgfpathmoveto{\pgfqpoint{2.105475in}{2.181267in}}%
\pgfpathcurveto{\pgfqpoint{2.116525in}{2.181267in}}{\pgfqpoint{2.127124in}{2.185657in}}{\pgfqpoint{2.134938in}{2.193471in}}%
\pgfpathcurveto{\pgfqpoint{2.142751in}{2.201285in}}{\pgfqpoint{2.147141in}{2.211884in}}{\pgfqpoint{2.147141in}{2.222934in}}%
\pgfpathcurveto{\pgfqpoint{2.147141in}{2.233984in}}{\pgfqpoint{2.142751in}{2.244583in}}{\pgfqpoint{2.134938in}{2.252397in}}%
\pgfpathcurveto{\pgfqpoint{2.127124in}{2.260210in}}{\pgfqpoint{2.116525in}{2.264600in}}{\pgfqpoint{2.105475in}{2.264600in}}%
\pgfpathcurveto{\pgfqpoint{2.094425in}{2.264600in}}{\pgfqpoint{2.083826in}{2.260210in}}{\pgfqpoint{2.076012in}{2.252397in}}%
\pgfpathcurveto{\pgfqpoint{2.068198in}{2.244583in}}{\pgfqpoint{2.063808in}{2.233984in}}{\pgfqpoint{2.063808in}{2.222934in}}%
\pgfpathcurveto{\pgfqpoint{2.063808in}{2.211884in}}{\pgfqpoint{2.068198in}{2.201285in}}{\pgfqpoint{2.076012in}{2.193471in}}%
\pgfpathcurveto{\pgfqpoint{2.083826in}{2.185657in}}{\pgfqpoint{2.094425in}{2.181267in}}{\pgfqpoint{2.105475in}{2.181267in}}%
\pgfpathclose%
\pgfusepath{stroke,fill}%
\end{pgfscope}%
\begin{pgfscope}%
\pgfpathrectangle{\pgfqpoint{0.511823in}{0.504323in}}{\pgfqpoint{3.218177in}{3.225677in}} %
\pgfusepath{clip}%
\pgfsetbuttcap%
\pgfsetroundjoin%
\definecolor{currentfill}{rgb}{0.000000,0.000000,0.545098}%
\pgfsetfillcolor{currentfill}%
\pgfsetfillopacity{0.400000}%
\pgfsetlinewidth{0.501875pt}%
\definecolor{currentstroke}{rgb}{0.000000,0.000000,0.545098}%
\pgfsetstrokecolor{currentstroke}%
\pgfsetstrokeopacity{0.400000}%
\pgfsetdash{}{0pt}%
\pgfpathmoveto{\pgfqpoint{1.869221in}{1.932840in}}%
\pgfpathcurveto{\pgfqpoint{1.880271in}{1.932840in}}{\pgfqpoint{1.890870in}{1.937230in}}{\pgfqpoint{1.898684in}{1.945044in}}%
\pgfpathcurveto{\pgfqpoint{1.906497in}{1.952858in}}{\pgfqpoint{1.910888in}{1.963457in}}{\pgfqpoint{1.910888in}{1.974507in}}%
\pgfpathcurveto{\pgfqpoint{1.910888in}{1.985557in}}{\pgfqpoint{1.906497in}{1.996156in}}{\pgfqpoint{1.898684in}{2.003970in}}%
\pgfpathcurveto{\pgfqpoint{1.890870in}{2.011783in}}{\pgfqpoint{1.880271in}{2.016173in}}{\pgfqpoint{1.869221in}{2.016173in}}%
\pgfpathcurveto{\pgfqpoint{1.858171in}{2.016173in}}{\pgfqpoint{1.847572in}{2.011783in}}{\pgfqpoint{1.839758in}{2.003970in}}%
\pgfpathcurveto{\pgfqpoint{1.831945in}{1.996156in}}{\pgfqpoint{1.827554in}{1.985557in}}{\pgfqpoint{1.827554in}{1.974507in}}%
\pgfpathcurveto{\pgfqpoint{1.827554in}{1.963457in}}{\pgfqpoint{1.831945in}{1.952858in}}{\pgfqpoint{1.839758in}{1.945044in}}%
\pgfpathcurveto{\pgfqpoint{1.847572in}{1.937230in}}{\pgfqpoint{1.858171in}{1.932840in}}{\pgfqpoint{1.869221in}{1.932840in}}%
\pgfpathclose%
\pgfusepath{stroke,fill}%
\end{pgfscope}%
\begin{pgfscope}%
\pgfpathrectangle{\pgfqpoint{0.511823in}{0.504323in}}{\pgfqpoint{3.218177in}{3.225677in}} %
\pgfusepath{clip}%
\pgfsetbuttcap%
\pgfsetroundjoin%
\definecolor{currentfill}{rgb}{0.000000,0.000000,0.545098}%
\pgfsetfillcolor{currentfill}%
\pgfsetfillopacity{0.400000}%
\pgfsetlinewidth{0.501875pt}%
\definecolor{currentstroke}{rgb}{0.000000,0.000000,0.545098}%
\pgfsetstrokecolor{currentstroke}%
\pgfsetstrokeopacity{0.400000}%
\pgfsetdash{}{0pt}%
\pgfpathmoveto{\pgfqpoint{2.025279in}{2.112821in}}%
\pgfpathcurveto{\pgfqpoint{2.036330in}{2.112821in}}{\pgfqpoint{2.046929in}{2.117211in}}{\pgfqpoint{2.054742in}{2.125025in}}%
\pgfpathcurveto{\pgfqpoint{2.062556in}{2.132839in}}{\pgfqpoint{2.066946in}{2.143438in}}{\pgfqpoint{2.066946in}{2.154488in}}%
\pgfpathcurveto{\pgfqpoint{2.066946in}{2.165538in}}{\pgfqpoint{2.062556in}{2.176137in}}{\pgfqpoint{2.054742in}{2.183951in}}%
\pgfpathcurveto{\pgfqpoint{2.046929in}{2.191764in}}{\pgfqpoint{2.036330in}{2.196154in}}{\pgfqpoint{2.025279in}{2.196154in}}%
\pgfpathcurveto{\pgfqpoint{2.014229in}{2.196154in}}{\pgfqpoint{2.003630in}{2.191764in}}{\pgfqpoint{1.995817in}{2.183951in}}%
\pgfpathcurveto{\pgfqpoint{1.988003in}{2.176137in}}{\pgfqpoint{1.983613in}{2.165538in}}{\pgfqpoint{1.983613in}{2.154488in}}%
\pgfpathcurveto{\pgfqpoint{1.983613in}{2.143438in}}{\pgfqpoint{1.988003in}{2.132839in}}{\pgfqpoint{1.995817in}{2.125025in}}%
\pgfpathcurveto{\pgfqpoint{2.003630in}{2.117211in}}{\pgfqpoint{2.014229in}{2.112821in}}{\pgfqpoint{2.025279in}{2.112821in}}%
\pgfpathclose%
\pgfusepath{stroke,fill}%
\end{pgfscope}%
\begin{pgfscope}%
\pgfpathrectangle{\pgfqpoint{0.511823in}{0.504323in}}{\pgfqpoint{3.218177in}{3.225677in}} %
\pgfusepath{clip}%
\pgfsetbuttcap%
\pgfsetroundjoin%
\definecolor{currentfill}{rgb}{0.000000,0.000000,0.545098}%
\pgfsetfillcolor{currentfill}%
\pgfsetfillopacity{0.400000}%
\pgfsetlinewidth{0.501875pt}%
\definecolor{currentstroke}{rgb}{0.000000,0.000000,0.545098}%
\pgfsetstrokecolor{currentstroke}%
\pgfsetstrokeopacity{0.400000}%
\pgfsetdash{}{0pt}%
\pgfpathmoveto{\pgfqpoint{1.930295in}{2.017085in}}%
\pgfpathcurveto{\pgfqpoint{1.941345in}{2.017085in}}{\pgfqpoint{1.951944in}{2.021475in}}{\pgfqpoint{1.959758in}{2.029288in}}%
\pgfpathcurveto{\pgfqpoint{1.967571in}{2.037102in}}{\pgfqpoint{1.971962in}{2.047701in}}{\pgfqpoint{1.971962in}{2.058751in}}%
\pgfpathcurveto{\pgfqpoint{1.971962in}{2.069801in}}{\pgfqpoint{1.967571in}{2.080400in}}{\pgfqpoint{1.959758in}{2.088214in}}%
\pgfpathcurveto{\pgfqpoint{1.951944in}{2.096028in}}{\pgfqpoint{1.941345in}{2.100418in}}{\pgfqpoint{1.930295in}{2.100418in}}%
\pgfpathcurveto{\pgfqpoint{1.919245in}{2.100418in}}{\pgfqpoint{1.908646in}{2.096028in}}{\pgfqpoint{1.900832in}{2.088214in}}%
\pgfpathcurveto{\pgfqpoint{1.893019in}{2.080400in}}{\pgfqpoint{1.888628in}{2.069801in}}{\pgfqpoint{1.888628in}{2.058751in}}%
\pgfpathcurveto{\pgfqpoint{1.888628in}{2.047701in}}{\pgfqpoint{1.893019in}{2.037102in}}{\pgfqpoint{1.900832in}{2.029288in}}%
\pgfpathcurveto{\pgfqpoint{1.908646in}{2.021475in}}{\pgfqpoint{1.919245in}{2.017085in}}{\pgfqpoint{1.930295in}{2.017085in}}%
\pgfpathclose%
\pgfusepath{stroke,fill}%
\end{pgfscope}%
\begin{pgfscope}%
\pgfpathrectangle{\pgfqpoint{0.511823in}{0.504323in}}{\pgfqpoint{3.218177in}{3.225677in}} %
\pgfusepath{clip}%
\pgfsetbuttcap%
\pgfsetroundjoin%
\definecolor{currentfill}{rgb}{0.000000,0.000000,0.545098}%
\pgfsetfillcolor{currentfill}%
\pgfsetfillopacity{0.400000}%
\pgfsetlinewidth{0.501875pt}%
\definecolor{currentstroke}{rgb}{0.000000,0.000000,0.545098}%
\pgfsetstrokecolor{currentstroke}%
\pgfsetstrokeopacity{0.400000}%
\pgfsetdash{}{0pt}%
\pgfpathmoveto{\pgfqpoint{2.047113in}{2.156103in}}%
\pgfpathcurveto{\pgfqpoint{2.058163in}{2.156103in}}{\pgfqpoint{2.068762in}{2.160494in}}{\pgfqpoint{2.076576in}{2.168307in}}%
\pgfpathcurveto{\pgfqpoint{2.084390in}{2.176121in}}{\pgfqpoint{2.088780in}{2.186720in}}{\pgfqpoint{2.088780in}{2.197770in}}%
\pgfpathcurveto{\pgfqpoint{2.088780in}{2.208820in}}{\pgfqpoint{2.084390in}{2.219419in}}{\pgfqpoint{2.076576in}{2.227233in}}%
\pgfpathcurveto{\pgfqpoint{2.068762in}{2.235046in}}{\pgfqpoint{2.058163in}{2.239437in}}{\pgfqpoint{2.047113in}{2.239437in}}%
\pgfpathcurveto{\pgfqpoint{2.036063in}{2.239437in}}{\pgfqpoint{2.025464in}{2.235046in}}{\pgfqpoint{2.017650in}{2.227233in}}%
\pgfpathcurveto{\pgfqpoint{2.009837in}{2.219419in}}{\pgfqpoint{2.005446in}{2.208820in}}{\pgfqpoint{2.005446in}{2.197770in}}%
\pgfpathcurveto{\pgfqpoint{2.005446in}{2.186720in}}{\pgfqpoint{2.009837in}{2.176121in}}{\pgfqpoint{2.017650in}{2.168307in}}%
\pgfpathcurveto{\pgfqpoint{2.025464in}{2.160494in}}{\pgfqpoint{2.036063in}{2.156103in}}{\pgfqpoint{2.047113in}{2.156103in}}%
\pgfpathclose%
\pgfusepath{stroke,fill}%
\end{pgfscope}%
\begin{pgfscope}%
\pgfpathrectangle{\pgfqpoint{0.511823in}{0.504323in}}{\pgfqpoint{3.218177in}{3.225677in}} %
\pgfusepath{clip}%
\pgfsetbuttcap%
\pgfsetroundjoin%
\definecolor{currentfill}{rgb}{0.000000,0.000000,0.545098}%
\pgfsetfillcolor{currentfill}%
\pgfsetfillopacity{0.400000}%
\pgfsetlinewidth{0.501875pt}%
\definecolor{currentstroke}{rgb}{0.000000,0.000000,0.545098}%
\pgfsetstrokecolor{currentstroke}%
\pgfsetstrokeopacity{0.400000}%
\pgfsetdash{}{0pt}%
\pgfpathmoveto{\pgfqpoint{1.891086in}{1.990848in}}%
\pgfpathcurveto{\pgfqpoint{1.902136in}{1.990848in}}{\pgfqpoint{1.912735in}{1.995238in}}{\pgfqpoint{1.920548in}{2.003052in}}%
\pgfpathcurveto{\pgfqpoint{1.928362in}{2.010865in}}{\pgfqpoint{1.932752in}{2.021464in}}{\pgfqpoint{1.932752in}{2.032514in}}%
\pgfpathcurveto{\pgfqpoint{1.932752in}{2.043565in}}{\pgfqpoint{1.928362in}{2.054164in}}{\pgfqpoint{1.920548in}{2.061977in}}%
\pgfpathcurveto{\pgfqpoint{1.912735in}{2.069791in}}{\pgfqpoint{1.902136in}{2.074181in}}{\pgfqpoint{1.891086in}{2.074181in}}%
\pgfpathcurveto{\pgfqpoint{1.880035in}{2.074181in}}{\pgfqpoint{1.869436in}{2.069791in}}{\pgfqpoint{1.861623in}{2.061977in}}%
\pgfpathcurveto{\pgfqpoint{1.853809in}{2.054164in}}{\pgfqpoint{1.849419in}{2.043565in}}{\pgfqpoint{1.849419in}{2.032514in}}%
\pgfpathcurveto{\pgfqpoint{1.849419in}{2.021464in}}{\pgfqpoint{1.853809in}{2.010865in}}{\pgfqpoint{1.861623in}{2.003052in}}%
\pgfpathcurveto{\pgfqpoint{1.869436in}{1.995238in}}{\pgfqpoint{1.880035in}{1.990848in}}{\pgfqpoint{1.891086in}{1.990848in}}%
\pgfpathclose%
\pgfusepath{stroke,fill}%
\end{pgfscope}%
\begin{pgfscope}%
\pgfpathrectangle{\pgfqpoint{0.511823in}{0.504323in}}{\pgfqpoint{3.218177in}{3.225677in}} %
\pgfusepath{clip}%
\pgfsetbuttcap%
\pgfsetroundjoin%
\definecolor{currentfill}{rgb}{0.000000,0.000000,0.545098}%
\pgfsetfillcolor{currentfill}%
\pgfsetfillopacity{0.400000}%
\pgfsetlinewidth{0.501875pt}%
\definecolor{currentstroke}{rgb}{0.000000,0.000000,0.545098}%
\pgfsetstrokecolor{currentstroke}%
\pgfsetstrokeopacity{0.400000}%
\pgfsetdash{}{0pt}%
\pgfpathmoveto{\pgfqpoint{1.956303in}{2.073152in}}%
\pgfpathcurveto{\pgfqpoint{1.967353in}{2.073152in}}{\pgfqpoint{1.977952in}{2.077542in}}{\pgfqpoint{1.985766in}{2.085355in}}%
\pgfpathcurveto{\pgfqpoint{1.993580in}{2.093169in}}{\pgfqpoint{1.997970in}{2.103768in}}{\pgfqpoint{1.997970in}{2.114818in}}%
\pgfpathcurveto{\pgfqpoint{1.997970in}{2.125868in}}{\pgfqpoint{1.993580in}{2.136467in}}{\pgfqpoint{1.985766in}{2.144281in}}%
\pgfpathcurveto{\pgfqpoint{1.977952in}{2.152095in}}{\pgfqpoint{1.967353in}{2.156485in}}{\pgfqpoint{1.956303in}{2.156485in}}%
\pgfpathcurveto{\pgfqpoint{1.945253in}{2.156485in}}{\pgfqpoint{1.934654in}{2.152095in}}{\pgfqpoint{1.926840in}{2.144281in}}%
\pgfpathcurveto{\pgfqpoint{1.919027in}{2.136467in}}{\pgfqpoint{1.914636in}{2.125868in}}{\pgfqpoint{1.914636in}{2.114818in}}%
\pgfpathcurveto{\pgfqpoint{1.914636in}{2.103768in}}{\pgfqpoint{1.919027in}{2.093169in}}{\pgfqpoint{1.926840in}{2.085355in}}%
\pgfpathcurveto{\pgfqpoint{1.934654in}{2.077542in}}{\pgfqpoint{1.945253in}{2.073152in}}{\pgfqpoint{1.956303in}{2.073152in}}%
\pgfpathclose%
\pgfusepath{stroke,fill}%
\end{pgfscope}%
\begin{pgfscope}%
\pgfpathrectangle{\pgfqpoint{0.511823in}{0.504323in}}{\pgfqpoint{3.218177in}{3.225677in}} %
\pgfusepath{clip}%
\pgfsetbuttcap%
\pgfsetroundjoin%
\definecolor{currentfill}{rgb}{0.000000,0.000000,0.545098}%
\pgfsetfillcolor{currentfill}%
\pgfsetfillopacity{0.400000}%
\pgfsetlinewidth{0.501875pt}%
\definecolor{currentstroke}{rgb}{0.000000,0.000000,0.545098}%
\pgfsetstrokecolor{currentstroke}%
\pgfsetstrokeopacity{0.400000}%
\pgfsetdash{}{0pt}%
\pgfpathmoveto{\pgfqpoint{1.776995in}{1.878678in}}%
\pgfpathcurveto{\pgfqpoint{1.788045in}{1.878678in}}{\pgfqpoint{1.798644in}{1.883068in}}{\pgfqpoint{1.806457in}{1.890882in}}%
\pgfpathcurveto{\pgfqpoint{1.814271in}{1.898696in}}{\pgfqpoint{1.818661in}{1.909295in}}{\pgfqpoint{1.818661in}{1.920345in}}%
\pgfpathcurveto{\pgfqpoint{1.818661in}{1.931395in}}{\pgfqpoint{1.814271in}{1.941994in}}{\pgfqpoint{1.806457in}{1.949808in}}%
\pgfpathcurveto{\pgfqpoint{1.798644in}{1.957621in}}{\pgfqpoint{1.788045in}{1.962012in}}{\pgfqpoint{1.776995in}{1.962012in}}%
\pgfpathcurveto{\pgfqpoint{1.765945in}{1.962012in}}{\pgfqpoint{1.755346in}{1.957621in}}{\pgfqpoint{1.747532in}{1.949808in}}%
\pgfpathcurveto{\pgfqpoint{1.739718in}{1.941994in}}{\pgfqpoint{1.735328in}{1.931395in}}{\pgfqpoint{1.735328in}{1.920345in}}%
\pgfpathcurveto{\pgfqpoint{1.735328in}{1.909295in}}{\pgfqpoint{1.739718in}{1.898696in}}{\pgfqpoint{1.747532in}{1.890882in}}%
\pgfpathcurveto{\pgfqpoint{1.755346in}{1.883068in}}{\pgfqpoint{1.765945in}{1.878678in}}{\pgfqpoint{1.776995in}{1.878678in}}%
\pgfpathclose%
\pgfusepath{stroke,fill}%
\end{pgfscope}%
\begin{pgfscope}%
\pgfpathrectangle{\pgfqpoint{0.511823in}{0.504323in}}{\pgfqpoint{3.218177in}{3.225677in}} %
\pgfusepath{clip}%
\pgfsetbuttcap%
\pgfsetroundjoin%
\definecolor{currentfill}{rgb}{0.000000,0.000000,0.545098}%
\pgfsetfillcolor{currentfill}%
\pgfsetfillopacity{0.400000}%
\pgfsetlinewidth{0.501875pt}%
\definecolor{currentstroke}{rgb}{0.000000,0.000000,0.545098}%
\pgfsetstrokecolor{currentstroke}%
\pgfsetstrokeopacity{0.400000}%
\pgfsetdash{}{0pt}%
\pgfpathmoveto{\pgfqpoint{1.938969in}{2.071861in}}%
\pgfpathcurveto{\pgfqpoint{1.950019in}{2.071861in}}{\pgfqpoint{1.960618in}{2.076251in}}{\pgfqpoint{1.968432in}{2.084065in}}%
\pgfpathcurveto{\pgfqpoint{1.976246in}{2.091878in}}{\pgfqpoint{1.980636in}{2.102477in}}{\pgfqpoint{1.980636in}{2.113528in}}%
\pgfpathcurveto{\pgfqpoint{1.980636in}{2.124578in}}{\pgfqpoint{1.976246in}{2.135177in}}{\pgfqpoint{1.968432in}{2.142990in}}%
\pgfpathcurveto{\pgfqpoint{1.960618in}{2.150804in}}{\pgfqpoint{1.950019in}{2.155194in}}{\pgfqpoint{1.938969in}{2.155194in}}%
\pgfpathcurveto{\pgfqpoint{1.927919in}{2.155194in}}{\pgfqpoint{1.917320in}{2.150804in}}{\pgfqpoint{1.909506in}{2.142990in}}%
\pgfpathcurveto{\pgfqpoint{1.901693in}{2.135177in}}{\pgfqpoint{1.897303in}{2.124578in}}{\pgfqpoint{1.897303in}{2.113528in}}%
\pgfpathcurveto{\pgfqpoint{1.897303in}{2.102477in}}{\pgfqpoint{1.901693in}{2.091878in}}{\pgfqpoint{1.909506in}{2.084065in}}%
\pgfpathcurveto{\pgfqpoint{1.917320in}{2.076251in}}{\pgfqpoint{1.927919in}{2.071861in}}{\pgfqpoint{1.938969in}{2.071861in}}%
\pgfpathclose%
\pgfusepath{stroke,fill}%
\end{pgfscope}%
\begin{pgfscope}%
\pgfpathrectangle{\pgfqpoint{0.511823in}{0.504323in}}{\pgfqpoint{3.218177in}{3.225677in}} %
\pgfusepath{clip}%
\pgfsetbuttcap%
\pgfsetroundjoin%
\definecolor{currentfill}{rgb}{0.000000,0.000000,0.545098}%
\pgfsetfillcolor{currentfill}%
\pgfsetfillopacity{0.400000}%
\pgfsetlinewidth{0.501875pt}%
\definecolor{currentstroke}{rgb}{0.000000,0.000000,0.545098}%
\pgfsetstrokecolor{currentstroke}%
\pgfsetstrokeopacity{0.400000}%
\pgfsetdash{}{0pt}%
\pgfpathmoveto{\pgfqpoint{2.027246in}{2.182689in}}%
\pgfpathcurveto{\pgfqpoint{2.038296in}{2.182689in}}{\pgfqpoint{2.048895in}{2.187079in}}{\pgfqpoint{2.056708in}{2.194892in}}%
\pgfpathcurveto{\pgfqpoint{2.064522in}{2.202706in}}{\pgfqpoint{2.068912in}{2.213305in}}{\pgfqpoint{2.068912in}{2.224355in}}%
\pgfpathcurveto{\pgfqpoint{2.068912in}{2.235405in}}{\pgfqpoint{2.064522in}{2.246004in}}{\pgfqpoint{2.056708in}{2.253818in}}%
\pgfpathcurveto{\pgfqpoint{2.048895in}{2.261632in}}{\pgfqpoint{2.038296in}{2.266022in}}{\pgfqpoint{2.027246in}{2.266022in}}%
\pgfpathcurveto{\pgfqpoint{2.016195in}{2.266022in}}{\pgfqpoint{2.005596in}{2.261632in}}{\pgfqpoint{1.997783in}{2.253818in}}%
\pgfpathcurveto{\pgfqpoint{1.989969in}{2.246004in}}{\pgfqpoint{1.985579in}{2.235405in}}{\pgfqpoint{1.985579in}{2.224355in}}%
\pgfpathcurveto{\pgfqpoint{1.985579in}{2.213305in}}{\pgfqpoint{1.989969in}{2.202706in}}{\pgfqpoint{1.997783in}{2.194892in}}%
\pgfpathcurveto{\pgfqpoint{2.005596in}{2.187079in}}{\pgfqpoint{2.016195in}{2.182689in}}{\pgfqpoint{2.027246in}{2.182689in}}%
\pgfpathclose%
\pgfusepath{stroke,fill}%
\end{pgfscope}%
\begin{pgfscope}%
\pgfpathrectangle{\pgfqpoint{0.511823in}{0.504323in}}{\pgfqpoint{3.218177in}{3.225677in}} %
\pgfusepath{clip}%
\pgfsetbuttcap%
\pgfsetroundjoin%
\definecolor{currentfill}{rgb}{0.000000,0.000000,0.545098}%
\pgfsetfillcolor{currentfill}%
\pgfsetfillopacity{0.400000}%
\pgfsetlinewidth{0.501875pt}%
\definecolor{currentstroke}{rgb}{0.000000,0.000000,0.545098}%
\pgfsetstrokecolor{currentstroke}%
\pgfsetstrokeopacity{0.400000}%
\pgfsetdash{}{0pt}%
\pgfpathmoveto{\pgfqpoint{1.808037in}{1.938778in}}%
\pgfpathcurveto{\pgfqpoint{1.819088in}{1.938778in}}{\pgfqpoint{1.829687in}{1.943168in}}{\pgfqpoint{1.837500in}{1.950982in}}%
\pgfpathcurveto{\pgfqpoint{1.845314in}{1.958796in}}{\pgfqpoint{1.849704in}{1.969395in}}{\pgfqpoint{1.849704in}{1.980445in}}%
\pgfpathcurveto{\pgfqpoint{1.849704in}{1.991495in}}{\pgfqpoint{1.845314in}{2.002094in}}{\pgfqpoint{1.837500in}{2.009908in}}%
\pgfpathcurveto{\pgfqpoint{1.829687in}{2.017721in}}{\pgfqpoint{1.819088in}{2.022112in}}{\pgfqpoint{1.808037in}{2.022112in}}%
\pgfpathcurveto{\pgfqpoint{1.796987in}{2.022112in}}{\pgfqpoint{1.786388in}{2.017721in}}{\pgfqpoint{1.778575in}{2.009908in}}%
\pgfpathcurveto{\pgfqpoint{1.770761in}{2.002094in}}{\pgfqpoint{1.766371in}{1.991495in}}{\pgfqpoint{1.766371in}{1.980445in}}%
\pgfpathcurveto{\pgfqpoint{1.766371in}{1.969395in}}{\pgfqpoint{1.770761in}{1.958796in}}{\pgfqpoint{1.778575in}{1.950982in}}%
\pgfpathcurveto{\pgfqpoint{1.786388in}{1.943168in}}{\pgfqpoint{1.796987in}{1.938778in}}{\pgfqpoint{1.808037in}{1.938778in}}%
\pgfpathclose%
\pgfusepath{stroke,fill}%
\end{pgfscope}%
\begin{pgfscope}%
\pgfpathrectangle{\pgfqpoint{0.511823in}{0.504323in}}{\pgfqpoint{3.218177in}{3.225677in}} %
\pgfusepath{clip}%
\pgfsetbuttcap%
\pgfsetroundjoin%
\definecolor{currentfill}{rgb}{0.000000,0.000000,0.545098}%
\pgfsetfillcolor{currentfill}%
\pgfsetfillopacity{0.400000}%
\pgfsetlinewidth{0.501875pt}%
\definecolor{currentstroke}{rgb}{0.000000,0.000000,0.545098}%
\pgfsetstrokecolor{currentstroke}%
\pgfsetstrokeopacity{0.400000}%
\pgfsetdash{}{0pt}%
\pgfpathmoveto{\pgfqpoint{2.025982in}{2.201240in}}%
\pgfpathcurveto{\pgfqpoint{2.037032in}{2.201240in}}{\pgfqpoint{2.047631in}{2.205630in}}{\pgfqpoint{2.055445in}{2.213444in}}%
\pgfpathcurveto{\pgfqpoint{2.063259in}{2.221258in}}{\pgfqpoint{2.067649in}{2.231857in}}{\pgfqpoint{2.067649in}{2.242907in}}%
\pgfpathcurveto{\pgfqpoint{2.067649in}{2.253957in}}{\pgfqpoint{2.063259in}{2.264556in}}{\pgfqpoint{2.055445in}{2.272370in}}%
\pgfpathcurveto{\pgfqpoint{2.047631in}{2.280183in}}{\pgfqpoint{2.037032in}{2.284574in}}{\pgfqpoint{2.025982in}{2.284574in}}%
\pgfpathcurveto{\pgfqpoint{2.014932in}{2.284574in}}{\pgfqpoint{2.004333in}{2.280183in}}{\pgfqpoint{1.996519in}{2.272370in}}%
\pgfpathcurveto{\pgfqpoint{1.988706in}{2.264556in}}{\pgfqpoint{1.984316in}{2.253957in}}{\pgfqpoint{1.984316in}{2.242907in}}%
\pgfpathcurveto{\pgfqpoint{1.984316in}{2.231857in}}{\pgfqpoint{1.988706in}{2.221258in}}{\pgfqpoint{1.996519in}{2.213444in}}%
\pgfpathcurveto{\pgfqpoint{2.004333in}{2.205630in}}{\pgfqpoint{2.014932in}{2.201240in}}{\pgfqpoint{2.025982in}{2.201240in}}%
\pgfpathclose%
\pgfusepath{stroke,fill}%
\end{pgfscope}%
\begin{pgfscope}%
\pgfpathrectangle{\pgfqpoint{0.511823in}{0.504323in}}{\pgfqpoint{3.218177in}{3.225677in}} %
\pgfusepath{clip}%
\pgfsetbuttcap%
\pgfsetroundjoin%
\definecolor{currentfill}{rgb}{0.000000,0.000000,0.545098}%
\pgfsetfillcolor{currentfill}%
\pgfsetfillopacity{0.400000}%
\pgfsetlinewidth{0.501875pt}%
\definecolor{currentstroke}{rgb}{0.000000,0.000000,0.545098}%
\pgfsetstrokecolor{currentstroke}%
\pgfsetstrokeopacity{0.400000}%
\pgfsetdash{}{0pt}%
\pgfpathmoveto{\pgfqpoint{1.864000in}{2.021319in}}%
\pgfpathcurveto{\pgfqpoint{1.875051in}{2.021319in}}{\pgfqpoint{1.885650in}{2.025709in}}{\pgfqpoint{1.893463in}{2.033523in}}%
\pgfpathcurveto{\pgfqpoint{1.901277in}{2.041336in}}{\pgfqpoint{1.905667in}{2.051935in}}{\pgfqpoint{1.905667in}{2.062985in}}%
\pgfpathcurveto{\pgfqpoint{1.905667in}{2.074036in}}{\pgfqpoint{1.901277in}{2.084635in}}{\pgfqpoint{1.893463in}{2.092448in}}%
\pgfpathcurveto{\pgfqpoint{1.885650in}{2.100262in}}{\pgfqpoint{1.875051in}{2.104652in}}{\pgfqpoint{1.864000in}{2.104652in}}%
\pgfpathcurveto{\pgfqpoint{1.852950in}{2.104652in}}{\pgfqpoint{1.842351in}{2.100262in}}{\pgfqpoint{1.834538in}{2.092448in}}%
\pgfpathcurveto{\pgfqpoint{1.826724in}{2.084635in}}{\pgfqpoint{1.822334in}{2.074036in}}{\pgfqpoint{1.822334in}{2.062985in}}%
\pgfpathcurveto{\pgfqpoint{1.822334in}{2.051935in}}{\pgfqpoint{1.826724in}{2.041336in}}{\pgfqpoint{1.834538in}{2.033523in}}%
\pgfpathcurveto{\pgfqpoint{1.842351in}{2.025709in}}{\pgfqpoint{1.852950in}{2.021319in}}{\pgfqpoint{1.864000in}{2.021319in}}%
\pgfpathclose%
\pgfusepath{stroke,fill}%
\end{pgfscope}%
\begin{pgfscope}%
\pgfpathrectangle{\pgfqpoint{0.511823in}{0.504323in}}{\pgfqpoint{3.218177in}{3.225677in}} %
\pgfusepath{clip}%
\pgfsetbuttcap%
\pgfsetroundjoin%
\definecolor{currentfill}{rgb}{0.000000,0.000000,0.545098}%
\pgfsetfillcolor{currentfill}%
\pgfsetfillopacity{0.400000}%
\pgfsetlinewidth{0.501875pt}%
\definecolor{currentstroke}{rgb}{0.000000,0.000000,0.545098}%
\pgfsetstrokecolor{currentstroke}%
\pgfsetstrokeopacity{0.400000}%
\pgfsetdash{}{0pt}%
\pgfpathmoveto{\pgfqpoint{1.913826in}{2.089124in}}%
\pgfpathcurveto{\pgfqpoint{1.924876in}{2.089124in}}{\pgfqpoint{1.935475in}{2.093514in}}{\pgfqpoint{1.943289in}{2.101328in}}%
\pgfpathcurveto{\pgfqpoint{1.951103in}{2.109142in}}{\pgfqpoint{1.955493in}{2.119741in}}{\pgfqpoint{1.955493in}{2.130791in}}%
\pgfpathcurveto{\pgfqpoint{1.955493in}{2.141841in}}{\pgfqpoint{1.951103in}{2.152440in}}{\pgfqpoint{1.943289in}{2.160253in}}%
\pgfpathcurveto{\pgfqpoint{1.935475in}{2.168067in}}{\pgfqpoint{1.924876in}{2.172457in}}{\pgfqpoint{1.913826in}{2.172457in}}%
\pgfpathcurveto{\pgfqpoint{1.902776in}{2.172457in}}{\pgfqpoint{1.892177in}{2.168067in}}{\pgfqpoint{1.884363in}{2.160253in}}%
\pgfpathcurveto{\pgfqpoint{1.876550in}{2.152440in}}{\pgfqpoint{1.872160in}{2.141841in}}{\pgfqpoint{1.872160in}{2.130791in}}%
\pgfpathcurveto{\pgfqpoint{1.872160in}{2.119741in}}{\pgfqpoint{1.876550in}{2.109142in}}{\pgfqpoint{1.884363in}{2.101328in}}%
\pgfpathcurveto{\pgfqpoint{1.892177in}{2.093514in}}{\pgfqpoint{1.902776in}{2.089124in}}{\pgfqpoint{1.913826in}{2.089124in}}%
\pgfpathclose%
\pgfusepath{stroke,fill}%
\end{pgfscope}%
\begin{pgfscope}%
\pgfpathrectangle{\pgfqpoint{0.511823in}{0.504323in}}{\pgfqpoint{3.218177in}{3.225677in}} %
\pgfusepath{clip}%
\pgfsetbuttcap%
\pgfsetroundjoin%
\definecolor{currentfill}{rgb}{0.000000,0.000000,0.545098}%
\pgfsetfillcolor{currentfill}%
\pgfsetfillopacity{0.400000}%
\pgfsetlinewidth{0.501875pt}%
\definecolor{currentstroke}{rgb}{0.000000,0.000000,0.545098}%
\pgfsetstrokecolor{currentstroke}%
\pgfsetstrokeopacity{0.400000}%
\pgfsetdash{}{0pt}%
\pgfpathmoveto{\pgfqpoint{1.686207in}{1.828015in}}%
\pgfpathcurveto{\pgfqpoint{1.697258in}{1.828015in}}{\pgfqpoint{1.707857in}{1.832406in}}{\pgfqpoint{1.715670in}{1.840219in}}%
\pgfpathcurveto{\pgfqpoint{1.723484in}{1.848033in}}{\pgfqpoint{1.727874in}{1.858632in}}{\pgfqpoint{1.727874in}{1.869682in}}%
\pgfpathcurveto{\pgfqpoint{1.727874in}{1.880732in}}{\pgfqpoint{1.723484in}{1.891331in}}{\pgfqpoint{1.715670in}{1.899145in}}%
\pgfpathcurveto{\pgfqpoint{1.707857in}{1.906958in}}{\pgfqpoint{1.697258in}{1.911349in}}{\pgfqpoint{1.686207in}{1.911349in}}%
\pgfpathcurveto{\pgfqpoint{1.675157in}{1.911349in}}{\pgfqpoint{1.664558in}{1.906958in}}{\pgfqpoint{1.656745in}{1.899145in}}%
\pgfpathcurveto{\pgfqpoint{1.648931in}{1.891331in}}{\pgfqpoint{1.644541in}{1.880732in}}{\pgfqpoint{1.644541in}{1.869682in}}%
\pgfpathcurveto{\pgfqpoint{1.644541in}{1.858632in}}{\pgfqpoint{1.648931in}{1.848033in}}{\pgfqpoint{1.656745in}{1.840219in}}%
\pgfpathcurveto{\pgfqpoint{1.664558in}{1.832406in}}{\pgfqpoint{1.675157in}{1.828015in}}{\pgfqpoint{1.686207in}{1.828015in}}%
\pgfpathclose%
\pgfusepath{stroke,fill}%
\end{pgfscope}%
\begin{pgfscope}%
\pgfpathrectangle{\pgfqpoint{0.511823in}{0.504323in}}{\pgfqpoint{3.218177in}{3.225677in}} %
\pgfusepath{clip}%
\pgfsetbuttcap%
\pgfsetroundjoin%
\definecolor{currentfill}{rgb}{0.000000,0.000000,0.545098}%
\pgfsetfillcolor{currentfill}%
\pgfsetfillopacity{0.400000}%
\pgfsetlinewidth{0.501875pt}%
\definecolor{currentstroke}{rgb}{0.000000,0.000000,0.545098}%
\pgfsetstrokecolor{currentstroke}%
\pgfsetstrokeopacity{0.400000}%
\pgfsetdash{}{0pt}%
\pgfpathmoveto{\pgfqpoint{1.996021in}{2.206373in}}%
\pgfpathcurveto{\pgfqpoint{2.007071in}{2.206373in}}{\pgfqpoint{2.017670in}{2.210763in}}{\pgfqpoint{2.025484in}{2.218577in}}%
\pgfpathcurveto{\pgfqpoint{2.033298in}{2.226391in}}{\pgfqpoint{2.037688in}{2.236990in}}{\pgfqpoint{2.037688in}{2.248040in}}%
\pgfpathcurveto{\pgfqpoint{2.037688in}{2.259090in}}{\pgfqpoint{2.033298in}{2.269689in}}{\pgfqpoint{2.025484in}{2.277503in}}%
\pgfpathcurveto{\pgfqpoint{2.017670in}{2.285316in}}{\pgfqpoint{2.007071in}{2.289706in}}{\pgfqpoint{1.996021in}{2.289706in}}%
\pgfpathcurveto{\pgfqpoint{1.984971in}{2.289706in}}{\pgfqpoint{1.974372in}{2.285316in}}{\pgfqpoint{1.966558in}{2.277503in}}%
\pgfpathcurveto{\pgfqpoint{1.958745in}{2.269689in}}{\pgfqpoint{1.954354in}{2.259090in}}{\pgfqpoint{1.954354in}{2.248040in}}%
\pgfpathcurveto{\pgfqpoint{1.954354in}{2.236990in}}{\pgfqpoint{1.958745in}{2.226391in}}{\pgfqpoint{1.966558in}{2.218577in}}%
\pgfpathcurveto{\pgfqpoint{1.974372in}{2.210763in}}{\pgfqpoint{1.984971in}{2.206373in}}{\pgfqpoint{1.996021in}{2.206373in}}%
\pgfpathclose%
\pgfusepath{stroke,fill}%
\end{pgfscope}%
\begin{pgfscope}%
\pgfpathrectangle{\pgfqpoint{0.511823in}{0.504323in}}{\pgfqpoint{3.218177in}{3.225677in}} %
\pgfusepath{clip}%
\pgfsetbuttcap%
\pgfsetroundjoin%
\definecolor{currentfill}{rgb}{0.000000,0.000000,0.545098}%
\pgfsetfillcolor{currentfill}%
\pgfsetfillopacity{0.400000}%
\pgfsetlinewidth{0.501875pt}%
\definecolor{currentstroke}{rgb}{0.000000,0.000000,0.545098}%
\pgfsetstrokecolor{currentstroke}%
\pgfsetstrokeopacity{0.400000}%
\pgfsetdash{}{0pt}%
\pgfpathmoveto{\pgfqpoint{1.832183in}{2.019318in}}%
\pgfpathcurveto{\pgfqpoint{1.843234in}{2.019318in}}{\pgfqpoint{1.853833in}{2.023709in}}{\pgfqpoint{1.861646in}{2.031522in}}%
\pgfpathcurveto{\pgfqpoint{1.869460in}{2.039336in}}{\pgfqpoint{1.873850in}{2.049935in}}{\pgfqpoint{1.873850in}{2.060985in}}%
\pgfpathcurveto{\pgfqpoint{1.873850in}{2.072035in}}{\pgfqpoint{1.869460in}{2.082634in}}{\pgfqpoint{1.861646in}{2.090448in}}%
\pgfpathcurveto{\pgfqpoint{1.853833in}{2.098261in}}{\pgfqpoint{1.843234in}{2.102652in}}{\pgfqpoint{1.832183in}{2.102652in}}%
\pgfpathcurveto{\pgfqpoint{1.821133in}{2.102652in}}{\pgfqpoint{1.810534in}{2.098261in}}{\pgfqpoint{1.802721in}{2.090448in}}%
\pgfpathcurveto{\pgfqpoint{1.794907in}{2.082634in}}{\pgfqpoint{1.790517in}{2.072035in}}{\pgfqpoint{1.790517in}{2.060985in}}%
\pgfpathcurveto{\pgfqpoint{1.790517in}{2.049935in}}{\pgfqpoint{1.794907in}{2.039336in}}{\pgfqpoint{1.802721in}{2.031522in}}%
\pgfpathcurveto{\pgfqpoint{1.810534in}{2.023709in}}{\pgfqpoint{1.821133in}{2.019318in}}{\pgfqpoint{1.832183in}{2.019318in}}%
\pgfpathclose%
\pgfusepath{stroke,fill}%
\end{pgfscope}%
\begin{pgfscope}%
\pgfpathrectangle{\pgfqpoint{0.511823in}{0.504323in}}{\pgfqpoint{3.218177in}{3.225677in}} %
\pgfusepath{clip}%
\pgfsetbuttcap%
\pgfsetroundjoin%
\definecolor{currentfill}{rgb}{0.000000,0.000000,0.545098}%
\pgfsetfillcolor{currentfill}%
\pgfsetfillopacity{0.400000}%
\pgfsetlinewidth{0.501875pt}%
\definecolor{currentstroke}{rgb}{0.000000,0.000000,0.545098}%
\pgfsetstrokecolor{currentstroke}%
\pgfsetstrokeopacity{0.400000}%
\pgfsetdash{}{0pt}%
\pgfpathmoveto{\pgfqpoint{1.854782in}{2.055719in}}%
\pgfpathcurveto{\pgfqpoint{1.865833in}{2.055719in}}{\pgfqpoint{1.876432in}{2.060110in}}{\pgfqpoint{1.884245in}{2.067923in}}%
\pgfpathcurveto{\pgfqpoint{1.892059in}{2.075737in}}{\pgfqpoint{1.896449in}{2.086336in}}{\pgfqpoint{1.896449in}{2.097386in}}%
\pgfpathcurveto{\pgfqpoint{1.896449in}{2.108436in}}{\pgfqpoint{1.892059in}{2.119035in}}{\pgfqpoint{1.884245in}{2.126849in}}%
\pgfpathcurveto{\pgfqpoint{1.876432in}{2.134662in}}{\pgfqpoint{1.865833in}{2.139053in}}{\pgfqpoint{1.854782in}{2.139053in}}%
\pgfpathcurveto{\pgfqpoint{1.843732in}{2.139053in}}{\pgfqpoint{1.833133in}{2.134662in}}{\pgfqpoint{1.825320in}{2.126849in}}%
\pgfpathcurveto{\pgfqpoint{1.817506in}{2.119035in}}{\pgfqpoint{1.813116in}{2.108436in}}{\pgfqpoint{1.813116in}{2.097386in}}%
\pgfpathcurveto{\pgfqpoint{1.813116in}{2.086336in}}{\pgfqpoint{1.817506in}{2.075737in}}{\pgfqpoint{1.825320in}{2.067923in}}%
\pgfpathcurveto{\pgfqpoint{1.833133in}{2.060110in}}{\pgfqpoint{1.843732in}{2.055719in}}{\pgfqpoint{1.854782in}{2.055719in}}%
\pgfpathclose%
\pgfusepath{stroke,fill}%
\end{pgfscope}%
\begin{pgfscope}%
\pgfpathrectangle{\pgfqpoint{0.511823in}{0.504323in}}{\pgfqpoint{3.218177in}{3.225677in}} %
\pgfusepath{clip}%
\pgfsetbuttcap%
\pgfsetroundjoin%
\definecolor{currentfill}{rgb}{0.000000,0.000000,0.545098}%
\pgfsetfillcolor{currentfill}%
\pgfsetfillopacity{0.400000}%
\pgfsetlinewidth{0.501875pt}%
\definecolor{currentstroke}{rgb}{0.000000,0.000000,0.545098}%
\pgfsetstrokecolor{currentstroke}%
\pgfsetstrokeopacity{0.400000}%
\pgfsetdash{}{0pt}%
\pgfpathmoveto{\pgfqpoint{1.790440in}{1.986496in}}%
\pgfpathcurveto{\pgfqpoint{1.801490in}{1.986496in}}{\pgfqpoint{1.812089in}{1.990886in}}{\pgfqpoint{1.819903in}{1.998700in}}%
\pgfpathcurveto{\pgfqpoint{1.827717in}{2.006514in}}{\pgfqpoint{1.832107in}{2.017113in}}{\pgfqpoint{1.832107in}{2.028163in}}%
\pgfpathcurveto{\pgfqpoint{1.832107in}{2.039213in}}{\pgfqpoint{1.827717in}{2.049812in}}{\pgfqpoint{1.819903in}{2.057626in}}%
\pgfpathcurveto{\pgfqpoint{1.812089in}{2.065439in}}{\pgfqpoint{1.801490in}{2.069829in}}{\pgfqpoint{1.790440in}{2.069829in}}%
\pgfpathcurveto{\pgfqpoint{1.779390in}{2.069829in}}{\pgfqpoint{1.768791in}{2.065439in}}{\pgfqpoint{1.760977in}{2.057626in}}%
\pgfpathcurveto{\pgfqpoint{1.753164in}{2.049812in}}{\pgfqpoint{1.748774in}{2.039213in}}{\pgfqpoint{1.748774in}{2.028163in}}%
\pgfpathcurveto{\pgfqpoint{1.748774in}{2.017113in}}{\pgfqpoint{1.753164in}{2.006514in}}{\pgfqpoint{1.760977in}{1.998700in}}%
\pgfpathcurveto{\pgfqpoint{1.768791in}{1.990886in}}{\pgfqpoint{1.779390in}{1.986496in}}{\pgfqpoint{1.790440in}{1.986496in}}%
\pgfpathclose%
\pgfusepath{stroke,fill}%
\end{pgfscope}%
\begin{pgfscope}%
\pgfpathrectangle{\pgfqpoint{0.511823in}{0.504323in}}{\pgfqpoint{3.218177in}{3.225677in}} %
\pgfusepath{clip}%
\pgfsetbuttcap%
\pgfsetroundjoin%
\definecolor{currentfill}{rgb}{0.000000,0.000000,0.545098}%
\pgfsetfillcolor{currentfill}%
\pgfsetfillopacity{0.400000}%
\pgfsetlinewidth{0.501875pt}%
\definecolor{currentstroke}{rgb}{0.000000,0.000000,0.545098}%
\pgfsetstrokecolor{currentstroke}%
\pgfsetstrokeopacity{0.400000}%
\pgfsetdash{}{0pt}%
\pgfpathmoveto{\pgfqpoint{1.808750in}{2.017805in}}%
\pgfpathcurveto{\pgfqpoint{1.819800in}{2.017805in}}{\pgfqpoint{1.830399in}{2.022196in}}{\pgfqpoint{1.838212in}{2.030009in}}%
\pgfpathcurveto{\pgfqpoint{1.846026in}{2.037823in}}{\pgfqpoint{1.850416in}{2.048422in}}{\pgfqpoint{1.850416in}{2.059472in}}%
\pgfpathcurveto{\pgfqpoint{1.850416in}{2.070522in}}{\pgfqpoint{1.846026in}{2.081121in}}{\pgfqpoint{1.838212in}{2.088935in}}%
\pgfpathcurveto{\pgfqpoint{1.830399in}{2.096749in}}{\pgfqpoint{1.819800in}{2.101139in}}{\pgfqpoint{1.808750in}{2.101139in}}%
\pgfpathcurveto{\pgfqpoint{1.797699in}{2.101139in}}{\pgfqpoint{1.787100in}{2.096749in}}{\pgfqpoint{1.779287in}{2.088935in}}%
\pgfpathcurveto{\pgfqpoint{1.771473in}{2.081121in}}{\pgfqpoint{1.767083in}{2.070522in}}{\pgfqpoint{1.767083in}{2.059472in}}%
\pgfpathcurveto{\pgfqpoint{1.767083in}{2.048422in}}{\pgfqpoint{1.771473in}{2.037823in}}{\pgfqpoint{1.779287in}{2.030009in}}%
\pgfpathcurveto{\pgfqpoint{1.787100in}{2.022196in}}{\pgfqpoint{1.797699in}{2.017805in}}{\pgfqpoint{1.808750in}{2.017805in}}%
\pgfpathclose%
\pgfusepath{stroke,fill}%
\end{pgfscope}%
\begin{pgfscope}%
\pgfpathrectangle{\pgfqpoint{0.511823in}{0.504323in}}{\pgfqpoint{3.218177in}{3.225677in}} %
\pgfusepath{clip}%
\pgfsetbuttcap%
\pgfsetroundjoin%
\definecolor{currentfill}{rgb}{0.000000,0.000000,0.545098}%
\pgfsetfillcolor{currentfill}%
\pgfsetfillopacity{0.400000}%
\pgfsetlinewidth{0.501875pt}%
\definecolor{currentstroke}{rgb}{0.000000,0.000000,0.545098}%
\pgfsetstrokecolor{currentstroke}%
\pgfsetstrokeopacity{0.400000}%
\pgfsetdash{}{0pt}%
\pgfpathmoveto{\pgfqpoint{1.974530in}{2.231747in}}%
\pgfpathcurveto{\pgfqpoint{1.985580in}{2.231747in}}{\pgfqpoint{1.996179in}{2.236137in}}{\pgfqpoint{2.003993in}{2.243951in}}%
\pgfpathcurveto{\pgfqpoint{2.011806in}{2.251765in}}{\pgfqpoint{2.016197in}{2.262364in}}{\pgfqpoint{2.016197in}{2.273414in}}%
\pgfpathcurveto{\pgfqpoint{2.016197in}{2.284464in}}{\pgfqpoint{2.011806in}{2.295063in}}{\pgfqpoint{2.003993in}{2.302877in}}%
\pgfpathcurveto{\pgfqpoint{1.996179in}{2.310690in}}{\pgfqpoint{1.985580in}{2.315081in}}{\pgfqpoint{1.974530in}{2.315081in}}%
\pgfpathcurveto{\pgfqpoint{1.963480in}{2.315081in}}{\pgfqpoint{1.952881in}{2.310690in}}{\pgfqpoint{1.945067in}{2.302877in}}%
\pgfpathcurveto{\pgfqpoint{1.937253in}{2.295063in}}{\pgfqpoint{1.932863in}{2.284464in}}{\pgfqpoint{1.932863in}{2.273414in}}%
\pgfpathcurveto{\pgfqpoint{1.932863in}{2.262364in}}{\pgfqpoint{1.937253in}{2.251765in}}{\pgfqpoint{1.945067in}{2.243951in}}%
\pgfpathcurveto{\pgfqpoint{1.952881in}{2.236137in}}{\pgfqpoint{1.963480in}{2.231747in}}{\pgfqpoint{1.974530in}{2.231747in}}%
\pgfpathclose%
\pgfusepath{stroke,fill}%
\end{pgfscope}%
\begin{pgfscope}%
\pgfpathrectangle{\pgfqpoint{0.511823in}{0.504323in}}{\pgfqpoint{3.218177in}{3.225677in}} %
\pgfusepath{clip}%
\pgfsetbuttcap%
\pgfsetroundjoin%
\definecolor{currentfill}{rgb}{0.000000,0.000000,0.545098}%
\pgfsetfillcolor{currentfill}%
\pgfsetfillopacity{0.400000}%
\pgfsetlinewidth{0.501875pt}%
\definecolor{currentstroke}{rgb}{0.000000,0.000000,0.545098}%
\pgfsetstrokecolor{currentstroke}%
\pgfsetstrokeopacity{0.400000}%
\pgfsetdash{}{0pt}%
\pgfpathmoveto{\pgfqpoint{1.837836in}{2.072165in}}%
\pgfpathcurveto{\pgfqpoint{1.848886in}{2.072165in}}{\pgfqpoint{1.859485in}{2.076555in}}{\pgfqpoint{1.867299in}{2.084369in}}%
\pgfpathcurveto{\pgfqpoint{1.875112in}{2.092183in}}{\pgfqpoint{1.879503in}{2.102782in}}{\pgfqpoint{1.879503in}{2.113832in}}%
\pgfpathcurveto{\pgfqpoint{1.879503in}{2.124882in}}{\pgfqpoint{1.875112in}{2.135481in}}{\pgfqpoint{1.867299in}{2.143295in}}%
\pgfpathcurveto{\pgfqpoint{1.859485in}{2.151108in}}{\pgfqpoint{1.848886in}{2.155499in}}{\pgfqpoint{1.837836in}{2.155499in}}%
\pgfpathcurveto{\pgfqpoint{1.826786in}{2.155499in}}{\pgfqpoint{1.816187in}{2.151108in}}{\pgfqpoint{1.808373in}{2.143295in}}%
\pgfpathcurveto{\pgfqpoint{1.800559in}{2.135481in}}{\pgfqpoint{1.796169in}{2.124882in}}{\pgfqpoint{1.796169in}{2.113832in}}%
\pgfpathcurveto{\pgfqpoint{1.796169in}{2.102782in}}{\pgfqpoint{1.800559in}{2.092183in}}{\pgfqpoint{1.808373in}{2.084369in}}%
\pgfpathcurveto{\pgfqpoint{1.816187in}{2.076555in}}{\pgfqpoint{1.826786in}{2.072165in}}{\pgfqpoint{1.837836in}{2.072165in}}%
\pgfpathclose%
\pgfusepath{stroke,fill}%
\end{pgfscope}%
\begin{pgfscope}%
\pgfpathrectangle{\pgfqpoint{0.511823in}{0.504323in}}{\pgfqpoint{3.218177in}{3.225677in}} %
\pgfusepath{clip}%
\pgfsetbuttcap%
\pgfsetroundjoin%
\definecolor{currentfill}{rgb}{0.000000,0.000000,0.545098}%
\pgfsetfillcolor{currentfill}%
\pgfsetfillopacity{0.400000}%
\pgfsetlinewidth{0.501875pt}%
\definecolor{currentstroke}{rgb}{0.000000,0.000000,0.545098}%
\pgfsetstrokecolor{currentstroke}%
\pgfsetstrokeopacity{0.400000}%
\pgfsetdash{}{0pt}%
\pgfpathmoveto{\pgfqpoint{1.944077in}{2.214647in}}%
\pgfpathcurveto{\pgfqpoint{1.955127in}{2.214647in}}{\pgfqpoint{1.965726in}{2.219037in}}{\pgfqpoint{1.973539in}{2.226851in}}%
\pgfpathcurveto{\pgfqpoint{1.981353in}{2.234664in}}{\pgfqpoint{1.985743in}{2.245263in}}{\pgfqpoint{1.985743in}{2.256314in}}%
\pgfpathcurveto{\pgfqpoint{1.985743in}{2.267364in}}{\pgfqpoint{1.981353in}{2.277963in}}{\pgfqpoint{1.973539in}{2.285776in}}%
\pgfpathcurveto{\pgfqpoint{1.965726in}{2.293590in}}{\pgfqpoint{1.955127in}{2.297980in}}{\pgfqpoint{1.944077in}{2.297980in}}%
\pgfpathcurveto{\pgfqpoint{1.933026in}{2.297980in}}{\pgfqpoint{1.922427in}{2.293590in}}{\pgfqpoint{1.914614in}{2.285776in}}%
\pgfpathcurveto{\pgfqpoint{1.906800in}{2.277963in}}{\pgfqpoint{1.902410in}{2.267364in}}{\pgfqpoint{1.902410in}{2.256314in}}%
\pgfpathcurveto{\pgfqpoint{1.902410in}{2.245263in}}{\pgfqpoint{1.906800in}{2.234664in}}{\pgfqpoint{1.914614in}{2.226851in}}%
\pgfpathcurveto{\pgfqpoint{1.922427in}{2.219037in}}{\pgfqpoint{1.933026in}{2.214647in}}{\pgfqpoint{1.944077in}{2.214647in}}%
\pgfpathclose%
\pgfusepath{stroke,fill}%
\end{pgfscope}%
\begin{pgfscope}%
\pgfpathrectangle{\pgfqpoint{0.511823in}{0.504323in}}{\pgfqpoint{3.218177in}{3.225677in}} %
\pgfusepath{clip}%
\pgfsetbuttcap%
\pgfsetroundjoin%
\definecolor{currentfill}{rgb}{0.000000,0.000000,0.545098}%
\pgfsetfillcolor{currentfill}%
\pgfsetfillopacity{0.400000}%
\pgfsetlinewidth{0.501875pt}%
\definecolor{currentstroke}{rgb}{0.000000,0.000000,0.545098}%
\pgfsetstrokecolor{currentstroke}%
\pgfsetstrokeopacity{0.400000}%
\pgfsetdash{}{0pt}%
\pgfpathmoveto{\pgfqpoint{1.854346in}{2.111939in}}%
\pgfpathcurveto{\pgfqpoint{1.865397in}{2.111939in}}{\pgfqpoint{1.875996in}{2.116330in}}{\pgfqpoint{1.883809in}{2.124143in}}%
\pgfpathcurveto{\pgfqpoint{1.891623in}{2.131957in}}{\pgfqpoint{1.896013in}{2.142556in}}{\pgfqpoint{1.896013in}{2.153606in}}%
\pgfpathcurveto{\pgfqpoint{1.896013in}{2.164656in}}{\pgfqpoint{1.891623in}{2.175255in}}{\pgfqpoint{1.883809in}{2.183069in}}%
\pgfpathcurveto{\pgfqpoint{1.875996in}{2.190882in}}{\pgfqpoint{1.865397in}{2.195273in}}{\pgfqpoint{1.854346in}{2.195273in}}%
\pgfpathcurveto{\pgfqpoint{1.843296in}{2.195273in}}{\pgfqpoint{1.832697in}{2.190882in}}{\pgfqpoint{1.824884in}{2.183069in}}%
\pgfpathcurveto{\pgfqpoint{1.817070in}{2.175255in}}{\pgfqpoint{1.812680in}{2.164656in}}{\pgfqpoint{1.812680in}{2.153606in}}%
\pgfpathcurveto{\pgfqpoint{1.812680in}{2.142556in}}{\pgfqpoint{1.817070in}{2.131957in}}{\pgfqpoint{1.824884in}{2.124143in}}%
\pgfpathcurveto{\pgfqpoint{1.832697in}{2.116330in}}{\pgfqpoint{1.843296in}{2.111939in}}{\pgfqpoint{1.854346in}{2.111939in}}%
\pgfpathclose%
\pgfusepath{stroke,fill}%
\end{pgfscope}%
\begin{pgfscope}%
\pgfpathrectangle{\pgfqpoint{0.511823in}{0.504323in}}{\pgfqpoint{3.218177in}{3.225677in}} %
\pgfusepath{clip}%
\pgfsetbuttcap%
\pgfsetroundjoin%
\definecolor{currentfill}{rgb}{0.000000,0.000000,0.545098}%
\pgfsetfillcolor{currentfill}%
\pgfsetfillopacity{0.400000}%
\pgfsetlinewidth{0.501875pt}%
\definecolor{currentstroke}{rgb}{0.000000,0.000000,0.545098}%
\pgfsetstrokecolor{currentstroke}%
\pgfsetstrokeopacity{0.400000}%
\pgfsetdash{}{0pt}%
\pgfpathmoveto{\pgfqpoint{1.890333in}{2.167342in}}%
\pgfpathcurveto{\pgfqpoint{1.901383in}{2.167342in}}{\pgfqpoint{1.911982in}{2.171733in}}{\pgfqpoint{1.919796in}{2.179546in}}%
\pgfpathcurveto{\pgfqpoint{1.927610in}{2.187360in}}{\pgfqpoint{1.932000in}{2.197959in}}{\pgfqpoint{1.932000in}{2.209009in}}%
\pgfpathcurveto{\pgfqpoint{1.932000in}{2.220059in}}{\pgfqpoint{1.927610in}{2.230658in}}{\pgfqpoint{1.919796in}{2.238472in}}%
\pgfpathcurveto{\pgfqpoint{1.911982in}{2.246285in}}{\pgfqpoint{1.901383in}{2.250676in}}{\pgfqpoint{1.890333in}{2.250676in}}%
\pgfpathcurveto{\pgfqpoint{1.879283in}{2.250676in}}{\pgfqpoint{1.868684in}{2.246285in}}{\pgfqpoint{1.860870in}{2.238472in}}%
\pgfpathcurveto{\pgfqpoint{1.853057in}{2.230658in}}{\pgfqpoint{1.848667in}{2.220059in}}{\pgfqpoint{1.848667in}{2.209009in}}%
\pgfpathcurveto{\pgfqpoint{1.848667in}{2.197959in}}{\pgfqpoint{1.853057in}{2.187360in}}{\pgfqpoint{1.860870in}{2.179546in}}%
\pgfpathcurveto{\pgfqpoint{1.868684in}{2.171733in}}{\pgfqpoint{1.879283in}{2.167342in}}{\pgfqpoint{1.890333in}{2.167342in}}%
\pgfpathclose%
\pgfusepath{stroke,fill}%
\end{pgfscope}%
\begin{pgfscope}%
\pgfpathrectangle{\pgfqpoint{0.511823in}{0.504323in}}{\pgfqpoint{3.218177in}{3.225677in}} %
\pgfusepath{clip}%
\pgfsetbuttcap%
\pgfsetroundjoin%
\definecolor{currentfill}{rgb}{0.000000,0.000000,0.545098}%
\pgfsetfillcolor{currentfill}%
\pgfsetfillopacity{0.400000}%
\pgfsetlinewidth{0.501875pt}%
\definecolor{currentstroke}{rgb}{0.000000,0.000000,0.545098}%
\pgfsetstrokecolor{currentstroke}%
\pgfsetstrokeopacity{0.400000}%
\pgfsetdash{}{0pt}%
\pgfpathmoveto{\pgfqpoint{1.818577in}{2.085817in}}%
\pgfpathcurveto{\pgfqpoint{1.829627in}{2.085817in}}{\pgfqpoint{1.840226in}{2.090207in}}{\pgfqpoint{1.848039in}{2.098020in}}%
\pgfpathcurveto{\pgfqpoint{1.855853in}{2.105834in}}{\pgfqpoint{1.860243in}{2.116433in}}{\pgfqpoint{1.860243in}{2.127483in}}%
\pgfpathcurveto{\pgfqpoint{1.860243in}{2.138533in}}{\pgfqpoint{1.855853in}{2.149132in}}{\pgfqpoint{1.848039in}{2.156946in}}%
\pgfpathcurveto{\pgfqpoint{1.840226in}{2.164760in}}{\pgfqpoint{1.829627in}{2.169150in}}{\pgfqpoint{1.818577in}{2.169150in}}%
\pgfpathcurveto{\pgfqpoint{1.807527in}{2.169150in}}{\pgfqpoint{1.796927in}{2.164760in}}{\pgfqpoint{1.789114in}{2.156946in}}%
\pgfpathcurveto{\pgfqpoint{1.781300in}{2.149132in}}{\pgfqpoint{1.776910in}{2.138533in}}{\pgfqpoint{1.776910in}{2.127483in}}%
\pgfpathcurveto{\pgfqpoint{1.776910in}{2.116433in}}{\pgfqpoint{1.781300in}{2.105834in}}{\pgfqpoint{1.789114in}{2.098020in}}%
\pgfpathcurveto{\pgfqpoint{1.796927in}{2.090207in}}{\pgfqpoint{1.807527in}{2.085817in}}{\pgfqpoint{1.818577in}{2.085817in}}%
\pgfpathclose%
\pgfusepath{stroke,fill}%
\end{pgfscope}%
\begin{pgfscope}%
\pgfpathrectangle{\pgfqpoint{0.511823in}{0.504323in}}{\pgfqpoint{3.218177in}{3.225677in}} %
\pgfusepath{clip}%
\pgfsetbuttcap%
\pgfsetroundjoin%
\definecolor{currentfill}{rgb}{0.000000,0.000000,0.545098}%
\pgfsetfillcolor{currentfill}%
\pgfsetfillopacity{0.400000}%
\pgfsetlinewidth{0.501875pt}%
\definecolor{currentstroke}{rgb}{0.000000,0.000000,0.545098}%
\pgfsetstrokecolor{currentstroke}%
\pgfsetstrokeopacity{0.400000}%
\pgfsetdash{}{0pt}%
\pgfpathmoveto{\pgfqpoint{1.886279in}{2.182472in}}%
\pgfpathcurveto{\pgfqpoint{1.897329in}{2.182472in}}{\pgfqpoint{1.907928in}{2.186863in}}{\pgfqpoint{1.915742in}{2.194676in}}%
\pgfpathcurveto{\pgfqpoint{1.923556in}{2.202490in}}{\pgfqpoint{1.927946in}{2.213089in}}{\pgfqpoint{1.927946in}{2.224139in}}%
\pgfpathcurveto{\pgfqpoint{1.927946in}{2.235189in}}{\pgfqpoint{1.923556in}{2.245788in}}{\pgfqpoint{1.915742in}{2.253602in}}%
\pgfpathcurveto{\pgfqpoint{1.907928in}{2.261416in}}{\pgfqpoint{1.897329in}{2.265806in}}{\pgfqpoint{1.886279in}{2.265806in}}%
\pgfpathcurveto{\pgfqpoint{1.875229in}{2.265806in}}{\pgfqpoint{1.864630in}{2.261416in}}{\pgfqpoint{1.856817in}{2.253602in}}%
\pgfpathcurveto{\pgfqpoint{1.849003in}{2.245788in}}{\pgfqpoint{1.844613in}{2.235189in}}{\pgfqpoint{1.844613in}{2.224139in}}%
\pgfpathcurveto{\pgfqpoint{1.844613in}{2.213089in}}{\pgfqpoint{1.849003in}{2.202490in}}{\pgfqpoint{1.856817in}{2.194676in}}%
\pgfpathcurveto{\pgfqpoint{1.864630in}{2.186863in}}{\pgfqpoint{1.875229in}{2.182472in}}{\pgfqpoint{1.886279in}{2.182472in}}%
\pgfpathclose%
\pgfusepath{stroke,fill}%
\end{pgfscope}%
\begin{pgfscope}%
\pgfpathrectangle{\pgfqpoint{0.511823in}{0.504323in}}{\pgfqpoint{3.218177in}{3.225677in}} %
\pgfusepath{clip}%
\pgfsetbuttcap%
\pgfsetroundjoin%
\definecolor{currentfill}{rgb}{0.000000,0.000000,0.545098}%
\pgfsetfillcolor{currentfill}%
\pgfsetfillopacity{0.400000}%
\pgfsetlinewidth{0.501875pt}%
\definecolor{currentstroke}{rgb}{0.000000,0.000000,0.545098}%
\pgfsetstrokecolor{currentstroke}%
\pgfsetstrokeopacity{0.400000}%
\pgfsetdash{}{0pt}%
\pgfpathmoveto{\pgfqpoint{2.048424in}{2.402610in}}%
\pgfpathcurveto{\pgfqpoint{2.059474in}{2.402610in}}{\pgfqpoint{2.070073in}{2.407000in}}{\pgfqpoint{2.077887in}{2.414814in}}%
\pgfpathcurveto{\pgfqpoint{2.085701in}{2.422628in}}{\pgfqpoint{2.090091in}{2.433227in}}{\pgfqpoint{2.090091in}{2.444277in}}%
\pgfpathcurveto{\pgfqpoint{2.090091in}{2.455327in}}{\pgfqpoint{2.085701in}{2.465926in}}{\pgfqpoint{2.077887in}{2.473740in}}%
\pgfpathcurveto{\pgfqpoint{2.070073in}{2.481553in}}{\pgfqpoint{2.059474in}{2.485944in}}{\pgfqpoint{2.048424in}{2.485944in}}%
\pgfpathcurveto{\pgfqpoint{2.037374in}{2.485944in}}{\pgfqpoint{2.026775in}{2.481553in}}{\pgfqpoint{2.018962in}{2.473740in}}%
\pgfpathcurveto{\pgfqpoint{2.011148in}{2.465926in}}{\pgfqpoint{2.006758in}{2.455327in}}{\pgfqpoint{2.006758in}{2.444277in}}%
\pgfpathcurveto{\pgfqpoint{2.006758in}{2.433227in}}{\pgfqpoint{2.011148in}{2.422628in}}{\pgfqpoint{2.018962in}{2.414814in}}%
\pgfpathcurveto{\pgfqpoint{2.026775in}{2.407000in}}{\pgfqpoint{2.037374in}{2.402610in}}{\pgfqpoint{2.048424in}{2.402610in}}%
\pgfpathclose%
\pgfusepath{stroke,fill}%
\end{pgfscope}%
\begin{pgfscope}%
\pgfpathrectangle{\pgfqpoint{0.511823in}{0.504323in}}{\pgfqpoint{3.218177in}{3.225677in}} %
\pgfusepath{clip}%
\pgfsetbuttcap%
\pgfsetroundjoin%
\definecolor{currentfill}{rgb}{0.000000,0.000000,0.545098}%
\pgfsetfillcolor{currentfill}%
\pgfsetfillopacity{0.400000}%
\pgfsetlinewidth{0.501875pt}%
\definecolor{currentstroke}{rgb}{0.000000,0.000000,0.545098}%
\pgfsetstrokecolor{currentstroke}%
\pgfsetstrokeopacity{0.400000}%
\pgfsetdash{}{0pt}%
\pgfpathmoveto{\pgfqpoint{1.893220in}{2.212146in}}%
\pgfpathcurveto{\pgfqpoint{1.904270in}{2.212146in}}{\pgfqpoint{1.914869in}{2.216536in}}{\pgfqpoint{1.922683in}{2.224350in}}%
\pgfpathcurveto{\pgfqpoint{1.930496in}{2.232163in}}{\pgfqpoint{1.934887in}{2.242762in}}{\pgfqpoint{1.934887in}{2.253812in}}%
\pgfpathcurveto{\pgfqpoint{1.934887in}{2.264862in}}{\pgfqpoint{1.930496in}{2.275461in}}{\pgfqpoint{1.922683in}{2.283275in}}%
\pgfpathcurveto{\pgfqpoint{1.914869in}{2.291089in}}{\pgfqpoint{1.904270in}{2.295479in}}{\pgfqpoint{1.893220in}{2.295479in}}%
\pgfpathcurveto{\pgfqpoint{1.882170in}{2.295479in}}{\pgfqpoint{1.871571in}{2.291089in}}{\pgfqpoint{1.863757in}{2.283275in}}%
\pgfpathcurveto{\pgfqpoint{1.855944in}{2.275461in}}{\pgfqpoint{1.851553in}{2.264862in}}{\pgfqpoint{1.851553in}{2.253812in}}%
\pgfpathcurveto{\pgfqpoint{1.851553in}{2.242762in}}{\pgfqpoint{1.855944in}{2.232163in}}{\pgfqpoint{1.863757in}{2.224350in}}%
\pgfpathcurveto{\pgfqpoint{1.871571in}{2.216536in}}{\pgfqpoint{1.882170in}{2.212146in}}{\pgfqpoint{1.893220in}{2.212146in}}%
\pgfpathclose%
\pgfusepath{stroke,fill}%
\end{pgfscope}%
\begin{pgfscope}%
\pgfpathrectangle{\pgfqpoint{0.511823in}{0.504323in}}{\pgfqpoint{3.218177in}{3.225677in}} %
\pgfusepath{clip}%
\pgfsetbuttcap%
\pgfsetroundjoin%
\definecolor{currentfill}{rgb}{0.000000,0.000000,0.545098}%
\pgfsetfillcolor{currentfill}%
\pgfsetfillopacity{0.400000}%
\pgfsetlinewidth{0.501875pt}%
\definecolor{currentstroke}{rgb}{0.000000,0.000000,0.545098}%
\pgfsetstrokecolor{currentstroke}%
\pgfsetstrokeopacity{0.400000}%
\pgfsetdash{}{0pt}%
\pgfpathmoveto{\pgfqpoint{1.876607in}{2.200867in}}%
\pgfpathcurveto{\pgfqpoint{1.887657in}{2.200867in}}{\pgfqpoint{1.898256in}{2.205258in}}{\pgfqpoint{1.906070in}{2.213071in}}%
\pgfpathcurveto{\pgfqpoint{1.913884in}{2.220885in}}{\pgfqpoint{1.918274in}{2.231484in}}{\pgfqpoint{1.918274in}{2.242534in}}%
\pgfpathcurveto{\pgfqpoint{1.918274in}{2.253584in}}{\pgfqpoint{1.913884in}{2.264183in}}{\pgfqpoint{1.906070in}{2.271997in}}%
\pgfpathcurveto{\pgfqpoint{1.898256in}{2.279810in}}{\pgfqpoint{1.887657in}{2.284201in}}{\pgfqpoint{1.876607in}{2.284201in}}%
\pgfpathcurveto{\pgfqpoint{1.865557in}{2.284201in}}{\pgfqpoint{1.854958in}{2.279810in}}{\pgfqpoint{1.847144in}{2.271997in}}%
\pgfpathcurveto{\pgfqpoint{1.839331in}{2.264183in}}{\pgfqpoint{1.834940in}{2.253584in}}{\pgfqpoint{1.834940in}{2.242534in}}%
\pgfpathcurveto{\pgfqpoint{1.834940in}{2.231484in}}{\pgfqpoint{1.839331in}{2.220885in}}{\pgfqpoint{1.847144in}{2.213071in}}%
\pgfpathcurveto{\pgfqpoint{1.854958in}{2.205258in}}{\pgfqpoint{1.865557in}{2.200867in}}{\pgfqpoint{1.876607in}{2.200867in}}%
\pgfpathclose%
\pgfusepath{stroke,fill}%
\end{pgfscope}%
\begin{pgfscope}%
\pgfpathrectangle{\pgfqpoint{0.511823in}{0.504323in}}{\pgfqpoint{3.218177in}{3.225677in}} %
\pgfusepath{clip}%
\pgfsetbuttcap%
\pgfsetroundjoin%
\definecolor{currentfill}{rgb}{0.000000,0.000000,0.545098}%
\pgfsetfillcolor{currentfill}%
\pgfsetfillopacity{0.400000}%
\pgfsetlinewidth{0.501875pt}%
\definecolor{currentstroke}{rgb}{0.000000,0.000000,0.545098}%
\pgfsetstrokecolor{currentstroke}%
\pgfsetstrokeopacity{0.400000}%
\pgfsetdash{}{0pt}%
\pgfpathmoveto{\pgfqpoint{1.796296in}{2.105280in}}%
\pgfpathcurveto{\pgfqpoint{1.807346in}{2.105280in}}{\pgfqpoint{1.817945in}{2.109670in}}{\pgfqpoint{1.825759in}{2.117484in}}%
\pgfpathcurveto{\pgfqpoint{1.833573in}{2.125298in}}{\pgfqpoint{1.837963in}{2.135897in}}{\pgfqpoint{1.837963in}{2.146947in}}%
\pgfpathcurveto{\pgfqpoint{1.837963in}{2.157997in}}{\pgfqpoint{1.833573in}{2.168596in}}{\pgfqpoint{1.825759in}{2.176410in}}%
\pgfpathcurveto{\pgfqpoint{1.817945in}{2.184223in}}{\pgfqpoint{1.807346in}{2.188613in}}{\pgfqpoint{1.796296in}{2.188613in}}%
\pgfpathcurveto{\pgfqpoint{1.785246in}{2.188613in}}{\pgfqpoint{1.774647in}{2.184223in}}{\pgfqpoint{1.766833in}{2.176410in}}%
\pgfpathcurveto{\pgfqpoint{1.759020in}{2.168596in}}{\pgfqpoint{1.754629in}{2.157997in}}{\pgfqpoint{1.754629in}{2.146947in}}%
\pgfpathcurveto{\pgfqpoint{1.754629in}{2.135897in}}{\pgfqpoint{1.759020in}{2.125298in}}{\pgfqpoint{1.766833in}{2.117484in}}%
\pgfpathcurveto{\pgfqpoint{1.774647in}{2.109670in}}{\pgfqpoint{1.785246in}{2.105280in}}{\pgfqpoint{1.796296in}{2.105280in}}%
\pgfpathclose%
\pgfusepath{stroke,fill}%
\end{pgfscope}%
\begin{pgfscope}%
\pgfpathrectangle{\pgfqpoint{0.511823in}{0.504323in}}{\pgfqpoint{3.218177in}{3.225677in}} %
\pgfusepath{clip}%
\pgfsetbuttcap%
\pgfsetroundjoin%
\definecolor{currentfill}{rgb}{0.000000,0.000000,0.545098}%
\pgfsetfillcolor{currentfill}%
\pgfsetfillopacity{0.400000}%
\pgfsetlinewidth{0.501875pt}%
\definecolor{currentstroke}{rgb}{0.000000,0.000000,0.545098}%
\pgfsetstrokecolor{currentstroke}%
\pgfsetstrokeopacity{0.400000}%
\pgfsetdash{}{0pt}%
\pgfpathmoveto{\pgfqpoint{1.904443in}{2.258884in}}%
\pgfpathcurveto{\pgfqpoint{1.915493in}{2.258884in}}{\pgfqpoint{1.926092in}{2.263274in}}{\pgfqpoint{1.933906in}{2.271088in}}%
\pgfpathcurveto{\pgfqpoint{1.941719in}{2.278902in}}{\pgfqpoint{1.946110in}{2.289501in}}{\pgfqpoint{1.946110in}{2.300551in}}%
\pgfpathcurveto{\pgfqpoint{1.946110in}{2.311601in}}{\pgfqpoint{1.941719in}{2.322200in}}{\pgfqpoint{1.933906in}{2.330014in}}%
\pgfpathcurveto{\pgfqpoint{1.926092in}{2.337827in}}{\pgfqpoint{1.915493in}{2.342217in}}{\pgfqpoint{1.904443in}{2.342217in}}%
\pgfpathcurveto{\pgfqpoint{1.893393in}{2.342217in}}{\pgfqpoint{1.882794in}{2.337827in}}{\pgfqpoint{1.874980in}{2.330014in}}%
\pgfpathcurveto{\pgfqpoint{1.867167in}{2.322200in}}{\pgfqpoint{1.862776in}{2.311601in}}{\pgfqpoint{1.862776in}{2.300551in}}%
\pgfpathcurveto{\pgfqpoint{1.862776in}{2.289501in}}{\pgfqpoint{1.867167in}{2.278902in}}{\pgfqpoint{1.874980in}{2.271088in}}%
\pgfpathcurveto{\pgfqpoint{1.882794in}{2.263274in}}{\pgfqpoint{1.893393in}{2.258884in}}{\pgfqpoint{1.904443in}{2.258884in}}%
\pgfpathclose%
\pgfusepath{stroke,fill}%
\end{pgfscope}%
\begin{pgfscope}%
\pgfpathrectangle{\pgfqpoint{0.511823in}{0.504323in}}{\pgfqpoint{3.218177in}{3.225677in}} %
\pgfusepath{clip}%
\pgfsetbuttcap%
\pgfsetroundjoin%
\definecolor{currentfill}{rgb}{0.000000,0.000000,0.545098}%
\pgfsetfillcolor{currentfill}%
\pgfsetfillopacity{0.400000}%
\pgfsetlinewidth{0.501875pt}%
\definecolor{currentstroke}{rgb}{0.000000,0.000000,0.545098}%
\pgfsetstrokecolor{currentstroke}%
\pgfsetstrokeopacity{0.400000}%
\pgfsetdash{}{0pt}%
\pgfpathmoveto{\pgfqpoint{1.925336in}{2.297742in}}%
\pgfpathcurveto{\pgfqpoint{1.936386in}{2.297742in}}{\pgfqpoint{1.946985in}{2.302133in}}{\pgfqpoint{1.954798in}{2.309946in}}%
\pgfpathcurveto{\pgfqpoint{1.962612in}{2.317760in}}{\pgfqpoint{1.967002in}{2.328359in}}{\pgfqpoint{1.967002in}{2.339409in}}%
\pgfpathcurveto{\pgfqpoint{1.967002in}{2.350459in}}{\pgfqpoint{1.962612in}{2.361058in}}{\pgfqpoint{1.954798in}{2.368872in}}%
\pgfpathcurveto{\pgfqpoint{1.946985in}{2.376686in}}{\pgfqpoint{1.936386in}{2.381076in}}{\pgfqpoint{1.925336in}{2.381076in}}%
\pgfpathcurveto{\pgfqpoint{1.914286in}{2.381076in}}{\pgfqpoint{1.903686in}{2.376686in}}{\pgfqpoint{1.895873in}{2.368872in}}%
\pgfpathcurveto{\pgfqpoint{1.888059in}{2.361058in}}{\pgfqpoint{1.883669in}{2.350459in}}{\pgfqpoint{1.883669in}{2.339409in}}%
\pgfpathcurveto{\pgfqpoint{1.883669in}{2.328359in}}{\pgfqpoint{1.888059in}{2.317760in}}{\pgfqpoint{1.895873in}{2.309946in}}%
\pgfpathcurveto{\pgfqpoint{1.903686in}{2.302133in}}{\pgfqpoint{1.914286in}{2.297742in}}{\pgfqpoint{1.925336in}{2.297742in}}%
\pgfpathclose%
\pgfusepath{stroke,fill}%
\end{pgfscope}%
\begin{pgfscope}%
\pgfpathrectangle{\pgfqpoint{0.511823in}{0.504323in}}{\pgfqpoint{3.218177in}{3.225677in}} %
\pgfusepath{clip}%
\pgfsetbuttcap%
\pgfsetroundjoin%
\definecolor{currentfill}{rgb}{0.000000,0.000000,0.545098}%
\pgfsetfillcolor{currentfill}%
\pgfsetfillopacity{0.400000}%
\pgfsetlinewidth{0.501875pt}%
\definecolor{currentstroke}{rgb}{0.000000,0.000000,0.545098}%
\pgfsetstrokecolor{currentstroke}%
\pgfsetstrokeopacity{0.400000}%
\pgfsetdash{}{0pt}%
\pgfpathmoveto{\pgfqpoint{1.688937in}{1.990428in}}%
\pgfpathcurveto{\pgfqpoint{1.699987in}{1.990428in}}{\pgfqpoint{1.710586in}{1.994818in}}{\pgfqpoint{1.718400in}{2.002632in}}%
\pgfpathcurveto{\pgfqpoint{1.726213in}{2.010446in}}{\pgfqpoint{1.730604in}{2.021045in}}{\pgfqpoint{1.730604in}{2.032095in}}%
\pgfpathcurveto{\pgfqpoint{1.730604in}{2.043145in}}{\pgfqpoint{1.726213in}{2.053744in}}{\pgfqpoint{1.718400in}{2.061558in}}%
\pgfpathcurveto{\pgfqpoint{1.710586in}{2.069371in}}{\pgfqpoint{1.699987in}{2.073761in}}{\pgfqpoint{1.688937in}{2.073761in}}%
\pgfpathcurveto{\pgfqpoint{1.677887in}{2.073761in}}{\pgfqpoint{1.667288in}{2.069371in}}{\pgfqpoint{1.659474in}{2.061558in}}%
\pgfpathcurveto{\pgfqpoint{1.651660in}{2.053744in}}{\pgfqpoint{1.647270in}{2.043145in}}{\pgfqpoint{1.647270in}{2.032095in}}%
\pgfpathcurveto{\pgfqpoint{1.647270in}{2.021045in}}{\pgfqpoint{1.651660in}{2.010446in}}{\pgfqpoint{1.659474in}{2.002632in}}%
\pgfpathcurveto{\pgfqpoint{1.667288in}{1.994818in}}{\pgfqpoint{1.677887in}{1.990428in}}{\pgfqpoint{1.688937in}{1.990428in}}%
\pgfpathclose%
\pgfusepath{stroke,fill}%
\end{pgfscope}%
\begin{pgfscope}%
\pgfpathrectangle{\pgfqpoint{0.511823in}{0.504323in}}{\pgfqpoint{3.218177in}{3.225677in}} %
\pgfusepath{clip}%
\pgfsetbuttcap%
\pgfsetroundjoin%
\definecolor{currentfill}{rgb}{0.000000,0.000000,0.545098}%
\pgfsetfillcolor{currentfill}%
\pgfsetfillopacity{0.400000}%
\pgfsetlinewidth{0.501875pt}%
\definecolor{currentstroke}{rgb}{0.000000,0.000000,0.545098}%
\pgfsetstrokecolor{currentstroke}%
\pgfsetstrokeopacity{0.400000}%
\pgfsetdash{}{0pt}%
\pgfpathmoveto{\pgfqpoint{1.708508in}{2.026109in}}%
\pgfpathcurveto{\pgfqpoint{1.719558in}{2.026109in}}{\pgfqpoint{1.730157in}{2.030500in}}{\pgfqpoint{1.737971in}{2.038313in}}%
\pgfpathcurveto{\pgfqpoint{1.745785in}{2.046127in}}{\pgfqpoint{1.750175in}{2.056726in}}{\pgfqpoint{1.750175in}{2.067776in}}%
\pgfpathcurveto{\pgfqpoint{1.750175in}{2.078826in}}{\pgfqpoint{1.745785in}{2.089425in}}{\pgfqpoint{1.737971in}{2.097239in}}%
\pgfpathcurveto{\pgfqpoint{1.730157in}{2.105052in}}{\pgfqpoint{1.719558in}{2.109443in}}{\pgfqpoint{1.708508in}{2.109443in}}%
\pgfpathcurveto{\pgfqpoint{1.697458in}{2.109443in}}{\pgfqpoint{1.686859in}{2.105052in}}{\pgfqpoint{1.679045in}{2.097239in}}%
\pgfpathcurveto{\pgfqpoint{1.671232in}{2.089425in}}{\pgfqpoint{1.666842in}{2.078826in}}{\pgfqpoint{1.666842in}{2.067776in}}%
\pgfpathcurveto{\pgfqpoint{1.666842in}{2.056726in}}{\pgfqpoint{1.671232in}{2.046127in}}{\pgfqpoint{1.679045in}{2.038313in}}%
\pgfpathcurveto{\pgfqpoint{1.686859in}{2.030500in}}{\pgfqpoint{1.697458in}{2.026109in}}{\pgfqpoint{1.708508in}{2.026109in}}%
\pgfpathclose%
\pgfusepath{stroke,fill}%
\end{pgfscope}%
\begin{pgfscope}%
\pgfpathrectangle{\pgfqpoint{0.511823in}{0.504323in}}{\pgfqpoint{3.218177in}{3.225677in}} %
\pgfusepath{clip}%
\pgfsetbuttcap%
\pgfsetroundjoin%
\definecolor{currentfill}{rgb}{0.000000,0.000000,0.545098}%
\pgfsetfillcolor{currentfill}%
\pgfsetfillopacity{0.400000}%
\pgfsetlinewidth{0.501875pt}%
\definecolor{currentstroke}{rgb}{0.000000,0.000000,0.545098}%
\pgfsetstrokecolor{currentstroke}%
\pgfsetstrokeopacity{0.400000}%
\pgfsetdash{}{0pt}%
\pgfpathmoveto{\pgfqpoint{1.963923in}{2.384304in}}%
\pgfpathcurveto{\pgfqpoint{1.974973in}{2.384304in}}{\pgfqpoint{1.985572in}{2.388694in}}{\pgfqpoint{1.993385in}{2.396508in}}%
\pgfpathcurveto{\pgfqpoint{2.001199in}{2.404322in}}{\pgfqpoint{2.005589in}{2.414921in}}{\pgfqpoint{2.005589in}{2.425971in}}%
\pgfpathcurveto{\pgfqpoint{2.005589in}{2.437021in}}{\pgfqpoint{2.001199in}{2.447620in}}{\pgfqpoint{1.993385in}{2.455434in}}%
\pgfpathcurveto{\pgfqpoint{1.985572in}{2.463247in}}{\pgfqpoint{1.974973in}{2.467638in}}{\pgfqpoint{1.963923in}{2.467638in}}%
\pgfpathcurveto{\pgfqpoint{1.952872in}{2.467638in}}{\pgfqpoint{1.942273in}{2.463247in}}{\pgfqpoint{1.934460in}{2.455434in}}%
\pgfpathcurveto{\pgfqpoint{1.926646in}{2.447620in}}{\pgfqpoint{1.922256in}{2.437021in}}{\pgfqpoint{1.922256in}{2.425971in}}%
\pgfpathcurveto{\pgfqpoint{1.922256in}{2.414921in}}{\pgfqpoint{1.926646in}{2.404322in}}{\pgfqpoint{1.934460in}{2.396508in}}%
\pgfpathcurveto{\pgfqpoint{1.942273in}{2.388694in}}{\pgfqpoint{1.952872in}{2.384304in}}{\pgfqpoint{1.963923in}{2.384304in}}%
\pgfpathclose%
\pgfusepath{stroke,fill}%
\end{pgfscope}%
\begin{pgfscope}%
\pgfpathrectangle{\pgfqpoint{0.511823in}{0.504323in}}{\pgfqpoint{3.218177in}{3.225677in}} %
\pgfusepath{clip}%
\pgfsetbuttcap%
\pgfsetroundjoin%
\definecolor{currentfill}{rgb}{0.000000,0.000000,0.545098}%
\pgfsetfillcolor{currentfill}%
\pgfsetfillopacity{0.400000}%
\pgfsetlinewidth{0.501875pt}%
\definecolor{currentstroke}{rgb}{0.000000,0.000000,0.545098}%
\pgfsetstrokecolor{currentstroke}%
\pgfsetstrokeopacity{0.400000}%
\pgfsetdash{}{0pt}%
\pgfpathmoveto{\pgfqpoint{1.833407in}{2.216716in}}%
\pgfpathcurveto{\pgfqpoint{1.844457in}{2.216716in}}{\pgfqpoint{1.855056in}{2.221106in}}{\pgfqpoint{1.862870in}{2.228920in}}%
\pgfpathcurveto{\pgfqpoint{1.870683in}{2.236734in}}{\pgfqpoint{1.875074in}{2.247333in}}{\pgfqpoint{1.875074in}{2.258383in}}%
\pgfpathcurveto{\pgfqpoint{1.875074in}{2.269433in}}{\pgfqpoint{1.870683in}{2.280032in}}{\pgfqpoint{1.862870in}{2.287846in}}%
\pgfpathcurveto{\pgfqpoint{1.855056in}{2.295659in}}{\pgfqpoint{1.844457in}{2.300049in}}{\pgfqpoint{1.833407in}{2.300049in}}%
\pgfpathcurveto{\pgfqpoint{1.822357in}{2.300049in}}{\pgfqpoint{1.811758in}{2.295659in}}{\pgfqpoint{1.803944in}{2.287846in}}%
\pgfpathcurveto{\pgfqpoint{1.796131in}{2.280032in}}{\pgfqpoint{1.791740in}{2.269433in}}{\pgfqpoint{1.791740in}{2.258383in}}%
\pgfpathcurveto{\pgfqpoint{1.791740in}{2.247333in}}{\pgfqpoint{1.796131in}{2.236734in}}{\pgfqpoint{1.803944in}{2.228920in}}%
\pgfpathcurveto{\pgfqpoint{1.811758in}{2.221106in}}{\pgfqpoint{1.822357in}{2.216716in}}{\pgfqpoint{1.833407in}{2.216716in}}%
\pgfpathclose%
\pgfusepath{stroke,fill}%
\end{pgfscope}%
\begin{pgfscope}%
\pgfpathrectangle{\pgfqpoint{0.511823in}{0.504323in}}{\pgfqpoint{3.218177in}{3.225677in}} %
\pgfusepath{clip}%
\pgfsetbuttcap%
\pgfsetroundjoin%
\definecolor{currentfill}{rgb}{0.000000,0.000000,0.545098}%
\pgfsetfillcolor{currentfill}%
\pgfsetfillopacity{0.400000}%
\pgfsetlinewidth{0.501875pt}%
\definecolor{currentstroke}{rgb}{0.000000,0.000000,0.545098}%
\pgfsetstrokecolor{currentstroke}%
\pgfsetstrokeopacity{0.400000}%
\pgfsetdash{}{0pt}%
\pgfpathmoveto{\pgfqpoint{1.806108in}{2.189672in}}%
\pgfpathcurveto{\pgfqpoint{1.817159in}{2.189672in}}{\pgfqpoint{1.827758in}{2.194063in}}{\pgfqpoint{1.835571in}{2.201876in}}%
\pgfpathcurveto{\pgfqpoint{1.843385in}{2.209690in}}{\pgfqpoint{1.847775in}{2.220289in}}{\pgfqpoint{1.847775in}{2.231339in}}%
\pgfpathcurveto{\pgfqpoint{1.847775in}{2.242389in}}{\pgfqpoint{1.843385in}{2.252988in}}{\pgfqpoint{1.835571in}{2.260802in}}%
\pgfpathcurveto{\pgfqpoint{1.827758in}{2.268615in}}{\pgfqpoint{1.817159in}{2.273006in}}{\pgfqpoint{1.806108in}{2.273006in}}%
\pgfpathcurveto{\pgfqpoint{1.795058in}{2.273006in}}{\pgfqpoint{1.784459in}{2.268615in}}{\pgfqpoint{1.776646in}{2.260802in}}%
\pgfpathcurveto{\pgfqpoint{1.768832in}{2.252988in}}{\pgfqpoint{1.764442in}{2.242389in}}{\pgfqpoint{1.764442in}{2.231339in}}%
\pgfpathcurveto{\pgfqpoint{1.764442in}{2.220289in}}{\pgfqpoint{1.768832in}{2.209690in}}{\pgfqpoint{1.776646in}{2.201876in}}%
\pgfpathcurveto{\pgfqpoint{1.784459in}{2.194063in}}{\pgfqpoint{1.795058in}{2.189672in}}{\pgfqpoint{1.806108in}{2.189672in}}%
\pgfpathclose%
\pgfusepath{stroke,fill}%
\end{pgfscope}%
\begin{pgfscope}%
\pgfpathrectangle{\pgfqpoint{0.511823in}{0.504323in}}{\pgfqpoint{3.218177in}{3.225677in}} %
\pgfusepath{clip}%
\pgfsetbuttcap%
\pgfsetroundjoin%
\definecolor{currentfill}{rgb}{0.000000,0.000000,0.545098}%
\pgfsetfillcolor{currentfill}%
\pgfsetfillopacity{0.400000}%
\pgfsetlinewidth{0.501875pt}%
\definecolor{currentstroke}{rgb}{0.000000,0.000000,0.545098}%
\pgfsetstrokecolor{currentstroke}%
\pgfsetstrokeopacity{0.400000}%
\pgfsetdash{}{0pt}%
\pgfpathmoveto{\pgfqpoint{1.832810in}{2.237475in}}%
\pgfpathcurveto{\pgfqpoint{1.843860in}{2.237475in}}{\pgfqpoint{1.854459in}{2.241865in}}{\pgfqpoint{1.862273in}{2.249679in}}%
\pgfpathcurveto{\pgfqpoint{1.870087in}{2.257492in}}{\pgfqpoint{1.874477in}{2.268091in}}{\pgfqpoint{1.874477in}{2.279141in}}%
\pgfpathcurveto{\pgfqpoint{1.874477in}{2.290192in}}{\pgfqpoint{1.870087in}{2.300791in}}{\pgfqpoint{1.862273in}{2.308604in}}%
\pgfpathcurveto{\pgfqpoint{1.854459in}{2.316418in}}{\pgfqpoint{1.843860in}{2.320808in}}{\pgfqpoint{1.832810in}{2.320808in}}%
\pgfpathcurveto{\pgfqpoint{1.821760in}{2.320808in}}{\pgfqpoint{1.811161in}{2.316418in}}{\pgfqpoint{1.803347in}{2.308604in}}%
\pgfpathcurveto{\pgfqpoint{1.795534in}{2.300791in}}{\pgfqpoint{1.791144in}{2.290192in}}{\pgfqpoint{1.791144in}{2.279141in}}%
\pgfpathcurveto{\pgfqpoint{1.791144in}{2.268091in}}{\pgfqpoint{1.795534in}{2.257492in}}{\pgfqpoint{1.803347in}{2.249679in}}%
\pgfpathcurveto{\pgfqpoint{1.811161in}{2.241865in}}{\pgfqpoint{1.821760in}{2.237475in}}{\pgfqpoint{1.832810in}{2.237475in}}%
\pgfpathclose%
\pgfusepath{stroke,fill}%
\end{pgfscope}%
\begin{pgfscope}%
\pgfpathrectangle{\pgfqpoint{0.511823in}{0.504323in}}{\pgfqpoint{3.218177in}{3.225677in}} %
\pgfusepath{clip}%
\pgfsetbuttcap%
\pgfsetroundjoin%
\definecolor{currentfill}{rgb}{0.000000,0.000000,0.545098}%
\pgfsetfillcolor{currentfill}%
\pgfsetfillopacity{0.400000}%
\pgfsetlinewidth{0.501875pt}%
\definecolor{currentstroke}{rgb}{0.000000,0.000000,0.545098}%
\pgfsetstrokecolor{currentstroke}%
\pgfsetstrokeopacity{0.400000}%
\pgfsetdash{}{0pt}%
\pgfpathmoveto{\pgfqpoint{1.866534in}{2.295740in}}%
\pgfpathcurveto{\pgfqpoint{1.877584in}{2.295740in}}{\pgfqpoint{1.888183in}{2.300130in}}{\pgfqpoint{1.895996in}{2.307944in}}%
\pgfpathcurveto{\pgfqpoint{1.903810in}{2.315758in}}{\pgfqpoint{1.908200in}{2.326357in}}{\pgfqpoint{1.908200in}{2.337407in}}%
\pgfpathcurveto{\pgfqpoint{1.908200in}{2.348457in}}{\pgfqpoint{1.903810in}{2.359056in}}{\pgfqpoint{1.895996in}{2.366870in}}%
\pgfpathcurveto{\pgfqpoint{1.888183in}{2.374683in}}{\pgfqpoint{1.877584in}{2.379074in}}{\pgfqpoint{1.866534in}{2.379074in}}%
\pgfpathcurveto{\pgfqpoint{1.855484in}{2.379074in}}{\pgfqpoint{1.844884in}{2.374683in}}{\pgfqpoint{1.837071in}{2.366870in}}%
\pgfpathcurveto{\pgfqpoint{1.829257in}{2.359056in}}{\pgfqpoint{1.824867in}{2.348457in}}{\pgfqpoint{1.824867in}{2.337407in}}%
\pgfpathcurveto{\pgfqpoint{1.824867in}{2.326357in}}{\pgfqpoint{1.829257in}{2.315758in}}{\pgfqpoint{1.837071in}{2.307944in}}%
\pgfpathcurveto{\pgfqpoint{1.844884in}{2.300130in}}{\pgfqpoint{1.855484in}{2.295740in}}{\pgfqpoint{1.866534in}{2.295740in}}%
\pgfpathclose%
\pgfusepath{stroke,fill}%
\end{pgfscope}%
\begin{pgfscope}%
\pgfpathrectangle{\pgfqpoint{0.511823in}{0.504323in}}{\pgfqpoint{3.218177in}{3.225677in}} %
\pgfusepath{clip}%
\pgfsetbuttcap%
\pgfsetroundjoin%
\definecolor{currentfill}{rgb}{0.000000,0.000000,0.545098}%
\pgfsetfillcolor{currentfill}%
\pgfsetfillopacity{0.400000}%
\pgfsetlinewidth{0.501875pt}%
\definecolor{currentstroke}{rgb}{0.000000,0.000000,0.545098}%
\pgfsetstrokecolor{currentstroke}%
\pgfsetstrokeopacity{0.400000}%
\pgfsetdash{}{0pt}%
\pgfpathmoveto{\pgfqpoint{1.839332in}{2.268679in}}%
\pgfpathcurveto{\pgfqpoint{1.850382in}{2.268679in}}{\pgfqpoint{1.860981in}{2.273069in}}{\pgfqpoint{1.868795in}{2.280883in}}%
\pgfpathcurveto{\pgfqpoint{1.876608in}{2.288697in}}{\pgfqpoint{1.880999in}{2.299296in}}{\pgfqpoint{1.880999in}{2.310346in}}%
\pgfpathcurveto{\pgfqpoint{1.880999in}{2.321396in}}{\pgfqpoint{1.876608in}{2.331995in}}{\pgfqpoint{1.868795in}{2.339809in}}%
\pgfpathcurveto{\pgfqpoint{1.860981in}{2.347622in}}{\pgfqpoint{1.850382in}{2.352013in}}{\pgfqpoint{1.839332in}{2.352013in}}%
\pgfpathcurveto{\pgfqpoint{1.828282in}{2.352013in}}{\pgfqpoint{1.817683in}{2.347622in}}{\pgfqpoint{1.809869in}{2.339809in}}%
\pgfpathcurveto{\pgfqpoint{1.802056in}{2.331995in}}{\pgfqpoint{1.797665in}{2.321396in}}{\pgfqpoint{1.797665in}{2.310346in}}%
\pgfpathcurveto{\pgfqpoint{1.797665in}{2.299296in}}{\pgfqpoint{1.802056in}{2.288697in}}{\pgfqpoint{1.809869in}{2.280883in}}%
\pgfpathcurveto{\pgfqpoint{1.817683in}{2.273069in}}{\pgfqpoint{1.828282in}{2.268679in}}{\pgfqpoint{1.839332in}{2.268679in}}%
\pgfpathclose%
\pgfusepath{stroke,fill}%
\end{pgfscope}%
\begin{pgfscope}%
\pgfpathrectangle{\pgfqpoint{0.511823in}{0.504323in}}{\pgfqpoint{3.218177in}{3.225677in}} %
\pgfusepath{clip}%
\pgfsetbuttcap%
\pgfsetroundjoin%
\definecolor{currentfill}{rgb}{0.000000,0.000000,0.545098}%
\pgfsetfillcolor{currentfill}%
\pgfsetfillopacity{0.400000}%
\pgfsetlinewidth{0.501875pt}%
\definecolor{currentstroke}{rgb}{0.000000,0.000000,0.545098}%
\pgfsetstrokecolor{currentstroke}%
\pgfsetstrokeopacity{0.400000}%
\pgfsetdash{}{0pt}%
\pgfpathmoveto{\pgfqpoint{1.759465in}{2.166275in}}%
\pgfpathcurveto{\pgfqpoint{1.770515in}{2.166275in}}{\pgfqpoint{1.781114in}{2.170665in}}{\pgfqpoint{1.788927in}{2.178479in}}%
\pgfpathcurveto{\pgfqpoint{1.796741in}{2.186292in}}{\pgfqpoint{1.801131in}{2.196891in}}{\pgfqpoint{1.801131in}{2.207941in}}%
\pgfpathcurveto{\pgfqpoint{1.801131in}{2.218992in}}{\pgfqpoint{1.796741in}{2.229591in}}{\pgfqpoint{1.788927in}{2.237404in}}%
\pgfpathcurveto{\pgfqpoint{1.781114in}{2.245218in}}{\pgfqpoint{1.770515in}{2.249608in}}{\pgfqpoint{1.759465in}{2.249608in}}%
\pgfpathcurveto{\pgfqpoint{1.748415in}{2.249608in}}{\pgfqpoint{1.737816in}{2.245218in}}{\pgfqpoint{1.730002in}{2.237404in}}%
\pgfpathcurveto{\pgfqpoint{1.722188in}{2.229591in}}{\pgfqpoint{1.717798in}{2.218992in}}{\pgfqpoint{1.717798in}{2.207941in}}%
\pgfpathcurveto{\pgfqpoint{1.717798in}{2.196891in}}{\pgfqpoint{1.722188in}{2.186292in}}{\pgfqpoint{1.730002in}{2.178479in}}%
\pgfpathcurveto{\pgfqpoint{1.737816in}{2.170665in}}{\pgfqpoint{1.748415in}{2.166275in}}{\pgfqpoint{1.759465in}{2.166275in}}%
\pgfpathclose%
\pgfusepath{stroke,fill}%
\end{pgfscope}%
\begin{pgfscope}%
\pgfpathrectangle{\pgfqpoint{0.511823in}{0.504323in}}{\pgfqpoint{3.218177in}{3.225677in}} %
\pgfusepath{clip}%
\pgfsetbuttcap%
\pgfsetroundjoin%
\definecolor{currentfill}{rgb}{0.000000,0.000000,0.545098}%
\pgfsetfillcolor{currentfill}%
\pgfsetfillopacity{0.400000}%
\pgfsetlinewidth{0.501875pt}%
\definecolor{currentstroke}{rgb}{0.000000,0.000000,0.545098}%
\pgfsetstrokecolor{currentstroke}%
\pgfsetstrokeopacity{0.400000}%
\pgfsetdash{}{0pt}%
\pgfpathmoveto{\pgfqpoint{1.919548in}{2.406126in}}%
\pgfpathcurveto{\pgfqpoint{1.930598in}{2.406126in}}{\pgfqpoint{1.941197in}{2.410516in}}{\pgfqpoint{1.949010in}{2.418330in}}%
\pgfpathcurveto{\pgfqpoint{1.956824in}{2.426143in}}{\pgfqpoint{1.961214in}{2.436743in}}{\pgfqpoint{1.961214in}{2.447793in}}%
\pgfpathcurveto{\pgfqpoint{1.961214in}{2.458843in}}{\pgfqpoint{1.956824in}{2.469442in}}{\pgfqpoint{1.949010in}{2.477255in}}%
\pgfpathcurveto{\pgfqpoint{1.941197in}{2.485069in}}{\pgfqpoint{1.930598in}{2.489459in}}{\pgfqpoint{1.919548in}{2.489459in}}%
\pgfpathcurveto{\pgfqpoint{1.908497in}{2.489459in}}{\pgfqpoint{1.897898in}{2.485069in}}{\pgfqpoint{1.890085in}{2.477255in}}%
\pgfpathcurveto{\pgfqpoint{1.882271in}{2.469442in}}{\pgfqpoint{1.877881in}{2.458843in}}{\pgfqpoint{1.877881in}{2.447793in}}%
\pgfpathcurveto{\pgfqpoint{1.877881in}{2.436743in}}{\pgfqpoint{1.882271in}{2.426143in}}{\pgfqpoint{1.890085in}{2.418330in}}%
\pgfpathcurveto{\pgfqpoint{1.897898in}{2.410516in}}{\pgfqpoint{1.908497in}{2.406126in}}{\pgfqpoint{1.919548in}{2.406126in}}%
\pgfpathclose%
\pgfusepath{stroke,fill}%
\end{pgfscope}%
\begin{pgfscope}%
\pgfpathrectangle{\pgfqpoint{0.511823in}{0.504323in}}{\pgfqpoint{3.218177in}{3.225677in}} %
\pgfusepath{clip}%
\pgfsetbuttcap%
\pgfsetroundjoin%
\definecolor{currentfill}{rgb}{0.000000,0.000000,0.545098}%
\pgfsetfillcolor{currentfill}%
\pgfsetfillopacity{0.400000}%
\pgfsetlinewidth{0.501875pt}%
\definecolor{currentstroke}{rgb}{0.000000,0.000000,0.545098}%
\pgfsetstrokecolor{currentstroke}%
\pgfsetstrokeopacity{0.400000}%
\pgfsetdash{}{0pt}%
\pgfpathmoveto{\pgfqpoint{1.766065in}{2.196979in}}%
\pgfpathcurveto{\pgfqpoint{1.777115in}{2.196979in}}{\pgfqpoint{1.787715in}{2.201370in}}{\pgfqpoint{1.795528in}{2.209183in}}%
\pgfpathcurveto{\pgfqpoint{1.803342in}{2.216997in}}{\pgfqpoint{1.807732in}{2.227596in}}{\pgfqpoint{1.807732in}{2.238646in}}%
\pgfpathcurveto{\pgfqpoint{1.807732in}{2.249696in}}{\pgfqpoint{1.803342in}{2.260295in}}{\pgfqpoint{1.795528in}{2.268109in}}%
\pgfpathcurveto{\pgfqpoint{1.787715in}{2.275923in}}{\pgfqpoint{1.777115in}{2.280313in}}{\pgfqpoint{1.766065in}{2.280313in}}%
\pgfpathcurveto{\pgfqpoint{1.755015in}{2.280313in}}{\pgfqpoint{1.744416in}{2.275923in}}{\pgfqpoint{1.736603in}{2.268109in}}%
\pgfpathcurveto{\pgfqpoint{1.728789in}{2.260295in}}{\pgfqpoint{1.724399in}{2.249696in}}{\pgfqpoint{1.724399in}{2.238646in}}%
\pgfpathcurveto{\pgfqpoint{1.724399in}{2.227596in}}{\pgfqpoint{1.728789in}{2.216997in}}{\pgfqpoint{1.736603in}{2.209183in}}%
\pgfpathcurveto{\pgfqpoint{1.744416in}{2.201370in}}{\pgfqpoint{1.755015in}{2.196979in}}{\pgfqpoint{1.766065in}{2.196979in}}%
\pgfpathclose%
\pgfusepath{stroke,fill}%
\end{pgfscope}%
\begin{pgfscope}%
\pgfpathrectangle{\pgfqpoint{0.511823in}{0.504323in}}{\pgfqpoint{3.218177in}{3.225677in}} %
\pgfusepath{clip}%
\pgfsetbuttcap%
\pgfsetroundjoin%
\definecolor{currentfill}{rgb}{0.000000,0.000000,0.545098}%
\pgfsetfillcolor{currentfill}%
\pgfsetfillopacity{0.400000}%
\pgfsetlinewidth{0.501875pt}%
\definecolor{currentstroke}{rgb}{0.000000,0.000000,0.545098}%
\pgfsetstrokecolor{currentstroke}%
\pgfsetstrokeopacity{0.400000}%
\pgfsetdash{}{0pt}%
\pgfpathmoveto{\pgfqpoint{1.697287in}{2.107900in}}%
\pgfpathcurveto{\pgfqpoint{1.708337in}{2.107900in}}{\pgfqpoint{1.718936in}{2.112290in}}{\pgfqpoint{1.726749in}{2.120104in}}%
\pgfpathcurveto{\pgfqpoint{1.734563in}{2.127918in}}{\pgfqpoint{1.738953in}{2.138517in}}{\pgfqpoint{1.738953in}{2.149567in}}%
\pgfpathcurveto{\pgfqpoint{1.738953in}{2.160617in}}{\pgfqpoint{1.734563in}{2.171216in}}{\pgfqpoint{1.726749in}{2.179030in}}%
\pgfpathcurveto{\pgfqpoint{1.718936in}{2.186843in}}{\pgfqpoint{1.708337in}{2.191234in}}{\pgfqpoint{1.697287in}{2.191234in}}%
\pgfpathcurveto{\pgfqpoint{1.686237in}{2.191234in}}{\pgfqpoint{1.675637in}{2.186843in}}{\pgfqpoint{1.667824in}{2.179030in}}%
\pgfpathcurveto{\pgfqpoint{1.660010in}{2.171216in}}{\pgfqpoint{1.655620in}{2.160617in}}{\pgfqpoint{1.655620in}{2.149567in}}%
\pgfpathcurveto{\pgfqpoint{1.655620in}{2.138517in}}{\pgfqpoint{1.660010in}{2.127918in}}{\pgfqpoint{1.667824in}{2.120104in}}%
\pgfpathcurveto{\pgfqpoint{1.675637in}{2.112290in}}{\pgfqpoint{1.686237in}{2.107900in}}{\pgfqpoint{1.697287in}{2.107900in}}%
\pgfpathclose%
\pgfusepath{stroke,fill}%
\end{pgfscope}%
\begin{pgfscope}%
\pgfpathrectangle{\pgfqpoint{0.511823in}{0.504323in}}{\pgfqpoint{3.218177in}{3.225677in}} %
\pgfusepath{clip}%
\pgfsetbuttcap%
\pgfsetroundjoin%
\definecolor{currentfill}{rgb}{0.000000,0.000000,0.545098}%
\pgfsetfillcolor{currentfill}%
\pgfsetfillopacity{0.400000}%
\pgfsetlinewidth{0.501875pt}%
\definecolor{currentstroke}{rgb}{0.000000,0.000000,0.545098}%
\pgfsetstrokecolor{currentstroke}%
\pgfsetstrokeopacity{0.400000}%
\pgfsetdash{}{0pt}%
\pgfpathmoveto{\pgfqpoint{1.714938in}{2.143940in}}%
\pgfpathcurveto{\pgfqpoint{1.725988in}{2.143940in}}{\pgfqpoint{1.736587in}{2.148331in}}{\pgfqpoint{1.744401in}{2.156144in}}%
\pgfpathcurveto{\pgfqpoint{1.752215in}{2.163958in}}{\pgfqpoint{1.756605in}{2.174557in}}{\pgfqpoint{1.756605in}{2.185607in}}%
\pgfpathcurveto{\pgfqpoint{1.756605in}{2.196657in}}{\pgfqpoint{1.752215in}{2.207256in}}{\pgfqpoint{1.744401in}{2.215070in}}%
\pgfpathcurveto{\pgfqpoint{1.736587in}{2.222883in}}{\pgfqpoint{1.725988in}{2.227274in}}{\pgfqpoint{1.714938in}{2.227274in}}%
\pgfpathcurveto{\pgfqpoint{1.703888in}{2.227274in}}{\pgfqpoint{1.693289in}{2.222883in}}{\pgfqpoint{1.685475in}{2.215070in}}%
\pgfpathcurveto{\pgfqpoint{1.677662in}{2.207256in}}{\pgfqpoint{1.673271in}{2.196657in}}{\pgfqpoint{1.673271in}{2.185607in}}%
\pgfpathcurveto{\pgfqpoint{1.673271in}{2.174557in}}{\pgfqpoint{1.677662in}{2.163958in}}{\pgfqpoint{1.685475in}{2.156144in}}%
\pgfpathcurveto{\pgfqpoint{1.693289in}{2.148331in}}{\pgfqpoint{1.703888in}{2.143940in}}{\pgfqpoint{1.714938in}{2.143940in}}%
\pgfpathclose%
\pgfusepath{stroke,fill}%
\end{pgfscope}%
\begin{pgfscope}%
\pgfpathrectangle{\pgfqpoint{0.511823in}{0.504323in}}{\pgfqpoint{3.218177in}{3.225677in}} %
\pgfusepath{clip}%
\pgfsetbuttcap%
\pgfsetroundjoin%
\definecolor{currentfill}{rgb}{0.000000,0.000000,0.545098}%
\pgfsetfillcolor{currentfill}%
\pgfsetfillopacity{0.400000}%
\pgfsetlinewidth{0.501875pt}%
\definecolor{currentstroke}{rgb}{0.000000,0.000000,0.545098}%
\pgfsetstrokecolor{currentstroke}%
\pgfsetstrokeopacity{0.400000}%
\pgfsetdash{}{0pt}%
\pgfpathmoveto{\pgfqpoint{1.739943in}{2.191273in}}%
\pgfpathcurveto{\pgfqpoint{1.750993in}{2.191273in}}{\pgfqpoint{1.761592in}{2.195663in}}{\pgfqpoint{1.769406in}{2.203477in}}%
\pgfpathcurveto{\pgfqpoint{1.777220in}{2.211290in}}{\pgfqpoint{1.781610in}{2.221889in}}{\pgfqpoint{1.781610in}{2.232940in}}%
\pgfpathcurveto{\pgfqpoint{1.781610in}{2.243990in}}{\pgfqpoint{1.777220in}{2.254589in}}{\pgfqpoint{1.769406in}{2.262402in}}%
\pgfpathcurveto{\pgfqpoint{1.761592in}{2.270216in}}{\pgfqpoint{1.750993in}{2.274606in}}{\pgfqpoint{1.739943in}{2.274606in}}%
\pgfpathcurveto{\pgfqpoint{1.728893in}{2.274606in}}{\pgfqpoint{1.718294in}{2.270216in}}{\pgfqpoint{1.710480in}{2.262402in}}%
\pgfpathcurveto{\pgfqpoint{1.702667in}{2.254589in}}{\pgfqpoint{1.698277in}{2.243990in}}{\pgfqpoint{1.698277in}{2.232940in}}%
\pgfpathcurveto{\pgfqpoint{1.698277in}{2.221889in}}{\pgfqpoint{1.702667in}{2.211290in}}{\pgfqpoint{1.710480in}{2.203477in}}%
\pgfpathcurveto{\pgfqpoint{1.718294in}{2.195663in}}{\pgfqpoint{1.728893in}{2.191273in}}{\pgfqpoint{1.739943in}{2.191273in}}%
\pgfpathclose%
\pgfusepath{stroke,fill}%
\end{pgfscope}%
\begin{pgfscope}%
\pgfpathrectangle{\pgfqpoint{0.511823in}{0.504323in}}{\pgfqpoint{3.218177in}{3.225677in}} %
\pgfusepath{clip}%
\pgfsetbuttcap%
\pgfsetroundjoin%
\definecolor{currentfill}{rgb}{0.000000,0.000000,0.545098}%
\pgfsetfillcolor{currentfill}%
\pgfsetfillopacity{0.400000}%
\pgfsetlinewidth{0.501875pt}%
\definecolor{currentstroke}{rgb}{0.000000,0.000000,0.545098}%
\pgfsetstrokecolor{currentstroke}%
\pgfsetstrokeopacity{0.400000}%
\pgfsetdash{}{0pt}%
\pgfpathmoveto{\pgfqpoint{1.717280in}{2.168532in}}%
\pgfpathcurveto{\pgfqpoint{1.728330in}{2.168532in}}{\pgfqpoint{1.738929in}{2.172922in}}{\pgfqpoint{1.746742in}{2.180736in}}%
\pgfpathcurveto{\pgfqpoint{1.754556in}{2.188550in}}{\pgfqpoint{1.758946in}{2.199149in}}{\pgfqpoint{1.758946in}{2.210199in}}%
\pgfpathcurveto{\pgfqpoint{1.758946in}{2.221249in}}{\pgfqpoint{1.754556in}{2.231848in}}{\pgfqpoint{1.746742in}{2.239662in}}%
\pgfpathcurveto{\pgfqpoint{1.738929in}{2.247475in}}{\pgfqpoint{1.728330in}{2.251866in}}{\pgfqpoint{1.717280in}{2.251866in}}%
\pgfpathcurveto{\pgfqpoint{1.706230in}{2.251866in}}{\pgfqpoint{1.695631in}{2.247475in}}{\pgfqpoint{1.687817in}{2.239662in}}%
\pgfpathcurveto{\pgfqpoint{1.680003in}{2.231848in}}{\pgfqpoint{1.675613in}{2.221249in}}{\pgfqpoint{1.675613in}{2.210199in}}%
\pgfpathcurveto{\pgfqpoint{1.675613in}{2.199149in}}{\pgfqpoint{1.680003in}{2.188550in}}{\pgfqpoint{1.687817in}{2.180736in}}%
\pgfpathcurveto{\pgfqpoint{1.695631in}{2.172922in}}{\pgfqpoint{1.706230in}{2.168532in}}{\pgfqpoint{1.717280in}{2.168532in}}%
\pgfpathclose%
\pgfusepath{stroke,fill}%
\end{pgfscope}%
\begin{pgfscope}%
\pgfpathrectangle{\pgfqpoint{0.511823in}{0.504323in}}{\pgfqpoint{3.218177in}{3.225677in}} %
\pgfusepath{clip}%
\pgfsetbuttcap%
\pgfsetroundjoin%
\definecolor{currentfill}{rgb}{0.000000,0.000000,0.545098}%
\pgfsetfillcolor{currentfill}%
\pgfsetfillopacity{0.400000}%
\pgfsetlinewidth{0.501875pt}%
\definecolor{currentstroke}{rgb}{0.000000,0.000000,0.545098}%
\pgfsetstrokecolor{currentstroke}%
\pgfsetstrokeopacity{0.400000}%
\pgfsetdash{}{0pt}%
\pgfpathmoveto{\pgfqpoint{1.725532in}{2.191600in}}%
\pgfpathcurveto{\pgfqpoint{1.736582in}{2.191600in}}{\pgfqpoint{1.747181in}{2.195990in}}{\pgfqpoint{1.754994in}{2.203804in}}%
\pgfpathcurveto{\pgfqpoint{1.762808in}{2.211617in}}{\pgfqpoint{1.767198in}{2.222216in}}{\pgfqpoint{1.767198in}{2.233266in}}%
\pgfpathcurveto{\pgfqpoint{1.767198in}{2.244317in}}{\pgfqpoint{1.762808in}{2.254916in}}{\pgfqpoint{1.754994in}{2.262729in}}%
\pgfpathcurveto{\pgfqpoint{1.747181in}{2.270543in}}{\pgfqpoint{1.736582in}{2.274933in}}{\pgfqpoint{1.725532in}{2.274933in}}%
\pgfpathcurveto{\pgfqpoint{1.714481in}{2.274933in}}{\pgfqpoint{1.703882in}{2.270543in}}{\pgfqpoint{1.696069in}{2.262729in}}%
\pgfpathcurveto{\pgfqpoint{1.688255in}{2.254916in}}{\pgfqpoint{1.683865in}{2.244317in}}{\pgfqpoint{1.683865in}{2.233266in}}%
\pgfpathcurveto{\pgfqpoint{1.683865in}{2.222216in}}{\pgfqpoint{1.688255in}{2.211617in}}{\pgfqpoint{1.696069in}{2.203804in}}%
\pgfpathcurveto{\pgfqpoint{1.703882in}{2.195990in}}{\pgfqpoint{1.714481in}{2.191600in}}{\pgfqpoint{1.725532in}{2.191600in}}%
\pgfpathclose%
\pgfusepath{stroke,fill}%
\end{pgfscope}%
\begin{pgfscope}%
\pgfpathrectangle{\pgfqpoint{0.511823in}{0.504323in}}{\pgfqpoint{3.218177in}{3.225677in}} %
\pgfusepath{clip}%
\pgfsetbuttcap%
\pgfsetroundjoin%
\definecolor{currentfill}{rgb}{0.000000,0.000000,0.545098}%
\pgfsetfillcolor{currentfill}%
\pgfsetfillopacity{0.400000}%
\pgfsetlinewidth{0.501875pt}%
\definecolor{currentstroke}{rgb}{0.000000,0.000000,0.545098}%
\pgfsetstrokecolor{currentstroke}%
\pgfsetstrokeopacity{0.400000}%
\pgfsetdash{}{0pt}%
\pgfpathmoveto{\pgfqpoint{1.755402in}{2.247464in}}%
\pgfpathcurveto{\pgfqpoint{1.766452in}{2.247464in}}{\pgfqpoint{1.777051in}{2.251854in}}{\pgfqpoint{1.784865in}{2.259668in}}%
\pgfpathcurveto{\pgfqpoint{1.792678in}{2.267481in}}{\pgfqpoint{1.797069in}{2.278080in}}{\pgfqpoint{1.797069in}{2.289131in}}%
\pgfpathcurveto{\pgfqpoint{1.797069in}{2.300181in}}{\pgfqpoint{1.792678in}{2.310780in}}{\pgfqpoint{1.784865in}{2.318593in}}%
\pgfpathcurveto{\pgfqpoint{1.777051in}{2.326407in}}{\pgfqpoint{1.766452in}{2.330797in}}{\pgfqpoint{1.755402in}{2.330797in}}%
\pgfpathcurveto{\pgfqpoint{1.744352in}{2.330797in}}{\pgfqpoint{1.733753in}{2.326407in}}{\pgfqpoint{1.725939in}{2.318593in}}%
\pgfpathcurveto{\pgfqpoint{1.718126in}{2.310780in}}{\pgfqpoint{1.713735in}{2.300181in}}{\pgfqpoint{1.713735in}{2.289131in}}%
\pgfpathcurveto{\pgfqpoint{1.713735in}{2.278080in}}{\pgfqpoint{1.718126in}{2.267481in}}{\pgfqpoint{1.725939in}{2.259668in}}%
\pgfpathcurveto{\pgfqpoint{1.733753in}{2.251854in}}{\pgfqpoint{1.744352in}{2.247464in}}{\pgfqpoint{1.755402in}{2.247464in}}%
\pgfpathclose%
\pgfusepath{stroke,fill}%
\end{pgfscope}%
\begin{pgfscope}%
\pgfpathrectangle{\pgfqpoint{0.511823in}{0.504323in}}{\pgfqpoint{3.218177in}{3.225677in}} %
\pgfusepath{clip}%
\pgfsetbuttcap%
\pgfsetroundjoin%
\definecolor{currentfill}{rgb}{0.000000,0.000000,0.545098}%
\pgfsetfillcolor{currentfill}%
\pgfsetfillopacity{0.400000}%
\pgfsetlinewidth{0.501875pt}%
\definecolor{currentstroke}{rgb}{0.000000,0.000000,0.545098}%
\pgfsetstrokecolor{currentstroke}%
\pgfsetstrokeopacity{0.400000}%
\pgfsetdash{}{0pt}%
\pgfpathmoveto{\pgfqpoint{1.500014in}{1.872017in}}%
\pgfpathcurveto{\pgfqpoint{1.511064in}{1.872017in}}{\pgfqpoint{1.521663in}{1.876407in}}{\pgfqpoint{1.529476in}{1.884221in}}%
\pgfpathcurveto{\pgfqpoint{1.537290in}{1.892034in}}{\pgfqpoint{1.541680in}{1.902633in}}{\pgfqpoint{1.541680in}{1.913684in}}%
\pgfpathcurveto{\pgfqpoint{1.541680in}{1.924734in}}{\pgfqpoint{1.537290in}{1.935333in}}{\pgfqpoint{1.529476in}{1.943146in}}%
\pgfpathcurveto{\pgfqpoint{1.521663in}{1.950960in}}{\pgfqpoint{1.511064in}{1.955350in}}{\pgfqpoint{1.500014in}{1.955350in}}%
\pgfpathcurveto{\pgfqpoint{1.488963in}{1.955350in}}{\pgfqpoint{1.478364in}{1.950960in}}{\pgfqpoint{1.470551in}{1.943146in}}%
\pgfpathcurveto{\pgfqpoint{1.462737in}{1.935333in}}{\pgfqpoint{1.458347in}{1.924734in}}{\pgfqpoint{1.458347in}{1.913684in}}%
\pgfpathcurveto{\pgfqpoint{1.458347in}{1.902633in}}{\pgfqpoint{1.462737in}{1.892034in}}{\pgfqpoint{1.470551in}{1.884221in}}%
\pgfpathcurveto{\pgfqpoint{1.478364in}{1.876407in}}{\pgfqpoint{1.488963in}{1.872017in}}{\pgfqpoint{1.500014in}{1.872017in}}%
\pgfpathclose%
\pgfusepath{stroke,fill}%
\end{pgfscope}%
\begin{pgfscope}%
\pgfpathrectangle{\pgfqpoint{0.511823in}{0.504323in}}{\pgfqpoint{3.218177in}{3.225677in}} %
\pgfusepath{clip}%
\pgfsetbuttcap%
\pgfsetroundjoin%
\definecolor{currentfill}{rgb}{0.000000,0.000000,0.545098}%
\pgfsetfillcolor{currentfill}%
\pgfsetfillopacity{0.400000}%
\pgfsetlinewidth{0.501875pt}%
\definecolor{currentstroke}{rgb}{0.000000,0.000000,0.545098}%
\pgfsetstrokecolor{currentstroke}%
\pgfsetstrokeopacity{0.400000}%
\pgfsetdash{}{0pt}%
\pgfpathmoveto{\pgfqpoint{1.668913in}{2.138392in}}%
\pgfpathcurveto{\pgfqpoint{1.679963in}{2.138392in}}{\pgfqpoint{1.690562in}{2.142782in}}{\pgfqpoint{1.698376in}{2.150596in}}%
\pgfpathcurveto{\pgfqpoint{1.706189in}{2.158410in}}{\pgfqpoint{1.710580in}{2.169009in}}{\pgfqpoint{1.710580in}{2.180059in}}%
\pgfpathcurveto{\pgfqpoint{1.710580in}{2.191109in}}{\pgfqpoint{1.706189in}{2.201708in}}{\pgfqpoint{1.698376in}{2.209522in}}%
\pgfpathcurveto{\pgfqpoint{1.690562in}{2.217335in}}{\pgfqpoint{1.679963in}{2.221726in}}{\pgfqpoint{1.668913in}{2.221726in}}%
\pgfpathcurveto{\pgfqpoint{1.657863in}{2.221726in}}{\pgfqpoint{1.647264in}{2.217335in}}{\pgfqpoint{1.639450in}{2.209522in}}%
\pgfpathcurveto{\pgfqpoint{1.631637in}{2.201708in}}{\pgfqpoint{1.627246in}{2.191109in}}{\pgfqpoint{1.627246in}{2.180059in}}%
\pgfpathcurveto{\pgfqpoint{1.627246in}{2.169009in}}{\pgfqpoint{1.631637in}{2.158410in}}{\pgfqpoint{1.639450in}{2.150596in}}%
\pgfpathcurveto{\pgfqpoint{1.647264in}{2.142782in}}{\pgfqpoint{1.657863in}{2.138392in}}{\pgfqpoint{1.668913in}{2.138392in}}%
\pgfpathclose%
\pgfusepath{stroke,fill}%
\end{pgfscope}%
\begin{pgfscope}%
\pgfpathrectangle{\pgfqpoint{0.511823in}{0.504323in}}{\pgfqpoint{3.218177in}{3.225677in}} %
\pgfusepath{clip}%
\pgfsetbuttcap%
\pgfsetroundjoin%
\definecolor{currentfill}{rgb}{0.000000,0.000000,0.545098}%
\pgfsetfillcolor{currentfill}%
\pgfsetfillopacity{0.400000}%
\pgfsetlinewidth{0.501875pt}%
\definecolor{currentstroke}{rgb}{0.000000,0.000000,0.545098}%
\pgfsetstrokecolor{currentstroke}%
\pgfsetstrokeopacity{0.400000}%
\pgfsetdash{}{0pt}%
\pgfpathmoveto{\pgfqpoint{1.782671in}{2.323816in}}%
\pgfpathcurveto{\pgfqpoint{1.793721in}{2.323816in}}{\pgfqpoint{1.804320in}{2.328206in}}{\pgfqpoint{1.812134in}{2.336020in}}%
\pgfpathcurveto{\pgfqpoint{1.819947in}{2.343834in}}{\pgfqpoint{1.824338in}{2.354433in}}{\pgfqpoint{1.824338in}{2.365483in}}%
\pgfpathcurveto{\pgfqpoint{1.824338in}{2.376533in}}{\pgfqpoint{1.819947in}{2.387132in}}{\pgfqpoint{1.812134in}{2.394946in}}%
\pgfpathcurveto{\pgfqpoint{1.804320in}{2.402759in}}{\pgfqpoint{1.793721in}{2.407149in}}{\pgfqpoint{1.782671in}{2.407149in}}%
\pgfpathcurveto{\pgfqpoint{1.771621in}{2.407149in}}{\pgfqpoint{1.761022in}{2.402759in}}{\pgfqpoint{1.753208in}{2.394946in}}%
\pgfpathcurveto{\pgfqpoint{1.745395in}{2.387132in}}{\pgfqpoint{1.741004in}{2.376533in}}{\pgfqpoint{1.741004in}{2.365483in}}%
\pgfpathcurveto{\pgfqpoint{1.741004in}{2.354433in}}{\pgfqpoint{1.745395in}{2.343834in}}{\pgfqpoint{1.753208in}{2.336020in}}%
\pgfpathcurveto{\pgfqpoint{1.761022in}{2.328206in}}{\pgfqpoint{1.771621in}{2.323816in}}{\pgfqpoint{1.782671in}{2.323816in}}%
\pgfpathclose%
\pgfusepath{stroke,fill}%
\end{pgfscope}%
\begin{pgfscope}%
\pgfpathrectangle{\pgfqpoint{0.511823in}{0.504323in}}{\pgfqpoint{3.218177in}{3.225677in}} %
\pgfusepath{clip}%
\pgfsetbuttcap%
\pgfsetroundjoin%
\definecolor{currentfill}{rgb}{0.000000,0.000000,0.545098}%
\pgfsetfillcolor{currentfill}%
\pgfsetfillopacity{0.400000}%
\pgfsetlinewidth{0.501875pt}%
\definecolor{currentstroke}{rgb}{0.000000,0.000000,0.545098}%
\pgfsetstrokecolor{currentstroke}%
\pgfsetstrokeopacity{0.400000}%
\pgfsetdash{}{0pt}%
\pgfpathmoveto{\pgfqpoint{1.750485in}{2.286045in}}%
\pgfpathcurveto{\pgfqpoint{1.761536in}{2.286045in}}{\pgfqpoint{1.772135in}{2.290435in}}{\pgfqpoint{1.779948in}{2.298249in}}%
\pgfpathcurveto{\pgfqpoint{1.787762in}{2.306062in}}{\pgfqpoint{1.792152in}{2.316661in}}{\pgfqpoint{1.792152in}{2.327711in}}%
\pgfpathcurveto{\pgfqpoint{1.792152in}{2.338762in}}{\pgfqpoint{1.787762in}{2.349361in}}{\pgfqpoint{1.779948in}{2.357174in}}%
\pgfpathcurveto{\pgfqpoint{1.772135in}{2.364988in}}{\pgfqpoint{1.761536in}{2.369378in}}{\pgfqpoint{1.750485in}{2.369378in}}%
\pgfpathcurveto{\pgfqpoint{1.739435in}{2.369378in}}{\pgfqpoint{1.728836in}{2.364988in}}{\pgfqpoint{1.721023in}{2.357174in}}%
\pgfpathcurveto{\pgfqpoint{1.713209in}{2.349361in}}{\pgfqpoint{1.708819in}{2.338762in}}{\pgfqpoint{1.708819in}{2.327711in}}%
\pgfpathcurveto{\pgfqpoint{1.708819in}{2.316661in}}{\pgfqpoint{1.713209in}{2.306062in}}{\pgfqpoint{1.721023in}{2.298249in}}%
\pgfpathcurveto{\pgfqpoint{1.728836in}{2.290435in}}{\pgfqpoint{1.739435in}{2.286045in}}{\pgfqpoint{1.750485in}{2.286045in}}%
\pgfpathclose%
\pgfusepath{stroke,fill}%
\end{pgfscope}%
\begin{pgfscope}%
\pgfpathrectangle{\pgfqpoint{0.511823in}{0.504323in}}{\pgfqpoint{3.218177in}{3.225677in}} %
\pgfusepath{clip}%
\pgfsetbuttcap%
\pgfsetroundjoin%
\definecolor{currentfill}{rgb}{0.000000,0.000000,0.545098}%
\pgfsetfillcolor{currentfill}%
\pgfsetfillopacity{0.400000}%
\pgfsetlinewidth{0.501875pt}%
\definecolor{currentstroke}{rgb}{0.000000,0.000000,0.545098}%
\pgfsetstrokecolor{currentstroke}%
\pgfsetstrokeopacity{0.400000}%
\pgfsetdash{}{0pt}%
\pgfpathmoveto{\pgfqpoint{1.660327in}{2.157349in}}%
\pgfpathcurveto{\pgfqpoint{1.671377in}{2.157349in}}{\pgfqpoint{1.681976in}{2.161739in}}{\pgfqpoint{1.689790in}{2.169552in}}%
\pgfpathcurveto{\pgfqpoint{1.697603in}{2.177366in}}{\pgfqpoint{1.701994in}{2.187965in}}{\pgfqpoint{1.701994in}{2.199015in}}%
\pgfpathcurveto{\pgfqpoint{1.701994in}{2.210065in}}{\pgfqpoint{1.697603in}{2.220664in}}{\pgfqpoint{1.689790in}{2.228478in}}%
\pgfpathcurveto{\pgfqpoint{1.681976in}{2.236292in}}{\pgfqpoint{1.671377in}{2.240682in}}{\pgfqpoint{1.660327in}{2.240682in}}%
\pgfpathcurveto{\pgfqpoint{1.649277in}{2.240682in}}{\pgfqpoint{1.638678in}{2.236292in}}{\pgfqpoint{1.630864in}{2.228478in}}%
\pgfpathcurveto{\pgfqpoint{1.623051in}{2.220664in}}{\pgfqpoint{1.618660in}{2.210065in}}{\pgfqpoint{1.618660in}{2.199015in}}%
\pgfpathcurveto{\pgfqpoint{1.618660in}{2.187965in}}{\pgfqpoint{1.623051in}{2.177366in}}{\pgfqpoint{1.630864in}{2.169552in}}%
\pgfpathcurveto{\pgfqpoint{1.638678in}{2.161739in}}{\pgfqpoint{1.649277in}{2.157349in}}{\pgfqpoint{1.660327in}{2.157349in}}%
\pgfpathclose%
\pgfusepath{stroke,fill}%
\end{pgfscope}%
\begin{pgfscope}%
\pgfpathrectangle{\pgfqpoint{0.511823in}{0.504323in}}{\pgfqpoint{3.218177in}{3.225677in}} %
\pgfusepath{clip}%
\pgfsetbuttcap%
\pgfsetroundjoin%
\definecolor{currentfill}{rgb}{0.000000,0.000000,0.545098}%
\pgfsetfillcolor{currentfill}%
\pgfsetfillopacity{0.400000}%
\pgfsetlinewidth{0.501875pt}%
\definecolor{currentstroke}{rgb}{0.000000,0.000000,0.545098}%
\pgfsetstrokecolor{currentstroke}%
\pgfsetstrokeopacity{0.400000}%
\pgfsetdash{}{0pt}%
\pgfpathmoveto{\pgfqpoint{1.658987in}{2.166156in}}%
\pgfpathcurveto{\pgfqpoint{1.670037in}{2.166156in}}{\pgfqpoint{1.680636in}{2.170546in}}{\pgfqpoint{1.688449in}{2.178360in}}%
\pgfpathcurveto{\pgfqpoint{1.696263in}{2.186173in}}{\pgfqpoint{1.700653in}{2.196772in}}{\pgfqpoint{1.700653in}{2.207822in}}%
\pgfpathcurveto{\pgfqpoint{1.700653in}{2.218872in}}{\pgfqpoint{1.696263in}{2.229471in}}{\pgfqpoint{1.688449in}{2.237285in}}%
\pgfpathcurveto{\pgfqpoint{1.680636in}{2.245099in}}{\pgfqpoint{1.670037in}{2.249489in}}{\pgfqpoint{1.658987in}{2.249489in}}%
\pgfpathcurveto{\pgfqpoint{1.647937in}{2.249489in}}{\pgfqpoint{1.637338in}{2.245099in}}{\pgfqpoint{1.629524in}{2.237285in}}%
\pgfpathcurveto{\pgfqpoint{1.621710in}{2.229471in}}{\pgfqpoint{1.617320in}{2.218872in}}{\pgfqpoint{1.617320in}{2.207822in}}%
\pgfpathcurveto{\pgfqpoint{1.617320in}{2.196772in}}{\pgfqpoint{1.621710in}{2.186173in}}{\pgfqpoint{1.629524in}{2.178360in}}%
\pgfpathcurveto{\pgfqpoint{1.637338in}{2.170546in}}{\pgfqpoint{1.647937in}{2.166156in}}{\pgfqpoint{1.658987in}{2.166156in}}%
\pgfpathclose%
\pgfusepath{stroke,fill}%
\end{pgfscope}%
\begin{pgfscope}%
\pgfpathrectangle{\pgfqpoint{0.511823in}{0.504323in}}{\pgfqpoint{3.218177in}{3.225677in}} %
\pgfusepath{clip}%
\pgfsetbuttcap%
\pgfsetroundjoin%
\definecolor{currentfill}{rgb}{0.000000,0.000000,0.545098}%
\pgfsetfillcolor{currentfill}%
\pgfsetfillopacity{0.400000}%
\pgfsetlinewidth{0.501875pt}%
\definecolor{currentstroke}{rgb}{0.000000,0.000000,0.545098}%
\pgfsetstrokecolor{currentstroke}%
\pgfsetstrokeopacity{0.400000}%
\pgfsetdash{}{0pt}%
\pgfpathmoveto{\pgfqpoint{1.772591in}{2.356754in}}%
\pgfpathcurveto{\pgfqpoint{1.783641in}{2.356754in}}{\pgfqpoint{1.794240in}{2.361144in}}{\pgfqpoint{1.802054in}{2.368957in}}%
\pgfpathcurveto{\pgfqpoint{1.809867in}{2.376771in}}{\pgfqpoint{1.814258in}{2.387370in}}{\pgfqpoint{1.814258in}{2.398420in}}%
\pgfpathcurveto{\pgfqpoint{1.814258in}{2.409470in}}{\pgfqpoint{1.809867in}{2.420069in}}{\pgfqpoint{1.802054in}{2.427883in}}%
\pgfpathcurveto{\pgfqpoint{1.794240in}{2.435697in}}{\pgfqpoint{1.783641in}{2.440087in}}{\pgfqpoint{1.772591in}{2.440087in}}%
\pgfpathcurveto{\pgfqpoint{1.761541in}{2.440087in}}{\pgfqpoint{1.750942in}{2.435697in}}{\pgfqpoint{1.743128in}{2.427883in}}%
\pgfpathcurveto{\pgfqpoint{1.735315in}{2.420069in}}{\pgfqpoint{1.730924in}{2.409470in}}{\pgfqpoint{1.730924in}{2.398420in}}%
\pgfpathcurveto{\pgfqpoint{1.730924in}{2.387370in}}{\pgfqpoint{1.735315in}{2.376771in}}{\pgfqpoint{1.743128in}{2.368957in}}%
\pgfpathcurveto{\pgfqpoint{1.750942in}{2.361144in}}{\pgfqpoint{1.761541in}{2.356754in}}{\pgfqpoint{1.772591in}{2.356754in}}%
\pgfpathclose%
\pgfusepath{stroke,fill}%
\end{pgfscope}%
\begin{pgfscope}%
\pgfpathrectangle{\pgfqpoint{0.511823in}{0.504323in}}{\pgfqpoint{3.218177in}{3.225677in}} %
\pgfusepath{clip}%
\pgfsetbuttcap%
\pgfsetroundjoin%
\definecolor{currentfill}{rgb}{0.000000,0.000000,0.545098}%
\pgfsetfillcolor{currentfill}%
\pgfsetfillopacity{0.400000}%
\pgfsetlinewidth{0.501875pt}%
\definecolor{currentstroke}{rgb}{0.000000,0.000000,0.545098}%
\pgfsetstrokecolor{currentstroke}%
\pgfsetstrokeopacity{0.400000}%
\pgfsetdash{}{0pt}%
\pgfpathmoveto{\pgfqpoint{1.673858in}{2.211977in}}%
\pgfpathcurveto{\pgfqpoint{1.684908in}{2.211977in}}{\pgfqpoint{1.695507in}{2.216367in}}{\pgfqpoint{1.703321in}{2.224181in}}%
\pgfpathcurveto{\pgfqpoint{1.711135in}{2.231995in}}{\pgfqpoint{1.715525in}{2.242594in}}{\pgfqpoint{1.715525in}{2.253644in}}%
\pgfpathcurveto{\pgfqpoint{1.715525in}{2.264694in}}{\pgfqpoint{1.711135in}{2.275293in}}{\pgfqpoint{1.703321in}{2.283106in}}%
\pgfpathcurveto{\pgfqpoint{1.695507in}{2.290920in}}{\pgfqpoint{1.684908in}{2.295310in}}{\pgfqpoint{1.673858in}{2.295310in}}%
\pgfpathcurveto{\pgfqpoint{1.662808in}{2.295310in}}{\pgfqpoint{1.652209in}{2.290920in}}{\pgfqpoint{1.644395in}{2.283106in}}%
\pgfpathcurveto{\pgfqpoint{1.636582in}{2.275293in}}{\pgfqpoint{1.632191in}{2.264694in}}{\pgfqpoint{1.632191in}{2.253644in}}%
\pgfpathcurveto{\pgfqpoint{1.632191in}{2.242594in}}{\pgfqpoint{1.636582in}{2.231995in}}{\pgfqpoint{1.644395in}{2.224181in}}%
\pgfpathcurveto{\pgfqpoint{1.652209in}{2.216367in}}{\pgfqpoint{1.662808in}{2.211977in}}{\pgfqpoint{1.673858in}{2.211977in}}%
\pgfpathclose%
\pgfusepath{stroke,fill}%
\end{pgfscope}%
\begin{pgfscope}%
\pgfpathrectangle{\pgfqpoint{0.511823in}{0.504323in}}{\pgfqpoint{3.218177in}{3.225677in}} %
\pgfusepath{clip}%
\pgfsetbuttcap%
\pgfsetroundjoin%
\definecolor{currentfill}{rgb}{0.000000,0.000000,0.545098}%
\pgfsetfillcolor{currentfill}%
\pgfsetfillopacity{0.400000}%
\pgfsetlinewidth{0.501875pt}%
\definecolor{currentstroke}{rgb}{0.000000,0.000000,0.545098}%
\pgfsetstrokecolor{currentstroke}%
\pgfsetstrokeopacity{0.400000}%
\pgfsetdash{}{0pt}%
\pgfpathmoveto{\pgfqpoint{1.649729in}{2.184701in}}%
\pgfpathcurveto{\pgfqpoint{1.660779in}{2.184701in}}{\pgfqpoint{1.671378in}{2.189092in}}{\pgfqpoint{1.679192in}{2.196905in}}%
\pgfpathcurveto{\pgfqpoint{1.687005in}{2.204719in}}{\pgfqpoint{1.691396in}{2.215318in}}{\pgfqpoint{1.691396in}{2.226368in}}%
\pgfpathcurveto{\pgfqpoint{1.691396in}{2.237418in}}{\pgfqpoint{1.687005in}{2.248017in}}{\pgfqpoint{1.679192in}{2.255831in}}%
\pgfpathcurveto{\pgfqpoint{1.671378in}{2.263644in}}{\pgfqpoint{1.660779in}{2.268035in}}{\pgfqpoint{1.649729in}{2.268035in}}%
\pgfpathcurveto{\pgfqpoint{1.638679in}{2.268035in}}{\pgfqpoint{1.628080in}{2.263644in}}{\pgfqpoint{1.620266in}{2.255831in}}%
\pgfpathcurveto{\pgfqpoint{1.612453in}{2.248017in}}{\pgfqpoint{1.608062in}{2.237418in}}{\pgfqpoint{1.608062in}{2.226368in}}%
\pgfpathcurveto{\pgfqpoint{1.608062in}{2.215318in}}{\pgfqpoint{1.612453in}{2.204719in}}{\pgfqpoint{1.620266in}{2.196905in}}%
\pgfpathcurveto{\pgfqpoint{1.628080in}{2.189092in}}{\pgfqpoint{1.638679in}{2.184701in}}{\pgfqpoint{1.649729in}{2.184701in}}%
\pgfpathclose%
\pgfusepath{stroke,fill}%
\end{pgfscope}%
\begin{pgfscope}%
\pgfpathrectangle{\pgfqpoint{0.511823in}{0.504323in}}{\pgfqpoint{3.218177in}{3.225677in}} %
\pgfusepath{clip}%
\pgfsetbuttcap%
\pgfsetroundjoin%
\definecolor{currentfill}{rgb}{0.000000,0.000000,0.545098}%
\pgfsetfillcolor{currentfill}%
\pgfsetfillopacity{0.400000}%
\pgfsetlinewidth{0.501875pt}%
\definecolor{currentstroke}{rgb}{0.000000,0.000000,0.545098}%
\pgfsetstrokecolor{currentstroke}%
\pgfsetstrokeopacity{0.400000}%
\pgfsetdash{}{0pt}%
\pgfpathmoveto{\pgfqpoint{1.802471in}{2.442724in}}%
\pgfpathcurveto{\pgfqpoint{1.813522in}{2.442724in}}{\pgfqpoint{1.824121in}{2.447114in}}{\pgfqpoint{1.831934in}{2.454928in}}%
\pgfpathcurveto{\pgfqpoint{1.839748in}{2.462741in}}{\pgfqpoint{1.844138in}{2.473340in}}{\pgfqpoint{1.844138in}{2.484390in}}%
\pgfpathcurveto{\pgfqpoint{1.844138in}{2.495441in}}{\pgfqpoint{1.839748in}{2.506040in}}{\pgfqpoint{1.831934in}{2.513853in}}%
\pgfpathcurveto{\pgfqpoint{1.824121in}{2.521667in}}{\pgfqpoint{1.813522in}{2.526057in}}{\pgfqpoint{1.802471in}{2.526057in}}%
\pgfpathcurveto{\pgfqpoint{1.791421in}{2.526057in}}{\pgfqpoint{1.780822in}{2.521667in}}{\pgfqpoint{1.773009in}{2.513853in}}%
\pgfpathcurveto{\pgfqpoint{1.765195in}{2.506040in}}{\pgfqpoint{1.760805in}{2.495441in}}{\pgfqpoint{1.760805in}{2.484390in}}%
\pgfpathcurveto{\pgfqpoint{1.760805in}{2.473340in}}{\pgfqpoint{1.765195in}{2.462741in}}{\pgfqpoint{1.773009in}{2.454928in}}%
\pgfpathcurveto{\pgfqpoint{1.780822in}{2.447114in}}{\pgfqpoint{1.791421in}{2.442724in}}{\pgfqpoint{1.802471in}{2.442724in}}%
\pgfpathclose%
\pgfusepath{stroke,fill}%
\end{pgfscope}%
\begin{pgfscope}%
\pgfpathrectangle{\pgfqpoint{0.511823in}{0.504323in}}{\pgfqpoint{3.218177in}{3.225677in}} %
\pgfusepath{clip}%
\pgfsetbuttcap%
\pgfsetroundjoin%
\definecolor{currentfill}{rgb}{0.000000,0.000000,0.545098}%
\pgfsetfillcolor{currentfill}%
\pgfsetfillopacity{0.400000}%
\pgfsetlinewidth{0.501875pt}%
\definecolor{currentstroke}{rgb}{0.000000,0.000000,0.545098}%
\pgfsetstrokecolor{currentstroke}%
\pgfsetstrokeopacity{0.400000}%
\pgfsetdash{}{0pt}%
\pgfpathmoveto{\pgfqpoint{1.649814in}{2.207495in}}%
\pgfpathcurveto{\pgfqpoint{1.660864in}{2.207495in}}{\pgfqpoint{1.671463in}{2.211885in}}{\pgfqpoint{1.679277in}{2.219698in}}%
\pgfpathcurveto{\pgfqpoint{1.687091in}{2.227512in}}{\pgfqpoint{1.691481in}{2.238111in}}{\pgfqpoint{1.691481in}{2.249161in}}%
\pgfpathcurveto{\pgfqpoint{1.691481in}{2.260211in}}{\pgfqpoint{1.687091in}{2.270810in}}{\pgfqpoint{1.679277in}{2.278624in}}%
\pgfpathcurveto{\pgfqpoint{1.671463in}{2.286438in}}{\pgfqpoint{1.660864in}{2.290828in}}{\pgfqpoint{1.649814in}{2.290828in}}%
\pgfpathcurveto{\pgfqpoint{1.638764in}{2.290828in}}{\pgfqpoint{1.628165in}{2.286438in}}{\pgfqpoint{1.620351in}{2.278624in}}%
\pgfpathcurveto{\pgfqpoint{1.612538in}{2.270810in}}{\pgfqpoint{1.608148in}{2.260211in}}{\pgfqpoint{1.608148in}{2.249161in}}%
\pgfpathcurveto{\pgfqpoint{1.608148in}{2.238111in}}{\pgfqpoint{1.612538in}{2.227512in}}{\pgfqpoint{1.620351in}{2.219698in}}%
\pgfpathcurveto{\pgfqpoint{1.628165in}{2.211885in}}{\pgfqpoint{1.638764in}{2.207495in}}{\pgfqpoint{1.649814in}{2.207495in}}%
\pgfpathclose%
\pgfusepath{stroke,fill}%
\end{pgfscope}%
\begin{pgfscope}%
\pgfpathrectangle{\pgfqpoint{0.511823in}{0.504323in}}{\pgfqpoint{3.218177in}{3.225677in}} %
\pgfusepath{clip}%
\pgfsetbuttcap%
\pgfsetroundjoin%
\definecolor{currentfill}{rgb}{0.000000,0.000000,0.545098}%
\pgfsetfillcolor{currentfill}%
\pgfsetfillopacity{0.400000}%
\pgfsetlinewidth{0.501875pt}%
\definecolor{currentstroke}{rgb}{0.000000,0.000000,0.545098}%
\pgfsetstrokecolor{currentstroke}%
\pgfsetstrokeopacity{0.400000}%
\pgfsetdash{}{0pt}%
\pgfpathmoveto{\pgfqpoint{1.713535in}{2.323468in}}%
\pgfpathcurveto{\pgfqpoint{1.724586in}{2.323468in}}{\pgfqpoint{1.735185in}{2.327859in}}{\pgfqpoint{1.742998in}{2.335672in}}%
\pgfpathcurveto{\pgfqpoint{1.750812in}{2.343486in}}{\pgfqpoint{1.755202in}{2.354085in}}{\pgfqpoint{1.755202in}{2.365135in}}%
\pgfpathcurveto{\pgfqpoint{1.755202in}{2.376185in}}{\pgfqpoint{1.750812in}{2.386784in}}{\pgfqpoint{1.742998in}{2.394598in}}%
\pgfpathcurveto{\pgfqpoint{1.735185in}{2.402411in}}{\pgfqpoint{1.724586in}{2.406802in}}{\pgfqpoint{1.713535in}{2.406802in}}%
\pgfpathcurveto{\pgfqpoint{1.702485in}{2.406802in}}{\pgfqpoint{1.691886in}{2.402411in}}{\pgfqpoint{1.684073in}{2.394598in}}%
\pgfpathcurveto{\pgfqpoint{1.676259in}{2.386784in}}{\pgfqpoint{1.671869in}{2.376185in}}{\pgfqpoint{1.671869in}{2.365135in}}%
\pgfpathcurveto{\pgfqpoint{1.671869in}{2.354085in}}{\pgfqpoint{1.676259in}{2.343486in}}{\pgfqpoint{1.684073in}{2.335672in}}%
\pgfpathcurveto{\pgfqpoint{1.691886in}{2.327859in}}{\pgfqpoint{1.702485in}{2.323468in}}{\pgfqpoint{1.713535in}{2.323468in}}%
\pgfpathclose%
\pgfusepath{stroke,fill}%
\end{pgfscope}%
\begin{pgfscope}%
\pgfpathrectangle{\pgfqpoint{0.511823in}{0.504323in}}{\pgfqpoint{3.218177in}{3.225677in}} %
\pgfusepath{clip}%
\pgfsetbuttcap%
\pgfsetroundjoin%
\definecolor{currentfill}{rgb}{0.000000,0.000000,0.545098}%
\pgfsetfillcolor{currentfill}%
\pgfsetfillopacity{0.400000}%
\pgfsetlinewidth{0.501875pt}%
\definecolor{currentstroke}{rgb}{0.000000,0.000000,0.545098}%
\pgfsetstrokecolor{currentstroke}%
\pgfsetstrokeopacity{0.400000}%
\pgfsetdash{}{0pt}%
\pgfpathmoveto{\pgfqpoint{1.660300in}{2.247972in}}%
\pgfpathcurveto{\pgfqpoint{1.671350in}{2.247972in}}{\pgfqpoint{1.681949in}{2.252362in}}{\pgfqpoint{1.689763in}{2.260176in}}%
\pgfpathcurveto{\pgfqpoint{1.697577in}{2.267989in}}{\pgfqpoint{1.701967in}{2.278589in}}{\pgfqpoint{1.701967in}{2.289639in}}%
\pgfpathcurveto{\pgfqpoint{1.701967in}{2.300689in}}{\pgfqpoint{1.697577in}{2.311288in}}{\pgfqpoint{1.689763in}{2.319101in}}%
\pgfpathcurveto{\pgfqpoint{1.681949in}{2.326915in}}{\pgfqpoint{1.671350in}{2.331305in}}{\pgfqpoint{1.660300in}{2.331305in}}%
\pgfpathcurveto{\pgfqpoint{1.649250in}{2.331305in}}{\pgfqpoint{1.638651in}{2.326915in}}{\pgfqpoint{1.630837in}{2.319101in}}%
\pgfpathcurveto{\pgfqpoint{1.623024in}{2.311288in}}{\pgfqpoint{1.618634in}{2.300689in}}{\pgfqpoint{1.618634in}{2.289639in}}%
\pgfpathcurveto{\pgfqpoint{1.618634in}{2.278589in}}{\pgfqpoint{1.623024in}{2.267989in}}{\pgfqpoint{1.630837in}{2.260176in}}%
\pgfpathcurveto{\pgfqpoint{1.638651in}{2.252362in}}{\pgfqpoint{1.649250in}{2.247972in}}{\pgfqpoint{1.660300in}{2.247972in}}%
\pgfpathclose%
\pgfusepath{stroke,fill}%
\end{pgfscope}%
\begin{pgfscope}%
\pgfpathrectangle{\pgfqpoint{0.511823in}{0.504323in}}{\pgfqpoint{3.218177in}{3.225677in}} %
\pgfusepath{clip}%
\pgfsetbuttcap%
\pgfsetroundjoin%
\definecolor{currentfill}{rgb}{0.000000,0.000000,0.545098}%
\pgfsetfillcolor{currentfill}%
\pgfsetfillopacity{0.400000}%
\pgfsetlinewidth{0.501875pt}%
\definecolor{currentstroke}{rgb}{0.000000,0.000000,0.545098}%
\pgfsetstrokecolor{currentstroke}%
\pgfsetstrokeopacity{0.400000}%
\pgfsetdash{}{0pt}%
\pgfpathmoveto{\pgfqpoint{1.661101in}{2.261208in}}%
\pgfpathcurveto{\pgfqpoint{1.672151in}{2.261208in}}{\pgfqpoint{1.682750in}{2.265598in}}{\pgfqpoint{1.690564in}{2.273412in}}%
\pgfpathcurveto{\pgfqpoint{1.698377in}{2.281225in}}{\pgfqpoint{1.702768in}{2.291824in}}{\pgfqpoint{1.702768in}{2.302875in}}%
\pgfpathcurveto{\pgfqpoint{1.702768in}{2.313925in}}{\pgfqpoint{1.698377in}{2.324524in}}{\pgfqpoint{1.690564in}{2.332337in}}%
\pgfpathcurveto{\pgfqpoint{1.682750in}{2.340151in}}{\pgfqpoint{1.672151in}{2.344541in}}{\pgfqpoint{1.661101in}{2.344541in}}%
\pgfpathcurveto{\pgfqpoint{1.650051in}{2.344541in}}{\pgfqpoint{1.639452in}{2.340151in}}{\pgfqpoint{1.631638in}{2.332337in}}%
\pgfpathcurveto{\pgfqpoint{1.623824in}{2.324524in}}{\pgfqpoint{1.619434in}{2.313925in}}{\pgfqpoint{1.619434in}{2.302875in}}%
\pgfpathcurveto{\pgfqpoint{1.619434in}{2.291824in}}{\pgfqpoint{1.623824in}{2.281225in}}{\pgfqpoint{1.631638in}{2.273412in}}%
\pgfpathcurveto{\pgfqpoint{1.639452in}{2.265598in}}{\pgfqpoint{1.650051in}{2.261208in}}{\pgfqpoint{1.661101in}{2.261208in}}%
\pgfpathclose%
\pgfusepath{stroke,fill}%
\end{pgfscope}%
\begin{pgfscope}%
\pgfpathrectangle{\pgfqpoint{0.511823in}{0.504323in}}{\pgfqpoint{3.218177in}{3.225677in}} %
\pgfusepath{clip}%
\pgfsetbuttcap%
\pgfsetroundjoin%
\definecolor{currentfill}{rgb}{0.000000,0.000000,0.545098}%
\pgfsetfillcolor{currentfill}%
\pgfsetfillopacity{0.400000}%
\pgfsetlinewidth{0.501875pt}%
\definecolor{currentstroke}{rgb}{0.000000,0.000000,0.545098}%
\pgfsetstrokecolor{currentstroke}%
\pgfsetstrokeopacity{0.400000}%
\pgfsetdash{}{0pt}%
\pgfpathmoveto{\pgfqpoint{1.591345in}{2.156348in}}%
\pgfpathcurveto{\pgfqpoint{1.602395in}{2.156348in}}{\pgfqpoint{1.612994in}{2.160739in}}{\pgfqpoint{1.620808in}{2.168552in}}%
\pgfpathcurveto{\pgfqpoint{1.628621in}{2.176366in}}{\pgfqpoint{1.633012in}{2.186965in}}{\pgfqpoint{1.633012in}{2.198015in}}%
\pgfpathcurveto{\pgfqpoint{1.633012in}{2.209065in}}{\pgfqpoint{1.628621in}{2.219664in}}{\pgfqpoint{1.620808in}{2.227478in}}%
\pgfpathcurveto{\pgfqpoint{1.612994in}{2.235291in}}{\pgfqpoint{1.602395in}{2.239682in}}{\pgfqpoint{1.591345in}{2.239682in}}%
\pgfpathcurveto{\pgfqpoint{1.580295in}{2.239682in}}{\pgfqpoint{1.569696in}{2.235291in}}{\pgfqpoint{1.561882in}{2.227478in}}%
\pgfpathcurveto{\pgfqpoint{1.554069in}{2.219664in}}{\pgfqpoint{1.549678in}{2.209065in}}{\pgfqpoint{1.549678in}{2.198015in}}%
\pgfpathcurveto{\pgfqpoint{1.549678in}{2.186965in}}{\pgfqpoint{1.554069in}{2.176366in}}{\pgfqpoint{1.561882in}{2.168552in}}%
\pgfpathcurveto{\pgfqpoint{1.569696in}{2.160739in}}{\pgfqpoint{1.580295in}{2.156348in}}{\pgfqpoint{1.591345in}{2.156348in}}%
\pgfpathclose%
\pgfusepath{stroke,fill}%
\end{pgfscope}%
\begin{pgfscope}%
\pgfpathrectangle{\pgfqpoint{0.511823in}{0.504323in}}{\pgfqpoint{3.218177in}{3.225677in}} %
\pgfusepath{clip}%
\pgfsetbuttcap%
\pgfsetroundjoin%
\definecolor{currentfill}{rgb}{0.000000,0.000000,0.545098}%
\pgfsetfillcolor{currentfill}%
\pgfsetfillopacity{0.400000}%
\pgfsetlinewidth{0.501875pt}%
\definecolor{currentstroke}{rgb}{0.000000,0.000000,0.545098}%
\pgfsetstrokecolor{currentstroke}%
\pgfsetstrokeopacity{0.400000}%
\pgfsetdash{}{0pt}%
\pgfpathmoveto{\pgfqpoint{1.593908in}{2.171990in}}%
\pgfpathcurveto{\pgfqpoint{1.604958in}{2.171990in}}{\pgfqpoint{1.615557in}{2.176380in}}{\pgfqpoint{1.623371in}{2.184194in}}%
\pgfpathcurveto{\pgfqpoint{1.631185in}{2.192008in}}{\pgfqpoint{1.635575in}{2.202607in}}{\pgfqpoint{1.635575in}{2.213657in}}%
\pgfpathcurveto{\pgfqpoint{1.635575in}{2.224707in}}{\pgfqpoint{1.631185in}{2.235306in}}{\pgfqpoint{1.623371in}{2.243120in}}%
\pgfpathcurveto{\pgfqpoint{1.615557in}{2.250933in}}{\pgfqpoint{1.604958in}{2.255323in}}{\pgfqpoint{1.593908in}{2.255323in}}%
\pgfpathcurveto{\pgfqpoint{1.582858in}{2.255323in}}{\pgfqpoint{1.572259in}{2.250933in}}{\pgfqpoint{1.564446in}{2.243120in}}%
\pgfpathcurveto{\pgfqpoint{1.556632in}{2.235306in}}{\pgfqpoint{1.552242in}{2.224707in}}{\pgfqpoint{1.552242in}{2.213657in}}%
\pgfpathcurveto{\pgfqpoint{1.552242in}{2.202607in}}{\pgfqpoint{1.556632in}{2.192008in}}{\pgfqpoint{1.564446in}{2.184194in}}%
\pgfpathcurveto{\pgfqpoint{1.572259in}{2.176380in}}{\pgfqpoint{1.582858in}{2.171990in}}{\pgfqpoint{1.593908in}{2.171990in}}%
\pgfpathclose%
\pgfusepath{stroke,fill}%
\end{pgfscope}%
\begin{pgfscope}%
\pgfpathrectangle{\pgfqpoint{0.511823in}{0.504323in}}{\pgfqpoint{3.218177in}{3.225677in}} %
\pgfusepath{clip}%
\pgfsetbuttcap%
\pgfsetroundjoin%
\definecolor{currentfill}{rgb}{0.000000,0.000000,0.545098}%
\pgfsetfillcolor{currentfill}%
\pgfsetfillopacity{0.400000}%
\pgfsetlinewidth{0.501875pt}%
\definecolor{currentstroke}{rgb}{0.000000,0.000000,0.545098}%
\pgfsetstrokecolor{currentstroke}%
\pgfsetstrokeopacity{0.400000}%
\pgfsetdash{}{0pt}%
\pgfpathmoveto{\pgfqpoint{1.721459in}{2.400444in}}%
\pgfpathcurveto{\pgfqpoint{1.732509in}{2.400444in}}{\pgfqpoint{1.743108in}{2.404834in}}{\pgfqpoint{1.750922in}{2.412648in}}%
\pgfpathcurveto{\pgfqpoint{1.758735in}{2.420461in}}{\pgfqpoint{1.763126in}{2.431060in}}{\pgfqpoint{1.763126in}{2.442111in}}%
\pgfpathcurveto{\pgfqpoint{1.763126in}{2.453161in}}{\pgfqpoint{1.758735in}{2.463760in}}{\pgfqpoint{1.750922in}{2.471573in}}%
\pgfpathcurveto{\pgfqpoint{1.743108in}{2.479387in}}{\pgfqpoint{1.732509in}{2.483777in}}{\pgfqpoint{1.721459in}{2.483777in}}%
\pgfpathcurveto{\pgfqpoint{1.710409in}{2.483777in}}{\pgfqpoint{1.699810in}{2.479387in}}{\pgfqpoint{1.691996in}{2.471573in}}%
\pgfpathcurveto{\pgfqpoint{1.684183in}{2.463760in}}{\pgfqpoint{1.679792in}{2.453161in}}{\pgfqpoint{1.679792in}{2.442111in}}%
\pgfpathcurveto{\pgfqpoint{1.679792in}{2.431060in}}{\pgfqpoint{1.684183in}{2.420461in}}{\pgfqpoint{1.691996in}{2.412648in}}%
\pgfpathcurveto{\pgfqpoint{1.699810in}{2.404834in}}{\pgfqpoint{1.710409in}{2.400444in}}{\pgfqpoint{1.721459in}{2.400444in}}%
\pgfpathclose%
\pgfusepath{stroke,fill}%
\end{pgfscope}%
\begin{pgfscope}%
\pgfpathrectangle{\pgfqpoint{0.511823in}{0.504323in}}{\pgfqpoint{3.218177in}{3.225677in}} %
\pgfusepath{clip}%
\pgfsetbuttcap%
\pgfsetroundjoin%
\definecolor{currentfill}{rgb}{0.000000,0.000000,0.545098}%
\pgfsetfillcolor{currentfill}%
\pgfsetfillopacity{0.400000}%
\pgfsetlinewidth{0.501875pt}%
\definecolor{currentstroke}{rgb}{0.000000,0.000000,0.545098}%
\pgfsetstrokecolor{currentstroke}%
\pgfsetstrokeopacity{0.400000}%
\pgfsetdash{}{0pt}%
\pgfpathmoveto{\pgfqpoint{1.563781in}{2.143440in}}%
\pgfpathcurveto{\pgfqpoint{1.574831in}{2.143440in}}{\pgfqpoint{1.585430in}{2.147830in}}{\pgfqpoint{1.593244in}{2.155644in}}%
\pgfpathcurveto{\pgfqpoint{1.601057in}{2.163457in}}{\pgfqpoint{1.605448in}{2.174056in}}{\pgfqpoint{1.605448in}{2.185106in}}%
\pgfpathcurveto{\pgfqpoint{1.605448in}{2.196157in}}{\pgfqpoint{1.601057in}{2.206756in}}{\pgfqpoint{1.593244in}{2.214569in}}%
\pgfpathcurveto{\pgfqpoint{1.585430in}{2.222383in}}{\pgfqpoint{1.574831in}{2.226773in}}{\pgfqpoint{1.563781in}{2.226773in}}%
\pgfpathcurveto{\pgfqpoint{1.552731in}{2.226773in}}{\pgfqpoint{1.542132in}{2.222383in}}{\pgfqpoint{1.534318in}{2.214569in}}%
\pgfpathcurveto{\pgfqpoint{1.526504in}{2.206756in}}{\pgfqpoint{1.522114in}{2.196157in}}{\pgfqpoint{1.522114in}{2.185106in}}%
\pgfpathcurveto{\pgfqpoint{1.522114in}{2.174056in}}{\pgfqpoint{1.526504in}{2.163457in}}{\pgfqpoint{1.534318in}{2.155644in}}%
\pgfpathcurveto{\pgfqpoint{1.542132in}{2.147830in}}{\pgfqpoint{1.552731in}{2.143440in}}{\pgfqpoint{1.563781in}{2.143440in}}%
\pgfpathclose%
\pgfusepath{stroke,fill}%
\end{pgfscope}%
\begin{pgfscope}%
\pgfpathrectangle{\pgfqpoint{0.511823in}{0.504323in}}{\pgfqpoint{3.218177in}{3.225677in}} %
\pgfusepath{clip}%
\pgfsetbuttcap%
\pgfsetroundjoin%
\definecolor{currentfill}{rgb}{0.000000,0.000000,0.545098}%
\pgfsetfillcolor{currentfill}%
\pgfsetfillopacity{0.400000}%
\pgfsetlinewidth{0.501875pt}%
\definecolor{currentstroke}{rgb}{0.000000,0.000000,0.545098}%
\pgfsetstrokecolor{currentstroke}%
\pgfsetstrokeopacity{0.400000}%
\pgfsetdash{}{0pt}%
\pgfpathmoveto{\pgfqpoint{1.647811in}{2.299901in}}%
\pgfpathcurveto{\pgfqpoint{1.658861in}{2.299901in}}{\pgfqpoint{1.669460in}{2.304292in}}{\pgfqpoint{1.677274in}{2.312105in}}%
\pgfpathcurveto{\pgfqpoint{1.685087in}{2.319919in}}{\pgfqpoint{1.689478in}{2.330518in}}{\pgfqpoint{1.689478in}{2.341568in}}%
\pgfpathcurveto{\pgfqpoint{1.689478in}{2.352618in}}{\pgfqpoint{1.685087in}{2.363217in}}{\pgfqpoint{1.677274in}{2.371031in}}%
\pgfpathcurveto{\pgfqpoint{1.669460in}{2.378844in}}{\pgfqpoint{1.658861in}{2.383235in}}{\pgfqpoint{1.647811in}{2.383235in}}%
\pgfpathcurveto{\pgfqpoint{1.636761in}{2.383235in}}{\pgfqpoint{1.626162in}{2.378844in}}{\pgfqpoint{1.618348in}{2.371031in}}%
\pgfpathcurveto{\pgfqpoint{1.610535in}{2.363217in}}{\pgfqpoint{1.606144in}{2.352618in}}{\pgfqpoint{1.606144in}{2.341568in}}%
\pgfpathcurveto{\pgfqpoint{1.606144in}{2.330518in}}{\pgfqpoint{1.610535in}{2.319919in}}{\pgfqpoint{1.618348in}{2.312105in}}%
\pgfpathcurveto{\pgfqpoint{1.626162in}{2.304292in}}{\pgfqpoint{1.636761in}{2.299901in}}{\pgfqpoint{1.647811in}{2.299901in}}%
\pgfpathclose%
\pgfusepath{stroke,fill}%
\end{pgfscope}%
\begin{pgfscope}%
\pgfpathrectangle{\pgfqpoint{0.511823in}{0.504323in}}{\pgfqpoint{3.218177in}{3.225677in}} %
\pgfusepath{clip}%
\pgfsetbuttcap%
\pgfsetroundjoin%
\definecolor{currentfill}{rgb}{0.000000,0.000000,0.545098}%
\pgfsetfillcolor{currentfill}%
\pgfsetfillopacity{0.400000}%
\pgfsetlinewidth{0.501875pt}%
\definecolor{currentstroke}{rgb}{0.000000,0.000000,0.545098}%
\pgfsetstrokecolor{currentstroke}%
\pgfsetstrokeopacity{0.400000}%
\pgfsetdash{}{0pt}%
\pgfpathmoveto{\pgfqpoint{1.550114in}{2.142502in}}%
\pgfpathcurveto{\pgfqpoint{1.561164in}{2.142502in}}{\pgfqpoint{1.571763in}{2.146892in}}{\pgfqpoint{1.579576in}{2.154706in}}%
\pgfpathcurveto{\pgfqpoint{1.587390in}{2.162520in}}{\pgfqpoint{1.591780in}{2.173119in}}{\pgfqpoint{1.591780in}{2.184169in}}%
\pgfpathcurveto{\pgfqpoint{1.591780in}{2.195219in}}{\pgfqpoint{1.587390in}{2.205818in}}{\pgfqpoint{1.579576in}{2.213632in}}%
\pgfpathcurveto{\pgfqpoint{1.571763in}{2.221445in}}{\pgfqpoint{1.561164in}{2.225835in}}{\pgfqpoint{1.550114in}{2.225835in}}%
\pgfpathcurveto{\pgfqpoint{1.539063in}{2.225835in}}{\pgfqpoint{1.528464in}{2.221445in}}{\pgfqpoint{1.520651in}{2.213632in}}%
\pgfpathcurveto{\pgfqpoint{1.512837in}{2.205818in}}{\pgfqpoint{1.508447in}{2.195219in}}{\pgfqpoint{1.508447in}{2.184169in}}%
\pgfpathcurveto{\pgfqpoint{1.508447in}{2.173119in}}{\pgfqpoint{1.512837in}{2.162520in}}{\pgfqpoint{1.520651in}{2.154706in}}%
\pgfpathcurveto{\pgfqpoint{1.528464in}{2.146892in}}{\pgfqpoint{1.539063in}{2.142502in}}{\pgfqpoint{1.550114in}{2.142502in}}%
\pgfpathclose%
\pgfusepath{stroke,fill}%
\end{pgfscope}%
\begin{pgfscope}%
\pgfpathrectangle{\pgfqpoint{0.511823in}{0.504323in}}{\pgfqpoint{3.218177in}{3.225677in}} %
\pgfusepath{clip}%
\pgfsetbuttcap%
\pgfsetroundjoin%
\definecolor{currentfill}{rgb}{0.000000,0.000000,0.545098}%
\pgfsetfillcolor{currentfill}%
\pgfsetfillopacity{0.400000}%
\pgfsetlinewidth{0.501875pt}%
\definecolor{currentstroke}{rgb}{0.000000,0.000000,0.545098}%
\pgfsetstrokecolor{currentstroke}%
\pgfsetstrokeopacity{0.400000}%
\pgfsetdash{}{0pt}%
\pgfpathmoveto{\pgfqpoint{1.604356in}{2.249043in}}%
\pgfpathcurveto{\pgfqpoint{1.615407in}{2.249043in}}{\pgfqpoint{1.626006in}{2.253433in}}{\pgfqpoint{1.633819in}{2.261247in}}%
\pgfpathcurveto{\pgfqpoint{1.641633in}{2.269060in}}{\pgfqpoint{1.646023in}{2.279659in}}{\pgfqpoint{1.646023in}{2.290710in}}%
\pgfpathcurveto{\pgfqpoint{1.646023in}{2.301760in}}{\pgfqpoint{1.641633in}{2.312359in}}{\pgfqpoint{1.633819in}{2.320172in}}%
\pgfpathcurveto{\pgfqpoint{1.626006in}{2.327986in}}{\pgfqpoint{1.615407in}{2.332376in}}{\pgfqpoint{1.604356in}{2.332376in}}%
\pgfpathcurveto{\pgfqpoint{1.593306in}{2.332376in}}{\pgfqpoint{1.582707in}{2.327986in}}{\pgfqpoint{1.574894in}{2.320172in}}%
\pgfpathcurveto{\pgfqpoint{1.567080in}{2.312359in}}{\pgfqpoint{1.562690in}{2.301760in}}{\pgfqpoint{1.562690in}{2.290710in}}%
\pgfpathcurveto{\pgfqpoint{1.562690in}{2.279659in}}{\pgfqpoint{1.567080in}{2.269060in}}{\pgfqpoint{1.574894in}{2.261247in}}%
\pgfpathcurveto{\pgfqpoint{1.582707in}{2.253433in}}{\pgfqpoint{1.593306in}{2.249043in}}{\pgfqpoint{1.604356in}{2.249043in}}%
\pgfpathclose%
\pgfusepath{stroke,fill}%
\end{pgfscope}%
\begin{pgfscope}%
\pgfpathrectangle{\pgfqpoint{0.511823in}{0.504323in}}{\pgfqpoint{3.218177in}{3.225677in}} %
\pgfusepath{clip}%
\pgfsetbuttcap%
\pgfsetroundjoin%
\definecolor{currentfill}{rgb}{0.000000,0.000000,0.545098}%
\pgfsetfillcolor{currentfill}%
\pgfsetfillopacity{0.400000}%
\pgfsetlinewidth{0.501875pt}%
\definecolor{currentstroke}{rgb}{0.000000,0.000000,0.545098}%
\pgfsetstrokecolor{currentstroke}%
\pgfsetstrokeopacity{0.400000}%
\pgfsetdash{}{0pt}%
\pgfpathmoveto{\pgfqpoint{1.611471in}{2.273919in}}%
\pgfpathcurveto{\pgfqpoint{1.622522in}{2.273919in}}{\pgfqpoint{1.633121in}{2.278309in}}{\pgfqpoint{1.640934in}{2.286123in}}%
\pgfpathcurveto{\pgfqpoint{1.648748in}{2.293937in}}{\pgfqpoint{1.653138in}{2.304536in}}{\pgfqpoint{1.653138in}{2.315586in}}%
\pgfpathcurveto{\pgfqpoint{1.653138in}{2.326636in}}{\pgfqpoint{1.648748in}{2.337235in}}{\pgfqpoint{1.640934in}{2.345049in}}%
\pgfpathcurveto{\pgfqpoint{1.633121in}{2.352862in}}{\pgfqpoint{1.622522in}{2.357252in}}{\pgfqpoint{1.611471in}{2.357252in}}%
\pgfpathcurveto{\pgfqpoint{1.600421in}{2.357252in}}{\pgfqpoint{1.589822in}{2.352862in}}{\pgfqpoint{1.582009in}{2.345049in}}%
\pgfpathcurveto{\pgfqpoint{1.574195in}{2.337235in}}{\pgfqpoint{1.569805in}{2.326636in}}{\pgfqpoint{1.569805in}{2.315586in}}%
\pgfpathcurveto{\pgfqpoint{1.569805in}{2.304536in}}{\pgfqpoint{1.574195in}{2.293937in}}{\pgfqpoint{1.582009in}{2.286123in}}%
\pgfpathcurveto{\pgfqpoint{1.589822in}{2.278309in}}{\pgfqpoint{1.600421in}{2.273919in}}{\pgfqpoint{1.611471in}{2.273919in}}%
\pgfpathclose%
\pgfusepath{stroke,fill}%
\end{pgfscope}%
\begin{pgfscope}%
\pgfpathrectangle{\pgfqpoint{0.511823in}{0.504323in}}{\pgfqpoint{3.218177in}{3.225677in}} %
\pgfusepath{clip}%
\pgfsetbuttcap%
\pgfsetroundjoin%
\definecolor{currentfill}{rgb}{0.000000,0.000000,0.545098}%
\pgfsetfillcolor{currentfill}%
\pgfsetfillopacity{0.400000}%
\pgfsetlinewidth{0.501875pt}%
\definecolor{currentstroke}{rgb}{0.000000,0.000000,0.545098}%
\pgfsetstrokecolor{currentstroke}%
\pgfsetstrokeopacity{0.400000}%
\pgfsetdash{}{0pt}%
\pgfpathmoveto{\pgfqpoint{1.508077in}{2.102412in}}%
\pgfpathcurveto{\pgfqpoint{1.519127in}{2.102412in}}{\pgfqpoint{1.529726in}{2.106802in}}{\pgfqpoint{1.537540in}{2.114616in}}%
\pgfpathcurveto{\pgfqpoint{1.545354in}{2.122429in}}{\pgfqpoint{1.549744in}{2.133028in}}{\pgfqpoint{1.549744in}{2.144078in}}%
\pgfpathcurveto{\pgfqpoint{1.549744in}{2.155128in}}{\pgfqpoint{1.545354in}{2.165727in}}{\pgfqpoint{1.537540in}{2.173541in}}%
\pgfpathcurveto{\pgfqpoint{1.529726in}{2.181355in}}{\pgfqpoint{1.519127in}{2.185745in}}{\pgfqpoint{1.508077in}{2.185745in}}%
\pgfpathcurveto{\pgfqpoint{1.497027in}{2.185745in}}{\pgfqpoint{1.486428in}{2.181355in}}{\pgfqpoint{1.478614in}{2.173541in}}%
\pgfpathcurveto{\pgfqpoint{1.470801in}{2.165727in}}{\pgfqpoint{1.466411in}{2.155128in}}{\pgfqpoint{1.466411in}{2.144078in}}%
\pgfpathcurveto{\pgfqpoint{1.466411in}{2.133028in}}{\pgfqpoint{1.470801in}{2.122429in}}{\pgfqpoint{1.478614in}{2.114616in}}%
\pgfpathcurveto{\pgfqpoint{1.486428in}{2.106802in}}{\pgfqpoint{1.497027in}{2.102412in}}{\pgfqpoint{1.508077in}{2.102412in}}%
\pgfpathclose%
\pgfusepath{stroke,fill}%
\end{pgfscope}%
\begin{pgfscope}%
\pgfpathrectangle{\pgfqpoint{0.511823in}{0.504323in}}{\pgfqpoint{3.218177in}{3.225677in}} %
\pgfusepath{clip}%
\pgfsetbuttcap%
\pgfsetroundjoin%
\definecolor{currentfill}{rgb}{0.000000,0.000000,0.545098}%
\pgfsetfillcolor{currentfill}%
\pgfsetfillopacity{0.400000}%
\pgfsetlinewidth{0.501875pt}%
\definecolor{currentstroke}{rgb}{0.000000,0.000000,0.545098}%
\pgfsetstrokecolor{currentstroke}%
\pgfsetstrokeopacity{0.400000}%
\pgfsetdash{}{0pt}%
\pgfpathmoveto{\pgfqpoint{1.671988in}{2.407735in}}%
\pgfpathcurveto{\pgfqpoint{1.683039in}{2.407735in}}{\pgfqpoint{1.693638in}{2.412125in}}{\pgfqpoint{1.701451in}{2.419939in}}%
\pgfpathcurveto{\pgfqpoint{1.709265in}{2.427752in}}{\pgfqpoint{1.713655in}{2.438351in}}{\pgfqpoint{1.713655in}{2.449402in}}%
\pgfpathcurveto{\pgfqpoint{1.713655in}{2.460452in}}{\pgfqpoint{1.709265in}{2.471051in}}{\pgfqpoint{1.701451in}{2.478864in}}%
\pgfpathcurveto{\pgfqpoint{1.693638in}{2.486678in}}{\pgfqpoint{1.683039in}{2.491068in}}{\pgfqpoint{1.671988in}{2.491068in}}%
\pgfpathcurveto{\pgfqpoint{1.660938in}{2.491068in}}{\pgfqpoint{1.650339in}{2.486678in}}{\pgfqpoint{1.642526in}{2.478864in}}%
\pgfpathcurveto{\pgfqpoint{1.634712in}{2.471051in}}{\pgfqpoint{1.630322in}{2.460452in}}{\pgfqpoint{1.630322in}{2.449402in}}%
\pgfpathcurveto{\pgfqpoint{1.630322in}{2.438351in}}{\pgfqpoint{1.634712in}{2.427752in}}{\pgfqpoint{1.642526in}{2.419939in}}%
\pgfpathcurveto{\pgfqpoint{1.650339in}{2.412125in}}{\pgfqpoint{1.660938in}{2.407735in}}{\pgfqpoint{1.671988in}{2.407735in}}%
\pgfpathclose%
\pgfusepath{stroke,fill}%
\end{pgfscope}%
\begin{pgfscope}%
\pgfpathrectangle{\pgfqpoint{0.511823in}{0.504323in}}{\pgfqpoint{3.218177in}{3.225677in}} %
\pgfusepath{clip}%
\pgfsetbuttcap%
\pgfsetroundjoin%
\definecolor{currentfill}{rgb}{0.000000,0.000000,0.545098}%
\pgfsetfillcolor{currentfill}%
\pgfsetfillopacity{0.400000}%
\pgfsetlinewidth{0.501875pt}%
\definecolor{currentstroke}{rgb}{0.000000,0.000000,0.545098}%
\pgfsetstrokecolor{currentstroke}%
\pgfsetstrokeopacity{0.400000}%
\pgfsetdash{}{0pt}%
\pgfpathmoveto{\pgfqpoint{1.585945in}{2.265885in}}%
\pgfpathcurveto{\pgfqpoint{1.596995in}{2.265885in}}{\pgfqpoint{1.607594in}{2.270276in}}{\pgfqpoint{1.615408in}{2.278089in}}%
\pgfpathcurveto{\pgfqpoint{1.623221in}{2.285903in}}{\pgfqpoint{1.627612in}{2.296502in}}{\pgfqpoint{1.627612in}{2.307552in}}%
\pgfpathcurveto{\pgfqpoint{1.627612in}{2.318602in}}{\pgfqpoint{1.623221in}{2.329201in}}{\pgfqpoint{1.615408in}{2.337015in}}%
\pgfpathcurveto{\pgfqpoint{1.607594in}{2.344829in}}{\pgfqpoint{1.596995in}{2.349219in}}{\pgfqpoint{1.585945in}{2.349219in}}%
\pgfpathcurveto{\pgfqpoint{1.574895in}{2.349219in}}{\pgfqpoint{1.564296in}{2.344829in}}{\pgfqpoint{1.556482in}{2.337015in}}%
\pgfpathcurveto{\pgfqpoint{1.548669in}{2.329201in}}{\pgfqpoint{1.544278in}{2.318602in}}{\pgfqpoint{1.544278in}{2.307552in}}%
\pgfpathcurveto{\pgfqpoint{1.544278in}{2.296502in}}{\pgfqpoint{1.548669in}{2.285903in}}{\pgfqpoint{1.556482in}{2.278089in}}%
\pgfpathcurveto{\pgfqpoint{1.564296in}{2.270276in}}{\pgfqpoint{1.574895in}{2.265885in}}{\pgfqpoint{1.585945in}{2.265885in}}%
\pgfpathclose%
\pgfusepath{stroke,fill}%
\end{pgfscope}%
\begin{pgfscope}%
\pgfpathrectangle{\pgfqpoint{0.511823in}{0.504323in}}{\pgfqpoint{3.218177in}{3.225677in}} %
\pgfusepath{clip}%
\pgfsetbuttcap%
\pgfsetroundjoin%
\definecolor{currentfill}{rgb}{0.000000,0.000000,0.545098}%
\pgfsetfillcolor{currentfill}%
\pgfsetfillopacity{0.400000}%
\pgfsetlinewidth{0.501875pt}%
\definecolor{currentstroke}{rgb}{0.000000,0.000000,0.545098}%
\pgfsetstrokecolor{currentstroke}%
\pgfsetstrokeopacity{0.400000}%
\pgfsetdash{}{0pt}%
\pgfpathmoveto{\pgfqpoint{1.664826in}{2.422254in}}%
\pgfpathcurveto{\pgfqpoint{1.675876in}{2.422254in}}{\pgfqpoint{1.686475in}{2.426644in}}{\pgfqpoint{1.694289in}{2.434458in}}%
\pgfpathcurveto{\pgfqpoint{1.702102in}{2.442271in}}{\pgfqpoint{1.706493in}{2.452870in}}{\pgfqpoint{1.706493in}{2.463920in}}%
\pgfpathcurveto{\pgfqpoint{1.706493in}{2.474971in}}{\pgfqpoint{1.702102in}{2.485570in}}{\pgfqpoint{1.694289in}{2.493383in}}%
\pgfpathcurveto{\pgfqpoint{1.686475in}{2.501197in}}{\pgfqpoint{1.675876in}{2.505587in}}{\pgfqpoint{1.664826in}{2.505587in}}%
\pgfpathcurveto{\pgfqpoint{1.653776in}{2.505587in}}{\pgfqpoint{1.643177in}{2.501197in}}{\pgfqpoint{1.635363in}{2.493383in}}%
\pgfpathcurveto{\pgfqpoint{1.627549in}{2.485570in}}{\pgfqpoint{1.623159in}{2.474971in}}{\pgfqpoint{1.623159in}{2.463920in}}%
\pgfpathcurveto{\pgfqpoint{1.623159in}{2.452870in}}{\pgfqpoint{1.627549in}{2.442271in}}{\pgfqpoint{1.635363in}{2.434458in}}%
\pgfpathcurveto{\pgfqpoint{1.643177in}{2.426644in}}{\pgfqpoint{1.653776in}{2.422254in}}{\pgfqpoint{1.664826in}{2.422254in}}%
\pgfpathclose%
\pgfusepath{stroke,fill}%
\end{pgfscope}%
\begin{pgfscope}%
\pgfpathrectangle{\pgfqpoint{0.511823in}{0.504323in}}{\pgfqpoint{3.218177in}{3.225677in}} %
\pgfusepath{clip}%
\pgfsetbuttcap%
\pgfsetroundjoin%
\definecolor{currentfill}{rgb}{0.000000,0.000000,0.545098}%
\pgfsetfillcolor{currentfill}%
\pgfsetfillopacity{0.400000}%
\pgfsetlinewidth{0.501875pt}%
\definecolor{currentstroke}{rgb}{0.000000,0.000000,0.545098}%
\pgfsetstrokecolor{currentstroke}%
\pgfsetstrokeopacity{0.400000}%
\pgfsetdash{}{0pt}%
\pgfpathmoveto{\pgfqpoint{1.529920in}{2.188490in}}%
\pgfpathcurveto{\pgfqpoint{1.540970in}{2.188490in}}{\pgfqpoint{1.551569in}{2.192880in}}{\pgfqpoint{1.559383in}{2.200694in}}%
\pgfpathcurveto{\pgfqpoint{1.567197in}{2.208507in}}{\pgfqpoint{1.571587in}{2.219106in}}{\pgfqpoint{1.571587in}{2.230156in}}%
\pgfpathcurveto{\pgfqpoint{1.571587in}{2.241207in}}{\pgfqpoint{1.567197in}{2.251806in}}{\pgfqpoint{1.559383in}{2.259619in}}%
\pgfpathcurveto{\pgfqpoint{1.551569in}{2.267433in}}{\pgfqpoint{1.540970in}{2.271823in}}{\pgfqpoint{1.529920in}{2.271823in}}%
\pgfpathcurveto{\pgfqpoint{1.518870in}{2.271823in}}{\pgfqpoint{1.508271in}{2.267433in}}{\pgfqpoint{1.500457in}{2.259619in}}%
\pgfpathcurveto{\pgfqpoint{1.492644in}{2.251806in}}{\pgfqpoint{1.488254in}{2.241207in}}{\pgfqpoint{1.488254in}{2.230156in}}%
\pgfpathcurveto{\pgfqpoint{1.488254in}{2.219106in}}{\pgfqpoint{1.492644in}{2.208507in}}{\pgfqpoint{1.500457in}{2.200694in}}%
\pgfpathcurveto{\pgfqpoint{1.508271in}{2.192880in}}{\pgfqpoint{1.518870in}{2.188490in}}{\pgfqpoint{1.529920in}{2.188490in}}%
\pgfpathclose%
\pgfusepath{stroke,fill}%
\end{pgfscope}%
\begin{pgfscope}%
\pgfpathrectangle{\pgfqpoint{0.511823in}{0.504323in}}{\pgfqpoint{3.218177in}{3.225677in}} %
\pgfusepath{clip}%
\pgfsetbuttcap%
\pgfsetroundjoin%
\definecolor{currentfill}{rgb}{0.000000,0.000000,0.545098}%
\pgfsetfillcolor{currentfill}%
\pgfsetfillopacity{0.400000}%
\pgfsetlinewidth{0.501875pt}%
\definecolor{currentstroke}{rgb}{0.000000,0.000000,0.545098}%
\pgfsetstrokecolor{currentstroke}%
\pgfsetstrokeopacity{0.400000}%
\pgfsetdash{}{0pt}%
\pgfpathmoveto{\pgfqpoint{1.634696in}{2.394548in}}%
\pgfpathcurveto{\pgfqpoint{1.645746in}{2.394548in}}{\pgfqpoint{1.656345in}{2.398938in}}{\pgfqpoint{1.664159in}{2.406752in}}%
\pgfpathcurveto{\pgfqpoint{1.671972in}{2.414566in}}{\pgfqpoint{1.676363in}{2.425165in}}{\pgfqpoint{1.676363in}{2.436215in}}%
\pgfpathcurveto{\pgfqpoint{1.676363in}{2.447265in}}{\pgfqpoint{1.671972in}{2.457864in}}{\pgfqpoint{1.664159in}{2.465678in}}%
\pgfpathcurveto{\pgfqpoint{1.656345in}{2.473491in}}{\pgfqpoint{1.645746in}{2.477881in}}{\pgfqpoint{1.634696in}{2.477881in}}%
\pgfpathcurveto{\pgfqpoint{1.623646in}{2.477881in}}{\pgfqpoint{1.613047in}{2.473491in}}{\pgfqpoint{1.605233in}{2.465678in}}%
\pgfpathcurveto{\pgfqpoint{1.597420in}{2.457864in}}{\pgfqpoint{1.593029in}{2.447265in}}{\pgfqpoint{1.593029in}{2.436215in}}%
\pgfpathcurveto{\pgfqpoint{1.593029in}{2.425165in}}{\pgfqpoint{1.597420in}{2.414566in}}{\pgfqpoint{1.605233in}{2.406752in}}%
\pgfpathcurveto{\pgfqpoint{1.613047in}{2.398938in}}{\pgfqpoint{1.623646in}{2.394548in}}{\pgfqpoint{1.634696in}{2.394548in}}%
\pgfpathclose%
\pgfusepath{stroke,fill}%
\end{pgfscope}%
\begin{pgfscope}%
\pgfpathrectangle{\pgfqpoint{0.511823in}{0.504323in}}{\pgfqpoint{3.218177in}{3.225677in}} %
\pgfusepath{clip}%
\pgfsetbuttcap%
\pgfsetroundjoin%
\definecolor{currentfill}{rgb}{0.000000,0.000000,0.545098}%
\pgfsetfillcolor{currentfill}%
\pgfsetfillopacity{0.400000}%
\pgfsetlinewidth{0.501875pt}%
\definecolor{currentstroke}{rgb}{0.000000,0.000000,0.545098}%
\pgfsetstrokecolor{currentstroke}%
\pgfsetstrokeopacity{0.400000}%
\pgfsetdash{}{0pt}%
\pgfpathmoveto{\pgfqpoint{1.712989in}{2.554427in}}%
\pgfpathcurveto{\pgfqpoint{1.724039in}{2.554427in}}{\pgfqpoint{1.734638in}{2.558817in}}{\pgfqpoint{1.742452in}{2.566630in}}%
\pgfpathcurveto{\pgfqpoint{1.750265in}{2.574444in}}{\pgfqpoint{1.754656in}{2.585043in}}{\pgfqpoint{1.754656in}{2.596093in}}%
\pgfpathcurveto{\pgfqpoint{1.754656in}{2.607143in}}{\pgfqpoint{1.750265in}{2.617742in}}{\pgfqpoint{1.742452in}{2.625556in}}%
\pgfpathcurveto{\pgfqpoint{1.734638in}{2.633370in}}{\pgfqpoint{1.724039in}{2.637760in}}{\pgfqpoint{1.712989in}{2.637760in}}%
\pgfpathcurveto{\pgfqpoint{1.701939in}{2.637760in}}{\pgfqpoint{1.691340in}{2.633370in}}{\pgfqpoint{1.683526in}{2.625556in}}%
\pgfpathcurveto{\pgfqpoint{1.675713in}{2.617742in}}{\pgfqpoint{1.671322in}{2.607143in}}{\pgfqpoint{1.671322in}{2.596093in}}%
\pgfpathcurveto{\pgfqpoint{1.671322in}{2.585043in}}{\pgfqpoint{1.675713in}{2.574444in}}{\pgfqpoint{1.683526in}{2.566630in}}%
\pgfpathcurveto{\pgfqpoint{1.691340in}{2.558817in}}{\pgfqpoint{1.701939in}{2.554427in}}{\pgfqpoint{1.712989in}{2.554427in}}%
\pgfpathclose%
\pgfusepath{stroke,fill}%
\end{pgfscope}%
\begin{pgfscope}%
\pgfpathrectangle{\pgfqpoint{0.511823in}{0.504323in}}{\pgfqpoint{3.218177in}{3.225677in}} %
\pgfusepath{clip}%
\pgfsetbuttcap%
\pgfsetroundjoin%
\definecolor{currentfill}{rgb}{0.000000,0.000000,0.545098}%
\pgfsetfillcolor{currentfill}%
\pgfsetfillopacity{0.400000}%
\pgfsetlinewidth{0.501875pt}%
\definecolor{currentstroke}{rgb}{0.000000,0.000000,0.545098}%
\pgfsetstrokecolor{currentstroke}%
\pgfsetstrokeopacity{0.400000}%
\pgfsetdash{}{0pt}%
\pgfpathmoveto{\pgfqpoint{1.581886in}{2.323145in}}%
\pgfpathcurveto{\pgfqpoint{1.592936in}{2.323145in}}{\pgfqpoint{1.603535in}{2.327535in}}{\pgfqpoint{1.611348in}{2.335349in}}%
\pgfpathcurveto{\pgfqpoint{1.619162in}{2.343162in}}{\pgfqpoint{1.623552in}{2.353761in}}{\pgfqpoint{1.623552in}{2.364811in}}%
\pgfpathcurveto{\pgfqpoint{1.623552in}{2.375862in}}{\pgfqpoint{1.619162in}{2.386461in}}{\pgfqpoint{1.611348in}{2.394274in}}%
\pgfpathcurveto{\pgfqpoint{1.603535in}{2.402088in}}{\pgfqpoint{1.592936in}{2.406478in}}{\pgfqpoint{1.581886in}{2.406478in}}%
\pgfpathcurveto{\pgfqpoint{1.570836in}{2.406478in}}{\pgfqpoint{1.560237in}{2.402088in}}{\pgfqpoint{1.552423in}{2.394274in}}%
\pgfpathcurveto{\pgfqpoint{1.544609in}{2.386461in}}{\pgfqpoint{1.540219in}{2.375862in}}{\pgfqpoint{1.540219in}{2.364811in}}%
\pgfpathcurveto{\pgfqpoint{1.540219in}{2.353761in}}{\pgfqpoint{1.544609in}{2.343162in}}{\pgfqpoint{1.552423in}{2.335349in}}%
\pgfpathcurveto{\pgfqpoint{1.560237in}{2.327535in}}{\pgfqpoint{1.570836in}{2.323145in}}{\pgfqpoint{1.581886in}{2.323145in}}%
\pgfpathclose%
\pgfusepath{stroke,fill}%
\end{pgfscope}%
\begin{pgfscope}%
\pgfpathrectangle{\pgfqpoint{0.511823in}{0.504323in}}{\pgfqpoint{3.218177in}{3.225677in}} %
\pgfusepath{clip}%
\pgfsetbuttcap%
\pgfsetroundjoin%
\definecolor{currentfill}{rgb}{0.000000,0.000000,0.545098}%
\pgfsetfillcolor{currentfill}%
\pgfsetfillopacity{0.400000}%
\pgfsetlinewidth{0.501875pt}%
\definecolor{currentstroke}{rgb}{0.000000,0.000000,0.545098}%
\pgfsetstrokecolor{currentstroke}%
\pgfsetstrokeopacity{0.400000}%
\pgfsetdash{}{0pt}%
\pgfpathmoveto{\pgfqpoint{1.525689in}{2.230072in}}%
\pgfpathcurveto{\pgfqpoint{1.536739in}{2.230072in}}{\pgfqpoint{1.547338in}{2.234462in}}{\pgfqpoint{1.555152in}{2.242276in}}%
\pgfpathcurveto{\pgfqpoint{1.562966in}{2.250089in}}{\pgfqpoint{1.567356in}{2.260688in}}{\pgfqpoint{1.567356in}{2.271739in}}%
\pgfpathcurveto{\pgfqpoint{1.567356in}{2.282789in}}{\pgfqpoint{1.562966in}{2.293388in}}{\pgfqpoint{1.555152in}{2.301201in}}%
\pgfpathcurveto{\pgfqpoint{1.547338in}{2.309015in}}{\pgfqpoint{1.536739in}{2.313405in}}{\pgfqpoint{1.525689in}{2.313405in}}%
\pgfpathcurveto{\pgfqpoint{1.514639in}{2.313405in}}{\pgfqpoint{1.504040in}{2.309015in}}{\pgfqpoint{1.496227in}{2.301201in}}%
\pgfpathcurveto{\pgfqpoint{1.488413in}{2.293388in}}{\pgfqpoint{1.484023in}{2.282789in}}{\pgfqpoint{1.484023in}{2.271739in}}%
\pgfpathcurveto{\pgfqpoint{1.484023in}{2.260688in}}{\pgfqpoint{1.488413in}{2.250089in}}{\pgfqpoint{1.496227in}{2.242276in}}%
\pgfpathcurveto{\pgfqpoint{1.504040in}{2.234462in}}{\pgfqpoint{1.514639in}{2.230072in}}{\pgfqpoint{1.525689in}{2.230072in}}%
\pgfpathclose%
\pgfusepath{stroke,fill}%
\end{pgfscope}%
\begin{pgfscope}%
\pgfpathrectangle{\pgfqpoint{0.511823in}{0.504323in}}{\pgfqpoint{3.218177in}{3.225677in}} %
\pgfusepath{clip}%
\pgfsetbuttcap%
\pgfsetroundjoin%
\definecolor{currentfill}{rgb}{0.000000,0.000000,0.545098}%
\pgfsetfillcolor{currentfill}%
\pgfsetfillopacity{0.400000}%
\pgfsetlinewidth{0.501875pt}%
\definecolor{currentstroke}{rgb}{0.000000,0.000000,0.545098}%
\pgfsetstrokecolor{currentstroke}%
\pgfsetstrokeopacity{0.400000}%
\pgfsetdash{}{0pt}%
\pgfpathmoveto{\pgfqpoint{1.585709in}{2.357451in}}%
\pgfpathcurveto{\pgfqpoint{1.596759in}{2.357451in}}{\pgfqpoint{1.607358in}{2.361841in}}{\pgfqpoint{1.615172in}{2.369654in}}%
\pgfpathcurveto{\pgfqpoint{1.622986in}{2.377468in}}{\pgfqpoint{1.627376in}{2.388067in}}{\pgfqpoint{1.627376in}{2.399117in}}%
\pgfpathcurveto{\pgfqpoint{1.627376in}{2.410167in}}{\pgfqpoint{1.622986in}{2.420766in}}{\pgfqpoint{1.615172in}{2.428580in}}%
\pgfpathcurveto{\pgfqpoint{1.607358in}{2.436394in}}{\pgfqpoint{1.596759in}{2.440784in}}{\pgfqpoint{1.585709in}{2.440784in}}%
\pgfpathcurveto{\pgfqpoint{1.574659in}{2.440784in}}{\pgfqpoint{1.564060in}{2.436394in}}{\pgfqpoint{1.556247in}{2.428580in}}%
\pgfpathcurveto{\pgfqpoint{1.548433in}{2.420766in}}{\pgfqpoint{1.544043in}{2.410167in}}{\pgfqpoint{1.544043in}{2.399117in}}%
\pgfpathcurveto{\pgfqpoint{1.544043in}{2.388067in}}{\pgfqpoint{1.548433in}{2.377468in}}{\pgfqpoint{1.556247in}{2.369654in}}%
\pgfpathcurveto{\pgfqpoint{1.564060in}{2.361841in}}{\pgfqpoint{1.574659in}{2.357451in}}{\pgfqpoint{1.585709in}{2.357451in}}%
\pgfpathclose%
\pgfusepath{stroke,fill}%
\end{pgfscope}%
\begin{pgfscope}%
\pgfpathrectangle{\pgfqpoint{0.511823in}{0.504323in}}{\pgfqpoint{3.218177in}{3.225677in}} %
\pgfusepath{clip}%
\pgfsetbuttcap%
\pgfsetroundjoin%
\definecolor{currentfill}{rgb}{0.000000,0.000000,0.545098}%
\pgfsetfillcolor{currentfill}%
\pgfsetfillopacity{0.400000}%
\pgfsetlinewidth{0.501875pt}%
\definecolor{currentstroke}{rgb}{0.000000,0.000000,0.545098}%
\pgfsetstrokecolor{currentstroke}%
\pgfsetstrokeopacity{0.400000}%
\pgfsetdash{}{0pt}%
\pgfpathmoveto{\pgfqpoint{1.560391in}{2.322524in}}%
\pgfpathcurveto{\pgfqpoint{1.571441in}{2.322524in}}{\pgfqpoint{1.582040in}{2.326914in}}{\pgfqpoint{1.589854in}{2.334728in}}%
\pgfpathcurveto{\pgfqpoint{1.597667in}{2.342541in}}{\pgfqpoint{1.602057in}{2.353140in}}{\pgfqpoint{1.602057in}{2.364191in}}%
\pgfpathcurveto{\pgfqpoint{1.602057in}{2.375241in}}{\pgfqpoint{1.597667in}{2.385840in}}{\pgfqpoint{1.589854in}{2.393653in}}%
\pgfpathcurveto{\pgfqpoint{1.582040in}{2.401467in}}{\pgfqpoint{1.571441in}{2.405857in}}{\pgfqpoint{1.560391in}{2.405857in}}%
\pgfpathcurveto{\pgfqpoint{1.549341in}{2.405857in}}{\pgfqpoint{1.538742in}{2.401467in}}{\pgfqpoint{1.530928in}{2.393653in}}%
\pgfpathcurveto{\pgfqpoint{1.523114in}{2.385840in}}{\pgfqpoint{1.518724in}{2.375241in}}{\pgfqpoint{1.518724in}{2.364191in}}%
\pgfpathcurveto{\pgfqpoint{1.518724in}{2.353140in}}{\pgfqpoint{1.523114in}{2.342541in}}{\pgfqpoint{1.530928in}{2.334728in}}%
\pgfpathcurveto{\pgfqpoint{1.538742in}{2.326914in}}{\pgfqpoint{1.549341in}{2.322524in}}{\pgfqpoint{1.560391in}{2.322524in}}%
\pgfpathclose%
\pgfusepath{stroke,fill}%
\end{pgfscope}%
\begin{pgfscope}%
\pgfpathrectangle{\pgfqpoint{0.511823in}{0.504323in}}{\pgfqpoint{3.218177in}{3.225677in}} %
\pgfusepath{clip}%
\pgfsetbuttcap%
\pgfsetroundjoin%
\definecolor{currentfill}{rgb}{0.000000,0.000000,0.545098}%
\pgfsetfillcolor{currentfill}%
\pgfsetfillopacity{0.400000}%
\pgfsetlinewidth{0.501875pt}%
\definecolor{currentstroke}{rgb}{0.000000,0.000000,0.545098}%
\pgfsetstrokecolor{currentstroke}%
\pgfsetstrokeopacity{0.400000}%
\pgfsetdash{}{0pt}%
\pgfpathmoveto{\pgfqpoint{1.534575in}{2.286025in}}%
\pgfpathcurveto{\pgfqpoint{1.545625in}{2.286025in}}{\pgfqpoint{1.556224in}{2.290415in}}{\pgfqpoint{1.564038in}{2.298229in}}%
\pgfpathcurveto{\pgfqpoint{1.571852in}{2.306043in}}{\pgfqpoint{1.576242in}{2.316642in}}{\pgfqpoint{1.576242in}{2.327692in}}%
\pgfpathcurveto{\pgfqpoint{1.576242in}{2.338742in}}{\pgfqpoint{1.571852in}{2.349341in}}{\pgfqpoint{1.564038in}{2.357155in}}%
\pgfpathcurveto{\pgfqpoint{1.556224in}{2.364968in}}{\pgfqpoint{1.545625in}{2.369359in}}{\pgfqpoint{1.534575in}{2.369359in}}%
\pgfpathcurveto{\pgfqpoint{1.523525in}{2.369359in}}{\pgfqpoint{1.512926in}{2.364968in}}{\pgfqpoint{1.505112in}{2.357155in}}%
\pgfpathcurveto{\pgfqpoint{1.497299in}{2.349341in}}{\pgfqpoint{1.492909in}{2.338742in}}{\pgfqpoint{1.492909in}{2.327692in}}%
\pgfpathcurveto{\pgfqpoint{1.492909in}{2.316642in}}{\pgfqpoint{1.497299in}{2.306043in}}{\pgfqpoint{1.505112in}{2.298229in}}%
\pgfpathcurveto{\pgfqpoint{1.512926in}{2.290415in}}{\pgfqpoint{1.523525in}{2.286025in}}{\pgfqpoint{1.534575in}{2.286025in}}%
\pgfpathclose%
\pgfusepath{stroke,fill}%
\end{pgfscope}%
\begin{pgfscope}%
\pgfpathrectangle{\pgfqpoint{0.511823in}{0.504323in}}{\pgfqpoint{3.218177in}{3.225677in}} %
\pgfusepath{clip}%
\pgfsetbuttcap%
\pgfsetroundjoin%
\definecolor{currentfill}{rgb}{0.000000,0.000000,0.545098}%
\pgfsetfillcolor{currentfill}%
\pgfsetfillopacity{0.400000}%
\pgfsetlinewidth{0.501875pt}%
\definecolor{currentstroke}{rgb}{0.000000,0.000000,0.545098}%
\pgfsetstrokecolor{currentstroke}%
\pgfsetstrokeopacity{0.400000}%
\pgfsetdash{}{0pt}%
\pgfpathmoveto{\pgfqpoint{1.534100in}{2.298477in}}%
\pgfpathcurveto{\pgfqpoint{1.545150in}{2.298477in}}{\pgfqpoint{1.555749in}{2.302868in}}{\pgfqpoint{1.563562in}{2.310681in}}%
\pgfpathcurveto{\pgfqpoint{1.571376in}{2.318495in}}{\pgfqpoint{1.575766in}{2.329094in}}{\pgfqpoint{1.575766in}{2.340144in}}%
\pgfpathcurveto{\pgfqpoint{1.575766in}{2.351194in}}{\pgfqpoint{1.571376in}{2.361793in}}{\pgfqpoint{1.563562in}{2.369607in}}%
\pgfpathcurveto{\pgfqpoint{1.555749in}{2.377421in}}{\pgfqpoint{1.545150in}{2.381811in}}{\pgfqpoint{1.534100in}{2.381811in}}%
\pgfpathcurveto{\pgfqpoint{1.523049in}{2.381811in}}{\pgfqpoint{1.512450in}{2.377421in}}{\pgfqpoint{1.504637in}{2.369607in}}%
\pgfpathcurveto{\pgfqpoint{1.496823in}{2.361793in}}{\pgfqpoint{1.492433in}{2.351194in}}{\pgfqpoint{1.492433in}{2.340144in}}%
\pgfpathcurveto{\pgfqpoint{1.492433in}{2.329094in}}{\pgfqpoint{1.496823in}{2.318495in}}{\pgfqpoint{1.504637in}{2.310681in}}%
\pgfpathcurveto{\pgfqpoint{1.512450in}{2.302868in}}{\pgfqpoint{1.523049in}{2.298477in}}{\pgfqpoint{1.534100in}{2.298477in}}%
\pgfpathclose%
\pgfusepath{stroke,fill}%
\end{pgfscope}%
\begin{pgfscope}%
\pgfpathrectangle{\pgfqpoint{0.511823in}{0.504323in}}{\pgfqpoint{3.218177in}{3.225677in}} %
\pgfusepath{clip}%
\pgfsetbuttcap%
\pgfsetroundjoin%
\definecolor{currentfill}{rgb}{0.000000,0.000000,0.545098}%
\pgfsetfillcolor{currentfill}%
\pgfsetfillopacity{0.400000}%
\pgfsetlinewidth{0.501875pt}%
\definecolor{currentstroke}{rgb}{0.000000,0.000000,0.545098}%
\pgfsetstrokecolor{currentstroke}%
\pgfsetstrokeopacity{0.400000}%
\pgfsetdash{}{0pt}%
\pgfpathmoveto{\pgfqpoint{1.502547in}{2.249784in}}%
\pgfpathcurveto{\pgfqpoint{1.513597in}{2.249784in}}{\pgfqpoint{1.524196in}{2.254174in}}{\pgfqpoint{1.532010in}{2.261988in}}%
\pgfpathcurveto{\pgfqpoint{1.539823in}{2.269801in}}{\pgfqpoint{1.544214in}{2.280400in}}{\pgfqpoint{1.544214in}{2.291451in}}%
\pgfpathcurveto{\pgfqpoint{1.544214in}{2.302501in}}{\pgfqpoint{1.539823in}{2.313100in}}{\pgfqpoint{1.532010in}{2.320913in}}%
\pgfpathcurveto{\pgfqpoint{1.524196in}{2.328727in}}{\pgfqpoint{1.513597in}{2.333117in}}{\pgfqpoint{1.502547in}{2.333117in}}%
\pgfpathcurveto{\pgfqpoint{1.491497in}{2.333117in}}{\pgfqpoint{1.480898in}{2.328727in}}{\pgfqpoint{1.473084in}{2.320913in}}%
\pgfpathcurveto{\pgfqpoint{1.465271in}{2.313100in}}{\pgfqpoint{1.460880in}{2.302501in}}{\pgfqpoint{1.460880in}{2.291451in}}%
\pgfpathcurveto{\pgfqpoint{1.460880in}{2.280400in}}{\pgfqpoint{1.465271in}{2.269801in}}{\pgfqpoint{1.473084in}{2.261988in}}%
\pgfpathcurveto{\pgfqpoint{1.480898in}{2.254174in}}{\pgfqpoint{1.491497in}{2.249784in}}{\pgfqpoint{1.502547in}{2.249784in}}%
\pgfpathclose%
\pgfusepath{stroke,fill}%
\end{pgfscope}%
\begin{pgfscope}%
\pgfpathrectangle{\pgfqpoint{0.511823in}{0.504323in}}{\pgfqpoint{3.218177in}{3.225677in}} %
\pgfusepath{clip}%
\pgfsetbuttcap%
\pgfsetroundjoin%
\definecolor{currentfill}{rgb}{0.000000,0.000000,0.545098}%
\pgfsetfillcolor{currentfill}%
\pgfsetfillopacity{0.400000}%
\pgfsetlinewidth{0.501875pt}%
\definecolor{currentstroke}{rgb}{0.000000,0.000000,0.545098}%
\pgfsetstrokecolor{currentstroke}%
\pgfsetstrokeopacity{0.400000}%
\pgfsetdash{}{0pt}%
\pgfpathmoveto{\pgfqpoint{1.553460in}{2.364276in}}%
\pgfpathcurveto{\pgfqpoint{1.564510in}{2.364276in}}{\pgfqpoint{1.575110in}{2.368666in}}{\pgfqpoint{1.582923in}{2.376479in}}%
\pgfpathcurveto{\pgfqpoint{1.590737in}{2.384293in}}{\pgfqpoint{1.595127in}{2.394892in}}{\pgfqpoint{1.595127in}{2.405942in}}%
\pgfpathcurveto{\pgfqpoint{1.595127in}{2.416992in}}{\pgfqpoint{1.590737in}{2.427591in}}{\pgfqpoint{1.582923in}{2.435405in}}%
\pgfpathcurveto{\pgfqpoint{1.575110in}{2.443219in}}{\pgfqpoint{1.564510in}{2.447609in}}{\pgfqpoint{1.553460in}{2.447609in}}%
\pgfpathcurveto{\pgfqpoint{1.542410in}{2.447609in}}{\pgfqpoint{1.531811in}{2.443219in}}{\pgfqpoint{1.523998in}{2.435405in}}%
\pgfpathcurveto{\pgfqpoint{1.516184in}{2.427591in}}{\pgfqpoint{1.511794in}{2.416992in}}{\pgfqpoint{1.511794in}{2.405942in}}%
\pgfpathcurveto{\pgfqpoint{1.511794in}{2.394892in}}{\pgfqpoint{1.516184in}{2.384293in}}{\pgfqpoint{1.523998in}{2.376479in}}%
\pgfpathcurveto{\pgfqpoint{1.531811in}{2.368666in}}{\pgfqpoint{1.542410in}{2.364276in}}{\pgfqpoint{1.553460in}{2.364276in}}%
\pgfpathclose%
\pgfusepath{stroke,fill}%
\end{pgfscope}%
\begin{pgfscope}%
\pgfpathrectangle{\pgfqpoint{0.511823in}{0.504323in}}{\pgfqpoint{3.218177in}{3.225677in}} %
\pgfusepath{clip}%
\pgfsetbuttcap%
\pgfsetroundjoin%
\definecolor{currentfill}{rgb}{0.000000,0.000000,0.545098}%
\pgfsetfillcolor{currentfill}%
\pgfsetfillopacity{0.400000}%
\pgfsetlinewidth{0.501875pt}%
\definecolor{currentstroke}{rgb}{0.000000,0.000000,0.545098}%
\pgfsetstrokecolor{currentstroke}%
\pgfsetstrokeopacity{0.400000}%
\pgfsetdash{}{0pt}%
\pgfpathmoveto{\pgfqpoint{1.502329in}{2.275975in}}%
\pgfpathcurveto{\pgfqpoint{1.513379in}{2.275975in}}{\pgfqpoint{1.523978in}{2.280365in}}{\pgfqpoint{1.531792in}{2.288179in}}%
\pgfpathcurveto{\pgfqpoint{1.539605in}{2.295992in}}{\pgfqpoint{1.543996in}{2.306591in}}{\pgfqpoint{1.543996in}{2.317641in}}%
\pgfpathcurveto{\pgfqpoint{1.543996in}{2.328692in}}{\pgfqpoint{1.539605in}{2.339291in}}{\pgfqpoint{1.531792in}{2.347104in}}%
\pgfpathcurveto{\pgfqpoint{1.523978in}{2.354918in}}{\pgfqpoint{1.513379in}{2.359308in}}{\pgfqpoint{1.502329in}{2.359308in}}%
\pgfpathcurveto{\pgfqpoint{1.491279in}{2.359308in}}{\pgfqpoint{1.480680in}{2.354918in}}{\pgfqpoint{1.472866in}{2.347104in}}%
\pgfpathcurveto{\pgfqpoint{1.465053in}{2.339291in}}{\pgfqpoint{1.460662in}{2.328692in}}{\pgfqpoint{1.460662in}{2.317641in}}%
\pgfpathcurveto{\pgfqpoint{1.460662in}{2.306591in}}{\pgfqpoint{1.465053in}{2.295992in}}{\pgfqpoint{1.472866in}{2.288179in}}%
\pgfpathcurveto{\pgfqpoint{1.480680in}{2.280365in}}{\pgfqpoint{1.491279in}{2.275975in}}{\pgfqpoint{1.502329in}{2.275975in}}%
\pgfpathclose%
\pgfusepath{stroke,fill}%
\end{pgfscope}%
\begin{pgfscope}%
\pgfpathrectangle{\pgfqpoint{0.511823in}{0.504323in}}{\pgfqpoint{3.218177in}{3.225677in}} %
\pgfusepath{clip}%
\pgfsetbuttcap%
\pgfsetroundjoin%
\definecolor{currentfill}{rgb}{0.000000,0.000000,0.545098}%
\pgfsetfillcolor{currentfill}%
\pgfsetfillopacity{0.400000}%
\pgfsetlinewidth{0.501875pt}%
\definecolor{currentstroke}{rgb}{0.000000,0.000000,0.545098}%
\pgfsetstrokecolor{currentstroke}%
\pgfsetstrokeopacity{0.400000}%
\pgfsetdash{}{0pt}%
\pgfpathmoveto{\pgfqpoint{1.470751in}{2.225705in}}%
\pgfpathcurveto{\pgfqpoint{1.481801in}{2.225705in}}{\pgfqpoint{1.492400in}{2.230095in}}{\pgfqpoint{1.500214in}{2.237909in}}%
\pgfpathcurveto{\pgfqpoint{1.508027in}{2.245722in}}{\pgfqpoint{1.512418in}{2.256321in}}{\pgfqpoint{1.512418in}{2.267371in}}%
\pgfpathcurveto{\pgfqpoint{1.512418in}{2.278422in}}{\pgfqpoint{1.508027in}{2.289021in}}{\pgfqpoint{1.500214in}{2.296834in}}%
\pgfpathcurveto{\pgfqpoint{1.492400in}{2.304648in}}{\pgfqpoint{1.481801in}{2.309038in}}{\pgfqpoint{1.470751in}{2.309038in}}%
\pgfpathcurveto{\pgfqpoint{1.459701in}{2.309038in}}{\pgfqpoint{1.449102in}{2.304648in}}{\pgfqpoint{1.441288in}{2.296834in}}%
\pgfpathcurveto{\pgfqpoint{1.433475in}{2.289021in}}{\pgfqpoint{1.429084in}{2.278422in}}{\pgfqpoint{1.429084in}{2.267371in}}%
\pgfpathcurveto{\pgfqpoint{1.429084in}{2.256321in}}{\pgfqpoint{1.433475in}{2.245722in}}{\pgfqpoint{1.441288in}{2.237909in}}%
\pgfpathcurveto{\pgfqpoint{1.449102in}{2.230095in}}{\pgfqpoint{1.459701in}{2.225705in}}{\pgfqpoint{1.470751in}{2.225705in}}%
\pgfpathclose%
\pgfusepath{stroke,fill}%
\end{pgfscope}%
\begin{pgfscope}%
\pgfpathrectangle{\pgfqpoint{0.511823in}{0.504323in}}{\pgfqpoint{3.218177in}{3.225677in}} %
\pgfusepath{clip}%
\pgfsetbuttcap%
\pgfsetroundjoin%
\definecolor{currentfill}{rgb}{0.000000,0.000000,0.545098}%
\pgfsetfillcolor{currentfill}%
\pgfsetfillopacity{0.400000}%
\pgfsetlinewidth{0.501875pt}%
\definecolor{currentstroke}{rgb}{0.000000,0.000000,0.545098}%
\pgfsetstrokecolor{currentstroke}%
\pgfsetstrokeopacity{0.400000}%
\pgfsetdash{}{0pt}%
\pgfpathmoveto{\pgfqpoint{1.557478in}{2.415775in}}%
\pgfpathcurveto{\pgfqpoint{1.568528in}{2.415775in}}{\pgfqpoint{1.579127in}{2.420166in}}{\pgfqpoint{1.586941in}{2.427979in}}%
\pgfpathcurveto{\pgfqpoint{1.594755in}{2.435793in}}{\pgfqpoint{1.599145in}{2.446392in}}{\pgfqpoint{1.599145in}{2.457442in}}%
\pgfpathcurveto{\pgfqpoint{1.599145in}{2.468492in}}{\pgfqpoint{1.594755in}{2.479091in}}{\pgfqpoint{1.586941in}{2.486905in}}%
\pgfpathcurveto{\pgfqpoint{1.579127in}{2.494718in}}{\pgfqpoint{1.568528in}{2.499109in}}{\pgfqpoint{1.557478in}{2.499109in}}%
\pgfpathcurveto{\pgfqpoint{1.546428in}{2.499109in}}{\pgfqpoint{1.535829in}{2.494718in}}{\pgfqpoint{1.528015in}{2.486905in}}%
\pgfpathcurveto{\pgfqpoint{1.520202in}{2.479091in}}{\pgfqpoint{1.515812in}{2.468492in}}{\pgfqpoint{1.515812in}{2.457442in}}%
\pgfpathcurveto{\pgfqpoint{1.515812in}{2.446392in}}{\pgfqpoint{1.520202in}{2.435793in}}{\pgfqpoint{1.528015in}{2.427979in}}%
\pgfpathcurveto{\pgfqpoint{1.535829in}{2.420166in}}{\pgfqpoint{1.546428in}{2.415775in}}{\pgfqpoint{1.557478in}{2.415775in}}%
\pgfpathclose%
\pgfusepath{stroke,fill}%
\end{pgfscope}%
\begin{pgfscope}%
\pgfpathrectangle{\pgfqpoint{0.511823in}{0.504323in}}{\pgfqpoint{3.218177in}{3.225677in}} %
\pgfusepath{clip}%
\pgfsetbuttcap%
\pgfsetroundjoin%
\definecolor{currentfill}{rgb}{0.000000,0.000000,0.545098}%
\pgfsetfillcolor{currentfill}%
\pgfsetfillopacity{0.400000}%
\pgfsetlinewidth{0.501875pt}%
\definecolor{currentstroke}{rgb}{0.000000,0.000000,0.545098}%
\pgfsetstrokecolor{currentstroke}%
\pgfsetstrokeopacity{0.400000}%
\pgfsetdash{}{0pt}%
\pgfpathmoveto{\pgfqpoint{1.497802in}{2.307982in}}%
\pgfpathcurveto{\pgfqpoint{1.508852in}{2.307982in}}{\pgfqpoint{1.519451in}{2.312372in}}{\pgfqpoint{1.527265in}{2.320186in}}%
\pgfpathcurveto{\pgfqpoint{1.535078in}{2.327999in}}{\pgfqpoint{1.539469in}{2.338599in}}{\pgfqpoint{1.539469in}{2.349649in}}%
\pgfpathcurveto{\pgfqpoint{1.539469in}{2.360699in}}{\pgfqpoint{1.535078in}{2.371298in}}{\pgfqpoint{1.527265in}{2.379111in}}%
\pgfpathcurveto{\pgfqpoint{1.519451in}{2.386925in}}{\pgfqpoint{1.508852in}{2.391315in}}{\pgfqpoint{1.497802in}{2.391315in}}%
\pgfpathcurveto{\pgfqpoint{1.486752in}{2.391315in}}{\pgfqpoint{1.476153in}{2.386925in}}{\pgfqpoint{1.468339in}{2.379111in}}%
\pgfpathcurveto{\pgfqpoint{1.460526in}{2.371298in}}{\pgfqpoint{1.456135in}{2.360699in}}{\pgfqpoint{1.456135in}{2.349649in}}%
\pgfpathcurveto{\pgfqpoint{1.456135in}{2.338599in}}{\pgfqpoint{1.460526in}{2.327999in}}{\pgfqpoint{1.468339in}{2.320186in}}%
\pgfpathcurveto{\pgfqpoint{1.476153in}{2.312372in}}{\pgfqpoint{1.486752in}{2.307982in}}{\pgfqpoint{1.497802in}{2.307982in}}%
\pgfpathclose%
\pgfusepath{stroke,fill}%
\end{pgfscope}%
\begin{pgfscope}%
\pgfpathrectangle{\pgfqpoint{0.511823in}{0.504323in}}{\pgfqpoint{3.218177in}{3.225677in}} %
\pgfusepath{clip}%
\pgfsetbuttcap%
\pgfsetroundjoin%
\definecolor{currentfill}{rgb}{0.000000,0.000000,0.545098}%
\pgfsetfillcolor{currentfill}%
\pgfsetfillopacity{0.400000}%
\pgfsetlinewidth{0.501875pt}%
\definecolor{currentstroke}{rgb}{0.000000,0.000000,0.545098}%
\pgfsetstrokecolor{currentstroke}%
\pgfsetstrokeopacity{0.400000}%
\pgfsetdash{}{0pt}%
\pgfpathmoveto{\pgfqpoint{1.644842in}{2.626937in}}%
\pgfpathcurveto{\pgfqpoint{1.655892in}{2.626937in}}{\pgfqpoint{1.666491in}{2.631327in}}{\pgfqpoint{1.674305in}{2.639141in}}%
\pgfpathcurveto{\pgfqpoint{1.682118in}{2.646955in}}{\pgfqpoint{1.686509in}{2.657554in}}{\pgfqpoint{1.686509in}{2.668604in}}%
\pgfpathcurveto{\pgfqpoint{1.686509in}{2.679654in}}{\pgfqpoint{1.682118in}{2.690253in}}{\pgfqpoint{1.674305in}{2.698067in}}%
\pgfpathcurveto{\pgfqpoint{1.666491in}{2.705880in}}{\pgfqpoint{1.655892in}{2.710270in}}{\pgfqpoint{1.644842in}{2.710270in}}%
\pgfpathcurveto{\pgfqpoint{1.633792in}{2.710270in}}{\pgfqpoint{1.623193in}{2.705880in}}{\pgfqpoint{1.615379in}{2.698067in}}%
\pgfpathcurveto{\pgfqpoint{1.607566in}{2.690253in}}{\pgfqpoint{1.603175in}{2.679654in}}{\pgfqpoint{1.603175in}{2.668604in}}%
\pgfpathcurveto{\pgfqpoint{1.603175in}{2.657554in}}{\pgfqpoint{1.607566in}{2.646955in}}{\pgfqpoint{1.615379in}{2.639141in}}%
\pgfpathcurveto{\pgfqpoint{1.623193in}{2.631327in}}{\pgfqpoint{1.633792in}{2.626937in}}{\pgfqpoint{1.644842in}{2.626937in}}%
\pgfpathclose%
\pgfusepath{stroke,fill}%
\end{pgfscope}%
\begin{pgfscope}%
\pgfpathrectangle{\pgfqpoint{0.511823in}{0.504323in}}{\pgfqpoint{3.218177in}{3.225677in}} %
\pgfusepath{clip}%
\pgfsetbuttcap%
\pgfsetroundjoin%
\definecolor{currentfill}{rgb}{0.000000,0.000000,0.545098}%
\pgfsetfillcolor{currentfill}%
\pgfsetfillopacity{0.400000}%
\pgfsetlinewidth{0.501875pt}%
\definecolor{currentstroke}{rgb}{0.000000,0.000000,0.545098}%
\pgfsetstrokecolor{currentstroke}%
\pgfsetstrokeopacity{0.400000}%
\pgfsetdash{}{0pt}%
\pgfpathmoveto{\pgfqpoint{1.481986in}{2.303268in}}%
\pgfpathcurveto{\pgfqpoint{1.493037in}{2.303268in}}{\pgfqpoint{1.503636in}{2.307658in}}{\pgfqpoint{1.511449in}{2.315472in}}%
\pgfpathcurveto{\pgfqpoint{1.519263in}{2.323285in}}{\pgfqpoint{1.523653in}{2.333884in}}{\pgfqpoint{1.523653in}{2.344934in}}%
\pgfpathcurveto{\pgfqpoint{1.523653in}{2.355984in}}{\pgfqpoint{1.519263in}{2.366583in}}{\pgfqpoint{1.511449in}{2.374397in}}%
\pgfpathcurveto{\pgfqpoint{1.503636in}{2.382211in}}{\pgfqpoint{1.493037in}{2.386601in}}{\pgfqpoint{1.481986in}{2.386601in}}%
\pgfpathcurveto{\pgfqpoint{1.470936in}{2.386601in}}{\pgfqpoint{1.460337in}{2.382211in}}{\pgfqpoint{1.452524in}{2.374397in}}%
\pgfpathcurveto{\pgfqpoint{1.444710in}{2.366583in}}{\pgfqpoint{1.440320in}{2.355984in}}{\pgfqpoint{1.440320in}{2.344934in}}%
\pgfpathcurveto{\pgfqpoint{1.440320in}{2.333884in}}{\pgfqpoint{1.444710in}{2.323285in}}{\pgfqpoint{1.452524in}{2.315472in}}%
\pgfpathcurveto{\pgfqpoint{1.460337in}{2.307658in}}{\pgfqpoint{1.470936in}{2.303268in}}{\pgfqpoint{1.481986in}{2.303268in}}%
\pgfpathclose%
\pgfusepath{stroke,fill}%
\end{pgfscope}%
\begin{pgfscope}%
\pgfpathrectangle{\pgfqpoint{0.511823in}{0.504323in}}{\pgfqpoint{3.218177in}{3.225677in}} %
\pgfusepath{clip}%
\pgfsetbuttcap%
\pgfsetroundjoin%
\definecolor{currentfill}{rgb}{0.000000,0.000000,0.545098}%
\pgfsetfillcolor{currentfill}%
\pgfsetfillopacity{0.400000}%
\pgfsetlinewidth{0.501875pt}%
\definecolor{currentstroke}{rgb}{0.000000,0.000000,0.545098}%
\pgfsetstrokecolor{currentstroke}%
\pgfsetstrokeopacity{0.400000}%
\pgfsetdash{}{0pt}%
\pgfpathmoveto{\pgfqpoint{1.451300in}{2.252752in}}%
\pgfpathcurveto{\pgfqpoint{1.462350in}{2.252752in}}{\pgfqpoint{1.472949in}{2.257142in}}{\pgfqpoint{1.480762in}{2.264955in}}%
\pgfpathcurveto{\pgfqpoint{1.488576in}{2.272769in}}{\pgfqpoint{1.492966in}{2.283368in}}{\pgfqpoint{1.492966in}{2.294418in}}%
\pgfpathcurveto{\pgfqpoint{1.492966in}{2.305468in}}{\pgfqpoint{1.488576in}{2.316067in}}{\pgfqpoint{1.480762in}{2.323881in}}%
\pgfpathcurveto{\pgfqpoint{1.472949in}{2.331695in}}{\pgfqpoint{1.462350in}{2.336085in}}{\pgfqpoint{1.451300in}{2.336085in}}%
\pgfpathcurveto{\pgfqpoint{1.440249in}{2.336085in}}{\pgfqpoint{1.429650in}{2.331695in}}{\pgfqpoint{1.421837in}{2.323881in}}%
\pgfpathcurveto{\pgfqpoint{1.414023in}{2.316067in}}{\pgfqpoint{1.409633in}{2.305468in}}{\pgfqpoint{1.409633in}{2.294418in}}%
\pgfpathcurveto{\pgfqpoint{1.409633in}{2.283368in}}{\pgfqpoint{1.414023in}{2.272769in}}{\pgfqpoint{1.421837in}{2.264955in}}%
\pgfpathcurveto{\pgfqpoint{1.429650in}{2.257142in}}{\pgfqpoint{1.440249in}{2.252752in}}{\pgfqpoint{1.451300in}{2.252752in}}%
\pgfpathclose%
\pgfusepath{stroke,fill}%
\end{pgfscope}%
\begin{pgfscope}%
\pgfpathrectangle{\pgfqpoint{0.511823in}{0.504323in}}{\pgfqpoint{3.218177in}{3.225677in}} %
\pgfusepath{clip}%
\pgfsetbuttcap%
\pgfsetroundjoin%
\definecolor{currentfill}{rgb}{0.000000,0.000000,0.545098}%
\pgfsetfillcolor{currentfill}%
\pgfsetfillopacity{0.400000}%
\pgfsetlinewidth{0.501875pt}%
\definecolor{currentstroke}{rgb}{0.000000,0.000000,0.545098}%
\pgfsetstrokecolor{currentstroke}%
\pgfsetstrokeopacity{0.400000}%
\pgfsetdash{}{0pt}%
\pgfpathmoveto{\pgfqpoint{1.454763in}{2.273965in}}%
\pgfpathcurveto{\pgfqpoint{1.465813in}{2.273965in}}{\pgfqpoint{1.476412in}{2.278355in}}{\pgfqpoint{1.484225in}{2.286169in}}%
\pgfpathcurveto{\pgfqpoint{1.492039in}{2.293982in}}{\pgfqpoint{1.496429in}{2.304581in}}{\pgfqpoint{1.496429in}{2.315632in}}%
\pgfpathcurveto{\pgfqpoint{1.496429in}{2.326682in}}{\pgfqpoint{1.492039in}{2.337281in}}{\pgfqpoint{1.484225in}{2.345094in}}%
\pgfpathcurveto{\pgfqpoint{1.476412in}{2.352908in}}{\pgfqpoint{1.465813in}{2.357298in}}{\pgfqpoint{1.454763in}{2.357298in}}%
\pgfpathcurveto{\pgfqpoint{1.443713in}{2.357298in}}{\pgfqpoint{1.433114in}{2.352908in}}{\pgfqpoint{1.425300in}{2.345094in}}%
\pgfpathcurveto{\pgfqpoint{1.417486in}{2.337281in}}{\pgfqpoint{1.413096in}{2.326682in}}{\pgfqpoint{1.413096in}{2.315632in}}%
\pgfpathcurveto{\pgfqpoint{1.413096in}{2.304581in}}{\pgfqpoint{1.417486in}{2.293982in}}{\pgfqpoint{1.425300in}{2.286169in}}%
\pgfpathcurveto{\pgfqpoint{1.433114in}{2.278355in}}{\pgfqpoint{1.443713in}{2.273965in}}{\pgfqpoint{1.454763in}{2.273965in}}%
\pgfpathclose%
\pgfusepath{stroke,fill}%
\end{pgfscope}%
\begin{pgfscope}%
\pgfpathrectangle{\pgfqpoint{0.511823in}{0.504323in}}{\pgfqpoint{3.218177in}{3.225677in}} %
\pgfusepath{clip}%
\pgfsetbuttcap%
\pgfsetroundjoin%
\definecolor{currentfill}{rgb}{0.000000,0.000000,0.545098}%
\pgfsetfillcolor{currentfill}%
\pgfsetfillopacity{0.400000}%
\pgfsetlinewidth{0.501875pt}%
\definecolor{currentstroke}{rgb}{0.000000,0.000000,0.545098}%
\pgfsetstrokecolor{currentstroke}%
\pgfsetstrokeopacity{0.400000}%
\pgfsetdash{}{0pt}%
\pgfpathmoveto{\pgfqpoint{1.497825in}{2.380459in}}%
\pgfpathcurveto{\pgfqpoint{1.508875in}{2.380459in}}{\pgfqpoint{1.519474in}{2.384849in}}{\pgfqpoint{1.527287in}{2.392663in}}%
\pgfpathcurveto{\pgfqpoint{1.535101in}{2.400476in}}{\pgfqpoint{1.539491in}{2.411075in}}{\pgfqpoint{1.539491in}{2.422125in}}%
\pgfpathcurveto{\pgfqpoint{1.539491in}{2.433175in}}{\pgfqpoint{1.535101in}{2.443775in}}{\pgfqpoint{1.527287in}{2.451588in}}%
\pgfpathcurveto{\pgfqpoint{1.519474in}{2.459402in}}{\pgfqpoint{1.508875in}{2.463792in}}{\pgfqpoint{1.497825in}{2.463792in}}%
\pgfpathcurveto{\pgfqpoint{1.486774in}{2.463792in}}{\pgfqpoint{1.476175in}{2.459402in}}{\pgfqpoint{1.468362in}{2.451588in}}%
\pgfpathcurveto{\pgfqpoint{1.460548in}{2.443775in}}{\pgfqpoint{1.456158in}{2.433175in}}{\pgfqpoint{1.456158in}{2.422125in}}%
\pgfpathcurveto{\pgfqpoint{1.456158in}{2.411075in}}{\pgfqpoint{1.460548in}{2.400476in}}{\pgfqpoint{1.468362in}{2.392663in}}%
\pgfpathcurveto{\pgfqpoint{1.476175in}{2.384849in}}{\pgfqpoint{1.486774in}{2.380459in}}{\pgfqpoint{1.497825in}{2.380459in}}%
\pgfpathclose%
\pgfusepath{stroke,fill}%
\end{pgfscope}%
\begin{pgfscope}%
\pgfpathrectangle{\pgfqpoint{0.511823in}{0.504323in}}{\pgfqpoint{3.218177in}{3.225677in}} %
\pgfusepath{clip}%
\pgfsetbuttcap%
\pgfsetroundjoin%
\definecolor{currentfill}{rgb}{0.000000,0.000000,0.545098}%
\pgfsetfillcolor{currentfill}%
\pgfsetfillopacity{0.400000}%
\pgfsetlinewidth{0.501875pt}%
\definecolor{currentstroke}{rgb}{0.000000,0.000000,0.545098}%
\pgfsetstrokecolor{currentstroke}%
\pgfsetstrokeopacity{0.400000}%
\pgfsetdash{}{0pt}%
\pgfpathmoveto{\pgfqpoint{1.435942in}{2.261667in}}%
\pgfpathcurveto{\pgfqpoint{1.446992in}{2.261667in}}{\pgfqpoint{1.457591in}{2.266058in}}{\pgfqpoint{1.465405in}{2.273871in}}%
\pgfpathcurveto{\pgfqpoint{1.473219in}{2.281685in}}{\pgfqpoint{1.477609in}{2.292284in}}{\pgfqpoint{1.477609in}{2.303334in}}%
\pgfpathcurveto{\pgfqpoint{1.477609in}{2.314384in}}{\pgfqpoint{1.473219in}{2.324983in}}{\pgfqpoint{1.465405in}{2.332797in}}%
\pgfpathcurveto{\pgfqpoint{1.457591in}{2.340610in}}{\pgfqpoint{1.446992in}{2.345001in}}{\pgfqpoint{1.435942in}{2.345001in}}%
\pgfpathcurveto{\pgfqpoint{1.424892in}{2.345001in}}{\pgfqpoint{1.414293in}{2.340610in}}{\pgfqpoint{1.406479in}{2.332797in}}%
\pgfpathcurveto{\pgfqpoint{1.398666in}{2.324983in}}{\pgfqpoint{1.394276in}{2.314384in}}{\pgfqpoint{1.394276in}{2.303334in}}%
\pgfpathcurveto{\pgfqpoint{1.394276in}{2.292284in}}{\pgfqpoint{1.398666in}{2.281685in}}{\pgfqpoint{1.406479in}{2.273871in}}%
\pgfpathcurveto{\pgfqpoint{1.414293in}{2.266058in}}{\pgfqpoint{1.424892in}{2.261667in}}{\pgfqpoint{1.435942in}{2.261667in}}%
\pgfpathclose%
\pgfusepath{stroke,fill}%
\end{pgfscope}%
\begin{pgfscope}%
\pgfpathrectangle{\pgfqpoint{0.511823in}{0.504323in}}{\pgfqpoint{3.218177in}{3.225677in}} %
\pgfusepath{clip}%
\pgfsetbuttcap%
\pgfsetroundjoin%
\definecolor{currentfill}{rgb}{0.000000,0.000000,0.545098}%
\pgfsetfillcolor{currentfill}%
\pgfsetfillopacity{0.400000}%
\pgfsetlinewidth{0.501875pt}%
\definecolor{currentstroke}{rgb}{0.000000,0.000000,0.545098}%
\pgfsetstrokecolor{currentstroke}%
\pgfsetstrokeopacity{0.400000}%
\pgfsetdash{}{0pt}%
\pgfpathmoveto{\pgfqpoint{1.401011in}{2.199589in}}%
\pgfpathcurveto{\pgfqpoint{1.412061in}{2.199589in}}{\pgfqpoint{1.422660in}{2.203980in}}{\pgfqpoint{1.430474in}{2.211793in}}%
\pgfpathcurveto{\pgfqpoint{1.438287in}{2.219607in}}{\pgfqpoint{1.442678in}{2.230206in}}{\pgfqpoint{1.442678in}{2.241256in}}%
\pgfpathcurveto{\pgfqpoint{1.442678in}{2.252306in}}{\pgfqpoint{1.438287in}{2.262905in}}{\pgfqpoint{1.430474in}{2.270719in}}%
\pgfpathcurveto{\pgfqpoint{1.422660in}{2.278532in}}{\pgfqpoint{1.412061in}{2.282923in}}{\pgfqpoint{1.401011in}{2.282923in}}%
\pgfpathcurveto{\pgfqpoint{1.389961in}{2.282923in}}{\pgfqpoint{1.379362in}{2.278532in}}{\pgfqpoint{1.371548in}{2.270719in}}%
\pgfpathcurveto{\pgfqpoint{1.363734in}{2.262905in}}{\pgfqpoint{1.359344in}{2.252306in}}{\pgfqpoint{1.359344in}{2.241256in}}%
\pgfpathcurveto{\pgfqpoint{1.359344in}{2.230206in}}{\pgfqpoint{1.363734in}{2.219607in}}{\pgfqpoint{1.371548in}{2.211793in}}%
\pgfpathcurveto{\pgfqpoint{1.379362in}{2.203980in}}{\pgfqpoint{1.389961in}{2.199589in}}{\pgfqpoint{1.401011in}{2.199589in}}%
\pgfpathclose%
\pgfusepath{stroke,fill}%
\end{pgfscope}%
\begin{pgfscope}%
\pgfpathrectangle{\pgfqpoint{0.511823in}{0.504323in}}{\pgfqpoint{3.218177in}{3.225677in}} %
\pgfusepath{clip}%
\pgfsetbuttcap%
\pgfsetroundjoin%
\definecolor{currentfill}{rgb}{0.000000,0.000000,0.545098}%
\pgfsetfillcolor{currentfill}%
\pgfsetfillopacity{0.400000}%
\pgfsetlinewidth{0.501875pt}%
\definecolor{currentstroke}{rgb}{0.000000,0.000000,0.545098}%
\pgfsetstrokecolor{currentstroke}%
\pgfsetstrokeopacity{0.400000}%
\pgfsetdash{}{0pt}%
\pgfpathmoveto{\pgfqpoint{1.545601in}{2.531701in}}%
\pgfpathcurveto{\pgfqpoint{1.556651in}{2.531701in}}{\pgfqpoint{1.567250in}{2.536091in}}{\pgfqpoint{1.575064in}{2.543905in}}%
\pgfpathcurveto{\pgfqpoint{1.582877in}{2.551719in}}{\pgfqpoint{1.587267in}{2.562318in}}{\pgfqpoint{1.587267in}{2.573368in}}%
\pgfpathcurveto{\pgfqpoint{1.587267in}{2.584418in}}{\pgfqpoint{1.582877in}{2.595017in}}{\pgfqpoint{1.575064in}{2.602831in}}%
\pgfpathcurveto{\pgfqpoint{1.567250in}{2.610644in}}{\pgfqpoint{1.556651in}{2.615034in}}{\pgfqpoint{1.545601in}{2.615034in}}%
\pgfpathcurveto{\pgfqpoint{1.534551in}{2.615034in}}{\pgfqpoint{1.523952in}{2.610644in}}{\pgfqpoint{1.516138in}{2.602831in}}%
\pgfpathcurveto{\pgfqpoint{1.508324in}{2.595017in}}{\pgfqpoint{1.503934in}{2.584418in}}{\pgfqpoint{1.503934in}{2.573368in}}%
\pgfpathcurveto{\pgfqpoint{1.503934in}{2.562318in}}{\pgfqpoint{1.508324in}{2.551719in}}{\pgfqpoint{1.516138in}{2.543905in}}%
\pgfpathcurveto{\pgfqpoint{1.523952in}{2.536091in}}{\pgfqpoint{1.534551in}{2.531701in}}{\pgfqpoint{1.545601in}{2.531701in}}%
\pgfpathclose%
\pgfusepath{stroke,fill}%
\end{pgfscope}%
\begin{pgfscope}%
\pgfpathrectangle{\pgfqpoint{0.511823in}{0.504323in}}{\pgfqpoint{3.218177in}{3.225677in}} %
\pgfusepath{clip}%
\pgfsetbuttcap%
\pgfsetroundjoin%
\definecolor{currentfill}{rgb}{0.000000,0.000000,0.545098}%
\pgfsetfillcolor{currentfill}%
\pgfsetfillopacity{0.400000}%
\pgfsetlinewidth{0.501875pt}%
\definecolor{currentstroke}{rgb}{0.000000,0.000000,0.545098}%
\pgfsetstrokecolor{currentstroke}%
\pgfsetstrokeopacity{0.400000}%
\pgfsetdash{}{0pt}%
\pgfpathmoveto{\pgfqpoint{1.391629in}{2.206381in}}%
\pgfpathcurveto{\pgfqpoint{1.402679in}{2.206381in}}{\pgfqpoint{1.413278in}{2.210771in}}{\pgfqpoint{1.421092in}{2.218585in}}%
\pgfpathcurveto{\pgfqpoint{1.428906in}{2.226398in}}{\pgfqpoint{1.433296in}{2.236997in}}{\pgfqpoint{1.433296in}{2.248047in}}%
\pgfpathcurveto{\pgfqpoint{1.433296in}{2.259098in}}{\pgfqpoint{1.428906in}{2.269697in}}{\pgfqpoint{1.421092in}{2.277510in}}%
\pgfpathcurveto{\pgfqpoint{1.413278in}{2.285324in}}{\pgfqpoint{1.402679in}{2.289714in}}{\pgfqpoint{1.391629in}{2.289714in}}%
\pgfpathcurveto{\pgfqpoint{1.380579in}{2.289714in}}{\pgfqpoint{1.369980in}{2.285324in}}{\pgfqpoint{1.362166in}{2.277510in}}%
\pgfpathcurveto{\pgfqpoint{1.354353in}{2.269697in}}{\pgfqpoint{1.349963in}{2.259098in}}{\pgfqpoint{1.349963in}{2.248047in}}%
\pgfpathcurveto{\pgfqpoint{1.349963in}{2.236997in}}{\pgfqpoint{1.354353in}{2.226398in}}{\pgfqpoint{1.362166in}{2.218585in}}%
\pgfpathcurveto{\pgfqpoint{1.369980in}{2.210771in}}{\pgfqpoint{1.380579in}{2.206381in}}{\pgfqpoint{1.391629in}{2.206381in}}%
\pgfpathclose%
\pgfusepath{stroke,fill}%
\end{pgfscope}%
\begin{pgfscope}%
\pgfpathrectangle{\pgfqpoint{0.511823in}{0.504323in}}{\pgfqpoint{3.218177in}{3.225677in}} %
\pgfusepath{clip}%
\pgfsetbuttcap%
\pgfsetroundjoin%
\definecolor{currentfill}{rgb}{0.000000,0.000000,0.545098}%
\pgfsetfillcolor{currentfill}%
\pgfsetfillopacity{0.400000}%
\pgfsetlinewidth{0.501875pt}%
\definecolor{currentstroke}{rgb}{0.000000,0.000000,0.545098}%
\pgfsetstrokecolor{currentstroke}%
\pgfsetstrokeopacity{0.400000}%
\pgfsetdash{}{0pt}%
\pgfpathmoveto{\pgfqpoint{1.526395in}{2.522381in}}%
\pgfpathcurveto{\pgfqpoint{1.537446in}{2.522381in}}{\pgfqpoint{1.548045in}{2.526772in}}{\pgfqpoint{1.555858in}{2.534585in}}%
\pgfpathcurveto{\pgfqpoint{1.563672in}{2.542399in}}{\pgfqpoint{1.568062in}{2.552998in}}{\pgfqpoint{1.568062in}{2.564048in}}%
\pgfpathcurveto{\pgfqpoint{1.568062in}{2.575098in}}{\pgfqpoint{1.563672in}{2.585697in}}{\pgfqpoint{1.555858in}{2.593511in}}%
\pgfpathcurveto{\pgfqpoint{1.548045in}{2.601324in}}{\pgfqpoint{1.537446in}{2.605715in}}{\pgfqpoint{1.526395in}{2.605715in}}%
\pgfpathcurveto{\pgfqpoint{1.515345in}{2.605715in}}{\pgfqpoint{1.504746in}{2.601324in}}{\pgfqpoint{1.496933in}{2.593511in}}%
\pgfpathcurveto{\pgfqpoint{1.489119in}{2.585697in}}{\pgfqpoint{1.484729in}{2.575098in}}{\pgfqpoint{1.484729in}{2.564048in}}%
\pgfpathcurveto{\pgfqpoint{1.484729in}{2.552998in}}{\pgfqpoint{1.489119in}{2.542399in}}{\pgfqpoint{1.496933in}{2.534585in}}%
\pgfpathcurveto{\pgfqpoint{1.504746in}{2.526772in}}{\pgfqpoint{1.515345in}{2.522381in}}{\pgfqpoint{1.526395in}{2.522381in}}%
\pgfpathclose%
\pgfusepath{stroke,fill}%
\end{pgfscope}%
\begin{pgfscope}%
\pgfpathrectangle{\pgfqpoint{0.511823in}{0.504323in}}{\pgfqpoint{3.218177in}{3.225677in}} %
\pgfusepath{clip}%
\pgfsetbuttcap%
\pgfsetroundjoin%
\definecolor{currentfill}{rgb}{0.000000,0.000000,0.545098}%
\pgfsetfillcolor{currentfill}%
\pgfsetfillopacity{0.400000}%
\pgfsetlinewidth{0.501875pt}%
\definecolor{currentstroke}{rgb}{0.000000,0.000000,0.545098}%
\pgfsetstrokecolor{currentstroke}%
\pgfsetstrokeopacity{0.400000}%
\pgfsetdash{}{0pt}%
\pgfpathmoveto{\pgfqpoint{1.450736in}{2.368148in}}%
\pgfpathcurveto{\pgfqpoint{1.461786in}{2.368148in}}{\pgfqpoint{1.472385in}{2.372538in}}{\pgfqpoint{1.480199in}{2.380352in}}%
\pgfpathcurveto{\pgfqpoint{1.488012in}{2.388166in}}{\pgfqpoint{1.492403in}{2.398765in}}{\pgfqpoint{1.492403in}{2.409815in}}%
\pgfpathcurveto{\pgfqpoint{1.492403in}{2.420865in}}{\pgfqpoint{1.488012in}{2.431464in}}{\pgfqpoint{1.480199in}{2.439277in}}%
\pgfpathcurveto{\pgfqpoint{1.472385in}{2.447091in}}{\pgfqpoint{1.461786in}{2.451481in}}{\pgfqpoint{1.450736in}{2.451481in}}%
\pgfpathcurveto{\pgfqpoint{1.439686in}{2.451481in}}{\pgfqpoint{1.429087in}{2.447091in}}{\pgfqpoint{1.421273in}{2.439277in}}%
\pgfpathcurveto{\pgfqpoint{1.413460in}{2.431464in}}{\pgfqpoint{1.409069in}{2.420865in}}{\pgfqpoint{1.409069in}{2.409815in}}%
\pgfpathcurveto{\pgfqpoint{1.409069in}{2.398765in}}{\pgfqpoint{1.413460in}{2.388166in}}{\pgfqpoint{1.421273in}{2.380352in}}%
\pgfpathcurveto{\pgfqpoint{1.429087in}{2.372538in}}{\pgfqpoint{1.439686in}{2.368148in}}{\pgfqpoint{1.450736in}{2.368148in}}%
\pgfpathclose%
\pgfusepath{stroke,fill}%
\end{pgfscope}%
\begin{pgfscope}%
\pgfpathrectangle{\pgfqpoint{0.511823in}{0.504323in}}{\pgfqpoint{3.218177in}{3.225677in}} %
\pgfusepath{clip}%
\pgfsetbuttcap%
\pgfsetroundjoin%
\definecolor{currentfill}{rgb}{0.000000,0.000000,0.545098}%
\pgfsetfillcolor{currentfill}%
\pgfsetfillopacity{0.400000}%
\pgfsetlinewidth{0.501875pt}%
\definecolor{currentstroke}{rgb}{0.000000,0.000000,0.545098}%
\pgfsetstrokecolor{currentstroke}%
\pgfsetstrokeopacity{0.400000}%
\pgfsetdash{}{0pt}%
\pgfpathmoveto{\pgfqpoint{1.384616in}{2.232858in}}%
\pgfpathcurveto{\pgfqpoint{1.395666in}{2.232858in}}{\pgfqpoint{1.406265in}{2.237248in}}{\pgfqpoint{1.414079in}{2.245062in}}%
\pgfpathcurveto{\pgfqpoint{1.421893in}{2.252875in}}{\pgfqpoint{1.426283in}{2.263475in}}{\pgfqpoint{1.426283in}{2.274525in}}%
\pgfpathcurveto{\pgfqpoint{1.426283in}{2.285575in}}{\pgfqpoint{1.421893in}{2.296174in}}{\pgfqpoint{1.414079in}{2.303987in}}%
\pgfpathcurveto{\pgfqpoint{1.406265in}{2.311801in}}{\pgfqpoint{1.395666in}{2.316191in}}{\pgfqpoint{1.384616in}{2.316191in}}%
\pgfpathcurveto{\pgfqpoint{1.373566in}{2.316191in}}{\pgfqpoint{1.362967in}{2.311801in}}{\pgfqpoint{1.355154in}{2.303987in}}%
\pgfpathcurveto{\pgfqpoint{1.347340in}{2.296174in}}{\pgfqpoint{1.342950in}{2.285575in}}{\pgfqpoint{1.342950in}{2.274525in}}%
\pgfpathcurveto{\pgfqpoint{1.342950in}{2.263475in}}{\pgfqpoint{1.347340in}{2.252875in}}{\pgfqpoint{1.355154in}{2.245062in}}%
\pgfpathcurveto{\pgfqpoint{1.362967in}{2.237248in}}{\pgfqpoint{1.373566in}{2.232858in}}{\pgfqpoint{1.384616in}{2.232858in}}%
\pgfpathclose%
\pgfusepath{stroke,fill}%
\end{pgfscope}%
\begin{pgfscope}%
\pgfpathrectangle{\pgfqpoint{0.511823in}{0.504323in}}{\pgfqpoint{3.218177in}{3.225677in}} %
\pgfusepath{clip}%
\pgfsetbuttcap%
\pgfsetroundjoin%
\definecolor{currentfill}{rgb}{0.000000,0.000000,0.545098}%
\pgfsetfillcolor{currentfill}%
\pgfsetfillopacity{0.400000}%
\pgfsetlinewidth{0.501875pt}%
\definecolor{currentstroke}{rgb}{0.000000,0.000000,0.545098}%
\pgfsetstrokecolor{currentstroke}%
\pgfsetstrokeopacity{0.400000}%
\pgfsetdash{}{0pt}%
\pgfpathmoveto{\pgfqpoint{1.514842in}{2.547158in}}%
\pgfpathcurveto{\pgfqpoint{1.525892in}{2.547158in}}{\pgfqpoint{1.536491in}{2.551548in}}{\pgfqpoint{1.544305in}{2.559362in}}%
\pgfpathcurveto{\pgfqpoint{1.552119in}{2.567175in}}{\pgfqpoint{1.556509in}{2.577774in}}{\pgfqpoint{1.556509in}{2.588824in}}%
\pgfpathcurveto{\pgfqpoint{1.556509in}{2.599875in}}{\pgfqpoint{1.552119in}{2.610474in}}{\pgfqpoint{1.544305in}{2.618287in}}%
\pgfpathcurveto{\pgfqpoint{1.536491in}{2.626101in}}{\pgfqpoint{1.525892in}{2.630491in}}{\pgfqpoint{1.514842in}{2.630491in}}%
\pgfpathcurveto{\pgfqpoint{1.503792in}{2.630491in}}{\pgfqpoint{1.493193in}{2.626101in}}{\pgfqpoint{1.485380in}{2.618287in}}%
\pgfpathcurveto{\pgfqpoint{1.477566in}{2.610474in}}{\pgfqpoint{1.473176in}{2.599875in}}{\pgfqpoint{1.473176in}{2.588824in}}%
\pgfpathcurveto{\pgfqpoint{1.473176in}{2.577774in}}{\pgfqpoint{1.477566in}{2.567175in}}{\pgfqpoint{1.485380in}{2.559362in}}%
\pgfpathcurveto{\pgfqpoint{1.493193in}{2.551548in}}{\pgfqpoint{1.503792in}{2.547158in}}{\pgfqpoint{1.514842in}{2.547158in}}%
\pgfpathclose%
\pgfusepath{stroke,fill}%
\end{pgfscope}%
\begin{pgfscope}%
\pgfpathrectangle{\pgfqpoint{0.511823in}{0.504323in}}{\pgfqpoint{3.218177in}{3.225677in}} %
\pgfusepath{clip}%
\pgfsetbuttcap%
\pgfsetroundjoin%
\definecolor{currentfill}{rgb}{0.000000,0.000000,0.545098}%
\pgfsetfillcolor{currentfill}%
\pgfsetfillopacity{0.400000}%
\pgfsetlinewidth{0.501875pt}%
\definecolor{currentstroke}{rgb}{0.000000,0.000000,0.545098}%
\pgfsetstrokecolor{currentstroke}%
\pgfsetstrokeopacity{0.400000}%
\pgfsetdash{}{0pt}%
\pgfpathmoveto{\pgfqpoint{1.416169in}{2.335309in}}%
\pgfpathcurveto{\pgfqpoint{1.427220in}{2.335309in}}{\pgfqpoint{1.437819in}{2.339699in}}{\pgfqpoint{1.445632in}{2.347513in}}%
\pgfpathcurveto{\pgfqpoint{1.453446in}{2.355326in}}{\pgfqpoint{1.457836in}{2.365926in}}{\pgfqpoint{1.457836in}{2.376976in}}%
\pgfpathcurveto{\pgfqpoint{1.457836in}{2.388026in}}{\pgfqpoint{1.453446in}{2.398625in}}{\pgfqpoint{1.445632in}{2.406438in}}%
\pgfpathcurveto{\pgfqpoint{1.437819in}{2.414252in}}{\pgfqpoint{1.427220in}{2.418642in}}{\pgfqpoint{1.416169in}{2.418642in}}%
\pgfpathcurveto{\pgfqpoint{1.405119in}{2.418642in}}{\pgfqpoint{1.394520in}{2.414252in}}{\pgfqpoint{1.386707in}{2.406438in}}%
\pgfpathcurveto{\pgfqpoint{1.378893in}{2.398625in}}{\pgfqpoint{1.374503in}{2.388026in}}{\pgfqpoint{1.374503in}{2.376976in}}%
\pgfpathcurveto{\pgfqpoint{1.374503in}{2.365926in}}{\pgfqpoint{1.378893in}{2.355326in}}{\pgfqpoint{1.386707in}{2.347513in}}%
\pgfpathcurveto{\pgfqpoint{1.394520in}{2.339699in}}{\pgfqpoint{1.405119in}{2.335309in}}{\pgfqpoint{1.416169in}{2.335309in}}%
\pgfpathclose%
\pgfusepath{stroke,fill}%
\end{pgfscope}%
\begin{pgfscope}%
\pgfpathrectangle{\pgfqpoint{0.511823in}{0.504323in}}{\pgfqpoint{3.218177in}{3.225677in}} %
\pgfusepath{clip}%
\pgfsetbuttcap%
\pgfsetroundjoin%
\definecolor{currentfill}{rgb}{0.000000,0.000000,0.545098}%
\pgfsetfillcolor{currentfill}%
\pgfsetfillopacity{0.400000}%
\pgfsetlinewidth{0.501875pt}%
\definecolor{currentstroke}{rgb}{0.000000,0.000000,0.545098}%
\pgfsetstrokecolor{currentstroke}%
\pgfsetstrokeopacity{0.400000}%
\pgfsetdash{}{0pt}%
\pgfpathmoveto{\pgfqpoint{1.371085in}{2.245181in}}%
\pgfpathcurveto{\pgfqpoint{1.382136in}{2.245181in}}{\pgfqpoint{1.392735in}{2.249571in}}{\pgfqpoint{1.400548in}{2.257384in}}%
\pgfpathcurveto{\pgfqpoint{1.408362in}{2.265198in}}{\pgfqpoint{1.412752in}{2.275797in}}{\pgfqpoint{1.412752in}{2.286847in}}%
\pgfpathcurveto{\pgfqpoint{1.412752in}{2.297897in}}{\pgfqpoint{1.408362in}{2.308496in}}{\pgfqpoint{1.400548in}{2.316310in}}%
\pgfpathcurveto{\pgfqpoint{1.392735in}{2.324124in}}{\pgfqpoint{1.382136in}{2.328514in}}{\pgfqpoint{1.371085in}{2.328514in}}%
\pgfpathcurveto{\pgfqpoint{1.360035in}{2.328514in}}{\pgfqpoint{1.349436in}{2.324124in}}{\pgfqpoint{1.341623in}{2.316310in}}%
\pgfpathcurveto{\pgfqpoint{1.333809in}{2.308496in}}{\pgfqpoint{1.329419in}{2.297897in}}{\pgfqpoint{1.329419in}{2.286847in}}%
\pgfpathcurveto{\pgfqpoint{1.329419in}{2.275797in}}{\pgfqpoint{1.333809in}{2.265198in}}{\pgfqpoint{1.341623in}{2.257384in}}%
\pgfpathcurveto{\pgfqpoint{1.349436in}{2.249571in}}{\pgfqpoint{1.360035in}{2.245181in}}{\pgfqpoint{1.371085in}{2.245181in}}%
\pgfpathclose%
\pgfusepath{stroke,fill}%
\end{pgfscope}%
\begin{pgfscope}%
\pgfpathrectangle{\pgfqpoint{0.511823in}{0.504323in}}{\pgfqpoint{3.218177in}{3.225677in}} %
\pgfusepath{clip}%
\pgfsetbuttcap%
\pgfsetroundjoin%
\definecolor{currentfill}{rgb}{0.000000,0.000000,0.545098}%
\pgfsetfillcolor{currentfill}%
\pgfsetfillopacity{0.400000}%
\pgfsetlinewidth{0.501875pt}%
\definecolor{currentstroke}{rgb}{0.000000,0.000000,0.545098}%
\pgfsetstrokecolor{currentstroke}%
\pgfsetstrokeopacity{0.400000}%
\pgfsetdash{}{0pt}%
\pgfpathmoveto{\pgfqpoint{1.417733in}{2.370436in}}%
\pgfpathcurveto{\pgfqpoint{1.428784in}{2.370436in}}{\pgfqpoint{1.439383in}{2.374827in}}{\pgfqpoint{1.447196in}{2.382640in}}%
\pgfpathcurveto{\pgfqpoint{1.455010in}{2.390454in}}{\pgfqpoint{1.459400in}{2.401053in}}{\pgfqpoint{1.459400in}{2.412103in}}%
\pgfpathcurveto{\pgfqpoint{1.459400in}{2.423153in}}{\pgfqpoint{1.455010in}{2.433752in}}{\pgfqpoint{1.447196in}{2.441566in}}%
\pgfpathcurveto{\pgfqpoint{1.439383in}{2.449379in}}{\pgfqpoint{1.428784in}{2.453770in}}{\pgfqpoint{1.417733in}{2.453770in}}%
\pgfpathcurveto{\pgfqpoint{1.406683in}{2.453770in}}{\pgfqpoint{1.396084in}{2.449379in}}{\pgfqpoint{1.388271in}{2.441566in}}%
\pgfpathcurveto{\pgfqpoint{1.380457in}{2.433752in}}{\pgfqpoint{1.376067in}{2.423153in}}{\pgfqpoint{1.376067in}{2.412103in}}%
\pgfpathcurveto{\pgfqpoint{1.376067in}{2.401053in}}{\pgfqpoint{1.380457in}{2.390454in}}{\pgfqpoint{1.388271in}{2.382640in}}%
\pgfpathcurveto{\pgfqpoint{1.396084in}{2.374827in}}{\pgfqpoint{1.406683in}{2.370436in}}{\pgfqpoint{1.417733in}{2.370436in}}%
\pgfpathclose%
\pgfusepath{stroke,fill}%
\end{pgfscope}%
\begin{pgfscope}%
\pgfpathrectangle{\pgfqpoint{0.511823in}{0.504323in}}{\pgfqpoint{3.218177in}{3.225677in}} %
\pgfusepath{clip}%
\pgfsetbuttcap%
\pgfsetroundjoin%
\definecolor{currentfill}{rgb}{0.000000,0.000000,0.545098}%
\pgfsetfillcolor{currentfill}%
\pgfsetfillopacity{0.400000}%
\pgfsetlinewidth{0.501875pt}%
\definecolor{currentstroke}{rgb}{0.000000,0.000000,0.545098}%
\pgfsetstrokecolor{currentstroke}%
\pgfsetstrokeopacity{0.400000}%
\pgfsetdash{}{0pt}%
\pgfpathmoveto{\pgfqpoint{1.470670in}{2.513012in}}%
\pgfpathcurveto{\pgfqpoint{1.481721in}{2.513012in}}{\pgfqpoint{1.492320in}{2.517402in}}{\pgfqpoint{1.500133in}{2.525216in}}%
\pgfpathcurveto{\pgfqpoint{1.507947in}{2.533030in}}{\pgfqpoint{1.512337in}{2.543629in}}{\pgfqpoint{1.512337in}{2.554679in}}%
\pgfpathcurveto{\pgfqpoint{1.512337in}{2.565729in}}{\pgfqpoint{1.507947in}{2.576328in}}{\pgfqpoint{1.500133in}{2.584142in}}%
\pgfpathcurveto{\pgfqpoint{1.492320in}{2.591955in}}{\pgfqpoint{1.481721in}{2.596345in}}{\pgfqpoint{1.470670in}{2.596345in}}%
\pgfpathcurveto{\pgfqpoint{1.459620in}{2.596345in}}{\pgfqpoint{1.449021in}{2.591955in}}{\pgfqpoint{1.441208in}{2.584142in}}%
\pgfpathcurveto{\pgfqpoint{1.433394in}{2.576328in}}{\pgfqpoint{1.429004in}{2.565729in}}{\pgfqpoint{1.429004in}{2.554679in}}%
\pgfpathcurveto{\pgfqpoint{1.429004in}{2.543629in}}{\pgfqpoint{1.433394in}{2.533030in}}{\pgfqpoint{1.441208in}{2.525216in}}%
\pgfpathcurveto{\pgfqpoint{1.449021in}{2.517402in}}{\pgfqpoint{1.459620in}{2.513012in}}{\pgfqpoint{1.470670in}{2.513012in}}%
\pgfpathclose%
\pgfusepath{stroke,fill}%
\end{pgfscope}%
\begin{pgfscope}%
\pgfpathrectangle{\pgfqpoint{0.511823in}{0.504323in}}{\pgfqpoint{3.218177in}{3.225677in}} %
\pgfusepath{clip}%
\pgfsetbuttcap%
\pgfsetroundjoin%
\definecolor{currentfill}{rgb}{0.000000,0.000000,0.545098}%
\pgfsetfillcolor{currentfill}%
\pgfsetfillopacity{0.400000}%
\pgfsetlinewidth{0.501875pt}%
\definecolor{currentstroke}{rgb}{0.000000,0.000000,0.545098}%
\pgfsetstrokecolor{currentstroke}%
\pgfsetstrokeopacity{0.400000}%
\pgfsetdash{}{0pt}%
\pgfpathmoveto{\pgfqpoint{1.435542in}{2.445893in}}%
\pgfpathcurveto{\pgfqpoint{1.446592in}{2.445893in}}{\pgfqpoint{1.457191in}{2.450283in}}{\pgfqpoint{1.465005in}{2.458097in}}%
\pgfpathcurveto{\pgfqpoint{1.472819in}{2.465910in}}{\pgfqpoint{1.477209in}{2.476509in}}{\pgfqpoint{1.477209in}{2.487560in}}%
\pgfpathcurveto{\pgfqpoint{1.477209in}{2.498610in}}{\pgfqpoint{1.472819in}{2.509209in}}{\pgfqpoint{1.465005in}{2.517022in}}%
\pgfpathcurveto{\pgfqpoint{1.457191in}{2.524836in}}{\pgfqpoint{1.446592in}{2.529226in}}{\pgfqpoint{1.435542in}{2.529226in}}%
\pgfpathcurveto{\pgfqpoint{1.424492in}{2.529226in}}{\pgfqpoint{1.413893in}{2.524836in}}{\pgfqpoint{1.406079in}{2.517022in}}%
\pgfpathcurveto{\pgfqpoint{1.398266in}{2.509209in}}{\pgfqpoint{1.393875in}{2.498610in}}{\pgfqpoint{1.393875in}{2.487560in}}%
\pgfpathcurveto{\pgfqpoint{1.393875in}{2.476509in}}{\pgfqpoint{1.398266in}{2.465910in}}{\pgfqpoint{1.406079in}{2.458097in}}%
\pgfpathcurveto{\pgfqpoint{1.413893in}{2.450283in}}{\pgfqpoint{1.424492in}{2.445893in}}{\pgfqpoint{1.435542in}{2.445893in}}%
\pgfpathclose%
\pgfusepath{stroke,fill}%
\end{pgfscope}%
\begin{pgfscope}%
\pgfpathrectangle{\pgfqpoint{0.511823in}{0.504323in}}{\pgfqpoint{3.218177in}{3.225677in}} %
\pgfusepath{clip}%
\pgfsetbuttcap%
\pgfsetroundjoin%
\definecolor{currentfill}{rgb}{0.000000,0.000000,0.545098}%
\pgfsetfillcolor{currentfill}%
\pgfsetfillopacity{0.400000}%
\pgfsetlinewidth{0.501875pt}%
\definecolor{currentstroke}{rgb}{0.000000,0.000000,0.545098}%
\pgfsetstrokecolor{currentstroke}%
\pgfsetstrokeopacity{0.400000}%
\pgfsetdash{}{0pt}%
\pgfpathmoveto{\pgfqpoint{1.436179in}{2.464508in}}%
\pgfpathcurveto{\pgfqpoint{1.447229in}{2.464508in}}{\pgfqpoint{1.457828in}{2.468899in}}{\pgfqpoint{1.465642in}{2.476712in}}%
\pgfpathcurveto{\pgfqpoint{1.473455in}{2.484526in}}{\pgfqpoint{1.477845in}{2.495125in}}{\pgfqpoint{1.477845in}{2.506175in}}%
\pgfpathcurveto{\pgfqpoint{1.477845in}{2.517225in}}{\pgfqpoint{1.473455in}{2.527824in}}{\pgfqpoint{1.465642in}{2.535638in}}%
\pgfpathcurveto{\pgfqpoint{1.457828in}{2.543451in}}{\pgfqpoint{1.447229in}{2.547842in}}{\pgfqpoint{1.436179in}{2.547842in}}%
\pgfpathcurveto{\pgfqpoint{1.425129in}{2.547842in}}{\pgfqpoint{1.414530in}{2.543451in}}{\pgfqpoint{1.406716in}{2.535638in}}%
\pgfpathcurveto{\pgfqpoint{1.398902in}{2.527824in}}{\pgfqpoint{1.394512in}{2.517225in}}{\pgfqpoint{1.394512in}{2.506175in}}%
\pgfpathcurveto{\pgfqpoint{1.394512in}{2.495125in}}{\pgfqpoint{1.398902in}{2.484526in}}{\pgfqpoint{1.406716in}{2.476712in}}%
\pgfpathcurveto{\pgfqpoint{1.414530in}{2.468899in}}{\pgfqpoint{1.425129in}{2.464508in}}{\pgfqpoint{1.436179in}{2.464508in}}%
\pgfpathclose%
\pgfusepath{stroke,fill}%
\end{pgfscope}%
\begin{pgfscope}%
\pgfpathrectangle{\pgfqpoint{0.511823in}{0.504323in}}{\pgfqpoint{3.218177in}{3.225677in}} %
\pgfusepath{clip}%
\pgfsetbuttcap%
\pgfsetroundjoin%
\definecolor{currentfill}{rgb}{0.000000,0.000000,0.545098}%
\pgfsetfillcolor{currentfill}%
\pgfsetfillopacity{0.400000}%
\pgfsetlinewidth{0.501875pt}%
\definecolor{currentstroke}{rgb}{0.000000,0.000000,0.545098}%
\pgfsetstrokecolor{currentstroke}%
\pgfsetstrokeopacity{0.400000}%
\pgfsetdash{}{0pt}%
\pgfpathmoveto{\pgfqpoint{1.473606in}{2.573831in}}%
\pgfpathcurveto{\pgfqpoint{1.484656in}{2.573831in}}{\pgfqpoint{1.495255in}{2.578222in}}{\pgfqpoint{1.503069in}{2.586035in}}%
\pgfpathcurveto{\pgfqpoint{1.510882in}{2.593849in}}{\pgfqpoint{1.515273in}{2.604448in}}{\pgfqpoint{1.515273in}{2.615498in}}%
\pgfpathcurveto{\pgfqpoint{1.515273in}{2.626548in}}{\pgfqpoint{1.510882in}{2.637147in}}{\pgfqpoint{1.503069in}{2.644961in}}%
\pgfpathcurveto{\pgfqpoint{1.495255in}{2.652774in}}{\pgfqpoint{1.484656in}{2.657165in}}{\pgfqpoint{1.473606in}{2.657165in}}%
\pgfpathcurveto{\pgfqpoint{1.462556in}{2.657165in}}{\pgfqpoint{1.451957in}{2.652774in}}{\pgfqpoint{1.444143in}{2.644961in}}%
\pgfpathcurveto{\pgfqpoint{1.436330in}{2.637147in}}{\pgfqpoint{1.431939in}{2.626548in}}{\pgfqpoint{1.431939in}{2.615498in}}%
\pgfpathcurveto{\pgfqpoint{1.431939in}{2.604448in}}{\pgfqpoint{1.436330in}{2.593849in}}{\pgfqpoint{1.444143in}{2.586035in}}%
\pgfpathcurveto{\pgfqpoint{1.451957in}{2.578222in}}{\pgfqpoint{1.462556in}{2.573831in}}{\pgfqpoint{1.473606in}{2.573831in}}%
\pgfpathclose%
\pgfusepath{stroke,fill}%
\end{pgfscope}%
\begin{pgfscope}%
\pgfpathrectangle{\pgfqpoint{0.511823in}{0.504323in}}{\pgfqpoint{3.218177in}{3.225677in}} %
\pgfusepath{clip}%
\pgfsetbuttcap%
\pgfsetroundjoin%
\definecolor{currentfill}{rgb}{0.000000,0.000000,0.545098}%
\pgfsetfillcolor{currentfill}%
\pgfsetfillopacity{0.400000}%
\pgfsetlinewidth{0.501875pt}%
\definecolor{currentstroke}{rgb}{0.000000,0.000000,0.545098}%
\pgfsetstrokecolor{currentstroke}%
\pgfsetstrokeopacity{0.400000}%
\pgfsetdash{}{0pt}%
\pgfpathmoveto{\pgfqpoint{1.395318in}{2.398170in}}%
\pgfpathcurveto{\pgfqpoint{1.406368in}{2.398170in}}{\pgfqpoint{1.416967in}{2.402560in}}{\pgfqpoint{1.424780in}{2.410374in}}%
\pgfpathcurveto{\pgfqpoint{1.432594in}{2.418187in}}{\pgfqpoint{1.436984in}{2.428786in}}{\pgfqpoint{1.436984in}{2.439836in}}%
\pgfpathcurveto{\pgfqpoint{1.436984in}{2.450886in}}{\pgfqpoint{1.432594in}{2.461485in}}{\pgfqpoint{1.424780in}{2.469299in}}%
\pgfpathcurveto{\pgfqpoint{1.416967in}{2.477113in}}{\pgfqpoint{1.406368in}{2.481503in}}{\pgfqpoint{1.395318in}{2.481503in}}%
\pgfpathcurveto{\pgfqpoint{1.384268in}{2.481503in}}{\pgfqpoint{1.373668in}{2.477113in}}{\pgfqpoint{1.365855in}{2.469299in}}%
\pgfpathcurveto{\pgfqpoint{1.358041in}{2.461485in}}{\pgfqpoint{1.353651in}{2.450886in}}{\pgfqpoint{1.353651in}{2.439836in}}%
\pgfpathcurveto{\pgfqpoint{1.353651in}{2.428786in}}{\pgfqpoint{1.358041in}{2.418187in}}{\pgfqpoint{1.365855in}{2.410374in}}%
\pgfpathcurveto{\pgfqpoint{1.373668in}{2.402560in}}{\pgfqpoint{1.384268in}{2.398170in}}{\pgfqpoint{1.395318in}{2.398170in}}%
\pgfpathclose%
\pgfusepath{stroke,fill}%
\end{pgfscope}%
\begin{pgfscope}%
\pgfpathrectangle{\pgfqpoint{0.511823in}{0.504323in}}{\pgfqpoint{3.218177in}{3.225677in}} %
\pgfusepath{clip}%
\pgfsetbuttcap%
\pgfsetroundjoin%
\definecolor{currentfill}{rgb}{0.000000,0.000000,0.545098}%
\pgfsetfillcolor{currentfill}%
\pgfsetfillopacity{0.400000}%
\pgfsetlinewidth{0.501875pt}%
\definecolor{currentstroke}{rgb}{0.000000,0.000000,0.545098}%
\pgfsetstrokecolor{currentstroke}%
\pgfsetstrokeopacity{0.400000}%
\pgfsetdash{}{0pt}%
\pgfpathmoveto{\pgfqpoint{1.396815in}{2.418896in}}%
\pgfpathcurveto{\pgfqpoint{1.407865in}{2.418896in}}{\pgfqpoint{1.418464in}{2.423286in}}{\pgfqpoint{1.426278in}{2.431099in}}%
\pgfpathcurveto{\pgfqpoint{1.434091in}{2.438913in}}{\pgfqpoint{1.438482in}{2.449512in}}{\pgfqpoint{1.438482in}{2.460562in}}%
\pgfpathcurveto{\pgfqpoint{1.438482in}{2.471612in}}{\pgfqpoint{1.434091in}{2.482211in}}{\pgfqpoint{1.426278in}{2.490025in}}%
\pgfpathcurveto{\pgfqpoint{1.418464in}{2.497839in}}{\pgfqpoint{1.407865in}{2.502229in}}{\pgfqpoint{1.396815in}{2.502229in}}%
\pgfpathcurveto{\pgfqpoint{1.385765in}{2.502229in}}{\pgfqpoint{1.375166in}{2.497839in}}{\pgfqpoint{1.367352in}{2.490025in}}%
\pgfpathcurveto{\pgfqpoint{1.359539in}{2.482211in}}{\pgfqpoint{1.355148in}{2.471612in}}{\pgfqpoint{1.355148in}{2.460562in}}%
\pgfpathcurveto{\pgfqpoint{1.355148in}{2.449512in}}{\pgfqpoint{1.359539in}{2.438913in}}{\pgfqpoint{1.367352in}{2.431099in}}%
\pgfpathcurveto{\pgfqpoint{1.375166in}{2.423286in}}{\pgfqpoint{1.385765in}{2.418896in}}{\pgfqpoint{1.396815in}{2.418896in}}%
\pgfpathclose%
\pgfusepath{stroke,fill}%
\end{pgfscope}%
\begin{pgfscope}%
\pgfpathrectangle{\pgfqpoint{0.511823in}{0.504323in}}{\pgfqpoint{3.218177in}{3.225677in}} %
\pgfusepath{clip}%
\pgfsetbuttcap%
\pgfsetroundjoin%
\definecolor{currentfill}{rgb}{0.000000,0.000000,0.545098}%
\pgfsetfillcolor{currentfill}%
\pgfsetfillopacity{0.400000}%
\pgfsetlinewidth{0.501875pt}%
\definecolor{currentstroke}{rgb}{0.000000,0.000000,0.545098}%
\pgfsetstrokecolor{currentstroke}%
\pgfsetstrokeopacity{0.400000}%
\pgfsetdash{}{0pt}%
\pgfpathmoveto{\pgfqpoint{1.363225in}{2.351252in}}%
\pgfpathcurveto{\pgfqpoint{1.374275in}{2.351252in}}{\pgfqpoint{1.384874in}{2.355642in}}{\pgfqpoint{1.392687in}{2.363456in}}%
\pgfpathcurveto{\pgfqpoint{1.400501in}{2.371270in}}{\pgfqpoint{1.404891in}{2.381869in}}{\pgfqpoint{1.404891in}{2.392919in}}%
\pgfpathcurveto{\pgfqpoint{1.404891in}{2.403969in}}{\pgfqpoint{1.400501in}{2.414568in}}{\pgfqpoint{1.392687in}{2.422382in}}%
\pgfpathcurveto{\pgfqpoint{1.384874in}{2.430195in}}{\pgfqpoint{1.374275in}{2.434586in}}{\pgfqpoint{1.363225in}{2.434586in}}%
\pgfpathcurveto{\pgfqpoint{1.352174in}{2.434586in}}{\pgfqpoint{1.341575in}{2.430195in}}{\pgfqpoint{1.333762in}{2.422382in}}%
\pgfpathcurveto{\pgfqpoint{1.325948in}{2.414568in}}{\pgfqpoint{1.321558in}{2.403969in}}{\pgfqpoint{1.321558in}{2.392919in}}%
\pgfpathcurveto{\pgfqpoint{1.321558in}{2.381869in}}{\pgfqpoint{1.325948in}{2.371270in}}{\pgfqpoint{1.333762in}{2.363456in}}%
\pgfpathcurveto{\pgfqpoint{1.341575in}{2.355642in}}{\pgfqpoint{1.352174in}{2.351252in}}{\pgfqpoint{1.363225in}{2.351252in}}%
\pgfpathclose%
\pgfusepath{stroke,fill}%
\end{pgfscope}%
\begin{pgfscope}%
\pgfpathrectangle{\pgfqpoint{0.511823in}{0.504323in}}{\pgfqpoint{3.218177in}{3.225677in}} %
\pgfusepath{clip}%
\pgfsetbuttcap%
\pgfsetroundjoin%
\definecolor{currentfill}{rgb}{0.000000,0.000000,0.545098}%
\pgfsetfillcolor{currentfill}%
\pgfsetfillopacity{0.400000}%
\pgfsetlinewidth{0.501875pt}%
\definecolor{currentstroke}{rgb}{0.000000,0.000000,0.545098}%
\pgfsetstrokecolor{currentstroke}%
\pgfsetstrokeopacity{0.400000}%
\pgfsetdash{}{0pt}%
\pgfpathmoveto{\pgfqpoint{1.301627in}{2.210722in}}%
\pgfpathcurveto{\pgfqpoint{1.312677in}{2.210722in}}{\pgfqpoint{1.323276in}{2.215113in}}{\pgfqpoint{1.331089in}{2.222926in}}%
\pgfpathcurveto{\pgfqpoint{1.338903in}{2.230740in}}{\pgfqpoint{1.343293in}{2.241339in}}{\pgfqpoint{1.343293in}{2.252389in}}%
\pgfpathcurveto{\pgfqpoint{1.343293in}{2.263439in}}{\pgfqpoint{1.338903in}{2.274038in}}{\pgfqpoint{1.331089in}{2.281852in}}%
\pgfpathcurveto{\pgfqpoint{1.323276in}{2.289665in}}{\pgfqpoint{1.312677in}{2.294056in}}{\pgfqpoint{1.301627in}{2.294056in}}%
\pgfpathcurveto{\pgfqpoint{1.290576in}{2.294056in}}{\pgfqpoint{1.279977in}{2.289665in}}{\pgfqpoint{1.272164in}{2.281852in}}%
\pgfpathcurveto{\pgfqpoint{1.264350in}{2.274038in}}{\pgfqpoint{1.259960in}{2.263439in}}{\pgfqpoint{1.259960in}{2.252389in}}%
\pgfpathcurveto{\pgfqpoint{1.259960in}{2.241339in}}{\pgfqpoint{1.264350in}{2.230740in}}{\pgfqpoint{1.272164in}{2.222926in}}%
\pgfpathcurveto{\pgfqpoint{1.279977in}{2.215113in}}{\pgfqpoint{1.290576in}{2.210722in}}{\pgfqpoint{1.301627in}{2.210722in}}%
\pgfpathclose%
\pgfusepath{stroke,fill}%
\end{pgfscope}%
\begin{pgfscope}%
\pgfpathrectangle{\pgfqpoint{0.511823in}{0.504323in}}{\pgfqpoint{3.218177in}{3.225677in}} %
\pgfusepath{clip}%
\pgfsetbuttcap%
\pgfsetroundjoin%
\definecolor{currentfill}{rgb}{0.000000,0.000000,0.545098}%
\pgfsetfillcolor{currentfill}%
\pgfsetfillopacity{0.400000}%
\pgfsetlinewidth{0.501875pt}%
\definecolor{currentstroke}{rgb}{0.000000,0.000000,0.545098}%
\pgfsetstrokecolor{currentstroke}%
\pgfsetstrokeopacity{0.400000}%
\pgfsetdash{}{0pt}%
\pgfpathmoveto{\pgfqpoint{1.355298in}{2.364640in}}%
\pgfpathcurveto{\pgfqpoint{1.366348in}{2.364640in}}{\pgfqpoint{1.376947in}{2.369030in}}{\pgfqpoint{1.384761in}{2.376844in}}%
\pgfpathcurveto{\pgfqpoint{1.392574in}{2.384657in}}{\pgfqpoint{1.396965in}{2.395257in}}{\pgfqpoint{1.396965in}{2.406307in}}%
\pgfpathcurveto{\pgfqpoint{1.396965in}{2.417357in}}{\pgfqpoint{1.392574in}{2.427956in}}{\pgfqpoint{1.384761in}{2.435769in}}%
\pgfpathcurveto{\pgfqpoint{1.376947in}{2.443583in}}{\pgfqpoint{1.366348in}{2.447973in}}{\pgfqpoint{1.355298in}{2.447973in}}%
\pgfpathcurveto{\pgfqpoint{1.344248in}{2.447973in}}{\pgfqpoint{1.333649in}{2.443583in}}{\pgfqpoint{1.325835in}{2.435769in}}%
\pgfpathcurveto{\pgfqpoint{1.318022in}{2.427956in}}{\pgfqpoint{1.313631in}{2.417357in}}{\pgfqpoint{1.313631in}{2.406307in}}%
\pgfpathcurveto{\pgfqpoint{1.313631in}{2.395257in}}{\pgfqpoint{1.318022in}{2.384657in}}{\pgfqpoint{1.325835in}{2.376844in}}%
\pgfpathcurveto{\pgfqpoint{1.333649in}{2.369030in}}{\pgfqpoint{1.344248in}{2.364640in}}{\pgfqpoint{1.355298in}{2.364640in}}%
\pgfpathclose%
\pgfusepath{stroke,fill}%
\end{pgfscope}%
\begin{pgfscope}%
\pgfpathrectangle{\pgfqpoint{0.511823in}{0.504323in}}{\pgfqpoint{3.218177in}{3.225677in}} %
\pgfusepath{clip}%
\pgfsetbuttcap%
\pgfsetroundjoin%
\definecolor{currentfill}{rgb}{0.000000,0.000000,0.545098}%
\pgfsetfillcolor{currentfill}%
\pgfsetfillopacity{0.400000}%
\pgfsetlinewidth{0.501875pt}%
\definecolor{currentstroke}{rgb}{0.000000,0.000000,0.545098}%
\pgfsetstrokecolor{currentstroke}%
\pgfsetstrokeopacity{0.400000}%
\pgfsetdash{}{0pt}%
\pgfpathmoveto{\pgfqpoint{1.428078in}{2.571352in}}%
\pgfpathcurveto{\pgfqpoint{1.439128in}{2.571352in}}{\pgfqpoint{1.449727in}{2.575743in}}{\pgfqpoint{1.457541in}{2.583556in}}%
\pgfpathcurveto{\pgfqpoint{1.465354in}{2.591370in}}{\pgfqpoint{1.469745in}{2.601969in}}{\pgfqpoint{1.469745in}{2.613019in}}%
\pgfpathcurveto{\pgfqpoint{1.469745in}{2.624069in}}{\pgfqpoint{1.465354in}{2.634668in}}{\pgfqpoint{1.457541in}{2.642482in}}%
\pgfpathcurveto{\pgfqpoint{1.449727in}{2.650295in}}{\pgfqpoint{1.439128in}{2.654686in}}{\pgfqpoint{1.428078in}{2.654686in}}%
\pgfpathcurveto{\pgfqpoint{1.417028in}{2.654686in}}{\pgfqpoint{1.406429in}{2.650295in}}{\pgfqpoint{1.398615in}{2.642482in}}%
\pgfpathcurveto{\pgfqpoint{1.390801in}{2.634668in}}{\pgfqpoint{1.386411in}{2.624069in}}{\pgfqpoint{1.386411in}{2.613019in}}%
\pgfpathcurveto{\pgfqpoint{1.386411in}{2.601969in}}{\pgfqpoint{1.390801in}{2.591370in}}{\pgfqpoint{1.398615in}{2.583556in}}%
\pgfpathcurveto{\pgfqpoint{1.406429in}{2.575743in}}{\pgfqpoint{1.417028in}{2.571352in}}{\pgfqpoint{1.428078in}{2.571352in}}%
\pgfpathclose%
\pgfusepath{stroke,fill}%
\end{pgfscope}%
\begin{pgfscope}%
\pgfpathrectangle{\pgfqpoint{0.511823in}{0.504323in}}{\pgfqpoint{3.218177in}{3.225677in}} %
\pgfusepath{clip}%
\pgfsetbuttcap%
\pgfsetroundjoin%
\definecolor{currentfill}{rgb}{0.000000,0.000000,0.545098}%
\pgfsetfillcolor{currentfill}%
\pgfsetfillopacity{0.400000}%
\pgfsetlinewidth{0.501875pt}%
\definecolor{currentstroke}{rgb}{0.000000,0.000000,0.545098}%
\pgfsetstrokecolor{currentstroke}%
\pgfsetstrokeopacity{0.400000}%
\pgfsetdash{}{0pt}%
\pgfpathmoveto{\pgfqpoint{1.273429in}{2.183897in}}%
\pgfpathcurveto{\pgfqpoint{1.284480in}{2.183897in}}{\pgfqpoint{1.295079in}{2.188288in}}{\pgfqpoint{1.302892in}{2.196101in}}%
\pgfpathcurveto{\pgfqpoint{1.310706in}{2.203915in}}{\pgfqpoint{1.315096in}{2.214514in}}{\pgfqpoint{1.315096in}{2.225564in}}%
\pgfpathcurveto{\pgfqpoint{1.315096in}{2.236614in}}{\pgfqpoint{1.310706in}{2.247213in}}{\pgfqpoint{1.302892in}{2.255027in}}%
\pgfpathcurveto{\pgfqpoint{1.295079in}{2.262840in}}{\pgfqpoint{1.284480in}{2.267231in}}{\pgfqpoint{1.273429in}{2.267231in}}%
\pgfpathcurveto{\pgfqpoint{1.262379in}{2.267231in}}{\pgfqpoint{1.251780in}{2.262840in}}{\pgfqpoint{1.243967in}{2.255027in}}%
\pgfpathcurveto{\pgfqpoint{1.236153in}{2.247213in}}{\pgfqpoint{1.231763in}{2.236614in}}{\pgfqpoint{1.231763in}{2.225564in}}%
\pgfpathcurveto{\pgfqpoint{1.231763in}{2.214514in}}{\pgfqpoint{1.236153in}{2.203915in}}{\pgfqpoint{1.243967in}{2.196101in}}%
\pgfpathcurveto{\pgfqpoint{1.251780in}{2.188288in}}{\pgfqpoint{1.262379in}{2.183897in}}{\pgfqpoint{1.273429in}{2.183897in}}%
\pgfpathclose%
\pgfusepath{stroke,fill}%
\end{pgfscope}%
\begin{pgfscope}%
\pgfpathrectangle{\pgfqpoint{0.511823in}{0.504323in}}{\pgfqpoint{3.218177in}{3.225677in}} %
\pgfusepath{clip}%
\pgfsetbuttcap%
\pgfsetroundjoin%
\definecolor{currentfill}{rgb}{0.000000,0.000000,0.545098}%
\pgfsetfillcolor{currentfill}%
\pgfsetfillopacity{0.400000}%
\pgfsetlinewidth{0.501875pt}%
\definecolor{currentstroke}{rgb}{0.000000,0.000000,0.545098}%
\pgfsetstrokecolor{currentstroke}%
\pgfsetstrokeopacity{0.400000}%
\pgfsetdash{}{0pt}%
\pgfpathmoveto{\pgfqpoint{1.403023in}{2.543850in}}%
\pgfpathcurveto{\pgfqpoint{1.414073in}{2.543850in}}{\pgfqpoint{1.424672in}{2.548240in}}{\pgfqpoint{1.432485in}{2.556054in}}%
\pgfpathcurveto{\pgfqpoint{1.440299in}{2.563867in}}{\pgfqpoint{1.444689in}{2.574466in}}{\pgfqpoint{1.444689in}{2.585517in}}%
\pgfpathcurveto{\pgfqpoint{1.444689in}{2.596567in}}{\pgfqpoint{1.440299in}{2.607166in}}{\pgfqpoint{1.432485in}{2.614979in}}%
\pgfpathcurveto{\pgfqpoint{1.424672in}{2.622793in}}{\pgfqpoint{1.414073in}{2.627183in}}{\pgfqpoint{1.403023in}{2.627183in}}%
\pgfpathcurveto{\pgfqpoint{1.391972in}{2.627183in}}{\pgfqpoint{1.381373in}{2.622793in}}{\pgfqpoint{1.373560in}{2.614979in}}%
\pgfpathcurveto{\pgfqpoint{1.365746in}{2.607166in}}{\pgfqpoint{1.361356in}{2.596567in}}{\pgfqpoint{1.361356in}{2.585517in}}%
\pgfpathcurveto{\pgfqpoint{1.361356in}{2.574466in}}{\pgfqpoint{1.365746in}{2.563867in}}{\pgfqpoint{1.373560in}{2.556054in}}%
\pgfpathcurveto{\pgfqpoint{1.381373in}{2.548240in}}{\pgfqpoint{1.391972in}{2.543850in}}{\pgfqpoint{1.403023in}{2.543850in}}%
\pgfpathclose%
\pgfusepath{stroke,fill}%
\end{pgfscope}%
\begin{pgfscope}%
\pgfpathrectangle{\pgfqpoint{0.511823in}{0.504323in}}{\pgfqpoint{3.218177in}{3.225677in}} %
\pgfusepath{clip}%
\pgfsetbuttcap%
\pgfsetroundjoin%
\definecolor{currentfill}{rgb}{0.000000,0.000000,0.545098}%
\pgfsetfillcolor{currentfill}%
\pgfsetfillopacity{0.400000}%
\pgfsetlinewidth{0.501875pt}%
\definecolor{currentstroke}{rgb}{0.000000,0.000000,0.545098}%
\pgfsetstrokecolor{currentstroke}%
\pgfsetstrokeopacity{0.400000}%
\pgfsetdash{}{0pt}%
\pgfpathmoveto{\pgfqpoint{1.349547in}{2.419699in}}%
\pgfpathcurveto{\pgfqpoint{1.360597in}{2.419699in}}{\pgfqpoint{1.371196in}{2.424090in}}{\pgfqpoint{1.379010in}{2.431903in}}%
\pgfpathcurveto{\pgfqpoint{1.386823in}{2.439717in}}{\pgfqpoint{1.391214in}{2.450316in}}{\pgfqpoint{1.391214in}{2.461366in}}%
\pgfpathcurveto{\pgfqpoint{1.391214in}{2.472416in}}{\pgfqpoint{1.386823in}{2.483015in}}{\pgfqpoint{1.379010in}{2.490829in}}%
\pgfpathcurveto{\pgfqpoint{1.371196in}{2.498642in}}{\pgfqpoint{1.360597in}{2.503033in}}{\pgfqpoint{1.349547in}{2.503033in}}%
\pgfpathcurveto{\pgfqpoint{1.338497in}{2.503033in}}{\pgfqpoint{1.327898in}{2.498642in}}{\pgfqpoint{1.320084in}{2.490829in}}%
\pgfpathcurveto{\pgfqpoint{1.312271in}{2.483015in}}{\pgfqpoint{1.307880in}{2.472416in}}{\pgfqpoint{1.307880in}{2.461366in}}%
\pgfpathcurveto{\pgfqpoint{1.307880in}{2.450316in}}{\pgfqpoint{1.312271in}{2.439717in}}{\pgfqpoint{1.320084in}{2.431903in}}%
\pgfpathcurveto{\pgfqpoint{1.327898in}{2.424090in}}{\pgfqpoint{1.338497in}{2.419699in}}{\pgfqpoint{1.349547in}{2.419699in}}%
\pgfpathclose%
\pgfusepath{stroke,fill}%
\end{pgfscope}%
\begin{pgfscope}%
\pgfpathrectangle{\pgfqpoint{0.511823in}{0.504323in}}{\pgfqpoint{3.218177in}{3.225677in}} %
\pgfusepath{clip}%
\pgfsetbuttcap%
\pgfsetroundjoin%
\definecolor{currentfill}{rgb}{0.000000,0.000000,0.545098}%
\pgfsetfillcolor{currentfill}%
\pgfsetfillopacity{0.400000}%
\pgfsetlinewidth{0.501875pt}%
\definecolor{currentstroke}{rgb}{0.000000,0.000000,0.545098}%
\pgfsetstrokecolor{currentstroke}%
\pgfsetstrokeopacity{0.400000}%
\pgfsetdash{}{0pt}%
\pgfpathmoveto{\pgfqpoint{1.293059in}{2.284788in}}%
\pgfpathcurveto{\pgfqpoint{1.304109in}{2.284788in}}{\pgfqpoint{1.314708in}{2.289178in}}{\pgfqpoint{1.322522in}{2.296992in}}%
\pgfpathcurveto{\pgfqpoint{1.330336in}{2.304806in}}{\pgfqpoint{1.334726in}{2.315405in}}{\pgfqpoint{1.334726in}{2.326455in}}%
\pgfpathcurveto{\pgfqpoint{1.334726in}{2.337505in}}{\pgfqpoint{1.330336in}{2.348104in}}{\pgfqpoint{1.322522in}{2.355918in}}%
\pgfpathcurveto{\pgfqpoint{1.314708in}{2.363731in}}{\pgfqpoint{1.304109in}{2.368122in}}{\pgfqpoint{1.293059in}{2.368122in}}%
\pgfpathcurveto{\pgfqpoint{1.282009in}{2.368122in}}{\pgfqpoint{1.271410in}{2.363731in}}{\pgfqpoint{1.263596in}{2.355918in}}%
\pgfpathcurveto{\pgfqpoint{1.255783in}{2.348104in}}{\pgfqpoint{1.251393in}{2.337505in}}{\pgfqpoint{1.251393in}{2.326455in}}%
\pgfpathcurveto{\pgfqpoint{1.251393in}{2.315405in}}{\pgfqpoint{1.255783in}{2.304806in}}{\pgfqpoint{1.263596in}{2.296992in}}%
\pgfpathcurveto{\pgfqpoint{1.271410in}{2.289178in}}{\pgfqpoint{1.282009in}{2.284788in}}{\pgfqpoint{1.293059in}{2.284788in}}%
\pgfpathclose%
\pgfusepath{stroke,fill}%
\end{pgfscope}%
\begin{pgfscope}%
\pgfpathrectangle{\pgfqpoint{0.511823in}{0.504323in}}{\pgfqpoint{3.218177in}{3.225677in}} %
\pgfusepath{clip}%
\pgfsetbuttcap%
\pgfsetroundjoin%
\definecolor{currentfill}{rgb}{0.000000,0.000000,0.545098}%
\pgfsetfillcolor{currentfill}%
\pgfsetfillopacity{0.400000}%
\pgfsetlinewidth{0.501875pt}%
\definecolor{currentstroke}{rgb}{0.000000,0.000000,0.545098}%
\pgfsetstrokecolor{currentstroke}%
\pgfsetstrokeopacity{0.400000}%
\pgfsetdash{}{0pt}%
\pgfpathmoveto{\pgfqpoint{1.329126in}{2.400632in}}%
\pgfpathcurveto{\pgfqpoint{1.340176in}{2.400632in}}{\pgfqpoint{1.350775in}{2.405022in}}{\pgfqpoint{1.358589in}{2.412835in}}%
\pgfpathcurveto{\pgfqpoint{1.366403in}{2.420649in}}{\pgfqpoint{1.370793in}{2.431248in}}{\pgfqpoint{1.370793in}{2.442298in}}%
\pgfpathcurveto{\pgfqpoint{1.370793in}{2.453348in}}{\pgfqpoint{1.366403in}{2.463947in}}{\pgfqpoint{1.358589in}{2.471761in}}%
\pgfpathcurveto{\pgfqpoint{1.350775in}{2.479575in}}{\pgfqpoint{1.340176in}{2.483965in}}{\pgfqpoint{1.329126in}{2.483965in}}%
\pgfpathcurveto{\pgfqpoint{1.318076in}{2.483965in}}{\pgfqpoint{1.307477in}{2.479575in}}{\pgfqpoint{1.299663in}{2.471761in}}%
\pgfpathcurveto{\pgfqpoint{1.291850in}{2.463947in}}{\pgfqpoint{1.287459in}{2.453348in}}{\pgfqpoint{1.287459in}{2.442298in}}%
\pgfpathcurveto{\pgfqpoint{1.287459in}{2.431248in}}{\pgfqpoint{1.291850in}{2.420649in}}{\pgfqpoint{1.299663in}{2.412835in}}%
\pgfpathcurveto{\pgfqpoint{1.307477in}{2.405022in}}{\pgfqpoint{1.318076in}{2.400632in}}{\pgfqpoint{1.329126in}{2.400632in}}%
\pgfpathclose%
\pgfusepath{stroke,fill}%
\end{pgfscope}%
\begin{pgfscope}%
\pgfpathrectangle{\pgfqpoint{0.511823in}{0.504323in}}{\pgfqpoint{3.218177in}{3.225677in}} %
\pgfusepath{clip}%
\pgfsetbuttcap%
\pgfsetroundjoin%
\definecolor{currentfill}{rgb}{0.000000,0.000000,0.545098}%
\pgfsetfillcolor{currentfill}%
\pgfsetfillopacity{0.400000}%
\pgfsetlinewidth{0.501875pt}%
\definecolor{currentstroke}{rgb}{0.000000,0.000000,0.545098}%
\pgfsetstrokecolor{currentstroke}%
\pgfsetstrokeopacity{0.400000}%
\pgfsetdash{}{0pt}%
\pgfpathmoveto{\pgfqpoint{1.395473in}{2.602685in}}%
\pgfpathcurveto{\pgfqpoint{1.406523in}{2.602685in}}{\pgfqpoint{1.417122in}{2.607075in}}{\pgfqpoint{1.424936in}{2.614888in}}%
\pgfpathcurveto{\pgfqpoint{1.432749in}{2.622702in}}{\pgfqpoint{1.437140in}{2.633301in}}{\pgfqpoint{1.437140in}{2.644351in}}%
\pgfpathcurveto{\pgfqpoint{1.437140in}{2.655401in}}{\pgfqpoint{1.432749in}{2.666000in}}{\pgfqpoint{1.424936in}{2.673814in}}%
\pgfpathcurveto{\pgfqpoint{1.417122in}{2.681628in}}{\pgfqpoint{1.406523in}{2.686018in}}{\pgfqpoint{1.395473in}{2.686018in}}%
\pgfpathcurveto{\pgfqpoint{1.384423in}{2.686018in}}{\pgfqpoint{1.373824in}{2.681628in}}{\pgfqpoint{1.366010in}{2.673814in}}%
\pgfpathcurveto{\pgfqpoint{1.358197in}{2.666000in}}{\pgfqpoint{1.353806in}{2.655401in}}{\pgfqpoint{1.353806in}{2.644351in}}%
\pgfpathcurveto{\pgfqpoint{1.353806in}{2.633301in}}{\pgfqpoint{1.358197in}{2.622702in}}{\pgfqpoint{1.366010in}{2.614888in}}%
\pgfpathcurveto{\pgfqpoint{1.373824in}{2.607075in}}{\pgfqpoint{1.384423in}{2.602685in}}{\pgfqpoint{1.395473in}{2.602685in}}%
\pgfpathclose%
\pgfusepath{stroke,fill}%
\end{pgfscope}%
\begin{pgfscope}%
\pgfpathrectangle{\pgfqpoint{0.511823in}{0.504323in}}{\pgfqpoint{3.218177in}{3.225677in}} %
\pgfusepath{clip}%
\pgfsetbuttcap%
\pgfsetroundjoin%
\definecolor{currentfill}{rgb}{0.000000,0.000000,0.545098}%
\pgfsetfillcolor{currentfill}%
\pgfsetfillopacity{0.400000}%
\pgfsetlinewidth{0.501875pt}%
\definecolor{currentstroke}{rgb}{0.000000,0.000000,0.545098}%
\pgfsetstrokecolor{currentstroke}%
\pgfsetstrokeopacity{0.400000}%
\pgfsetdash{}{0pt}%
\pgfpathmoveto{\pgfqpoint{1.290462in}{2.329563in}}%
\pgfpathcurveto{\pgfqpoint{1.301512in}{2.329563in}}{\pgfqpoint{1.312111in}{2.333953in}}{\pgfqpoint{1.319925in}{2.341767in}}%
\pgfpathcurveto{\pgfqpoint{1.327738in}{2.349580in}}{\pgfqpoint{1.332129in}{2.360179in}}{\pgfqpoint{1.332129in}{2.371229in}}%
\pgfpathcurveto{\pgfqpoint{1.332129in}{2.382280in}}{\pgfqpoint{1.327738in}{2.392879in}}{\pgfqpoint{1.319925in}{2.400692in}}%
\pgfpathcurveto{\pgfqpoint{1.312111in}{2.408506in}}{\pgfqpoint{1.301512in}{2.412896in}}{\pgfqpoint{1.290462in}{2.412896in}}%
\pgfpathcurveto{\pgfqpoint{1.279412in}{2.412896in}}{\pgfqpoint{1.268813in}{2.408506in}}{\pgfqpoint{1.260999in}{2.400692in}}%
\pgfpathcurveto{\pgfqpoint{1.253185in}{2.392879in}}{\pgfqpoint{1.248795in}{2.382280in}}{\pgfqpoint{1.248795in}{2.371229in}}%
\pgfpathcurveto{\pgfqpoint{1.248795in}{2.360179in}}{\pgfqpoint{1.253185in}{2.349580in}}{\pgfqpoint{1.260999in}{2.341767in}}%
\pgfpathcurveto{\pgfqpoint{1.268813in}{2.333953in}}{\pgfqpoint{1.279412in}{2.329563in}}{\pgfqpoint{1.290462in}{2.329563in}}%
\pgfpathclose%
\pgfusepath{stroke,fill}%
\end{pgfscope}%
\begin{pgfscope}%
\pgfpathrectangle{\pgfqpoint{0.511823in}{0.504323in}}{\pgfqpoint{3.218177in}{3.225677in}} %
\pgfusepath{clip}%
\pgfsetbuttcap%
\pgfsetroundjoin%
\definecolor{currentfill}{rgb}{0.000000,0.000000,0.545098}%
\pgfsetfillcolor{currentfill}%
\pgfsetfillopacity{0.400000}%
\pgfsetlinewidth{0.501875pt}%
\definecolor{currentstroke}{rgb}{0.000000,0.000000,0.545098}%
\pgfsetstrokecolor{currentstroke}%
\pgfsetstrokeopacity{0.400000}%
\pgfsetdash{}{0pt}%
\pgfpathmoveto{\pgfqpoint{1.341298in}{2.491222in}}%
\pgfpathcurveto{\pgfqpoint{1.352348in}{2.491222in}}{\pgfqpoint{1.362947in}{2.495612in}}{\pgfqpoint{1.370761in}{2.503426in}}%
\pgfpathcurveto{\pgfqpoint{1.378574in}{2.511239in}}{\pgfqpoint{1.382964in}{2.521838in}}{\pgfqpoint{1.382964in}{2.532888in}}%
\pgfpathcurveto{\pgfqpoint{1.382964in}{2.543938in}}{\pgfqpoint{1.378574in}{2.554537in}}{\pgfqpoint{1.370761in}{2.562351in}}%
\pgfpathcurveto{\pgfqpoint{1.362947in}{2.570165in}}{\pgfqpoint{1.352348in}{2.574555in}}{\pgfqpoint{1.341298in}{2.574555in}}%
\pgfpathcurveto{\pgfqpoint{1.330248in}{2.574555in}}{\pgfqpoint{1.319649in}{2.570165in}}{\pgfqpoint{1.311835in}{2.562351in}}%
\pgfpathcurveto{\pgfqpoint{1.304021in}{2.554537in}}{\pgfqpoint{1.299631in}{2.543938in}}{\pgfqpoint{1.299631in}{2.532888in}}%
\pgfpathcurveto{\pgfqpoint{1.299631in}{2.521838in}}{\pgfqpoint{1.304021in}{2.511239in}}{\pgfqpoint{1.311835in}{2.503426in}}%
\pgfpathcurveto{\pgfqpoint{1.319649in}{2.495612in}}{\pgfqpoint{1.330248in}{2.491222in}}{\pgfqpoint{1.341298in}{2.491222in}}%
\pgfpathclose%
\pgfusepath{stroke,fill}%
\end{pgfscope}%
\begin{pgfscope}%
\pgfpathrectangle{\pgfqpoint{0.511823in}{0.504323in}}{\pgfqpoint{3.218177in}{3.225677in}} %
\pgfusepath{clip}%
\pgfsetbuttcap%
\pgfsetroundjoin%
\definecolor{currentfill}{rgb}{0.000000,0.000000,0.545098}%
\pgfsetfillcolor{currentfill}%
\pgfsetfillopacity{0.400000}%
\pgfsetlinewidth{0.501875pt}%
\definecolor{currentstroke}{rgb}{0.000000,0.000000,0.545098}%
\pgfsetstrokecolor{currentstroke}%
\pgfsetstrokeopacity{0.400000}%
\pgfsetdash{}{0pt}%
\pgfpathmoveto{\pgfqpoint{1.349034in}{2.533145in}}%
\pgfpathcurveto{\pgfqpoint{1.360084in}{2.533145in}}{\pgfqpoint{1.370683in}{2.537535in}}{\pgfqpoint{1.378497in}{2.545349in}}%
\pgfpathcurveto{\pgfqpoint{1.386310in}{2.553162in}}{\pgfqpoint{1.390701in}{2.563761in}}{\pgfqpoint{1.390701in}{2.574812in}}%
\pgfpathcurveto{\pgfqpoint{1.390701in}{2.585862in}}{\pgfqpoint{1.386310in}{2.596461in}}{\pgfqpoint{1.378497in}{2.604274in}}%
\pgfpathcurveto{\pgfqpoint{1.370683in}{2.612088in}}{\pgfqpoint{1.360084in}{2.616478in}}{\pgfqpoint{1.349034in}{2.616478in}}%
\pgfpathcurveto{\pgfqpoint{1.337984in}{2.616478in}}{\pgfqpoint{1.327385in}{2.612088in}}{\pgfqpoint{1.319571in}{2.604274in}}%
\pgfpathcurveto{\pgfqpoint{1.311757in}{2.596461in}}{\pgfqpoint{1.307367in}{2.585862in}}{\pgfqpoint{1.307367in}{2.574812in}}%
\pgfpathcurveto{\pgfqpoint{1.307367in}{2.563761in}}{\pgfqpoint{1.311757in}{2.553162in}}{\pgfqpoint{1.319571in}{2.545349in}}%
\pgfpathcurveto{\pgfqpoint{1.327385in}{2.537535in}}{\pgfqpoint{1.337984in}{2.533145in}}{\pgfqpoint{1.349034in}{2.533145in}}%
\pgfpathclose%
\pgfusepath{stroke,fill}%
\end{pgfscope}%
\begin{pgfscope}%
\pgfpathrectangle{\pgfqpoint{0.511823in}{0.504323in}}{\pgfqpoint{3.218177in}{3.225677in}} %
\pgfusepath{clip}%
\pgfsetbuttcap%
\pgfsetroundjoin%
\definecolor{currentfill}{rgb}{0.000000,0.000000,0.545098}%
\pgfsetfillcolor{currentfill}%
\pgfsetfillopacity{0.400000}%
\pgfsetlinewidth{0.501875pt}%
\definecolor{currentstroke}{rgb}{0.000000,0.000000,0.545098}%
\pgfsetstrokecolor{currentstroke}%
\pgfsetstrokeopacity{0.400000}%
\pgfsetdash{}{0pt}%
\pgfpathmoveto{\pgfqpoint{1.260107in}{2.296732in}}%
\pgfpathcurveto{\pgfqpoint{1.271157in}{2.296732in}}{\pgfqpoint{1.281756in}{2.301122in}}{\pgfqpoint{1.289570in}{2.308936in}}%
\pgfpathcurveto{\pgfqpoint{1.297383in}{2.316750in}}{\pgfqpoint{1.301773in}{2.327349in}}{\pgfqpoint{1.301773in}{2.338399in}}%
\pgfpathcurveto{\pgfqpoint{1.301773in}{2.349449in}}{\pgfqpoint{1.297383in}{2.360048in}}{\pgfqpoint{1.289570in}{2.367862in}}%
\pgfpathcurveto{\pgfqpoint{1.281756in}{2.375675in}}{\pgfqpoint{1.271157in}{2.380066in}}{\pgfqpoint{1.260107in}{2.380066in}}%
\pgfpathcurveto{\pgfqpoint{1.249057in}{2.380066in}}{\pgfqpoint{1.238458in}{2.375675in}}{\pgfqpoint{1.230644in}{2.367862in}}%
\pgfpathcurveto{\pgfqpoint{1.222830in}{2.360048in}}{\pgfqpoint{1.218440in}{2.349449in}}{\pgfqpoint{1.218440in}{2.338399in}}%
\pgfpathcurveto{\pgfqpoint{1.218440in}{2.327349in}}{\pgfqpoint{1.222830in}{2.316750in}}{\pgfqpoint{1.230644in}{2.308936in}}%
\pgfpathcurveto{\pgfqpoint{1.238458in}{2.301122in}}{\pgfqpoint{1.249057in}{2.296732in}}{\pgfqpoint{1.260107in}{2.296732in}}%
\pgfpathclose%
\pgfusepath{stroke,fill}%
\end{pgfscope}%
\begin{pgfscope}%
\pgfpathrectangle{\pgfqpoint{0.511823in}{0.504323in}}{\pgfqpoint{3.218177in}{3.225677in}} %
\pgfusepath{clip}%
\pgfsetbuttcap%
\pgfsetroundjoin%
\definecolor{currentfill}{rgb}{0.000000,0.000000,0.545098}%
\pgfsetfillcolor{currentfill}%
\pgfsetfillopacity{0.400000}%
\pgfsetlinewidth{0.501875pt}%
\definecolor{currentstroke}{rgb}{0.000000,0.000000,0.545098}%
\pgfsetstrokecolor{currentstroke}%
\pgfsetstrokeopacity{0.400000}%
\pgfsetdash{}{0pt}%
\pgfpathmoveto{\pgfqpoint{1.316118in}{2.478325in}}%
\pgfpathcurveto{\pgfqpoint{1.327168in}{2.478325in}}{\pgfqpoint{1.337767in}{2.482715in}}{\pgfqpoint{1.345581in}{2.490529in}}%
\pgfpathcurveto{\pgfqpoint{1.353394in}{2.498342in}}{\pgfqpoint{1.357785in}{2.508941in}}{\pgfqpoint{1.357785in}{2.519991in}}%
\pgfpathcurveto{\pgfqpoint{1.357785in}{2.531042in}}{\pgfqpoint{1.353394in}{2.541641in}}{\pgfqpoint{1.345581in}{2.549454in}}%
\pgfpathcurveto{\pgfqpoint{1.337767in}{2.557268in}}{\pgfqpoint{1.327168in}{2.561658in}}{\pgfqpoint{1.316118in}{2.561658in}}%
\pgfpathcurveto{\pgfqpoint{1.305068in}{2.561658in}}{\pgfqpoint{1.294469in}{2.557268in}}{\pgfqpoint{1.286655in}{2.549454in}}%
\pgfpathcurveto{\pgfqpoint{1.278842in}{2.541641in}}{\pgfqpoint{1.274451in}{2.531042in}}{\pgfqpoint{1.274451in}{2.519991in}}%
\pgfpathcurveto{\pgfqpoint{1.274451in}{2.508941in}}{\pgfqpoint{1.278842in}{2.498342in}}{\pgfqpoint{1.286655in}{2.490529in}}%
\pgfpathcurveto{\pgfqpoint{1.294469in}{2.482715in}}{\pgfqpoint{1.305068in}{2.478325in}}{\pgfqpoint{1.316118in}{2.478325in}}%
\pgfpathclose%
\pgfusepath{stroke,fill}%
\end{pgfscope}%
\begin{pgfscope}%
\pgfpathrectangle{\pgfqpoint{0.511823in}{0.504323in}}{\pgfqpoint{3.218177in}{3.225677in}} %
\pgfusepath{clip}%
\pgfsetbuttcap%
\pgfsetroundjoin%
\definecolor{currentfill}{rgb}{0.000000,0.000000,0.545098}%
\pgfsetfillcolor{currentfill}%
\pgfsetfillopacity{0.400000}%
\pgfsetlinewidth{0.501875pt}%
\definecolor{currentstroke}{rgb}{0.000000,0.000000,0.545098}%
\pgfsetstrokecolor{currentstroke}%
\pgfsetstrokeopacity{0.400000}%
\pgfsetdash{}{0pt}%
\pgfpathmoveto{\pgfqpoint{1.306004in}{2.468675in}}%
\pgfpathcurveto{\pgfqpoint{1.317054in}{2.468675in}}{\pgfqpoint{1.327653in}{2.473065in}}{\pgfqpoint{1.335467in}{2.480879in}}%
\pgfpathcurveto{\pgfqpoint{1.343280in}{2.488692in}}{\pgfqpoint{1.347671in}{2.499291in}}{\pgfqpoint{1.347671in}{2.510341in}}%
\pgfpathcurveto{\pgfqpoint{1.347671in}{2.521392in}}{\pgfqpoint{1.343280in}{2.531991in}}{\pgfqpoint{1.335467in}{2.539804in}}%
\pgfpathcurveto{\pgfqpoint{1.327653in}{2.547618in}}{\pgfqpoint{1.317054in}{2.552008in}}{\pgfqpoint{1.306004in}{2.552008in}}%
\pgfpathcurveto{\pgfqpoint{1.294954in}{2.552008in}}{\pgfqpoint{1.284355in}{2.547618in}}{\pgfqpoint{1.276541in}{2.539804in}}%
\pgfpathcurveto{\pgfqpoint{1.268728in}{2.531991in}}{\pgfqpoint{1.264337in}{2.521392in}}{\pgfqpoint{1.264337in}{2.510341in}}%
\pgfpathcurveto{\pgfqpoint{1.264337in}{2.499291in}}{\pgfqpoint{1.268728in}{2.488692in}}{\pgfqpoint{1.276541in}{2.480879in}}%
\pgfpathcurveto{\pgfqpoint{1.284355in}{2.473065in}}{\pgfqpoint{1.294954in}{2.468675in}}{\pgfqpoint{1.306004in}{2.468675in}}%
\pgfpathclose%
\pgfusepath{stroke,fill}%
\end{pgfscope}%
\begin{pgfscope}%
\pgfpathrectangle{\pgfqpoint{0.511823in}{0.504323in}}{\pgfqpoint{3.218177in}{3.225677in}} %
\pgfusepath{clip}%
\pgfsetbuttcap%
\pgfsetroundjoin%
\definecolor{currentfill}{rgb}{0.000000,0.000000,0.545098}%
\pgfsetfillcolor{currentfill}%
\pgfsetfillopacity{0.400000}%
\pgfsetlinewidth{0.501875pt}%
\definecolor{currentstroke}{rgb}{0.000000,0.000000,0.545098}%
\pgfsetstrokecolor{currentstroke}%
\pgfsetstrokeopacity{0.400000}%
\pgfsetdash{}{0pt}%
\pgfpathmoveto{\pgfqpoint{1.259963in}{2.351540in}}%
\pgfpathcurveto{\pgfqpoint{1.271013in}{2.351540in}}{\pgfqpoint{1.281612in}{2.355930in}}{\pgfqpoint{1.289426in}{2.363744in}}%
\pgfpathcurveto{\pgfqpoint{1.297239in}{2.371557in}}{\pgfqpoint{1.301630in}{2.382157in}}{\pgfqpoint{1.301630in}{2.393207in}}%
\pgfpathcurveto{\pgfqpoint{1.301630in}{2.404257in}}{\pgfqpoint{1.297239in}{2.414856in}}{\pgfqpoint{1.289426in}{2.422669in}}%
\pgfpathcurveto{\pgfqpoint{1.281612in}{2.430483in}}{\pgfqpoint{1.271013in}{2.434873in}}{\pgfqpoint{1.259963in}{2.434873in}}%
\pgfpathcurveto{\pgfqpoint{1.248913in}{2.434873in}}{\pgfqpoint{1.238314in}{2.430483in}}{\pgfqpoint{1.230500in}{2.422669in}}%
\pgfpathcurveto{\pgfqpoint{1.222687in}{2.414856in}}{\pgfqpoint{1.218296in}{2.404257in}}{\pgfqpoint{1.218296in}{2.393207in}}%
\pgfpathcurveto{\pgfqpoint{1.218296in}{2.382157in}}{\pgfqpoint{1.222687in}{2.371557in}}{\pgfqpoint{1.230500in}{2.363744in}}%
\pgfpathcurveto{\pgfqpoint{1.238314in}{2.355930in}}{\pgfqpoint{1.248913in}{2.351540in}}{\pgfqpoint{1.259963in}{2.351540in}}%
\pgfpathclose%
\pgfusepath{stroke,fill}%
\end{pgfscope}%
\begin{pgfscope}%
\pgfpathrectangle{\pgfqpoint{0.511823in}{0.504323in}}{\pgfqpoint{3.218177in}{3.225677in}} %
\pgfusepath{clip}%
\pgfsetbuttcap%
\pgfsetroundjoin%
\definecolor{currentfill}{rgb}{0.000000,0.000000,0.545098}%
\pgfsetfillcolor{currentfill}%
\pgfsetfillopacity{0.400000}%
\pgfsetlinewidth{0.501875pt}%
\definecolor{currentstroke}{rgb}{0.000000,0.000000,0.545098}%
\pgfsetstrokecolor{currentstroke}%
\pgfsetstrokeopacity{0.400000}%
\pgfsetdash{}{0pt}%
\pgfpathmoveto{\pgfqpoint{1.299473in}{2.489942in}}%
\pgfpathcurveto{\pgfqpoint{1.310523in}{2.489942in}}{\pgfqpoint{1.321122in}{2.494332in}}{\pgfqpoint{1.328936in}{2.502146in}}%
\pgfpathcurveto{\pgfqpoint{1.336749in}{2.509959in}}{\pgfqpoint{1.341139in}{2.520558in}}{\pgfqpoint{1.341139in}{2.531608in}}%
\pgfpathcurveto{\pgfqpoint{1.341139in}{2.542659in}}{\pgfqpoint{1.336749in}{2.553258in}}{\pgfqpoint{1.328936in}{2.561071in}}%
\pgfpathcurveto{\pgfqpoint{1.321122in}{2.568885in}}{\pgfqpoint{1.310523in}{2.573275in}}{\pgfqpoint{1.299473in}{2.573275in}}%
\pgfpathcurveto{\pgfqpoint{1.288423in}{2.573275in}}{\pgfqpoint{1.277824in}{2.568885in}}{\pgfqpoint{1.270010in}{2.561071in}}%
\pgfpathcurveto{\pgfqpoint{1.262196in}{2.553258in}}{\pgfqpoint{1.257806in}{2.542659in}}{\pgfqpoint{1.257806in}{2.531608in}}%
\pgfpathcurveto{\pgfqpoint{1.257806in}{2.520558in}}{\pgfqpoint{1.262196in}{2.509959in}}{\pgfqpoint{1.270010in}{2.502146in}}%
\pgfpathcurveto{\pgfqpoint{1.277824in}{2.494332in}}{\pgfqpoint{1.288423in}{2.489942in}}{\pgfqpoint{1.299473in}{2.489942in}}%
\pgfpathclose%
\pgfusepath{stroke,fill}%
\end{pgfscope}%
\begin{pgfscope}%
\pgfpathrectangle{\pgfqpoint{0.511823in}{0.504323in}}{\pgfqpoint{3.218177in}{3.225677in}} %
\pgfusepath{clip}%
\pgfsetbuttcap%
\pgfsetroundjoin%
\definecolor{currentfill}{rgb}{0.000000,0.000000,0.545098}%
\pgfsetfillcolor{currentfill}%
\pgfsetfillopacity{0.400000}%
\pgfsetlinewidth{0.501875pt}%
\definecolor{currentstroke}{rgb}{0.000000,0.000000,0.545098}%
\pgfsetstrokecolor{currentstroke}%
\pgfsetstrokeopacity{0.400000}%
\pgfsetdash{}{0pt}%
\pgfpathmoveto{\pgfqpoint{1.252739in}{2.368203in}}%
\pgfpathcurveto{\pgfqpoint{1.263789in}{2.368203in}}{\pgfqpoint{1.274388in}{2.372593in}}{\pgfqpoint{1.282202in}{2.380407in}}%
\pgfpathcurveto{\pgfqpoint{1.290015in}{2.388220in}}{\pgfqpoint{1.294406in}{2.398820in}}{\pgfqpoint{1.294406in}{2.409870in}}%
\pgfpathcurveto{\pgfqpoint{1.294406in}{2.420920in}}{\pgfqpoint{1.290015in}{2.431519in}}{\pgfqpoint{1.282202in}{2.439332in}}%
\pgfpathcurveto{\pgfqpoint{1.274388in}{2.447146in}}{\pgfqpoint{1.263789in}{2.451536in}}{\pgfqpoint{1.252739in}{2.451536in}}%
\pgfpathcurveto{\pgfqpoint{1.241689in}{2.451536in}}{\pgfqpoint{1.231090in}{2.447146in}}{\pgfqpoint{1.223276in}{2.439332in}}%
\pgfpathcurveto{\pgfqpoint{1.215463in}{2.431519in}}{\pgfqpoint{1.211072in}{2.420920in}}{\pgfqpoint{1.211072in}{2.409870in}}%
\pgfpathcurveto{\pgfqpoint{1.211072in}{2.398820in}}{\pgfqpoint{1.215463in}{2.388220in}}{\pgfqpoint{1.223276in}{2.380407in}}%
\pgfpathcurveto{\pgfqpoint{1.231090in}{2.372593in}}{\pgfqpoint{1.241689in}{2.368203in}}{\pgfqpoint{1.252739in}{2.368203in}}%
\pgfpathclose%
\pgfusepath{stroke,fill}%
\end{pgfscope}%
\begin{pgfscope}%
\pgfpathrectangle{\pgfqpoint{0.511823in}{0.504323in}}{\pgfqpoint{3.218177in}{3.225677in}} %
\pgfusepath{clip}%
\pgfsetbuttcap%
\pgfsetroundjoin%
\definecolor{currentfill}{rgb}{0.000000,0.000000,0.545098}%
\pgfsetfillcolor{currentfill}%
\pgfsetfillopacity{0.400000}%
\pgfsetlinewidth{0.501875pt}%
\definecolor{currentstroke}{rgb}{0.000000,0.000000,0.545098}%
\pgfsetstrokecolor{currentstroke}%
\pgfsetstrokeopacity{0.400000}%
\pgfsetdash{}{0pt}%
\pgfpathmoveto{\pgfqpoint{1.238610in}{2.344297in}}%
\pgfpathcurveto{\pgfqpoint{1.249660in}{2.344297in}}{\pgfqpoint{1.260259in}{2.348687in}}{\pgfqpoint{1.268073in}{2.356501in}}%
\pgfpathcurveto{\pgfqpoint{1.275886in}{2.364315in}}{\pgfqpoint{1.280277in}{2.374914in}}{\pgfqpoint{1.280277in}{2.385964in}}%
\pgfpathcurveto{\pgfqpoint{1.280277in}{2.397014in}}{\pgfqpoint{1.275886in}{2.407613in}}{\pgfqpoint{1.268073in}{2.415427in}}%
\pgfpathcurveto{\pgfqpoint{1.260259in}{2.423240in}}{\pgfqpoint{1.249660in}{2.427631in}}{\pgfqpoint{1.238610in}{2.427631in}}%
\pgfpathcurveto{\pgfqpoint{1.227560in}{2.427631in}}{\pgfqpoint{1.216961in}{2.423240in}}{\pgfqpoint{1.209147in}{2.415427in}}%
\pgfpathcurveto{\pgfqpoint{1.201334in}{2.407613in}}{\pgfqpoint{1.196943in}{2.397014in}}{\pgfqpoint{1.196943in}{2.385964in}}%
\pgfpathcurveto{\pgfqpoint{1.196943in}{2.374914in}}{\pgfqpoint{1.201334in}{2.364315in}}{\pgfqpoint{1.209147in}{2.356501in}}%
\pgfpathcurveto{\pgfqpoint{1.216961in}{2.348687in}}{\pgfqpoint{1.227560in}{2.344297in}}{\pgfqpoint{1.238610in}{2.344297in}}%
\pgfpathclose%
\pgfusepath{stroke,fill}%
\end{pgfscope}%
\begin{pgfscope}%
\pgfpathrectangle{\pgfqpoint{0.511823in}{0.504323in}}{\pgfqpoint{3.218177in}{3.225677in}} %
\pgfusepath{clip}%
\pgfsetbuttcap%
\pgfsetroundjoin%
\definecolor{currentfill}{rgb}{0.000000,0.000000,0.545098}%
\pgfsetfillcolor{currentfill}%
\pgfsetfillopacity{0.400000}%
\pgfsetlinewidth{0.501875pt}%
\definecolor{currentstroke}{rgb}{0.000000,0.000000,0.545098}%
\pgfsetstrokecolor{currentstroke}%
\pgfsetstrokeopacity{0.400000}%
\pgfsetdash{}{0pt}%
\pgfpathmoveto{\pgfqpoint{1.207939in}{2.268150in}}%
\pgfpathcurveto{\pgfqpoint{1.218989in}{2.268150in}}{\pgfqpoint{1.229588in}{2.272540in}}{\pgfqpoint{1.237401in}{2.280353in}}%
\pgfpathcurveto{\pgfqpoint{1.245215in}{2.288167in}}{\pgfqpoint{1.249605in}{2.298766in}}{\pgfqpoint{1.249605in}{2.309816in}}%
\pgfpathcurveto{\pgfqpoint{1.249605in}{2.320866in}}{\pgfqpoint{1.245215in}{2.331465in}}{\pgfqpoint{1.237401in}{2.339279in}}%
\pgfpathcurveto{\pgfqpoint{1.229588in}{2.347093in}}{\pgfqpoint{1.218989in}{2.351483in}}{\pgfqpoint{1.207939in}{2.351483in}}%
\pgfpathcurveto{\pgfqpoint{1.196888in}{2.351483in}}{\pgfqpoint{1.186289in}{2.347093in}}{\pgfqpoint{1.178476in}{2.339279in}}%
\pgfpathcurveto{\pgfqpoint{1.170662in}{2.331465in}}{\pgfqpoint{1.166272in}{2.320866in}}{\pgfqpoint{1.166272in}{2.309816in}}%
\pgfpathcurveto{\pgfqpoint{1.166272in}{2.298766in}}{\pgfqpoint{1.170662in}{2.288167in}}{\pgfqpoint{1.178476in}{2.280353in}}%
\pgfpathcurveto{\pgfqpoint{1.186289in}{2.272540in}}{\pgfqpoint{1.196888in}{2.268150in}}{\pgfqpoint{1.207939in}{2.268150in}}%
\pgfpathclose%
\pgfusepath{stroke,fill}%
\end{pgfscope}%
\begin{pgfscope}%
\pgfpathrectangle{\pgfqpoint{0.511823in}{0.504323in}}{\pgfqpoint{3.218177in}{3.225677in}} %
\pgfusepath{clip}%
\pgfsetbuttcap%
\pgfsetroundjoin%
\definecolor{currentfill}{rgb}{0.000000,0.000000,0.545098}%
\pgfsetfillcolor{currentfill}%
\pgfsetfillopacity{0.400000}%
\pgfsetlinewidth{0.501875pt}%
\definecolor{currentstroke}{rgb}{0.000000,0.000000,0.545098}%
\pgfsetstrokecolor{currentstroke}%
\pgfsetstrokeopacity{0.400000}%
\pgfsetdash{}{0pt}%
\pgfpathmoveto{\pgfqpoint{1.263160in}{2.461513in}}%
\pgfpathcurveto{\pgfqpoint{1.274210in}{2.461513in}}{\pgfqpoint{1.284809in}{2.465904in}}{\pgfqpoint{1.292623in}{2.473717in}}%
\pgfpathcurveto{\pgfqpoint{1.300437in}{2.481531in}}{\pgfqpoint{1.304827in}{2.492130in}}{\pgfqpoint{1.304827in}{2.503180in}}%
\pgfpathcurveto{\pgfqpoint{1.304827in}{2.514230in}}{\pgfqpoint{1.300437in}{2.524829in}}{\pgfqpoint{1.292623in}{2.532643in}}%
\pgfpathcurveto{\pgfqpoint{1.284809in}{2.540457in}}{\pgfqpoint{1.274210in}{2.544847in}}{\pgfqpoint{1.263160in}{2.544847in}}%
\pgfpathcurveto{\pgfqpoint{1.252110in}{2.544847in}}{\pgfqpoint{1.241511in}{2.540457in}}{\pgfqpoint{1.233697in}{2.532643in}}%
\pgfpathcurveto{\pgfqpoint{1.225884in}{2.524829in}}{\pgfqpoint{1.221493in}{2.514230in}}{\pgfqpoint{1.221493in}{2.503180in}}%
\pgfpathcurveto{\pgfqpoint{1.221493in}{2.492130in}}{\pgfqpoint{1.225884in}{2.481531in}}{\pgfqpoint{1.233697in}{2.473717in}}%
\pgfpathcurveto{\pgfqpoint{1.241511in}{2.465904in}}{\pgfqpoint{1.252110in}{2.461513in}}{\pgfqpoint{1.263160in}{2.461513in}}%
\pgfpathclose%
\pgfusepath{stroke,fill}%
\end{pgfscope}%
\begin{pgfscope}%
\pgfpathrectangle{\pgfqpoint{0.511823in}{0.504323in}}{\pgfqpoint{3.218177in}{3.225677in}} %
\pgfusepath{clip}%
\pgfsetbuttcap%
\pgfsetroundjoin%
\definecolor{currentfill}{rgb}{0.000000,0.000000,0.545098}%
\pgfsetfillcolor{currentfill}%
\pgfsetfillopacity{0.400000}%
\pgfsetlinewidth{0.501875pt}%
\definecolor{currentstroke}{rgb}{0.000000,0.000000,0.545098}%
\pgfsetstrokecolor{currentstroke}%
\pgfsetstrokeopacity{0.400000}%
\pgfsetdash{}{0pt}%
\pgfpathmoveto{\pgfqpoint{1.226783in}{2.366642in}}%
\pgfpathcurveto{\pgfqpoint{1.237833in}{2.366642in}}{\pgfqpoint{1.248432in}{2.371032in}}{\pgfqpoint{1.256246in}{2.378846in}}%
\pgfpathcurveto{\pgfqpoint{1.264059in}{2.386659in}}{\pgfqpoint{1.268449in}{2.397258in}}{\pgfqpoint{1.268449in}{2.408309in}}%
\pgfpathcurveto{\pgfqpoint{1.268449in}{2.419359in}}{\pgfqpoint{1.264059in}{2.429958in}}{\pgfqpoint{1.256246in}{2.437771in}}%
\pgfpathcurveto{\pgfqpoint{1.248432in}{2.445585in}}{\pgfqpoint{1.237833in}{2.449975in}}{\pgfqpoint{1.226783in}{2.449975in}}%
\pgfpathcurveto{\pgfqpoint{1.215733in}{2.449975in}}{\pgfqpoint{1.205134in}{2.445585in}}{\pgfqpoint{1.197320in}{2.437771in}}%
\pgfpathcurveto{\pgfqpoint{1.189506in}{2.429958in}}{\pgfqpoint{1.185116in}{2.419359in}}{\pgfqpoint{1.185116in}{2.408309in}}%
\pgfpathcurveto{\pgfqpoint{1.185116in}{2.397258in}}{\pgfqpoint{1.189506in}{2.386659in}}{\pgfqpoint{1.197320in}{2.378846in}}%
\pgfpathcurveto{\pgfqpoint{1.205134in}{2.371032in}}{\pgfqpoint{1.215733in}{2.366642in}}{\pgfqpoint{1.226783in}{2.366642in}}%
\pgfpathclose%
\pgfusepath{stroke,fill}%
\end{pgfscope}%
\begin{pgfscope}%
\pgfpathrectangle{\pgfqpoint{0.511823in}{0.504323in}}{\pgfqpoint{3.218177in}{3.225677in}} %
\pgfusepath{clip}%
\pgfsetbuttcap%
\pgfsetroundjoin%
\definecolor{currentfill}{rgb}{0.000000,0.000000,0.545098}%
\pgfsetfillcolor{currentfill}%
\pgfsetfillopacity{0.400000}%
\pgfsetlinewidth{0.501875pt}%
\definecolor{currentstroke}{rgb}{0.000000,0.000000,0.545098}%
\pgfsetstrokecolor{currentstroke}%
\pgfsetstrokeopacity{0.400000}%
\pgfsetdash{}{0pt}%
\pgfpathmoveto{\pgfqpoint{1.216179in}{2.352826in}}%
\pgfpathcurveto{\pgfqpoint{1.227229in}{2.352826in}}{\pgfqpoint{1.237828in}{2.357216in}}{\pgfqpoint{1.245641in}{2.365030in}}%
\pgfpathcurveto{\pgfqpoint{1.253455in}{2.372844in}}{\pgfqpoint{1.257845in}{2.383443in}}{\pgfqpoint{1.257845in}{2.394493in}}%
\pgfpathcurveto{\pgfqpoint{1.257845in}{2.405543in}}{\pgfqpoint{1.253455in}{2.416142in}}{\pgfqpoint{1.245641in}{2.423956in}}%
\pgfpathcurveto{\pgfqpoint{1.237828in}{2.431769in}}{\pgfqpoint{1.227229in}{2.436159in}}{\pgfqpoint{1.216179in}{2.436159in}}%
\pgfpathcurveto{\pgfqpoint{1.205129in}{2.436159in}}{\pgfqpoint{1.194530in}{2.431769in}}{\pgfqpoint{1.186716in}{2.423956in}}%
\pgfpathcurveto{\pgfqpoint{1.178902in}{2.416142in}}{\pgfqpoint{1.174512in}{2.405543in}}{\pgfqpoint{1.174512in}{2.394493in}}%
\pgfpathcurveto{\pgfqpoint{1.174512in}{2.383443in}}{\pgfqpoint{1.178902in}{2.372844in}}{\pgfqpoint{1.186716in}{2.365030in}}%
\pgfpathcurveto{\pgfqpoint{1.194530in}{2.357216in}}{\pgfqpoint{1.205129in}{2.352826in}}{\pgfqpoint{1.216179in}{2.352826in}}%
\pgfpathclose%
\pgfusepath{stroke,fill}%
\end{pgfscope}%
\begin{pgfscope}%
\pgfpathrectangle{\pgfqpoint{0.511823in}{0.504323in}}{\pgfqpoint{3.218177in}{3.225677in}} %
\pgfusepath{clip}%
\pgfsetbuttcap%
\pgfsetroundjoin%
\definecolor{currentfill}{rgb}{0.000000,0.000000,0.545098}%
\pgfsetfillcolor{currentfill}%
\pgfsetfillopacity{0.400000}%
\pgfsetlinewidth{0.501875pt}%
\definecolor{currentstroke}{rgb}{0.000000,0.000000,0.545098}%
\pgfsetstrokecolor{currentstroke}%
\pgfsetstrokeopacity{0.400000}%
\pgfsetdash{}{0pt}%
\pgfpathmoveto{\pgfqpoint{1.188972in}{2.284309in}}%
\pgfpathcurveto{\pgfqpoint{1.200022in}{2.284309in}}{\pgfqpoint{1.210621in}{2.288699in}}{\pgfqpoint{1.218435in}{2.296513in}}%
\pgfpathcurveto{\pgfqpoint{1.226248in}{2.304326in}}{\pgfqpoint{1.230638in}{2.314925in}}{\pgfqpoint{1.230638in}{2.325976in}}%
\pgfpathcurveto{\pgfqpoint{1.230638in}{2.337026in}}{\pgfqpoint{1.226248in}{2.347625in}}{\pgfqpoint{1.218435in}{2.355438in}}%
\pgfpathcurveto{\pgfqpoint{1.210621in}{2.363252in}}{\pgfqpoint{1.200022in}{2.367642in}}{\pgfqpoint{1.188972in}{2.367642in}}%
\pgfpathcurveto{\pgfqpoint{1.177922in}{2.367642in}}{\pgfqpoint{1.167323in}{2.363252in}}{\pgfqpoint{1.159509in}{2.355438in}}%
\pgfpathcurveto{\pgfqpoint{1.151695in}{2.347625in}}{\pgfqpoint{1.147305in}{2.337026in}}{\pgfqpoint{1.147305in}{2.325976in}}%
\pgfpathcurveto{\pgfqpoint{1.147305in}{2.314925in}}{\pgfqpoint{1.151695in}{2.304326in}}{\pgfqpoint{1.159509in}{2.296513in}}%
\pgfpathcurveto{\pgfqpoint{1.167323in}{2.288699in}}{\pgfqpoint{1.177922in}{2.284309in}}{\pgfqpoint{1.188972in}{2.284309in}}%
\pgfpathclose%
\pgfusepath{stroke,fill}%
\end{pgfscope}%
\begin{pgfscope}%
\pgfpathrectangle{\pgfqpoint{0.511823in}{0.504323in}}{\pgfqpoint{3.218177in}{3.225677in}} %
\pgfusepath{clip}%
\pgfsetbuttcap%
\pgfsetroundjoin%
\definecolor{currentfill}{rgb}{0.000000,0.000000,0.545098}%
\pgfsetfillcolor{currentfill}%
\pgfsetfillopacity{0.400000}%
\pgfsetlinewidth{0.501875pt}%
\definecolor{currentstroke}{rgb}{0.000000,0.000000,0.545098}%
\pgfsetstrokecolor{currentstroke}%
\pgfsetstrokeopacity{0.400000}%
\pgfsetdash{}{0pt}%
\pgfpathmoveto{\pgfqpoint{1.175650in}{2.260085in}}%
\pgfpathcurveto{\pgfqpoint{1.186700in}{2.260085in}}{\pgfqpoint{1.197299in}{2.264476in}}{\pgfqpoint{1.205113in}{2.272289in}}%
\pgfpathcurveto{\pgfqpoint{1.212926in}{2.280103in}}{\pgfqpoint{1.217317in}{2.290702in}}{\pgfqpoint{1.217317in}{2.301752in}}%
\pgfpathcurveto{\pgfqpoint{1.217317in}{2.312802in}}{\pgfqpoint{1.212926in}{2.323401in}}{\pgfqpoint{1.205113in}{2.331215in}}%
\pgfpathcurveto{\pgfqpoint{1.197299in}{2.339028in}}{\pgfqpoint{1.186700in}{2.343419in}}{\pgfqpoint{1.175650in}{2.343419in}}%
\pgfpathcurveto{\pgfqpoint{1.164600in}{2.343419in}}{\pgfqpoint{1.154001in}{2.339028in}}{\pgfqpoint{1.146187in}{2.331215in}}%
\pgfpathcurveto{\pgfqpoint{1.138374in}{2.323401in}}{\pgfqpoint{1.133983in}{2.312802in}}{\pgfqpoint{1.133983in}{2.301752in}}%
\pgfpathcurveto{\pgfqpoint{1.133983in}{2.290702in}}{\pgfqpoint{1.138374in}{2.280103in}}{\pgfqpoint{1.146187in}{2.272289in}}%
\pgfpathcurveto{\pgfqpoint{1.154001in}{2.264476in}}{\pgfqpoint{1.164600in}{2.260085in}}{\pgfqpoint{1.175650in}{2.260085in}}%
\pgfpathclose%
\pgfusepath{stroke,fill}%
\end{pgfscope}%
\begin{pgfscope}%
\pgfpathrectangle{\pgfqpoint{0.511823in}{0.504323in}}{\pgfqpoint{3.218177in}{3.225677in}} %
\pgfusepath{clip}%
\pgfsetbuttcap%
\pgfsetroundjoin%
\definecolor{currentfill}{rgb}{0.000000,0.000000,0.545098}%
\pgfsetfillcolor{currentfill}%
\pgfsetfillopacity{0.400000}%
\pgfsetlinewidth{0.501875pt}%
\definecolor{currentstroke}{rgb}{0.000000,0.000000,0.545098}%
\pgfsetstrokecolor{currentstroke}%
\pgfsetstrokeopacity{0.400000}%
\pgfsetdash{}{0pt}%
\pgfpathmoveto{\pgfqpoint{1.190598in}{2.330011in}}%
\pgfpathcurveto{\pgfqpoint{1.201648in}{2.330011in}}{\pgfqpoint{1.212247in}{2.334401in}}{\pgfqpoint{1.220061in}{2.342215in}}%
\pgfpathcurveto{\pgfqpoint{1.227874in}{2.350028in}}{\pgfqpoint{1.232264in}{2.360627in}}{\pgfqpoint{1.232264in}{2.371677in}}%
\pgfpathcurveto{\pgfqpoint{1.232264in}{2.382727in}}{\pgfqpoint{1.227874in}{2.393326in}}{\pgfqpoint{1.220061in}{2.401140in}}%
\pgfpathcurveto{\pgfqpoint{1.212247in}{2.408954in}}{\pgfqpoint{1.201648in}{2.413344in}}{\pgfqpoint{1.190598in}{2.413344in}}%
\pgfpathcurveto{\pgfqpoint{1.179548in}{2.413344in}}{\pgfqpoint{1.168949in}{2.408954in}}{\pgfqpoint{1.161135in}{2.401140in}}%
\pgfpathcurveto{\pgfqpoint{1.153321in}{2.393326in}}{\pgfqpoint{1.148931in}{2.382727in}}{\pgfqpoint{1.148931in}{2.371677in}}%
\pgfpathcurveto{\pgfqpoint{1.148931in}{2.360627in}}{\pgfqpoint{1.153321in}{2.350028in}}{\pgfqpoint{1.161135in}{2.342215in}}%
\pgfpathcurveto{\pgfqpoint{1.168949in}{2.334401in}}{\pgfqpoint{1.179548in}{2.330011in}}{\pgfqpoint{1.190598in}{2.330011in}}%
\pgfpathclose%
\pgfusepath{stroke,fill}%
\end{pgfscope}%
\begin{pgfscope}%
\pgfpathrectangle{\pgfqpoint{0.511823in}{0.504323in}}{\pgfqpoint{3.218177in}{3.225677in}} %
\pgfusepath{clip}%
\pgfsetbuttcap%
\pgfsetroundjoin%
\definecolor{currentfill}{rgb}{0.000000,0.000000,0.545098}%
\pgfsetfillcolor{currentfill}%
\pgfsetfillopacity{0.400000}%
\pgfsetlinewidth{0.501875pt}%
\definecolor{currentstroke}{rgb}{0.000000,0.000000,0.545098}%
\pgfsetstrokecolor{currentstroke}%
\pgfsetstrokeopacity{0.400000}%
\pgfsetdash{}{0pt}%
\pgfpathmoveto{\pgfqpoint{1.238162in}{2.512332in}}%
\pgfpathcurveto{\pgfqpoint{1.249212in}{2.512332in}}{\pgfqpoint{1.259811in}{2.516722in}}{\pgfqpoint{1.267625in}{2.524536in}}%
\pgfpathcurveto{\pgfqpoint{1.275439in}{2.532349in}}{\pgfqpoint{1.279829in}{2.542948in}}{\pgfqpoint{1.279829in}{2.553998in}}%
\pgfpathcurveto{\pgfqpoint{1.279829in}{2.565048in}}{\pgfqpoint{1.275439in}{2.575648in}}{\pgfqpoint{1.267625in}{2.583461in}}%
\pgfpathcurveto{\pgfqpoint{1.259811in}{2.591275in}}{\pgfqpoint{1.249212in}{2.595665in}}{\pgfqpoint{1.238162in}{2.595665in}}%
\pgfpathcurveto{\pgfqpoint{1.227112in}{2.595665in}}{\pgfqpoint{1.216513in}{2.591275in}}{\pgfqpoint{1.208699in}{2.583461in}}%
\pgfpathcurveto{\pgfqpoint{1.200886in}{2.575648in}}{\pgfqpoint{1.196496in}{2.565048in}}{\pgfqpoint{1.196496in}{2.553998in}}%
\pgfpathcurveto{\pgfqpoint{1.196496in}{2.542948in}}{\pgfqpoint{1.200886in}{2.532349in}}{\pgfqpoint{1.208699in}{2.524536in}}%
\pgfpathcurveto{\pgfqpoint{1.216513in}{2.516722in}}{\pgfqpoint{1.227112in}{2.512332in}}{\pgfqpoint{1.238162in}{2.512332in}}%
\pgfpathclose%
\pgfusepath{stroke,fill}%
\end{pgfscope}%
\begin{pgfscope}%
\pgfpathrectangle{\pgfqpoint{0.511823in}{0.504323in}}{\pgfqpoint{3.218177in}{3.225677in}} %
\pgfusepath{clip}%
\pgfsetbuttcap%
\pgfsetroundjoin%
\definecolor{currentfill}{rgb}{0.000000,0.000000,0.545098}%
\pgfsetfillcolor{currentfill}%
\pgfsetfillopacity{0.400000}%
\pgfsetlinewidth{0.501875pt}%
\definecolor{currentstroke}{rgb}{0.000000,0.000000,0.545098}%
\pgfsetstrokecolor{currentstroke}%
\pgfsetstrokeopacity{0.400000}%
\pgfsetdash{}{0pt}%
\pgfpathmoveto{\pgfqpoint{1.178297in}{2.329994in}}%
\pgfpathcurveto{\pgfqpoint{1.189347in}{2.329994in}}{\pgfqpoint{1.199946in}{2.334385in}}{\pgfqpoint{1.207759in}{2.342198in}}%
\pgfpathcurveto{\pgfqpoint{1.215573in}{2.350012in}}{\pgfqpoint{1.219963in}{2.360611in}}{\pgfqpoint{1.219963in}{2.371661in}}%
\pgfpathcurveto{\pgfqpoint{1.219963in}{2.382711in}}{\pgfqpoint{1.215573in}{2.393310in}}{\pgfqpoint{1.207759in}{2.401124in}}%
\pgfpathcurveto{\pgfqpoint{1.199946in}{2.408937in}}{\pgfqpoint{1.189347in}{2.413328in}}{\pgfqpoint{1.178297in}{2.413328in}}%
\pgfpathcurveto{\pgfqpoint{1.167246in}{2.413328in}}{\pgfqpoint{1.156647in}{2.408937in}}{\pgfqpoint{1.148834in}{2.401124in}}%
\pgfpathcurveto{\pgfqpoint{1.141020in}{2.393310in}}{\pgfqpoint{1.136630in}{2.382711in}}{\pgfqpoint{1.136630in}{2.371661in}}%
\pgfpathcurveto{\pgfqpoint{1.136630in}{2.360611in}}{\pgfqpoint{1.141020in}{2.350012in}}{\pgfqpoint{1.148834in}{2.342198in}}%
\pgfpathcurveto{\pgfqpoint{1.156647in}{2.334385in}}{\pgfqpoint{1.167246in}{2.329994in}}{\pgfqpoint{1.178297in}{2.329994in}}%
\pgfpathclose%
\pgfusepath{stroke,fill}%
\end{pgfscope}%
\begin{pgfscope}%
\pgfpathrectangle{\pgfqpoint{0.511823in}{0.504323in}}{\pgfqpoint{3.218177in}{3.225677in}} %
\pgfusepath{clip}%
\pgfsetbuttcap%
\pgfsetroundjoin%
\definecolor{currentfill}{rgb}{0.000000,0.000000,0.545098}%
\pgfsetfillcolor{currentfill}%
\pgfsetfillopacity{0.400000}%
\pgfsetlinewidth{0.501875pt}%
\definecolor{currentstroke}{rgb}{0.000000,0.000000,0.545098}%
\pgfsetstrokecolor{currentstroke}%
\pgfsetstrokeopacity{0.400000}%
\pgfsetdash{}{0pt}%
\pgfpathmoveto{\pgfqpoint{1.249979in}{2.600777in}}%
\pgfpathcurveto{\pgfqpoint{1.261030in}{2.600777in}}{\pgfqpoint{1.271629in}{2.605167in}}{\pgfqpoint{1.279442in}{2.612981in}}%
\pgfpathcurveto{\pgfqpoint{1.287256in}{2.620794in}}{\pgfqpoint{1.291646in}{2.631393in}}{\pgfqpoint{1.291646in}{2.642444in}}%
\pgfpathcurveto{\pgfqpoint{1.291646in}{2.653494in}}{\pgfqpoint{1.287256in}{2.664093in}}{\pgfqpoint{1.279442in}{2.671906in}}%
\pgfpathcurveto{\pgfqpoint{1.271629in}{2.679720in}}{\pgfqpoint{1.261030in}{2.684110in}}{\pgfqpoint{1.249979in}{2.684110in}}%
\pgfpathcurveto{\pgfqpoint{1.238929in}{2.684110in}}{\pgfqpoint{1.228330in}{2.679720in}}{\pgfqpoint{1.220517in}{2.671906in}}%
\pgfpathcurveto{\pgfqpoint{1.212703in}{2.664093in}}{\pgfqpoint{1.208313in}{2.653494in}}{\pgfqpoint{1.208313in}{2.642444in}}%
\pgfpathcurveto{\pgfqpoint{1.208313in}{2.631393in}}{\pgfqpoint{1.212703in}{2.620794in}}{\pgfqpoint{1.220517in}{2.612981in}}%
\pgfpathcurveto{\pgfqpoint{1.228330in}{2.605167in}}{\pgfqpoint{1.238929in}{2.600777in}}{\pgfqpoint{1.249979in}{2.600777in}}%
\pgfpathclose%
\pgfusepath{stroke,fill}%
\end{pgfscope}%
\begin{pgfscope}%
\pgfpathrectangle{\pgfqpoint{0.511823in}{0.504323in}}{\pgfqpoint{3.218177in}{3.225677in}} %
\pgfusepath{clip}%
\pgfsetbuttcap%
\pgfsetroundjoin%
\definecolor{currentfill}{rgb}{0.000000,0.000000,0.545098}%
\pgfsetfillcolor{currentfill}%
\pgfsetfillopacity{0.400000}%
\pgfsetlinewidth{0.501875pt}%
\definecolor{currentstroke}{rgb}{0.000000,0.000000,0.545098}%
\pgfsetstrokecolor{currentstroke}%
\pgfsetstrokeopacity{0.400000}%
\pgfsetdash{}{0pt}%
\pgfpathmoveto{\pgfqpoint{1.208238in}{2.478702in}}%
\pgfpathcurveto{\pgfqpoint{1.219288in}{2.478702in}}{\pgfqpoint{1.229887in}{2.483092in}}{\pgfqpoint{1.237701in}{2.490906in}}%
\pgfpathcurveto{\pgfqpoint{1.245514in}{2.498719in}}{\pgfqpoint{1.249905in}{2.509318in}}{\pgfqpoint{1.249905in}{2.520369in}}%
\pgfpathcurveto{\pgfqpoint{1.249905in}{2.531419in}}{\pgfqpoint{1.245514in}{2.542018in}}{\pgfqpoint{1.237701in}{2.549831in}}%
\pgfpathcurveto{\pgfqpoint{1.229887in}{2.557645in}}{\pgfqpoint{1.219288in}{2.562035in}}{\pgfqpoint{1.208238in}{2.562035in}}%
\pgfpathcurveto{\pgfqpoint{1.197188in}{2.562035in}}{\pgfqpoint{1.186589in}{2.557645in}}{\pgfqpoint{1.178775in}{2.549831in}}%
\pgfpathcurveto{\pgfqpoint{1.170962in}{2.542018in}}{\pgfqpoint{1.166571in}{2.531419in}}{\pgfqpoint{1.166571in}{2.520369in}}%
\pgfpathcurveto{\pgfqpoint{1.166571in}{2.509318in}}{\pgfqpoint{1.170962in}{2.498719in}}{\pgfqpoint{1.178775in}{2.490906in}}%
\pgfpathcurveto{\pgfqpoint{1.186589in}{2.483092in}}{\pgfqpoint{1.197188in}{2.478702in}}{\pgfqpoint{1.208238in}{2.478702in}}%
\pgfpathclose%
\pgfusepath{stroke,fill}%
\end{pgfscope}%
\begin{pgfscope}%
\pgfpathrectangle{\pgfqpoint{0.511823in}{0.504323in}}{\pgfqpoint{3.218177in}{3.225677in}} %
\pgfusepath{clip}%
\pgfsetbuttcap%
\pgfsetroundjoin%
\definecolor{currentfill}{rgb}{0.000000,0.000000,0.545098}%
\pgfsetfillcolor{currentfill}%
\pgfsetfillopacity{0.400000}%
\pgfsetlinewidth{0.501875pt}%
\definecolor{currentstroke}{rgb}{0.000000,0.000000,0.545098}%
\pgfsetstrokecolor{currentstroke}%
\pgfsetstrokeopacity{0.400000}%
\pgfsetdash{}{0pt}%
\pgfpathmoveto{\pgfqpoint{1.193642in}{2.450336in}}%
\pgfpathcurveto{\pgfqpoint{1.204692in}{2.450336in}}{\pgfqpoint{1.215291in}{2.454727in}}{\pgfqpoint{1.223105in}{2.462540in}}%
\pgfpathcurveto{\pgfqpoint{1.230919in}{2.470354in}}{\pgfqpoint{1.235309in}{2.480953in}}{\pgfqpoint{1.235309in}{2.492003in}}%
\pgfpathcurveto{\pgfqpoint{1.235309in}{2.503053in}}{\pgfqpoint{1.230919in}{2.513652in}}{\pgfqpoint{1.223105in}{2.521466in}}%
\pgfpathcurveto{\pgfqpoint{1.215291in}{2.529279in}}{\pgfqpoint{1.204692in}{2.533670in}}{\pgfqpoint{1.193642in}{2.533670in}}%
\pgfpathcurveto{\pgfqpoint{1.182592in}{2.533670in}}{\pgfqpoint{1.171993in}{2.529279in}}{\pgfqpoint{1.164179in}{2.521466in}}%
\pgfpathcurveto{\pgfqpoint{1.156366in}{2.513652in}}{\pgfqpoint{1.151976in}{2.503053in}}{\pgfqpoint{1.151976in}{2.492003in}}%
\pgfpathcurveto{\pgfqpoint{1.151976in}{2.480953in}}{\pgfqpoint{1.156366in}{2.470354in}}{\pgfqpoint{1.164179in}{2.462540in}}%
\pgfpathcurveto{\pgfqpoint{1.171993in}{2.454727in}}{\pgfqpoint{1.182592in}{2.450336in}}{\pgfqpoint{1.193642in}{2.450336in}}%
\pgfpathclose%
\pgfusepath{stroke,fill}%
\end{pgfscope}%
\begin{pgfscope}%
\pgfpathrectangle{\pgfqpoint{0.511823in}{0.504323in}}{\pgfqpoint{3.218177in}{3.225677in}} %
\pgfusepath{clip}%
\pgfsetbuttcap%
\pgfsetroundjoin%
\definecolor{currentfill}{rgb}{0.000000,0.000000,0.545098}%
\pgfsetfillcolor{currentfill}%
\pgfsetfillopacity{0.400000}%
\pgfsetlinewidth{0.501875pt}%
\definecolor{currentstroke}{rgb}{0.000000,0.000000,0.545098}%
\pgfsetstrokecolor{currentstroke}%
\pgfsetstrokeopacity{0.400000}%
\pgfsetdash{}{0pt}%
\pgfpathmoveto{\pgfqpoint{1.142514in}{2.289077in}}%
\pgfpathcurveto{\pgfqpoint{1.153564in}{2.289077in}}{\pgfqpoint{1.164163in}{2.293468in}}{\pgfqpoint{1.171977in}{2.301281in}}%
\pgfpathcurveto{\pgfqpoint{1.179790in}{2.309095in}}{\pgfqpoint{1.184181in}{2.319694in}}{\pgfqpoint{1.184181in}{2.330744in}}%
\pgfpathcurveto{\pgfqpoint{1.184181in}{2.341794in}}{\pgfqpoint{1.179790in}{2.352393in}}{\pgfqpoint{1.171977in}{2.360207in}}%
\pgfpathcurveto{\pgfqpoint{1.164163in}{2.368020in}}{\pgfqpoint{1.153564in}{2.372411in}}{\pgfqpoint{1.142514in}{2.372411in}}%
\pgfpathcurveto{\pgfqpoint{1.131464in}{2.372411in}}{\pgfqpoint{1.120865in}{2.368020in}}{\pgfqpoint{1.113051in}{2.360207in}}%
\pgfpathcurveto{\pgfqpoint{1.105238in}{2.352393in}}{\pgfqpoint{1.100847in}{2.341794in}}{\pgfqpoint{1.100847in}{2.330744in}}%
\pgfpathcurveto{\pgfqpoint{1.100847in}{2.319694in}}{\pgfqpoint{1.105238in}{2.309095in}}{\pgfqpoint{1.113051in}{2.301281in}}%
\pgfpathcurveto{\pgfqpoint{1.120865in}{2.293468in}}{\pgfqpoint{1.131464in}{2.289077in}}{\pgfqpoint{1.142514in}{2.289077in}}%
\pgfpathclose%
\pgfusepath{stroke,fill}%
\end{pgfscope}%
\begin{pgfscope}%
\pgfpathrectangle{\pgfqpoint{0.511823in}{0.504323in}}{\pgfqpoint{3.218177in}{3.225677in}} %
\pgfusepath{clip}%
\pgfsetbuttcap%
\pgfsetroundjoin%
\definecolor{currentfill}{rgb}{0.000000,0.000000,0.545098}%
\pgfsetfillcolor{currentfill}%
\pgfsetfillopacity{0.400000}%
\pgfsetlinewidth{0.501875pt}%
\definecolor{currentstroke}{rgb}{0.000000,0.000000,0.545098}%
\pgfsetstrokecolor{currentstroke}%
\pgfsetstrokeopacity{0.400000}%
\pgfsetdash{}{0pt}%
\pgfpathmoveto{\pgfqpoint{1.240009in}{2.668088in}}%
\pgfpathcurveto{\pgfqpoint{1.251059in}{2.668088in}}{\pgfqpoint{1.261658in}{2.672478in}}{\pgfqpoint{1.269472in}{2.680292in}}%
\pgfpathcurveto{\pgfqpoint{1.277285in}{2.688105in}}{\pgfqpoint{1.281675in}{2.698704in}}{\pgfqpoint{1.281675in}{2.709755in}}%
\pgfpathcurveto{\pgfqpoint{1.281675in}{2.720805in}}{\pgfqpoint{1.277285in}{2.731404in}}{\pgfqpoint{1.269472in}{2.739217in}}%
\pgfpathcurveto{\pgfqpoint{1.261658in}{2.747031in}}{\pgfqpoint{1.251059in}{2.751421in}}{\pgfqpoint{1.240009in}{2.751421in}}%
\pgfpathcurveto{\pgfqpoint{1.228959in}{2.751421in}}{\pgfqpoint{1.218360in}{2.747031in}}{\pgfqpoint{1.210546in}{2.739217in}}%
\pgfpathcurveto{\pgfqpoint{1.202732in}{2.731404in}}{\pgfqpoint{1.198342in}{2.720805in}}{\pgfqpoint{1.198342in}{2.709755in}}%
\pgfpathcurveto{\pgfqpoint{1.198342in}{2.698704in}}{\pgfqpoint{1.202732in}{2.688105in}}{\pgfqpoint{1.210546in}{2.680292in}}%
\pgfpathcurveto{\pgfqpoint{1.218360in}{2.672478in}}{\pgfqpoint{1.228959in}{2.668088in}}{\pgfqpoint{1.240009in}{2.668088in}}%
\pgfpathclose%
\pgfusepath{stroke,fill}%
\end{pgfscope}%
\begin{pgfscope}%
\pgfpathrectangle{\pgfqpoint{0.511823in}{0.504323in}}{\pgfqpoint{3.218177in}{3.225677in}} %
\pgfusepath{clip}%
\pgfsetbuttcap%
\pgfsetroundjoin%
\definecolor{currentfill}{rgb}{0.000000,0.000000,0.545098}%
\pgfsetfillcolor{currentfill}%
\pgfsetfillopacity{0.400000}%
\pgfsetlinewidth{0.501875pt}%
\definecolor{currentstroke}{rgb}{0.000000,0.000000,0.545098}%
\pgfsetstrokecolor{currentstroke}%
\pgfsetstrokeopacity{0.400000}%
\pgfsetdash{}{0pt}%
\pgfpathmoveto{\pgfqpoint{1.123690in}{2.263292in}}%
\pgfpathcurveto{\pgfqpoint{1.134740in}{2.263292in}}{\pgfqpoint{1.145339in}{2.267682in}}{\pgfqpoint{1.153153in}{2.275496in}}%
\pgfpathcurveto{\pgfqpoint{1.160966in}{2.283310in}}{\pgfqpoint{1.165357in}{2.293909in}}{\pgfqpoint{1.165357in}{2.304959in}}%
\pgfpathcurveto{\pgfqpoint{1.165357in}{2.316009in}}{\pgfqpoint{1.160966in}{2.326608in}}{\pgfqpoint{1.153153in}{2.334422in}}%
\pgfpathcurveto{\pgfqpoint{1.145339in}{2.342235in}}{\pgfqpoint{1.134740in}{2.346625in}}{\pgfqpoint{1.123690in}{2.346625in}}%
\pgfpathcurveto{\pgfqpoint{1.112640in}{2.346625in}}{\pgfqpoint{1.102041in}{2.342235in}}{\pgfqpoint{1.094227in}{2.334422in}}%
\pgfpathcurveto{\pgfqpoint{1.086414in}{2.326608in}}{\pgfqpoint{1.082023in}{2.316009in}}{\pgfqpoint{1.082023in}{2.304959in}}%
\pgfpathcurveto{\pgfqpoint{1.082023in}{2.293909in}}{\pgfqpoint{1.086414in}{2.283310in}}{\pgfqpoint{1.094227in}{2.275496in}}%
\pgfpathcurveto{\pgfqpoint{1.102041in}{2.267682in}}{\pgfqpoint{1.112640in}{2.263292in}}{\pgfqpoint{1.123690in}{2.263292in}}%
\pgfpathclose%
\pgfusepath{stroke,fill}%
\end{pgfscope}%
\begin{pgfscope}%
\pgfpathrectangle{\pgfqpoint{0.511823in}{0.504323in}}{\pgfqpoint{3.218177in}{3.225677in}} %
\pgfusepath{clip}%
\pgfsetbuttcap%
\pgfsetroundjoin%
\definecolor{currentfill}{rgb}{0.000000,0.000000,0.545098}%
\pgfsetfillcolor{currentfill}%
\pgfsetfillopacity{0.400000}%
\pgfsetlinewidth{0.501875pt}%
\definecolor{currentstroke}{rgb}{0.000000,0.000000,0.545098}%
\pgfsetstrokecolor{currentstroke}%
\pgfsetstrokeopacity{0.400000}%
\pgfsetdash{}{0pt}%
\pgfpathmoveto{\pgfqpoint{1.181074in}{2.501192in}}%
\pgfpathcurveto{\pgfqpoint{1.192124in}{2.501192in}}{\pgfqpoint{1.202723in}{2.505582in}}{\pgfqpoint{1.210536in}{2.513396in}}%
\pgfpathcurveto{\pgfqpoint{1.218350in}{2.521209in}}{\pgfqpoint{1.222740in}{2.531808in}}{\pgfqpoint{1.222740in}{2.542858in}}%
\pgfpathcurveto{\pgfqpoint{1.222740in}{2.553909in}}{\pgfqpoint{1.218350in}{2.564508in}}{\pgfqpoint{1.210536in}{2.572321in}}%
\pgfpathcurveto{\pgfqpoint{1.202723in}{2.580135in}}{\pgfqpoint{1.192124in}{2.584525in}}{\pgfqpoint{1.181074in}{2.584525in}}%
\pgfpathcurveto{\pgfqpoint{1.170023in}{2.584525in}}{\pgfqpoint{1.159424in}{2.580135in}}{\pgfqpoint{1.151611in}{2.572321in}}%
\pgfpathcurveto{\pgfqpoint{1.143797in}{2.564508in}}{\pgfqpoint{1.139407in}{2.553909in}}{\pgfqpoint{1.139407in}{2.542858in}}%
\pgfpathcurveto{\pgfqpoint{1.139407in}{2.531808in}}{\pgfqpoint{1.143797in}{2.521209in}}{\pgfqpoint{1.151611in}{2.513396in}}%
\pgfpathcurveto{\pgfqpoint{1.159424in}{2.505582in}}{\pgfqpoint{1.170023in}{2.501192in}}{\pgfqpoint{1.181074in}{2.501192in}}%
\pgfpathclose%
\pgfusepath{stroke,fill}%
\end{pgfscope}%
\begin{pgfscope}%
\pgfpathrectangle{\pgfqpoint{0.511823in}{0.504323in}}{\pgfqpoint{3.218177in}{3.225677in}} %
\pgfusepath{clip}%
\pgfsetbuttcap%
\pgfsetroundjoin%
\definecolor{currentfill}{rgb}{0.000000,0.000000,0.545098}%
\pgfsetfillcolor{currentfill}%
\pgfsetfillopacity{0.400000}%
\pgfsetlinewidth{0.501875pt}%
\definecolor{currentstroke}{rgb}{0.000000,0.000000,0.545098}%
\pgfsetstrokecolor{currentstroke}%
\pgfsetstrokeopacity{0.400000}%
\pgfsetdash{}{0pt}%
\pgfpathmoveto{\pgfqpoint{1.211248in}{2.641771in}}%
\pgfpathcurveto{\pgfqpoint{1.222298in}{2.641771in}}{\pgfqpoint{1.232897in}{2.646161in}}{\pgfqpoint{1.240711in}{2.653975in}}%
\pgfpathcurveto{\pgfqpoint{1.248524in}{2.661788in}}{\pgfqpoint{1.252914in}{2.672387in}}{\pgfqpoint{1.252914in}{2.683437in}}%
\pgfpathcurveto{\pgfqpoint{1.252914in}{2.694487in}}{\pgfqpoint{1.248524in}{2.705087in}}{\pgfqpoint{1.240711in}{2.712900in}}%
\pgfpathcurveto{\pgfqpoint{1.232897in}{2.720714in}}{\pgfqpoint{1.222298in}{2.725104in}}{\pgfqpoint{1.211248in}{2.725104in}}%
\pgfpathcurveto{\pgfqpoint{1.200198in}{2.725104in}}{\pgfqpoint{1.189599in}{2.720714in}}{\pgfqpoint{1.181785in}{2.712900in}}%
\pgfpathcurveto{\pgfqpoint{1.173971in}{2.705087in}}{\pgfqpoint{1.169581in}{2.694487in}}{\pgfqpoint{1.169581in}{2.683437in}}%
\pgfpathcurveto{\pgfqpoint{1.169581in}{2.672387in}}{\pgfqpoint{1.173971in}{2.661788in}}{\pgfqpoint{1.181785in}{2.653975in}}%
\pgfpathcurveto{\pgfqpoint{1.189599in}{2.646161in}}{\pgfqpoint{1.200198in}{2.641771in}}{\pgfqpoint{1.211248in}{2.641771in}}%
\pgfpathclose%
\pgfusepath{stroke,fill}%
\end{pgfscope}%
\begin{pgfscope}%
\pgfpathrectangle{\pgfqpoint{0.511823in}{0.504323in}}{\pgfqpoint{3.218177in}{3.225677in}} %
\pgfusepath{clip}%
\pgfsetbuttcap%
\pgfsetroundjoin%
\definecolor{currentfill}{rgb}{0.000000,0.000000,0.545098}%
\pgfsetfillcolor{currentfill}%
\pgfsetfillopacity{0.400000}%
\pgfsetlinewidth{0.501875pt}%
\definecolor{currentstroke}{rgb}{0.000000,0.000000,0.545098}%
\pgfsetstrokecolor{currentstroke}%
\pgfsetstrokeopacity{0.400000}%
\pgfsetdash{}{0pt}%
\pgfpathmoveto{\pgfqpoint{1.145723in}{2.416080in}}%
\pgfpathcurveto{\pgfqpoint{1.156773in}{2.416080in}}{\pgfqpoint{1.167372in}{2.420470in}}{\pgfqpoint{1.175186in}{2.428284in}}%
\pgfpathcurveto{\pgfqpoint{1.182999in}{2.436098in}}{\pgfqpoint{1.187389in}{2.446697in}}{\pgfqpoint{1.187389in}{2.457747in}}%
\pgfpathcurveto{\pgfqpoint{1.187389in}{2.468797in}}{\pgfqpoint{1.182999in}{2.479396in}}{\pgfqpoint{1.175186in}{2.487210in}}%
\pgfpathcurveto{\pgfqpoint{1.167372in}{2.495023in}}{\pgfqpoint{1.156773in}{2.499414in}}{\pgfqpoint{1.145723in}{2.499414in}}%
\pgfpathcurveto{\pgfqpoint{1.134673in}{2.499414in}}{\pgfqpoint{1.124074in}{2.495023in}}{\pgfqpoint{1.116260in}{2.487210in}}%
\pgfpathcurveto{\pgfqpoint{1.108446in}{2.479396in}}{\pgfqpoint{1.104056in}{2.468797in}}{\pgfqpoint{1.104056in}{2.457747in}}%
\pgfpathcurveto{\pgfqpoint{1.104056in}{2.446697in}}{\pgfqpoint{1.108446in}{2.436098in}}{\pgfqpoint{1.116260in}{2.428284in}}%
\pgfpathcurveto{\pgfqpoint{1.124074in}{2.420470in}}{\pgfqpoint{1.134673in}{2.416080in}}{\pgfqpoint{1.145723in}{2.416080in}}%
\pgfpathclose%
\pgfusepath{stroke,fill}%
\end{pgfscope}%
\begin{pgfscope}%
\pgfpathrectangle{\pgfqpoint{0.511823in}{0.504323in}}{\pgfqpoint{3.218177in}{3.225677in}} %
\pgfusepath{clip}%
\pgfsetbuttcap%
\pgfsetroundjoin%
\definecolor{currentfill}{rgb}{0.000000,0.000000,0.545098}%
\pgfsetfillcolor{currentfill}%
\pgfsetfillopacity{0.400000}%
\pgfsetlinewidth{0.501875pt}%
\definecolor{currentstroke}{rgb}{0.000000,0.000000,0.545098}%
\pgfsetstrokecolor{currentstroke}%
\pgfsetstrokeopacity{0.400000}%
\pgfsetdash{}{0pt}%
\pgfpathmoveto{\pgfqpoint{1.124871in}{2.359217in}}%
\pgfpathcurveto{\pgfqpoint{1.135921in}{2.359217in}}{\pgfqpoint{1.146520in}{2.363607in}}{\pgfqpoint{1.154334in}{2.371421in}}%
\pgfpathcurveto{\pgfqpoint{1.162147in}{2.379235in}}{\pgfqpoint{1.166538in}{2.389834in}}{\pgfqpoint{1.166538in}{2.400884in}}%
\pgfpathcurveto{\pgfqpoint{1.166538in}{2.411934in}}{\pgfqpoint{1.162147in}{2.422533in}}{\pgfqpoint{1.154334in}{2.430347in}}%
\pgfpathcurveto{\pgfqpoint{1.146520in}{2.438160in}}{\pgfqpoint{1.135921in}{2.442551in}}{\pgfqpoint{1.124871in}{2.442551in}}%
\pgfpathcurveto{\pgfqpoint{1.113821in}{2.442551in}}{\pgfqpoint{1.103222in}{2.438160in}}{\pgfqpoint{1.095408in}{2.430347in}}%
\pgfpathcurveto{\pgfqpoint{1.087595in}{2.422533in}}{\pgfqpoint{1.083204in}{2.411934in}}{\pgfqpoint{1.083204in}{2.400884in}}%
\pgfpathcurveto{\pgfqpoint{1.083204in}{2.389834in}}{\pgfqpoint{1.087595in}{2.379235in}}{\pgfqpoint{1.095408in}{2.371421in}}%
\pgfpathcurveto{\pgfqpoint{1.103222in}{2.363607in}}{\pgfqpoint{1.113821in}{2.359217in}}{\pgfqpoint{1.124871in}{2.359217in}}%
\pgfpathclose%
\pgfusepath{stroke,fill}%
\end{pgfscope}%
\begin{pgfscope}%
\pgfpathrectangle{\pgfqpoint{0.511823in}{0.504323in}}{\pgfqpoint{3.218177in}{3.225677in}} %
\pgfusepath{clip}%
\pgfsetbuttcap%
\pgfsetroundjoin%
\definecolor{currentfill}{rgb}{0.000000,0.000000,0.545098}%
\pgfsetfillcolor{currentfill}%
\pgfsetfillopacity{0.400000}%
\pgfsetlinewidth{0.501875pt}%
\definecolor{currentstroke}{rgb}{0.000000,0.000000,0.545098}%
\pgfsetstrokecolor{currentstroke}%
\pgfsetstrokeopacity{0.400000}%
\pgfsetdash{}{0pt}%
\pgfpathmoveto{\pgfqpoint{1.139132in}{2.440364in}}%
\pgfpathcurveto{\pgfqpoint{1.150182in}{2.440364in}}{\pgfqpoint{1.160781in}{2.444754in}}{\pgfqpoint{1.168595in}{2.452568in}}%
\pgfpathcurveto{\pgfqpoint{1.176409in}{2.460381in}}{\pgfqpoint{1.180799in}{2.470981in}}{\pgfqpoint{1.180799in}{2.482031in}}%
\pgfpathcurveto{\pgfqpoint{1.180799in}{2.493081in}}{\pgfqpoint{1.176409in}{2.503680in}}{\pgfqpoint{1.168595in}{2.511493in}}%
\pgfpathcurveto{\pgfqpoint{1.160781in}{2.519307in}}{\pgfqpoint{1.150182in}{2.523697in}}{\pgfqpoint{1.139132in}{2.523697in}}%
\pgfpathcurveto{\pgfqpoint{1.128082in}{2.523697in}}{\pgfqpoint{1.117483in}{2.519307in}}{\pgfqpoint{1.109670in}{2.511493in}}%
\pgfpathcurveto{\pgfqpoint{1.101856in}{2.503680in}}{\pgfqpoint{1.097466in}{2.493081in}}{\pgfqpoint{1.097466in}{2.482031in}}%
\pgfpathcurveto{\pgfqpoint{1.097466in}{2.470981in}}{\pgfqpoint{1.101856in}{2.460381in}}{\pgfqpoint{1.109670in}{2.452568in}}%
\pgfpathcurveto{\pgfqpoint{1.117483in}{2.444754in}}{\pgfqpoint{1.128082in}{2.440364in}}{\pgfqpoint{1.139132in}{2.440364in}}%
\pgfpathclose%
\pgfusepath{stroke,fill}%
\end{pgfscope}%
\begin{pgfscope}%
\pgfpathrectangle{\pgfqpoint{0.511823in}{0.504323in}}{\pgfqpoint{3.218177in}{3.225677in}} %
\pgfusepath{clip}%
\pgfsetbuttcap%
\pgfsetroundjoin%
\definecolor{currentfill}{rgb}{0.000000,0.000000,0.545098}%
\pgfsetfillcolor{currentfill}%
\pgfsetfillopacity{0.400000}%
\pgfsetlinewidth{0.501875pt}%
\definecolor{currentstroke}{rgb}{0.000000,0.000000,0.545098}%
\pgfsetstrokecolor{currentstroke}%
\pgfsetstrokeopacity{0.400000}%
\pgfsetdash{}{0pt}%
\pgfpathmoveto{\pgfqpoint{1.160902in}{2.554102in}}%
\pgfpathcurveto{\pgfqpoint{1.171952in}{2.554102in}}{\pgfqpoint{1.182551in}{2.558493in}}{\pgfqpoint{1.190364in}{2.566306in}}%
\pgfpathcurveto{\pgfqpoint{1.198178in}{2.574120in}}{\pgfqpoint{1.202568in}{2.584719in}}{\pgfqpoint{1.202568in}{2.595769in}}%
\pgfpathcurveto{\pgfqpoint{1.202568in}{2.606819in}}{\pgfqpoint{1.198178in}{2.617418in}}{\pgfqpoint{1.190364in}{2.625232in}}%
\pgfpathcurveto{\pgfqpoint{1.182551in}{2.633045in}}{\pgfqpoint{1.171952in}{2.637436in}}{\pgfqpoint{1.160902in}{2.637436in}}%
\pgfpathcurveto{\pgfqpoint{1.149851in}{2.637436in}}{\pgfqpoint{1.139252in}{2.633045in}}{\pgfqpoint{1.131439in}{2.625232in}}%
\pgfpathcurveto{\pgfqpoint{1.123625in}{2.617418in}}{\pgfqpoint{1.119235in}{2.606819in}}{\pgfqpoint{1.119235in}{2.595769in}}%
\pgfpathcurveto{\pgfqpoint{1.119235in}{2.584719in}}{\pgfqpoint{1.123625in}{2.574120in}}{\pgfqpoint{1.131439in}{2.566306in}}%
\pgfpathcurveto{\pgfqpoint{1.139252in}{2.558493in}}{\pgfqpoint{1.149851in}{2.554102in}}{\pgfqpoint{1.160902in}{2.554102in}}%
\pgfpathclose%
\pgfusepath{stroke,fill}%
\end{pgfscope}%
\begin{pgfscope}%
\pgfpathrectangle{\pgfqpoint{0.511823in}{0.504323in}}{\pgfqpoint{3.218177in}{3.225677in}} %
\pgfusepath{clip}%
\pgfsetbuttcap%
\pgfsetroundjoin%
\definecolor{currentfill}{rgb}{0.000000,0.000000,0.545098}%
\pgfsetfillcolor{currentfill}%
\pgfsetfillopacity{0.400000}%
\pgfsetlinewidth{0.501875pt}%
\definecolor{currentstroke}{rgb}{0.000000,0.000000,0.545098}%
\pgfsetstrokecolor{currentstroke}%
\pgfsetstrokeopacity{0.400000}%
\pgfsetdash{}{0pt}%
\pgfpathmoveto{\pgfqpoint{1.167450in}{2.608810in}}%
\pgfpathcurveto{\pgfqpoint{1.178500in}{2.608810in}}{\pgfqpoint{1.189099in}{2.613201in}}{\pgfqpoint{1.196913in}{2.621014in}}%
\pgfpathcurveto{\pgfqpoint{1.204726in}{2.628828in}}{\pgfqpoint{1.209117in}{2.639427in}}{\pgfqpoint{1.209117in}{2.650477in}}%
\pgfpathcurveto{\pgfqpoint{1.209117in}{2.661527in}}{\pgfqpoint{1.204726in}{2.672126in}}{\pgfqpoint{1.196913in}{2.679940in}}%
\pgfpathcurveto{\pgfqpoint{1.189099in}{2.687754in}}{\pgfqpoint{1.178500in}{2.692144in}}{\pgfqpoint{1.167450in}{2.692144in}}%
\pgfpathcurveto{\pgfqpoint{1.156400in}{2.692144in}}{\pgfqpoint{1.145801in}{2.687754in}}{\pgfqpoint{1.137987in}{2.679940in}}%
\pgfpathcurveto{\pgfqpoint{1.130174in}{2.672126in}}{\pgfqpoint{1.125783in}{2.661527in}}{\pgfqpoint{1.125783in}{2.650477in}}%
\pgfpathcurveto{\pgfqpoint{1.125783in}{2.639427in}}{\pgfqpoint{1.130174in}{2.628828in}}{\pgfqpoint{1.137987in}{2.621014in}}%
\pgfpathcurveto{\pgfqpoint{1.145801in}{2.613201in}}{\pgfqpoint{1.156400in}{2.608810in}}{\pgfqpoint{1.167450in}{2.608810in}}%
\pgfpathclose%
\pgfusepath{stroke,fill}%
\end{pgfscope}%
\begin{pgfscope}%
\pgfpathrectangle{\pgfqpoint{0.511823in}{0.504323in}}{\pgfqpoint{3.218177in}{3.225677in}} %
\pgfusepath{clip}%
\pgfsetbuttcap%
\pgfsetroundjoin%
\definecolor{currentfill}{rgb}{0.000000,0.000000,0.545098}%
\pgfsetfillcolor{currentfill}%
\pgfsetfillopacity{0.400000}%
\pgfsetlinewidth{0.501875pt}%
\definecolor{currentstroke}{rgb}{0.000000,0.000000,0.545098}%
\pgfsetstrokecolor{currentstroke}%
\pgfsetstrokeopacity{0.400000}%
\pgfsetdash{}{0pt}%
\pgfpathmoveto{\pgfqpoint{1.138915in}{2.519356in}}%
\pgfpathcurveto{\pgfqpoint{1.149965in}{2.519356in}}{\pgfqpoint{1.160564in}{2.523746in}}{\pgfqpoint{1.168377in}{2.531560in}}%
\pgfpathcurveto{\pgfqpoint{1.176191in}{2.539373in}}{\pgfqpoint{1.180581in}{2.549972in}}{\pgfqpoint{1.180581in}{2.561023in}}%
\pgfpathcurveto{\pgfqpoint{1.180581in}{2.572073in}}{\pgfqpoint{1.176191in}{2.582672in}}{\pgfqpoint{1.168377in}{2.590485in}}%
\pgfpathcurveto{\pgfqpoint{1.160564in}{2.598299in}}{\pgfqpoint{1.149965in}{2.602689in}}{\pgfqpoint{1.138915in}{2.602689in}}%
\pgfpathcurveto{\pgfqpoint{1.127865in}{2.602689in}}{\pgfqpoint{1.117266in}{2.598299in}}{\pgfqpoint{1.109452in}{2.590485in}}%
\pgfpathcurveto{\pgfqpoint{1.101638in}{2.582672in}}{\pgfqpoint{1.097248in}{2.572073in}}{\pgfqpoint{1.097248in}{2.561023in}}%
\pgfpathcurveto{\pgfqpoint{1.097248in}{2.549972in}}{\pgfqpoint{1.101638in}{2.539373in}}{\pgfqpoint{1.109452in}{2.531560in}}%
\pgfpathcurveto{\pgfqpoint{1.117266in}{2.523746in}}{\pgfqpoint{1.127865in}{2.519356in}}{\pgfqpoint{1.138915in}{2.519356in}}%
\pgfpathclose%
\pgfusepath{stroke,fill}%
\end{pgfscope}%
\begin{pgfscope}%
\pgfpathrectangle{\pgfqpoint{0.511823in}{0.504323in}}{\pgfqpoint{3.218177in}{3.225677in}} %
\pgfusepath{clip}%
\pgfsetbuttcap%
\pgfsetroundjoin%
\definecolor{currentfill}{rgb}{0.000000,0.000000,0.545098}%
\pgfsetfillcolor{currentfill}%
\pgfsetfillopacity{0.400000}%
\pgfsetlinewidth{0.501875pt}%
\definecolor{currentstroke}{rgb}{0.000000,0.000000,0.545098}%
\pgfsetstrokecolor{currentstroke}%
\pgfsetstrokeopacity{0.400000}%
\pgfsetdash{}{0pt}%
\pgfpathmoveto{\pgfqpoint{1.084700in}{2.318911in}}%
\pgfpathcurveto{\pgfqpoint{1.095750in}{2.318911in}}{\pgfqpoint{1.106349in}{2.323301in}}{\pgfqpoint{1.114163in}{2.331115in}}%
\pgfpathcurveto{\pgfqpoint{1.121976in}{2.338928in}}{\pgfqpoint{1.126366in}{2.349527in}}{\pgfqpoint{1.126366in}{2.360577in}}%
\pgfpathcurveto{\pgfqpoint{1.126366in}{2.371628in}}{\pgfqpoint{1.121976in}{2.382227in}}{\pgfqpoint{1.114163in}{2.390040in}}%
\pgfpathcurveto{\pgfqpoint{1.106349in}{2.397854in}}{\pgfqpoint{1.095750in}{2.402244in}}{\pgfqpoint{1.084700in}{2.402244in}}%
\pgfpathcurveto{\pgfqpoint{1.073650in}{2.402244in}}{\pgfqpoint{1.063051in}{2.397854in}}{\pgfqpoint{1.055237in}{2.390040in}}%
\pgfpathcurveto{\pgfqpoint{1.047423in}{2.382227in}}{\pgfqpoint{1.043033in}{2.371628in}}{\pgfqpoint{1.043033in}{2.360577in}}%
\pgfpathcurveto{\pgfqpoint{1.043033in}{2.349527in}}{\pgfqpoint{1.047423in}{2.338928in}}{\pgfqpoint{1.055237in}{2.331115in}}%
\pgfpathcurveto{\pgfqpoint{1.063051in}{2.323301in}}{\pgfqpoint{1.073650in}{2.318911in}}{\pgfqpoint{1.084700in}{2.318911in}}%
\pgfpathclose%
\pgfusepath{stroke,fill}%
\end{pgfscope}%
\begin{pgfscope}%
\pgfpathrectangle{\pgfqpoint{0.511823in}{0.504323in}}{\pgfqpoint{3.218177in}{3.225677in}} %
\pgfusepath{clip}%
\pgfsetbuttcap%
\pgfsetroundjoin%
\definecolor{currentfill}{rgb}{0.000000,0.000000,0.545098}%
\pgfsetfillcolor{currentfill}%
\pgfsetfillopacity{0.400000}%
\pgfsetlinewidth{0.501875pt}%
\definecolor{currentstroke}{rgb}{0.000000,0.000000,0.545098}%
\pgfsetstrokecolor{currentstroke}%
\pgfsetstrokeopacity{0.400000}%
\pgfsetdash{}{0pt}%
\pgfpathmoveto{\pgfqpoint{1.090155in}{2.367741in}}%
\pgfpathcurveto{\pgfqpoint{1.101205in}{2.367741in}}{\pgfqpoint{1.111804in}{2.372131in}}{\pgfqpoint{1.119618in}{2.379945in}}%
\pgfpathcurveto{\pgfqpoint{1.127431in}{2.387759in}}{\pgfqpoint{1.131822in}{2.398358in}}{\pgfqpoint{1.131822in}{2.409408in}}%
\pgfpathcurveto{\pgfqpoint{1.131822in}{2.420458in}}{\pgfqpoint{1.127431in}{2.431057in}}{\pgfqpoint{1.119618in}{2.438871in}}%
\pgfpathcurveto{\pgfqpoint{1.111804in}{2.446684in}}{\pgfqpoint{1.101205in}{2.451074in}}{\pgfqpoint{1.090155in}{2.451074in}}%
\pgfpathcurveto{\pgfqpoint{1.079105in}{2.451074in}}{\pgfqpoint{1.068506in}{2.446684in}}{\pgfqpoint{1.060692in}{2.438871in}}%
\pgfpathcurveto{\pgfqpoint{1.052878in}{2.431057in}}{\pgfqpoint{1.048488in}{2.420458in}}{\pgfqpoint{1.048488in}{2.409408in}}%
\pgfpathcurveto{\pgfqpoint{1.048488in}{2.398358in}}{\pgfqpoint{1.052878in}{2.387759in}}{\pgfqpoint{1.060692in}{2.379945in}}%
\pgfpathcurveto{\pgfqpoint{1.068506in}{2.372131in}}{\pgfqpoint{1.079105in}{2.367741in}}{\pgfqpoint{1.090155in}{2.367741in}}%
\pgfpathclose%
\pgfusepath{stroke,fill}%
\end{pgfscope}%
\begin{pgfscope}%
\pgfpathrectangle{\pgfqpoint{0.511823in}{0.504323in}}{\pgfqpoint{3.218177in}{3.225677in}} %
\pgfusepath{clip}%
\pgfsetbuttcap%
\pgfsetroundjoin%
\definecolor{currentfill}{rgb}{0.000000,0.000000,0.545098}%
\pgfsetfillcolor{currentfill}%
\pgfsetfillopacity{0.400000}%
\pgfsetlinewidth{0.501875pt}%
\definecolor{currentstroke}{rgb}{0.000000,0.000000,0.545098}%
\pgfsetstrokecolor{currentstroke}%
\pgfsetstrokeopacity{0.400000}%
\pgfsetdash{}{0pt}%
\pgfpathmoveto{\pgfqpoint{1.108680in}{2.474843in}}%
\pgfpathcurveto{\pgfqpoint{1.119730in}{2.474843in}}{\pgfqpoint{1.130329in}{2.479233in}}{\pgfqpoint{1.138143in}{2.487047in}}%
\pgfpathcurveto{\pgfqpoint{1.145957in}{2.494860in}}{\pgfqpoint{1.150347in}{2.505459in}}{\pgfqpoint{1.150347in}{2.516509in}}%
\pgfpathcurveto{\pgfqpoint{1.150347in}{2.527560in}}{\pgfqpoint{1.145957in}{2.538159in}}{\pgfqpoint{1.138143in}{2.545972in}}%
\pgfpathcurveto{\pgfqpoint{1.130329in}{2.553786in}}{\pgfqpoint{1.119730in}{2.558176in}}{\pgfqpoint{1.108680in}{2.558176in}}%
\pgfpathcurveto{\pgfqpoint{1.097630in}{2.558176in}}{\pgfqpoint{1.087031in}{2.553786in}}{\pgfqpoint{1.079217in}{2.545972in}}%
\pgfpathcurveto{\pgfqpoint{1.071404in}{2.538159in}}{\pgfqpoint{1.067013in}{2.527560in}}{\pgfqpoint{1.067013in}{2.516509in}}%
\pgfpathcurveto{\pgfqpoint{1.067013in}{2.505459in}}{\pgfqpoint{1.071404in}{2.494860in}}{\pgfqpoint{1.079217in}{2.487047in}}%
\pgfpathcurveto{\pgfqpoint{1.087031in}{2.479233in}}{\pgfqpoint{1.097630in}{2.474843in}}{\pgfqpoint{1.108680in}{2.474843in}}%
\pgfpathclose%
\pgfusepath{stroke,fill}%
\end{pgfscope}%
\begin{pgfscope}%
\pgfpathrectangle{\pgfqpoint{0.511823in}{0.504323in}}{\pgfqpoint{3.218177in}{3.225677in}} %
\pgfusepath{clip}%
\pgfsetbuttcap%
\pgfsetroundjoin%
\definecolor{currentfill}{rgb}{0.000000,0.000000,0.545098}%
\pgfsetfillcolor{currentfill}%
\pgfsetfillopacity{0.400000}%
\pgfsetlinewidth{0.501875pt}%
\definecolor{currentstroke}{rgb}{0.000000,0.000000,0.545098}%
\pgfsetstrokecolor{currentstroke}%
\pgfsetstrokeopacity{0.400000}%
\pgfsetdash{}{0pt}%
\pgfpathmoveto{\pgfqpoint{1.061982in}{2.297364in}}%
\pgfpathcurveto{\pgfqpoint{1.073032in}{2.297364in}}{\pgfqpoint{1.083631in}{2.301754in}}{\pgfqpoint{1.091445in}{2.309567in}}%
\pgfpathcurveto{\pgfqpoint{1.099259in}{2.317381in}}{\pgfqpoint{1.103649in}{2.327980in}}{\pgfqpoint{1.103649in}{2.339030in}}%
\pgfpathcurveto{\pgfqpoint{1.103649in}{2.350080in}}{\pgfqpoint{1.099259in}{2.360679in}}{\pgfqpoint{1.091445in}{2.368493in}}%
\pgfpathcurveto{\pgfqpoint{1.083631in}{2.376307in}}{\pgfqpoint{1.073032in}{2.380697in}}{\pgfqpoint{1.061982in}{2.380697in}}%
\pgfpathcurveto{\pgfqpoint{1.050932in}{2.380697in}}{\pgfqpoint{1.040333in}{2.376307in}}{\pgfqpoint{1.032519in}{2.368493in}}%
\pgfpathcurveto{\pgfqpoint{1.024706in}{2.360679in}}{\pgfqpoint{1.020316in}{2.350080in}}{\pgfqpoint{1.020316in}{2.339030in}}%
\pgfpathcurveto{\pgfqpoint{1.020316in}{2.327980in}}{\pgfqpoint{1.024706in}{2.317381in}}{\pgfqpoint{1.032519in}{2.309567in}}%
\pgfpathcurveto{\pgfqpoint{1.040333in}{2.301754in}}{\pgfqpoint{1.050932in}{2.297364in}}{\pgfqpoint{1.061982in}{2.297364in}}%
\pgfpathclose%
\pgfusepath{stroke,fill}%
\end{pgfscope}%
\begin{pgfscope}%
\pgfpathrectangle{\pgfqpoint{0.511823in}{0.504323in}}{\pgfqpoint{3.218177in}{3.225677in}} %
\pgfusepath{clip}%
\pgfsetbuttcap%
\pgfsetroundjoin%
\definecolor{currentfill}{rgb}{0.000000,0.000000,0.545098}%
\pgfsetfillcolor{currentfill}%
\pgfsetfillopacity{0.400000}%
\pgfsetlinewidth{0.501875pt}%
\definecolor{currentstroke}{rgb}{0.000000,0.000000,0.545098}%
\pgfsetstrokecolor{currentstroke}%
\pgfsetstrokeopacity{0.400000}%
\pgfsetdash{}{0pt}%
\pgfpathmoveto{\pgfqpoint{1.084295in}{2.423652in}}%
\pgfpathcurveto{\pgfqpoint{1.095345in}{2.423652in}}{\pgfqpoint{1.105944in}{2.428042in}}{\pgfqpoint{1.113757in}{2.435856in}}%
\pgfpathcurveto{\pgfqpoint{1.121571in}{2.443669in}}{\pgfqpoint{1.125961in}{2.454268in}}{\pgfqpoint{1.125961in}{2.465319in}}%
\pgfpathcurveto{\pgfqpoint{1.125961in}{2.476369in}}{\pgfqpoint{1.121571in}{2.486968in}}{\pgfqpoint{1.113757in}{2.494781in}}%
\pgfpathcurveto{\pgfqpoint{1.105944in}{2.502595in}}{\pgfqpoint{1.095345in}{2.506985in}}{\pgfqpoint{1.084295in}{2.506985in}}%
\pgfpathcurveto{\pgfqpoint{1.073244in}{2.506985in}}{\pgfqpoint{1.062645in}{2.502595in}}{\pgfqpoint{1.054832in}{2.494781in}}%
\pgfpathcurveto{\pgfqpoint{1.047018in}{2.486968in}}{\pgfqpoint{1.042628in}{2.476369in}}{\pgfqpoint{1.042628in}{2.465319in}}%
\pgfpathcurveto{\pgfqpoint{1.042628in}{2.454268in}}{\pgfqpoint{1.047018in}{2.443669in}}{\pgfqpoint{1.054832in}{2.435856in}}%
\pgfpathcurveto{\pgfqpoint{1.062645in}{2.428042in}}{\pgfqpoint{1.073244in}{2.423652in}}{\pgfqpoint{1.084295in}{2.423652in}}%
\pgfpathclose%
\pgfusepath{stroke,fill}%
\end{pgfscope}%
\begin{pgfscope}%
\pgfpathrectangle{\pgfqpoint{0.511823in}{0.504323in}}{\pgfqpoint{3.218177in}{3.225677in}} %
\pgfusepath{clip}%
\pgfsetbuttcap%
\pgfsetroundjoin%
\definecolor{currentfill}{rgb}{0.000000,0.000000,0.545098}%
\pgfsetfillcolor{currentfill}%
\pgfsetfillopacity{0.400000}%
\pgfsetlinewidth{0.501875pt}%
\definecolor{currentstroke}{rgb}{0.000000,0.000000,0.545098}%
\pgfsetstrokecolor{currentstroke}%
\pgfsetstrokeopacity{0.400000}%
\pgfsetdash{}{0pt}%
\pgfpathmoveto{\pgfqpoint{1.144126in}{2.724893in}}%
\pgfpathcurveto{\pgfqpoint{1.155176in}{2.724893in}}{\pgfqpoint{1.165775in}{2.729283in}}{\pgfqpoint{1.173589in}{2.737097in}}%
\pgfpathcurveto{\pgfqpoint{1.181403in}{2.744911in}}{\pgfqpoint{1.185793in}{2.755510in}}{\pgfqpoint{1.185793in}{2.766560in}}%
\pgfpathcurveto{\pgfqpoint{1.185793in}{2.777610in}}{\pgfqpoint{1.181403in}{2.788209in}}{\pgfqpoint{1.173589in}{2.796023in}}%
\pgfpathcurveto{\pgfqpoint{1.165775in}{2.803836in}}{\pgfqpoint{1.155176in}{2.808226in}}{\pgfqpoint{1.144126in}{2.808226in}}%
\pgfpathcurveto{\pgfqpoint{1.133076in}{2.808226in}}{\pgfqpoint{1.122477in}{2.803836in}}{\pgfqpoint{1.114663in}{2.796023in}}%
\pgfpathcurveto{\pgfqpoint{1.106850in}{2.788209in}}{\pgfqpoint{1.102459in}{2.777610in}}{\pgfqpoint{1.102459in}{2.766560in}}%
\pgfpathcurveto{\pgfqpoint{1.102459in}{2.755510in}}{\pgfqpoint{1.106850in}{2.744911in}}{\pgfqpoint{1.114663in}{2.737097in}}%
\pgfpathcurveto{\pgfqpoint{1.122477in}{2.729283in}}{\pgfqpoint{1.133076in}{2.724893in}}{\pgfqpoint{1.144126in}{2.724893in}}%
\pgfpathclose%
\pgfusepath{stroke,fill}%
\end{pgfscope}%
\begin{pgfscope}%
\pgfpathrectangle{\pgfqpoint{0.511823in}{0.504323in}}{\pgfqpoint{3.218177in}{3.225677in}} %
\pgfusepath{clip}%
\pgfsetbuttcap%
\pgfsetroundjoin%
\definecolor{currentfill}{rgb}{0.000000,0.000000,0.545098}%
\pgfsetfillcolor{currentfill}%
\pgfsetfillopacity{0.400000}%
\pgfsetlinewidth{0.501875pt}%
\definecolor{currentstroke}{rgb}{0.000000,0.000000,0.545098}%
\pgfsetstrokecolor{currentstroke}%
\pgfsetstrokeopacity{0.400000}%
\pgfsetdash{}{0pt}%
\pgfpathmoveto{\pgfqpoint{1.050858in}{2.327056in}}%
\pgfpathcurveto{\pgfqpoint{1.061908in}{2.327056in}}{\pgfqpoint{1.072507in}{2.331447in}}{\pgfqpoint{1.080321in}{2.339260in}}%
\pgfpathcurveto{\pgfqpoint{1.088135in}{2.347074in}}{\pgfqpoint{1.092525in}{2.357673in}}{\pgfqpoint{1.092525in}{2.368723in}}%
\pgfpathcurveto{\pgfqpoint{1.092525in}{2.379773in}}{\pgfqpoint{1.088135in}{2.390372in}}{\pgfqpoint{1.080321in}{2.398186in}}%
\pgfpathcurveto{\pgfqpoint{1.072507in}{2.405999in}}{\pgfqpoint{1.061908in}{2.410390in}}{\pgfqpoint{1.050858in}{2.410390in}}%
\pgfpathcurveto{\pgfqpoint{1.039808in}{2.410390in}}{\pgfqpoint{1.029209in}{2.405999in}}{\pgfqpoint{1.021395in}{2.398186in}}%
\pgfpathcurveto{\pgfqpoint{1.013582in}{2.390372in}}{\pgfqpoint{1.009192in}{2.379773in}}{\pgfqpoint{1.009192in}{2.368723in}}%
\pgfpathcurveto{\pgfqpoint{1.009192in}{2.357673in}}{\pgfqpoint{1.013582in}{2.347074in}}{\pgfqpoint{1.021395in}{2.339260in}}%
\pgfpathcurveto{\pgfqpoint{1.029209in}{2.331447in}}{\pgfqpoint{1.039808in}{2.327056in}}{\pgfqpoint{1.050858in}{2.327056in}}%
\pgfpathclose%
\pgfusepath{stroke,fill}%
\end{pgfscope}%
\begin{pgfscope}%
\pgfpathrectangle{\pgfqpoint{0.511823in}{0.504323in}}{\pgfqpoint{3.218177in}{3.225677in}} %
\pgfusepath{clip}%
\pgfsetbuttcap%
\pgfsetroundjoin%
\definecolor{currentfill}{rgb}{0.000000,0.000000,0.545098}%
\pgfsetfillcolor{currentfill}%
\pgfsetfillopacity{0.400000}%
\pgfsetlinewidth{0.501875pt}%
\definecolor{currentstroke}{rgb}{0.000000,0.000000,0.545098}%
\pgfsetstrokecolor{currentstroke}%
\pgfsetstrokeopacity{0.400000}%
\pgfsetdash{}{0pt}%
\pgfpathmoveto{\pgfqpoint{1.102212in}{2.596704in}}%
\pgfpathcurveto{\pgfqpoint{1.113263in}{2.596704in}}{\pgfqpoint{1.123862in}{2.601094in}}{\pgfqpoint{1.131675in}{2.608908in}}%
\pgfpathcurveto{\pgfqpoint{1.139489in}{2.616721in}}{\pgfqpoint{1.143879in}{2.627320in}}{\pgfqpoint{1.143879in}{2.638371in}}%
\pgfpathcurveto{\pgfqpoint{1.143879in}{2.649421in}}{\pgfqpoint{1.139489in}{2.660020in}}{\pgfqpoint{1.131675in}{2.667833in}}%
\pgfpathcurveto{\pgfqpoint{1.123862in}{2.675647in}}{\pgfqpoint{1.113263in}{2.680037in}}{\pgfqpoint{1.102212in}{2.680037in}}%
\pgfpathcurveto{\pgfqpoint{1.091162in}{2.680037in}}{\pgfqpoint{1.080563in}{2.675647in}}{\pgfqpoint{1.072750in}{2.667833in}}%
\pgfpathcurveto{\pgfqpoint{1.064936in}{2.660020in}}{\pgfqpoint{1.060546in}{2.649421in}}{\pgfqpoint{1.060546in}{2.638371in}}%
\pgfpathcurveto{\pgfqpoint{1.060546in}{2.627320in}}{\pgfqpoint{1.064936in}{2.616721in}}{\pgfqpoint{1.072750in}{2.608908in}}%
\pgfpathcurveto{\pgfqpoint{1.080563in}{2.601094in}}{\pgfqpoint{1.091162in}{2.596704in}}{\pgfqpoint{1.102212in}{2.596704in}}%
\pgfpathclose%
\pgfusepath{stroke,fill}%
\end{pgfscope}%
\begin{pgfscope}%
\pgfpathrectangle{\pgfqpoint{0.511823in}{0.504323in}}{\pgfqpoint{3.218177in}{3.225677in}} %
\pgfusepath{clip}%
\pgfsetbuttcap%
\pgfsetroundjoin%
\definecolor{currentfill}{rgb}{0.000000,0.000000,0.545098}%
\pgfsetfillcolor{currentfill}%
\pgfsetfillopacity{0.400000}%
\pgfsetlinewidth{0.501875pt}%
\definecolor{currentstroke}{rgb}{0.000000,0.000000,0.545098}%
\pgfsetstrokecolor{currentstroke}%
\pgfsetstrokeopacity{0.400000}%
\pgfsetdash{}{0pt}%
\pgfpathmoveto{\pgfqpoint{1.042586in}{2.344240in}}%
\pgfpathcurveto{\pgfqpoint{1.053636in}{2.344240in}}{\pgfqpoint{1.064235in}{2.348631in}}{\pgfqpoint{1.072049in}{2.356444in}}%
\pgfpathcurveto{\pgfqpoint{1.079862in}{2.364258in}}{\pgfqpoint{1.084252in}{2.374857in}}{\pgfqpoint{1.084252in}{2.385907in}}%
\pgfpathcurveto{\pgfqpoint{1.084252in}{2.396957in}}{\pgfqpoint{1.079862in}{2.407556in}}{\pgfqpoint{1.072049in}{2.415370in}}%
\pgfpathcurveto{\pgfqpoint{1.064235in}{2.423184in}}{\pgfqpoint{1.053636in}{2.427574in}}{\pgfqpoint{1.042586in}{2.427574in}}%
\pgfpathcurveto{\pgfqpoint{1.031536in}{2.427574in}}{\pgfqpoint{1.020937in}{2.423184in}}{\pgfqpoint{1.013123in}{2.415370in}}%
\pgfpathcurveto{\pgfqpoint{1.005309in}{2.407556in}}{\pgfqpoint{1.000919in}{2.396957in}}{\pgfqpoint{1.000919in}{2.385907in}}%
\pgfpathcurveto{\pgfqpoint{1.000919in}{2.374857in}}{\pgfqpoint{1.005309in}{2.364258in}}{\pgfqpoint{1.013123in}{2.356444in}}%
\pgfpathcurveto{\pgfqpoint{1.020937in}{2.348631in}}{\pgfqpoint{1.031536in}{2.344240in}}{\pgfqpoint{1.042586in}{2.344240in}}%
\pgfpathclose%
\pgfusepath{stroke,fill}%
\end{pgfscope}%
\begin{pgfscope}%
\pgfpathrectangle{\pgfqpoint{0.511823in}{0.504323in}}{\pgfqpoint{3.218177in}{3.225677in}} %
\pgfusepath{clip}%
\pgfsetbuttcap%
\pgfsetroundjoin%
\definecolor{currentfill}{rgb}{0.000000,0.000000,0.545098}%
\pgfsetfillcolor{currentfill}%
\pgfsetfillopacity{0.400000}%
\pgfsetlinewidth{0.501875pt}%
\definecolor{currentstroke}{rgb}{0.000000,0.000000,0.545098}%
\pgfsetstrokecolor{currentstroke}%
\pgfsetstrokeopacity{0.400000}%
\pgfsetdash{}{0pt}%
\pgfpathmoveto{\pgfqpoint{1.051569in}{2.417082in}}%
\pgfpathcurveto{\pgfqpoint{1.062619in}{2.417082in}}{\pgfqpoint{1.073219in}{2.421472in}}{\pgfqpoint{1.081032in}{2.429286in}}%
\pgfpathcurveto{\pgfqpoint{1.088846in}{2.437100in}}{\pgfqpoint{1.093236in}{2.447699in}}{\pgfqpoint{1.093236in}{2.458749in}}%
\pgfpathcurveto{\pgfqpoint{1.093236in}{2.469799in}}{\pgfqpoint{1.088846in}{2.480398in}}{\pgfqpoint{1.081032in}{2.488211in}}%
\pgfpathcurveto{\pgfqpoint{1.073219in}{2.496025in}}{\pgfqpoint{1.062619in}{2.500415in}}{\pgfqpoint{1.051569in}{2.500415in}}%
\pgfpathcurveto{\pgfqpoint{1.040519in}{2.500415in}}{\pgfqpoint{1.029920in}{2.496025in}}{\pgfqpoint{1.022107in}{2.488211in}}%
\pgfpathcurveto{\pgfqpoint{1.014293in}{2.480398in}}{\pgfqpoint{1.009903in}{2.469799in}}{\pgfqpoint{1.009903in}{2.458749in}}%
\pgfpathcurveto{\pgfqpoint{1.009903in}{2.447699in}}{\pgfqpoint{1.014293in}{2.437100in}}{\pgfqpoint{1.022107in}{2.429286in}}%
\pgfpathcurveto{\pgfqpoint{1.029920in}{2.421472in}}{\pgfqpoint{1.040519in}{2.417082in}}{\pgfqpoint{1.051569in}{2.417082in}}%
\pgfpathclose%
\pgfusepath{stroke,fill}%
\end{pgfscope}%
\begin{pgfscope}%
\pgfpathrectangle{\pgfqpoint{0.511823in}{0.504323in}}{\pgfqpoint{3.218177in}{3.225677in}} %
\pgfusepath{clip}%
\pgfsetbuttcap%
\pgfsetroundjoin%
\definecolor{currentfill}{rgb}{0.000000,0.000000,0.545098}%
\pgfsetfillcolor{currentfill}%
\pgfsetfillopacity{0.400000}%
\pgfsetlinewidth{0.501875pt}%
\definecolor{currentstroke}{rgb}{0.000000,0.000000,0.545098}%
\pgfsetstrokecolor{currentstroke}%
\pgfsetstrokeopacity{0.400000}%
\pgfsetdash{}{0pt}%
\pgfpathmoveto{\pgfqpoint{1.052183in}{2.450969in}}%
\pgfpathcurveto{\pgfqpoint{1.063234in}{2.450969in}}{\pgfqpoint{1.073833in}{2.455360in}}{\pgfqpoint{1.081646in}{2.463173in}}%
\pgfpathcurveto{\pgfqpoint{1.089460in}{2.470987in}}{\pgfqpoint{1.093850in}{2.481586in}}{\pgfqpoint{1.093850in}{2.492636in}}%
\pgfpathcurveto{\pgfqpoint{1.093850in}{2.503686in}}{\pgfqpoint{1.089460in}{2.514285in}}{\pgfqpoint{1.081646in}{2.522099in}}%
\pgfpathcurveto{\pgfqpoint{1.073833in}{2.529913in}}{\pgfqpoint{1.063234in}{2.534303in}}{\pgfqpoint{1.052183in}{2.534303in}}%
\pgfpathcurveto{\pgfqpoint{1.041133in}{2.534303in}}{\pgfqpoint{1.030534in}{2.529913in}}{\pgfqpoint{1.022721in}{2.522099in}}%
\pgfpathcurveto{\pgfqpoint{1.014907in}{2.514285in}}{\pgfqpoint{1.010517in}{2.503686in}}{\pgfqpoint{1.010517in}{2.492636in}}%
\pgfpathcurveto{\pgfqpoint{1.010517in}{2.481586in}}{\pgfqpoint{1.014907in}{2.470987in}}{\pgfqpoint{1.022721in}{2.463173in}}%
\pgfpathcurveto{\pgfqpoint{1.030534in}{2.455360in}}{\pgfqpoint{1.041133in}{2.450969in}}{\pgfqpoint{1.052183in}{2.450969in}}%
\pgfpathclose%
\pgfusepath{stroke,fill}%
\end{pgfscope}%
\begin{pgfscope}%
\pgfpathrectangle{\pgfqpoint{0.511823in}{0.504323in}}{\pgfqpoint{3.218177in}{3.225677in}} %
\pgfusepath{clip}%
\pgfsetbuttcap%
\pgfsetroundjoin%
\definecolor{currentfill}{rgb}{0.000000,0.000000,0.545098}%
\pgfsetfillcolor{currentfill}%
\pgfsetfillopacity{0.400000}%
\pgfsetlinewidth{0.501875pt}%
\definecolor{currentstroke}{rgb}{0.000000,0.000000,0.545098}%
\pgfsetstrokecolor{currentstroke}%
\pgfsetstrokeopacity{0.400000}%
\pgfsetdash{}{0pt}%
\pgfpathmoveto{\pgfqpoint{1.011480in}{2.277796in}}%
\pgfpathcurveto{\pgfqpoint{1.022530in}{2.277796in}}{\pgfqpoint{1.033129in}{2.282186in}}{\pgfqpoint{1.040943in}{2.290000in}}%
\pgfpathcurveto{\pgfqpoint{1.048756in}{2.297813in}}{\pgfqpoint{1.053146in}{2.308412in}}{\pgfqpoint{1.053146in}{2.319463in}}%
\pgfpathcurveto{\pgfqpoint{1.053146in}{2.330513in}}{\pgfqpoint{1.048756in}{2.341112in}}{\pgfqpoint{1.040943in}{2.348925in}}%
\pgfpathcurveto{\pgfqpoint{1.033129in}{2.356739in}}{\pgfqpoint{1.022530in}{2.361129in}}{\pgfqpoint{1.011480in}{2.361129in}}%
\pgfpathcurveto{\pgfqpoint{1.000430in}{2.361129in}}{\pgfqpoint{0.989831in}{2.356739in}}{\pgfqpoint{0.982017in}{2.348925in}}%
\pgfpathcurveto{\pgfqpoint{0.974203in}{2.341112in}}{\pgfqpoint{0.969813in}{2.330513in}}{\pgfqpoint{0.969813in}{2.319463in}}%
\pgfpathcurveto{\pgfqpoint{0.969813in}{2.308412in}}{\pgfqpoint{0.974203in}{2.297813in}}{\pgfqpoint{0.982017in}{2.290000in}}%
\pgfpathcurveto{\pgfqpoint{0.989831in}{2.282186in}}{\pgfqpoint{1.000430in}{2.277796in}}{\pgfqpoint{1.011480in}{2.277796in}}%
\pgfpathclose%
\pgfusepath{stroke,fill}%
\end{pgfscope}%
\begin{pgfscope}%
\pgfpathrectangle{\pgfqpoint{0.511823in}{0.504323in}}{\pgfqpoint{3.218177in}{3.225677in}} %
\pgfusepath{clip}%
\pgfsetbuttcap%
\pgfsetroundjoin%
\definecolor{currentfill}{rgb}{0.000000,0.000000,0.545098}%
\pgfsetfillcolor{currentfill}%
\pgfsetfillopacity{0.400000}%
\pgfsetlinewidth{0.501875pt}%
\definecolor{currentstroke}{rgb}{0.000000,0.000000,0.545098}%
\pgfsetstrokecolor{currentstroke}%
\pgfsetstrokeopacity{0.400000}%
\pgfsetdash{}{0pt}%
\pgfpathmoveto{\pgfqpoint{1.039094in}{2.448831in}}%
\pgfpathcurveto{\pgfqpoint{1.050144in}{2.448831in}}{\pgfqpoint{1.060743in}{2.453222in}}{\pgfqpoint{1.068556in}{2.461035in}}%
\pgfpathcurveto{\pgfqpoint{1.076370in}{2.468849in}}{\pgfqpoint{1.080760in}{2.479448in}}{\pgfqpoint{1.080760in}{2.490498in}}%
\pgfpathcurveto{\pgfqpoint{1.080760in}{2.501548in}}{\pgfqpoint{1.076370in}{2.512147in}}{\pgfqpoint{1.068556in}{2.519961in}}%
\pgfpathcurveto{\pgfqpoint{1.060743in}{2.527774in}}{\pgfqpoint{1.050144in}{2.532165in}}{\pgfqpoint{1.039094in}{2.532165in}}%
\pgfpathcurveto{\pgfqpoint{1.028043in}{2.532165in}}{\pgfqpoint{1.017444in}{2.527774in}}{\pgfqpoint{1.009631in}{2.519961in}}%
\pgfpathcurveto{\pgfqpoint{1.001817in}{2.512147in}}{\pgfqpoint{0.997427in}{2.501548in}}{\pgfqpoint{0.997427in}{2.490498in}}%
\pgfpathcurveto{\pgfqpoint{0.997427in}{2.479448in}}{\pgfqpoint{1.001817in}{2.468849in}}{\pgfqpoint{1.009631in}{2.461035in}}%
\pgfpathcurveto{\pgfqpoint{1.017444in}{2.453222in}}{\pgfqpoint{1.028043in}{2.448831in}}{\pgfqpoint{1.039094in}{2.448831in}}%
\pgfpathclose%
\pgfusepath{stroke,fill}%
\end{pgfscope}%
\begin{pgfscope}%
\pgfpathrectangle{\pgfqpoint{0.511823in}{0.504323in}}{\pgfqpoint{3.218177in}{3.225677in}} %
\pgfusepath{clip}%
\pgfsetbuttcap%
\pgfsetroundjoin%
\definecolor{currentfill}{rgb}{0.000000,0.000000,0.545098}%
\pgfsetfillcolor{currentfill}%
\pgfsetfillopacity{0.400000}%
\pgfsetlinewidth{0.501875pt}%
\definecolor{currentstroke}{rgb}{0.000000,0.000000,0.545098}%
\pgfsetstrokecolor{currentstroke}%
\pgfsetstrokeopacity{0.400000}%
\pgfsetdash{}{0pt}%
\pgfpathmoveto{\pgfqpoint{1.065587in}{2.620220in}}%
\pgfpathcurveto{\pgfqpoint{1.076637in}{2.620220in}}{\pgfqpoint{1.087236in}{2.624611in}}{\pgfqpoint{1.095050in}{2.632424in}}%
\pgfpathcurveto{\pgfqpoint{1.102863in}{2.640238in}}{\pgfqpoint{1.107254in}{2.650837in}}{\pgfqpoint{1.107254in}{2.661887in}}%
\pgfpathcurveto{\pgfqpoint{1.107254in}{2.672937in}}{\pgfqpoint{1.102863in}{2.683536in}}{\pgfqpoint{1.095050in}{2.691350in}}%
\pgfpathcurveto{\pgfqpoint{1.087236in}{2.699163in}}{\pgfqpoint{1.076637in}{2.703554in}}{\pgfqpoint{1.065587in}{2.703554in}}%
\pgfpathcurveto{\pgfqpoint{1.054537in}{2.703554in}}{\pgfqpoint{1.043938in}{2.699163in}}{\pgfqpoint{1.036124in}{2.691350in}}%
\pgfpathcurveto{\pgfqpoint{1.028310in}{2.683536in}}{\pgfqpoint{1.023920in}{2.672937in}}{\pgfqpoint{1.023920in}{2.661887in}}%
\pgfpathcurveto{\pgfqpoint{1.023920in}{2.650837in}}{\pgfqpoint{1.028310in}{2.640238in}}{\pgfqpoint{1.036124in}{2.632424in}}%
\pgfpathcurveto{\pgfqpoint{1.043938in}{2.624611in}}{\pgfqpoint{1.054537in}{2.620220in}}{\pgfqpoint{1.065587in}{2.620220in}}%
\pgfpathclose%
\pgfusepath{stroke,fill}%
\end{pgfscope}%
\begin{pgfscope}%
\pgfpathrectangle{\pgfqpoint{0.511823in}{0.504323in}}{\pgfqpoint{3.218177in}{3.225677in}} %
\pgfusepath{clip}%
\pgfsetbuttcap%
\pgfsetroundjoin%
\definecolor{currentfill}{rgb}{0.000000,0.000000,0.545098}%
\pgfsetfillcolor{currentfill}%
\pgfsetfillopacity{0.400000}%
\pgfsetlinewidth{0.501875pt}%
\definecolor{currentstroke}{rgb}{0.000000,0.000000,0.545098}%
\pgfsetstrokecolor{currentstroke}%
\pgfsetstrokeopacity{0.400000}%
\pgfsetdash{}{0pt}%
\pgfpathmoveto{\pgfqpoint{1.045638in}{2.550843in}}%
\pgfpathcurveto{\pgfqpoint{1.056688in}{2.550843in}}{\pgfqpoint{1.067287in}{2.555233in}}{\pgfqpoint{1.075101in}{2.563047in}}%
\pgfpathcurveto{\pgfqpoint{1.082914in}{2.570860in}}{\pgfqpoint{1.087305in}{2.581459in}}{\pgfqpoint{1.087305in}{2.592510in}}%
\pgfpathcurveto{\pgfqpoint{1.087305in}{2.603560in}}{\pgfqpoint{1.082914in}{2.614159in}}{\pgfqpoint{1.075101in}{2.621972in}}%
\pgfpathcurveto{\pgfqpoint{1.067287in}{2.629786in}}{\pgfqpoint{1.056688in}{2.634176in}}{\pgfqpoint{1.045638in}{2.634176in}}%
\pgfpathcurveto{\pgfqpoint{1.034588in}{2.634176in}}{\pgfqpoint{1.023989in}{2.629786in}}{\pgfqpoint{1.016175in}{2.621972in}}%
\pgfpathcurveto{\pgfqpoint{1.008362in}{2.614159in}}{\pgfqpoint{1.003971in}{2.603560in}}{\pgfqpoint{1.003971in}{2.592510in}}%
\pgfpathcurveto{\pgfqpoint{1.003971in}{2.581459in}}{\pgfqpoint{1.008362in}{2.570860in}}{\pgfqpoint{1.016175in}{2.563047in}}%
\pgfpathcurveto{\pgfqpoint{1.023989in}{2.555233in}}{\pgfqpoint{1.034588in}{2.550843in}}{\pgfqpoint{1.045638in}{2.550843in}}%
\pgfpathclose%
\pgfusepath{stroke,fill}%
\end{pgfscope}%
\begin{pgfscope}%
\pgfpathrectangle{\pgfqpoint{0.511823in}{0.504323in}}{\pgfqpoint{3.218177in}{3.225677in}} %
\pgfusepath{clip}%
\pgfsetbuttcap%
\pgfsetroundjoin%
\definecolor{currentfill}{rgb}{0.000000,0.000000,0.545098}%
\pgfsetfillcolor{currentfill}%
\pgfsetfillopacity{0.400000}%
\pgfsetlinewidth{0.501875pt}%
\definecolor{currentstroke}{rgb}{0.000000,0.000000,0.545098}%
\pgfsetstrokecolor{currentstroke}%
\pgfsetstrokeopacity{0.400000}%
\pgfsetdash{}{0pt}%
\pgfpathmoveto{\pgfqpoint{1.038735in}{2.549523in}}%
\pgfpathcurveto{\pgfqpoint{1.049785in}{2.549523in}}{\pgfqpoint{1.060384in}{2.553913in}}{\pgfqpoint{1.068198in}{2.561727in}}%
\pgfpathcurveto{\pgfqpoint{1.076011in}{2.569540in}}{\pgfqpoint{1.080402in}{2.580139in}}{\pgfqpoint{1.080402in}{2.591189in}}%
\pgfpathcurveto{\pgfqpoint{1.080402in}{2.602240in}}{\pgfqpoint{1.076011in}{2.612839in}}{\pgfqpoint{1.068198in}{2.620652in}}%
\pgfpathcurveto{\pgfqpoint{1.060384in}{2.628466in}}{\pgfqpoint{1.049785in}{2.632856in}}{\pgfqpoint{1.038735in}{2.632856in}}%
\pgfpathcurveto{\pgfqpoint{1.027685in}{2.632856in}}{\pgfqpoint{1.017086in}{2.628466in}}{\pgfqpoint{1.009272in}{2.620652in}}%
\pgfpathcurveto{\pgfqpoint{1.001458in}{2.612839in}}{\pgfqpoint{0.997068in}{2.602240in}}{\pgfqpoint{0.997068in}{2.591189in}}%
\pgfpathcurveto{\pgfqpoint{0.997068in}{2.580139in}}{\pgfqpoint{1.001458in}{2.569540in}}{\pgfqpoint{1.009272in}{2.561727in}}%
\pgfpathcurveto{\pgfqpoint{1.017086in}{2.553913in}}{\pgfqpoint{1.027685in}{2.549523in}}{\pgfqpoint{1.038735in}{2.549523in}}%
\pgfpathclose%
\pgfusepath{stroke,fill}%
\end{pgfscope}%
\begin{pgfscope}%
\pgfpathrectangle{\pgfqpoint{0.511823in}{0.504323in}}{\pgfqpoint{3.218177in}{3.225677in}} %
\pgfusepath{clip}%
\pgfsetbuttcap%
\pgfsetroundjoin%
\definecolor{currentfill}{rgb}{0.000000,0.000000,0.545098}%
\pgfsetfillcolor{currentfill}%
\pgfsetfillopacity{0.400000}%
\pgfsetlinewidth{0.501875pt}%
\definecolor{currentstroke}{rgb}{0.000000,0.000000,0.545098}%
\pgfsetstrokecolor{currentstroke}%
\pgfsetstrokeopacity{0.400000}%
\pgfsetdash{}{0pt}%
\pgfpathmoveto{\pgfqpoint{0.999176in}{2.367684in}}%
\pgfpathcurveto{\pgfqpoint{1.010226in}{2.367684in}}{\pgfqpoint{1.020825in}{2.372074in}}{\pgfqpoint{1.028639in}{2.379888in}}%
\pgfpathcurveto{\pgfqpoint{1.036452in}{2.387702in}}{\pgfqpoint{1.040843in}{2.398301in}}{\pgfqpoint{1.040843in}{2.409351in}}%
\pgfpathcurveto{\pgfqpoint{1.040843in}{2.420401in}}{\pgfqpoint{1.036452in}{2.431000in}}{\pgfqpoint{1.028639in}{2.438814in}}%
\pgfpathcurveto{\pgfqpoint{1.020825in}{2.446627in}}{\pgfqpoint{1.010226in}{2.451017in}}{\pgfqpoint{0.999176in}{2.451017in}}%
\pgfpathcurveto{\pgfqpoint{0.988126in}{2.451017in}}{\pgfqpoint{0.977527in}{2.446627in}}{\pgfqpoint{0.969713in}{2.438814in}}%
\pgfpathcurveto{\pgfqpoint{0.961900in}{2.431000in}}{\pgfqpoint{0.957509in}{2.420401in}}{\pgfqpoint{0.957509in}{2.409351in}}%
\pgfpathcurveto{\pgfqpoint{0.957509in}{2.398301in}}{\pgfqpoint{0.961900in}{2.387702in}}{\pgfqpoint{0.969713in}{2.379888in}}%
\pgfpathcurveto{\pgfqpoint{0.977527in}{2.372074in}}{\pgfqpoint{0.988126in}{2.367684in}}{\pgfqpoint{0.999176in}{2.367684in}}%
\pgfpathclose%
\pgfusepath{stroke,fill}%
\end{pgfscope}%
\begin{pgfscope}%
\pgfpathrectangle{\pgfqpoint{0.511823in}{0.504323in}}{\pgfqpoint{3.218177in}{3.225677in}} %
\pgfusepath{clip}%
\pgfsetbuttcap%
\pgfsetroundjoin%
\definecolor{currentfill}{rgb}{0.000000,0.000000,0.545098}%
\pgfsetfillcolor{currentfill}%
\pgfsetfillopacity{0.400000}%
\pgfsetlinewidth{0.501875pt}%
\definecolor{currentstroke}{rgb}{0.000000,0.000000,0.545098}%
\pgfsetstrokecolor{currentstroke}%
\pgfsetstrokeopacity{0.400000}%
\pgfsetdash{}{0pt}%
\pgfpathmoveto{\pgfqpoint{1.007585in}{2.448904in}}%
\pgfpathcurveto{\pgfqpoint{1.018635in}{2.448904in}}{\pgfqpoint{1.029234in}{2.453294in}}{\pgfqpoint{1.037048in}{2.461107in}}%
\pgfpathcurveto{\pgfqpoint{1.044862in}{2.468921in}}{\pgfqpoint{1.049252in}{2.479520in}}{\pgfqpoint{1.049252in}{2.490570in}}%
\pgfpathcurveto{\pgfqpoint{1.049252in}{2.501620in}}{\pgfqpoint{1.044862in}{2.512219in}}{\pgfqpoint{1.037048in}{2.520033in}}%
\pgfpathcurveto{\pgfqpoint{1.029234in}{2.527847in}}{\pgfqpoint{1.018635in}{2.532237in}}{\pgfqpoint{1.007585in}{2.532237in}}%
\pgfpathcurveto{\pgfqpoint{0.996535in}{2.532237in}}{\pgfqpoint{0.985936in}{2.527847in}}{\pgfqpoint{0.978122in}{2.520033in}}%
\pgfpathcurveto{\pgfqpoint{0.970309in}{2.512219in}}{\pgfqpoint{0.965918in}{2.501620in}}{\pgfqpoint{0.965918in}{2.490570in}}%
\pgfpathcurveto{\pgfqpoint{0.965918in}{2.479520in}}{\pgfqpoint{0.970309in}{2.468921in}}{\pgfqpoint{0.978122in}{2.461107in}}%
\pgfpathcurveto{\pgfqpoint{0.985936in}{2.453294in}}{\pgfqpoint{0.996535in}{2.448904in}}{\pgfqpoint{1.007585in}{2.448904in}}%
\pgfpathclose%
\pgfusepath{stroke,fill}%
\end{pgfscope}%
\begin{pgfscope}%
\pgfpathrectangle{\pgfqpoint{0.511823in}{0.504323in}}{\pgfqpoint{3.218177in}{3.225677in}} %
\pgfusepath{clip}%
\pgfsetbuttcap%
\pgfsetroundjoin%
\definecolor{currentfill}{rgb}{0.000000,0.000000,0.545098}%
\pgfsetfillcolor{currentfill}%
\pgfsetfillopacity{0.400000}%
\pgfsetlinewidth{0.501875pt}%
\definecolor{currentstroke}{rgb}{0.000000,0.000000,0.545098}%
\pgfsetstrokecolor{currentstroke}%
\pgfsetstrokeopacity{0.400000}%
\pgfsetdash{}{0pt}%
\pgfpathmoveto{\pgfqpoint{1.008830in}{2.492163in}}%
\pgfpathcurveto{\pgfqpoint{1.019880in}{2.492163in}}{\pgfqpoint{1.030479in}{2.496553in}}{\pgfqpoint{1.038292in}{2.504367in}}%
\pgfpathcurveto{\pgfqpoint{1.046106in}{2.512180in}}{\pgfqpoint{1.050496in}{2.522779in}}{\pgfqpoint{1.050496in}{2.533829in}}%
\pgfpathcurveto{\pgfqpoint{1.050496in}{2.544880in}}{\pgfqpoint{1.046106in}{2.555479in}}{\pgfqpoint{1.038292in}{2.563292in}}%
\pgfpathcurveto{\pgfqpoint{1.030479in}{2.571106in}}{\pgfqpoint{1.019880in}{2.575496in}}{\pgfqpoint{1.008830in}{2.575496in}}%
\pgfpathcurveto{\pgfqpoint{0.997779in}{2.575496in}}{\pgfqpoint{0.987180in}{2.571106in}}{\pgfqpoint{0.979367in}{2.563292in}}%
\pgfpathcurveto{\pgfqpoint{0.971553in}{2.555479in}}{\pgfqpoint{0.967163in}{2.544880in}}{\pgfqpoint{0.967163in}{2.533829in}}%
\pgfpathcurveto{\pgfqpoint{0.967163in}{2.522779in}}{\pgfqpoint{0.971553in}{2.512180in}}{\pgfqpoint{0.979367in}{2.504367in}}%
\pgfpathcurveto{\pgfqpoint{0.987180in}{2.496553in}}{\pgfqpoint{0.997779in}{2.492163in}}{\pgfqpoint{1.008830in}{2.492163in}}%
\pgfpathclose%
\pgfusepath{stroke,fill}%
\end{pgfscope}%
\begin{pgfscope}%
\pgfpathrectangle{\pgfqpoint{0.511823in}{0.504323in}}{\pgfqpoint{3.218177in}{3.225677in}} %
\pgfusepath{clip}%
\pgfsetbuttcap%
\pgfsetroundjoin%
\definecolor{currentfill}{rgb}{0.000000,0.000000,0.545098}%
\pgfsetfillcolor{currentfill}%
\pgfsetfillopacity{0.400000}%
\pgfsetlinewidth{0.501875pt}%
\definecolor{currentstroke}{rgb}{0.000000,0.000000,0.545098}%
\pgfsetstrokecolor{currentstroke}%
\pgfsetstrokeopacity{0.400000}%
\pgfsetdash{}{0pt}%
\pgfpathmoveto{\pgfqpoint{1.012467in}{2.551174in}}%
\pgfpathcurveto{\pgfqpoint{1.023517in}{2.551174in}}{\pgfqpoint{1.034116in}{2.555564in}}{\pgfqpoint{1.041930in}{2.563378in}}%
\pgfpathcurveto{\pgfqpoint{1.049743in}{2.571191in}}{\pgfqpoint{1.054134in}{2.581790in}}{\pgfqpoint{1.054134in}{2.592840in}}%
\pgfpathcurveto{\pgfqpoint{1.054134in}{2.603890in}}{\pgfqpoint{1.049743in}{2.614489in}}{\pgfqpoint{1.041930in}{2.622303in}}%
\pgfpathcurveto{\pgfqpoint{1.034116in}{2.630117in}}{\pgfqpoint{1.023517in}{2.634507in}}{\pgfqpoint{1.012467in}{2.634507in}}%
\pgfpathcurveto{\pgfqpoint{1.001417in}{2.634507in}}{\pgfqpoint{0.990818in}{2.630117in}}{\pgfqpoint{0.983004in}{2.622303in}}%
\pgfpathcurveto{\pgfqpoint{0.975191in}{2.614489in}}{\pgfqpoint{0.970800in}{2.603890in}}{\pgfqpoint{0.970800in}{2.592840in}}%
\pgfpathcurveto{\pgfqpoint{0.970800in}{2.581790in}}{\pgfqpoint{0.975191in}{2.571191in}}{\pgfqpoint{0.983004in}{2.563378in}}%
\pgfpathcurveto{\pgfqpoint{0.990818in}{2.555564in}}{\pgfqpoint{1.001417in}{2.551174in}}{\pgfqpoint{1.012467in}{2.551174in}}%
\pgfpathclose%
\pgfusepath{stroke,fill}%
\end{pgfscope}%
\begin{pgfscope}%
\pgfpathrectangle{\pgfqpoint{0.511823in}{0.504323in}}{\pgfqpoint{3.218177in}{3.225677in}} %
\pgfusepath{clip}%
\pgfsetbuttcap%
\pgfsetroundjoin%
\definecolor{currentfill}{rgb}{0.000000,0.000000,0.545098}%
\pgfsetfillcolor{currentfill}%
\pgfsetfillopacity{0.400000}%
\pgfsetlinewidth{0.501875pt}%
\definecolor{currentstroke}{rgb}{0.000000,0.000000,0.545098}%
\pgfsetstrokecolor{currentstroke}%
\pgfsetstrokeopacity{0.400000}%
\pgfsetdash{}{0pt}%
\pgfpathmoveto{\pgfqpoint{1.019528in}{2.633085in}}%
\pgfpathcurveto{\pgfqpoint{1.030579in}{2.633085in}}{\pgfqpoint{1.041178in}{2.637476in}}{\pgfqpoint{1.048991in}{2.645289in}}%
\pgfpathcurveto{\pgfqpoint{1.056805in}{2.653103in}}{\pgfqpoint{1.061195in}{2.663702in}}{\pgfqpoint{1.061195in}{2.674752in}}%
\pgfpathcurveto{\pgfqpoint{1.061195in}{2.685802in}}{\pgfqpoint{1.056805in}{2.696401in}}{\pgfqpoint{1.048991in}{2.704215in}}%
\pgfpathcurveto{\pgfqpoint{1.041178in}{2.712029in}}{\pgfqpoint{1.030579in}{2.716419in}}{\pgfqpoint{1.019528in}{2.716419in}}%
\pgfpathcurveto{\pgfqpoint{1.008478in}{2.716419in}}{\pgfqpoint{0.997879in}{2.712029in}}{\pgfqpoint{0.990066in}{2.704215in}}%
\pgfpathcurveto{\pgfqpoint{0.982252in}{2.696401in}}{\pgfqpoint{0.977862in}{2.685802in}}{\pgfqpoint{0.977862in}{2.674752in}}%
\pgfpathcurveto{\pgfqpoint{0.977862in}{2.663702in}}{\pgfqpoint{0.982252in}{2.653103in}}{\pgfqpoint{0.990066in}{2.645289in}}%
\pgfpathcurveto{\pgfqpoint{0.997879in}{2.637476in}}{\pgfqpoint{1.008478in}{2.633085in}}{\pgfqpoint{1.019528in}{2.633085in}}%
\pgfpathclose%
\pgfusepath{stroke,fill}%
\end{pgfscope}%
\begin{pgfscope}%
\pgfpathrectangle{\pgfqpoint{0.511823in}{0.504323in}}{\pgfqpoint{3.218177in}{3.225677in}} %
\pgfusepath{clip}%
\pgfsetbuttcap%
\pgfsetroundjoin%
\definecolor{currentfill}{rgb}{0.000000,0.000000,0.545098}%
\pgfsetfillcolor{currentfill}%
\pgfsetfillopacity{0.400000}%
\pgfsetlinewidth{0.501875pt}%
\definecolor{currentstroke}{rgb}{0.000000,0.000000,0.545098}%
\pgfsetstrokecolor{currentstroke}%
\pgfsetstrokeopacity{0.400000}%
\pgfsetdash{}{0pt}%
\pgfpathmoveto{\pgfqpoint{0.966019in}{2.348274in}}%
\pgfpathcurveto{\pgfqpoint{0.977069in}{2.348274in}}{\pgfqpoint{0.987668in}{2.352665in}}{\pgfqpoint{0.995481in}{2.360478in}}%
\pgfpathcurveto{\pgfqpoint{1.003295in}{2.368292in}}{\pgfqpoint{1.007685in}{2.378891in}}{\pgfqpoint{1.007685in}{2.389941in}}%
\pgfpathcurveto{\pgfqpoint{1.007685in}{2.400991in}}{\pgfqpoint{1.003295in}{2.411590in}}{\pgfqpoint{0.995481in}{2.419404in}}%
\pgfpathcurveto{\pgfqpoint{0.987668in}{2.427217in}}{\pgfqpoint{0.977069in}{2.431608in}}{\pgfqpoint{0.966019in}{2.431608in}}%
\pgfpathcurveto{\pgfqpoint{0.954969in}{2.431608in}}{\pgfqpoint{0.944370in}{2.427217in}}{\pgfqpoint{0.936556in}{2.419404in}}%
\pgfpathcurveto{\pgfqpoint{0.928742in}{2.411590in}}{\pgfqpoint{0.924352in}{2.400991in}}{\pgfqpoint{0.924352in}{2.389941in}}%
\pgfpathcurveto{\pgfqpoint{0.924352in}{2.378891in}}{\pgfqpoint{0.928742in}{2.368292in}}{\pgfqpoint{0.936556in}{2.360478in}}%
\pgfpathcurveto{\pgfqpoint{0.944370in}{2.352665in}}{\pgfqpoint{0.954969in}{2.348274in}}{\pgfqpoint{0.966019in}{2.348274in}}%
\pgfpathclose%
\pgfusepath{stroke,fill}%
\end{pgfscope}%
\begin{pgfscope}%
\pgfpathrectangle{\pgfqpoint{0.511823in}{0.504323in}}{\pgfqpoint{3.218177in}{3.225677in}} %
\pgfusepath{clip}%
\pgfsetbuttcap%
\pgfsetroundjoin%
\definecolor{currentfill}{rgb}{0.000000,0.000000,0.545098}%
\pgfsetfillcolor{currentfill}%
\pgfsetfillopacity{0.400000}%
\pgfsetlinewidth{0.501875pt}%
\definecolor{currentstroke}{rgb}{0.000000,0.000000,0.545098}%
\pgfsetstrokecolor{currentstroke}%
\pgfsetstrokeopacity{0.400000}%
\pgfsetdash{}{0pt}%
\pgfpathmoveto{\pgfqpoint{1.013017in}{2.678587in}}%
\pgfpathcurveto{\pgfqpoint{1.024067in}{2.678587in}}{\pgfqpoint{1.034666in}{2.682977in}}{\pgfqpoint{1.042479in}{2.690791in}}%
\pgfpathcurveto{\pgfqpoint{1.050293in}{2.698605in}}{\pgfqpoint{1.054683in}{2.709204in}}{\pgfqpoint{1.054683in}{2.720254in}}%
\pgfpathcurveto{\pgfqpoint{1.054683in}{2.731304in}}{\pgfqpoint{1.050293in}{2.741903in}}{\pgfqpoint{1.042479in}{2.749717in}}%
\pgfpathcurveto{\pgfqpoint{1.034666in}{2.757530in}}{\pgfqpoint{1.024067in}{2.761921in}}{\pgfqpoint{1.013017in}{2.761921in}}%
\pgfpathcurveto{\pgfqpoint{1.001966in}{2.761921in}}{\pgfqpoint{0.991367in}{2.757530in}}{\pgfqpoint{0.983554in}{2.749717in}}%
\pgfpathcurveto{\pgfqpoint{0.975740in}{2.741903in}}{\pgfqpoint{0.971350in}{2.731304in}}{\pgfqpoint{0.971350in}{2.720254in}}%
\pgfpathcurveto{\pgfqpoint{0.971350in}{2.709204in}}{\pgfqpoint{0.975740in}{2.698605in}}{\pgfqpoint{0.983554in}{2.690791in}}%
\pgfpathcurveto{\pgfqpoint{0.991367in}{2.682977in}}{\pgfqpoint{1.001966in}{2.678587in}}{\pgfqpoint{1.013017in}{2.678587in}}%
\pgfpathclose%
\pgfusepath{stroke,fill}%
\end{pgfscope}%
\begin{pgfscope}%
\pgfpathrectangle{\pgfqpoint{0.511823in}{0.504323in}}{\pgfqpoint{3.218177in}{3.225677in}} %
\pgfusepath{clip}%
\pgfsetbuttcap%
\pgfsetroundjoin%
\definecolor{currentfill}{rgb}{0.000000,0.000000,0.545098}%
\pgfsetfillcolor{currentfill}%
\pgfsetfillopacity{0.400000}%
\pgfsetlinewidth{0.501875pt}%
\definecolor{currentstroke}{rgb}{0.000000,0.000000,0.545098}%
\pgfsetstrokecolor{currentstroke}%
\pgfsetstrokeopacity{0.400000}%
\pgfsetdash{}{0pt}%
\pgfpathmoveto{\pgfqpoint{0.968104in}{2.436982in}}%
\pgfpathcurveto{\pgfqpoint{0.979154in}{2.436982in}}{\pgfqpoint{0.989753in}{2.441372in}}{\pgfqpoint{0.997567in}{2.449186in}}%
\pgfpathcurveto{\pgfqpoint{1.005380in}{2.456999in}}{\pgfqpoint{1.009771in}{2.467598in}}{\pgfqpoint{1.009771in}{2.478648in}}%
\pgfpathcurveto{\pgfqpoint{1.009771in}{2.489699in}}{\pgfqpoint{1.005380in}{2.500298in}}{\pgfqpoint{0.997567in}{2.508111in}}%
\pgfpathcurveto{\pgfqpoint{0.989753in}{2.515925in}}{\pgfqpoint{0.979154in}{2.520315in}}{\pgfqpoint{0.968104in}{2.520315in}}%
\pgfpathcurveto{\pgfqpoint{0.957054in}{2.520315in}}{\pgfqpoint{0.946455in}{2.515925in}}{\pgfqpoint{0.938641in}{2.508111in}}%
\pgfpathcurveto{\pgfqpoint{0.930828in}{2.500298in}}{\pgfqpoint{0.926437in}{2.489699in}}{\pgfqpoint{0.926437in}{2.478648in}}%
\pgfpathcurveto{\pgfqpoint{0.926437in}{2.467598in}}{\pgfqpoint{0.930828in}{2.456999in}}{\pgfqpoint{0.938641in}{2.449186in}}%
\pgfpathcurveto{\pgfqpoint{0.946455in}{2.441372in}}{\pgfqpoint{0.957054in}{2.436982in}}{\pgfqpoint{0.968104in}{2.436982in}}%
\pgfpathclose%
\pgfusepath{stroke,fill}%
\end{pgfscope}%
\begin{pgfscope}%
\pgfpathrectangle{\pgfqpoint{0.511823in}{0.504323in}}{\pgfqpoint{3.218177in}{3.225677in}} %
\pgfusepath{clip}%
\pgfsetbuttcap%
\pgfsetroundjoin%
\definecolor{currentfill}{rgb}{0.000000,0.000000,0.545098}%
\pgfsetfillcolor{currentfill}%
\pgfsetfillopacity{0.400000}%
\pgfsetlinewidth{0.501875pt}%
\definecolor{currentstroke}{rgb}{0.000000,0.000000,0.545098}%
\pgfsetstrokecolor{currentstroke}%
\pgfsetstrokeopacity{0.400000}%
\pgfsetdash{}{0pt}%
\pgfpathmoveto{\pgfqpoint{0.962670in}{2.442023in}}%
\pgfpathcurveto{\pgfqpoint{0.973720in}{2.442023in}}{\pgfqpoint{0.984319in}{2.446414in}}{\pgfqpoint{0.992133in}{2.454227in}}%
\pgfpathcurveto{\pgfqpoint{0.999946in}{2.462041in}}{\pgfqpoint{1.004337in}{2.472640in}}{\pgfqpoint{1.004337in}{2.483690in}}%
\pgfpathcurveto{\pgfqpoint{1.004337in}{2.494740in}}{\pgfqpoint{0.999946in}{2.505339in}}{\pgfqpoint{0.992133in}{2.513153in}}%
\pgfpathcurveto{\pgfqpoint{0.984319in}{2.520967in}}{\pgfqpoint{0.973720in}{2.525357in}}{\pgfqpoint{0.962670in}{2.525357in}}%
\pgfpathcurveto{\pgfqpoint{0.951620in}{2.525357in}}{\pgfqpoint{0.941021in}{2.520967in}}{\pgfqpoint{0.933207in}{2.513153in}}%
\pgfpathcurveto{\pgfqpoint{0.925394in}{2.505339in}}{\pgfqpoint{0.921003in}{2.494740in}}{\pgfqpoint{0.921003in}{2.483690in}}%
\pgfpathcurveto{\pgfqpoint{0.921003in}{2.472640in}}{\pgfqpoint{0.925394in}{2.462041in}}{\pgfqpoint{0.933207in}{2.454227in}}%
\pgfpathcurveto{\pgfqpoint{0.941021in}{2.446414in}}{\pgfqpoint{0.951620in}{2.442023in}}{\pgfqpoint{0.962670in}{2.442023in}}%
\pgfpathclose%
\pgfusepath{stroke,fill}%
\end{pgfscope}%
\begin{pgfscope}%
\pgfpathrectangle{\pgfqpoint{0.511823in}{0.504323in}}{\pgfqpoint{3.218177in}{3.225677in}} %
\pgfusepath{clip}%
\pgfsetbuttcap%
\pgfsetroundjoin%
\definecolor{currentfill}{rgb}{0.000000,0.000000,0.545098}%
\pgfsetfillcolor{currentfill}%
\pgfsetfillopacity{0.400000}%
\pgfsetlinewidth{0.501875pt}%
\definecolor{currentstroke}{rgb}{0.000000,0.000000,0.545098}%
\pgfsetstrokecolor{currentstroke}%
\pgfsetstrokeopacity{0.400000}%
\pgfsetdash{}{0pt}%
\pgfpathmoveto{\pgfqpoint{0.939046in}{2.325849in}}%
\pgfpathcurveto{\pgfqpoint{0.950096in}{2.325849in}}{\pgfqpoint{0.960695in}{2.330239in}}{\pgfqpoint{0.968509in}{2.338053in}}%
\pgfpathcurveto{\pgfqpoint{0.976323in}{2.345866in}}{\pgfqpoint{0.980713in}{2.356465in}}{\pgfqpoint{0.980713in}{2.367515in}}%
\pgfpathcurveto{\pgfqpoint{0.980713in}{2.378566in}}{\pgfqpoint{0.976323in}{2.389165in}}{\pgfqpoint{0.968509in}{2.396978in}}%
\pgfpathcurveto{\pgfqpoint{0.960695in}{2.404792in}}{\pgfqpoint{0.950096in}{2.409182in}}{\pgfqpoint{0.939046in}{2.409182in}}%
\pgfpathcurveto{\pgfqpoint{0.927996in}{2.409182in}}{\pgfqpoint{0.917397in}{2.404792in}}{\pgfqpoint{0.909583in}{2.396978in}}%
\pgfpathcurveto{\pgfqpoint{0.901770in}{2.389165in}}{\pgfqpoint{0.897380in}{2.378566in}}{\pgfqpoint{0.897380in}{2.367515in}}%
\pgfpathcurveto{\pgfqpoint{0.897380in}{2.356465in}}{\pgfqpoint{0.901770in}{2.345866in}}{\pgfqpoint{0.909583in}{2.338053in}}%
\pgfpathcurveto{\pgfqpoint{0.917397in}{2.330239in}}{\pgfqpoint{0.927996in}{2.325849in}}{\pgfqpoint{0.939046in}{2.325849in}}%
\pgfpathclose%
\pgfusepath{stroke,fill}%
\end{pgfscope}%
\begin{pgfscope}%
\pgfpathrectangle{\pgfqpoint{0.511823in}{0.504323in}}{\pgfqpoint{3.218177in}{3.225677in}} %
\pgfusepath{clip}%
\pgfsetbuttcap%
\pgfsetroundjoin%
\definecolor{currentfill}{rgb}{0.000000,0.000000,0.545098}%
\pgfsetfillcolor{currentfill}%
\pgfsetfillopacity{0.400000}%
\pgfsetlinewidth{0.501875pt}%
\definecolor{currentstroke}{rgb}{0.000000,0.000000,0.545098}%
\pgfsetstrokecolor{currentstroke}%
\pgfsetstrokeopacity{0.400000}%
\pgfsetdash{}{0pt}%
\pgfpathmoveto{\pgfqpoint{0.953761in}{2.466091in}}%
\pgfpathcurveto{\pgfqpoint{0.964811in}{2.466091in}}{\pgfqpoint{0.975410in}{2.470481in}}{\pgfqpoint{0.983224in}{2.478295in}}%
\pgfpathcurveto{\pgfqpoint{0.991037in}{2.486109in}}{\pgfqpoint{0.995428in}{2.496708in}}{\pgfqpoint{0.995428in}{2.507758in}}%
\pgfpathcurveto{\pgfqpoint{0.995428in}{2.518808in}}{\pgfqpoint{0.991037in}{2.529407in}}{\pgfqpoint{0.983224in}{2.537221in}}%
\pgfpathcurveto{\pgfqpoint{0.975410in}{2.545034in}}{\pgfqpoint{0.964811in}{2.549425in}}{\pgfqpoint{0.953761in}{2.549425in}}%
\pgfpathcurveto{\pgfqpoint{0.942711in}{2.549425in}}{\pgfqpoint{0.932112in}{2.545034in}}{\pgfqpoint{0.924298in}{2.537221in}}%
\pgfpathcurveto{\pgfqpoint{0.916485in}{2.529407in}}{\pgfqpoint{0.912094in}{2.518808in}}{\pgfqpoint{0.912094in}{2.507758in}}%
\pgfpathcurveto{\pgfqpoint{0.912094in}{2.496708in}}{\pgfqpoint{0.916485in}{2.486109in}}{\pgfqpoint{0.924298in}{2.478295in}}%
\pgfpathcurveto{\pgfqpoint{0.932112in}{2.470481in}}{\pgfqpoint{0.942711in}{2.466091in}}{\pgfqpoint{0.953761in}{2.466091in}}%
\pgfpathclose%
\pgfusepath{stroke,fill}%
\end{pgfscope}%
\begin{pgfscope}%
\pgfpathrectangle{\pgfqpoint{0.511823in}{0.504323in}}{\pgfqpoint{3.218177in}{3.225677in}} %
\pgfusepath{clip}%
\pgfsetbuttcap%
\pgfsetroundjoin%
\definecolor{currentfill}{rgb}{0.000000,0.000000,0.545098}%
\pgfsetfillcolor{currentfill}%
\pgfsetfillopacity{0.400000}%
\pgfsetlinewidth{0.501875pt}%
\definecolor{currentstroke}{rgb}{0.000000,0.000000,0.545098}%
\pgfsetstrokecolor{currentstroke}%
\pgfsetstrokeopacity{0.400000}%
\pgfsetdash{}{0pt}%
\pgfpathmoveto{\pgfqpoint{0.944776in}{2.447240in}}%
\pgfpathcurveto{\pgfqpoint{0.955826in}{2.447240in}}{\pgfqpoint{0.966425in}{2.451630in}}{\pgfqpoint{0.974239in}{2.459444in}}%
\pgfpathcurveto{\pgfqpoint{0.982052in}{2.467257in}}{\pgfqpoint{0.986443in}{2.477856in}}{\pgfqpoint{0.986443in}{2.488906in}}%
\pgfpathcurveto{\pgfqpoint{0.986443in}{2.499957in}}{\pgfqpoint{0.982052in}{2.510556in}}{\pgfqpoint{0.974239in}{2.518369in}}%
\pgfpathcurveto{\pgfqpoint{0.966425in}{2.526183in}}{\pgfqpoint{0.955826in}{2.530573in}}{\pgfqpoint{0.944776in}{2.530573in}}%
\pgfpathcurveto{\pgfqpoint{0.933726in}{2.530573in}}{\pgfqpoint{0.923127in}{2.526183in}}{\pgfqpoint{0.915313in}{2.518369in}}%
\pgfpathcurveto{\pgfqpoint{0.907500in}{2.510556in}}{\pgfqpoint{0.903109in}{2.499957in}}{\pgfqpoint{0.903109in}{2.488906in}}%
\pgfpathcurveto{\pgfqpoint{0.903109in}{2.477856in}}{\pgfqpoint{0.907500in}{2.467257in}}{\pgfqpoint{0.915313in}{2.459444in}}%
\pgfpathcurveto{\pgfqpoint{0.923127in}{2.451630in}}{\pgfqpoint{0.933726in}{2.447240in}}{\pgfqpoint{0.944776in}{2.447240in}}%
\pgfpathclose%
\pgfusepath{stroke,fill}%
\end{pgfscope}%
\begin{pgfscope}%
\pgfpathrectangle{\pgfqpoint{0.511823in}{0.504323in}}{\pgfqpoint{3.218177in}{3.225677in}} %
\pgfusepath{clip}%
\pgfsetbuttcap%
\pgfsetroundjoin%
\definecolor{currentfill}{rgb}{0.000000,0.000000,0.545098}%
\pgfsetfillcolor{currentfill}%
\pgfsetfillopacity{0.400000}%
\pgfsetlinewidth{0.501875pt}%
\definecolor{currentstroke}{rgb}{0.000000,0.000000,0.545098}%
\pgfsetstrokecolor{currentstroke}%
\pgfsetstrokeopacity{0.400000}%
\pgfsetdash{}{0pt}%
\pgfpathmoveto{\pgfqpoint{0.921477in}{2.324803in}}%
\pgfpathcurveto{\pgfqpoint{0.932528in}{2.324803in}}{\pgfqpoint{0.943127in}{2.329194in}}{\pgfqpoint{0.950940in}{2.337007in}}%
\pgfpathcurveto{\pgfqpoint{0.958754in}{2.344821in}}{\pgfqpoint{0.963144in}{2.355420in}}{\pgfqpoint{0.963144in}{2.366470in}}%
\pgfpathcurveto{\pgfqpoint{0.963144in}{2.377520in}}{\pgfqpoint{0.958754in}{2.388119in}}{\pgfqpoint{0.950940in}{2.395933in}}%
\pgfpathcurveto{\pgfqpoint{0.943127in}{2.403747in}}{\pgfqpoint{0.932528in}{2.408137in}}{\pgfqpoint{0.921477in}{2.408137in}}%
\pgfpathcurveto{\pgfqpoint{0.910427in}{2.408137in}}{\pgfqpoint{0.899828in}{2.403747in}}{\pgfqpoint{0.892015in}{2.395933in}}%
\pgfpathcurveto{\pgfqpoint{0.884201in}{2.388119in}}{\pgfqpoint{0.879811in}{2.377520in}}{\pgfqpoint{0.879811in}{2.366470in}}%
\pgfpathcurveto{\pgfqpoint{0.879811in}{2.355420in}}{\pgfqpoint{0.884201in}{2.344821in}}{\pgfqpoint{0.892015in}{2.337007in}}%
\pgfpathcurveto{\pgfqpoint{0.899828in}{2.329194in}}{\pgfqpoint{0.910427in}{2.324803in}}{\pgfqpoint{0.921477in}{2.324803in}}%
\pgfpathclose%
\pgfusepath{stroke,fill}%
\end{pgfscope}%
\begin{pgfscope}%
\pgfpathrectangle{\pgfqpoint{0.511823in}{0.504323in}}{\pgfqpoint{3.218177in}{3.225677in}} %
\pgfusepath{clip}%
\pgfsetbuttcap%
\pgfsetroundjoin%
\definecolor{currentfill}{rgb}{0.000000,0.000000,0.545098}%
\pgfsetfillcolor{currentfill}%
\pgfsetfillopacity{0.400000}%
\pgfsetlinewidth{0.501875pt}%
\definecolor{currentstroke}{rgb}{0.000000,0.000000,0.545098}%
\pgfsetstrokecolor{currentstroke}%
\pgfsetstrokeopacity{0.400000}%
\pgfsetdash{}{0pt}%
\pgfpathmoveto{\pgfqpoint{0.931488in}{2.440978in}}%
\pgfpathcurveto{\pgfqpoint{0.942538in}{2.440978in}}{\pgfqpoint{0.953137in}{2.445368in}}{\pgfqpoint{0.960951in}{2.453182in}}%
\pgfpathcurveto{\pgfqpoint{0.968764in}{2.460996in}}{\pgfqpoint{0.973155in}{2.471595in}}{\pgfqpoint{0.973155in}{2.482645in}}%
\pgfpathcurveto{\pgfqpoint{0.973155in}{2.493695in}}{\pgfqpoint{0.968764in}{2.504294in}}{\pgfqpoint{0.960951in}{2.512107in}}%
\pgfpathcurveto{\pgfqpoint{0.953137in}{2.519921in}}{\pgfqpoint{0.942538in}{2.524311in}}{\pgfqpoint{0.931488in}{2.524311in}}%
\pgfpathcurveto{\pgfqpoint{0.920438in}{2.524311in}}{\pgfqpoint{0.909839in}{2.519921in}}{\pgfqpoint{0.902025in}{2.512107in}}%
\pgfpathcurveto{\pgfqpoint{0.894212in}{2.504294in}}{\pgfqpoint{0.889821in}{2.493695in}}{\pgfqpoint{0.889821in}{2.482645in}}%
\pgfpathcurveto{\pgfqpoint{0.889821in}{2.471595in}}{\pgfqpoint{0.894212in}{2.460996in}}{\pgfqpoint{0.902025in}{2.453182in}}%
\pgfpathcurveto{\pgfqpoint{0.909839in}{2.445368in}}{\pgfqpoint{0.920438in}{2.440978in}}{\pgfqpoint{0.931488in}{2.440978in}}%
\pgfpathclose%
\pgfusepath{stroke,fill}%
\end{pgfscope}%
\begin{pgfscope}%
\pgfpathrectangle{\pgfqpoint{0.511823in}{0.504323in}}{\pgfqpoint{3.218177in}{3.225677in}} %
\pgfusepath{clip}%
\pgfsetbuttcap%
\pgfsetroundjoin%
\definecolor{currentfill}{rgb}{0.000000,0.000000,0.545098}%
\pgfsetfillcolor{currentfill}%
\pgfsetfillopacity{0.400000}%
\pgfsetlinewidth{0.501875pt}%
\definecolor{currentstroke}{rgb}{0.000000,0.000000,0.545098}%
\pgfsetstrokecolor{currentstroke}%
\pgfsetstrokeopacity{0.400000}%
\pgfsetdash{}{0pt}%
\pgfpathmoveto{\pgfqpoint{0.932098in}{2.492251in}}%
\pgfpathcurveto{\pgfqpoint{0.943148in}{2.492251in}}{\pgfqpoint{0.953748in}{2.496641in}}{\pgfqpoint{0.961561in}{2.504455in}}%
\pgfpathcurveto{\pgfqpoint{0.969375in}{2.512268in}}{\pgfqpoint{0.973765in}{2.522868in}}{\pgfqpoint{0.973765in}{2.533918in}}%
\pgfpathcurveto{\pgfqpoint{0.973765in}{2.544968in}}{\pgfqpoint{0.969375in}{2.555567in}}{\pgfqpoint{0.961561in}{2.563380in}}%
\pgfpathcurveto{\pgfqpoint{0.953748in}{2.571194in}}{\pgfqpoint{0.943148in}{2.575584in}}{\pgfqpoint{0.932098in}{2.575584in}}%
\pgfpathcurveto{\pgfqpoint{0.921048in}{2.575584in}}{\pgfqpoint{0.910449in}{2.571194in}}{\pgfqpoint{0.902636in}{2.563380in}}%
\pgfpathcurveto{\pgfqpoint{0.894822in}{2.555567in}}{\pgfqpoint{0.890432in}{2.544968in}}{\pgfqpoint{0.890432in}{2.533918in}}%
\pgfpathcurveto{\pgfqpoint{0.890432in}{2.522868in}}{\pgfqpoint{0.894822in}{2.512268in}}{\pgfqpoint{0.902636in}{2.504455in}}%
\pgfpathcurveto{\pgfqpoint{0.910449in}{2.496641in}}{\pgfqpoint{0.921048in}{2.492251in}}{\pgfqpoint{0.932098in}{2.492251in}}%
\pgfpathclose%
\pgfusepath{stroke,fill}%
\end{pgfscope}%
\begin{pgfscope}%
\pgfpathrectangle{\pgfqpoint{0.511823in}{0.504323in}}{\pgfqpoint{3.218177in}{3.225677in}} %
\pgfusepath{clip}%
\pgfsetbuttcap%
\pgfsetroundjoin%
\definecolor{currentfill}{rgb}{0.000000,0.000000,0.545098}%
\pgfsetfillcolor{currentfill}%
\pgfsetfillopacity{0.400000}%
\pgfsetlinewidth{0.501875pt}%
\definecolor{currentstroke}{rgb}{0.000000,0.000000,0.545098}%
\pgfsetstrokecolor{currentstroke}%
\pgfsetstrokeopacity{0.400000}%
\pgfsetdash{}{0pt}%
\pgfpathmoveto{\pgfqpoint{0.917395in}{2.427665in}}%
\pgfpathcurveto{\pgfqpoint{0.928445in}{2.427665in}}{\pgfqpoint{0.939044in}{2.432055in}}{\pgfqpoint{0.946858in}{2.439869in}}%
\pgfpathcurveto{\pgfqpoint{0.954672in}{2.447683in}}{\pgfqpoint{0.959062in}{2.458282in}}{\pgfqpoint{0.959062in}{2.469332in}}%
\pgfpathcurveto{\pgfqpoint{0.959062in}{2.480382in}}{\pgfqpoint{0.954672in}{2.490981in}}{\pgfqpoint{0.946858in}{2.498795in}}%
\pgfpathcurveto{\pgfqpoint{0.939044in}{2.506608in}}{\pgfqpoint{0.928445in}{2.510998in}}{\pgfqpoint{0.917395in}{2.510998in}}%
\pgfpathcurveto{\pgfqpoint{0.906345in}{2.510998in}}{\pgfqpoint{0.895746in}{2.506608in}}{\pgfqpoint{0.887932in}{2.498795in}}%
\pgfpathcurveto{\pgfqpoint{0.880119in}{2.490981in}}{\pgfqpoint{0.875728in}{2.480382in}}{\pgfqpoint{0.875728in}{2.469332in}}%
\pgfpathcurveto{\pgfqpoint{0.875728in}{2.458282in}}{\pgfqpoint{0.880119in}{2.447683in}}{\pgfqpoint{0.887932in}{2.439869in}}%
\pgfpathcurveto{\pgfqpoint{0.895746in}{2.432055in}}{\pgfqpoint{0.906345in}{2.427665in}}{\pgfqpoint{0.917395in}{2.427665in}}%
\pgfpathclose%
\pgfusepath{stroke,fill}%
\end{pgfscope}%
\begin{pgfscope}%
\pgfpathrectangle{\pgfqpoint{0.511823in}{0.504323in}}{\pgfqpoint{3.218177in}{3.225677in}} %
\pgfusepath{clip}%
\pgfsetbuttcap%
\pgfsetroundjoin%
\definecolor{currentfill}{rgb}{0.000000,0.000000,0.545098}%
\pgfsetfillcolor{currentfill}%
\pgfsetfillopacity{0.400000}%
\pgfsetlinewidth{0.501875pt}%
\definecolor{currentstroke}{rgb}{0.000000,0.000000,0.545098}%
\pgfsetstrokecolor{currentstroke}%
\pgfsetstrokeopacity{0.400000}%
\pgfsetdash{}{0pt}%
\pgfpathmoveto{\pgfqpoint{0.929702in}{2.574313in}}%
\pgfpathcurveto{\pgfqpoint{0.940752in}{2.574313in}}{\pgfqpoint{0.951351in}{2.578703in}}{\pgfqpoint{0.959164in}{2.586517in}}%
\pgfpathcurveto{\pgfqpoint{0.966978in}{2.594330in}}{\pgfqpoint{0.971368in}{2.604929in}}{\pgfqpoint{0.971368in}{2.615980in}}%
\pgfpathcurveto{\pgfqpoint{0.971368in}{2.627030in}}{\pgfqpoint{0.966978in}{2.637629in}}{\pgfqpoint{0.959164in}{2.645442in}}%
\pgfpathcurveto{\pgfqpoint{0.951351in}{2.653256in}}{\pgfqpoint{0.940752in}{2.657646in}}{\pgfqpoint{0.929702in}{2.657646in}}%
\pgfpathcurveto{\pgfqpoint{0.918652in}{2.657646in}}{\pgfqpoint{0.908052in}{2.653256in}}{\pgfqpoint{0.900239in}{2.645442in}}%
\pgfpathcurveto{\pgfqpoint{0.892425in}{2.637629in}}{\pgfqpoint{0.888035in}{2.627030in}}{\pgfqpoint{0.888035in}{2.615980in}}%
\pgfpathcurveto{\pgfqpoint{0.888035in}{2.604929in}}{\pgfqpoint{0.892425in}{2.594330in}}{\pgfqpoint{0.900239in}{2.586517in}}%
\pgfpathcurveto{\pgfqpoint{0.908052in}{2.578703in}}{\pgfqpoint{0.918652in}{2.574313in}}{\pgfqpoint{0.929702in}{2.574313in}}%
\pgfpathclose%
\pgfusepath{stroke,fill}%
\end{pgfscope}%
\begin{pgfscope}%
\pgfpathrectangle{\pgfqpoint{0.511823in}{0.504323in}}{\pgfqpoint{3.218177in}{3.225677in}} %
\pgfusepath{clip}%
\pgfsetbuttcap%
\pgfsetroundjoin%
\definecolor{currentfill}{rgb}{0.000000,0.000000,0.545098}%
\pgfsetfillcolor{currentfill}%
\pgfsetfillopacity{0.400000}%
\pgfsetlinewidth{0.501875pt}%
\definecolor{currentstroke}{rgb}{0.000000,0.000000,0.545098}%
\pgfsetstrokecolor{currentstroke}%
\pgfsetstrokeopacity{0.400000}%
\pgfsetdash{}{0pt}%
\pgfpathmoveto{\pgfqpoint{0.896093in}{2.353770in}}%
\pgfpathcurveto{\pgfqpoint{0.907143in}{2.353770in}}{\pgfqpoint{0.917742in}{2.358160in}}{\pgfqpoint{0.925556in}{2.365974in}}%
\pgfpathcurveto{\pgfqpoint{0.933370in}{2.373788in}}{\pgfqpoint{0.937760in}{2.384387in}}{\pgfqpoint{0.937760in}{2.395437in}}%
\pgfpathcurveto{\pgfqpoint{0.937760in}{2.406487in}}{\pgfqpoint{0.933370in}{2.417086in}}{\pgfqpoint{0.925556in}{2.424900in}}%
\pgfpathcurveto{\pgfqpoint{0.917742in}{2.432713in}}{\pgfqpoint{0.907143in}{2.437103in}}{\pgfqpoint{0.896093in}{2.437103in}}%
\pgfpathcurveto{\pgfqpoint{0.885043in}{2.437103in}}{\pgfqpoint{0.874444in}{2.432713in}}{\pgfqpoint{0.866630in}{2.424900in}}%
\pgfpathcurveto{\pgfqpoint{0.858817in}{2.417086in}}{\pgfqpoint{0.854427in}{2.406487in}}{\pgfqpoint{0.854427in}{2.395437in}}%
\pgfpathcurveto{\pgfqpoint{0.854427in}{2.384387in}}{\pgfqpoint{0.858817in}{2.373788in}}{\pgfqpoint{0.866630in}{2.365974in}}%
\pgfpathcurveto{\pgfqpoint{0.874444in}{2.358160in}}{\pgfqpoint{0.885043in}{2.353770in}}{\pgfqpoint{0.896093in}{2.353770in}}%
\pgfpathclose%
\pgfusepath{stroke,fill}%
\end{pgfscope}%
\begin{pgfscope}%
\pgfpathrectangle{\pgfqpoint{0.511823in}{0.504323in}}{\pgfqpoint{3.218177in}{3.225677in}} %
\pgfusepath{clip}%
\pgfsetbuttcap%
\pgfsetroundjoin%
\definecolor{currentfill}{rgb}{0.000000,0.000000,0.545098}%
\pgfsetfillcolor{currentfill}%
\pgfsetfillopacity{0.400000}%
\pgfsetlinewidth{0.501875pt}%
\definecolor{currentstroke}{rgb}{0.000000,0.000000,0.545098}%
\pgfsetstrokecolor{currentstroke}%
\pgfsetstrokeopacity{0.400000}%
\pgfsetdash{}{0pt}%
\pgfpathmoveto{\pgfqpoint{0.918221in}{2.589272in}}%
\pgfpathcurveto{\pgfqpoint{0.929271in}{2.589272in}}{\pgfqpoint{0.939870in}{2.593662in}}{\pgfqpoint{0.947684in}{2.601475in}}%
\pgfpathcurveto{\pgfqpoint{0.955497in}{2.609289in}}{\pgfqpoint{0.959887in}{2.619888in}}{\pgfqpoint{0.959887in}{2.630938in}}%
\pgfpathcurveto{\pgfqpoint{0.959887in}{2.641988in}}{\pgfqpoint{0.955497in}{2.652587in}}{\pgfqpoint{0.947684in}{2.660401in}}%
\pgfpathcurveto{\pgfqpoint{0.939870in}{2.668215in}}{\pgfqpoint{0.929271in}{2.672605in}}{\pgfqpoint{0.918221in}{2.672605in}}%
\pgfpathcurveto{\pgfqpoint{0.907171in}{2.672605in}}{\pgfqpoint{0.896572in}{2.668215in}}{\pgfqpoint{0.888758in}{2.660401in}}%
\pgfpathcurveto{\pgfqpoint{0.880944in}{2.652587in}}{\pgfqpoint{0.876554in}{2.641988in}}{\pgfqpoint{0.876554in}{2.630938in}}%
\pgfpathcurveto{\pgfqpoint{0.876554in}{2.619888in}}{\pgfqpoint{0.880944in}{2.609289in}}{\pgfqpoint{0.888758in}{2.601475in}}%
\pgfpathcurveto{\pgfqpoint{0.896572in}{2.593662in}}{\pgfqpoint{0.907171in}{2.589272in}}{\pgfqpoint{0.918221in}{2.589272in}}%
\pgfpathclose%
\pgfusepath{stroke,fill}%
\end{pgfscope}%
\begin{pgfscope}%
\pgfpathrectangle{\pgfqpoint{0.511823in}{0.504323in}}{\pgfqpoint{3.218177in}{3.225677in}} %
\pgfusepath{clip}%
\pgfsetbuttcap%
\pgfsetroundjoin%
\definecolor{currentfill}{rgb}{0.000000,0.000000,0.545098}%
\pgfsetfillcolor{currentfill}%
\pgfsetfillopacity{0.400000}%
\pgfsetlinewidth{0.501875pt}%
\definecolor{currentstroke}{rgb}{0.000000,0.000000,0.545098}%
\pgfsetstrokecolor{currentstroke}%
\pgfsetstrokeopacity{0.400000}%
\pgfsetdash{}{0pt}%
\pgfpathmoveto{\pgfqpoint{0.896377in}{2.457899in}}%
\pgfpathcurveto{\pgfqpoint{0.907427in}{2.457899in}}{\pgfqpoint{0.918027in}{2.462289in}}{\pgfqpoint{0.925840in}{2.470103in}}%
\pgfpathcurveto{\pgfqpoint{0.933654in}{2.477916in}}{\pgfqpoint{0.938044in}{2.488516in}}{\pgfqpoint{0.938044in}{2.499566in}}%
\pgfpathcurveto{\pgfqpoint{0.938044in}{2.510616in}}{\pgfqpoint{0.933654in}{2.521215in}}{\pgfqpoint{0.925840in}{2.529028in}}%
\pgfpathcurveto{\pgfqpoint{0.918027in}{2.536842in}}{\pgfqpoint{0.907427in}{2.541232in}}{\pgfqpoint{0.896377in}{2.541232in}}%
\pgfpathcurveto{\pgfqpoint{0.885327in}{2.541232in}}{\pgfqpoint{0.874728in}{2.536842in}}{\pgfqpoint{0.866915in}{2.529028in}}%
\pgfpathcurveto{\pgfqpoint{0.859101in}{2.521215in}}{\pgfqpoint{0.854711in}{2.510616in}}{\pgfqpoint{0.854711in}{2.499566in}}%
\pgfpathcurveto{\pgfqpoint{0.854711in}{2.488516in}}{\pgfqpoint{0.859101in}{2.477916in}}{\pgfqpoint{0.866915in}{2.470103in}}%
\pgfpathcurveto{\pgfqpoint{0.874728in}{2.462289in}}{\pgfqpoint{0.885327in}{2.457899in}}{\pgfqpoint{0.896377in}{2.457899in}}%
\pgfpathclose%
\pgfusepath{stroke,fill}%
\end{pgfscope}%
\begin{pgfscope}%
\pgfpathrectangle{\pgfqpoint{0.511823in}{0.504323in}}{\pgfqpoint{3.218177in}{3.225677in}} %
\pgfusepath{clip}%
\pgfsetbuttcap%
\pgfsetroundjoin%
\definecolor{currentfill}{rgb}{0.000000,0.000000,0.545098}%
\pgfsetfillcolor{currentfill}%
\pgfsetfillopacity{0.400000}%
\pgfsetlinewidth{0.501875pt}%
\definecolor{currentstroke}{rgb}{0.000000,0.000000,0.545098}%
\pgfsetstrokecolor{currentstroke}%
\pgfsetstrokeopacity{0.400000}%
\pgfsetdash{}{0pt}%
\pgfpathmoveto{\pgfqpoint{0.897905in}{2.527042in}}%
\pgfpathcurveto{\pgfqpoint{0.908955in}{2.527042in}}{\pgfqpoint{0.919554in}{2.531433in}}{\pgfqpoint{0.927368in}{2.539246in}}%
\pgfpathcurveto{\pgfqpoint{0.935181in}{2.547060in}}{\pgfqpoint{0.939571in}{2.557659in}}{\pgfqpoint{0.939571in}{2.568709in}}%
\pgfpathcurveto{\pgfqpoint{0.939571in}{2.579759in}}{\pgfqpoint{0.935181in}{2.590358in}}{\pgfqpoint{0.927368in}{2.598172in}}%
\pgfpathcurveto{\pgfqpoint{0.919554in}{2.605986in}}{\pgfqpoint{0.908955in}{2.610376in}}{\pgfqpoint{0.897905in}{2.610376in}}%
\pgfpathcurveto{\pgfqpoint{0.886855in}{2.610376in}}{\pgfqpoint{0.876256in}{2.605986in}}{\pgfqpoint{0.868442in}{2.598172in}}%
\pgfpathcurveto{\pgfqpoint{0.860628in}{2.590358in}}{\pgfqpoint{0.856238in}{2.579759in}}{\pgfqpoint{0.856238in}{2.568709in}}%
\pgfpathcurveto{\pgfqpoint{0.856238in}{2.557659in}}{\pgfqpoint{0.860628in}{2.547060in}}{\pgfqpoint{0.868442in}{2.539246in}}%
\pgfpathcurveto{\pgfqpoint{0.876256in}{2.531433in}}{\pgfqpoint{0.886855in}{2.527042in}}{\pgfqpoint{0.897905in}{2.527042in}}%
\pgfpathclose%
\pgfusepath{stroke,fill}%
\end{pgfscope}%
\begin{pgfscope}%
\pgfpathrectangle{\pgfqpoint{0.511823in}{0.504323in}}{\pgfqpoint{3.218177in}{3.225677in}} %
\pgfusepath{clip}%
\pgfsetbuttcap%
\pgfsetroundjoin%
\definecolor{currentfill}{rgb}{0.000000,0.000000,0.545098}%
\pgfsetfillcolor{currentfill}%
\pgfsetfillopacity{0.400000}%
\pgfsetlinewidth{0.501875pt}%
\definecolor{currentstroke}{rgb}{0.000000,0.000000,0.545098}%
\pgfsetstrokecolor{currentstroke}%
\pgfsetstrokeopacity{0.400000}%
\pgfsetdash{}{0pt}%
\pgfpathmoveto{\pgfqpoint{0.861936in}{2.255324in}}%
\pgfpathcurveto{\pgfqpoint{0.872986in}{2.255324in}}{\pgfqpoint{0.883585in}{2.259714in}}{\pgfqpoint{0.891399in}{2.267527in}}%
\pgfpathcurveto{\pgfqpoint{0.899212in}{2.275341in}}{\pgfqpoint{0.903603in}{2.285940in}}{\pgfqpoint{0.903603in}{2.296990in}}%
\pgfpathcurveto{\pgfqpoint{0.903603in}{2.308040in}}{\pgfqpoint{0.899212in}{2.318639in}}{\pgfqpoint{0.891399in}{2.326453in}}%
\pgfpathcurveto{\pgfqpoint{0.883585in}{2.334267in}}{\pgfqpoint{0.872986in}{2.338657in}}{\pgfqpoint{0.861936in}{2.338657in}}%
\pgfpathcurveto{\pgfqpoint{0.850886in}{2.338657in}}{\pgfqpoint{0.840287in}{2.334267in}}{\pgfqpoint{0.832473in}{2.326453in}}%
\pgfpathcurveto{\pgfqpoint{0.824660in}{2.318639in}}{\pgfqpoint{0.820269in}{2.308040in}}{\pgfqpoint{0.820269in}{2.296990in}}%
\pgfpathcurveto{\pgfqpoint{0.820269in}{2.285940in}}{\pgfqpoint{0.824660in}{2.275341in}}{\pgfqpoint{0.832473in}{2.267527in}}%
\pgfpathcurveto{\pgfqpoint{0.840287in}{2.259714in}}{\pgfqpoint{0.850886in}{2.255324in}}{\pgfqpoint{0.861936in}{2.255324in}}%
\pgfpathclose%
\pgfusepath{stroke,fill}%
\end{pgfscope}%
\begin{pgfscope}%
\pgfpathrectangle{\pgfqpoint{0.511823in}{0.504323in}}{\pgfqpoint{3.218177in}{3.225677in}} %
\pgfusepath{clip}%
\pgfsetbuttcap%
\pgfsetroundjoin%
\definecolor{currentfill}{rgb}{0.000000,0.000000,0.545098}%
\pgfsetfillcolor{currentfill}%
\pgfsetfillopacity{0.400000}%
\pgfsetlinewidth{0.501875pt}%
\definecolor{currentstroke}{rgb}{0.000000,0.000000,0.545098}%
\pgfsetstrokecolor{currentstroke}%
\pgfsetstrokeopacity{0.400000}%
\pgfsetdash{}{0pt}%
\pgfpathmoveto{\pgfqpoint{0.881988in}{2.498376in}}%
\pgfpathcurveto{\pgfqpoint{0.893038in}{2.498376in}}{\pgfqpoint{0.903637in}{2.502766in}}{\pgfqpoint{0.911451in}{2.510580in}}%
\pgfpathcurveto{\pgfqpoint{0.919265in}{2.518394in}}{\pgfqpoint{0.923655in}{2.528993in}}{\pgfqpoint{0.923655in}{2.540043in}}%
\pgfpathcurveto{\pgfqpoint{0.923655in}{2.551093in}}{\pgfqpoint{0.919265in}{2.561692in}}{\pgfqpoint{0.911451in}{2.569506in}}%
\pgfpathcurveto{\pgfqpoint{0.903637in}{2.577319in}}{\pgfqpoint{0.893038in}{2.581709in}}{\pgfqpoint{0.881988in}{2.581709in}}%
\pgfpathcurveto{\pgfqpoint{0.870938in}{2.581709in}}{\pgfqpoint{0.860339in}{2.577319in}}{\pgfqpoint{0.852525in}{2.569506in}}%
\pgfpathcurveto{\pgfqpoint{0.844712in}{2.561692in}}{\pgfqpoint{0.840322in}{2.551093in}}{\pgfqpoint{0.840322in}{2.540043in}}%
\pgfpathcurveto{\pgfqpoint{0.840322in}{2.528993in}}{\pgfqpoint{0.844712in}{2.518394in}}{\pgfqpoint{0.852525in}{2.510580in}}%
\pgfpathcurveto{\pgfqpoint{0.860339in}{2.502766in}}{\pgfqpoint{0.870938in}{2.498376in}}{\pgfqpoint{0.881988in}{2.498376in}}%
\pgfpathclose%
\pgfusepath{stroke,fill}%
\end{pgfscope}%
\begin{pgfscope}%
\pgfpathrectangle{\pgfqpoint{0.511823in}{0.504323in}}{\pgfqpoint{3.218177in}{3.225677in}} %
\pgfusepath{clip}%
\pgfsetbuttcap%
\pgfsetroundjoin%
\definecolor{currentfill}{rgb}{0.000000,0.000000,0.545098}%
\pgfsetfillcolor{currentfill}%
\pgfsetfillopacity{0.400000}%
\pgfsetlinewidth{0.501875pt}%
\definecolor{currentstroke}{rgb}{0.000000,0.000000,0.545098}%
\pgfsetstrokecolor{currentstroke}%
\pgfsetstrokeopacity{0.400000}%
\pgfsetdash{}{0pt}%
\pgfpathmoveto{\pgfqpoint{0.873309in}{2.475463in}}%
\pgfpathcurveto{\pgfqpoint{0.884359in}{2.475463in}}{\pgfqpoint{0.894959in}{2.479853in}}{\pgfqpoint{0.902772in}{2.487666in}}%
\pgfpathcurveto{\pgfqpoint{0.910586in}{2.495480in}}{\pgfqpoint{0.914976in}{2.506079in}}{\pgfqpoint{0.914976in}{2.517129in}}%
\pgfpathcurveto{\pgfqpoint{0.914976in}{2.528179in}}{\pgfqpoint{0.910586in}{2.538778in}}{\pgfqpoint{0.902772in}{2.546592in}}%
\pgfpathcurveto{\pgfqpoint{0.894959in}{2.554406in}}{\pgfqpoint{0.884359in}{2.558796in}}{\pgfqpoint{0.873309in}{2.558796in}}%
\pgfpathcurveto{\pgfqpoint{0.862259in}{2.558796in}}{\pgfqpoint{0.851660in}{2.554406in}}{\pgfqpoint{0.843847in}{2.546592in}}%
\pgfpathcurveto{\pgfqpoint{0.836033in}{2.538778in}}{\pgfqpoint{0.831643in}{2.528179in}}{\pgfqpoint{0.831643in}{2.517129in}}%
\pgfpathcurveto{\pgfqpoint{0.831643in}{2.506079in}}{\pgfqpoint{0.836033in}{2.495480in}}{\pgfqpoint{0.843847in}{2.487666in}}%
\pgfpathcurveto{\pgfqpoint{0.851660in}{2.479853in}}{\pgfqpoint{0.862259in}{2.475463in}}{\pgfqpoint{0.873309in}{2.475463in}}%
\pgfpathclose%
\pgfusepath{stroke,fill}%
\end{pgfscope}%
\begin{pgfscope}%
\pgfpathrectangle{\pgfqpoint{0.511823in}{0.504323in}}{\pgfqpoint{3.218177in}{3.225677in}} %
\pgfusepath{clip}%
\pgfsetbuttcap%
\pgfsetroundjoin%
\definecolor{currentfill}{rgb}{0.000000,0.000000,0.545098}%
\pgfsetfillcolor{currentfill}%
\pgfsetfillopacity{0.400000}%
\pgfsetlinewidth{0.501875pt}%
\definecolor{currentstroke}{rgb}{0.000000,0.000000,0.545098}%
\pgfsetstrokecolor{currentstroke}%
\pgfsetstrokeopacity{0.400000}%
\pgfsetdash{}{0pt}%
\pgfpathmoveto{\pgfqpoint{0.879163in}{2.598630in}}%
\pgfpathcurveto{\pgfqpoint{0.890213in}{2.598630in}}{\pgfqpoint{0.900812in}{2.603020in}}{\pgfqpoint{0.908626in}{2.610834in}}%
\pgfpathcurveto{\pgfqpoint{0.916439in}{2.618647in}}{\pgfqpoint{0.920830in}{2.629246in}}{\pgfqpoint{0.920830in}{2.640297in}}%
\pgfpathcurveto{\pgfqpoint{0.920830in}{2.651347in}}{\pgfqpoint{0.916439in}{2.661946in}}{\pgfqpoint{0.908626in}{2.669759in}}%
\pgfpathcurveto{\pgfqpoint{0.900812in}{2.677573in}}{\pgfqpoint{0.890213in}{2.681963in}}{\pgfqpoint{0.879163in}{2.681963in}}%
\pgfpathcurveto{\pgfqpoint{0.868113in}{2.681963in}}{\pgfqpoint{0.857514in}{2.677573in}}{\pgfqpoint{0.849700in}{2.669759in}}%
\pgfpathcurveto{\pgfqpoint{0.841886in}{2.661946in}}{\pgfqpoint{0.837496in}{2.651347in}}{\pgfqpoint{0.837496in}{2.640297in}}%
\pgfpathcurveto{\pgfqpoint{0.837496in}{2.629246in}}{\pgfqpoint{0.841886in}{2.618647in}}{\pgfqpoint{0.849700in}{2.610834in}}%
\pgfpathcurveto{\pgfqpoint{0.857514in}{2.603020in}}{\pgfqpoint{0.868113in}{2.598630in}}{\pgfqpoint{0.879163in}{2.598630in}}%
\pgfpathclose%
\pgfusepath{stroke,fill}%
\end{pgfscope}%
\begin{pgfscope}%
\pgfpathrectangle{\pgfqpoint{0.511823in}{0.504323in}}{\pgfqpoint{3.218177in}{3.225677in}} %
\pgfusepath{clip}%
\pgfsetbuttcap%
\pgfsetroundjoin%
\definecolor{currentfill}{rgb}{0.000000,0.000000,0.545098}%
\pgfsetfillcolor{currentfill}%
\pgfsetfillopacity{0.400000}%
\pgfsetlinewidth{0.501875pt}%
\definecolor{currentstroke}{rgb}{0.000000,0.000000,0.545098}%
\pgfsetstrokecolor{currentstroke}%
\pgfsetstrokeopacity{0.400000}%
\pgfsetdash{}{0pt}%
\pgfpathmoveto{\pgfqpoint{0.847829in}{2.339053in}}%
\pgfpathcurveto{\pgfqpoint{0.858879in}{2.339053in}}{\pgfqpoint{0.869478in}{2.343444in}}{\pgfqpoint{0.877292in}{2.351257in}}%
\pgfpathcurveto{\pgfqpoint{0.885105in}{2.359071in}}{\pgfqpoint{0.889496in}{2.369670in}}{\pgfqpoint{0.889496in}{2.380720in}}%
\pgfpathcurveto{\pgfqpoint{0.889496in}{2.391770in}}{\pgfqpoint{0.885105in}{2.402369in}}{\pgfqpoint{0.877292in}{2.410183in}}%
\pgfpathcurveto{\pgfqpoint{0.869478in}{2.417996in}}{\pgfqpoint{0.858879in}{2.422387in}}{\pgfqpoint{0.847829in}{2.422387in}}%
\pgfpathcurveto{\pgfqpoint{0.836779in}{2.422387in}}{\pgfqpoint{0.826180in}{2.417996in}}{\pgfqpoint{0.818366in}{2.410183in}}%
\pgfpathcurveto{\pgfqpoint{0.810552in}{2.402369in}}{\pgfqpoint{0.806162in}{2.391770in}}{\pgfqpoint{0.806162in}{2.380720in}}%
\pgfpathcurveto{\pgfqpoint{0.806162in}{2.369670in}}{\pgfqpoint{0.810552in}{2.359071in}}{\pgfqpoint{0.818366in}{2.351257in}}%
\pgfpathcurveto{\pgfqpoint{0.826180in}{2.343444in}}{\pgfqpoint{0.836779in}{2.339053in}}{\pgfqpoint{0.847829in}{2.339053in}}%
\pgfpathclose%
\pgfusepath{stroke,fill}%
\end{pgfscope}%
\begin{pgfscope}%
\pgfpathrectangle{\pgfqpoint{0.511823in}{0.504323in}}{\pgfqpoint{3.218177in}{3.225677in}} %
\pgfusepath{clip}%
\pgfsetbuttcap%
\pgfsetroundjoin%
\definecolor{currentfill}{rgb}{0.000000,0.000000,0.545098}%
\pgfsetfillcolor{currentfill}%
\pgfsetfillopacity{0.400000}%
\pgfsetlinewidth{0.501875pt}%
\definecolor{currentstroke}{rgb}{0.000000,0.000000,0.545098}%
\pgfsetstrokecolor{currentstroke}%
\pgfsetstrokeopacity{0.400000}%
\pgfsetdash{}{0pt}%
\pgfpathmoveto{\pgfqpoint{0.870729in}{2.652402in}}%
\pgfpathcurveto{\pgfqpoint{0.881779in}{2.652402in}}{\pgfqpoint{0.892378in}{2.656792in}}{\pgfqpoint{0.900191in}{2.664605in}}%
\pgfpathcurveto{\pgfqpoint{0.908005in}{2.672419in}}{\pgfqpoint{0.912395in}{2.683018in}}{\pgfqpoint{0.912395in}{2.694068in}}%
\pgfpathcurveto{\pgfqpoint{0.912395in}{2.705118in}}{\pgfqpoint{0.908005in}{2.715717in}}{\pgfqpoint{0.900191in}{2.723531in}}%
\pgfpathcurveto{\pgfqpoint{0.892378in}{2.731345in}}{\pgfqpoint{0.881779in}{2.735735in}}{\pgfqpoint{0.870729in}{2.735735in}}%
\pgfpathcurveto{\pgfqpoint{0.859678in}{2.735735in}}{\pgfqpoint{0.849079in}{2.731345in}}{\pgfqpoint{0.841266in}{2.723531in}}%
\pgfpathcurveto{\pgfqpoint{0.833452in}{2.715717in}}{\pgfqpoint{0.829062in}{2.705118in}}{\pgfqpoint{0.829062in}{2.694068in}}%
\pgfpathcurveto{\pgfqpoint{0.829062in}{2.683018in}}{\pgfqpoint{0.833452in}{2.672419in}}{\pgfqpoint{0.841266in}{2.664605in}}%
\pgfpathcurveto{\pgfqpoint{0.849079in}{2.656792in}}{\pgfqpoint{0.859678in}{2.652402in}}{\pgfqpoint{0.870729in}{2.652402in}}%
\pgfpathclose%
\pgfusepath{stroke,fill}%
\end{pgfscope}%
\begin{pgfscope}%
\pgfpathrectangle{\pgfqpoint{0.511823in}{0.504323in}}{\pgfqpoint{3.218177in}{3.225677in}} %
\pgfusepath{clip}%
\pgfsetbuttcap%
\pgfsetroundjoin%
\definecolor{currentfill}{rgb}{0.000000,0.000000,0.545098}%
\pgfsetfillcolor{currentfill}%
\pgfsetfillopacity{0.400000}%
\pgfsetlinewidth{0.501875pt}%
\definecolor{currentstroke}{rgb}{0.000000,0.000000,0.545098}%
\pgfsetstrokecolor{currentstroke}%
\pgfsetstrokeopacity{0.400000}%
\pgfsetdash{}{0pt}%
\pgfpathmoveto{\pgfqpoint{0.837383in}{2.351932in}}%
\pgfpathcurveto{\pgfqpoint{0.848433in}{2.351932in}}{\pgfqpoint{0.859032in}{2.356322in}}{\pgfqpoint{0.866846in}{2.364136in}}%
\pgfpathcurveto{\pgfqpoint{0.874659in}{2.371949in}}{\pgfqpoint{0.879049in}{2.382548in}}{\pgfqpoint{0.879049in}{2.393599in}}%
\pgfpathcurveto{\pgfqpoint{0.879049in}{2.404649in}}{\pgfqpoint{0.874659in}{2.415248in}}{\pgfqpoint{0.866846in}{2.423061in}}%
\pgfpathcurveto{\pgfqpoint{0.859032in}{2.430875in}}{\pgfqpoint{0.848433in}{2.435265in}}{\pgfqpoint{0.837383in}{2.435265in}}%
\pgfpathcurveto{\pgfqpoint{0.826333in}{2.435265in}}{\pgfqpoint{0.815734in}{2.430875in}}{\pgfqpoint{0.807920in}{2.423061in}}%
\pgfpathcurveto{\pgfqpoint{0.800106in}{2.415248in}}{\pgfqpoint{0.795716in}{2.404649in}}{\pgfqpoint{0.795716in}{2.393599in}}%
\pgfpathcurveto{\pgfqpoint{0.795716in}{2.382548in}}{\pgfqpoint{0.800106in}{2.371949in}}{\pgfqpoint{0.807920in}{2.364136in}}%
\pgfpathcurveto{\pgfqpoint{0.815734in}{2.356322in}}{\pgfqpoint{0.826333in}{2.351932in}}{\pgfqpoint{0.837383in}{2.351932in}}%
\pgfpathclose%
\pgfusepath{stroke,fill}%
\end{pgfscope}%
\begin{pgfscope}%
\pgfpathrectangle{\pgfqpoint{0.511823in}{0.504323in}}{\pgfqpoint{3.218177in}{3.225677in}} %
\pgfusepath{clip}%
\pgfsetbuttcap%
\pgfsetroundjoin%
\definecolor{currentfill}{rgb}{0.000000,0.000000,0.545098}%
\pgfsetfillcolor{currentfill}%
\pgfsetfillopacity{0.400000}%
\pgfsetlinewidth{0.501875pt}%
\definecolor{currentstroke}{rgb}{0.000000,0.000000,0.545098}%
\pgfsetstrokecolor{currentstroke}%
\pgfsetstrokeopacity{0.400000}%
\pgfsetdash{}{0pt}%
\pgfpathmoveto{\pgfqpoint{0.856613in}{2.647615in}}%
\pgfpathcurveto{\pgfqpoint{0.867663in}{2.647615in}}{\pgfqpoint{0.878262in}{2.652005in}}{\pgfqpoint{0.886076in}{2.659819in}}%
\pgfpathcurveto{\pgfqpoint{0.893889in}{2.667632in}}{\pgfqpoint{0.898280in}{2.678231in}}{\pgfqpoint{0.898280in}{2.689282in}}%
\pgfpathcurveto{\pgfqpoint{0.898280in}{2.700332in}}{\pgfqpoint{0.893889in}{2.710931in}}{\pgfqpoint{0.886076in}{2.718744in}}%
\pgfpathcurveto{\pgfqpoint{0.878262in}{2.726558in}}{\pgfqpoint{0.867663in}{2.730948in}}{\pgfqpoint{0.856613in}{2.730948in}}%
\pgfpathcurveto{\pgfqpoint{0.845563in}{2.730948in}}{\pgfqpoint{0.834964in}{2.726558in}}{\pgfqpoint{0.827150in}{2.718744in}}%
\pgfpathcurveto{\pgfqpoint{0.819337in}{2.710931in}}{\pgfqpoint{0.814946in}{2.700332in}}{\pgfqpoint{0.814946in}{2.689282in}}%
\pgfpathcurveto{\pgfqpoint{0.814946in}{2.678231in}}{\pgfqpoint{0.819337in}{2.667632in}}{\pgfqpoint{0.827150in}{2.659819in}}%
\pgfpathcurveto{\pgfqpoint{0.834964in}{2.652005in}}{\pgfqpoint{0.845563in}{2.647615in}}{\pgfqpoint{0.856613in}{2.647615in}}%
\pgfpathclose%
\pgfusepath{stroke,fill}%
\end{pgfscope}%
\begin{pgfscope}%
\pgfpathrectangle{\pgfqpoint{0.511823in}{0.504323in}}{\pgfqpoint{3.218177in}{3.225677in}} %
\pgfusepath{clip}%
\pgfsetbuttcap%
\pgfsetroundjoin%
\definecolor{currentfill}{rgb}{0.000000,0.000000,0.545098}%
\pgfsetfillcolor{currentfill}%
\pgfsetfillopacity{0.400000}%
\pgfsetlinewidth{0.501875pt}%
\definecolor{currentstroke}{rgb}{0.000000,0.000000,0.545098}%
\pgfsetstrokecolor{currentstroke}%
\pgfsetstrokeopacity{0.400000}%
\pgfsetdash{}{0pt}%
\pgfpathmoveto{\pgfqpoint{0.842425in}{2.557205in}}%
\pgfpathcurveto{\pgfqpoint{0.853475in}{2.557205in}}{\pgfqpoint{0.864074in}{2.561595in}}{\pgfqpoint{0.871887in}{2.569409in}}%
\pgfpathcurveto{\pgfqpoint{0.879701in}{2.577223in}}{\pgfqpoint{0.884091in}{2.587822in}}{\pgfqpoint{0.884091in}{2.598872in}}%
\pgfpathcurveto{\pgfqpoint{0.884091in}{2.609922in}}{\pgfqpoint{0.879701in}{2.620521in}}{\pgfqpoint{0.871887in}{2.628335in}}%
\pgfpathcurveto{\pgfqpoint{0.864074in}{2.636148in}}{\pgfqpoint{0.853475in}{2.640539in}}{\pgfqpoint{0.842425in}{2.640539in}}%
\pgfpathcurveto{\pgfqpoint{0.831375in}{2.640539in}}{\pgfqpoint{0.820776in}{2.636148in}}{\pgfqpoint{0.812962in}{2.628335in}}%
\pgfpathcurveto{\pgfqpoint{0.805148in}{2.620521in}}{\pgfqpoint{0.800758in}{2.609922in}}{\pgfqpoint{0.800758in}{2.598872in}}%
\pgfpathcurveto{\pgfqpoint{0.800758in}{2.587822in}}{\pgfqpoint{0.805148in}{2.577223in}}{\pgfqpoint{0.812962in}{2.569409in}}%
\pgfpathcurveto{\pgfqpoint{0.820776in}{2.561595in}}{\pgfqpoint{0.831375in}{2.557205in}}{\pgfqpoint{0.842425in}{2.557205in}}%
\pgfpathclose%
\pgfusepath{stroke,fill}%
\end{pgfscope}%
\begin{pgfscope}%
\pgfpathrectangle{\pgfqpoint{0.511823in}{0.504323in}}{\pgfqpoint{3.218177in}{3.225677in}} %
\pgfusepath{clip}%
\pgfsetbuttcap%
\pgfsetroundjoin%
\definecolor{currentfill}{rgb}{0.000000,0.000000,0.545098}%
\pgfsetfillcolor{currentfill}%
\pgfsetfillopacity{0.400000}%
\pgfsetlinewidth{0.501875pt}%
\definecolor{currentstroke}{rgb}{0.000000,0.000000,0.545098}%
\pgfsetstrokecolor{currentstroke}%
\pgfsetstrokeopacity{0.400000}%
\pgfsetdash{}{0pt}%
\pgfpathmoveto{\pgfqpoint{0.823148in}{2.393584in}}%
\pgfpathcurveto{\pgfqpoint{0.834198in}{2.393584in}}{\pgfqpoint{0.844797in}{2.397974in}}{\pgfqpoint{0.852611in}{2.405788in}}%
\pgfpathcurveto{\pgfqpoint{0.860424in}{2.413602in}}{\pgfqpoint{0.864815in}{2.424201in}}{\pgfqpoint{0.864815in}{2.435251in}}%
\pgfpathcurveto{\pgfqpoint{0.864815in}{2.446301in}}{\pgfqpoint{0.860424in}{2.456900in}}{\pgfqpoint{0.852611in}{2.464714in}}%
\pgfpathcurveto{\pgfqpoint{0.844797in}{2.472527in}}{\pgfqpoint{0.834198in}{2.476917in}}{\pgfqpoint{0.823148in}{2.476917in}}%
\pgfpathcurveto{\pgfqpoint{0.812098in}{2.476917in}}{\pgfqpoint{0.801499in}{2.472527in}}{\pgfqpoint{0.793685in}{2.464714in}}%
\pgfpathcurveto{\pgfqpoint{0.785872in}{2.456900in}}{\pgfqpoint{0.781481in}{2.446301in}}{\pgfqpoint{0.781481in}{2.435251in}}%
\pgfpathcurveto{\pgfqpoint{0.781481in}{2.424201in}}{\pgfqpoint{0.785872in}{2.413602in}}{\pgfqpoint{0.793685in}{2.405788in}}%
\pgfpathcurveto{\pgfqpoint{0.801499in}{2.397974in}}{\pgfqpoint{0.812098in}{2.393584in}}{\pgfqpoint{0.823148in}{2.393584in}}%
\pgfpathclose%
\pgfusepath{stroke,fill}%
\end{pgfscope}%
\begin{pgfscope}%
\pgfpathrectangle{\pgfqpoint{0.511823in}{0.504323in}}{\pgfqpoint{3.218177in}{3.225677in}} %
\pgfusepath{clip}%
\pgfsetbuttcap%
\pgfsetroundjoin%
\definecolor{currentfill}{rgb}{0.000000,0.000000,0.545098}%
\pgfsetfillcolor{currentfill}%
\pgfsetfillopacity{0.400000}%
\pgfsetlinewidth{0.501875pt}%
\definecolor{currentstroke}{rgb}{0.000000,0.000000,0.545098}%
\pgfsetstrokecolor{currentstroke}%
\pgfsetstrokeopacity{0.400000}%
\pgfsetdash{}{0pt}%
\pgfpathmoveto{\pgfqpoint{0.844058in}{2.754068in}}%
\pgfpathcurveto{\pgfqpoint{0.855108in}{2.754068in}}{\pgfqpoint{0.865707in}{2.758459in}}{\pgfqpoint{0.873521in}{2.766272in}}%
\pgfpathcurveto{\pgfqpoint{0.881335in}{2.774086in}}{\pgfqpoint{0.885725in}{2.784685in}}{\pgfqpoint{0.885725in}{2.795735in}}%
\pgfpathcurveto{\pgfqpoint{0.885725in}{2.806785in}}{\pgfqpoint{0.881335in}{2.817384in}}{\pgfqpoint{0.873521in}{2.825198in}}%
\pgfpathcurveto{\pgfqpoint{0.865707in}{2.833012in}}{\pgfqpoint{0.855108in}{2.837402in}}{\pgfqpoint{0.844058in}{2.837402in}}%
\pgfpathcurveto{\pgfqpoint{0.833008in}{2.837402in}}{\pgfqpoint{0.822409in}{2.833012in}}{\pgfqpoint{0.814595in}{2.825198in}}%
\pgfpathcurveto{\pgfqpoint{0.806782in}{2.817384in}}{\pgfqpoint{0.802392in}{2.806785in}}{\pgfqpoint{0.802392in}{2.795735in}}%
\pgfpathcurveto{\pgfqpoint{0.802392in}{2.784685in}}{\pgfqpoint{0.806782in}{2.774086in}}{\pgfqpoint{0.814595in}{2.766272in}}%
\pgfpathcurveto{\pgfqpoint{0.822409in}{2.758459in}}{\pgfqpoint{0.833008in}{2.754068in}}{\pgfqpoint{0.844058in}{2.754068in}}%
\pgfpathclose%
\pgfusepath{stroke,fill}%
\end{pgfscope}%
\begin{pgfscope}%
\pgfpathrectangle{\pgfqpoint{0.511823in}{0.504323in}}{\pgfqpoint{3.218177in}{3.225677in}} %
\pgfusepath{clip}%
\pgfsetbuttcap%
\pgfsetroundjoin%
\definecolor{currentfill}{rgb}{0.000000,0.000000,0.545098}%
\pgfsetfillcolor{currentfill}%
\pgfsetfillopacity{0.400000}%
\pgfsetlinewidth{0.501875pt}%
\definecolor{currentstroke}{rgb}{0.000000,0.000000,0.545098}%
\pgfsetstrokecolor{currentstroke}%
\pgfsetstrokeopacity{0.400000}%
\pgfsetdash{}{0pt}%
\pgfpathmoveto{\pgfqpoint{0.807412in}{2.340013in}}%
\pgfpathcurveto{\pgfqpoint{0.818462in}{2.340013in}}{\pgfqpoint{0.829061in}{2.344403in}}{\pgfqpoint{0.836875in}{2.352217in}}%
\pgfpathcurveto{\pgfqpoint{0.844689in}{2.360031in}}{\pgfqpoint{0.849079in}{2.370630in}}{\pgfqpoint{0.849079in}{2.381680in}}%
\pgfpathcurveto{\pgfqpoint{0.849079in}{2.392730in}}{\pgfqpoint{0.844689in}{2.403329in}}{\pgfqpoint{0.836875in}{2.411143in}}%
\pgfpathcurveto{\pgfqpoint{0.829061in}{2.418956in}}{\pgfqpoint{0.818462in}{2.423346in}}{\pgfqpoint{0.807412in}{2.423346in}}%
\pgfpathcurveto{\pgfqpoint{0.796362in}{2.423346in}}{\pgfqpoint{0.785763in}{2.418956in}}{\pgfqpoint{0.777949in}{2.411143in}}%
\pgfpathcurveto{\pgfqpoint{0.770136in}{2.403329in}}{\pgfqpoint{0.765746in}{2.392730in}}{\pgfqpoint{0.765746in}{2.381680in}}%
\pgfpathcurveto{\pgfqpoint{0.765746in}{2.370630in}}{\pgfqpoint{0.770136in}{2.360031in}}{\pgfqpoint{0.777949in}{2.352217in}}%
\pgfpathcurveto{\pgfqpoint{0.785763in}{2.344403in}}{\pgfqpoint{0.796362in}{2.340013in}}{\pgfqpoint{0.807412in}{2.340013in}}%
\pgfpathclose%
\pgfusepath{stroke,fill}%
\end{pgfscope}%
\begin{pgfscope}%
\pgfpathrectangle{\pgfqpoint{0.511823in}{0.504323in}}{\pgfqpoint{3.218177in}{3.225677in}} %
\pgfusepath{clip}%
\pgfsetbuttcap%
\pgfsetroundjoin%
\definecolor{currentfill}{rgb}{0.000000,0.000000,0.545098}%
\pgfsetfillcolor{currentfill}%
\pgfsetfillopacity{0.400000}%
\pgfsetlinewidth{0.501875pt}%
\definecolor{currentstroke}{rgb}{0.000000,0.000000,0.545098}%
\pgfsetstrokecolor{currentstroke}%
\pgfsetstrokeopacity{0.400000}%
\pgfsetdash{}{0pt}%
\pgfpathmoveto{\pgfqpoint{0.810203in}{2.466465in}}%
\pgfpathcurveto{\pgfqpoint{0.821254in}{2.466465in}}{\pgfqpoint{0.831853in}{2.470855in}}{\pgfqpoint{0.839666in}{2.478669in}}%
\pgfpathcurveto{\pgfqpoint{0.847480in}{2.486483in}}{\pgfqpoint{0.851870in}{2.497082in}}{\pgfqpoint{0.851870in}{2.508132in}}%
\pgfpathcurveto{\pgfqpoint{0.851870in}{2.519182in}}{\pgfqpoint{0.847480in}{2.529781in}}{\pgfqpoint{0.839666in}{2.537595in}}%
\pgfpathcurveto{\pgfqpoint{0.831853in}{2.545408in}}{\pgfqpoint{0.821254in}{2.549799in}}{\pgfqpoint{0.810203in}{2.549799in}}%
\pgfpathcurveto{\pgfqpoint{0.799153in}{2.549799in}}{\pgfqpoint{0.788554in}{2.545408in}}{\pgfqpoint{0.780741in}{2.537595in}}%
\pgfpathcurveto{\pgfqpoint{0.772927in}{2.529781in}}{\pgfqpoint{0.768537in}{2.519182in}}{\pgfqpoint{0.768537in}{2.508132in}}%
\pgfpathcurveto{\pgfqpoint{0.768537in}{2.497082in}}{\pgfqpoint{0.772927in}{2.486483in}}{\pgfqpoint{0.780741in}{2.478669in}}%
\pgfpathcurveto{\pgfqpoint{0.788554in}{2.470855in}}{\pgfqpoint{0.799153in}{2.466465in}}{\pgfqpoint{0.810203in}{2.466465in}}%
\pgfpathclose%
\pgfusepath{stroke,fill}%
\end{pgfscope}%
\begin{pgfscope}%
\pgfpathrectangle{\pgfqpoint{0.511823in}{0.504323in}}{\pgfqpoint{3.218177in}{3.225677in}} %
\pgfusepath{clip}%
\pgfsetbuttcap%
\pgfsetroundjoin%
\definecolor{currentfill}{rgb}{0.000000,0.000000,0.545098}%
\pgfsetfillcolor{currentfill}%
\pgfsetfillopacity{0.400000}%
\pgfsetlinewidth{0.501875pt}%
\definecolor{currentstroke}{rgb}{0.000000,0.000000,0.545098}%
\pgfsetstrokecolor{currentstroke}%
\pgfsetstrokeopacity{0.400000}%
\pgfsetdash{}{0pt}%
\pgfpathmoveto{\pgfqpoint{0.807029in}{2.513523in}}%
\pgfpathcurveto{\pgfqpoint{0.818079in}{2.513523in}}{\pgfqpoint{0.828678in}{2.517913in}}{\pgfqpoint{0.836491in}{2.525727in}}%
\pgfpathcurveto{\pgfqpoint{0.844305in}{2.533540in}}{\pgfqpoint{0.848695in}{2.544139in}}{\pgfqpoint{0.848695in}{2.555189in}}%
\pgfpathcurveto{\pgfqpoint{0.848695in}{2.566240in}}{\pgfqpoint{0.844305in}{2.576839in}}{\pgfqpoint{0.836491in}{2.584652in}}%
\pgfpathcurveto{\pgfqpoint{0.828678in}{2.592466in}}{\pgfqpoint{0.818079in}{2.596856in}}{\pgfqpoint{0.807029in}{2.596856in}}%
\pgfpathcurveto{\pgfqpoint{0.795978in}{2.596856in}}{\pgfqpoint{0.785379in}{2.592466in}}{\pgfqpoint{0.777566in}{2.584652in}}%
\pgfpathcurveto{\pgfqpoint{0.769752in}{2.576839in}}{\pgfqpoint{0.765362in}{2.566240in}}{\pgfqpoint{0.765362in}{2.555189in}}%
\pgfpathcurveto{\pgfqpoint{0.765362in}{2.544139in}}{\pgfqpoint{0.769752in}{2.533540in}}{\pgfqpoint{0.777566in}{2.525727in}}%
\pgfpathcurveto{\pgfqpoint{0.785379in}{2.517913in}}{\pgfqpoint{0.795978in}{2.513523in}}{\pgfqpoint{0.807029in}{2.513523in}}%
\pgfpathclose%
\pgfusepath{stroke,fill}%
\end{pgfscope}%
\begin{pgfscope}%
\pgfpathrectangle{\pgfqpoint{0.511823in}{0.504323in}}{\pgfqpoint{3.218177in}{3.225677in}} %
\pgfusepath{clip}%
\pgfsetbuttcap%
\pgfsetroundjoin%
\definecolor{currentfill}{rgb}{0.000000,0.000000,0.545098}%
\pgfsetfillcolor{currentfill}%
\pgfsetfillopacity{0.400000}%
\pgfsetlinewidth{0.501875pt}%
\definecolor{currentstroke}{rgb}{0.000000,0.000000,0.545098}%
\pgfsetstrokecolor{currentstroke}%
\pgfsetstrokeopacity{0.400000}%
\pgfsetdash{}{0pt}%
\pgfpathmoveto{\pgfqpoint{0.803053in}{2.552652in}}%
\pgfpathcurveto{\pgfqpoint{0.814103in}{2.552652in}}{\pgfqpoint{0.824702in}{2.557042in}}{\pgfqpoint{0.832516in}{2.564856in}}%
\pgfpathcurveto{\pgfqpoint{0.840330in}{2.572670in}}{\pgfqpoint{0.844720in}{2.583269in}}{\pgfqpoint{0.844720in}{2.594319in}}%
\pgfpathcurveto{\pgfqpoint{0.844720in}{2.605369in}}{\pgfqpoint{0.840330in}{2.615968in}}{\pgfqpoint{0.832516in}{2.623782in}}%
\pgfpathcurveto{\pgfqpoint{0.824702in}{2.631595in}}{\pgfqpoint{0.814103in}{2.635985in}}{\pgfqpoint{0.803053in}{2.635985in}}%
\pgfpathcurveto{\pgfqpoint{0.792003in}{2.635985in}}{\pgfqpoint{0.781404in}{2.631595in}}{\pgfqpoint{0.773590in}{2.623782in}}%
\pgfpathcurveto{\pgfqpoint{0.765777in}{2.615968in}}{\pgfqpoint{0.761387in}{2.605369in}}{\pgfqpoint{0.761387in}{2.594319in}}%
\pgfpathcurveto{\pgfqpoint{0.761387in}{2.583269in}}{\pgfqpoint{0.765777in}{2.572670in}}{\pgfqpoint{0.773590in}{2.564856in}}%
\pgfpathcurveto{\pgfqpoint{0.781404in}{2.557042in}}{\pgfqpoint{0.792003in}{2.552652in}}{\pgfqpoint{0.803053in}{2.552652in}}%
\pgfpathclose%
\pgfusepath{stroke,fill}%
\end{pgfscope}%
\begin{pgfscope}%
\pgfpathrectangle{\pgfqpoint{0.511823in}{0.504323in}}{\pgfqpoint{3.218177in}{3.225677in}} %
\pgfusepath{clip}%
\pgfsetbuttcap%
\pgfsetroundjoin%
\definecolor{currentfill}{rgb}{0.000000,0.000000,0.545098}%
\pgfsetfillcolor{currentfill}%
\pgfsetfillopacity{0.400000}%
\pgfsetlinewidth{0.501875pt}%
\definecolor{currentstroke}{rgb}{0.000000,0.000000,0.545098}%
\pgfsetstrokecolor{currentstroke}%
\pgfsetstrokeopacity{0.400000}%
\pgfsetdash{}{0pt}%
\pgfpathmoveto{\pgfqpoint{0.794485in}{2.516574in}}%
\pgfpathcurveto{\pgfqpoint{0.805535in}{2.516574in}}{\pgfqpoint{0.816134in}{2.520964in}}{\pgfqpoint{0.823948in}{2.528778in}}%
\pgfpathcurveto{\pgfqpoint{0.831761in}{2.536591in}}{\pgfqpoint{0.836152in}{2.547190in}}{\pgfqpoint{0.836152in}{2.558240in}}%
\pgfpathcurveto{\pgfqpoint{0.836152in}{2.569291in}}{\pgfqpoint{0.831761in}{2.579890in}}{\pgfqpoint{0.823948in}{2.587703in}}%
\pgfpathcurveto{\pgfqpoint{0.816134in}{2.595517in}}{\pgfqpoint{0.805535in}{2.599907in}}{\pgfqpoint{0.794485in}{2.599907in}}%
\pgfpathcurveto{\pgfqpoint{0.783435in}{2.599907in}}{\pgfqpoint{0.772836in}{2.595517in}}{\pgfqpoint{0.765022in}{2.587703in}}%
\pgfpathcurveto{\pgfqpoint{0.757209in}{2.579890in}}{\pgfqpoint{0.752818in}{2.569291in}}{\pgfqpoint{0.752818in}{2.558240in}}%
\pgfpathcurveto{\pgfqpoint{0.752818in}{2.547190in}}{\pgfqpoint{0.757209in}{2.536591in}}{\pgfqpoint{0.765022in}{2.528778in}}%
\pgfpathcurveto{\pgfqpoint{0.772836in}{2.520964in}}{\pgfqpoint{0.783435in}{2.516574in}}{\pgfqpoint{0.794485in}{2.516574in}}%
\pgfpathclose%
\pgfusepath{stroke,fill}%
\end{pgfscope}%
\begin{pgfscope}%
\pgfpathrectangle{\pgfqpoint{0.511823in}{0.504323in}}{\pgfqpoint{3.218177in}{3.225677in}} %
\pgfusepath{clip}%
\pgfsetbuttcap%
\pgfsetroundjoin%
\definecolor{currentfill}{rgb}{0.000000,0.000000,0.545098}%
\pgfsetfillcolor{currentfill}%
\pgfsetfillopacity{0.400000}%
\pgfsetlinewidth{0.501875pt}%
\definecolor{currentstroke}{rgb}{0.000000,0.000000,0.545098}%
\pgfsetstrokecolor{currentstroke}%
\pgfsetstrokeopacity{0.400000}%
\pgfsetdash{}{0pt}%
\pgfpathmoveto{\pgfqpoint{0.791960in}{2.587489in}}%
\pgfpathcurveto{\pgfqpoint{0.803010in}{2.587489in}}{\pgfqpoint{0.813609in}{2.591879in}}{\pgfqpoint{0.821423in}{2.599693in}}%
\pgfpathcurveto{\pgfqpoint{0.829237in}{2.607506in}}{\pgfqpoint{0.833627in}{2.618105in}}{\pgfqpoint{0.833627in}{2.629155in}}%
\pgfpathcurveto{\pgfqpoint{0.833627in}{2.640206in}}{\pgfqpoint{0.829237in}{2.650805in}}{\pgfqpoint{0.821423in}{2.658618in}}%
\pgfpathcurveto{\pgfqpoint{0.813609in}{2.666432in}}{\pgfqpoint{0.803010in}{2.670822in}}{\pgfqpoint{0.791960in}{2.670822in}}%
\pgfpathcurveto{\pgfqpoint{0.780910in}{2.670822in}}{\pgfqpoint{0.770311in}{2.666432in}}{\pgfqpoint{0.762497in}{2.658618in}}%
\pgfpathcurveto{\pgfqpoint{0.754684in}{2.650805in}}{\pgfqpoint{0.750293in}{2.640206in}}{\pgfqpoint{0.750293in}{2.629155in}}%
\pgfpathcurveto{\pgfqpoint{0.750293in}{2.618105in}}{\pgfqpoint{0.754684in}{2.607506in}}{\pgfqpoint{0.762497in}{2.599693in}}%
\pgfpathcurveto{\pgfqpoint{0.770311in}{2.591879in}}{\pgfqpoint{0.780910in}{2.587489in}}{\pgfqpoint{0.791960in}{2.587489in}}%
\pgfpathclose%
\pgfusepath{stroke,fill}%
\end{pgfscope}%
\begin{pgfscope}%
\pgfpathrectangle{\pgfqpoint{0.511823in}{0.504323in}}{\pgfqpoint{3.218177in}{3.225677in}} %
\pgfusepath{clip}%
\pgfsetbuttcap%
\pgfsetroundjoin%
\definecolor{currentfill}{rgb}{0.000000,0.000000,0.545098}%
\pgfsetfillcolor{currentfill}%
\pgfsetfillopacity{0.400000}%
\pgfsetlinewidth{0.501875pt}%
\definecolor{currentstroke}{rgb}{0.000000,0.000000,0.545098}%
\pgfsetstrokecolor{currentstroke}%
\pgfsetstrokeopacity{0.400000}%
\pgfsetdash{}{0pt}%
\pgfpathmoveto{\pgfqpoint{0.783242in}{2.545881in}}%
\pgfpathcurveto{\pgfqpoint{0.794292in}{2.545881in}}{\pgfqpoint{0.804891in}{2.550271in}}{\pgfqpoint{0.812705in}{2.558084in}}%
\pgfpathcurveto{\pgfqpoint{0.820518in}{2.565898in}}{\pgfqpoint{0.824908in}{2.576497in}}{\pgfqpoint{0.824908in}{2.587547in}}%
\pgfpathcurveto{\pgfqpoint{0.824908in}{2.598597in}}{\pgfqpoint{0.820518in}{2.609196in}}{\pgfqpoint{0.812705in}{2.617010in}}%
\pgfpathcurveto{\pgfqpoint{0.804891in}{2.624824in}}{\pgfqpoint{0.794292in}{2.629214in}}{\pgfqpoint{0.783242in}{2.629214in}}%
\pgfpathcurveto{\pgfqpoint{0.772192in}{2.629214in}}{\pgfqpoint{0.761593in}{2.624824in}}{\pgfqpoint{0.753779in}{2.617010in}}%
\pgfpathcurveto{\pgfqpoint{0.745965in}{2.609196in}}{\pgfqpoint{0.741575in}{2.598597in}}{\pgfqpoint{0.741575in}{2.587547in}}%
\pgfpathcurveto{\pgfqpoint{0.741575in}{2.576497in}}{\pgfqpoint{0.745965in}{2.565898in}}{\pgfqpoint{0.753779in}{2.558084in}}%
\pgfpathcurveto{\pgfqpoint{0.761593in}{2.550271in}}{\pgfqpoint{0.772192in}{2.545881in}}{\pgfqpoint{0.783242in}{2.545881in}}%
\pgfpathclose%
\pgfusepath{stroke,fill}%
\end{pgfscope}%
\begin{pgfscope}%
\pgfpathrectangle{\pgfqpoint{0.511823in}{0.504323in}}{\pgfqpoint{3.218177in}{3.225677in}} %
\pgfusepath{clip}%
\pgfsetbuttcap%
\pgfsetroundjoin%
\definecolor{currentfill}{rgb}{0.000000,0.000000,0.545098}%
\pgfsetfillcolor{currentfill}%
\pgfsetfillopacity{0.400000}%
\pgfsetlinewidth{0.501875pt}%
\definecolor{currentstroke}{rgb}{0.000000,0.000000,0.545098}%
\pgfsetstrokecolor{currentstroke}%
\pgfsetstrokeopacity{0.400000}%
\pgfsetdash{}{0pt}%
\pgfpathmoveto{\pgfqpoint{0.774300in}{2.493554in}}%
\pgfpathcurveto{\pgfqpoint{0.785350in}{2.493554in}}{\pgfqpoint{0.795949in}{2.497944in}}{\pgfqpoint{0.803762in}{2.505758in}}%
\pgfpathcurveto{\pgfqpoint{0.811576in}{2.513572in}}{\pgfqpoint{0.815966in}{2.524171in}}{\pgfqpoint{0.815966in}{2.535221in}}%
\pgfpathcurveto{\pgfqpoint{0.815966in}{2.546271in}}{\pgfqpoint{0.811576in}{2.556870in}}{\pgfqpoint{0.803762in}{2.564684in}}%
\pgfpathcurveto{\pgfqpoint{0.795949in}{2.572497in}}{\pgfqpoint{0.785350in}{2.576887in}}{\pgfqpoint{0.774300in}{2.576887in}}%
\pgfpathcurveto{\pgfqpoint{0.763250in}{2.576887in}}{\pgfqpoint{0.752650in}{2.572497in}}{\pgfqpoint{0.744837in}{2.564684in}}%
\pgfpathcurveto{\pgfqpoint{0.737023in}{2.556870in}}{\pgfqpoint{0.732633in}{2.546271in}}{\pgfqpoint{0.732633in}{2.535221in}}%
\pgfpathcurveto{\pgfqpoint{0.732633in}{2.524171in}}{\pgfqpoint{0.737023in}{2.513572in}}{\pgfqpoint{0.744837in}{2.505758in}}%
\pgfpathcurveto{\pgfqpoint{0.752650in}{2.497944in}}{\pgfqpoint{0.763250in}{2.493554in}}{\pgfqpoint{0.774300in}{2.493554in}}%
\pgfpathclose%
\pgfusepath{stroke,fill}%
\end{pgfscope}%
\begin{pgfscope}%
\pgfpathrectangle{\pgfqpoint{0.511823in}{0.504323in}}{\pgfqpoint{3.218177in}{3.225677in}} %
\pgfusepath{clip}%
\pgfsetbuttcap%
\pgfsetroundjoin%
\definecolor{currentfill}{rgb}{0.000000,0.000000,0.545098}%
\pgfsetfillcolor{currentfill}%
\pgfsetfillopacity{0.400000}%
\pgfsetlinewidth{0.501875pt}%
\definecolor{currentstroke}{rgb}{0.000000,0.000000,0.545098}%
\pgfsetstrokecolor{currentstroke}%
\pgfsetstrokeopacity{0.400000}%
\pgfsetdash{}{0pt}%
\pgfpathmoveto{\pgfqpoint{0.762805in}{2.375123in}}%
\pgfpathcurveto{\pgfqpoint{0.773855in}{2.375123in}}{\pgfqpoint{0.784454in}{2.379513in}}{\pgfqpoint{0.792268in}{2.387327in}}%
\pgfpathcurveto{\pgfqpoint{0.800081in}{2.395140in}}{\pgfqpoint{0.804472in}{2.405739in}}{\pgfqpoint{0.804472in}{2.416789in}}%
\pgfpathcurveto{\pgfqpoint{0.804472in}{2.427840in}}{\pgfqpoint{0.800081in}{2.438439in}}{\pgfqpoint{0.792268in}{2.446252in}}%
\pgfpathcurveto{\pgfqpoint{0.784454in}{2.454066in}}{\pgfqpoint{0.773855in}{2.458456in}}{\pgfqpoint{0.762805in}{2.458456in}}%
\pgfpathcurveto{\pgfqpoint{0.751755in}{2.458456in}}{\pgfqpoint{0.741156in}{2.454066in}}{\pgfqpoint{0.733342in}{2.446252in}}%
\pgfpathcurveto{\pgfqpoint{0.725529in}{2.438439in}}{\pgfqpoint{0.721138in}{2.427840in}}{\pgfqpoint{0.721138in}{2.416789in}}%
\pgfpathcurveto{\pgfqpoint{0.721138in}{2.405739in}}{\pgfqpoint{0.725529in}{2.395140in}}{\pgfqpoint{0.733342in}{2.387327in}}%
\pgfpathcurveto{\pgfqpoint{0.741156in}{2.379513in}}{\pgfqpoint{0.751755in}{2.375123in}}{\pgfqpoint{0.762805in}{2.375123in}}%
\pgfpathclose%
\pgfusepath{stroke,fill}%
\end{pgfscope}%
\begin{pgfscope}%
\pgfpathrectangle{\pgfqpoint{0.511823in}{0.504323in}}{\pgfqpoint{3.218177in}{3.225677in}} %
\pgfusepath{clip}%
\pgfsetbuttcap%
\pgfsetroundjoin%
\definecolor{currentfill}{rgb}{0.000000,0.000000,0.545098}%
\pgfsetfillcolor{currentfill}%
\pgfsetfillopacity{0.400000}%
\pgfsetlinewidth{0.501875pt}%
\definecolor{currentstroke}{rgb}{0.000000,0.000000,0.545098}%
\pgfsetstrokecolor{currentstroke}%
\pgfsetstrokeopacity{0.400000}%
\pgfsetdash{}{0pt}%
\pgfpathmoveto{\pgfqpoint{0.759862in}{2.447439in}}%
\pgfpathcurveto{\pgfqpoint{0.770912in}{2.447439in}}{\pgfqpoint{0.781511in}{2.451829in}}{\pgfqpoint{0.789325in}{2.459642in}}%
\pgfpathcurveto{\pgfqpoint{0.797139in}{2.467456in}}{\pgfqpoint{0.801529in}{2.478055in}}{\pgfqpoint{0.801529in}{2.489105in}}%
\pgfpathcurveto{\pgfqpoint{0.801529in}{2.500155in}}{\pgfqpoint{0.797139in}{2.510754in}}{\pgfqpoint{0.789325in}{2.518568in}}%
\pgfpathcurveto{\pgfqpoint{0.781511in}{2.526382in}}{\pgfqpoint{0.770912in}{2.530772in}}{\pgfqpoint{0.759862in}{2.530772in}}%
\pgfpathcurveto{\pgfqpoint{0.748812in}{2.530772in}}{\pgfqpoint{0.738213in}{2.526382in}}{\pgfqpoint{0.730399in}{2.518568in}}%
\pgfpathcurveto{\pgfqpoint{0.722586in}{2.510754in}}{\pgfqpoint{0.718195in}{2.500155in}}{\pgfqpoint{0.718195in}{2.489105in}}%
\pgfpathcurveto{\pgfqpoint{0.718195in}{2.478055in}}{\pgfqpoint{0.722586in}{2.467456in}}{\pgfqpoint{0.730399in}{2.459642in}}%
\pgfpathcurveto{\pgfqpoint{0.738213in}{2.451829in}}{\pgfqpoint{0.748812in}{2.447439in}}{\pgfqpoint{0.759862in}{2.447439in}}%
\pgfpathclose%
\pgfusepath{stroke,fill}%
\end{pgfscope}%
\begin{pgfscope}%
\pgfpathrectangle{\pgfqpoint{0.511823in}{0.504323in}}{\pgfqpoint{3.218177in}{3.225677in}} %
\pgfusepath{clip}%
\pgfsetbuttcap%
\pgfsetroundjoin%
\definecolor{currentfill}{rgb}{0.000000,0.000000,0.545098}%
\pgfsetfillcolor{currentfill}%
\pgfsetfillopacity{0.400000}%
\pgfsetlinewidth{0.501875pt}%
\definecolor{currentstroke}{rgb}{0.000000,0.000000,0.545098}%
\pgfsetstrokecolor{currentstroke}%
\pgfsetstrokeopacity{0.400000}%
\pgfsetdash{}{0pt}%
\pgfpathmoveto{\pgfqpoint{0.759476in}{2.601542in}}%
\pgfpathcurveto{\pgfqpoint{0.770526in}{2.601542in}}{\pgfqpoint{0.781125in}{2.605933in}}{\pgfqpoint{0.788938in}{2.613746in}}%
\pgfpathcurveto{\pgfqpoint{0.796752in}{2.621560in}}{\pgfqpoint{0.801142in}{2.632159in}}{\pgfqpoint{0.801142in}{2.643209in}}%
\pgfpathcurveto{\pgfqpoint{0.801142in}{2.654259in}}{\pgfqpoint{0.796752in}{2.664858in}}{\pgfqpoint{0.788938in}{2.672672in}}%
\pgfpathcurveto{\pgfqpoint{0.781125in}{2.680485in}}{\pgfqpoint{0.770526in}{2.684876in}}{\pgfqpoint{0.759476in}{2.684876in}}%
\pgfpathcurveto{\pgfqpoint{0.748426in}{2.684876in}}{\pgfqpoint{0.737826in}{2.680485in}}{\pgfqpoint{0.730013in}{2.672672in}}%
\pgfpathcurveto{\pgfqpoint{0.722199in}{2.664858in}}{\pgfqpoint{0.717809in}{2.654259in}}{\pgfqpoint{0.717809in}{2.643209in}}%
\pgfpathcurveto{\pgfqpoint{0.717809in}{2.632159in}}{\pgfqpoint{0.722199in}{2.621560in}}{\pgfqpoint{0.730013in}{2.613746in}}%
\pgfpathcurveto{\pgfqpoint{0.737826in}{2.605933in}}{\pgfqpoint{0.748426in}{2.601542in}}{\pgfqpoint{0.759476in}{2.601542in}}%
\pgfpathclose%
\pgfusepath{stroke,fill}%
\end{pgfscope}%
\begin{pgfscope}%
\pgfpathrectangle{\pgfqpoint{0.511823in}{0.504323in}}{\pgfqpoint{3.218177in}{3.225677in}} %
\pgfusepath{clip}%
\pgfsetbuttcap%
\pgfsetroundjoin%
\definecolor{currentfill}{rgb}{0.000000,0.000000,0.545098}%
\pgfsetfillcolor{currentfill}%
\pgfsetfillopacity{0.400000}%
\pgfsetlinewidth{0.501875pt}%
\definecolor{currentstroke}{rgb}{0.000000,0.000000,0.545098}%
\pgfsetstrokecolor{currentstroke}%
\pgfsetstrokeopacity{0.400000}%
\pgfsetdash{}{0pt}%
\pgfpathmoveto{\pgfqpoint{0.744199in}{2.346484in}}%
\pgfpathcurveto{\pgfqpoint{0.755250in}{2.346484in}}{\pgfqpoint{0.765849in}{2.350874in}}{\pgfqpoint{0.773662in}{2.358688in}}%
\pgfpathcurveto{\pgfqpoint{0.781476in}{2.366502in}}{\pgfqpoint{0.785866in}{2.377101in}}{\pgfqpoint{0.785866in}{2.388151in}}%
\pgfpathcurveto{\pgfqpoint{0.785866in}{2.399201in}}{\pgfqpoint{0.781476in}{2.409800in}}{\pgfqpoint{0.773662in}{2.417614in}}%
\pgfpathcurveto{\pgfqpoint{0.765849in}{2.425427in}}{\pgfqpoint{0.755250in}{2.429817in}}{\pgfqpoint{0.744199in}{2.429817in}}%
\pgfpathcurveto{\pgfqpoint{0.733149in}{2.429817in}}{\pgfqpoint{0.722550in}{2.425427in}}{\pgfqpoint{0.714737in}{2.417614in}}%
\pgfpathcurveto{\pgfqpoint{0.706923in}{2.409800in}}{\pgfqpoint{0.702533in}{2.399201in}}{\pgfqpoint{0.702533in}{2.388151in}}%
\pgfpathcurveto{\pgfqpoint{0.702533in}{2.377101in}}{\pgfqpoint{0.706923in}{2.366502in}}{\pgfqpoint{0.714737in}{2.358688in}}%
\pgfpathcurveto{\pgfqpoint{0.722550in}{2.350874in}}{\pgfqpoint{0.733149in}{2.346484in}}{\pgfqpoint{0.744199in}{2.346484in}}%
\pgfpathclose%
\pgfusepath{stroke,fill}%
\end{pgfscope}%
\begin{pgfscope}%
\pgfpathrectangle{\pgfqpoint{0.511823in}{0.504323in}}{\pgfqpoint{3.218177in}{3.225677in}} %
\pgfusepath{clip}%
\pgfsetbuttcap%
\pgfsetroundjoin%
\definecolor{currentfill}{rgb}{0.000000,0.000000,0.545098}%
\pgfsetfillcolor{currentfill}%
\pgfsetfillopacity{0.400000}%
\pgfsetlinewidth{0.501875pt}%
\definecolor{currentstroke}{rgb}{0.000000,0.000000,0.545098}%
\pgfsetstrokecolor{currentstroke}%
\pgfsetstrokeopacity{0.400000}%
\pgfsetdash{}{0pt}%
\pgfpathmoveto{\pgfqpoint{0.743226in}{2.504271in}}%
\pgfpathcurveto{\pgfqpoint{0.754276in}{2.504271in}}{\pgfqpoint{0.764875in}{2.508662in}}{\pgfqpoint{0.772689in}{2.516475in}}%
\pgfpathcurveto{\pgfqpoint{0.780502in}{2.524289in}}{\pgfqpoint{0.784893in}{2.534888in}}{\pgfqpoint{0.784893in}{2.545938in}}%
\pgfpathcurveto{\pgfqpoint{0.784893in}{2.556988in}}{\pgfqpoint{0.780502in}{2.567587in}}{\pgfqpoint{0.772689in}{2.575401in}}%
\pgfpathcurveto{\pgfqpoint{0.764875in}{2.583215in}}{\pgfqpoint{0.754276in}{2.587605in}}{\pgfqpoint{0.743226in}{2.587605in}}%
\pgfpathcurveto{\pgfqpoint{0.732176in}{2.587605in}}{\pgfqpoint{0.721577in}{2.583215in}}{\pgfqpoint{0.713763in}{2.575401in}}%
\pgfpathcurveto{\pgfqpoint{0.705949in}{2.567587in}}{\pgfqpoint{0.701559in}{2.556988in}}{\pgfqpoint{0.701559in}{2.545938in}}%
\pgfpathcurveto{\pgfqpoint{0.701559in}{2.534888in}}{\pgfqpoint{0.705949in}{2.524289in}}{\pgfqpoint{0.713763in}{2.516475in}}%
\pgfpathcurveto{\pgfqpoint{0.721577in}{2.508662in}}{\pgfqpoint{0.732176in}{2.504271in}}{\pgfqpoint{0.743226in}{2.504271in}}%
\pgfpathclose%
\pgfusepath{stroke,fill}%
\end{pgfscope}%
\begin{pgfscope}%
\pgfpathrectangle{\pgfqpoint{0.511823in}{0.504323in}}{\pgfqpoint{3.218177in}{3.225677in}} %
\pgfusepath{clip}%
\pgfsetbuttcap%
\pgfsetroundjoin%
\definecolor{currentfill}{rgb}{0.000000,0.000000,0.545098}%
\pgfsetfillcolor{currentfill}%
\pgfsetfillopacity{0.400000}%
\pgfsetlinewidth{0.501875pt}%
\definecolor{currentstroke}{rgb}{0.000000,0.000000,0.545098}%
\pgfsetstrokecolor{currentstroke}%
\pgfsetstrokeopacity{0.400000}%
\pgfsetdash{}{0pt}%
\pgfpathmoveto{\pgfqpoint{0.729704in}{2.237964in}}%
\pgfpathcurveto{\pgfqpoint{0.740754in}{2.237964in}}{\pgfqpoint{0.751353in}{2.242354in}}{\pgfqpoint{0.759167in}{2.250168in}}%
\pgfpathcurveto{\pgfqpoint{0.766981in}{2.257982in}}{\pgfqpoint{0.771371in}{2.268581in}}{\pgfqpoint{0.771371in}{2.279631in}}%
\pgfpathcurveto{\pgfqpoint{0.771371in}{2.290681in}}{\pgfqpoint{0.766981in}{2.301280in}}{\pgfqpoint{0.759167in}{2.309094in}}%
\pgfpathcurveto{\pgfqpoint{0.751353in}{2.316907in}}{\pgfqpoint{0.740754in}{2.321298in}}{\pgfqpoint{0.729704in}{2.321298in}}%
\pgfpathcurveto{\pgfqpoint{0.718654in}{2.321298in}}{\pgfqpoint{0.708055in}{2.316907in}}{\pgfqpoint{0.700241in}{2.309094in}}%
\pgfpathcurveto{\pgfqpoint{0.692428in}{2.301280in}}{\pgfqpoint{0.688037in}{2.290681in}}{\pgfqpoint{0.688037in}{2.279631in}}%
\pgfpathcurveto{\pgfqpoint{0.688037in}{2.268581in}}{\pgfqpoint{0.692428in}{2.257982in}}{\pgfqpoint{0.700241in}{2.250168in}}%
\pgfpathcurveto{\pgfqpoint{0.708055in}{2.242354in}}{\pgfqpoint{0.718654in}{2.237964in}}{\pgfqpoint{0.729704in}{2.237964in}}%
\pgfpathclose%
\pgfusepath{stroke,fill}%
\end{pgfscope}%
\begin{pgfscope}%
\pgfpathrectangle{\pgfqpoint{0.511823in}{0.504323in}}{\pgfqpoint{3.218177in}{3.225677in}} %
\pgfusepath{clip}%
\pgfsetbuttcap%
\pgfsetroundjoin%
\definecolor{currentfill}{rgb}{0.000000,0.000000,0.545098}%
\pgfsetfillcolor{currentfill}%
\pgfsetfillopacity{0.400000}%
\pgfsetlinewidth{0.501875pt}%
\definecolor{currentstroke}{rgb}{0.000000,0.000000,0.545098}%
\pgfsetstrokecolor{currentstroke}%
\pgfsetstrokeopacity{0.400000}%
\pgfsetdash{}{0pt}%
\pgfpathmoveto{\pgfqpoint{0.728251in}{2.405027in}}%
\pgfpathcurveto{\pgfqpoint{0.739301in}{2.405027in}}{\pgfqpoint{0.749900in}{2.409417in}}{\pgfqpoint{0.757714in}{2.417230in}}%
\pgfpathcurveto{\pgfqpoint{0.765528in}{2.425044in}}{\pgfqpoint{0.769918in}{2.435643in}}{\pgfqpoint{0.769918in}{2.446693in}}%
\pgfpathcurveto{\pgfqpoint{0.769918in}{2.457743in}}{\pgfqpoint{0.765528in}{2.468342in}}{\pgfqpoint{0.757714in}{2.476156in}}%
\pgfpathcurveto{\pgfqpoint{0.749900in}{2.483970in}}{\pgfqpoint{0.739301in}{2.488360in}}{\pgfqpoint{0.728251in}{2.488360in}}%
\pgfpathcurveto{\pgfqpoint{0.717201in}{2.488360in}}{\pgfqpoint{0.706602in}{2.483970in}}{\pgfqpoint{0.698788in}{2.476156in}}%
\pgfpathcurveto{\pgfqpoint{0.690975in}{2.468342in}}{\pgfqpoint{0.686585in}{2.457743in}}{\pgfqpoint{0.686585in}{2.446693in}}%
\pgfpathcurveto{\pgfqpoint{0.686585in}{2.435643in}}{\pgfqpoint{0.690975in}{2.425044in}}{\pgfqpoint{0.698788in}{2.417230in}}%
\pgfpathcurveto{\pgfqpoint{0.706602in}{2.409417in}}{\pgfqpoint{0.717201in}{2.405027in}}{\pgfqpoint{0.728251in}{2.405027in}}%
\pgfpathclose%
\pgfusepath{stroke,fill}%
\end{pgfscope}%
\begin{pgfscope}%
\pgfpathrectangle{\pgfqpoint{0.511823in}{0.504323in}}{\pgfqpoint{3.218177in}{3.225677in}} %
\pgfusepath{clip}%
\pgfsetbuttcap%
\pgfsetroundjoin%
\definecolor{currentfill}{rgb}{0.000000,0.000000,0.545098}%
\pgfsetfillcolor{currentfill}%
\pgfsetfillopacity{0.400000}%
\pgfsetlinewidth{0.501875pt}%
\definecolor{currentstroke}{rgb}{0.000000,0.000000,0.545098}%
\pgfsetstrokecolor{currentstroke}%
\pgfsetstrokeopacity{0.400000}%
\pgfsetdash{}{0pt}%
\pgfpathmoveto{\pgfqpoint{0.719955in}{2.290667in}}%
\pgfpathcurveto{\pgfqpoint{0.731005in}{2.290667in}}{\pgfqpoint{0.741604in}{2.295057in}}{\pgfqpoint{0.749418in}{2.302870in}}%
\pgfpathcurveto{\pgfqpoint{0.757231in}{2.310684in}}{\pgfqpoint{0.761621in}{2.321283in}}{\pgfqpoint{0.761621in}{2.332333in}}%
\pgfpathcurveto{\pgfqpoint{0.761621in}{2.343383in}}{\pgfqpoint{0.757231in}{2.353982in}}{\pgfqpoint{0.749418in}{2.361796in}}%
\pgfpathcurveto{\pgfqpoint{0.741604in}{2.369610in}}{\pgfqpoint{0.731005in}{2.374000in}}{\pgfqpoint{0.719955in}{2.374000in}}%
\pgfpathcurveto{\pgfqpoint{0.708905in}{2.374000in}}{\pgfqpoint{0.698306in}{2.369610in}}{\pgfqpoint{0.690492in}{2.361796in}}%
\pgfpathcurveto{\pgfqpoint{0.682678in}{2.353982in}}{\pgfqpoint{0.678288in}{2.343383in}}{\pgfqpoint{0.678288in}{2.332333in}}%
\pgfpathcurveto{\pgfqpoint{0.678288in}{2.321283in}}{\pgfqpoint{0.682678in}{2.310684in}}{\pgfqpoint{0.690492in}{2.302870in}}%
\pgfpathcurveto{\pgfqpoint{0.698306in}{2.295057in}}{\pgfqpoint{0.708905in}{2.290667in}}{\pgfqpoint{0.719955in}{2.290667in}}%
\pgfpathclose%
\pgfusepath{stroke,fill}%
\end{pgfscope}%
\begin{pgfscope}%
\pgfpathrectangle{\pgfqpoint{0.511823in}{0.504323in}}{\pgfqpoint{3.218177in}{3.225677in}} %
\pgfusepath{clip}%
\pgfsetbuttcap%
\pgfsetroundjoin%
\definecolor{currentfill}{rgb}{0.000000,0.000000,0.545098}%
\pgfsetfillcolor{currentfill}%
\pgfsetfillopacity{0.400000}%
\pgfsetlinewidth{0.501875pt}%
\definecolor{currentstroke}{rgb}{0.000000,0.000000,0.545098}%
\pgfsetstrokecolor{currentstroke}%
\pgfsetstrokeopacity{0.400000}%
\pgfsetdash{}{0pt}%
\pgfpathmoveto{\pgfqpoint{0.720357in}{2.644694in}}%
\pgfpathcurveto{\pgfqpoint{0.731407in}{2.644694in}}{\pgfqpoint{0.742006in}{2.649084in}}{\pgfqpoint{0.749820in}{2.656898in}}%
\pgfpathcurveto{\pgfqpoint{0.757633in}{2.664711in}}{\pgfqpoint{0.762024in}{2.675310in}}{\pgfqpoint{0.762024in}{2.686360in}}%
\pgfpathcurveto{\pgfqpoint{0.762024in}{2.697411in}}{\pgfqpoint{0.757633in}{2.708010in}}{\pgfqpoint{0.749820in}{2.715823in}}%
\pgfpathcurveto{\pgfqpoint{0.742006in}{2.723637in}}{\pgfqpoint{0.731407in}{2.728027in}}{\pgfqpoint{0.720357in}{2.728027in}}%
\pgfpathcurveto{\pgfqpoint{0.709307in}{2.728027in}}{\pgfqpoint{0.698708in}{2.723637in}}{\pgfqpoint{0.690894in}{2.715823in}}%
\pgfpathcurveto{\pgfqpoint{0.683081in}{2.708010in}}{\pgfqpoint{0.678690in}{2.697411in}}{\pgfqpoint{0.678690in}{2.686360in}}%
\pgfpathcurveto{\pgfqpoint{0.678690in}{2.675310in}}{\pgfqpoint{0.683081in}{2.664711in}}{\pgfqpoint{0.690894in}{2.656898in}}%
\pgfpathcurveto{\pgfqpoint{0.698708in}{2.649084in}}{\pgfqpoint{0.709307in}{2.644694in}}{\pgfqpoint{0.720357in}{2.644694in}}%
\pgfpathclose%
\pgfusepath{stroke,fill}%
\end{pgfscope}%
\begin{pgfscope}%
\pgfpathrectangle{\pgfqpoint{0.511823in}{0.504323in}}{\pgfqpoint{3.218177in}{3.225677in}} %
\pgfusepath{clip}%
\pgfsetbuttcap%
\pgfsetroundjoin%
\definecolor{currentfill}{rgb}{0.000000,0.000000,0.545098}%
\pgfsetfillcolor{currentfill}%
\pgfsetfillopacity{0.400000}%
\pgfsetlinewidth{0.501875pt}%
\definecolor{currentstroke}{rgb}{0.000000,0.000000,0.545098}%
\pgfsetstrokecolor{currentstroke}%
\pgfsetstrokeopacity{0.400000}%
\pgfsetdash{}{0pt}%
\pgfpathmoveto{\pgfqpoint{0.708450in}{2.266195in}}%
\pgfpathcurveto{\pgfqpoint{0.719500in}{2.266195in}}{\pgfqpoint{0.730099in}{2.270586in}}{\pgfqpoint{0.737913in}{2.278399in}}%
\pgfpathcurveto{\pgfqpoint{0.745727in}{2.286213in}}{\pgfqpoint{0.750117in}{2.296812in}}{\pgfqpoint{0.750117in}{2.307862in}}%
\pgfpathcurveto{\pgfqpoint{0.750117in}{2.318912in}}{\pgfqpoint{0.745727in}{2.329511in}}{\pgfqpoint{0.737913in}{2.337325in}}%
\pgfpathcurveto{\pgfqpoint{0.730099in}{2.345139in}}{\pgfqpoint{0.719500in}{2.349529in}}{\pgfqpoint{0.708450in}{2.349529in}}%
\pgfpathcurveto{\pgfqpoint{0.697400in}{2.349529in}}{\pgfqpoint{0.686801in}{2.345139in}}{\pgfqpoint{0.678988in}{2.337325in}}%
\pgfpathcurveto{\pgfqpoint{0.671174in}{2.329511in}}{\pgfqpoint{0.666784in}{2.318912in}}{\pgfqpoint{0.666784in}{2.307862in}}%
\pgfpathcurveto{\pgfqpoint{0.666784in}{2.296812in}}{\pgfqpoint{0.671174in}{2.286213in}}{\pgfqpoint{0.678988in}{2.278399in}}%
\pgfpathcurveto{\pgfqpoint{0.686801in}{2.270586in}}{\pgfqpoint{0.697400in}{2.266195in}}{\pgfqpoint{0.708450in}{2.266195in}}%
\pgfpathclose%
\pgfusepath{stroke,fill}%
\end{pgfscope}%
\begin{pgfscope}%
\pgfpathrectangle{\pgfqpoint{0.511823in}{0.504323in}}{\pgfqpoint{3.218177in}{3.225677in}} %
\pgfusepath{clip}%
\pgfsetbuttcap%
\pgfsetroundjoin%
\definecolor{currentfill}{rgb}{0.000000,0.000000,0.545098}%
\pgfsetfillcolor{currentfill}%
\pgfsetfillopacity{0.400000}%
\pgfsetlinewidth{0.501875pt}%
\definecolor{currentstroke}{rgb}{0.000000,0.000000,0.545098}%
\pgfsetstrokecolor{currentstroke}%
\pgfsetstrokeopacity{0.400000}%
\pgfsetdash{}{0pt}%
\pgfpathmoveto{\pgfqpoint{0.704931in}{2.461662in}}%
\pgfpathcurveto{\pgfqpoint{0.715981in}{2.461662in}}{\pgfqpoint{0.726580in}{2.466052in}}{\pgfqpoint{0.734394in}{2.473866in}}%
\pgfpathcurveto{\pgfqpoint{0.742207in}{2.481679in}}{\pgfqpoint{0.746598in}{2.492278in}}{\pgfqpoint{0.746598in}{2.503329in}}%
\pgfpathcurveto{\pgfqpoint{0.746598in}{2.514379in}}{\pgfqpoint{0.742207in}{2.524978in}}{\pgfqpoint{0.734394in}{2.532791in}}%
\pgfpathcurveto{\pgfqpoint{0.726580in}{2.540605in}}{\pgfqpoint{0.715981in}{2.544995in}}{\pgfqpoint{0.704931in}{2.544995in}}%
\pgfpathcurveto{\pgfqpoint{0.693881in}{2.544995in}}{\pgfqpoint{0.683282in}{2.540605in}}{\pgfqpoint{0.675468in}{2.532791in}}%
\pgfpathcurveto{\pgfqpoint{0.667655in}{2.524978in}}{\pgfqpoint{0.663264in}{2.514379in}}{\pgfqpoint{0.663264in}{2.503329in}}%
\pgfpathcurveto{\pgfqpoint{0.663264in}{2.492278in}}{\pgfqpoint{0.667655in}{2.481679in}}{\pgfqpoint{0.675468in}{2.473866in}}%
\pgfpathcurveto{\pgfqpoint{0.683282in}{2.466052in}}{\pgfqpoint{0.693881in}{2.461662in}}{\pgfqpoint{0.704931in}{2.461662in}}%
\pgfpathclose%
\pgfusepath{stroke,fill}%
\end{pgfscope}%
\begin{pgfscope}%
\pgfpathrectangle{\pgfqpoint{0.511823in}{0.504323in}}{\pgfqpoint{3.218177in}{3.225677in}} %
\pgfusepath{clip}%
\pgfsetbuttcap%
\pgfsetroundjoin%
\definecolor{currentfill}{rgb}{0.000000,0.000000,0.545098}%
\pgfsetfillcolor{currentfill}%
\pgfsetfillopacity{0.400000}%
\pgfsetlinewidth{0.501875pt}%
\definecolor{currentstroke}{rgb}{0.000000,0.000000,0.545098}%
\pgfsetstrokecolor{currentstroke}%
\pgfsetstrokeopacity{0.400000}%
\pgfsetdash{}{0pt}%
\pgfpathmoveto{\pgfqpoint{0.699576in}{2.581198in}}%
\pgfpathcurveto{\pgfqpoint{0.710626in}{2.581198in}}{\pgfqpoint{0.721225in}{2.585588in}}{\pgfqpoint{0.729038in}{2.593402in}}%
\pgfpathcurveto{\pgfqpoint{0.736852in}{2.601215in}}{\pgfqpoint{0.741242in}{2.611815in}}{\pgfqpoint{0.741242in}{2.622865in}}%
\pgfpathcurveto{\pgfqpoint{0.741242in}{2.633915in}}{\pgfqpoint{0.736852in}{2.644514in}}{\pgfqpoint{0.729038in}{2.652327in}}%
\pgfpathcurveto{\pgfqpoint{0.721225in}{2.660141in}}{\pgfqpoint{0.710626in}{2.664531in}}{\pgfqpoint{0.699576in}{2.664531in}}%
\pgfpathcurveto{\pgfqpoint{0.688526in}{2.664531in}}{\pgfqpoint{0.677927in}{2.660141in}}{\pgfqpoint{0.670113in}{2.652327in}}%
\pgfpathcurveto{\pgfqpoint{0.662299in}{2.644514in}}{\pgfqpoint{0.657909in}{2.633915in}}{\pgfqpoint{0.657909in}{2.622865in}}%
\pgfpathcurveto{\pgfqpoint{0.657909in}{2.611815in}}{\pgfqpoint{0.662299in}{2.601215in}}{\pgfqpoint{0.670113in}{2.593402in}}%
\pgfpathcurveto{\pgfqpoint{0.677927in}{2.585588in}}{\pgfqpoint{0.688526in}{2.581198in}}{\pgfqpoint{0.699576in}{2.581198in}}%
\pgfpathclose%
\pgfusepath{stroke,fill}%
\end{pgfscope}%
\begin{pgfscope}%
\pgfpathrectangle{\pgfqpoint{0.511823in}{0.504323in}}{\pgfqpoint{3.218177in}{3.225677in}} %
\pgfusepath{clip}%
\pgfsetbuttcap%
\pgfsetroundjoin%
\definecolor{currentfill}{rgb}{0.000000,0.000000,0.545098}%
\pgfsetfillcolor{currentfill}%
\pgfsetfillopacity{0.400000}%
\pgfsetlinewidth{0.501875pt}%
\definecolor{currentstroke}{rgb}{0.000000,0.000000,0.545098}%
\pgfsetstrokecolor{currentstroke}%
\pgfsetstrokeopacity{0.400000}%
\pgfsetdash{}{0pt}%
\pgfpathmoveto{\pgfqpoint{0.692958in}{2.566469in}}%
\pgfpathcurveto{\pgfqpoint{0.704008in}{2.566469in}}{\pgfqpoint{0.714607in}{2.570859in}}{\pgfqpoint{0.722420in}{2.578673in}}%
\pgfpathcurveto{\pgfqpoint{0.730234in}{2.586486in}}{\pgfqpoint{0.734624in}{2.597085in}}{\pgfqpoint{0.734624in}{2.608136in}}%
\pgfpathcurveto{\pgfqpoint{0.734624in}{2.619186in}}{\pgfqpoint{0.730234in}{2.629785in}}{\pgfqpoint{0.722420in}{2.637598in}}%
\pgfpathcurveto{\pgfqpoint{0.714607in}{2.645412in}}{\pgfqpoint{0.704008in}{2.649802in}}{\pgfqpoint{0.692958in}{2.649802in}}%
\pgfpathcurveto{\pgfqpoint{0.681907in}{2.649802in}}{\pgfqpoint{0.671308in}{2.645412in}}{\pgfqpoint{0.663495in}{2.637598in}}%
\pgfpathcurveto{\pgfqpoint{0.655681in}{2.629785in}}{\pgfqpoint{0.651291in}{2.619186in}}{\pgfqpoint{0.651291in}{2.608136in}}%
\pgfpathcurveto{\pgfqpoint{0.651291in}{2.597085in}}{\pgfqpoint{0.655681in}{2.586486in}}{\pgfqpoint{0.663495in}{2.578673in}}%
\pgfpathcurveto{\pgfqpoint{0.671308in}{2.570859in}}{\pgfqpoint{0.681907in}{2.566469in}}{\pgfqpoint{0.692958in}{2.566469in}}%
\pgfpathclose%
\pgfusepath{stroke,fill}%
\end{pgfscope}%
\begin{pgfscope}%
\pgfpathrectangle{\pgfqpoint{0.511823in}{0.504323in}}{\pgfqpoint{3.218177in}{3.225677in}} %
\pgfusepath{clip}%
\pgfsetbuttcap%
\pgfsetroundjoin%
\definecolor{currentfill}{rgb}{0.000000,0.000000,0.545098}%
\pgfsetfillcolor{currentfill}%
\pgfsetfillopacity{0.400000}%
\pgfsetlinewidth{0.501875pt}%
\definecolor{currentstroke}{rgb}{0.000000,0.000000,0.545098}%
\pgfsetstrokecolor{currentstroke}%
\pgfsetstrokeopacity{0.400000}%
\pgfsetdash{}{0pt}%
\pgfpathmoveto{\pgfqpoint{0.686439in}{2.586759in}}%
\pgfpathcurveto{\pgfqpoint{0.697489in}{2.586759in}}{\pgfqpoint{0.708088in}{2.591149in}}{\pgfqpoint{0.715902in}{2.598963in}}%
\pgfpathcurveto{\pgfqpoint{0.723716in}{2.606777in}}{\pgfqpoint{0.728106in}{2.617376in}}{\pgfqpoint{0.728106in}{2.628426in}}%
\pgfpathcurveto{\pgfqpoint{0.728106in}{2.639476in}}{\pgfqpoint{0.723716in}{2.650075in}}{\pgfqpoint{0.715902in}{2.657889in}}%
\pgfpathcurveto{\pgfqpoint{0.708088in}{2.665702in}}{\pgfqpoint{0.697489in}{2.670093in}}{\pgfqpoint{0.686439in}{2.670093in}}%
\pgfpathcurveto{\pgfqpoint{0.675389in}{2.670093in}}{\pgfqpoint{0.664790in}{2.665702in}}{\pgfqpoint{0.656976in}{2.657889in}}%
\pgfpathcurveto{\pgfqpoint{0.649163in}{2.650075in}}{\pgfqpoint{0.644772in}{2.639476in}}{\pgfqpoint{0.644772in}{2.628426in}}%
\pgfpathcurveto{\pgfqpoint{0.644772in}{2.617376in}}{\pgfqpoint{0.649163in}{2.606777in}}{\pgfqpoint{0.656976in}{2.598963in}}%
\pgfpathcurveto{\pgfqpoint{0.664790in}{2.591149in}}{\pgfqpoint{0.675389in}{2.586759in}}{\pgfqpoint{0.686439in}{2.586759in}}%
\pgfpathclose%
\pgfusepath{stroke,fill}%
\end{pgfscope}%
\begin{pgfscope}%
\pgfsetrectcap%
\pgfsetmiterjoin%
\pgfsetlinewidth{3.011250pt}%
\definecolor{currentstroke}{rgb}{0.941176,0.941176,0.941176}%
\pgfsetstrokecolor{currentstroke}%
\pgfsetdash{}{0pt}%
\pgfpathmoveto{\pgfqpoint{0.511823in}{0.504323in}}%
\pgfpathlineto{\pgfqpoint{0.511823in}{3.730000in}}%
\pgfusepath{stroke}%
\end{pgfscope}%
\begin{pgfscope}%
\pgfsetrectcap%
\pgfsetmiterjoin%
\pgfsetlinewidth{3.011250pt}%
\definecolor{currentstroke}{rgb}{0.941176,0.941176,0.941176}%
\pgfsetstrokecolor{currentstroke}%
\pgfsetdash{}{0pt}%
\pgfpathmoveto{\pgfqpoint{3.730000in}{0.504323in}}%
\pgfpathlineto{\pgfqpoint{3.730000in}{3.730000in}}%
\pgfusepath{stroke}%
\end{pgfscope}%
\begin{pgfscope}%
\pgfsetrectcap%
\pgfsetmiterjoin%
\pgfsetlinewidth{3.011250pt}%
\definecolor{currentstroke}{rgb}{0.941176,0.941176,0.941176}%
\pgfsetstrokecolor{currentstroke}%
\pgfsetdash{}{0pt}%
\pgfpathmoveto{\pgfqpoint{0.511823in}{0.504323in}}%
\pgfpathlineto{\pgfqpoint{3.730000in}{0.504323in}}%
\pgfusepath{stroke}%
\end{pgfscope}%
\begin{pgfscope}%
\pgfsetrectcap%
\pgfsetmiterjoin%
\pgfsetlinewidth{3.011250pt}%
\definecolor{currentstroke}{rgb}{0.941176,0.941176,0.941176}%
\pgfsetstrokecolor{currentstroke}%
\pgfsetdash{}{0pt}%
\pgfpathmoveto{\pgfqpoint{0.511823in}{3.730000in}}%
\pgfpathlineto{\pgfqpoint{3.730000in}{3.730000in}}%
\pgfusepath{stroke}%
\end{pgfscope}%
\begin{pgfscope}%
\pgfsetbuttcap%
\pgfsetmiterjoin%
\definecolor{currentfill}{rgb}{0.941176,0.941176,0.941176}%
\pgfsetfillcolor{currentfill}%
\pgfsetfillopacity{0.800000}%
\pgfsetlinewidth{0.501875pt}%
\definecolor{currentstroke}{rgb}{0.800000,0.800000,0.800000}%
\pgfsetstrokecolor{currentstroke}%
\pgfsetstrokeopacity{0.800000}%
\pgfsetdash{}{0pt}%
\pgfpathmoveto{\pgfqpoint{2.510417in}{3.210834in}}%
\pgfpathlineto{\pgfqpoint{3.632778in}{3.210834in}}%
\pgfpathquadraticcurveto{\pgfqpoint{3.660556in}{3.210834in}}{\pgfqpoint{3.660556in}{3.238611in}}%
\pgfpathlineto{\pgfqpoint{3.660556in}{3.632778in}}%
\pgfpathquadraticcurveto{\pgfqpoint{3.660556in}{3.660556in}}{\pgfqpoint{3.632778in}{3.660556in}}%
\pgfpathlineto{\pgfqpoint{2.510417in}{3.660556in}}%
\pgfpathquadraticcurveto{\pgfqpoint{2.482639in}{3.660556in}}{\pgfqpoint{2.482639in}{3.632778in}}%
\pgfpathlineto{\pgfqpoint{2.482639in}{3.238611in}}%
\pgfpathquadraticcurveto{\pgfqpoint{2.482639in}{3.210834in}}{\pgfqpoint{2.510417in}{3.210834in}}%
\pgfpathclose%
\pgfusepath{stroke,fill}%
\end{pgfscope}%
\begin{pgfscope}%
\pgfsetbuttcap%
\pgfsetroundjoin%
\definecolor{currentfill}{rgb}{0.501961,0.000000,0.000000}%
\pgfsetfillcolor{currentfill}%
\pgfsetfillopacity{0.400000}%
\pgfsetlinewidth{0.501875pt}%
\definecolor{currentstroke}{rgb}{0.501961,0.000000,0.000000}%
\pgfsetstrokecolor{currentstroke}%
\pgfsetstrokeopacity{0.400000}%
\pgfsetdash{}{0pt}%
\pgfpathmoveto{\pgfqpoint{2.677083in}{3.500903in}}%
\pgfpathcurveto{\pgfqpoint{2.688133in}{3.500903in}}{\pgfqpoint{2.698732in}{3.505293in}}{\pgfqpoint{2.706546in}{3.513107in}}%
\pgfpathcurveto{\pgfqpoint{2.714360in}{3.520920in}}{\pgfqpoint{2.718750in}{3.531519in}}{\pgfqpoint{2.718750in}{3.542569in}}%
\pgfpathcurveto{\pgfqpoint{2.718750in}{3.553620in}}{\pgfqpoint{2.714360in}{3.564219in}}{\pgfqpoint{2.706546in}{3.572032in}}%
\pgfpathcurveto{\pgfqpoint{2.698732in}{3.579846in}}{\pgfqpoint{2.688133in}{3.584236in}}{\pgfqpoint{2.677083in}{3.584236in}}%
\pgfpathcurveto{\pgfqpoint{2.666033in}{3.584236in}}{\pgfqpoint{2.655434in}{3.579846in}}{\pgfqpoint{2.647621in}{3.572032in}}%
\pgfpathcurveto{\pgfqpoint{2.639807in}{3.564219in}}{\pgfqpoint{2.635417in}{3.553620in}}{\pgfqpoint{2.635417in}{3.542569in}}%
\pgfpathcurveto{\pgfqpoint{2.635417in}{3.531519in}}{\pgfqpoint{2.639807in}{3.520920in}}{\pgfqpoint{2.647621in}{3.513107in}}%
\pgfpathcurveto{\pgfqpoint{2.655434in}{3.505293in}}{\pgfqpoint{2.666033in}{3.500903in}}{\pgfqpoint{2.677083in}{3.500903in}}%
\pgfpathclose%
\pgfusepath{stroke,fill}%
\end{pgfscope}%
\begin{pgfscope}%
\pgftext[x=2.927083in,y=3.506111in,left,base]{\rmfamily\fontsize{10.000000}{12.000000}\selectfont Background}%
\end{pgfscope}%
\begin{pgfscope}%
\pgfsetbuttcap%
\pgfsetroundjoin%
\definecolor{currentfill}{rgb}{0.000000,0.000000,0.545098}%
\pgfsetfillcolor{currentfill}%
\pgfsetfillopacity{0.400000}%
\pgfsetlinewidth{0.501875pt}%
\definecolor{currentstroke}{rgb}{0.000000,0.000000,0.545098}%
\pgfsetstrokecolor{currentstroke}%
\pgfsetstrokeopacity{0.400000}%
\pgfsetdash{}{0pt}%
\pgfpathmoveto{\pgfqpoint{2.677083in}{3.296875in}}%
\pgfpathcurveto{\pgfqpoint{2.688133in}{3.296875in}}{\pgfqpoint{2.698732in}{3.301265in}}{\pgfqpoint{2.706546in}{3.309079in}}%
\pgfpathcurveto{\pgfqpoint{2.714360in}{3.316893in}}{\pgfqpoint{2.718750in}{3.327492in}}{\pgfqpoint{2.718750in}{3.338542in}}%
\pgfpathcurveto{\pgfqpoint{2.718750in}{3.349592in}}{\pgfqpoint{2.714360in}{3.360191in}}{\pgfqpoint{2.706546in}{3.368005in}}%
\pgfpathcurveto{\pgfqpoint{2.698732in}{3.375818in}}{\pgfqpoint{2.688133in}{3.380208in}}{\pgfqpoint{2.677083in}{3.380208in}}%
\pgfpathcurveto{\pgfqpoint{2.666033in}{3.380208in}}{\pgfqpoint{2.655434in}{3.375818in}}{\pgfqpoint{2.647621in}{3.368005in}}%
\pgfpathcurveto{\pgfqpoint{2.639807in}{3.360191in}}{\pgfqpoint{2.635417in}{3.349592in}}{\pgfqpoint{2.635417in}{3.338542in}}%
\pgfpathcurveto{\pgfqpoint{2.635417in}{3.327492in}}{\pgfqpoint{2.639807in}{3.316893in}}{\pgfqpoint{2.647621in}{3.309079in}}%
\pgfpathcurveto{\pgfqpoint{2.655434in}{3.301265in}}{\pgfqpoint{2.666033in}{3.296875in}}{\pgfqpoint{2.677083in}{3.296875in}}%
\pgfpathclose%
\pgfusepath{stroke,fill}%
\end{pgfscope}%
\begin{pgfscope}%
\pgftext[x=2.927083in,y=3.302083in,left,base]{\rmfamily\fontsize{10.000000}{12.000000}\selectfont Signal}%
\end{pgfscope}%
\end{pgfpicture}%
\makeatother%
\endgroup%

\end{sidecaption}
\label{fig:nonlinear_decision_boundary}
\end{figure}
If we consider a two or three dimensional feature space, this distribution of the events is typically easy to visualize, and can often be explained through a simple kinematical relation between the features. Thus, instead of a box, we might draw a ring, or perhaps some other shape as a `lasso' around the events. This kind of delineation between signal and background events is known in machine learning terms as a \emph{decision boundary}. As we go to higher dimensional feature spaces, choosing this decision boundary by visual inspection becomes virtually impossible - instead, we resort to looking at one and two dimensional slices of the space. 

However, inspecting these lower dimensional slices may fail to fully capture the potentially complicated correlations among the features, and we might end up rejecting too many signal events, or accepting too many background events. Take for example the distribution of data from a toy experiment in \autoref{fig:nonlinear_decision_boundary}. The decision boundary that separates the signal events (blue) and the background events (red) is curved. Applying rectilinear cuts in this feature space would clearly be suboptimal. Machine learning (ML) techniques (or, as they are sometimes referred to in the particle physics literature, multivariate techniques) can harness the correlations between features more efficiently than manually attempting to find the optimal decision boundary in feature space.

In the context of particle physics, a \emph{binary classifier} is a function that takes as input various features of an event, and outputs a score for the event corresponding to whether it is more signal-like or more background-like. The form of the function itself depends on a number of parameters, which are \emph{a priori} unknown. In \emph{supervised learning} (which encompasses the most commonly-used ML techniques), these parameters are set by feeding the classifier a set of \emph{labeled training data}. After the parameters are determined, we can evaluate the performance of the classifier by using it on \emph{test data}, which are unlabeled. While there are a number of commonly-used classifiers to choose from, for our study, we chose the Gradient Boosted Decision Tree classifier \footnote{For a description of this classifier and a good review of statistical learning in general, see \citep{Hastie2011}.}, since it performs quite well `out-of-the-box', with minimal tuning. This classifier is actually an example of an \emph{ensemble method} --- it combines the results from a large number of weak classifiers to assign scores to events. This combination is much more resistant to \emph{overtraining}, that is, the combined classifier will not be too specialized to the training set and can perform well on test sets as well.

While the analysis in \autoref{ch:LightChargedHiggs} does not employ ML techniques, the analyses in chapters \ref{ch:ExoticHiggs} and \ref{ch:DM_100_TeV} do.
