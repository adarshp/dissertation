\chapter{The Standard Model and Beyond}\label{ch:theory}

\newcommand{\ct}{\cos\theta_w}
\newcommand{\st}{\sin\theta_w}

In this chapter, we will provide some of the theoretical background that our research builds upon. We will first provide a lightning-quick review of the Standard Model particle content and gauge structure, followed by an introduction to Two-Higgs Doublet Models and the Minimal Supersymmetric Standard Model. For excellent reviews of the Standard Model, please see \citep{Cheng1985,Schwartz2014,Zee2010,Peskin:1995ev}. For a detailed review of Two-Higgs Doublet Models, see  \cite{Branco2012}, For a pedagogical introduction to supersymmetry and the MSSM, see \cite{Martin1997}. Our treatment of the subjects is adapted from these articles. 

\section{The Standard Model}\label{sec:sm}

The fundamental constituents of matter are fermions. The interactions between them are mediated by particles known as gauge bosons, which arise from the gauge structure of the Standard Model:
\[SU(3)\times SU(2)\times U(1)_Y\]
This means that each fermion transforms in a unique way under transformations corresponding to these gauge groups. In addition to the fermions and gauge bosons, the standard model contains a scalar $SU(2)$ doublet \emph{H}, known as the Higgs field. In \autoref{tab:SM_fields}, we collect the fundamental fields of the SM and their transformation properties under different gauge groups. The fermion labels \emph{u,d,e} and $\nu$ are collective labels for three generations of fermions, shown in \autoref{tab:fermion_generations}. The fields \emph{g}, $W_\mu$, and $B_\mu$ are vector bosons that that correspond to the $SU(3)$, $SU(2)$ and $U(1)$ gauge group respectively.

\begin{table}
  \begin{sidecaption}{The fields of the Standard Model, grouped by their charges under the relevant gauge groups.}[tab:SM_fields]
  \[
  \begin{array}{lccccc}
    \toprule
   \text{Type}                & \text{Field}              & \text{Spin}      & SU(3)      & SU(2)      & U(1)_Y \\\midrule
   \text{Scalar}              & H                         & 0                & \mathbf{1} & \mathbf{2} & \slantfrac{1}{2}\\\midrule
   \text{Fermions}            & Q = \vdoublet{u_L}{d_L}   & \slantfrac{1}{2} & \mathbf{3} & \mathbf{2} & \slantfrac{1}{6}\\
                              & u_R                       &                  & \mathbf{3} & \mathbf{1} & \slantfrac{2}{3}\\
                              & d_R                       &                  & \mathbf{3} & \mathbf{1} & -\slantfrac{1}{3}\\
                              & L = \vdoublet{e_L}{\nu_L} &                  & \mathbf{1} & \mathbf{2} & -\slantfrac{1}{2}\\
                              & e_R                       &                  & \mathbf{1} & \mathbf{1} & -1\\\midrule
    \text{Vector bosons}      & g                         & 1                & \mathbf{8} & \mathbf{1} & 0 \\
                              & W_\mu                     &                  & \mathbf{1} & \mathbf{3} & 0 \\
                              & B_\mu                     &                  & \mathbf{1} & \mathbf{1} & 0\\
    \bottomrule
  \end{array}
\]
\end{sidecaption}
\end{table}

The ground state that we inhabit spontaneously breaks the larger symmetry $SU(2)\times U(1)_Y$ to the symmetry $U(1)_\text{EM}$. The subscript \emph{Y} denotes the hypercharge of the field, while the subscript EM denotes electromagnetism. In this process of electroweak symmetry breaking (which we will revisit in more detail in \autoref{subsec:ewsb}), the three gauge fields $W_\mu$, corresponding to the $SU(2)$ gauge symmetry, combine with the $B_\mu$ field corresponding to the $U(1)_Y$ symmetry to form the massive gauge bosons $W^\pm$ and $Z$, and a massless photon, $A_\mu$ (often denoted by $\gamma$ instead). The \emph{W} and \emph{Z} bosons mediate the weak interactions that responsible to phenomena such as radioactive decay, while the photon mediates electromagnetic interactions. The gauge field \emph{g} corresponding to the $SU(3)$ gauge symmetry is known as the gluon, and mediates the strong force between nucleons. 


\begin{table}
  \raggedright
\strictpagecheck
\begin{adjustwidth}{-0.5in}{0in}
  \begin{tabular}{cllll}
    \toprule
    Generation & \multicolumn{2}{c}{Quarks} & $e$: Leptons & $\nu$: Neutrinos \\ \cmidrule(r){2-3}
     & $u$: Up & $d$: Down &                                       & \\\midrule
    I           & $u:$ up                    & $d:$ down     & $e:$ electron                         & $\nu_e:$ electron neutrino\\
    II          & $c:$ charm                 & $s:$ strange  & $\mu:$ muon                           & $\nu_\mu:$ muon neutrino\\
    III         & $t:$ top                   & $b:$ bottom   & $\tau:$ tau                           & $\nu_\tau:$ tau neutrino\\
    \bottomrule
  \end{tabular}
  \caption{The fermion sector of the Standard Model, grouped by generation.}
  \label{tab:fermion_generations}
\end{adjustwidth}
\end{table}
\subsection{Electroweak symmetry breaking}\label{subsec:ewsb}
In 1957, Schwinger proposed the unification of the weak and electromagnetic interactions, citing their vectorial nature. In 1961, Glashow proposed a model for weak interactions governed by symmetry under the product group $SU(2)\times U(1)$. The trouble was, experiments had shown that the weak interactions had a short range, implying that the vector bosons that mediated them must be massive. On the other hand, gauge symmetry prohibits mass terms for the intermediate vector bosons. Glashow's theory included these mass terms, but they were put in by hand, and spoiled the renormalizability of the theory. A possible way out was to break the gauge symmetry \emph{spontaneously} rather than explicitly. Spontaneous symmetry breaking refers to the phenomenon where the ground state of a system does not respect the symmetry of the full Lagrangian. However, Goldstone's theorem states that for every spontaneously broken symmetry, there exists a set of massless spin-0 bosons corresponding to the generators of the symmetry group - since such particles have never been observed, this was an undesirable feature of the theory. 

Thus we seem to have to choose between two undesirable outcomes: a set of massless gauge bosons, or a set of massless scalars, both of which go against what we actually see in nature. Remarkably, each of these problems would turn out to be the solution for the other, through the marvelous \emph{Higgs mechanism}. This mechanism was first put forth by Philip W. Anderson in the context of condensed matter systems. In 1964, three groups -- Peter Higgs, the duo of Francois Englert and Robert Brout, and the trio of Gerald Guralnik, C. R. Hagen, and Tom Kibble -- independently and almost simultaneously proposed that this mechanism could explain how the the weak interaction mediators can acquire mass. There is likely a strong case for naming the mechanism after all of these physicists, but for the sake of expediency, we will simply refer to it as the Higgs mechanism.

The Higgs mechanism describes the spontaneous breaking of electroweak symmetry, that is, the breakdown of the product group $SU(2)\times U(1)_Y\rightarrow U(1)_{EM}$. This is achieved by adding to the theory a scalar field $SU(2)$ doublet, $H$:
$$H = \vdoublet{H^+}{H^0}$$
The terms of the Lagrangian that involve this field can be written as follows:
\begin{align}
  \label{eq:higgs_kinetic}
  \mathcal{L}_{\text{Higgs}} &= \frac{1}{2}\left|D_\mu H\right|^2-V(H)\\
  \label{eq:higgs_potential}
  V(H) &= \left(H^\dag H-\frac{v^2}{2}\right)^2
\end{align}
The gauge covariant derivative for the Standard Model takes the form
$$D_\mu = \partial_\mu + igW_\mu^a\frac{\tau^a}{2}+ig'YB_\mu.$$
where $W_\mu^a$ and $B_\mu$ are the gauge fields corresponding to the $SU(2)$ and $U(1)_Y$ symmetries respectively, \emph{g} and \emph{g'} are coupling constants, and $\tau^a$ are the Pauli matrices. The potential $V(H)$ reaches its minimum when $H^\dag H = v^2 / 2$. We pick a vacuum expectation value (VEV) for $H$ that breaks the neutral sector symmetry (corresponding to $H^0$) but not the charged symmetry (corresponding to $H^{+}$), since we wish to keep photons massless. Thus, we can pick the VEV:
$$H = \vdoublet{0}{\frac{v}{\sqrt{2}}}$$
Plugging this into the kinetic term of the Lagrangian, and isolating the terms quadratic in the gauge fields, we get
\[\frac{g^2v^2}{4}W_\mu^-W_\mu^++\frac{1}{2}\frac{v^2}{4}(g^2+g'^2)
\left(\ct W_\mu^3 - \st B_\mu\right)^2\]
where $W_\mu^\pm = (W_\mu^1\pm iW_\mu^2)/\sqrt{2}$  and $\theta_w = \tan^{-1}(g'/g)$ is the \emph{Weinberg angle}, which parameterizes the mixing between the photon $A_\mu$ and the $Z$ boson.
\begin{equation}\label{eq:z_a_mixing}
\vdoublet{Z_\mu}{A_\mu} = \fourmatrix{\ct}{-\st}{\st}{\ct}
\vdoublet{W_{\mu}^{3}}{B_{\mu}}
\end{equation}
The product $W_\mu^+W_\mu^-$ can be interpreted as a mass term for a generic charged vector boson $W_\mu^\pm$. Rewriting our mass terms for the vector boson fields with the above redefinitions, we get
\[\mathcal{L}_{\text{mass terms}}=\frac{e^2v^2}{4\sin^2\theta_w}W_\mu W^\mu+
\frac{e^2v^2}{8\sin^2\theta_w\cos^2\theta_w}Z_\mu Z^\mu\]
After accounting for symmetry factors, we get the masses of the W and Z bosons as:
\begin{align*}
  m_W &=\frac{ev}{2\st} & 
  m_Z &=\frac{ev}{\sin2\theta_w}
\end{align*}
We see that there is no mass term for $A_\mu$, which means that this gauge field remains massless. Thus we see that the \emph{W} and \emph{Z} bosons gain mass, limiting their range, while the photon remains massless, corresponding to what is experimentally observed. Remarkably, we have done this without violating gauge invariance. Colloquially, we say that the \emph{W} and \emph{Z} bosons have `eaten' the Goldstone boson degrees of freedom to obtain mass. 

Peter Higgs was the first to postulate the existence of a physical \emph{Higgs boson} that could be produced by perturbing the vacuum. To see how this plays out, let us parameterize the Higgs doublet field as 
\[H = \frac{1}{\sqrt{2}}\exp\left(i\xi^a(x)\frac{\tau^a}{v}\right)\vdoublet{0}{v + h(x)}.\]
where $\tau^a$ are the Pauli matrices, $\xi^a$ are functions of spacetime, $\langle\xi\rangle=\langle h\rangle=0$, and \emph{v} is the vacuum expectation value of the Higgs doublet, experimentally measured to be about 246 GeV. The simplest gauge choice is the unitary gauge, where $\xi(x)=0$. In this gauge, the potential $V(H)$ from \eqref{eq:higgs_potential} becomes:
\[V(H) = -\frac{1}{2}m_h^2 h^2 - \lambda vh^3 -\frac{1}{4}\lambda h^4\]
The field $h$ is the physical Higgs boson. It is the only fundamental scalar in the Standard Model, and was discovered in 2012, nearly 50 years after it was first predicted. The coefficient of the term quadratic in $h$ is its mass, $m_h$. This mass is a free parameter that had to be determined experimentally. The second and third terms in the above expression represent the trilinear and and quartic self-coupling of the Higgs boson.
\subsection{Quark mass and gauge eigenstates}
While the quarks listed in \autoref{tab:fermion_generations} are gauge eigenstates, with well defined transformations under the gauge groups, the physical states that we observe at a collider will in fact be mass eigenstates. The two are related by the \emph{Cabibbo-Kobayashi-Maskawa} (\textsc{ckm}) matrix:
\sidepar{The Cabibbo-Kobayashi-Maskawa quark mixing matrix}
\begin{equation}
  V_\text{CKM} =
  \begin{pmatrix}
    V_{ud} & V_{us} & V_{ub}\\
    V_{cd} & V_{cs} & V_{cb}\\
    V_{td} & V_{ts} & V_{tb}
  \end{pmatrix}
\end{equation}
The elements of the matrix denote the level of mixing between the gauge eigenstates. Experimentally, the matrix has been determined to be nearly diagonal.

\subsection{Asymptotic freedom and QCD}

In the 50s and 60s, experiments were devised to unravel the structure of the proton. They found two seemingly incompatible results: on the one hand, colliding protons resulted in the production of a large number of pions collinear with the beam axis, implying that the protons could not absorb a large momentum transfer. However, deep inelastic scattering experiments showed that it was possible for an energetic electron to undergo hard electrmagnetic scattering off a proton. These disparate phenomena were reconciled by the introduction of the \emph{parton model} by Bjorken and Feynman. In this model, hadrons were comprised of a collection of loosely bound pieces, known as partons. To an energetic incoming electron, these partons would appear approximately free, allowing it to scatter with a large momentum transfer off of one of them. The struck parton will then exchange momentum softly via the strong interaction among the other partons, which results in the production of a \emph{jet} of hadrons, collinear with the direction of the original struck parton. The deeper reason for this behavior is that these partons are charged under a non-Abelian gauge group, $SU(3)$. The phenomenon of the weakening of the strong interaction at large momentum transfers is known as \emph{asymptotic freedom}, and it is a property of non-Abelian gauge theories.
\begin{marginfigure}[-3in]
  \strictpagecheck
  \includegraphics[width=0.8\textwidth]{images/proton_innards}
\caption{A representation of the innards of a proton, showing the dynamic structure. Image source:\citep{proton_structure}}
\end{marginfigure}
The $\beta$-function of a gauge theory describes the evolution of the coupling constant with energy. For non-Abelian gauge theories, it takes the following form (up to leading order):%
\sidepar{Leading order $\beta$-function for non-Abelian gauge theories}
\begin{equation}
  \beta(g_R) = \mu\frac{d}{d\mu}g_R = -\frac{g_R^3}{(4\pi)^2}\left[\frac{11}{3}C_A -\frac{4}{3}n_fT_F\right]
\end{equation}
where $\mu$ is the renormalization scale, $C_A$ is the quadratic Casimir operator for the adjoint representation, and $T_F$ is the \emph{index} of the fundamental representation.
In the case of the strong interaction, the relevant gauge group is $SU(3)$. Thus $C_A = 3$ and $T_F = \frac{1}{2}$. If $n_f$, the number of quark flavors is less than 17 \footnote{Which it is, since the number of quark flavors is six, as can be seen in \autoref{tab:fermion_generations}.}, the $\beta$ function has a \emph{negative} sign, which means that the coupling \emph{decreases} at higher energies. This is the underlying basis of asymptotic freedom. The theory of the strong interaction is known as \emph{quantum chromodynamics}, or QCD for short. The analogue of electric charge for this theory is a quantity referred to as \emph{color} (hence the name of the theory). The gauge fields that mediate this interaction, that correspond to the generators of $SU(3)$, are known as \emph{gluons}. At a hadron collider, it is of crucial importance to understand the physics of gluons and jets.

The Standard Model of particle physics has been remarkably successful, and over the decades has yielded some of the most precise measurements in all of physics. However, as mentioned in \autoref{ch:introduction}, there still remain unresolved issues with the SM. In the next two sections we will discuss some of these issues in more depth, and introduce extensions to the SM that potentially resolve them.

\section{Two-Higgs Doublet Models}\label{sec:2HDMs}

Intuitively, it is not too hard to imagine, based on the complex structure of the fermion and gauge sectors, that the scalar sector of the Standard Model might well contain members other than a single $SU(2)$ doublet. An extended scalar sector can potentially alleviate some of the unresolved issues facing the Standard Model.
% Strong CP problem
One of these issues is the \emph{strong CP problem}, which can be summarized as follows. Lagrangians for Yang-Mills theories can have a renormalizable term that is gauge-invariant but but violates CP, of the form
\[\mathcal{L}_{\theta} = \theta\epsilon^{\mu\nu\alpha\beta}F_{\mu\nu}^aF_{\alpha\beta}^a\]
where $\theta$ is some angle, $F_{\mu\nu}$ is the field strength tensor, and $\epsilon^{\mu\nu\alpha\beta}$ is an antisymmetric tensor. This term is a total derivative, since it can be written as $2\theta\partial_\mu(\epsilon^{\mu\nu\alpha\beta}A_\nu^aF_{\alpha\beta}^a)$. Thus, it should not contribute to perturbative effects. However, this term can potentially contribute to non-perturbative effects. Additionally, this term can be modified by chiral rotations of the form $\psi\rightarrow \exp(i\gamma_5\theta_F)\psi$. Since physical observables should be independent of the choice of the basis, i.e. $\theta_F$, we should absorb it into $\theta$ by defining a basis-independent phase: $\bar{\theta} = \theta-\theta_F$. For $SU(2)$ and $U(1)$ gauge symmetries, this phase can be set to zero by performing appropriate chiral rotations of the fermion fields. However, no such choice exists for the term corresponding to the the $SU(3)$ group, and thus the CP-violating term in the QCD Lagrangian could in principle by non-zero. A non-zero value of $\bar{\theta}$ would be manifested as non-perturbative effects. For example, the neutron would then have a non-zero electric dipole moment. However, experiments have shown that $\bar{\theta}$ must be vanishingly small, with a stringent upper bound: $\bar{\theta}<10^{-10}$.

Adding an additional scalar doublet allows us to impose a global $U(1)$ symmetry on the Lagrangian, known as \emph{Peccei-Quinn} symmetry. If this symmetry is spontaneously broken, a Goldstone boson arises, which can then be chirally rotated such that $\bar{\theta}$ becomes effectively zero for the ground state.
% Baryon asymmetry
Another motivation for $2$HDMs is their ability to explain the observed baryon asymmetry of the universe. That is, the amount of matter in the universe is much larger than the amount of antimatter in it. The CP violation in the weak sector the Standard Model cannot account for this imbalance, but $2$HDMs, with their complex scalar sector and possible new sources of CP violation, can.
%
The strongest motivation for $2$HDMs is, however, their connection to supersymmetric models, such as the MSSM, that have the potential to resolve the hierarchy problem, discussed in the next section. The scalar sector of the MSSM has a structure similar to that of a $2$HDM - it requires an additional scalar doublet to give mass to both up and down type fermions, and for the cancellation of anomalies. With these motivations in mind, let us look at the scalar potential that results from two scalar $SU(2)$ doublets.

\subsection{The $2$HDM scalar potential}
The most general renormalizable scalar potential for two scalar doublets $\Phi_1$ and $\Phi_2$ with hypercharge $+1$ is:
\strictpagecheck
\sidepar{Scalar potential for a Two-Higgs Doublet Model}
\begin{align*}
  V(\Phi_1,\Phi_2) &= m_{11}^2|\Phi_1|^2 + m_{22}^2|\Phi_2|^2 - m_{12}^2\left(\Phi_1^\dagger\Phi_2 + \text{h.c.}\right)\\
&+\frac{\lambda_1}{2}|\Phi_1|^4 + \frac{\lambda_2}{2}|\Phi_2|^4+\lambda_3|\Phi_1|^2|\Phi_2|^2 + \lambda_4|\Phi_1^\dagger\Phi_2|^2\\
&+\left[\frac{\lambda_5}{2}\left(\Phi_1^\dagger\Phi_2 \right)^2+\lambda_6|\Phi_1|^2(\Phi_1^\dagger\Phi_2)+\lambda_7|\Phi_2|^2(\Phi_1^\dagger\Phi_2) + \text{h.c.}\right]
\label{eq:2HDM_scalar_potential}
\end{align*}
where h.c. stands for the hermitian conjugate of the terms immediately preceding it. The parameters $m_{11}^2, m_{22}^2$, and $\lambda_{1,2,3,4}$ are real while $m_{12}^2$ and $\lambda_{5,6,7}$ can be complex. Naively, it would seem that this potential has 14 degrees of freedom - six from the real parameters, and eight from the complex parameters. However, it should be noted that we have the freedom to perform basis transformations, that is, we can write the potential in terms of new doublets $\Phi_a' = \sum_{b=1}^2U_{ab}\Phi_b$. where $U_{ab}$ is a $2\times 2$ unitary matrix. The condition of unitarity implies that $U$ has three degrees of freedom, which can absorb three out of the 14 degrees of freedom listed earlier. Thus, only 11 out of the original 14 degrees of freedom are physical.

In principle, we could proceed with these 11 parameters, however, there are a couple of reasons to attempt to reduce this number. The first is that a large number of free parameters makes a theory less falsifiable, thus reducing its predictive power. The second is that in order to distinguish between pseudoscalars and scalars, CP must be conserved in the Higgs sector. Finally, the potential in \eqref{eq:2HDM_scalar_potential} allows for tree-level flavor-changing neutral currents (FCNC), which are experimentally measured to be highly suppressed. These can be eliminated by introducing discrete or continuous symmetries. Imposing a discrete symmetry such as $\mathcal{Z}_2$, that is, the Lagrangian is invariant under the reflection of one of the doublets: $\Phi_i\rightarrow-\Phi_i$, removes the terms that are odd in $\Phi_i$. This effectively sets $\lambda_6=\lambda_7 = 0$. In principle, this should set $m_{12} = 0$ as well, but we retain this term since it breaks the $\mathcal{Z}_2$ symmetry softly, which relaxes the experimental bounds on the mass spectrum.
After imposing these constraints, all the remaining parameters $\lambda_{1,2,3,4,5}, m_{11,12,22}$ are real. From here on, we will only consider $2$HDMs with these constraints. 

There are four such models, classified based on the coupling patterns of the fermions to the two Higgs doublets. In Type I $2$HDMs, all the quarks couple to only one of the Higgs doublets (chosen by convention to be $\Phi_2$. In Type II $2$HDMs, the up-type right-handed quarks (\emph{u,c,t}) couple to $\Phi_2$, and the down-type right-handed quarks (\emph{d,s,b}) couple to $\Phi_1$. In both of these models, the right-handed leptons couple to the same doublet as the down-type quarks. There are two other models that do not have tree-level FCNCs. The lepton-specific model is similar to the Type I model, except in this case, the right-handed leptons couple to $\Phi_1$. Similarly, the `flipped' model is similar to the Type II $2$HDM, except that the leptons couple to $\Phi_2$. The coupling patterns for these models are collected in \autoref{tab:no_FCNC_$2$HDMs}.
\begin{margintable}[1cm]
\small{
  \begin{tabular}{lccc}
	\toprule
    Model & $u_R^i$ & $d_R^i$  & $e_R^i$\\
    \midrule
    Type I          & $\Phi_2$ & $\Phi_2$ & $\Phi_2$\\
    Type II         & $\Phi_2$ & $\Phi_1$ & $\Phi_1$\\
    Lepton-specific & $\Phi_2$ & $\Phi_2$ & $\Phi_1$\\
    Flipped         & $\Phi_2$ & $\Phi_1$ & $\Phi_2$\\
    \bottomrule
  \end{tabular}}
  \caption{$2$HDMs with flavor conservation.}
  \label{tab:no_FCNC_$2$HDMs}
\end{margintable}

This potential is minimized for non-zero vacuum expectation values of $\Phi_i$:
\begin{align}
\langle\Phi_i\rangle_0=\vdoublet{0}{\frac{v_i}{\sqrt{2}}}
\end{align}
The complex scalar $SU(2)$ doublets $\Phi_i$ can be expressed in terms of eight real fields as follows:
\begin{equation}\label{eq:2HDM_doublet_components}
\Phi_i = \vdoublet{\phi_i^+}{\frac{1}{\sqrt{2}}(v_i+\rho_i+i\eta_i)}
\end{equation}
The process of electroweak symmetry breaking causes three of these fields to be `eaten' by the \emph{W} and \emph{Z} bosons, and the remaining five are manifested as physical scalar fields. These consist of a pair of CP-even neutral scalars \emph{h} and \emph{H}, a CP-odd pseudoscalar \emph{A}, and a charged scalar $H^\pm$.

\subsection{The $2$HDM mass spectrum}

In this section, we will analyze the mass spectrum of flavor-conserving $2$HDMs. To do so, we construct the mass matrices by taking derivatives of the scalar potential:
\[M_{ij} = \frac{\partial V(\Phi_1,\Phi_2)}{\partial\phi_i\partial\phi_j}\]
where $\phi_{i}$ can be any of the fields $\phi_i^+,\rho_i,\eta_i$ in \eqref{eq:2HDM_doublet_components}. We will also adopt the notation: 
Applying this procedure to the charged scalar components $\phi_i^\pm$, we obtain their mass matrix $M_{\phi^{\pm}}$:
\[M_{\phi^\pm} = \left[m_{12}^2-(\lambda_4+\lambda_5)v_1v_2\right]\fourmatrix{v_2/v_1}{-1}{-1}{v_1/v_2}\]
Diagonalizing this matrix gives us the mass of the charged Higgses:
\[m_{H^\pm}^2 = (v_1^2+v_2^2)[m_{12}^2/v_1v_2-(\lambda_4+\lambda_5)]\]
Similarly, the mass matrix for the pseudoscalars is given by
\[M_{\eta} = \frac{m_A^2}{v_1^2+v_2^2}\fourmatrix{v_2^2}{-v_1v_2}{-v_1v_2}{v_1^2}\]
Diagonalizing this matrix gives us a massless Goldstone boson $G^0$, corresponding to a zero eigenvalue, and a pseudoscalar Higgs boson $A$, with mass given by:
\[m_A^2 = (v_1^2+v_2^2)[m_{12}^2/v_1v_2-2\lambda_5]\]
The diagonalization process amounts to a rotation of the basis vectors by some angle.For the CP-odd and the charged scalars, this angle is the same, and is denoted as $\beta$. Let us now adopt the notation 
\[s_\theta,c_\theta,t_\theta = \sin\theta,\cos\theta,\tan\theta\]
for conciseness. In this notation, the mass eigenstates are given by
\begin{align}
\vdoublet{A}{G^0} = \fourmatrix{s_\beta}{-c_\beta}{c_\beta}{s_\beta}\vdoublet{\eta_1}{\eta_2}&&\text{and}&&
\vdoublet{H^\pm}{G^\pm} = \fourmatrix{s_\beta}{-c_\beta}{c_\beta}{s_\beta}\vdoublet{\phi_1^+}{\phi_2^+}.
\end{align}
The angle $\beta$ turns out to be a very important one for studying $2$HDMs. It also represents the ratio of the vacuum expectation values of the neutral components of the two Higgs doublets: $t_\beta = v_2/v_1$.
Finally, the mass matrix for the CP-even scalars is given by:
\[M_{\rho} = -\fourmatrix{m_{12}^2\frac{v_2}{v_1}+\lambda_1v_1^2}{-m_{12}^2+\lambda_{345}v_1v_2}{-m_{12}^2+\lambda_{345}v_1v_2}{m_{12}^2\frac{v_2}{v_1}+\lambda_1v_2^2}\]
where $\lambda_{345} = \lambda_3+\lambda_4+\lambda_5$. This matrix is diagonalized by rotation of the basis vectors by the angle $\alpha$:
\[\vdoublet{h}{H} = \fourmatrix{s_\alpha}{-c_\alpha}{-c_\alpha}{-s_\alpha}\vdoublet{\rho_1}{\rho_2}\]
with the mass eigenstates denoted $h,H$. Traditionally, $h$ is taken to be the ligher of the two particles. 

%\begin{align}
%v &= v_1^2 + v_2^2
%\end{align}
Thus we see that the physical spectrum of $2$HDMs contains five mass eigenstates: the CP-even higgses $h$ and $H$, the CP-odd pseudoscalar Higgs $A$, and a pair of charged Higgses $H^\pm$. Incidentally, the Standard Model Higgs is a combination of the CP-even scalars:
\begin{equation}
h_\text{SM} = \rho_1\cos\beta + \rho_2\sin\beta = h\sin(\alpha-\beta)-H\cos(\alpha-\beta)
\label{eq:h_SM}
\end{equation}


\subsection{Interactions}

\newcommand{\sbma}{s_{\beta-\alpha}}
\newcommand{\cbma}{c_{\beta-\alpha}}
\newcommand{\casb}{c_\alpha/s_\beta}
\newcommand{\sacb}{s_\alpha/c_\beta}
\newcommand{\sasb}{s_\alpha/s_\beta}

In this section, we discuss the interactions that are relevant to our analyses. Namely the ones governing the exotic decays of Higgses to other, lighter Higgses, and the subsequent decays of the lighter Higgses to SM fermions. The decays to SM fermions is governed by the Yukawa terms in the Lagrangian:
\sidepar{Yukawa terms for flavor-conserving 2HDMs}
\begin{align*}
&\mathcal{L}^{\mathrm{2HDM}}_{\text{Yukawa}} = - \sum_{f = u, d, l} \frac{m_f}{v}
\left(\xi_h^f \overline{f}fh+\xi_H^f \overline{f}fH-i\xi_A^f \overline{f}\gamma_5fA \right)\\
&-\left\{\frac{\sqrt{2}V_{ud}}{v}\overline{u}\left(m_u\xi_A^uP_l+m_d\xi_A^dP_R\right)dH^+ + \frac{\sqrt{2}m_l\xi^l_A}{v}\overline{\nu}_Ll_RH^+ + h.c.\right\}
\label{eq:2HDM_Yukawa_couplings}
\end{align*}
where $f$ is a fermion with mass $m_f$, the fields $u$ and $d$ are up and down type quarks with masses $m_u$ and $m_d$ and CKM mixing $V_{ud}$, $l$ is a lepton with mass $m_l$, and $\nu_L$ is a neutrino. 
The factors $\xi$ in the above expressions depend on the specific model being considered. For the Type II $2$HDM, they take on the values listed in \autoref{tab:xi_factors}.
\begin{margintable}[3in]
  \[
    \begin{array}{lr}
      \toprule
      \text{Coefficient}       & \text{Value} \\
      \midrule
      \xi_{hVV}   & \sbma          \\
      \xi_{h}^u   & \casb          \\
      \xi_{h}^d   & -\sacb         \\
      \xi_{h}^l   & -\sacb         \\
      \xi_{HVV}   & \cbma          \\
      \xi_{H}^u   & \sasb          \\
      \xi_{H}^d   & \casb          \\
      \xi_{H}^l   & \casb          \\
      \xi_{AVV}   & 0              \\
      \xi_{A}^u   & 1/t_\beta      \\
      \xi_{A}^d   & t_\beta        \\
      \xi_{A}^l   & t_\beta\\      
      \bottomrule
\end{array}
\]
\caption{List of the factors $\xi$ that determine the Yukawa couplings of Higgs bosons in Type II $2$HDMs.}
\label{tab:xi_factors}
\end{margintable}

For the exotic decays, we can obtain the coupling strengths from the kinetic terms for the fields $\Phi_i$, similar to what we do for the SM electroweak interactions. With a little bit of work (we omit the details for brevity), we can extract the following couplings: (see \citep{Kling2016a} for details):
\begin{align*}
& g_{AH^\pm W^\mp} & = && \frac{g}{2}(p_{H^+}-p_{A})^\mu\\
& g_{hAZ}          & = && is_{\beta-\alpha}\frac{g}{2c_{\theta_w}}(p_A-p_h)^\mu\\
& g_{hH^\pm W^\mp} & = && -is_{\beta-\alpha}\frac{g}{2}(p_{H^\pm}-p_h)^\mu\\
& g_{HAZ}          & = && ic_{\beta-\alpha}\frac{g}{2c_{\theta_w}}(p_A-p_H)^\mu\\
& g_{HH^\pm W^\mp} & = && -ic_{\beta-\alpha}\frac{g}{2}(p_{H^\pm}-p_H)^\mu\\
\end{align*}

\subsection{The Type II $2$HDM}

As mentioned at the beginning of this chapter, The Type II $2$HDM is of special interest since it has the same fermion-Higgs doublet coupling pattern as the MSSM. We will examine the MSSM in more detail in the next section, but will note that if we can recover the tree-level MSSM scalar potential from a Type II $2$HDM by setting the parameters $\lambda_i$ to the following values:
\begin{align}
\lambda_{1,2} = \frac{g^2+g'^2}{2} &,& \lambda_3 = \frac{g^2-g'^2}{4} &,& \lambda_4 = -\frac{g^2}{2}&,&\lambda_{5,6,7} = 0.
\end{align}
It should be noted, however, that these relations do not hold beyond the tree-level for a generic non-supersymmetrized $2$HDM.
The analyses in chapters \ref{ch:LightChargedHiggs} and \ref{ch:ExoticHiggs} are designed to probe the parameter space of a Type II $2$HDM.
\section{The Minimal Supersymmetric Standard Model}\label{sec:supersymmetry}

Historically, examining nature at increasing energy scales (and correspondingly decreasing length scales ) has consistently yielded new physics. For example, higher-energy experiments were able to probe the structure of the weak interactions, precisely at the energy scale that the 4-Fermi theory started to fail. Similarly, the challenges such as the ones listed at the beginning of \autoref{sec:2HDMs} most likely point to new physics at higher energy scales, between the currently explored weak scale and the reduced Planck scale:
\begin{equation*}
M_P = 1/\sqrt{8\pi G} \approx 2.4\times 10^{18} \text{ GeV}.
\end{equation*}
However, the SM Higgs potential is extremely sensitive to new physics at high energies. The square of the mass of the SM Higgs boson receives large quantum corrections from any new physics at high energies that couples to the Higgs sector. For example, if the Higgs couples to a heavy fermion \emph{f} through a term of the form $-\lambda_fH\bar{f}f$, the one-loop correction to the square of the Higgs mass (\autoref{fig:one_loop_fermion}) takes the form
\begin{equation}
\Delta m_H^2 = -\frac{|\lambda_f|^2}{8\pi^2}\Lambda_\text{UV}^2 + ...
\label{eq:one_loop_fermion}
\end{equation}
\begin{marginfigure}[-1in]
  \strictpagecheck
\feynmandiagram [layered layout, horizontal=b to c] { 
  a [particle=\(h\)] -- [scalar] b -- [fermion, half left, edge label=\(f\)] c -- [fermion, half left] b, c -- [scalar] d,
};
\caption{Feynman diagram for the one-loop fermionic correction to the SM Higgs mass}
\label{fig:one_loop_fermion}
\end{marginfigure}
where $\Lambda_{UV}$ is some cutoff momentum where the effects of the new physics are expected to manifest themselves. Similarly, the one-loop correction from a heavy scalar \emph{S} through the term $-\lambda_S|H|^2|S|^2$ (\autoref{fig:one_loop_scalar}) takes the form
\begin{equation}
  \Delta m_H^2 = \frac{\lambda_S}{16\pi^2}\left[\Lambda_\text{UV}^2 + ...\right]
\label{eq:one_loop_scalar}
\end{equation}
\begin{marginfigure}
\begin{tikzpicture}
\begin{feynman}
  \vertex (a){\(h\)};
	\vertex [right=of a] (b);
	\vertex [right=of b] (c);
    \vertex [above=of b] (d);
\diagram*{
	{
      [edges = scalar]
      (a) -- (b) -- (c),
      (b) -- [half left, edge label=\(S\)] (d) -- [half left] (b),
    },
};
\end{feynman}
\end{tikzpicture}
\caption{Feynman diagram for the one-loop scalar correction to the SM Higgs mass}
\label{fig:one_loop_scalar}
\end{marginfigure}
\strictpagecheck
In both these cases, the size of the correction scales quadratically with the momentum cutoff $\Lambda_{UV}$. Higher-order loop corrections can be shown to be similarly large as well. Thus the `natural' mass of the Higgs would seem to be on the the order of $\Lambda_{UV}$, which could even be as high as the Planck scale. In contrast, the actual mass that we measure is only about 126 GeV. Thus there is a \emph{hierarchy} between the observed and the `natural' mass of the SM Higgs, one of many orders of magnitude. \footnote{Note that even though only the mass of the SM Higgs is directly sensitive to $\Lambda_{UV}$, this sensitivity is propagated to all the other SM particles through their couplings to the SM Higgs.}
Thus it would seem that any UV completion of the SM would have to come with a host of parameters to tune the counterterms enough to cancel out the quadratic divergences and result in the physical mass we observe experimentally.
It is obviously undesirable to have to manually tune a large number of parameters to be able to come up with UV completions of the Standard Model - it would be analogous to the geocentric Ptolemaians adding an ever-increasing number of epicycles to explain what would ultimately be more simply and accurately described by Copernicus's heliocentric theory. 
Looking at the forms of the one-loop corrections in \eqref{eq:one_loop_fermion} and \eqref{eq:one_loop_scalar}, we can see that the contribution from the fermion \emph{f} will be exactly canceled out by the contributions from two complex scalars with $\lambda_S = |\lambda_f|^2$, the corrections from the scalar and the fermion will cancel out exactly. This suggests that the simplest way to ensure that all quadratic divergences from new physics at high energy scales cancel out is to require some kind of symmetry between fermions and scalars, ensuring that there is a scalar partner for each fermion, or vice versa.

This symmetry is known as \emph{supersymmetry}. It is a rich mathematical structure with far-reaching consequences, a lot of which are beyond the scope of this work. For a pedagogical review of supersymmetry, we refer the reader to \citep{Martin1997}. This section provides a necessarily condensed version of the treatment there. 
The term `supersymmetry' refers to the invariance of the Lagrangian under supersymmetry transformations of the form\sidefootnote{Of course, this is not the precise form of the transformation - as can be seen by performing some rudimentary dimensional analysis. We write done more precise transformations later.}
\begin{align*}
  Q|\text{Boson}\rangle = |\text{Fermion}\rangle &&\text{and}&& Q|\text{Fermion}\rangle = |\text{Boson}\rangle.
\end{align*}
The minimal phenomenologically viable supersymmetric extension to the Standard Model is known as the \emph{Minimal Supersymmetric Standard Model}, or the MSSM. Although the hierarchy problem has been the main driving force behind the development of supersymmetry, there are other benefits that the MSSM provides as well.
One of them is that the MSSM has the right particle content to unify the strong and electroweak couplings at a high energy scale, as seen in \autoref{fig:gauge_coupling_unification}.

\begin{marginfigure}[-6.5in]
    \includegraphics[width=\textwidth]{images/gauge_coupling_unification}
  \caption{2-loop RG evolution of inverse gauge couplings in the SM (dashed lines) and the MSSM (solid lines). The sparticle masses are varied between 0.5-1.5 TeV, and $\alpha_3(m_Z)$ is varied between 0.117 and 0.121. Source: \citep{Martin1997}.}
  \label{fig:gauge_coupling_unification}
\end{marginfigure}

The third major motivation (and the one most relevant to this dissertation) is the fact that the MSSM results in a viable dark matter candidate. The MSSM admits a new kind of discrete symmetry known as \emph{R-parity}. It is an analogue of baryon and lepton number conservation. The Lagrangian of the MSSM is defined to be invariant under the action of the operator $P_R$ on the fields. The eigenvalues of this operator are $(-1)^{3(B-L)+2s}$, where \emph{B, L}, and \emph{s} represent the baryon number, lepton number, and spin of the particle, respectively. The consequence of this is that the lightest supersymmetric particle (LSP) must be absolutely stable, that is, it cannot decay further into other particles, thus making it a good candidate for particle dark matter.

\subsection{The supersymmetry algebra}
In 1967, Coleman and Mandula \citep{Coleman1967} showed that, given certain assumptions, the only possible Lie group symmetries allowed for relativistic interacting field theories in four dimensions are direct products of the Poincar\'e group and an internal symmetry group (i.e. a gauge symmetry) \citep{Mandula2015}.
A Lie algebra is defined by the commutation relations of its generators. If we extend the notion of a Lie algebra to include \emph{anticommutation} relations \citep{Wess1992}, the theorem no longer applies. These kinds of algebras are known as \emph{graded Lie algebras}, or \emph{superalgebras}. The trio of Haag, Łopuszanski, and Sohnius \citep{Haag1975} applied a similar treatment as Coleman and Mandula to determine the most general superalgebra consistent with relativistic quantum field theory. The generators $Q_\alpha$ of this algebra are known as fermionic operators, and must transform as Weyl spinors. The anticommutation (and commutation) relations take the form
\sidepar{The supersymmetry algebra}
\begin{align}
  \begin{split}
  \{Q_\alpha, \bar{Q}_{\dot{\beta}}\} &= 2(\sigma^\mu)_{\alpha\dot{\beta}}P^\mu\\
  \{Q_\alpha, Q_\beta\} &= \{\bar{Q}_{\dot{\alpha}}, \bar{Q}_{\dot{\beta}}\} = 0\\
  [P^\mu,Q_\alpha] &= [P^\mu,\bar{Q}_{\dot{\alpha}}] = 0
\end{split}
\label{eq:susy_algebra}
\end{align}
where $P_\mu = i\partial_\mu$ is the generator of translations. 

The first supersymmetric Lagrangian density in four-dimensions was formulated by Wess and Zumino \citep{Wess1974}. The approach taken was to define infinitesimal `supergauge' transformations for scalars and spinors that generated a closed algebraic structure. Soon afterwards, Salam and Strathdee \citep{Salam1974} created a more systematic approach to constructing these transformations, by giving them a geometric interpretation. 

In this picture, supersymmetric transformations can be viewed as translations in a manifold known as \emph{superspace}, which is obtained by adding the `fermionic' coordinates $\theta^\alpha$, and $\bar{\theta}^{\dot{\beta}}$ to the regular spacetime coordinates $x^\mu$. 
\sidepar{Generators of superspace translations}
Thus, the generators of these translations are represented by
\begin{align}
  Q_\alpha &= \frac{\partial}{\partial\theta^\alpha}-i(\sigma^\mu \bar{\theta})_\alpha \partial_\mu&&\text{and}&&
  \bar{Q}_{\dot{\beta}} = \frac{\partial}{\partial\bar{\theta}^{\dot{\beta}}}-i(\sigma^\mu\bar{\theta})_{\dot{\beta}}\partial_\mu.
\label{eq:susy_operators_diff_form}
\end{align}
These transformations act on objects known as superfields, which are complex scalar fields parameterized by the coordinates $(x^\mu,\theta,\bar{\theta})$.
\sidepar{Infinitesimal SUSY transformations}
Under an infinitesimal supersymmetry transformation, a general superfield $S(x,\theta,\bar{\theta})$ transforms as
\begin{align}
S\rightarrow S' = (1+i\xi^\alpha Q_\alpha + i\bar{\xi}_{\dot{\alpha}}\bar{Q}^{\dot{\alpha}})S
\label{eq:gen_susy_transformation}
\end{align}
The fermionic coordinate $\theta$ has two components: $\theta^\alpha = (\theta^1,\theta^2)$. Each of these two components is a \emph{Grassmannian} variable, satisfying the anticommutation relation $\{\theta^i,\theta^i\} = 0$. This implies that $(\theta^i)^2 = 0$. This means that a Taylor expansion in powers of a fermionic coordinate must always terminate with a finite number of terms. For a general superfield \emph{S}, the Taylor expansion about the fermionic coordinates $\theta,\bar{\theta}$ takes the form
\sidepar{General superfield expansion}
\begin{align}
  \begin{split}
  S(x,\theta,\bar{\theta}) = a &+ \theta\xi + \bar{\theta}\chi^\dagger + \theta\theta b + \bar{\theta}\bar{\theta}c+\bar{\theta}\bar{\sigma}^\mu\theta v_\mu \\
  &+ \bar{\theta}\bar{\theta}\theta\eta + \theta\theta\bar{\theta}\zeta^\dagger+\theta\theta\bar{\theta}\bar{\theta}d,
\end{split}
  \label{eq:general_superfield_expansion}
\end{align}
where \emph{a,b,c} and \emph{d} are complex-valued scalar fields, $\xi,\chi,\eta$ and $\zeta$ are spinor fields, and $v_\mu$ is a vector field. These coefficients in the Taylor expansion will be identified with the regular (non-super) fields that we see in the SM and its supersymmetrized version, the MSSM. The matter content of the MSSM can be derived from expanding superfields of a particular type, known as \emph{chiral} superfields. The expansion of a generic chiral superfield $\Phi$ takes the form:
\sidepar{Chiral superfield expansion}
\begin{align}
  \begin{split}
  \Phi(y,\theta) &= \phi(x) + \sqrt{2}\theta\psi(x)+\theta\theta F(x)\\
  &+i\theta\sigma\bar{\theta}\partial_\mu\phi(x)-\frac{1}{2}\theta\sigma\bar{\theta}\theta\sigma^\nu\bar{\theta}\partial_\mu\partial_\nu\phi(x) + \sqrt{2}\theta i\theta\sigma\bar{\theta}\partial_\mu\psi(x)
\end{split}
  \label{eq:chiral_superfield_expansion}
\end{align}
Similarly the gauge bosons come from the expansion of \emph{vector} superfields $V$, obtained by imposing the condition $V = V^\dagger$. 
\sidepar{Vector superfield expansion}
The expansion of $V$ takes the form:
\begin{align}
V = \bar{\theta}\bar{\sigma}^\mu\theta A_\mu+\bar{\theta}\bar{\theta}\theta\lambda+\theta\theta\bar{\theta}\lambda^\dagger+\frac{1}{2}\theta\theta\bar{\theta}\bar{\theta}D
\label{eq:vector_superfield_expansion}
\end{align}

We see that the expansion of the chiral superfield $\Phi$ contains a scalar field $\phi$, a Weyl fermion $\psi$, and a field $F$. Thus, choosing values of $\phi,\psi$, and $F$ fixes $\Phi$. These fields, known as the \emph{component fields} of $\Phi$ can be naturally placed in a collection known as a \emph{supermultiplet}. The scalar superpartners $\phi$ are referred to as \emph{sfermions}. 
Similarly, the fields $A_\mu,\lambda,$ and \emph{D} form another natural grouping, called a \emph{vector} supermultiplet. The fields $A_\mu$ and $\lambda$ are superpartners of each other, and fields such as $\lambda$ are generically known as \emph{gauginos}\footnote{After the co-inventor of the Wess-Zumino model, Bruno Zumino.}. The particle content of the MSSM, grouped into supermultiplets, is shown in tables \ref{tab:chiral_supermultiplets} and \ref{tab:gauge_supermultiplets}. For brevity, we have not shown all three fermion generations. Also, the fields $F$ and $D$ are not included, since they are \emph{auxiliary} fields, serving only to make sure that supersymmetry holds off-shell. The gauge structure of the MSSM is the same as that of the SM - there are no new gauge interactions.

The mass terms and Yukawa terms of the MSSM Lagrangian come from the collection of terms known as the \emph{superpotential}. 
\sidepar{The MSSM superpotential}
In terms of the chiral superfields in \autoref{tab:chiral_supermultiplets}, we can write the MSSM superpotential as:
\[W_\text{MSSM} = \bar{u}\mathbf{y_u}QH_u-\bar{d}\mathbf{y_d}Q H_d-\bar{e}\mathbf{y_e}L H_d+\mu H_u H_d \]
We have abbreviated terms such as $\mu(H_u)_\alpha (H_d)_\beta\epsilon^{\alpha\beta}$ to $\mu H_u H_d$, and the matrices $\mathbf{y}$ are $3\times3$ matrices of Yukawa couplings, with indices representing the fermion generations. Note also that we now see the necessity of having two Higgs doublets - one that couples to the up-type fermions, and the other to the down-type fermions. A term like $\overline{u}\mathbf{y_u}QH_d^*$ would lead to a non-conserved hypercharge.

\subsection{Softly broken supersymmetry}
The component fields of a superfield have the same mass, as can be seen from expansions of the chiral and vector superfields in \eqref{eq:chiral_superfield_expansion} and \eqref{eq:vector_superfield_expansion}. This means that, if supersymmetry holds, the superpartners of the SM particles should have been discovered by now. Evidently, if this symmetry exists, it is not evident in the ground state that we inhabit - that is, it must be spontaneously broken. The exact mechanism by which it is broken is still unknown - there are a number of competing models which introduce new physics at high energy scales. To remain model-agnostic, we can simply perform a parameterized explicit breaking by inserting the relevant terms into the Lagrangian by hand. The form of these terms is constrained by the requirement that the quadratic divergences in the radiative corrections to scalar masses must still vanish.

\begin{table}
  \begin{tabular}{cccccc}
    \toprule
                   & \multicolumn{2}{c}{Components}                  & \multicolumn{3}{c}{Group representation} \\ \cmidrule(r){2-3}\cmidrule(l){4-6}
    Superfield     & Spin-0                                          & Spin-$\slantfrac{1}{2}$                                                        & $SU(3)_c$          & $SU(2)_L$    & $U(1)_Y$\\\midrule
                   & Squarks                                         & Quarks                                                                         &                    &              & \\ \cmidrule(r){2-3}\\
    Q              & $\vdoublet{\widetilde{u}_L}{\widetilde{d}_L}$   & $\vdoublet{u_L}{d_L}$                                                          & $\mathbf{3}$       & $\mathbf{2}$ & $\frac{1}{2}$\\\\
    $\overline{u}$ & $\tilde{u}_R^*$                                 & $u_R^\dagger$                                                                  & $\bar{\mathbf{3}}$ & $\mathbf{1}$ & -$\frac{2}{3}$\\\\
    $\overline{d}$ & $\tilde{d}_R^*$                                 & $d_R^\dagger$                                                                  & $\bar{\mathbf{3}}$ & $\mathbf{1}$ & $\frac{1}{3}$\\\\\cmidrule{2-3}
                   & Sleptons                                        & Leptons                                                                        &                    &              & \\ \cmidrule{2-3}\\
    \emph{L}       & $\vdoublet{\widetilde{\nu}_L}{\widetilde{e}_L}$ & $\vdoublet{\nu_L}{e_L}$                                                        & $\mathbf{1}$       & $\mathbf{2}$ & -$\frac{1}{2}$\\\\
    $\overline{e}$ & $\tilde{e}_R^*$                                 & $e_R^\dagger$                                                                  & $\bar{\mathbf{1}}$ & $\mathbf{1}$ & $1$\\\\\cmidrule{2-3}
                   & Higgses                                         & Higgsinos                                                                      &                    &              & \\ \cmidrule{2-3}\\
    $H_u$          & $\vdoublet{H_u^+}{H_u^0}$                       & $\vdoublet{\widetilde{H}_u^+}{\widetilde{H}_u^0}$                              & $\mathbf{1}$       & $\mathbf{2}$ & $\frac{1}{2}$\\\\
    $H_d$          & $\vdoublet{H_d^0}{H_d^-}$                       & $\vdoublet{\widetilde{H}_d^0}{\widetilde{H}_d^-}$                              & $\mathbf{1}$       & $\mathbf{2}$ & -$\frac{1}{2}$\\\\
    \bottomrule
  \end{tabular}
  \caption{Chiral supermultiplets of the MSSM.}
  \label{tab:chiral_supermultiplets}
\end{table}

\begin{table}
  \begin{tabular}{ccccc}
    \toprule
\multicolumn{2}{c}{Components} & \multicolumn{3}{c}{Group representation} \\ \cmidrule(r){1-2}\cmidrule(l){3-5}
Spin-$\slantfrac{1}{2}$        & Spin-1                                                                         & $SU(3)_c$          & $SU(2)_L$    & $U(1)_Y$\\\midrule
Gluinos: $\widetilde{g}$       & Gluons: \emph{g}                                                               & $\mathbf{8}$       & $\mathbf{1}$ & 0 \\\\
Winos: $\widetilde{W}^0$       & W bosons: $W^\pm$                                                              & $\bar{\mathbf{1}}$ & $\mathbf{3}$ & 0\\\\
Binos: $\widetilde{B}^0$       & B bosons: $B^0$                                                                & $\bar{\mathbf{1}}$ & $\mathbf{3}$ & 0\\
    \bottomrule
  \end{tabular}
  \caption{Gauge supermultiplets of the MSSM.}
  \label{tab:gauge_supermultiplets}
\end{table}

The soft supersymmetry breaking terms, written in terms of component fields, are as follows.
\sidepar{Soft supersymmetry breaking terms}
\begin{align*}
  \begin{split}
    \mathcal{L}_\text{soft}^\text{MSSM} = &-\frac{1}{2}\left(M_1\widetilde{B}\widetilde{B}+
    M_2\widetilde{W}\widetilde{W} + M_3\widetilde{g}\widetilde{g} + \text{h.c.}\right)\\
    &-\left(\widetilde{\overline{u}}\mathbf{a_u}\widetilde{Q}H_u-
    \widetilde{\overline{d}}\mathbf{a_d}\widetilde{Q}H_d-
    \widetilde{\overline{e}}\mathbf{a_e}\widetilde{L}H_d+\text{h.c.}\right)\\
    &-\widetilde{Q}^\dagger\mathbf{m_Q^2}\widetilde{Q}
     -\widetilde{L}^\dagger\mathbf{m_L^2}\widetilde{L}
     -\widetilde{\overline{u}}^\dagger\mathbf{m_{\overline{u}}^2}\widetilde{\overline{u}}^\dagger
     -\widetilde{\overline{d}}^\dagger\mathbf{m_{\overline{d}}^2}\widetilde{\overline{d}}^\dagger
     -\widetilde{\overline{e}}^\dagger\mathbf{m_{\overline{e}}^2}\widetilde{\overline{e}}^\dagger\\
     &-m^2_{H_u}H^*_uH_u-m^2_{H_d}H_d^*H_d-(bH_uH_d+\text{h.c.}) 
  \end{split}
\end{align*}

The tildes denote superpartners of the corresponding SM field - for example, $\widetilde{g}$ is a gluino, $\widetilde{\bar{e}}$ is the scalar superpartner of a right-handed electron, and so on. The parameters $M_1$, $M_2$, $M_3$ govern the masses of the gauginos $\widetilde{g}, \widetilde{W}$, and $\widetilde{B}$.

\subsection{Phenomenological approximations}
The MSSM has a number of simplifying assumptions motivated both by convenience and compatibility with observed experimental data. For example, it is convenient to take the Yukawa matrices $\mathbf{y_u,y_d,y_e}$ in a simplified limit, where only the heaviest generations have non-zero Yukawa couplings. 
\begin{align*}
  \mathbf{y_u} \approx \begin{pmatrix}
    0 & 0 & 0\\
    0 & 0 & 0\\
    0 & 0 & y_t
  \end{pmatrix},&&
  \mathbf{y_d} \approx \begin{pmatrix}
    0 & 0 & 0\\
    0 & 0 & 0\\
    0 & 0 & y_b
  \end{pmatrix},&&
  \mathbf{y_e} \approx \begin{pmatrix}
    0 & 0 & 0\\
    0 & 0 & 0\\
    0 & 0 & y_\tau
  \end{pmatrix}
\end{align*}
Furthermore, to suppress excessive CP-violation and FCNCs in the MSSM, the following approximations are often made. First, we assume that there is minimal mixing among the sfermions:
\begin{align*}
  \mathbf{m_Q^2} = m_Q^2\mathbf{1},&&
  \mathbf{m_{\bar{u}}^2} = m_{\bar{u}}^2\mathbf{1},&&
  \mathbf{m_{\bar{d}}^2} = m_{\bar{d}}^2\mathbf{1},&&
  \mathbf{m_{\bar{e}}^2} = m_{\bar{e}}^2\mathbf{1},&&
  \mathbf{m_{\bar{L}}^2} = m_{\bar{L}}^2\mathbf{1}\\
\end{align*}
We also suppress the cubic scalar couplings of the first two families of sfermions by imposing the conditions: 
\begin{align*}
  \mathbf{a_u} = A_{u0}\mathbf{y_u},&&
  \mathbf{a_d} = A_{d0}\mathbf{y_d},&&
  \mathbf{a_e} = A_{e0}\mathbf{y_e}.\\
\end{align*}
Finally, requiring that the soft SUSY breaking parameters are real enables us to suppress excessive amounts of CP violation:
\begin{align*}
  \text{Im}(M_1)=
  \text{Im}(M_2)=
  \text{Im}(M_3)=
  \text{Im}(A_{u0})=
  \text{Im}(A_{d0})=
  \text{Im}(A_{e0})=0.
\end{align*}
With the above ingredients, one can work out myriad phenomenological consequences. In \autoref{ch:DM_100_TeV}, we examine the neutralino sector of the MSSM in slightly more detail, and present an analysis aimed at exploring it at a 100 TeV collider.
