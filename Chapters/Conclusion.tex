\chapter{Conclusion}\label{ch:Conclusion}
The discovery of the Higgs boson in 2012 completed the framework of the Standard Model of particle physics, which has been verified to unprecedent levels of precision. However, the journey is not yet over - the Standard Model either provides unsatisfactory answers to issues such as the hierarchy problem, or is silent on other quesions such as the nature of dark matter. Clearly, it is not the final word on the subject of particle physics. To address the issues that the SM does not, we must go beyond it, constructing new theories that provide satisfactory explanations of physics at high energy scales. 

Two-Higgs Doublet Models ($2$HDMs) are physically well-motivated extensions to the scalar sector of the SM. They predict a collection of Higgs bosons, (as opposed to the single Higgs boson of the SM): two CP-even states \emph{h} and \emph{H}, a CP-odd state \emph{A}, and a charged state $H^\pm$.
Traditional searches for these Higgses at particle colliders focus on them via their decays to SM final states. However, there are compelling scenarios in which these heavy scalars decay predominantly through exotic modes to non-SM final states. In certain regions of parameter space, these exotic modes can dominate the conventional decay modes to SM final states, and thus provide a complementary avenue for discovery. 

In \autoref{ch:LightChargedHiggs}, we presented an analysis aimed at discovering light charged Higgses produced via top quark decay at the 14 TeV LHC. We found that the exotic decay channel $H^\pm\rightarrow AW^\pm$ is able to probe the low $t_\beta$ region better than the conventional decay channels.

In \autoref{ch:ExoticHiggs}, we presented preliminary results from a project aimed at systematically covering a number of exotic decay scenarios through collider analyses for both the 14 TeV LHC and a future 100 TeV collider. We find that the preliminary results are promising, and cover a large swathe of $2$HDM parameter space, up to scalar masses of 3.5 TeV, for a wide range of values of $\tan\beta$, at a 100 TeV collider.

In addition, we also present an analysis aimed at discovering pair produced higgsino-like neutralinos that decay to bino-like neutralino LSPs at a 100 TeV collider. Both of these particles arise from the Minimal Supersymmetric Standard Model, described in \autoref{sec:supersymmetry}. This heavily-studied model can be viewed a special case of a $2$HDM that incorporates supersymmetry, a symmetry that connects fermions and bosons. This analysis is performed using the razor variables and boosted decision trees. We find that we are able to exclude Higgsinos up to 1.8 TeV for binos up to 1.3 TeV with this analysis.

In some ways, the three analyses presented in this dissertation can be viewed as a progression on two fronts. On the one hand, they involve increasingly high collider energies, going from 14 TeV to 100 TeV. On the other, they involve an increasing use of machine learning techniques. The first analysis solely uses rectangular selection cuts, the second qualitatively compares rectangular cut and machine learning strategies, and the third uses machine learning exclusively. The author believes that it is likely that this trend will be reflected in experimental particle physics research in the future, as we attempt to grapple with increasingly difficult questions at the frontier of the field. Increasing collider energies will be required to probe physically well-motivated scenarios with a TeV scale mass spectrum, while machine learning techniques will be increasingly needed to tackle the extremely large backgrounds and small signal rates at future colliders.

\paragraph{Open Science} The author strongly believes in the value of reproducible research and open science. In this spirit, the source code for the dissertation document has been made freely available online at

\url{https://github.com/adarshp/dissertation}.

\noindent The code for the analysis in \autoref{ch:DM_100_TeV} can be found at

\url{https://github.com/adarshp/Dark-Matter-at-100-TeV}

\noindent while the code for the analysis of the $H^\pm\rightarrow HW^\pm$ decay channel in \autoref{ch:ExoticHiggs} can be found at

\url{https://github.com/adarshp/ExoticHiggs}.

\noindent In addition, the code for both of these analyses utilizes a Python-based software framework, \texttt{clusterpheno}, written by the author during the course of his graduate work, for the purpose of performing large-scale event generation and analysis on the University of Arizona cluster. The source code for this framework can be found online at 

\url{https://github.com/adarshp/clusterpheno}.
