\chapter{Conclusion}\label{ch:Conclusion}
In some ways, the three analyses presented in this dissertation can be viewed as a progression on two fronts. On the one hand, they involve increasingly high collider energies, going from 14 TeV to 100 TeV. On the other, they involve an increasing use of machine learning techniques. The first analysis solely uses rectangular selection cuts, the second qualitatively compares rectangular cut and machine learning strategies, and the third uses machine learning exclusively. The author believes that it is likely that this trend will be reflected in experimental particle physics research in the future, as we attempt to grapple with increasingly difficult questions at the frontier of the field.

\paragraph{Open Science} The author strongly believes in the value of reproducible research and open science. In this spirit, the source code for the dissertation document has been made freely available online at

\url{https://github.com/adarshp/dissertation}.

\noindent The code for the analysis in \autoref{ch:DM_100_TeV} can be found at

\url{https://github.com/adarshp/Dark-Matter-at-100-TeV}

\noindent while the code for the analysis of the $H^\pm\rightarrow HW^\pm$ decay channel in \autoref{ch:ExoticHiggs} can be found at

\url{https://github.com/adarshp/ExoticHiggs}.

\noindent In addition, the code for both of these analyses utilizes a Python-based software framework, \texttt{clusterpheno}, written by the author during the course of his graduate work, for the purpose of performing large-scale event generation and analysis on the University of Arizona cluster. The source code for this framework can be found online at 

\url{https://github.com/adarshp/clusterpheno}.
