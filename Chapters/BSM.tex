\chapter{Beyond the Standard model}\label{ch:bsm}

In the previous chapter, we covered the Standard Model of particle physics and came up against its limits. In this chapter, we will cover some possible extensions to the Standard Model that can help us overcome those limits.

\section{Two-Higgs Doublet Models}

Intuitively, it is not too hard to imagine, based on the complex structure of the fermion and gauge sectors, that the scalar sector of the Standard Model might well contain members other than a single $SU(2)$ doublet. The possible extensions to the scalar sector are constrained by the value of the parameter $\rho$. For a $SU(2)\times U(1)$ gauge theory, this takes the form (at tree level):

\begin{equation}
\rho = \frac{\sum_\text{i=1}^n\left[I_i(I_i+1)-\frac{1}{4}Y_i^2Y_i^2\right]v_i}{\sum_\text{i=1}^n\frac{1}{2}Y_i^2v_i}
\end{equation}
%
where $n$ is the number of scalar multiplets with isospin\footnote{Short for \emph{isotopic spin}} $I_i$, hypercharge $Y_i$, and vacuum expectation values of the neutral components $v_i$. This parameter has been experimentally measured to be almost exactly one. The simplest extensions to the scalar sector that satisfy this constraint involve adding additional scalar singlets and doublets. The ones that add a single additional $SU(2)$ scalar doublet field are collectively known as \emph{Two-Higgs Doublet Models}, or 2HDMs. For a detailed review of 2HDMs, see \citep{Branco2012}, whose treatment we follow closely here.

Adding an additional scalar doublet alleviates some of the problems with the Standard Model that we mentioned at the end of the previous chapter. 
% Strong CP problem
Lagrangians for Yang-Mills theories can have a renormalizable term that is gauge-invariant but but violates CP, of the form
\[\mathcal{L}_{\theta} = \theta\epsilon^{\mu\nu\alpha\beta}F_{\mu\nu}^aF_{\alpha\beta}^a\]
where $\theta$ is some angle, $F_{\mu\nu}$ is the field strength tensor, and $\epsilon^{\mu\nu\alpha\beta}$ is an antisymmetric tensor\footnote{Levi-Civita?}. This term is a total derivative, since it can be written as $2\theta\partial_\mu(\epsilon^{\mu\nu\alpha\beta}A_\nu^aF_{\alpha\beta}^a)$. Thus, it should not contribute to perturbative effects. However, this term can potentially contribute to non-perturbative effects. Additionally, this term can be modified by chiral rotations of the form $\psi\rightarrow \exp(i\gamma_5\theta_F)\psi$. Since physical observables should be independent of the choice of the basis, i.e. $\theta_F$, we should absorb it into $\theta$ by defining a basis-independent phase: $\bar{\theta} = \theta-\theta_F$. For $SU(2)$ and $U(1)$ gauge symmetries, this phase can be set to zero by performing appropriate chiral rotations of the fermion fields. However, no such choice exists for the term corresponding to the the $SU(3)$ group, and thus the CP-violating term in the QCD Lagrangian could in principle by non-zero. A non-zero value of $\bar{\theta}$ would be manifested as non-perturbative effects. For example, the neutron would then have a non-zero electric dipole moment. However, experiments have shown that $\bar{\theta}$ must be vanishingly small, with a stringent upper bound: $\bar{\theta}<10^{-10}$.

Adding an additional scalar doublet allows us to impose a global $U(1)$ symmetry on the Lagrangian, known as \emph{Peccei-Quinn} symmetry. If this symmetry is spontaneously broken, a Goldstone boson arises, which can then be chirally rotated such that $\bar{\theta}$ becomes effectively zero for the ground state.
% Baryon asymmetry
Another motivation for 2HDMs is their ability to explain the observed baryon asymmetry of the universe. That is, the amount of matter in the universe is much larger than the amount of antimatter in it. The CP violation in the weak sector the Standard Model cannot account for this imbalance, but 2HDMs, with their complex scalar sector and possible new sources of CP violation, can.
%
The strongest motivation for 2HDMs is, however, the potential to resolve the hierarchy problem. More specifically, supersymmetric theories provide a way to resolve the hierarchy problem, as discussed in \autoref{sec:supersymmetry}. These theories require an additional scalar doublet to give mass to both up- and down-type fermions, and for the cancellation of anomalies.

\subsection{The 2HDM Lagrangian}
The most general renormalizable scalar potential for two doublets $\Phi_1$ and $\Phi_2$ with hypercharge $+1$ is:

\begin{align*}
  V(\Phi_1,\Phi_2) &= m_{11}^2|\Phi_1|^2 + m_{22}^2|\Phi_2|^2 - m_{12}^2\left(\Phi_1^\dagger\Phi_2 + h.c.\right)\\
&+\frac{\lambda_1}{2}|\Phi_1|^4 + \frac{\lambda_2}{2}|\Phi_2|^4+\lambda_3|\Phi_1|^2|\Phi_2|^2 + \lambda_4|\Phi_1^\dagger\Phi_2|^2\\
&+\frac{\lambda_5}{2}\left[\left(\Phi_1^\dagger\Phi_2 \right)^2+\lambda_6|\Phi_1|^2(\Phi_1^\dagger\Phi_2)+\lambda_7|\Phi_2|^2(\Phi_1^\dagger\Phi_2) + h.c.\right]
\label{eq:2HDM_scalar_potential}
\end{align*}
where $h.c.$ stands for the hermitian conjugate. The parameters $m_{11}, m_{22}$, and $\lambda_{1,2,3,4}$ are real while $m_{12}$ and $\lambda_{5,6,7}$ can be complex. Naively, it would seem that this potential has 14 degrees of freedom - six from the real parameters, and eight from the complex parameters. However, it should be noted that we have the freedom to perform basis transformations, that is, we can write the potential in terms of new doublets $\Phi_a' = \sum_{b=1}^2U_{ab}\Phi_b$. where $U_{ab}$ is a $2\times 2$ unitary matrix. The condition of unitarity implies that $U$ has three degrees of freedom, which can absorb three out of the 14 degrees of freedom listed earlier. Thus, only 11 out of the original 14 degrees of freedom are physical.

In principle, we could proceed with these 11 parameters, however, there are a couple of reasons to attempt to reduce this number. The first is that a large number of free parameters makes a theory less falsifiable, thus reducing its predictive power. The second is that in order to distinguish between pseudoscalars and scalars, CP must be conserved in the Higgs sector. Finally, the potential in \autoref{eq:2HDM_scalar_potential} allows for tree-level flavor-changing neutral currents (FCNC), which are experimentally measured to be highly suppressed. These can be eliminated by introducing discrete or continuous symmetries. Imposing a discrete symmetry such as $\mathcal{Z}_2$, that is, the Lagrangian is invariant under the reflection of one of the doublets: $\Phi_i\rightarrow-\Phi_i$, removes the terms that are odd in $\Phi_i$. This effectively sets $\lambda_6=\lambda_7 = 0$. In principle, this should set $m_{12} = 0$ as well, but we retain this term since it breaks the $\mathcal{Z}_2$ symmetry softly, which relaxes the experimental bounds on the mass spectrum.
After imposing these constraints, all the remaining parameters $\lambda_{1,2,3,4,5}, m_{11,12,22}$ are real. From here on, we will only consider 2HDMs with these constraints. 

There are four such models, classified based on the coupling patterns of the fermions to the two Higgs doublets. In type-I 2HDMs, all the quarks couple to only one of the Higgs doublets (chosen by convention to be $\Phi_2$. In type-II 2HDMs, the up-type right-handed quarks (\emph{u,c,t}) couple to $\Phi_2$, and the down-type right-handed quarks (\emph{d,s,b}) couple to $\Phi_1$. In both of these models, the right-handed leptons couple to the same doublet as the down-type quarks. There are two other models that do not have tree-level FCNCs. The lepton-specific model is similar to the type-I model, except in this case, the right-handed leptons couple to $\Phi_1$. Similarly, the `flipped' model is similar to the type-II 2HDM, except that the leptons couple to $\Phi_2$. The coupling patterns for these models are collected in \autoref{tab:no_FCNC_2HDMs}.

\begin{margintable}
\small{
  \begin{tabular}{lccc}
	\toprule
    Model & $u_R^i$ & $d_R^i$  & $e_R^i$\\
    \midrule
    Type I          & $\Phi_2$ & $\Phi_2$ & $\Phi_2$\\
    Type II         & $\Phi_2$ & $\Phi_1$ & $\Phi_1$\\
    Lepton-specific & $\Phi_2$ & $\Phi_2$ & $\Phi_1$\\
    Flipped         & $\Phi_2$ & $\Phi_1$ & $\Phi_2$\\
    \bottomrule
  \end{tabular}}
  \caption{2HDMs with flavor conservation.}
  \label{tab:no_FCNC_2HDMs}
\end{margintable}

This potential is minimized for non-zero vacuum expectation values of $\Phi_i$:
\begin{align}
\langle\Phi_i\rangle_0=\vdoublet{0}{\frac{v_i}{\sqrt{2}}}
\end{align}
The complex scalar $SU(2)$ doublets $\Phi_i$ can be expressed in terms of eight real fields as follows:
\begin{equation}\label{eq:2HDM_doublet_components}
\Phi_i = \vdoublet{\phi_i^+}{\frac{1}{\sqrt{2}}(v_i+\rho_i+i\eta_i)}
\end{equation}
The process of electroweak symmetry breaking causes three of these fields to be `eaten' by the \emph{W} and \emph{Z} bosons, and the remaining five are manifested as physical scalar fields. These consist of a pair of CP-even neutral scalars \emph{h} and \emph{H}, a CP-odd pseudoscalar \emph{A}, and a charged scalar $H^\pm$.

\subsection{The 2HDM mass spectrum}

In this section, we will analyze the mass spectrum of flavor-conserving 2HDMs. To do so, we construct the mass matrices by taking derivatives of the scalar potential:
\[M_{ij} = \frac{\partial V(\Phi_1,\Phi_2)}{\partial\phi_i\partial\phi_j}\]
where $\phi_{i}$ can be any of the fields $\phi_i^+,\rho_i,\eta_i$ in \autoref{eq:2HDM_doublet_components}. We will also adopt the notation: 
Applying this procedure to the charged scalar components $\phi_i^\pm$, we obtain their mass matrix $M_{\phi^{\pm}}$:
\[M_{\phi^\pm} = \left[m_{12}^2-(\lambda_4+\lambda_5)v_1v_2\right]\fourmatrix{v_2/v_1}{-1}{-1}{v_1/v_2}\]
Diagonalizing this matrix gives us the mass of the charged Higgses:
\[m_{H^\pm}^2 = (v_1^2+v_2^2)[m_{12}^2/v_1v_2-(\lambda_4+\lambda_5)]\]
Similarly, the mass matrix for the pseudoscalars is given by
\[M_{\eta} = \frac{m_A^2}{v_1^2+v_2^2}\fourmatrix{v_2^2}{-v_1v_2}{-v_1v_2}{v_1^2}\]
Diagonalizing this matrix gives us a massless Goldstone boson $G^0$, corresponding to a zero eigenvalue, and a pseudoscalar Higgs boson $A$, with mass given by:
\[m_A^2 = (v_1^2+v_2^2)[m_{12}^2/v_1v_2-2\lambda_5]\]
The diagonalization process amounts to a rotation of the basis vectors by some angle.For the CP-odd and the charged scalars, this angle is the same, and is denoted as $\beta$. Let us now adopt the notation 
\[s_\theta,c_\theta,t_\theta = \sin\theta,\cos\theta,\tan\theta\]
for conciseness. In this notation, the mass eigenstates are given by
\begin{align}
\vdoublet{A}{G^0} = \fourmatrix{s_\beta}{-c_\beta}{c_\beta}{s_\beta}\vdoublet{\eta_1}{\eta_2}&&\text{and}&&
\vdoublet{H^\pm}{G^\pm} = \fourmatrix{s_\beta}{-c_\beta}{c_\beta}{s_\beta}\vdoublet{\phi_1^+}{\phi_2^+}.
\end{align}
The angle $\beta$ turns out to be a very important one for studying 2HDMs. It also represents the ratio of the vacuum expectation values of the neutral components of the two Higgs doublets: $t_\beta = v_2/v_1$.
Finally, the mass matrix for the CP-even scalars is given by:
\[M_{\rho} = -\fourmatrix{m_{12}^2\frac{v_2}{v_1}+\lambda_1v_1^2}{-m_{12}^2+\lambda_{345}v_1v_2}{-m_{12}^2+\lambda_{345}v_1v_2}{m_{12}^2\frac{v_2}{v_1}+\lambda_1v_2^2}\]
where $\lambda_{345} = \lambda_3+\lambda_4+\lambda_5$. This matrix is diagonalized by rotation of the basis vectors by the angle $\alpha$:
\[\vdoublet{h}{H} = \fourmatrix{s_\alpha}{-c_\alpha}{-c_\alpha}{-s_\alpha}\vdoublet{\rho_1}{\rho_2}\]
with the mass eigenstates denoted $h,H$. Traditionally, $h$ is taken to be the ligher of the two particles. 

%\begin{align}
%v &= v_1^2 + v_2^2
%\end{align}
Thus we see that the physical spectrum of 2HDMs contains five mass eigenstates: the CP-even higgses $h$ and $H$, the CP-odd pseudoscalar Higgs $A$, and a pair of charged Higgses $H^\pm$. Incidentally, the standard model Higgs is a combination of the CP-even scalars:
\begin{equation}
h_\text{SM} = \rho_1\cos\beta + \rho_2\sin\beta = h\sin(\alpha-\beta)-H\cos(\alpha-\beta)
\label{eq:h_SM}
\end{equation}


\subsection{Interactions}

\newcommand{\sbma}{s_{\beta-\alpha}}
\newcommand{\cbma}{c_{\beta-\alpha}}
\newcommand{\casb}{c_\alpha/s_\beta}
\newcommand{\sacb}{s_\alpha/c_\beta}
\newcommand{\sasb}{s_\alpha/s_\beta}
\begin{table}
  \[
    \begin{array}{lrrrr}
      \toprule
         & \text{Type I} & \text{Type II} & \text{Lepton-specific} & \text{Flipped}\\
         \midrule
      \xi_{hVV} & \sbma & \sbma & \sbma & \sbma \\
      \xi_{h}^u & \casb & \casb & \casb & \casb \\
      \xi_{h}^d & \casb & -\sacb & \casb & -\sasb \\
      \xi_{h}^l & \casb & -\sacb & -\sasb & -\casb \\
      \xi_{HVV} & \cbma & \cbma  & \cbma & \cbma \\
      \xi_{H}^u & \sasb & \sasb & \sasb & \sasb \\
      \xi_{H}^d & \sasb & \casb & \sasb & \casb \\
      \xi_{H}^l & \sasb & \casb & \casb & \sasb \\
      \xi_{AVV} & 0     & 0     & 0     & 0 \\
      \xi_{A}^u & 1/t_\beta & 1/t_\beta & 1/t_\beta & 1/t_\beta \\
      \xi_{A}^d & -1/t_\beta & t_\beta & -1/t_\beta & t_\beta \\
      \xi_{A}^l & -1/t_\beta & t_\beta & t_\beta & -1/t_\beta \\
      \bottomrule
\end{array}\]
\caption{List of the factors $\xi$ that determine the Yukawa couplings of the 2HDM Higgs bosons.}
\label{tab:xi_factors}
\end{table}

We will first consider the Yukawa interactions of the mass eigenstates with fermions, which are given by:
\begin{align*}
\mathcal{L}^{\mathrm{2HDM}}_{\text{Yukawa}} =& - \sum_{f = u, d, l} \frac{m_f}{v}
\left(\xi_h^f \overline{f}fh+\xi_H^f \overline{f}fH-i\xi_A^f \overline{f}\gamma_5fA \right)\\
&-\left\{\frac{\sqrt{2}V_{ud}}{v}\overline{u}\left(m_u\xi_A^uP_l+m_d\xi_A^dP_R\right)dH^+ + \frac{\sqrt{2}m_l\xi^l_A}{v}\overline{\nu}_Ll_RH^+ + h.c.\right\}
\end{align*}
where $f$ is a fermion with mass $m_f$, $u,d$ refer to up- and down-type quarks with masses $m_u,m_d$ and CKM mixing $V_{ud}$, $l$ is a lepton with mass $m_l$, and $\nu_L$ is a neutrino. 
As for the Yukawa couplings to the vector bosons, we can recover them by recalling that the standard model Higgs is given by
\begin{equation}
h_\text{SM} = hs_{\alpha-\beta}-Hc_{\alpha-\beta}.
\end{equation}
Thus the couplings of \emph{h} and \emph{H} become rescaled versions of standard model Higgs couplings to the vector bosons. If we represent these couplings by $g_{h_\text{SM}VV}$, where $VV$ can be $ZZ$ or $W^+W^-$, then the Yukawa couplings of the 2HDM Higgses are given by
\begin{align}
g_{\phi VV} = \xi_{\phi VV}g_{h_\text{SM}VV},
\end{align}
where $\phi = h/H/A$. The factors $\xi$ in the above expressions depend on the specific model being considered, and are listed in \autoref{tab:xi_factors}.

With a little bit of work (we omit the details for brevity), we can extract the following couplings (see \citep{Kling2016a} for details):
\begin{align*}
g_{\gamma H^+H^-} &= -ie(p_{H^+}-p_{H^-})^\mu\\
g_{ZH^+H^-} &= -i\frac{gc_{2\theta_w}}{2c_{\theta_w}}(p_{H^+}-p_{H^-})^\mu\\
g_{AH^\pm W^\mp} &= \frac{g}{2}(p_{H^+}-p_{A})^\mu\\
g_{hAZ} &= is_{\beta-\alpha}\frac{g}{2c_{\theta_w}}(p_A-p_h)^\mu\\
g_{hH^\pm W^\mp} &= -is_{\beta-\alpha}\frac{g}{2}(p_{H^\pm}-p_h)^\mu\\
g_{HAZ} &= ic_{\beta-\alpha}\frac{g}{2c_{\theta_w}}(p_A-p_H)^\mu\\
g_{HH^\pm W^\mp} &= -ic_{\beta-\alpha}\frac{g}{2}(p_{H^\pm}-p_H)^\mu\\
\end{align*}

The cubic and quartic Higgs couplings can also be worked out explicitly, but we will decline to do so and instead refer the reader to the general expressions in Appendix C of  \citep{Branco2012} and the specific expressions in section 2.3.3 of \citep{Kling2016a}. 

\subsection{The Type-II 2HDM}

As mentioned at the beginning of this chapter, The type-II 2HDM is of special interest since it has the same fermion-Higgs doublet coupling pattern as the MSSM. In fact, the MSSM can be viewed as a special case of a type-II 2HDM, one that incorporates supersymmetry. We will examine the MSSM in more detail in the next section, but will note that if we can recover the tree-level MSSM scalar potential from a type-II 2HDM by setting the parameters $\lambda_i$ to the following values:
\begin{align}
\lambda_{1,2} = \frac{g^2+g'^2}{2} &,& \lambda_3 = \frac{g^2-g'^2}{4} &,& \lambda_4 = -\frac{g^2}{2}&,&\lambda_{5,6,7} = 0.
\end{align}
It should be noted, however, that these relations do not hold beyond the tree-level for a generic non-supersymmetrized 2HDM.

\section{Supersymmetry and the MSSM}\label{sec:supersymmetry}

\subsection{The hierarchy problem}
Historically, examining nature at increasing energy scales (and correspondingly decreasing length scales ) has consistently yielded new physics. For example, higher-energy experiments were able to probe the structure of the weak interactions, precisely at the energy scale that the 4-Fermi theory started to fail. Similarly, the challenges listed at the end of \autoref{ch:sm} most likely point to new physics at higher energy scales, between the currently explored weak scale to the reduced Planck scale:
\begin{equation*}
M_P = 1/\sqrt{8\pi G} \approx 2.4\times 10^{18} \text{ GeV}
\end{equation*}
However, the SM Higgs potential is extremely sensitive to new physics at high energies. The mass of the SM Higgs boson receives large quantum corrections from any new physics at high energies that couples to the Higgs sector. For example, if the Higgs couples to a heavy fermion \emph{f} through a term of the form $-\lambda_fH\bar{f}f$, the one-loop correction to the higgs mass takes the form
\begin{equation}
\Delta m_H^2 = -\frac{|\lambda_f|^2}{8\pi^2}\Lambda_\text{UV}^2 + ...
\label{eq:one_loop_fermion}
\end{equation}
where $\Lambda_{UV}$ is some cutoff momentum where the effects of the new physics are expected to manifest themselves. Similarly, the one-loop correction from a heavy scalar \emph{S} through the term $-\lambda_S|H|^2|S|^2$ takes the form
\begin{equation}
  \Delta m_H^2 = \frac{\lambda_S}{16\pi^2}\left[\Lambda_\text{UV}^2 + ...\right]
\label{eq:one_loop_scalar}
\end{equation}
In both these cases, the size of the correction scales quadratically with the momentum cutoff $\Lambda_{UV}$. Higher-order loop corrections can be shown to be similarly large as well. Thus the `natural' mass of the Higgs would seem to be on the the order of $\Lambda_{UV}$, which could even be as high as the Planck scale. In contrast, the actual mass that we measure is only about 126 GeV. Thus there is a \emph{hierarchy} between the observed and the `natural' mass of the SM Higgs, one of many orders of magnitude. \footnote{Note that even though only the mass of the SM Higgs is directly sensitive to $\Lambda_{UV}$, this sensitivity is propagated to all the other SM particles through their couplings to the SM Higgs.}
Thus it would seem that any UV completion of the SM would have to come with a host of parameters to tune the counterterms enough to cancel out the quadratic divergences and result in the physical mass we observe experimentally. If the new physics is at the Planck scale, the divergences must cancel to less than one part in $10^{-16}$ to give us a SM Higgs at the weak scale.
It is obviously undesirable to have to manually tune a large number of parameters to be able to come up with UV completions of the Standard Model - it would be analogous to the geocentric Ptolemaians adding an ever-increasing number of epicycles to explain what would ultimately be more simply and accurately described by Copernicus's heliocentric theory. 
Looking at the forms of the one-loop corrections in \autoref{eq:one_loop_fermion} and \autoref{eq:one_loop_scalar}, we can see that if $\lambda_f = 2\lambda_S$, the corrections from the scalar and the fermion will cancel out exactly. This suggests that the simplest way to ensure that all quadratic divergences from new physics at high energy scales cancel out is to require some kind of symmetry between fermions and scalars, ensuring that there is a scalar partner for each fermion, or vice versa.

This symmetry is known as \emph{supersymmetry}. It is a rich mathematical structure with far-reaching consequences, a lot of which are beyond the scope of this work. For a pedagogical review of supersymmetry, we refer the reader to \citep{Martin1997}. This section provides a necessarily condensed version of the treatment there.

\subsection{Stable dark matter candidate}
The MSSM has a new kind of symmetry known as \emph{R-parity}. It is an analogue of baryon and lepton number conservation. The Lagrangian of the MSSM is defined to be invariant under the action of the operator $P_R$ on the fields. The eigenvalues of this operator are $(-1)^{3(B-L)+2s}$, where \emph{B, L}, and \emph{s} represent the baryon number, lepton number, and spin of the particle, respectively. The  

The consequence of this is that the lightest supersymmetric particle (LSP) must be absolutely stable, that is, it cannot decay further into other particles.
\subsection{Gauge coupling unification}
The MSSM leads to the unification of the strong and electroweak forces at high energies, similar to how the electromagnetic and weak forces unify at high energies.

\begin{marginfigure}
\feynmandiagram [layered layout, horizontal=b to c] { 
  a [particle=\(h\)] -- [scalar] b -- [fermion, half left, edge label=\(f\)] c -- [fermion, half left] b, c -- [scalar] d,
};
\begin{tikzpicture}
\begin{feynman}
  \vertex (a){\(h\)};
	\vertex [right=of a] (b);
	\vertex [right=of b] (c);
    \vertex [above=of b] (d);
\diagram*{
	{
      [edges = scalar]
      (a) -- (b) -- (c),
      (b) -- [half left, edge label=\(S\)] (d) -- [half left] (b),
    },
};
\end{feynman}
\end{tikzpicture}
\caption{One-loop corrections from fermions and scalars to the Higgs mass.}
\end{marginfigure}

\subsection{Supersymmetry basics}
\subsection{Chiral and gauge supermultiplets}
All the particles in the MSSM are grouped into structures known as supermultiplets. They come in two varieties: \emph{chiral} and \emph{gauge} supermultiplets. The SM fermions and the components of the SM Higgs doublet reside inchiral supermultiplets, and the SM gauge bosons reside in gauge supermultiplets. Let us first look at what the gauge interactions of a generic chiral and gauge supermultiplet look like.
For a generic chiral supermultiplet consisting of a complex scalar $\phi$ and its superpartner, a two-component Weyl fermion $\psi$, the kinetic terms are 
\[\mathcal{L}_{\text{chiral,kinetic,fermion}} = -\partial^\mu\phi^*\partial_\mu\phi + i\psi^{\dagger}\overline{\sigma}^\mu\partial_\mu\psi.\]
If the Lagrangian density is invariant under gauge transformations of the chiral supermultiplet under a gauge group with generators $(T^a)_i^j$ and associated gauge fields $A_\mu^a$, the partial derivatives above can be promoted to covariant derivatives:
\begin{align}
  \nabla_\mu\phi_i &= \partial_\mu\phi_i - igA_\mu^a(T^a\phi)_i\label{eq:phi1}\\
  \nabla_\mu\phi^{*i} &= \partial_\mu\phi^{*i} + igA_\mu^a(\phi^*T^a)^{i}\label{eq:phi2}\\
  \nabla_\mu\psi_i &= \partial_\mu\psi_i - igA_\mu^a(T^a\psi)_i\label{eq:psi}
\end{align}

\subsubsection{Interactions}
\subsubsection{The superpotential}
The MSSM superpotential takes the form
\[W_\text{MSSM} = \bar{u}\mathbf{y_u}QH_u-\bar{d}\mathbf{y_d}Q H_d-\bar{e}\mathbf{y_e}L H_d+\mu H_u H_d \]
\subsection{The MSSM superpotential}
\section{Split Supersymmetry}

The non-SM interactions in our signal process are the higgsino-bino-$Z$ and the higgsino-bino-$h$ interactions. To determine their coupling, we can inspect the structure of the MSSM. We will first examine the gauge interactions of higgsino and bino gauge eigenstates. Upon doing this, it will be apparent that there are no interaction terms containing both pure higgsinos and pure binos. The reason is that electroweak symmetry breaking induces mixing among the neutralinos. After this, we will move on to the coupling involving the SM Higgs boson, and see how it is obtained. 
In the MSSM, there are two $SU(2)_L\times U(1)_Y$ Higgs doublets instead of the one that we encounter in the standard model.
\[H_u = \begin{pmatrix}H_u^+\\H_u^0\end{pmatrix};
H_d = \begin{pmatrix}H_d^0\\H_d^-\end{pmatrix}\]
Inserting the covariant derivatives into the kinetic terms will yield the interaction terms. Each of the fields $H_u$ and $H_d$ can be interpreted as the scalar field $\phi$ in \autoref{eq:phi1} and \autoref{eq:phi2}. Similarly, the supersymmetrizations of $H_u$ and $H_d$,
\[\widetilde{H_u} = \begin{pmatrix}\widetilde{H_u}^+\\\widetilde{H_u}^0\end{pmatrix};
\widetilde{H_d} = \begin{pmatrix}\widetilde{H_d}^0\\\widetilde{H_d}^-\end{pmatrix}\]
can be equated with the field $\psi$ in \autoref{eq:psi}. The component fields of these are called \emph{higgsinos}. Once we insert the generators and associated gauge fields for the $SU(2)_L\times U(1)_Y$ symmetry in the covariant derivatives, and mix the gauge fields in the manner dictated by electroweak symmetry breaking, the gauge interactions of the higgs and higgsino fields look a lot like the SM interactions. In fact, they will have the same strength. Since we are interested in neutral higgsino-like NLSPs for our signal process, let us isolate their particular interaction terms:
\[\frac{g}{\cos\theta_W}Z_\mu\left(\widetilde{H_u}^{0\dagger}\overline{\sigma}^\mu \widetilde{H_u}+
\widetilde{H_d}^{0\dagger}\overline{\sigma}^\mu \widetilde{H_d}\right)\]
Now let us turn to the interactions of gauge supermultiplets. A generic gauge supermultiplet will consist of a gauge boson $A_\mu^a$ and its two-component Weyl fermion superpartner, the \emph{gaugino} $\lambda^a$. The SM gauge bosons reside in such gauge supermultiplets, and so the index $a$ runs over the adjoint representation of the relevant gauge group. The gauge interactions of gauginos can be extracted from their kinetic term in the Lagrangian:
\[\mathcal{L}_{\text{gaugino,kinetic}} = i\lambda^{a\dagger}\overline{\sigma}^\mu\nabla_\mu\lambda^a\]
where the covariant derivative is given by
\[\nabla_\mu\lambda^a = \partial_\mu\lambda^a + gf^{abc}A_\mu^b\lambda^c.\]
The factors $f^{abc}$ represent the structure constants of the gauge group. Thus, the interaction term would look like:
\[ig\lambda^{a\dagger}\overline{\sigma}^\mu f^{abc}A_\mu^b\lambda^c\]
The bino $(\widetilde{B}^0)$ is the superpartner of the SM gauge boson associated with the $U(1)_Y$ (hypercharge) gauge symmetry. Thus, it will transform in the adjoint representation of $U(1)_Y$ as well. However, this implies that there can never be a interaction vertex containing a bino and a gauge boson, since the structure constants for $U(1)_Y$ are all simply 0. In addition, we have previously seen that examining the chiral supermultiplet conaining the higgsino does not yield a neutral higgsino-bino-$Z$ vertex either. The reason that we can consider such a vertex in our signal process is due to the fact that electroweak symmetry breaking induces mixing among neutral higgsinos, binos, and the neutral wino $(\widetilde{W}^0)$, the superpartner of the $W_\mu^3$ gauge field associated with the SM $SU(2)_L$ gauge symmetry that mixes with $B_\mu$ to form the $Z$ boson and photon.

\section{Dark Matter}\label{sec:dark_matter}
\subsection{History of Dark Matter}\label{subsec:history-of-dark-matter}

In 1933, the Swiss astronomer Fritz Zwicky turned his telescope to the night sky to measure the velocities of galaxies in a particular galaxy cluster known as the Coma cluster. He discovered that the visible matter in this cluster could not account for the speeds at which the galaxies were moving. He postulated that there must be some matter that does not emit light, but has mass and interacts only gravitationally with known matter. He termed this `Dunkle materie', or \emph{dark matter}. The evidence for dark matter has only continued to grow since then, and we now know that there is far more of it in the universe than there is regular, or \emph{baryonic} matter. The nature of dark matter is one of the most compelling mysteries in physics today.

Over the years, there have been many dark matter candidates\footnote{For a while, it was debated whether the effects of dark matter could instead be explained by a deviation of the gravitational force from the usual inverse square law at large length scales. This theory was termed Modified Newtonian Dynamics, or MOND \citep{Milgrom1983}. However, it was shown in 2006 \citep{Clowe2006} that MOND was fundamentally incompatible with the data from the bullet cluster.}, but the most widely accepted view today is that dark matter is comprised of a completely new kind of particle \footnote{Of course, this is not the only possibility. Instead of a single dark matter candidate particle, there could be a 'dark sector' comprised of multiple particles and interactions between them. See (DDM papers) for developments along this line.} that interacts only weakly with the particles of the Standard Model and moves at at non-relativistic speeds.

\strictpagecheck
\begin{figure}
  \begin{sidecaption}
    {Rotation curves for a number of spiral galaxies. \citep{Sofue2001}}
  \includegraphics[width=\textwidth]{images/rotation_curves}
\end{sidecaption}
\end{figure}
One of the most tantalizing clues to the nature of dark matter is the so-called 'WIMP miracle' - the remarkable coincidence that the observed dark matter density in the universe can arises naturally from a particle with mass and couplings comparable to the weak scale. This raises the hope that such particles might feasibly be detected at colliders or direct detection experiments.

\subsection{Dark matter, three ways}\label{dark-matter-three-ways}

There are three main methods of detecting WIMP dark matter. The first, direct detection, involves constructing a well shielded, giant vat of a relatively inert substance, and waiting for dark matter particles to interact with that substance. The second, termed indirect detection, involves searching for signs of dark matter particles annihilating each other in the cosmos. The third method, collider detection, involves producing dark matter through high energy particle collisions and searching for their associated signatures. The first two methods place relatively stringent constraints on the nature of dark matter, but have their own limitations. The smallest interaction cross-section between dark matter and regular matter that direct detection can measure is limited from below by the background of neutrinos from the sun, though there are creative methods that are being developed to deal with this background. Indirect detection suffers from large astrophysical uncertainties. Collider detection, therefore, is competitive with, and can even possibly surpass the other two methods.

\section{Calculation of the thermal relic density}

Assuming that the present-day abundance of WIMP dark matter is explained by a thermal history, the observed density of dark matter, known as the \emph{relic density} sets limits on particle dark matter candidates. We will briefly go over the standard calculation of the thermal relic density.

\newcommand{\fxv}{f(\mathbf{x},\mathbf{v})}

Consider the phase-space density of a single particle dark matter candidate in phase space, given by $\fxv$. That is, the probability of finding the particle with a position of $\mathbf{x}$ and velocity $\mathbf{v}$ is given by $\fxv d^3\mathbf{x}d^3\mathbf{v}$ \footnote{insert references}.
The time evolution of this phase-space density is given by Boltzmann's equation:
\begin{equation}
  \mathbf{L}f = \mathbf{C}f.
\end{equation}
Here, $\mathbf{L}$ is the Liouville operator, which in the non-relativistic limit reduces to 
\begin{equation}
  \mathbf{L}_\text{nr}f = \frac{\partial f}{\partial t} + \mathbf{\dot{x}}\frac{\partial f}{\partial\mathbf{x}} + \mathbf{\dot{v}}\frac{\partial f}{\partial \mathbf{v}} = \frac{d}{dt}
\end{equation}
The general-relativistic version of this operator is obtained by taking the total derivative along a world line parameterized by some affine parameter, say  $\lambda$.
\begin{equation}
  \mathbf{L} = \frac{d}{d\lambda} = \frac{dx^\mu}{d\lambda}\frac{\partial}{\partial x^\mu} + \frac{dp^\mu}{d\lambda}\frac{\partial}{\partial p^\mu}
\end{equation}
We can choose to normalize this parameter as follows:
\begin{equation}
  p^\mu = \frac{dx^\mu}{d\lambda} = m\frac{dx^\mu}{d\tau}
\end{equation}
where $\tau$ is the proper time. With this normalization, the geodesic equation becomes
\begin{equation}
  \frac{dp^\mu}{d\lambda}+\Gamma^{\mu}_{\alpha\beta}p^\alpha p^\beta = 0
\end{equation}
If we assume that there are no external non-gravitational forces, trajectory of the DM particle through phase space will be described by the geodesic. Thus, we can write the general-relativistic Liouville operator as:
\begin{equation}\label{eq:gr_liouville_operator}
  \mathbf{L} = p^\mu\frac{\partial}{\partial x^\mu} - \Gamma^\mu_{\alpha\beta}p^\alpha p^\beta\frac{\partial}{\partial p^\mu}
\end{equation}
The isotropy of the model implies that the phase-space density \emph{f} is devoid of directionality - it will depend only on the energy \emph{E} and time \emph{t}, and can be written as $f(E,t)$. Thus, all the partial derivatives in \autoref{eq:gr_liouville_operator} will vanish except for those corresponding to $\mu = 0$. If the connections $\Gamma^\mu_{\alpha\beta}$ are calculated for the FLRW metric (see \autoref{sec:sm_cosmology}), the Liouville operator simplifies to
\begin{equation}\label{eq:frw_liouville_operator}
  \mathbf{L}f = E\frac{\partial f}{\partial t} + H|\mathbf{p}|^2 \frac{\partial f}{\partial E}
\end{equation}
where $H$ is the Hubble parameter. The number density \emph{n} is given by the integral of $f(E,t)$ over all of phase-space, given by
\begin{align}
  n = 4\pi g\int dp p^2 f(E,t)
\end{align}
where $p = |\mathbf{p}|$, and $g$ is the number of spin degrees of freedom.
Dividing \autoref{eq:frw_liouville_operator} by E and integrating over the momentum coordinates gives us:
\begin{align}
  4\pi g\int dp p^2 \frac{\mathbf{L}f}{E} = 4\pi g\int dp p^2 \left(\frac{\partial f}{\partial t}- H\frac{p^2}{E}\frac{\partial f}{\partial E}\right)
\end{align}
Pulling the partial derivative outside the integral for the first term gives us
\begin{equation}
  \frac{dn}{dt} - 4\pi gH\int dp\frac{p^4}{E}\frac{\partial f}{\partial E}
\end{equation}
We know that $E^2 = p^2 - m^2$, thus $\frac{1}{E}\frac{\partial}{\partial E} = \frac{1}{p}\frac{\partial}{\partial p}$. Substituting this into the previous equation gives us:
\begin{equation}\label{eq:n_dot}
  \dot{n} - 4\pi gH\int dp p^3 \frac{\partial f}{\partial p}
\end{equation}
We can simplify the second term using integration by parts:
\begin{equation}
  \int dp p^3\frac{\partial f}{\partial p} = fp^3|_{0}^\infty - \int dp (3p^2 f)
\end{equation}
The first term on the right hand side of the above equation should evaluate to zero, since the probability of the particle having an infinite momentum must be zero. Substituting this expression into \autoref{eq:n_dot} and simplifying gives us
\begin{equation}\label{eq:simplified_n_density}
  \dot{n}+3Hn = 4\pi g\int \frac{dpp^2}{E}\mathbf{C}f
\end{equation}
The operator \textbf{C} is known as the \emph{collision} operator, and contains the information about how the dark matter particles interact with themselves and other particles. For a process of the form $1+2\rightarrow 3+4$, the collision term (on the right hand side of the above equation) for particle 1 takes the form  
\begin{marginfigure}
  \feynmandiagram[horizontal = chi2 to SM1]{
    {chi1 [particle = \(\chi\)], chi2 [particle = \(\chi\)],
    SM1 [particle = \(SM\)], SM2 [particle = \(SM\)]} -- v [blob],
  };
  \label{fig:dm_annihilation}
  \caption{Dark matter annihilating to SM particles.}
\end{marginfigure}
\begin{equation}\label{eq:collision_term}
  \begin{split}
  &-\sum_\text{spins}\int [f_1 f_2(1\pm f_3)(1\pm f_4)|\mathcal{M}_{12\rightarrow 34}|^2 + f_3 f_4(1\pm f_1)(1\pm f_2)|\mathcal{M}_{34\rightarrow 12}|^2]\\
&\times(2\pi)^4\delta^4(p_1+p_2-p_3-p_4)d\Pi_1 d\Pi_2 d\Pi_3 d\Pi_4
\end{split}
\end{equation}
where
\[d\Pi_i = \frac{d^3 p_i}{(2\pi)^3 2E_i}\]
are phase-space integration factors. This can be simplified considerably by assuming the following:
\begin{itemize}
  \item The early universe is in thermal equilibrium, and thus the phase-space distribution $f$ takes either Fermi-Dirac or Bose-Einstein form.
  \item The temperature of each species, $T_i$ is low enough that the condition $T_i \ll E_i - \mu_i$, where $\mu_i$ is its chemical potential. This means that their phase-space distributions are approximately of the Maxwell-Boltzmann form, and we can neglect the statistical factors: $(1+f)\approx 1$.
\end{itemize}
The amplitude for the forward and backwards processes should be equal:
\[|\mathcal{M}_{12\rightarrow 34}|^2 = |\mathcal{M}_{34\rightarrow 12}|^2 = |\mathcal{M}|^2 \]
Then the expression in \autoref{eq:collision_term} simplifies to
\begin{equation}
-\sum_\text{spins}\int(f_1 f_2 - f_3 f_4)|\mathcal{M}|^2(2\pi)^4\delta(p_1 + p_2 - p_3 - p_4)d\Pi_1 d\Pi_2 d\Pi_3 d\Pi_4
\end{equation}
We can relate the matrix element to the cross section as follows:
\[\sum_\text{spins}|\mathcal{M}_{ij\rightarrow kl}|^2\times(2\pi)^4\delta^4(p_i + p_j - p_k -p_l)d\Pi_jd\Pi_k = 4g_ig_j\sigma_{ij}\sqrt{(p_i\cdot p_j)^2 - (m_im_j)^2}\]
where $\sigma_{ij}$ is the scattering cross section. Let us define the M\o ller velocity,
\[v_\text{M\o l} = \frac{\sqrt{(p_i\cdot p_j)^2 - (m_im_j)^2}}{E_iE_j}\]
Substituting this in the collision term and noting that $f_i(2E_i)d\Pi_i = dn_i$, we get the collision term as
\[g_1\int \mathbf{C}f_1\frac{d^3p_1}{(2\pi)^3} = -\int[(\sigma v_\text{M\o l})_{12}dn_1dn_2 - (\sigma v_\text{M\o l})_{34}dn_3dn_4]\]
The quantity $\sigma v_\text{M\o l}$ is relatively independent of the number densities $n_i$, and thus can be taken outside the integral, to give us
\[\dot{n}_1 + 3Hn_1 = -\langle\sigma v\rangle_{12}n_1n_2 + \langle\sigma v\rangle_{34}n_3n_4.\]
where $v = v_\text{M\o l}$. Let us now apply this result to the specific $2\rightarrow 2$ annihilation case shown in \autoref{fig:dm_annihilation}. The incoming DM particles are identical, with number density \emph{n}. When the DM particles are in equilibrium with the SM particles, a condition known as \emph{detailed balance} holds that relates the number densities and annihilation cross sections for the SM and DM particles:
\[\langle\sigma v\rangle_{12}n^2_\text{eq}=\langle\sigma v\rangle_{34}n_3^\text{eq}n_4^\text{eq}\]
Thus, the Boltzmann equation can be written as
\[\dot{n}+3Hn = \langle\sigma v\rangle (n_\text{eq}^2 - n^2)\]
where $\langle\sigma v\rangle = \langle\sigma v\rangle_{12}$. Intuitively, the number density $n$ will decrease naturally as the universe expands. If we want to focus on the effects of collision, we should define a \emph{comoving number density} $Y = n/s$, where $s$ is the total entropy density of the universe. If the dominant contribution to \emph{s} comes from thermal radiation, then $sa^3$ will be approximately constant. Rewriting the Boltzmann equation in terms of $Y$ gives us
\[\frac{dY}{dt} = \langle\sigma v\rangle s (Y_\text{eq}^2 - Y^2)\]
As the early radiation-dominated universe expanded, it also cooled down. It is expedient to write the above equation in terms of temperature, using the variable $x = m/T$, where \emph{m} is the mass of the DM particle. 
\[\frac{dY}{dx} = -\frac{\langle\sigma v\rangle s}{H(x)x}\]
The thermally averaged cross-section times velocity is given by
\begin{equation}\label{eq:thermal_sigma_v}
  \langle\sigma v\rangle = \frac{\int\sigma vdn_1^\text{eq}dn_2^\text{eq}}{\int dn_1^\text{eq}dn_2^\text{eq}}
  =\frac{\int \sigma v e^{-E_1/T}e^{-E_2/T}d^3p_1d^3p_2}{\int e^{-E_1/T}e^{-E_2/T}d^3p_1d^3p_2}
\end{equation}
The momentum-space volume element in spherical coordinates is
\[d^3p_1d^3p_2 = 4\pi |\mathbf{p}_1|dE_14\pi |\mathbf{p}_2|dE_2\frac{1}{2}d(\cos\theta)\]
where $\theta$ is the angle between the 3-momenta $\mathbf{p}_1$ and $\mathbf{p}_2$. We can then change the integration variables to $E_+, E_-, s$, given by
\begin{align}
  E_+ &= E_1 + E_2\\
  E_- &= E_1 - E_2,\\
  s &= 2m^2 + 2E_1E_2-2\mathbf{p}_1\mathbf{p}_2\cos\theta
\end{align}
Thus the volume element in these variables becomes
\[d^3p_1d^3p_2 = 2\pi^2E_1E_2dE_+dE_-ds\]
And the integration limits become
\begin{align}
  |E_1| &\leq \sqrt{1-\frac{4m^2}{s}}\sqrt{E_+^2 - s}\\
  E_+ &\geq \sqrt{s}\\
  s &\geq 4m^2
\end{align}
With these variables, the numerator in \autoref{eq:thermal_sigma_v} becomes
\begin{align*}
  &2\pi^2\int dE_+\int dE_-\int ds(\sigma v)E_1E_2e^{-E_+/T}\\
  =&4\pi^2\int ds\sigma \frac{1}{2}\sqrt{s(s-4m^2)}\sqrt{1-\frac{4m^2}{s}}\int dE_+e^{-E_+/T}\sqrt{E_+-s}\\
  =&2\pi^2T\int ds\sigma(s-4m^2)\sqrt{s}K_1(\sqrt{s}/T)
\end{align*}
Similarly, the denominator simplifies to $[4\pi m^2TK_2(m/T)]^2$. The functions $K_i$ are modified Bessel functions of order \emph{i}\footnote{The ordinary differential equation
  \begin{equation*}
    x^2\ddot{y} + x\dot{y} - (x^2+n^2)y = 0
  \end{equation*}
  admits solutions of the form $y = c_1 I_n(x) + c_2K_n(x)$. The functions $I_n$ and $K_n$ are known as modified Bessel functions of the first and second kinds, respectively.
}. Thus \autoref{eq:thermal_sigma_v} simplifies to
\[\langle\sigma v\rangle = \frac{1}{8m^4TK_2^2(m/T)}\int_{4m^2}^\infty\sigma(s-4m^2)\sqrt{s}K_1(\sqrt{s}/T)ds\]
In this non-relativistic limit, this simplifies to
\[\langle\sigma v\rangle \approx b_0 + \frac{3}{2}b_1x^{-1} + ...\]
The coefficients $b_0, b_1, ...$ correspond to \emph{s}-wave, \emph{p}-wave annihilation, and so on.

\begin{figure}
  \begin{sidecaption}{Illustration of the freeze-out mechanism}
  % \includegraphics[width=\textwidth]{images/relic_density.pdf}
    %% Creator: Matplotlib, PGF backend
%%
%% To include the figure in your LaTeX document, write
%%   \input{<filename>.pgf}
%%
%% Make sure the required packages are loaded in your preamble
%%   \usepackage{pgf}
%%
%% Figures using additional raster images can only be included by \input if
%% they are in the same directory as the main LaTeX file. For loading figures
%% from other directories you can use the `import` package
%%   \usepackage{import}
%% and then include the figures with
%%   \import{<path to file>}{<filename>.pgf}
%%
%% Matplotlib used the following preamble
%%   \usepackage{fontspec}
%%   \setmonofont{Andale Mono}
%%
\begingroup%
\makeatletter%
\begin{pgfpicture}%
\pgfpathrectangle{\pgfpointorigin}{\pgfqpoint{3.888197in}{2.403038in}}%
\pgfusepath{use as bounding box, clip}%
\begin{pgfscope}%
\pgfsetbuttcap%
\pgfsetmiterjoin%
\definecolor{currentfill}{rgb}{0.941176,0.941176,0.941176}%
\pgfsetfillcolor{currentfill}%
\pgfsetlinewidth{0.000000pt}%
\definecolor{currentstroke}{rgb}{0.941176,0.941176,0.941176}%
\pgfsetstrokecolor{currentstroke}%
\pgfsetdash{}{0pt}%
\pgfpathmoveto{\pgfqpoint{0.000000in}{0.000000in}}%
\pgfpathlineto{\pgfqpoint{3.888197in}{0.000000in}}%
\pgfpathlineto{\pgfqpoint{3.888197in}{2.403038in}}%
\pgfpathlineto{\pgfqpoint{0.000000in}{2.403038in}}%
\pgfpathclose%
\pgfusepath{fill}%
\end{pgfscope}%
\begin{pgfscope}%
\pgfsetbuttcap%
\pgfsetmiterjoin%
\definecolor{currentfill}{rgb}{0.941176,0.941176,0.941176}%
\pgfsetfillcolor{currentfill}%
\pgfsetlinewidth{0.000000pt}%
\definecolor{currentstroke}{rgb}{0.000000,0.000000,0.000000}%
\pgfsetstrokecolor{currentstroke}%
\pgfsetstrokeopacity{0.000000}%
\pgfsetdash{}{0pt}%
\pgfpathmoveto{\pgfqpoint{0.676752in}{0.518611in}}%
\pgfpathlineto{\pgfqpoint{3.738197in}{0.518611in}}%
\pgfpathlineto{\pgfqpoint{3.738197in}{2.253038in}}%
\pgfpathlineto{\pgfqpoint{0.676752in}{2.253038in}}%
\pgfpathclose%
\pgfusepath{fill}%
\end{pgfscope}%
\begin{pgfscope}%
\pgfpathrectangle{\pgfqpoint{0.676752in}{0.518611in}}{\pgfqpoint{3.061445in}{1.734427in}} %
\pgfusepath{clip}%
\pgfsetbuttcap%
\pgfsetroundjoin%
\pgfsetlinewidth{1.003750pt}%
\definecolor{currentstroke}{rgb}{0.796078,0.796078,0.796078}%
\pgfsetstrokecolor{currentstroke}%
\pgfsetdash{}{0pt}%
\pgfpathmoveto{\pgfqpoint{0.815909in}{0.518611in}}%
\pgfpathlineto{\pgfqpoint{0.815909in}{2.253038in}}%
\pgfusepath{stroke}%
\end{pgfscope}%
\begin{pgfscope}%
\pgftext[x=0.815909in,y=0.469999in,,top]{\setmainfont{Minion Pro}\rmfamily\fontsize{10.000000}{12.000000}\selectfont \(\displaystyle {10^{0}}\)}%
\end{pgfscope}%
\begin{pgfscope}%
\pgfpathrectangle{\pgfqpoint{0.676752in}{0.518611in}}{\pgfqpoint{3.061445in}{1.734427in}} %
\pgfusepath{clip}%
\pgfsetbuttcap%
\pgfsetroundjoin%
\pgfsetlinewidth{1.003750pt}%
\definecolor{currentstroke}{rgb}{0.796078,0.796078,0.796078}%
\pgfsetstrokecolor{currentstroke}%
\pgfsetdash{}{0pt}%
\pgfpathmoveto{\pgfqpoint{1.743619in}{0.518611in}}%
\pgfpathlineto{\pgfqpoint{1.743619in}{2.253038in}}%
\pgfusepath{stroke}%
\end{pgfscope}%
\begin{pgfscope}%
\pgftext[x=1.743619in,y=0.469999in,,top]{\setmainfont{Minion Pro}\rmfamily\fontsize{10.000000}{12.000000}\selectfont \(\displaystyle {10^{1}}\)}%
\end{pgfscope}%
\begin{pgfscope}%
\pgfpathrectangle{\pgfqpoint{0.676752in}{0.518611in}}{\pgfqpoint{3.061445in}{1.734427in}} %
\pgfusepath{clip}%
\pgfsetbuttcap%
\pgfsetroundjoin%
\pgfsetlinewidth{1.003750pt}%
\definecolor{currentstroke}{rgb}{0.796078,0.796078,0.796078}%
\pgfsetstrokecolor{currentstroke}%
\pgfsetdash{}{0pt}%
\pgfpathmoveto{\pgfqpoint{2.671330in}{0.518611in}}%
\pgfpathlineto{\pgfqpoint{2.671330in}{2.253038in}}%
\pgfusepath{stroke}%
\end{pgfscope}%
\begin{pgfscope}%
\pgftext[x=2.671330in,y=0.469999in,,top]{\setmainfont{Minion Pro}\rmfamily\fontsize{10.000000}{12.000000}\selectfont \(\displaystyle {10^{2}}\)}%
\end{pgfscope}%
\begin{pgfscope}%
\pgfpathrectangle{\pgfqpoint{0.676752in}{0.518611in}}{\pgfqpoint{3.061445in}{1.734427in}} %
\pgfusepath{clip}%
\pgfsetbuttcap%
\pgfsetroundjoin%
\pgfsetlinewidth{1.003750pt}%
\definecolor{currentstroke}{rgb}{0.796078,0.796078,0.796078}%
\pgfsetstrokecolor{currentstroke}%
\pgfsetdash{}{0pt}%
\pgfpathmoveto{\pgfqpoint{3.599040in}{0.518611in}}%
\pgfpathlineto{\pgfqpoint{3.599040in}{2.253038in}}%
\pgfusepath{stroke}%
\end{pgfscope}%
\begin{pgfscope}%
\pgftext[x=3.599040in,y=0.469999in,,top]{\setmainfont{Minion Pro}\rmfamily\fontsize{10.000000}{12.000000}\selectfont \(\displaystyle {10^{3}}\)}%
\end{pgfscope}%
\begin{pgfscope}%
\pgftext[x=2.207475in,y=0.282222in,,top]{\setmainfont{Minion Pro}\rmfamily\fontsize{10.000000}{12.000000}\selectfont x}%
\end{pgfscope}%
\begin{pgfscope}%
\pgfpathrectangle{\pgfqpoint{0.676752in}{0.518611in}}{\pgfqpoint{3.061445in}{1.734427in}} %
\pgfusepath{clip}%
\pgfsetbuttcap%
\pgfsetroundjoin%
\pgfsetlinewidth{1.003750pt}%
\definecolor{currentstroke}{rgb}{0.796078,0.796078,0.796078}%
\pgfsetstrokecolor{currentstroke}%
\pgfsetdash{}{0pt}%
\pgfpathmoveto{\pgfqpoint{0.676752in}{0.518611in}}%
\pgfpathlineto{\pgfqpoint{3.738197in}{0.518611in}}%
\pgfusepath{stroke}%
\end{pgfscope}%
\begin{pgfscope}%
\pgftext[x=0.340139in,y=0.469166in,left,base]{\setmainfont{Minion Pro}\rmfamily\fontsize{10.000000}{12.000000}\selectfont \(\displaystyle {10^{-5}}\)}%
\end{pgfscope}%
\begin{pgfscope}%
\pgfpathrectangle{\pgfqpoint{0.676752in}{0.518611in}}{\pgfqpoint{3.061445in}{1.734427in}} %
\pgfusepath{clip}%
\pgfsetbuttcap%
\pgfsetroundjoin%
\pgfsetlinewidth{1.003750pt}%
\definecolor{currentstroke}{rgb}{0.796078,0.796078,0.796078}%
\pgfsetstrokecolor{currentstroke}%
\pgfsetdash{}{0pt}%
\pgfpathmoveto{\pgfqpoint{0.676752in}{0.843816in}}%
\pgfpathlineto{\pgfqpoint{3.738197in}{0.843816in}}%
\pgfusepath{stroke}%
\end{pgfscope}%
\begin{pgfscope}%
\pgftext[x=0.340139in,y=0.794371in,left,base]{\setmainfont{Minion Pro}\rmfamily\fontsize{10.000000}{12.000000}\selectfont \(\displaystyle {10^{-2}}\)}%
\end{pgfscope}%
\begin{pgfscope}%
\pgfpathrectangle{\pgfqpoint{0.676752in}{0.518611in}}{\pgfqpoint{3.061445in}{1.734427in}} %
\pgfusepath{clip}%
\pgfsetbuttcap%
\pgfsetroundjoin%
\pgfsetlinewidth{1.003750pt}%
\definecolor{currentstroke}{rgb}{0.796078,0.796078,0.796078}%
\pgfsetstrokecolor{currentstroke}%
\pgfsetdash{}{0pt}%
\pgfpathmoveto{\pgfqpoint{0.676752in}{1.169021in}}%
\pgfpathlineto{\pgfqpoint{3.738197in}{1.169021in}}%
\pgfusepath{stroke}%
\end{pgfscope}%
\begin{pgfscope}%
\pgftext[x=0.426945in,y=1.119576in,left,base]{\setmainfont{Minion Pro}\rmfamily\fontsize{10.000000}{12.000000}\selectfont \(\displaystyle {10^{1}}\)}%
\end{pgfscope}%
\begin{pgfscope}%
\pgfpathrectangle{\pgfqpoint{0.676752in}{0.518611in}}{\pgfqpoint{3.061445in}{1.734427in}} %
\pgfusepath{clip}%
\pgfsetbuttcap%
\pgfsetroundjoin%
\pgfsetlinewidth{1.003750pt}%
\definecolor{currentstroke}{rgb}{0.796078,0.796078,0.796078}%
\pgfsetstrokecolor{currentstroke}%
\pgfsetdash{}{0pt}%
\pgfpathmoveto{\pgfqpoint{0.676752in}{1.494226in}}%
\pgfpathlineto{\pgfqpoint{3.738197in}{1.494226in}}%
\pgfusepath{stroke}%
\end{pgfscope}%
\begin{pgfscope}%
\pgftext[x=0.426945in,y=1.444782in,left,base]{\setmainfont{Minion Pro}\rmfamily\fontsize{10.000000}{12.000000}\selectfont \(\displaystyle {10^{4}}\)}%
\end{pgfscope}%
\begin{pgfscope}%
\pgfpathrectangle{\pgfqpoint{0.676752in}{0.518611in}}{\pgfqpoint{3.061445in}{1.734427in}} %
\pgfusepath{clip}%
\pgfsetbuttcap%
\pgfsetroundjoin%
\pgfsetlinewidth{1.003750pt}%
\definecolor{currentstroke}{rgb}{0.796078,0.796078,0.796078}%
\pgfsetstrokecolor{currentstroke}%
\pgfsetdash{}{0pt}%
\pgfpathmoveto{\pgfqpoint{0.676752in}{1.819431in}}%
\pgfpathlineto{\pgfqpoint{3.738197in}{1.819431in}}%
\pgfusepath{stroke}%
\end{pgfscope}%
\begin{pgfscope}%
\pgftext[x=0.426945in,y=1.769987in,left,base]{\setmainfont{Minion Pro}\rmfamily\fontsize{10.000000}{12.000000}\selectfont \(\displaystyle {10^{7}}\)}%
\end{pgfscope}%
\begin{pgfscope}%
\pgfpathrectangle{\pgfqpoint{0.676752in}{0.518611in}}{\pgfqpoint{3.061445in}{1.734427in}} %
\pgfusepath{clip}%
\pgfsetbuttcap%
\pgfsetroundjoin%
\pgfsetlinewidth{1.003750pt}%
\definecolor{currentstroke}{rgb}{0.796078,0.796078,0.796078}%
\pgfsetstrokecolor{currentstroke}%
\pgfsetdash{}{0pt}%
\pgfpathmoveto{\pgfqpoint{0.676752in}{2.144636in}}%
\pgfpathlineto{\pgfqpoint{3.738197in}{2.144636in}}%
\pgfusepath{stroke}%
\end{pgfscope}%
\begin{pgfscope}%
\pgftext[x=0.371582in,y=2.095192in,left,base]{\setmainfont{Minion Pro}\rmfamily\fontsize{10.000000}{12.000000}\selectfont \(\displaystyle {10^{10}}\)}%
\end{pgfscope}%
\begin{pgfscope}%
\pgftext[x=0.284583in,y=1.385824in,,bottom,rotate=90.000000]{\setmainfont{Minion Pro}\rmfamily\fontsize{10.000000}{12.000000}\selectfont Number density}%
\end{pgfscope}%
\begin{pgfscope}%
\pgfpathrectangle{\pgfqpoint{0.676752in}{0.518611in}}{\pgfqpoint{3.061445in}{1.734427in}} %
\pgfusepath{clip}%
\pgfsetbuttcap%
\pgfsetroundjoin%
\pgfsetlinewidth{0.501875pt}%
\definecolor{currentstroke}{rgb}{0.000000,0.560784,0.835294}%
\pgfsetstrokecolor{currentstroke}%
\pgfsetdash{}{0pt}%
\pgfpathmoveto{\pgfqpoint{0.815909in}{2.169617in}}%
\pgfpathlineto{\pgfqpoint{0.854276in}{2.172283in}}%
\pgfpathlineto{\pgfqpoint{0.889306in}{2.173719in}}%
\pgfpathlineto{\pgfqpoint{0.979150in}{2.175356in}}%
\pgfpathlineto{\pgfqpoint{1.052567in}{2.174113in}}%
\pgfpathlineto{\pgfqpoint{1.114645in}{2.170883in}}%
\pgfpathlineto{\pgfqpoint{1.168423in}{2.166198in}}%
\pgfpathlineto{\pgfqpoint{1.230509in}{2.158266in}}%
\pgfpathlineto{\pgfqpoint{1.284292in}{2.148878in}}%
\pgfpathlineto{\pgfqpoint{1.331735in}{2.138379in}}%
\pgfpathlineto{\pgfqpoint{1.374174in}{2.127003in}}%
\pgfpathlineto{\pgfqpoint{1.421619in}{2.111801in}}%
\pgfpathlineto{\pgfqpoint{1.464061in}{2.095722in}}%
\pgfpathlineto{\pgfqpoint{1.502454in}{2.078933in}}%
\pgfpathlineto{\pgfqpoint{1.544164in}{2.058022in}}%
\pgfpathlineto{\pgfqpoint{1.581959in}{2.036425in}}%
\pgfpathlineto{\pgfqpoint{1.616510in}{2.014260in}}%
\pgfpathlineto{\pgfqpoint{1.653398in}{1.987800in}}%
\pgfpathlineto{\pgfqpoint{1.687190in}{1.960798in}}%
\pgfpathlineto{\pgfqpoint{1.722629in}{1.929380in}}%
\pgfpathlineto{\pgfqpoint{1.755201in}{1.897461in}}%
\pgfpathlineto{\pgfqpoint{1.788949in}{1.861044in}}%
\pgfpathlineto{\pgfqpoint{1.820087in}{1.824169in}}%
\pgfpathlineto{\pgfqpoint{1.852078in}{1.782741in}}%
\pgfpathlineto{\pgfqpoint{1.884562in}{1.736697in}}%
\pgfpathlineto{\pgfqpoint{1.914621in}{1.690231in}}%
\pgfpathlineto{\pgfqpoint{1.945041in}{1.639139in}}%
\pgfpathlineto{\pgfqpoint{1.975594in}{1.583423in}}%
\pgfpathlineto{\pgfqpoint{2.006097in}{1.523247in}}%
\pgfpathlineto{\pgfqpoint{2.046019in}{1.438903in}}%
\pgfpathlineto{\pgfqpoint{2.075332in}{1.378767in}}%
\pgfpathlineto{\pgfqpoint{2.090930in}{1.351591in}}%
\pgfpathlineto{\pgfqpoint{2.104305in}{1.332491in}}%
\pgfpathlineto{\pgfqpoint{2.117250in}{1.317734in}}%
\pgfpathlineto{\pgfqpoint{2.131333in}{1.305156in}}%
\pgfpathlineto{\pgfqpoint{2.146424in}{1.294647in}}%
\pgfpathlineto{\pgfqpoint{2.165234in}{1.284466in}}%
\pgfpathlineto{\pgfqpoint{2.187240in}{1.275231in}}%
\pgfpathlineto{\pgfqpoint{2.214411in}{1.266374in}}%
\pgfpathlineto{\pgfqpoint{2.248005in}{1.257923in}}%
\pgfpathlineto{\pgfqpoint{2.288492in}{1.250103in}}%
\pgfpathlineto{\pgfqpoint{2.339286in}{1.242653in}}%
\pgfpathlineto{\pgfqpoint{2.402047in}{1.235782in}}%
\pgfpathlineto{\pgfqpoint{2.480964in}{1.229477in}}%
\pgfpathlineto{\pgfqpoint{2.580563in}{1.223853in}}%
\pgfpathlineto{\pgfqpoint{2.707903in}{1.218991in}}%
\pgfpathlineto{\pgfqpoint{2.874923in}{1.214952in}}%
\pgfpathlineto{\pgfqpoint{3.102571in}{1.211807in}}%
\pgfpathlineto{\pgfqpoint{3.435820in}{1.209606in}}%
\pgfpathlineto{\pgfqpoint{3.599040in}{1.209052in}}%
\pgfpathlineto{\pgfqpoint{3.599040in}{1.209052in}}%
\pgfusepath{stroke}%
\end{pgfscope}%
\begin{pgfscope}%
\pgfpathrectangle{\pgfqpoint{0.676752in}{0.518611in}}{\pgfqpoint{3.061445in}{1.734427in}} %
\pgfusepath{clip}%
\pgfsetbuttcap%
\pgfsetroundjoin%
\pgfsetlinewidth{0.501875pt}%
\definecolor{currentstroke}{rgb}{0.988235,0.309804,0.188235}%
\pgfsetstrokecolor{currentstroke}%
\pgfsetdash{{5.600000pt}{2.400000pt}}{0.000000pt}%
\pgfpathmoveto{\pgfqpoint{0.815909in}{2.170262in}}%
\pgfpathlineto{\pgfqpoint{0.854276in}{2.172283in}}%
\pgfpathlineto{\pgfqpoint{0.889306in}{2.173719in}}%
\pgfpathlineto{\pgfqpoint{0.979150in}{2.175356in}}%
\pgfpathlineto{\pgfqpoint{1.052567in}{2.174113in}}%
\pgfpathlineto{\pgfqpoint{1.114645in}{2.170883in}}%
\pgfpathlineto{\pgfqpoint{1.168423in}{2.166198in}}%
\pgfpathlineto{\pgfqpoint{1.230509in}{2.158266in}}%
\pgfpathlineto{\pgfqpoint{1.284292in}{2.148878in}}%
\pgfpathlineto{\pgfqpoint{1.331735in}{2.138379in}}%
\pgfpathlineto{\pgfqpoint{1.374174in}{2.127003in}}%
\pgfpathlineto{\pgfqpoint{1.421619in}{2.111801in}}%
\pgfpathlineto{\pgfqpoint{1.464061in}{2.095722in}}%
\pgfpathlineto{\pgfqpoint{1.502454in}{2.078933in}}%
\pgfpathlineto{\pgfqpoint{1.544164in}{2.058022in}}%
\pgfpathlineto{\pgfqpoint{1.581959in}{2.036425in}}%
\pgfpathlineto{\pgfqpoint{1.616510in}{2.014260in}}%
\pgfpathlineto{\pgfqpoint{1.653398in}{1.987800in}}%
\pgfpathlineto{\pgfqpoint{1.687190in}{1.960798in}}%
\pgfpathlineto{\pgfqpoint{1.722629in}{1.929380in}}%
\pgfpathlineto{\pgfqpoint{1.755201in}{1.897461in}}%
\pgfpathlineto{\pgfqpoint{1.788949in}{1.861044in}}%
\pgfpathlineto{\pgfqpoint{1.820087in}{1.824169in}}%
\pgfpathlineto{\pgfqpoint{1.852078in}{1.782740in}}%
\pgfpathlineto{\pgfqpoint{1.884562in}{1.736694in}}%
\pgfpathlineto{\pgfqpoint{1.914621in}{1.690223in}}%
\pgfpathlineto{\pgfqpoint{1.945041in}{1.639112in}}%
\pgfpathlineto{\pgfqpoint{1.975594in}{1.583321in}}%
\pgfpathlineto{\pgfqpoint{2.006097in}{1.522817in}}%
\pgfpathlineto{\pgfqpoint{2.036403in}{1.457575in}}%
\pgfpathlineto{\pgfqpoint{2.066399in}{1.387575in}}%
\pgfpathlineto{\pgfqpoint{2.095998in}{1.312801in}}%
\pgfpathlineto{\pgfqpoint{2.125135in}{1.233244in}}%
\pgfpathlineto{\pgfqpoint{2.155215in}{1.144444in}}%
\pgfpathlineto{\pgfqpoint{2.184554in}{1.050811in}}%
\pgfpathlineto{\pgfqpoint{2.214411in}{0.947862in}}%
\pgfpathlineto{\pgfqpoint{2.243374in}{0.840052in}}%
\pgfpathlineto{\pgfqpoint{2.272565in}{0.722875in}}%
\pgfpathlineto{\pgfqpoint{2.301803in}{0.596299in}}%
\pgfpathlineto{\pgfqpoint{2.320838in}{0.508611in}}%
\pgfpathlineto{\pgfqpoint{2.320838in}{0.508611in}}%
\pgfusepath{stroke}%
\end{pgfscope}%
\begin{pgfscope}%
\pgfsetrectcap%
\pgfsetmiterjoin%
\pgfsetlinewidth{3.011250pt}%
\definecolor{currentstroke}{rgb}{0.941176,0.941176,0.941176}%
\pgfsetstrokecolor{currentstroke}%
\pgfsetdash{}{0pt}%
\pgfpathmoveto{\pgfqpoint{0.676752in}{0.518611in}}%
\pgfpathlineto{\pgfqpoint{0.676752in}{2.253038in}}%
\pgfusepath{stroke}%
\end{pgfscope}%
\begin{pgfscope}%
\pgfsetrectcap%
\pgfsetmiterjoin%
\pgfsetlinewidth{3.011250pt}%
\definecolor{currentstroke}{rgb}{0.941176,0.941176,0.941176}%
\pgfsetstrokecolor{currentstroke}%
\pgfsetdash{}{0pt}%
\pgfpathmoveto{\pgfqpoint{3.738197in}{0.518611in}}%
\pgfpathlineto{\pgfqpoint{3.738197in}{2.253038in}}%
\pgfusepath{stroke}%
\end{pgfscope}%
\begin{pgfscope}%
\pgfsetrectcap%
\pgfsetmiterjoin%
\pgfsetlinewidth{3.011250pt}%
\definecolor{currentstroke}{rgb}{0.941176,0.941176,0.941176}%
\pgfsetstrokecolor{currentstroke}%
\pgfsetdash{}{0pt}%
\pgfpathmoveto{\pgfqpoint{0.676752in}{0.518611in}}%
\pgfpathlineto{\pgfqpoint{3.738197in}{0.518611in}}%
\pgfusepath{stroke}%
\end{pgfscope}%
\begin{pgfscope}%
\pgfsetrectcap%
\pgfsetmiterjoin%
\pgfsetlinewidth{3.011250pt}%
\definecolor{currentstroke}{rgb}{0.941176,0.941176,0.941176}%
\pgfsetstrokecolor{currentstroke}%
\pgfsetdash{}{0pt}%
\pgfpathmoveto{\pgfqpoint{0.676752in}{2.253038in}}%
\pgfpathlineto{\pgfqpoint{3.738197in}{2.253038in}}%
\pgfusepath{stroke}%
\end{pgfscope}%
\end{pgfpicture}%
\makeatother%
\endgroup%

  \end{sidecaption}
\end{figure}
