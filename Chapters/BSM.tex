\chapter{Beyond the Standard model}\label{ch:bsm}

BSM physics

\section{Extending the scalar sector: The Two-Higgs Doublet Model}

- Citations - (Felix papers)\\

Two-Higgs Doublet Models (2HDMs)\footnote{For a detailed review, see\citep{Branco2012}} are a class of theories that implement a simple, yet theoreticaly well-motivated extension to the SM - they expand the scalar sector of the SM by introducing an additional $SU(2)$ Higgs doublet field. In fact, the MSSM and related theories (such as the NMSSM and split SUSY), are types of 2HDMs.

\subsection{The 2HDM Lagrangian}
The most general scalar potential for two doublets $\Phi_1$ and $\Phi_2$ with hypercharge $+1$ is:
\begin{align*}
  V &= m_{11}^2\Phi_1^\dagger\Phi_1 + m_{22}^2\Phi_2^\dagger\Phi_2 - m_{12}^2\left(\Phi_1^\dagger\Phi_2 + \Phi_2^\dagger\Phi_1 \right) + \frac{\lambda_1}{2}\left(\Phi_1^\dagger\Phi_1 \right)^2 + \frac{\lambda_2}{2}\left(\Phi_2^\dagger\Phi_2 \right)^2\\
&  +\lambda_3\Phi_1^\dagger\Phi_1 \Phi_2^\dagger\Phi_2 + \lambda_4\Phi_1^\dagger\Phi_2 \Phi_2^\dagger\Phi_1 + \frac{\lambda_5}{2}\left[\left(\Phi_1^\dagger\Phi_2 \right)^2+\left(\Phi_2^\dagger\Phi_1\right)^2 \right]
\end{align*}

\subsection{The Type-II 2HDM}

In the type-II 2HDM, $\Phi_2$ couples to right-handed up-type quarks, while $\Phi_1$ couples to right-handed down-type quarks and right-handed leptons. The MSSM, which we will discuss later, is an example of a type-II 2HDM.

\subsection{The 2HDM mass spectrum}

After electroweak symmetry breaking, there are five mass eigenstates: the CP-even higgses $h$ and $H$, the CP-odd pseudoscalar Higgs $A$, and a pair of charged Higgses $H^\pm$.

\subsection{Interactions}

The interactions of these additional scalar particles are given by the Yukawa terms of of the Lagrangian:
\begin{align*}
\mathcal{L}^{\mathrm{2HDM}}_{\text{Yukawa}} =& - \sum_{f = u, d, l} \frac{m_f}{v}
\left(\xi_h^f \overline{f}fh+\xi_H^f \overline{f}fH-i\xi_A^f \overline{f}\gamma_5fA \right)\\
&-\left\{\frac{\sqrt{2}V_{ud}}{v}\overline{u}\left(m_u\xi_A^uP_l+m_d\xi_A^dP_R\right)dH^+ + \frac{\sqrt{2}m_l\xi^l_A}{v}\overline{\nu}_Ll_RH^+ + h.c.\right\}
\end{align*}
\section{Supersymmetry and the MSSM}
\subsection{Motivations}
\subsubsection{The hierarchy problem}
The mass of the Higgs boson is not protected by any symmetry, and so is highly susceptible to quantum corrections. However, we see that the mass is 126 GeV. Thus, quantum corrections much larger than its mass have to be canceled out somehow. An elegant mechanism for doing this is \emph{supersymmetry}.
\subsubsection{Stable dark matter candidate}
The MSSM has a new kind of symmetry known as \emph{R-parity}. The consequence of this is that the lightest supersymmetric particle (LSP) must be absolutely stable, that is, it cannot decay further into other particles.
\subsubsection{Gauge coupling unification}
The MSSM leads to the unification of the strong and electroweak forces at high energies, similar to how the electromagnetic and weak forces unify at high energies.
\subsection{Supersymmetry basics}
\subsubsection{Chiral and gauge supermultiplets}
\subsubsection{Interactions}
\subsubsection{The superpotential}
\subsection{The MSSM superpotential}
\section{Split Supersymmetry}

The non-SM interactions in our signal process are the higgsino-bino-$Z$ and the higgsino-bino-$h$ interactions. To determine their coupling, we can inspect the structure of the MSSM. We will first examine the gauge interactions of higgsino and bino gauge eigenstates. Upon doing this, it will be apparent that there are no interaction terms containing both pure higgsinos and pure binos. The reason is that electroweak symmetry breaking induces mixing among the neutralinos. After this, we will move on to the coupling involving the SM Higgs boson, and see how it is obtained. 
All the particles in the MSSM are grouped into structures known as supermultiplets. They come in two varieties: \emph{chiral} and \emph{gauge} supermultiplets. The SM fermions and the components of the SM Higgs doublet reside inchiral supermultiplets, and the SM gauge bosons reside in gauge supermultiplets. Let us first look at what the gauge interactions of a generic chiral and gauge supermultiplet look like.
For a generic chiral supermultiplet consisting of a complex scalar $\phi$ and its superpartner, a two-component Weyl fermion $\psi$, the kinetic terms are 
\[\mathcal{L}_{\text{chiral,kinetic,fermion}} = -\partial^\mu\phi^*\partial_\mu\phi + i\psi^{\dagger}\overline{\sigma}^\mu\partial_\mu\psi.\]
If the Lagrangian density is invariant under gauge transformations of the chiral supermultiplet under a gauge group with generators $(T^a)_i^j$ and associated gauge fields $A_\mu^a$, the partial derivatives above can be promoted to covariant derivatives:
\begin{align}
  \nabla_\mu\phi_i &= \partial_\mu\phi_i - igA_\mu^a(T^a\phi)_i\label{eq:phi1}\\
  \nabla_\mu\phi^{*i} &= \partial_\mu\phi^{*i} + igA_\mu^a(\phi^*T^a)^{i}\label{eq:phi2}\\
  \nabla_\mu\psi_i &= \partial_\mu\psi_i - igA_\mu^a(T^a\psi)_i\label{eq:psi}
\end{align}
In the MSSM, there are two $SU(2)_L\times U(1)_Y$ Higgs doublets instead of the one that we encounter in the standard model.
\[H_u = \begin{pmatrix}H_u^+\\H_u^0\end{pmatrix};
H_d = \begin{pmatrix}H_d^0\\H_d^-\end{pmatrix}\]
Inserting the covariant derivatives into the kinetic terms will yield the interaction terms. Each of the fields $H_u$ and $H_d$ can be interpreted as the scalar field $\phi$ in \autoref{eq:phi1} and \autoref{eq:phi2}. Similarly, the supersymmetrizations of $H_u$ and $H_d$,
\[\widetilde{H_u} = \begin{pmatrix}\widetilde{H_u}^+\\\widetilde{H_u}^0\end{pmatrix};
\widetilde{H_d} = \begin{pmatrix}\widetilde{H_d}^0\\\widetilde{H_d}^-\end{pmatrix}\]
can be equated with the field $\psi$ in \autoref{eq:psi}. The component fields of these are called \emph{higgsinos}. Once we insert the generators and associated gauge fields for the $SU(2)_L\times U(1)_Y$ symmetry in the covariant derivatives, and mix the gauge fields in the manner dictated by electroweak symmetry breaking, the gauge interactions of the higgs and higgsino fields look a lot like the SM interactions. In fact, they will have the same strength. Since we are interested in neutral higgsino-like NLSPs for our signal process, let us isolate their particular interaction terms:
\[\frac{g}{\cos\theta_W}Z_\mu\left(\widetilde{H_u}^{0\dagger}\overline{\sigma}^\mu \widetilde{H_u}+
\widetilde{H_d}^{0\dagger}\overline{\sigma}^\mu \widetilde{H_d}\right)\]
Now let us turn to the interactions of gauge supermultiplets. A generic gauge supermultiplet will consist of a gauge boson $A_\mu^a$ and its two-component Weyl fermion superpartner, the \emph{gaugino} $\lambda^a$. The SM gauge bosons reside in such gauge supermultiplets, and so the index $a$ runs over the adjoint representation of the relevant gauge group. The gauge interactions of gauginos can be extracted from their kinetic term in the Lagrangian:
\[\mathcal{L}_{\text{gaugino,kinetic}} = i\lambda^{a\dagger}\overline{\sigma}^\mu\nabla_\mu\lambda^a\]
where the covariant derivative is given by
\[\nabla_\mu\lambda^a = \partial_\mu\lambda^a + gf^{abc}A_\mu^b\lambda^c.\]
The factors $f^{abc}$ represent the structure constants of the gauge group. Thus, the interaction term would look like:
\[ig\lambda^{a\dagger}\overline{\sigma}^\mu f^{abc}A_\mu^b\lambda^c\]
The bino $(\widetilde{B}^0)$ is the superpartner of the SM gauge boson associated with the $U(1)_Y$ (hypercharge) gauge symmetry. Thus, it will transform in the adjoint representation of $U(1)_Y$ as well. However, this implies that there can never be a interaction vertex containing a bino and a gauge boson, since the structure constants for $U(1)_Y$ are all simply 0. In addition, we have previously seen that examining the chiral supermultiplet conaining the higgsino does not yield a neutral higgsino-bino-$Z$ vertex either. The reason that we can consider such a vertex in our signal process is due to the fact that electroweak symmetry breaking induces mixing among neutral higgsinos, binos, and the neutral wino $(\widetilde{W}^0)$, the superpartner of the $W_\mu^3$ gauge field associated with the SM $SU(2)_L$ gauge symmetry that mixes with $B_\mu$ to form the $Z$ boson and photon.

\section{Dark Matter}
\subsection{History of Dark Matter}\label{history-of-dark-matter}

In 1933, the Swiss astronomer Fritz Zwicky turned his telescope to the night sky to measure the velocities of galaxies in a particular galaxy cluster known as the Coma cluster. He discovered that the visible matter in this cluster could not account for the speeds at which the galaxies were moving. He postulated that there must be some matter that does not emit light, but has mass and interacts only gravitationally with known matter. He termed this `Dunkle materie', or \emph{dark matter}. The evidence for dark matter has only continued to grow since then, and we now know that there is far more of it in the universe than there is regular, or \emph{baryonic} matter. The nature of dark matter is one of the most compelling mysteries in physics today.

Over the years, there have been many dark matter candidates\sidefootnote{For a while, it was debated whether the effects of dark matter could instead be explained by a deviation of the gravitational force from the usual inverse square law at large length scales. This theory was termed Modified Newtonian Dynamics, or MOND \citep{Milgrom1983}. However, it was shown in 2006 \citep{Clowe2006} that MOND was fundamentally incompatible with the data from the bullet cluster.}, but the most widely accepted view today is that dark matter is comprised of a completely new kind of particle \sidefootnote{Of course, this is not the only possibility. Instead of a single dark matter candidate particle, there could be a 'dark sector' comprised of multiple particles and interactions between them. See (DDM papers) for developments along this line.} that interacts only weakly with the particles of the Standard Model and moves at at non-relativistic speeds.

One of the most tantalizing clues to the nature of dark matter is the so-called 'WIMP miracle' - the remarkable coincidence that the observed dark matter density in the universe can arises naturally from a particle with mass and couplings comparable to the weak scale. This raises the hope that such particles might feasibly be detected at colliders or direct detection experiments.

\subsection{Dark matter, three ways}\label{dark-matter-three-ways}

There are three main methods of detecting WIMP dark matter. The first, direct detection, involves constructing a well shielded, giant vat of a relatively inert substance, and waiting for dark matter particles to interact with that substance. The second, termed indirect detection, involves searching for signs of dark matter particles annihilating each other in the cosmos. The third method, collider detection, involves producing dark matter through high energy particle collisions and searching for their associated signatures. The first two methods place relatively stringent constraints on the nature of dark matter, but have their own limitations. The smallest interaction cross-section between dark matter and regular matter that direct detection can measure is limited from below by the background of neutrinos from the sun, though there are creative methods that are being developed to deal with this background. Indirect detection suffers from large astrophysical uncertainties. Collider detection, therefore, is competitive with, and can even possibly surpass the other two methods.
