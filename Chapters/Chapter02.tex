\chapter{The Standard Model}\label{ch:sm}

For a (much) more detailed treatment, see \citep{Schwartz2014}.
\section{Quantum Field Theory}
So far, we have only spoken in terms of particles. However, the real constituents of the Standard Model are \emph{quantum fields}. A quantum field is a set of one or more functions of spacetime coordinates that transform. They are generally classified by how they transform under Lorentz transformations (scalar, vector, tensor, spinor, etc). 

A theory of quantum fields can be specified by writing down its \emph{Lagrangian density} (henceforth referred to as just the Lagrangian), which is a functional of the fields and their derivatives. The Lagrangian for a quantum field theory consists of two kinds of terms: kinetic terms, which involve the product of exactly two fields (or their derivatives with respect to spacetime coordinates), and interaction terms, which contain three or more fields (or their derivatives). Terms that contain exactly two of the same field (and no derivatives) are sometimes called \emph{mass terms}. 

On the one hand, it was impossible to put fermion mass terms into the SM Lagrangian by hand, while simultaneously preserving gauge invariance. On the other, we knew that the weak force mediators, the W and Z boson, were massive, but it was not possible to write down renormalizable mass terms for them. 

Remarkably, each of these two problems would turn out to be a solution for the other.

\section{Electroweak Symmetry Breaking}

Electroweak symmetry breaking is the cornerstone of the Standard Model. 

It is achieved by adding a single, complex doublet field to the previously discussed fermions and gauge bosons:

\begin{align*}
H = 
\left(
\begin{array}{c}
    H^{+}\\
    H^0
\end{array}
\right)
\end{align*}

The terms of the Lagrangian that involve this field can be written as follows:

\begin{align*}
  \mathcal{L}(H) &= \frac{1}{2}(D_\mu H)^\dag(D_\mu H)-V(H)-\frac{1}{4}F_{\mu\nu}F^{\mu\nu}\\
  V(H) &= \left(H^\dag H-\frac{v^2}{2}\right)
\end{align*}

This Lagrangian is invariant under $U(1)$ transformations. It is also invariant under $SU(2)$ transformations of the form $H\rightarrow e^{i\epsilon_a(x)T^a}H$. Here, $\epsilon_a (x)$ is a space-time dependent parameter and $T_a = \frac{\tau_a}{2}$, where $\tau_a$ are the familiar Pauli matrices.

Since we have three generators $(T_a)$ for the $SU(2)$ symmetries, and one (Y) for the $U(1)$ symmetry, we will introduce four gauge fields to construct the covariant derivative $D_\mu$.

\begin{align*}
  SU(2)\rightarrow W_\mu^1,\W_\mu^2, \W_\mu^3\\
  U(1)\rightarrow B_\mu
\end{align*}

The gauge covariant derivative takes the form
\[D_\mu = \partial_\mu + igW_\mu^a\frac{\tau^a}{2}+ig'YB_\mu.\]
The potential $V(H)$ reaches its minimum when $H^\dag H = v^2 / 2$. We can pick a vacuum expectation value for $H$ that breaks the neutral sector symmetry (corresponding to $H^0$) but not the charged symmetry (corresponding to $H^{+}$), since we know that electromagnetic gauge invariance is a good symmetry to keep, because photons are massless. Thus, we can write the doublet field as
\[H = \left(\begin{array}{c}0\\\frac{v}{\sqrt{2}}\end{array}\right).\]
We combine $W_\mu^1$ and $W_\mu^2$ to form two composite fields, $W^\pm$, as follows.
\[W_\mu^\pm = \frac{W_\mu^1\pm iW_\mu^2}{\sqrt{2}}\]
We use this redefinition and plug in our value for the vacuum expectation value into the kinetic term of the Lagrangian.
\begin{align*}
  (D_\mu H)^\dag(D_\mu H) &= (\begin{array}{cc} 0 & \frac{v}{\sqrt{2}}\end{array})
  \left(\partial_\mu - igW_\mu^a\frac{\tau^a}{2}
  -\frac{ig'}{2}B_\mu\right)
  \left(\partial_\mu+igW_\mu^a\frac{\tau^a}{2}+\frac{ig'}{2}B_\mu\right)
  \left(\begin{array}{c}0\\\frac{v}{\sqrt{2}}\end{array}\right)\\
  W_\mu^a\frac{\tau^a}{2} &= 
  \frac{1}{\sqrt{2}}
  \left(\begin{array}{cc}
    \frac{W_\mu^3}{\sqrt{2}} & W_\mu^+\\
    W_\mu^- & -\frac{W_\mu^3}{\sqrt{2}}
  \end{array}\right).
\end{align*}
If we simplify the above expression and isolate the mass terms, we get
\[\frac{g^2v^2}{4}W_\mu^-W_\mu^++\frac{1}{2}\frac{v^2}{4}(g^2+g'^2)
\left(\frac{gW_\mu^3}{\sqrt{g^2+g'^2}}-\frac{g'B_\mu}{\sqrt{g^2+g'^2}}\right)^2\]
We can make the following definitions:
\begin{align*}
  \frac{g}{\sqrt{g^2+g'^2}} = \cos\theta\\
  \frac{g'}{\sqrt{g^2+g'^2}} = \sin\theta
\end{align*}
We also make composite fields out of $W_\mu^3$ and $B_\mu$, namely the Z boson ($Z_\mu$) and the photon ($A_\mu$), with $Z_\mu$ defined as follows.
\[Z_\mu = \frac{gW_\mu^3}{\sqrt{g^2+g'^2}}-\frac{g'B_\mu^3}{\sqrt{g^2+g'^2}}\]
The product $W_\mu^+W_\mu^-$ can be written as a mass term for a generic $W_\mu$ boson that comes in two charges. Rewriting our mass terms for the vector boson fields with the above redefinitions, we get
\[\mathcal{L}_{\text{mass term}}=\frac{e^2v^2}{4\sin^2\theta}W_\mu W^\mu+
\frac{e^2v^2}{8\sin^2\theta\cos^2\theta}Z_\mu Z^\mu\]
After accounting for symmetry factors, we get the masses of the W and Z bosons as:
\begin{align*}
  m_W &=\frac{ev}{2\sin\theta}\\
  m_Z &=\frac{ev}{\sin2\theta}
\end{align*}
We see that there is no mass term for $A_\mu$, which means that this gauge field remains massless. Thus we see that the W and Z bosons gain mass, limiting their range, while the photon remains massless, corresponding to what is experimentally observed.
