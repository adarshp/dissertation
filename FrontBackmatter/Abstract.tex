%*******************************************************
% Abstract
%*******************************************************
\chapter*{Abstract}
\addcontentsline{toc}{chapter}{Abstract}
Despite its indisputable successes, the Standard Model of particle physics (SM) is widely considered to be an effective low-energy approximation to an underlying theory that describes physics at higher energy scales. While there are many candidates for such a theory, nearly all of them predict predict the existence of additional particles beyond those of the Standard Model. In this work, we present three analyses aimed at discovering new particles at current and future particle colliders.

The first two analyses are designed to probe extended scalar sectors, which often arise in theories beyond the Standard Model (BSM). The structure of these extended scalar sectors can be described by a physically well-motivated class of models, known collectively as Two-Higgs Doublet Models ($2$HDMs). The scalar mass spectrum of $2$HDMs is comprised of two CP-even states \emph{h} and \emph{H}, a CP-odd state \emph{A}, and a charged state $H^\pm$. Traditional searches for these states at particle colliders focus on finding them via their decays to SM particles. However, there are compelling scenarios in which these heavy scalars decay through exotic modes to non-SM final states. In certain regions of parameter space, these exotic modes can even dominate the conventional decay modes to SM final states, and thus provide a complementary avenue for discovering new Higgs bosons. 

The first analysis presented aims to discover charged Higgs bosons $H^\pm$ via top decay at the LHC. We find that the exotic decay modes outperform the conventional decay modes for regions of parameter space with low values of the $2$HDM parameter $\tan\beta$. 

The second analysis aims to systematically cover all the exotic decay scenarios that are consistent with theoretical and experimental constraints, at both the 14 TeV LHC and a future 100 TeV hadron collider. We find that the preliminary results are promising - we are able to exclude  a large swathe of $2$HDM parameter space, up to scalar masses of 3.5 TeV, for a wide range of values of $\tan\beta$.

In addition to these two analyses, we also present a third, aimed at discovering pair produced higgsinos that decay to binos at a 100 TeV collider. Higgsinos and binos are new fermion states that arise in the Minimal Supersymmetric Standard Model (MSSM). This heavily-studied model is the minimal phenomenologically viable incorporation of supersymmetry - a symmetry that connects fermions and bosons - into the Standard Model. The structure of the scalar sector MSSM at tree level reduces to that of a $2$HDM. In the scenario we consider, the bino is the lightest supersymmetric partner, which makes it a good candidate for dark matter. Using razor variables and boosted decision trees, we are able to exclude Higgsinos up to 1.8 TeV for binos up to 1.3 TeV.
