%*******************************************************
% Abstract
%*******************************************************
\chapter*{Abstract}
\phantomsection
\addcontentsline{toc}{chapter}{Abstract}
Despite its indisputable successes, the Standard Model of particle physics (SM) is widely considered to be an effective low-energy approximation to an underlying theory that describes physics at higher energy scales. 
The class of models known as Two-Higgs Doublet Models ($2$HDMs) are physically well-motivated extensions to the scalar sector of the SM. Through the addition of an additional $SU(2)$ Higgs doublet field, they extend the scalar mass spectrum so that it is comprised of two CP-even states \emph{h} and \emph{H}, a CP-odd state \emph{A}, and a charged state $H^\pm$.
Traditional searches for these states at particle colliders focus on finding them via their decays to SM particles. However, there are compelling scenarios in which these heavy scalars decay predominantly through exotic modes to non-SM final states. In certain regions of parameter space, these exotic modes can dominate the conventional decay modes to SM final states, and thus provide a complementary avenue for discovery. In this work, we present two analyses aimed at discovering new scalar states at the Large Hadron Collider (LHC) and a possible future hadron collider via exotic decay modes. The first is aimed at discovering charged Higgs bosons $H^\pm$ via top decay at the LHC. We find that the exotic decay modes outperform the conventional decay modes for regions of parameter space with low values of the $2$HDM parameter $\tan\beta$. The second analysis is aims to systematically cover a number of different exotic decay scenarios at both the 14 TeV LHC and a future 100 TeV hadron collider. We find that the preliminary results are promising, and cover a large swathe of $2$HDM parameter space, up to scalar masses of 3.5 TeV, for a wide range of values of $\tan\beta$.

In addition, we also present an analysis aimed at discovering pair produced Higgsinos that decay to binos at a 100 TeV collider. Higgsinos and binos are the fermionic counterparts of scalar states in the Minimal Supersymmetric Standard Model. This heavily-studied model can be viewed a special case of a $2$HDM that incorporates supersymmetry, a symmetry that connects fermions and bosons. This analysis is performed using the razor variables and boosted decision trees. We find that we are able to exclude Higgsinos up to 1.8 TeV for binos up to 1.3 TeV with this analysis.
